% !TEX TS-program = xelatex
% !TEX encoding = UTF-8 Unicode

%===========================================%
%  Preamble of the LaTeX version of SICP.   %
%  It will be read in by texi-to-latex.pl.  %
%===========================================%

\documentclass[11pt,oneside]{book}

% New line height: 1.05 * 1.2 = 1.26
\renewcommand{\baselinestretch}{1.05}

% Font settings
\usepackage[no-math]{fontspec} %-----------------------%  This exact order
\defaultfontfeatures{%                                 %  of declarations
  Scale=MatchLowercase, % needed here ...              %  is important for
}                                                      %  single quotes not 
\setmonofont{Inconsolata LGC}                          %  to turn out curly 
                                                       %  ("typographic")
\defaultfontfeatures{%                                 %  in verbatim blocks
  Scale=MatchLowercase, % ... and here again           %  of Scheme code.
  Mapping=tex-text,                                    %  Now the quote is
  SmallCapsFeatures={LetterSpace=2.5,WordSpace=1.05},  %  upright and safely
}                                                      %  copy-pasteable to 
\setmainfont{Linux Libertine O}                        %  the REPL without
\setsansfont{Linux Biolinum O} %-----------------------%  giving errors.

\usepackage{polyglossia}
\setdefaultlanguage[variant=american]{english}
\setotherlanguages{greek}

% To be able to use '\-/' in place of '-' inside \code{}
% so that long function names containing hyphens 
% can be broken up after the hyphen:
\usepackage[shortcuts]{extdash} 

% So that file names with multiple dots don't confuse 
% graphicx package when using \includegraphics command:
\usepackage[multidot]{grffile}
\usepackage{graphicx}

\usepackage[usenames,dvipsnames,x11names]{xcolor}
\usepackage{amsmath}

% To use Libertine letters and numbers,
% but tx-style operators in math environment:
\usepackage[libertine]{newtxmath} 

% Workaround to fix mismatched left and right math delimiters. Taken from: 
% http://tex.stackexchange.com/questions/63410/parentheses-differ-xelatex-fontspec-newtxmath-libertine
\DeclareSymbolFont{parenthesis}{T1}{fxl}{m}{n}
\DeclareMathDelimiter{(}{\mathopen}{parenthesis}{"28}{largesymbols}{"00}
\DeclareMathDelimiter{)}{\mathclose}{parenthesis}{"29}{largesymbols}{"01}
\DeclareMathDelimiter{[}{\mathopen}{parenthesis}{"5B}{largesymbols}{"02} 
\DeclareMathDelimiter{]}{\mathclose}{parenthesis}{"5D}{largesymbols}{"03} 
\DeclareMathDelimiter{\lbrace}{\mathopen}{parenthesis}{"7B}{largesymbols}{"08} 
\DeclareMathDelimiter{\rbrace}{\mathclose}{parenthesis}{"7D}{largesymbols}{"09}

\usepackage{fancyvrb}
\usepackage{imakeidx}
\usepackage[totoc,font=footnotesize]{idxlayout}
\usepackage{fancyhdr}
\pagestyle{plain}
\usepackage[final]{pdfpages} % inserts pages from a pdf file

% Page geometry for 10-inch tablets:
\usepackage[papersize={148mm,197mm},
            top=21mm,
            textwidth=111mm,  % increase one or both of these by 1mm or so to
            textheight=148mm, % make the source compile with TeX Live 2013
            hcentering,       % if you get "Too many unprocessed floats" error
]{geometry}

\usepackage{titlesec}  % to change the appearance of section titles 
\usepackage{listings}  % for syntax highlighted code listings
\usepackage{verbatim}  % for simple verbatim and comment environments
\usepackage{enumerate} % allows customized labels in enumerations
\usepackage{hyperref}  % makes cross references and URLs clickable 
\definecolor{LinkRed}{HTML}{80171F}
\hypersetup{
  pdfauthor={Harold Abelson, Gerald Jay Sussman, Julie Sussman},
  pdftitle={Structure and Interpretation of Computer Programs, 2nd ed.},
  pdfsubject={computer science, programming, abstraction},
  colorlinks=true,
  linkcolor=LinkRed,
  urlcolor=LinkRed,
}

% Document colors 
\definecolor{SchemeLight}  {HTML} {686868}
\definecolor{SchemeSteel}  {HTML} {888888}
\definecolor{SchemeDark}   {HTML} {262626}
\definecolor{SchemeBlue}   {HTML} {4172A3}
\definecolor{SchemeGreen}  {HTML} {487818}
\definecolor{SchemeBrown}  {HTML} {A07040}
\definecolor{SchemeRed}    {HTML} {AD4D3A}
\definecolor{SchemeViolet} {HTML} {7040A0}
\definecolor{DropCapGray}  {HTML} {B8B8B8}
\definecolor{ChapterGray}  {HTML} {C8C8C8}

\usepackage{lettrine}  % adds commands that make drop capitals
\renewcommand{\LettrineFontHook}{\rmfamily\bfseries\color{DropCapGray}}
\renewcommand{\DefaultLraise}{0.00}
\renewcommand{\DefaultLoversize}{0.02}
\renewcommand{\DefaultLhang}{0.12}
\setlength{\DefaultFindent}{1pt}
\setlength{\DefaultNindent}{0em}

\lstset{%
  % Scheme syntax highlighter
    columns=fixed,
    extendedchars=true,
    upquote=true,
    showstringspaces=false,
    sensitive=false,
    mathescape=true,
    escapechar=~,
    alsodigit={>,<,/,-,=,!,?,*},
    alsoletter=',
    morestring=[b]",
    morecomment=[l];,
    % Keyword list taken form functional.py in Pygments package:
    morekeywords={lambda, define, if, else, cond, and, or, case,%
      let, let*, letrec, begin, do, delay, set!, =>, quote,%
      quasiquote, unquote, unquote-splicing, define-syntax, let-syntax,%
      letrec-syntax, syntax-rules},
    % If keywords are quoted, they must not be highlighted:
    emph={'lambda, 'define, 'if, 'else, 'cond, 'and, 'or, 'case,%
      'let, 'let*, 'letrec, 'begin, 'do, 'delay, 'set!, '=>, 'quote,%
      'quasiquote, 'unquote, 'unquote-splicing, 'define-syntax, 'let-syntax,%
      'letrec-syntax, 'syntax-rules}, 
    emphstyle=\color{SchemeDark},
    % Paint error red:
    emph={[2]error},emphstyle=[2]\color{SchemeRed},%
    % Builtins taken from functional.py:
    emph={[3]*, +, -, /, <, <=, =, >, >=, abs, acos, angle,
        append, apply, asin, assoc, assq, assv, atan,
        boolean?, caaaar, caaadr, caaar, caadar, caaddr, caadr,
        caar, cadaar, cadadr, cadar, caddar, cadddr, caddr,
        cadr, call-with-current-continuation, call-with-input-file,
        call-with-output-file, call-with-values, call/cc, car,
        cdaaar, cdaadr, cdaar, cdadar, cdaddr, cdadr, cdar,
        cddaar, cddadr, cddar, cdddar, cddddr, cdddr, cddr,
        cdr, ceiling, char->integer, char-alphabetic?, char-ci<=?,
        char-ci<?, char-ci=?, char-ci>=?, char-ci>?, char-downcase,
        char-lower-case?, char-numeric?, char-ready?, char-upcase,
        char-upper-case?, char-whitespace?, char<=?, char<?, char=?,
        char>=?, char>?, char?, close-input-port, close-output-port,
        complex?, cons, cos, current-input-port, current-output-port,
        denominator, display, dynamic-wind, eof-object?, eq?,
        equal?, eqv?, eval, even?, exact->inexact, exact?, exp,
        expt, floor, for-each, force, gcd, imag-part,
        inexact->exact, inexact?, input-port?, integer->char,
        integer?, interaction-environment, lcm, length, list,
        list->string, list->vector, list-ref, list-tail, list?,
        load, log, magnitude, make-polar, make-rectangular,
        make-string, make-vector, map, max, member, memq, memv,
        min, modulo, negative?, newline, not, null-environment,
        null?, number->string, number?, numerator, odd?,
        open-input-file, open-output-file, output-port?, pair?,
        peek-char, port?, positive?, procedure?, quotient,
        rational?, rationalize, read, read-char, real-part, real?,
        remainder, reverse, round, scheme-report-environment,
        set-car!, set-cdr!, sin, sqrt, string, string->list,
        string->number, string->symbol, string-append, string-ci<=?,
        string-ci<?, string-ci=?, string-ci>=?, string-ci>?,
        string-copy, string-fill!, string-length, string-ref,
        string-set!, string<=?, string<?, string=?, string>=?,
        string>?, string?, substring, symbol->string, symbol?,
        tan, transcript-off, transcript-on, truncate, values,
        vector, vector->list, vector-fill!, vector-length,
        vector-ref, vector-set!, vector?, with-input-from-file,
        with-output-to-file, write, write-char, zero?},
    emphstyle=[3]\color{SchemeViolet},%
    %
    basicstyle=\color{SchemeLight}\ttfamily,
    keywordstyle=\color{SchemeBlue}\bfseries,
    identifierstyle=\color{SchemeDark},
    stringstyle=\color{SchemeGreen},
    commentstyle=\color{SchemeLight}\itshape,
}
  
\newcommand{\acronym}[1]{\textsc{\MakeLowercase{#1}}}
\newcommand{\newterm}[1]{\index{#1}\emph{#1}}
\newcommand{\strong}[1]{\textbf{#1}}
\newcommand{\var}[1]{\textsl{#1}}
\newcommand{\code}[1]{\texttt{#1}}
\newcommand{\link}[1]{\hyperref[#1]{#1}}
\newcommand{\heading}[1]{{\sffamily\bfseries #1}}
\newcommand{\dark}{\color{SchemeDark}}

\newenvironment{example}%
  {\verbatim\small}%
  {\endverbatim}

\newenvironment{smallexample}%
  {\verbatim\footnotesize}%
  {\endverbatim}

\lstnewenvironment{scheme}[1][]
{\lstset{basicstyle=\ttfamily\small\color{SchemeLight},#1}}
{}

\lstnewenvironment{smallscheme}[1][]
{\lstset{basicstyle=\ttfamily\footnotesize\color{SchemeLight},#1}}
{}

\titleformat{\chapter}[display]
  {\color{SchemeDark}\normalfont\sffamily\bfseries\LARGE}
  {\filright \color{ChapterGray}\fontsize{6em}{0em}\selectfont
    \oldstylenums{\thechapter}}
  {1em}
  {\filright}
  
\titleformat{\section}
{\color{SchemeDark}\normalfont\Large\sffamily\bfseries}
{\color{SchemeSteel}\thesection}{0.8em}{}

\titleformat{\subsection}
{\color{SchemeDark}\normalfont\large\sffamily\bfseries}
{\color{SchemeSteel}\thesubsection}{0.8em}{}

\titleformat{\subsubsection}
{\color{black}\normalfont\normalsize\sffamily\bfseries}
{\color{SchemeSteel}\thesubsubsection}{0.8em}{}

\setcounter{secnumdepth}{3}
\setcounter{tocdepth}{3}

\frenchspacing
\makeindex

%====================%
%  End of preamble.  %
%====================%

\begin{document}

\VerbatimFootnotes

\frontmatter

\includepdf[scale=0.92]{coverpage.pdf}

\pagebreak

\vspace*{\fill}
\thispagestyle{empty}

\begin{small}

\noindent
{\copyright}1996 by The Massachusetts Institute of Technology

\vspace{1.26em}
\noindent
Structure and Interpretation of Computer Programs,\\
second edition

\vspace{1.26em}
\noindent
Harold Abelson and Gerald Jay Sussman\\
with Julie Sussman, foreword by Alan J. Perlis

\vspace{1.6em}
\noindent
\includegraphics[width=3mm]{fig/icons/cc.pdf}
\includegraphics[width=3mm]{fig/icons/by.pdf}
\includegraphics[width=3mm]{fig/icons/sa.pdf}

\vspace{0.4em}
\noindent
This work is licensed under a Creative Commons\\ 
Attribution-ShareAlike 3.0 Unported License\\
(\href{http://creativecommons.org/licenses/by-sa/3.0/}{\acronym{CC BY-SA} 3.0}).
Based on a work at \href{http://mitpress.mit.edu/sicp/}{mitpress.mit.edu}.

\vspace{1.26em}
\noindent
The \acronym{MIT} Press\\
Cambridge, Massachusetts\\ 
London, England

\vspace{1.26em}
\noindent
McGraw-Hill Book Company\\
New York, St. Louis, San Francisco,\\ 
Montreal, Toronto

\vspace{1.26em}
\noindent
Unofficial Texinfo Format \href{http://sicpebook.wordpress.com}{2.andresraba5.2} (February 10, 2014),\\ 
based on \href{http://www.neilvandyke.org/sicp-texi/}{2.neilvandyke4} (January 10, 2007).

\end{small}

\pagebreak

\tableofcontents

%             **********************************************************
%         sicp
%            Structure and Interpretation of Computer Programs, 2e
%             Unofficial Texinfo Format
%
% utfversion      2.andresraba5.2 
% utfversiondate  February 10, 2014
%
%             This file is licensed under a Creative Commons 
%             Attribution-ShareAlike 3.0 Unported License 
%             http://creativecommons.org/licenses/by-sa/3.0/
%             
%             This is a Texinfo file.  To convert it to Info hypertext
%             format, you will need the `makeinfo' program from the GNU
%             Texinfo package.  To produce a PDF, use `texi2pdf'. 
%             For more information about this file,
%             see the text under `\label{UTF' below.}
%             
%             Various versions of sicp.texi and preformatted sicp.info
%             can be found at the following Web pages:
%             
%                 http://www.neilvandyke.org/sicp-texi/
%                 http://sicpebook.wordpress.com/
%                 [add your own here]
%             
%             **********************************************************

% HISTORY:
%
% * Version 1 (April, 2001) by Lytha Ayth.
%
% * Version 2 (April 20, 2001) by Lytha Ayth.
%
% * Version 2.nwv1 (March 11, 2002) by Neil W. Van Dyke.
%   Cosmetic change to heading in Info format, and comment changes.
% 
% * Version 2.neilvandyke1 (February 10, 2003) by Neil W. Van Dyke
%   Correction to Exercise 1.39 formula, spotted by Steve VanDevender.
%   Added URL of Abelson and Sussman video lectures.
%
% * Version 2.neilvandyke2 (unreleased)
%
% * Version 2.neilvandyke3 (April 20, 2006) by Neil W. Van Dyke
%   Pedro Kr\"oger patch to add missing Lisp example.
%
% * Version 2.neilvandyke4 (January 10, 2007) by Neil W. Van Dyke
%   Brad Walker patch to add \code{@dircategory} and \code{@direntry}.
%
% * Version 2.andresraba1 (May 23, 2011) by Andres Raba.
%   Mathematics typeset in TeX, figures redrawn in vector graphics,
%   typeface changed, cross-references improved, hyperlinks added,
%   known errors and typos corrected.
%
% * Version 2.andresraba2 (November 21, 2011) by Andres Raba.
%   Minor change to the appearance of diagrams. Adjusted page layout.
%   Fixed some typos. License changed from CC BY-NC to CC BY-SA.
%
% * Version 2.andresraba3 (November 22, 2012) by Andres Raba.
%   Improved layout and pagination. Included list of figures.
%   Added punctuation to displayed math. Updated citation links.

% * (Version 2.andresraba4 is a pocket version, described in
%   sicp-pocket.texi.)

% * Version 2.andresraba5 (September 20, 2013) by Andres Raba.
%   Texinfo source is converted to LaTeX. Pages are redesigned.

% The Algorithmic Language Scheme
\begin{comment}
* SICP: (sicp). Structure and Interpretation of Computer Programs
\end{comment}

\begin{comment} newterm{term}
% \term\
% {\term\}
\end{comment}

% @setchapternewpage odd
\begin{comment}
% {Structure and Interpretation}
\vspace{0.5em}
%  of Computer Programs
\vspace{0.2em}
% Second Edition
% Unofficial Texinfo Format {utfversion}
% Harold Abelson and Gerald Jay Sussman
\vspace{0.2em}
% with Julie Sussman
\vspace{0.2em}
% foreword by Alan J. Perlis
\vspace{0.4em}

% 0pt plus 1filll
\noindent
{\copyright} 1996 by The Massachusetts Institute of Technology

\vspace{0.5em}
\noindent
Structure and Interpretation of Computer Programs\\
second edition

\vspace{0.5em}
\noindent
Harold Abelson and Gerald Jay Sussman\\
with Julie Sussman, foreword by Alan J. Perlis

\vspace{0.7em}
\noindent
\includegraphics[width=6mm]{fig/icons/cc.pdf}
\includegraphics[width=6mm]{fig/icons/by.pdf}
\includegraphics[width=6mm]{fig/icons/sa.pdf}

\noindent
This work is licensed under a Creative Commons\\ 
Attribution-ShareAlike 3.0 Unported License 
(\href{http://creativecommons.org/licenses/by-sa/3.0/}{CC BY-SA 3.0}).\\
Based on a work at \href{http://mitpress.mit.edu/sicp/}{mitpress.mit.edu}.

\vspace{0.5em}
\noindent
The \acronym{MIT} Press\\
Cambridge, Massachusetts\\ 
London, England

\vspace{0.5em}
\noindent
McGraw-Hill Book Company\\
New York, St. Louis, San Francisco,\\ 
Montreal, Toronto

\vspace{0.5em}
\noindent
This book is one of a series of texts written by faculty of the Electrical\\
Engineering and Computer Science Department at the Massachusetts\\ Institute of
Technology.  It was edited and produced by The \acronym{MIT} Press\\ under a
joint production-distribution arrangement with the\\ McGraw-Hill Book Company.

\vspace{0.5em}
\noindent
Unofficial Texinfo Format \href{http://sicpebook.wordpress.com}{{utfversion}} ({utfversiondate}),\\
based on \href{http://www.neilvandyke.org/sicp-texi/}{2.neilvandyke4} (January 10, 2007).

\end{comment}


% to suppress the black boxes after overfull lines:
\label{Top}

\begin{comment}

% Structure and Interpretation of Computer Programs

\noindent
Second Edition\\
by Harold Abelson and Gerald Jay Sussman, with Julie Sussman\\
foreword by Alan J. Perlis\\
{\copyright} 1996 Massachusetts Institute of Technology

\noindent
Unofficial Texinfo Format version {utfversion} ({utfversiondate})

\end{comment}



\chapter*{Unofficial Texinfo Format}
\addcontentsline{toc}{chapter}{Unofficial Texinfo Format}
\label{UTF}

This is the second edition \acronym{SICP} book, from Unofficial Texinfo
Format.

You are probably reading it in an Info hypertext browser, such as the Info
mode of Emacs.  You might alternatively be reading it {\TeX}-formatted on your
screen or printer, though that would be silly.  And, if printed, expensive. 

The freely-distributed official \acronym{HTML}-and-\acronym{GIF} format was
first converted personally to Unofficial Texinfo Format (\acronym{UTF})
version 1 by Lytha Ayth during a long Emacs lovefest weekend in April, 2001.

The \acronym{UTF} is easier to search than the \acronym{HTML} format.  It is
also much more accessible to people running on modest computers, such as
donated '386-based PCs.  A 386 can, in theory, run Linux, Emacs, and a Scheme
interpreter simultaneously, but most 386s probably can't also run both Netscape
and the necessary X Window System without prematurely introducing budding young
underfunded hackers to the concept of \newterm{thrashing}.  UTF can also fit
uncompressed on a 1.44\acronym{MB} floppy diskette, which may come in handy for
installing UTF on PCs that do not have Internet or LAN access.

The Texinfo conversion has been a straight transliteration, to the extent
possible.  Like the {\TeX}-to-\acronym{HTML} conversion, this was not without
some introduction of breakage.  In the case of Unofficial Texinfo Format,
figures have suffered an amateurish resurrection of the lost art of
\acronym{ASCII}.  Also, it's quite possible that some errors of ambiguity
were introduced during the conversion of some of the copious superscripts (`\^{}')
and subscripts (`\_').  Divining \emph{which} has been left as an exercise to
the reader. But at least we don't put our brave astronauts at risk by encoding
the \emph{greater-than-or-equal} symbol as \code{<u>\&gt;</u>}.

If you modify \texttt{sicp.texi} to correct errors or improve the
\acronym{ASCII} art, then update the \code{@set utfversion {utfversion}}
line to reflect your delta.  For example, if you started with Lytha's version
\code{1}, and your name is Bob, then you could name your successive versions
\code{1.bob1}, \code{1.bob2}, \( \dots \) \code{1.bob\textit{n}}.  Also update
\code{utfversiondate}.  If you want to distribute your version on the Web, then
embedding the string ``sicp.texi'' somewhere in the file or Web page will make
it easier for people to find with Web search engines.

It is believed that the Unofficial Texinfo Format is in keeping with the
spirit of the graciously freely-distributed \acronym{HTML} version.  But you
never know when someone's armada of lawyers might need something to do, and get
their shorts all in a knot over some benign little thing, so think twice before
you use your full name or distribute Info, \acronym{DVI}, PostScript, or
\acronym{PDF} formats that might embed your account or machine name.

\vspace{0.5em}
\noindent
\textit{Peath, Lytha Ayth}

\vspace{1.0em}
\noindent
\textbf{Addendum:} See also the \acronym{SICP} video lectures by Abelson and Sussman:\\
at \href{http://groups.csail.mit.edu/mac/classes/6.001/abelson-sussman-lectures/}{\acronym{MIT CSAIL}} or 
\href{http://ocw.mit.edu/courses/electrical-engineering-and-computer-science/6-001-structure-and-interpretation-of-computer-programs-spring-2005/video-lectures/}{\acronym{MIT OCW}}.

\vspace{0.5em}
\noindent 
\textbf{Second Addendum:} Above is the original introduction to the \acronym{UTF} 
from 2001. Ten years later, \acronym{UTF} has been transformed: mathematical 
symbols and formulas are properly typeset, and figures drawn in 
vector graphics. The original text formulas and \acronym{ASCII} art figures 
are still there in the Texinfo source, but will display only when compiled 
to Info output. At the dawn of e-book readers and tablets, reading a 
\acronym{PDF} on screen is officially not silly anymore. Enjoy!

\enlargethispage{\baselineskip}

\vspace{0.5em}
\noindent
\textit{A.R, May, 2011}

\chapter*{Dedication}
\addcontentsline{toc}{chapter}{Dedication}
\label{Dedication}

\lettrine{T}{his book is dedicated}, in respect and admiration, to the spirit that lives in
the computer.

\begin{quote}
``I think that it's extraordinarily important that we in computer science keep
fun in computing.  When it started out, it was an awful lot of fun.  Of course,
the paying customers got shafted every now and then, and after a while we began
to take their complaints seriously.  We began to feel as if we really were
responsible for the successful, error-free perfect use of these machines.  I
don't think we are.  I think we're responsible for stretching them, setting
them off in new directions, and keeping fun in the house.  I hope the field of
computer science never loses its sense of fun.  Above all, I hope we don't
become missionaries.  Don't feel as if you're Bible salesmen.  The world has
too many of those already.  What you know about computing other people will
learn.  Don't feel as if the key to successful computing is only in your hands.
What's in your hands, I think and hope, is intelligence: the ability to see the
machine as more than when you were first led up to it, that you can make it
more.''

\enlargethispage{\baselineskip}

\noindent
---Alan J. Perlis (April 1, 1922 -- February 7, 1990)
\end{quote}

\chapter*{Foreword}
\addcontentsline{toc}{chapter}{Foreword}
\label{Foreword}

% \vspace{-0.6em}

\lettrine{E}{ducators, generals, dieticians}, psychologists, and parents program.  Ar\-mies,
students, and some societies are programmed.  An assault on large problems
employs a succession of programs, most of which spring into existence en route.
These programs are rife with issues that appear to be particular to the problem
at hand.  To appreciate programming as an intellectual activity in its own
right you must turn to computer programming; you must read and write computer
programs---many of them.  It doesn't matter much what the programs are about or
what applications they serve.  What does matter is how well they perform and
how smoothly they fit with other programs in the creation of still greater
programs.  The programmer must seek both perfection of part and adequacy of
collection.  In this book the use of ``program'' is focused on the creation,
execution, and study of programs written in a dialect of Lisp for execution on
a digital computer.  Using Lisp we restrict or limit not what we may program,
but only the notation for our program descriptions.

Our traffic with the subject matter of this book involves us with three foci of
phenomena: the human mind, collections of computer programs, and the computer.
Every computer program is a model, hatched in the mind, of a real or mental
process.  These processes, arising from human experience and thought, are huge
in number, intricate in detail, and at any time only partially understood.
They are modeled to our permanent satisfaction rarely by our computer programs.
Thus even though our programs are carefully handcrafted discrete collections of
symbols, mosaics of interlocking functions, they continually evolve: we change
them as our perception of the model deepens, enlarges, generalizes until the
model ultimately attains a metastable place within still another model with
which we struggle.  The source of the exhilaration associated with computer
programming is the continual unfolding within the mind and on the computer of
mechanisms expressed as programs and the explosion of perception they generate.
If art interprets our dreams, the computer executes them in the guise of
programs!

For all its power, the computer is a harsh taskmaster.  Its programs must be
correct, and what we wish to say must be said accurately in every detail.  As
in every other symbolic activity, we become convinced of program truth through
argument.  Lisp itself can be assigned a semantics (another model, by the way),
and if a program's function can be specified, say, in the predicate calculus,
the proof methods of logic can be used to make an acceptable correctness
argument.  Unfortunately, as programs get large and complicated, as they almost
always do, the adequacy, consistency, and correctness of the specifications
themselves become open to doubt, so that complete formal arguments of
correctness seldom accompany large programs.  Since large programs grow from
small ones, it is crucial that we develop an arsenal of standard program
structures of whose correctness we have become sure---we call them idioms---and
learn to combine them into larger structures using organizational techniques of
proven value.  These techniques are treated at length in this book, and
understanding them is essential to participation in the Promethean enterprise
called programming.  More than anything else, the uncovering and mastery of
powerful organizational techniques accelerates our ability to create large,
significant programs.  Conversely, since writing large programs is very taxing,
we are stimulated to invent new methods of reducing the mass of function and
detail to be fitted into large programs.

Unlike programs, computers must obey the laws of physics.  If they wish to
perform rapidly---a few nanoseconds per state change---they must transmit
electrons only small distances (at most \( 1 {1\over2} \) feet). The heat generated by
the huge number of devices so concentrated in space has to be removed.  An
exquisite engineering art has been developed balancing between multiplicity of
function and density of devices.  In any event, hardware always operates at a
level more primitive than that at which we care to program.  The processes that
transform our Lisp programs to ``machine'' programs are themselves abstract
models which we program.  Their study and creation give a great deal of insight
into the organizational programs associated with programming arbitrary models.
Of course the computer itself can be so modeled.  Think of it: the behavior of
the smallest physical switching element is modeled by quantum mechanics
described by differential equations whose detailed behavior is captured by
numerical approximations represented in computer programs executing on
computers composed of \( \dots \)!

It is not merely a matter of tactical convenience to separately identify the
three foci.  Even though, as they say, it's all in the head, this logical
separation induces an acceleration of symbolic traffic between these foci whose
richness, vitality, and potential is exceeded in human experience only by the
evolution of life itself.  At best, relationships between the foci are
metastable.  The computers are never large enough or fast enough.  Each
breakthrough in hardware technology leads to more massive programming
enterprises, new organizational principles, and an enrichment of abstract
models.  Every reader should ask himself periodically ``Toward what end, toward
what end?''---but do not ask it too often lest you pass up the fun of
programming for the constipation of bittersweet philosophy.

Among the programs we write, some (but never enough) perform a precise
mathematical function such as sorting or finding the maximum of a sequence of
numbers, determining primality, or finding the square root.  We call such
programs algorithms, and a great deal is known of their optimal behavior,
particularly with respect to the two important parameters of execution time and
data storage requirements.  A programmer should acquire good algorithms and
idioms.  Even though some programs resist precise specifications, it is the
responsibility of the programmer to estimate, and always to attempt to improve,
their performance.

Lisp is a survivor, having been in use for about a quarter of a century.  Among
the active programming languages only Fortran has had a longer life.  Both
languages have supported the programming needs of important areas of
application, Fortran for scientific and engineering computation and Lisp for
artificial intelligence.  These two areas continue to be important, and their
programmers are so devoted to these two languages that Lisp and Fortran may
well continue in active use for at least another quarter-century.

Lisp changes.  The Scheme dialect used in this text has evolved from the
original Lisp and differs from the latter in several important ways, including
static scoping for variable binding and permitting functions to yield functions
as values.  In its semantic structure Scheme is as closely akin to Algol 60 as
to early Lisps.  Algol 60, never to be an active language again, lives on in
the genes of Scheme and Pascal.  It would be difficult to find two languages
that are the communicating coin of two more different cultures than those
gathered around these two languages.  Pascal is for building
pyramids---imposing, breathtaking, static structures built by armies pushing
heavy blocks into place.  Lisp is for building organisms---imposing,
breathtaking, dynamic structures built by squads fitting fluctuating myriads of
simpler organisms into place.  The organizing principles used are the same in
both cases, except for one extraordinarily important difference: The
discretionary exportable functionality entrusted to the individual Lisp
programmer is more than an order of magnitude greater than that to be found
within Pascal enterprises.  Lisp programs inflate libraries with functions
whose utility transcends the application that produced them.  The list, Lisp's
native data structure, is largely responsible for such growth of utility.  The
simple structure and natural applicability of lists are reflected in functions
that are amazingly nonidiosyncratic.  In Pascal the plethora of declarable data
structures induces a specialization within functions that inhibits and
penalizes casual cooperation.  It is better to have 100 functions operate on
one data structure than to have 10 functions operate on 10 data structures.  As
a result the pyramid must stand unchanged for a millennium; the organism must
evolve or perish.

To illustrate this difference, compare the treatment of material and exercises
within this book with that in any first-course text using Pascal.  Do not labor
under the illusion that this is a text digestible at \acronym{MIT} only,
peculiar to the breed found there.  It is precisely what a serious book on
programming Lisp must be, no matter who the student is or where it is used.

Note that this is a text about programming, unlike most Lisp books, which are
used as a preparation for work in artificial intelligence.  After all, the
critical programming concerns of software engineering and artificial
intelligence tend to coalesce as the systems under investigation become larger.
This explains why there is such growing interest in Lisp outside of artificial
intelligence.

As one would expect from its goals, artificial intelligence research generates
many significant programming problems.  In other programming cultures this
spate of problems spawns new languages.  Indeed, in any very large programming
task a useful organizing principle is to control and isolate traffic within the
task modules via the invention of language.  These languages tend to become
less primitive as one approaches the boundaries of the system where we humans
interact most often.  As a result, such systems contain complex
language-processing functions replicated many times.  Lisp has such a simple
syntax and semantics that parsing can be treated as an elementary task.  Thus
parsing technology plays almost no role in Lisp programs, and the construction
of language processors is rarely an impediment to the rate of growth and change
of large Lisp systems.  Finally, it is this very simplicity of syntax and
semantics that is responsible for the burden and freedom borne by all Lisp
programmers.  No Lisp program of any size beyond a few lines can be written
without being saturated with discretionary functions.  Invent and fit; have
fits and reinvent!  We toast the Lisp programmer who pens his thoughts within
nests of parentheses.

\vspace{0.5em}
\noindent
Alan J. Perlis\\
New Haven, Connecticut

\chapter*{Preface to the Second Edition}
\addcontentsline{toc}{chapter}{Preface to the Second Edition}
\label{Preface}

\begin{quote}
Is it possible that software is not like anything else, that it is meant to be
discarded: that the whole point is to always see it as a soap bubble?

---Alan J. Perlis
\end{quote}

% \vspace{0.7em}

\noindent
\lettrine{T}{he material in this book} has been the basis of \acronym{MIT}'s entry-level
computer science subject since 1980.  We had been teaching this material for
four years when the first edition was published, and twelve more years have
elapsed until the appearance of this second edition.  We are pleased that our
work has been widely adopted and incorporated into other texts.  We have seen
our students take the ideas and programs in this book and build them in as the
core of new computer systems and languages.  In literal realization of an
ancient Talmudic pun, our students have become our builders.  We are lucky to
have such capable students and such accomplished builders.

In preparing this edition, we have incorporated hundreds of clarifications
suggested by our own teaching experience and the comments of colleagues at
\acronym{MIT} and elsewhere.  We have redesigned most of the major programming
systems in the book, including the generic-arithmetic system, the interpreters,
the register-machine simulator, and the compiler; and we have rewritten all the
program examples to ensure that any Scheme implementation conforming to the
\acronym{IEEE} Scheme standard (\link{IEEE 1990}) will be able to run the
code.

This edition emphasizes several new themes.  The most important of these is the
central role played by different approaches to dealing with time in
computational models: objects with state, concurrent programming, functional
programming, lazy evaluation, and nondeterministic programming.  We have
included new sections on concurrency and nondeterminism, and we have tried to
integrate this theme throughout the book.

The first edition of the book closely followed the syllabus of our
\acronym{MIT} one-semester subject.  With all the new material in the second
edition, it will not be possible to cover everything in a single semester, so
the instructor will have to pick and choose.  In our own teaching, we sometimes
skip the section on logic programming (\link{Section 4.4}), we have students use
the register-machine simulator but we do not cover its implementation (\link{Section 5.2}), 
and we give only a cursory overview of the compiler (\link{Section 5.5}).  
Even so, this is still an intense course.  Some instructors may
wish to cover only the first three or four chapters, leaving the other material
for subsequent courses.

The World-Wide-Web site \href{http://mitpress.mit.edu/sicp}{http://mitpress.mit.edu/sicp} provides
support for users of this book.  This includes programs from the book, sample
programming assignments, supplementary materials, and downloadable
implementations of the Scheme dialect of Lisp.

\chapter*{Preface to the First Edition}
\addcontentsline{toc}{chapter}{Preface to the First Edition}
\label{Preface 1e}

% \vspace{-0.6em}
\begin{quote}
A computer is like a violin.  You can imagine a novice trying first a
phonograph and then a violin.  The latter, he says, sounds terrible.  That is
the argument we have heard from our humanists and most of our computer
scientists.  Computer programs are good, they say, for particular purposes, but
they aren't flexible.  Neither is a violin, or a typewriter, until you learn
how to use it.

---Marvin Minsky, ``Why Programming Is a Good Medium for Expressing
Poorly-Understood and Sloppily-Formulated Ideas''
\end{quote}

% \vspace{0.8em}

\noindent
\lettrine[lhang=0.17]{``T}{he Structure and Interpretation of Computer Programs''} is the entry-level
subject in computer science at the Massachusetts Institute of Technology.  It
is required of all students at \acronym{MIT} who major in electrical
engineering or in computer science, as one-fourth of the ``common core
curriculum,'' which also includes two subjects on circuits and linear systems
and a subject on the design of digital systems.  We have been involved in the
development of this subject since 1978, and we have taught this material in its
present form since the fall of 1980 to between 600 and 700 students each year.
Most of these students have had little or no prior formal training in
computation, although many have played with computers a bit and a few have had
extensive programming or hardware-design experience.

Our design of this introductory computer-science subject reflects two major
concerns.  First, we want to establish the idea that a computer language is not
just a way of getting a computer to perform operations but rather that it is a
novel formal medium for expressing ideas about methodology.  Thus, programs
must be written for people to read, and only incidentally for machines to
execute.  Second, we believe that the essential material to be addressed by a
subject at this level is not the syntax of particular programming-language
constructs, nor clever algorithms for computing particular functions
efficiently, nor even the mathematical analysis of algorithms and the
foundations of computing, but rather the techniques used to control the
intellectual complexity of large software systems.

Our goal is that students who complete this subject should have a good feel for
the elements of style and the aesthetics of programming.  They should have
command of the major techniques for controlling complexity in a large
system. They should be capable of reading a 50-page-long program, if it is
written in an exemplary style. They should know what not to read, and what they
need not understand at any moment.  They should feel secure about modifying a
program, retaining the spirit and style of the original author.

These skills are by no means unique to computer programming.  The techniques we
teach and draw upon are common to all of engineering design.  We control
complexity by building abstractions that hide details when appropriate.  We
control complexity by establishing conventional interfaces that enable us to
construct systems by combining standard, well-understood pieces in a ``mix and
match'' way.  We control complexity by establishing new languages for
describing a design, each of which emphasizes particular aspects of the design
and deemphasizes others.

Underlying our approach to this subject is our conviction that ``computer
science'' is not a science and that its significance has little to do with
computers.  The computer revolution is a revolution in the way we think and in
the way we express what we think.  The essence of this change is the emergence
of what might best be called \newterm{procedural epistemology}---the study of
the structure of knowledge from an imperative point of view, as opposed to the
more declarative point of view taken by classical mathematical subjects.
Mathematics provides a framework for dealing precisely with notions of ``what
is.''  Computation provides a framework for dealing precisely with notions of
``how to.''

In teaching our material we use a dialect of the programming language Lisp.  We
never formally teach the language, because we don't have to.  We just use it,
and students pick it up in a few days.  This is one great advantage of
Lisp-like languages: They have very few ways of forming compound expressions,
and almost no syntactic structure.  All of the formal properties can be covered
in an hour, like the rules of chess.  After a short time we forget about
syntactic details of the language (because there are none) and get on with the
real issues---figuring out what we want to compute, how we will decompose
problems into manageable parts, and how we will work on the parts.  Another
advantage of Lisp is that it supports (but does not enforce) more of the
large-scale strategies for modular decomposition of programs than any other
language we know.  We can make procedural and data abstractions, we can use
higher-order functions to capture common patterns of usage, we can model local
state using assignment and data mutation, we can link parts of a program with
streams and delayed evaluation, and we can easily implement embedded languages.
All of this is embedded in an interactive environment with excellent support
for incremental program design, construction, testing, and debugging.  We thank
all the generations of Lisp wizards, starting with John McCarthy, who have
fashioned a fine tool of unprecedented power and elegance.

Scheme, the dialect of Lisp that we use, is an attempt to bring together the
power and elegance of Lisp and Algol.  From Lisp we take the metalinguistic
power that derives from the simple syntax, the uniform representation of
programs as data objects, and the garbage-collected heap-allocated data.  From
Algol we take lexical scoping and block structure, which are gifts from the
pioneers of programming-language design who were on the Algol committee.  We
wish to cite John Reynolds and Peter Landin for their insights into the
relationship of Church's λ-calculus to the structure of programming
languages.  We also recognize our debt to the mathematicians who scouted out
this territory decades before computers appeared on the scene.  These pioneers
include Alonzo Church, Barkley Rosser, Stephen Kleene, and Haskell Curry.

\chapter*{Acknowledgments}
\addcontentsline{toc}{chapter}{Acknowledgments}
\label{Acknowledgments}

\lettrine[findent=0pt]{W}{e would like to thank} the many people who have helped us develop this book and
this curriculum.

Our subject is a clear intellectual descendant of ``6.231,'' a wonderful
subject on programming linguistics and the λ-calculus taught at
\acronym{MIT} in the late 1960s by Jack Wozencraft and Arthur Evans, Jr.

We owe a great debt to Robert Fano, who reorganized \acronym{MIT}'s
introductory curriculum in electrical engineering and computer science to
emphasize the principles of engineering design.  He led us in starting out on
this enterprise and wrote the first set of subject notes from which this book
evolved.

Much of the style and aesthetics of programming that we try to teach were
developed in conjunction with Guy Lewis Steele Jr., who collaborated with
Gerald Jay Sussman in the initial development of the Scheme language.  In
addition, David Turner, Peter Henderson, Dan Friedman, David Wise, and Will
Clinger have taught us many of the techniques of the functional programming
community that appear in this book.

Joel Moses taught us about structuring large systems.  His experience with the
Macsyma system for symbolic computation provided the insight that one should
avoid complexities of control and concentrate on organizing the data to reflect
the real structure of the world being modeled.

Marvin Minsky and Seymour Papert formed many of our attitudes about programming
and its place in our intellectual lives.  To them we owe the understanding that
computation provides a means of expression for exploring ideas that would
otherwise be too complex to deal with precisely.  They emphasize that a
student's ability to write and modify programs provides a powerful medium in
which exploring becomes a natural activity.

We also strongly agree with Alan Perlis that programming is lots of fun and we
had better be careful to support the joy of programming.  Part of this joy
derives from observing great masters at work.  We are fortunate to have been
apprentice programmers at the feet of Bill Gosper and Richard Greenblatt.

It is difficult to identify all the people who have contributed to the
development of our curriculum.  We thank all the lecturers, recitation
instructors, and tutors who have worked with us over the past fifteen years and
put in many extra hours on our subject, especially Bill Siebert, Albert Meyer,
Joe Stoy, Randy Davis, Louis Braida, Eric Grimson, Rod Brooks, Lynn Stein and
Peter Szolovits.  We would like to specially acknowledge the outstanding
teaching contributions of Franklyn Turbak, now at Wellesley; his work in
undergraduate instruction set a standard that we can all aspire to.  We are
grateful to Jerry Saltzer and Jim Miller for helping us grapple with the
mysteries of concurrency, and to Peter Szolovits and David McAllester for their
contributions to the exposition of nondeterministic evaluation in \link{Chapter
4}.

Many people have put in significant effort presenting this material at other
universities.  Some of the people we have worked closely with are Jacob
Katzenelson at the Technion, Hardy Mayer at the University of California at
Irvine, Joe Stoy at Oxford, Elisha Sacks at Purdue, and Jan Komorowski at the
Norwegian University of Science and Technology.  We are exceptionally proud of
our colleagues who have received major teaching awards for their adaptations of
this subject at other universities, including Kenneth Yip at Yale, Brian Harvey
at the University of California at Berkeley, and Dan Huttenlocher at Cornell.

Al Moy\'e arranged for us to teach this material to engineers at
Hewlett-Packard, and for the production of videotapes of these lectures.  We
would like to thank the talented instructors---in particular Jim Miller, Bill
Siebert, and Mike Eisenberg---who have designed continuing education courses
incorporating these tapes and taught them at universities and industry all over
the world.

Many educators in other countries have put in significant work translating the
first edition.  Michel Briand, Pierre Chamard, and Andr\'e Pic produced a
French edition; Susanne Daniels-Herold produced a German edition; and Fumio
Motoyoshi produced a Japanese edition.  We do not know who produced the Chinese
edition, but we consider it an honor to have been selected as the subject of an
``unauthorized'' translation.

It is hard to enumerate all the people who have made technical contributions to
the development of the Scheme systems we use for instructional purposes.  In
addition to Guy Steele, principal wizards have included Chris Hanson, Joe
Bowbeer, Jim Miller, Guillermo Rozas, and Stephen Adams.  Others who have put
in significant time are Richard Stallman, Alan Bawden, Kent Pitman, Jon Taft,
Neil Mayle, John Lamping, Gwyn Osnos, Tracy Larrabee, George Carrette, Soma
Chaudhuri, Bill Chiarchiaro, Steven Kirsch, Leigh Klotz, Wayne Noss, Todd Cass,
Patrick O'Donnell, Kevin Theobald, Daniel Weise, Kenneth Sinclair, Anthony
Courtemanche, Henry M. Wu, Andrew Berlin, and Ruth Shyu.

Beyond the \acronym{MIT} implementation, we would like to thank the many people
who worked on the \acronym{IEEE} Scheme standard, including William Clinger and
Jonathan Rees, who edited the \( \rm R^4RS \), and Chris Haynes, David Bartley,
Chris Hanson, and Jim Miller, who prepared the \acronym{IEEE} standard.

Dan Friedman has been a long-time leader of the Scheme community.  The
community's broader work goes beyond issues of language design to encompass
significant educational innovations, such as the high-school curriculum based
on EdScheme by Schemer's Inc., and the wonderful books by Mike Eisenberg and by
Brian Harvey and Matthew Wright.

We appreciate the work of those who contributed to making this a real book,
especially Terry Ehling, Larry Cohen, and Paul Bethge at the \acronym{MIT}
Press.  Ella Mazel found the wonderful cover image.  For the second edition we
are particularly grateful to Bernard and Ella Mazel for help with the book
design, and to David Jones, {\TeX} wizard extraordinaire.  We also are indebted
to those readers who made penetrating comments on the new draft: Jacob
Katzenelson, Hardy Mayer, Jim Miller, and especially Brian Harvey, who did unto
this book as Julie did unto his book \textit{Simply Scheme}.

Finally, we would like to acknowledge the support of the organizations that
have encouraged this work over the years, including support from
Hewlett-Packard, made possible by Ira Goldstein and Joel Birnbaum, and support
from \acronym{DARPA}, made possible by Bob Kahn.

\mainmatter

\chapter{Building Abstractions with Procedures}
\label{Chapter 1}

\begin{quote}
The acts of the mind, wherein it exerts its power over simple ideas, are
chiefly these three: 1. Combining several simple ideas into one compound one,
and thus all complex ideas are made.  2. The second is bringing two ideas,
whether simple or complex, together, and setting them by one another so as to
take a view of them at once, without uniting them into one, by which it gets
all its ideas of relations.  3.  The third is separating them from all other
ideas that accompany them in their real existence: this is called abstraction,
and thus all its general ideas are made.

---John Locke, \emph{An Essay Concerning Human Understanding} (1690)
\end{quote}

% \vspace{1.0em}

\noindent
\lettrine[findent=0pt]{W}{e are about to study} the idea of a \newterm{computational process}.
Computational processes are abstract beings that inhabit computers.  As they
evolve, processes manipulate other abstract things called \newterm{data}.  The
evolution of a process is directed by a pattern of rules called a
\newterm{program}.  People create programs to direct processes.  In effect, we
conjure the spirits of the computer with our spells.

A computational process is indeed much like a sorcerer's idea of a spirit.  It
cannot be seen or touched.  It is not composed of matter at all.  However, it
is very real.  It can perform intellectual work.  It can answer questions.  It
can affect the world by disbursing money at a bank or by controlling a robot
arm in a factory.  The programs we use to conjure processes are like a
sorcerer's spells.  They are carefully composed from symbolic expressions in
arcane and esoteric \newterm{programming languages} that prescribe the tasks we
want our processes to perform.

A computational process, in a correctly working computer, executes programs
precisely and accurately.  Thus, like the sorcerer's apprentice, novice
programmers must learn to understand and to anticipate the consequences of
their conjuring.  Even small errors (usually called \newterm{bugs} or
\newterm{glitches}) in programs can have complex and unanticipated
consequences.

Fortunately, learning to program is considerably less dangerous than learning
sorcery, because the spirits we deal with are conveniently contained in a
secure way.  Real-world programming, however, requires care, expertise, and
wisdom.  A small bug in a computer-aided design program, for example, can lead
to the catastrophic collapse of an airplane or a dam or the self-destruction of
an industrial robot.

Master software engineers have the ability to organize programs so that they
can be reasonably sure that the resulting processes will perform the tasks
intended.  They can visualize the behavior of their systems in advance.  They
know how to structure programs so that unanticipated problems do not lead to
catastrophic consequences, and when problems do arise, they can \newterm{debug}
their programs.  Well-designed computational systems, like well-designed
automobiles or nuclear reactors, are designed in a modular manner, so that the
parts can be constructed, replaced, and debugged separately.

\subsubsection*{Programming in Lisp}

We need an appropriate language for describing processes, and we will use for
this purpose the programming language Lisp.  Just as our everyday thoughts are
usually expressed in our natural language (such as English, French, or
Japanese), and descriptions of quantitative phenomena are expressed with
mathematical notations, our procedural thoughts will be expressed in Lisp.
Lisp was invented in the late 1950s as a formalism for reasoning about the use
of certain kinds of logical expressions, called \newterm{recursion equations},
as a model for computation.  The language was conceived by John McCarthy and is
based on his paper ``Recursive Functions of Symbolic Expressions and Their
Computation by Machine'' (\link{McCarthy 1960}).

Despite its inception as a mathematical formalism, Lisp is a practical
programming language.  A Lisp \newterm{interpreter} is a machine that carries
out processes described in the Lisp language.  The first Lisp interpreter was
implemented by McCarthy with the help of colleagues and students in the
Artificial Intelligence Group of the \acronym{MIT} Research Laboratory of
Electronics and in the \acronym{MIT} Computation Center.\footnote{The
\textit{Lisp 1 Programmer's Manual} appeared in 1960, and the \textit{Lisp 1.5
Programmer's Manual} (\link{McCarthy et al. 1965}) was published in 1962.  The early history
of Lisp is described in \link{McCarthy 1978}.}  Lisp, whose name is an acronym for
LISt Processing, was designed to provide symbol-manipulating capabilities for
attacking programming problems such as the symbolic differentiation and
integration of algebraic expressions.  It included for this purpose new data
objects known as atoms and lists, which most strikingly set it apart from all
other languages of the period.

Lisp was not the product of a concerted design effort.  Instead, it evolved
informally in an experimental manner in response to users' needs and to
pragmatic implementation considerations.  Lisp's informal evolution has
continued through the years, and the community of Lisp users has traditionally
resisted attempts to promulgate any ``official'' definition of the language.
This evolution, together with the flexibility and elegance of the initial
conception, has enabled Lisp, which is the second oldest language in widespread
use today (only Fortran is older), to continually adapt to encompass the most
modern ideas about program design.  Thus, Lisp is by now a family of dialects,
which, while sharing most of the original features, may differ from one another
in significant ways.  The dialect of Lisp used in this book is called
Scheme.\footnote{The two dialects in which most major Lisp programs of the
1970s were written are MacLisp (\link{Moon 1978}; \link{Pitman 1983}), developed at the
\acronym{MIT} Project \acronym{MAC}, and Interlisp (\link{Teitelman 1974}), developed
at Bolt Beranek and Newman Inc. and the Xerox Palo Alto Research Center.
Portable Standard Lisp (\link{Hearn 1969}; \link{Griss 1981}) was a Lisp dialect designed to
be easily portable between different machines.  MacLisp spawned a number of
subdialects, such as Franz Lisp, which was developed at the University of
California at Berkeley, and Zetalisp (\link{Moon and Weinreb 1981}), which was based on a
special-purpose processor designed at the \acronym{MIT} Artificial Intelligence
Laboratory to run Lisp very efficiently.  The Lisp dialect used in this book,
called Scheme (\link{Steele and Sussman 1975}), was invented in 1975 by Guy Lewis Steele Jr. and
Gerald Jay Sussman of the \acronym{MIT} Artificial Intelligence Laboratory and
later reimplemented for instructional use at \acronym{MIT}.  Scheme became an
\acronym{IEEE} standard in 1990 (\link{IEEE 1990}).  The Common Lisp dialect
(\link{Steele 1982}, \link{Steele 1990}) was developed by the Lisp community to combine
features from the earlier Lisp dialects to make an industrial standard for
Lisp.  Common Lisp became an \acronym{ANSI} standard in 1994 (\link{ANSI 1994}).}

Because of its experimental character and its emphasis on symbol manipulation,
Lisp was at first very inefficient for numerical computations, at least in
comparison with Fortran.  Over the years, however, Lisp compilers have been
developed that translate programs into machine code that can perform numerical
computations reasonably efficiently.  And for special applications, Lisp has
been used with great effectiveness.\footnote{One such special application was a
breakthrough computation of scientific importance---an integration of the
motion of the Solar System that extended previous results by nearly two orders
of magnitude, and demonstrated that the dynamics of the Solar System is
chaotic.  This computation was made possible by new integration algorithms, a
special-purpose compiler, and a special-purpose computer all implemented with
the aid of software tools written in Lisp (\link{Abelson et al. 1992}; \link{Sussman and Wisdom 1992}).}  
Although Lisp has not yet overcome its old reputation as
hopelessly inefficient, Lisp is now used in many applications where efficiency
is not the central concern.  For example, Lisp has become a language of choice
for operating-system shell languages and for extension languages for editors
and computer-aided design systems.

\enlargethispage{\baselineskip}

If Lisp is not a mainstream language, why are we using it as the framework for
our discussion of programming?  Because the language possesses unique features
that make it an excellent medium for studying important programming constructs
and data structures and for relating them to the linguistic features that
support them.  The most significant of these features is the fact that Lisp
descriptions of processes, called \newterm{procedures}, can themselves be
represented and manipulated as Lisp data.  The importance of this is that there
are powerful program-design techniques that rely on the ability to blur the
traditional distinction between ``passive'' data and ``active'' processes.  As
we shall discover, Lisp's flexibility in handling procedures as data makes it
one of the most convenient languages in existence for exploring these
techniques.  The ability to represent procedures as data also makes Lisp an
excellent language for writing programs that must manipulate other programs as
data, such as the interpreters and compilers that support computer languages.
Above and beyond these considerations, programming in Lisp is great fun.



\section{The Elements of Programming}
\label{Section 1.1}

A powerful programming language is more than just a means for instructing a
computer to perform tasks.  The language also serves as a framework within
which we organize our ideas about processes.  Thus, when we describe a
language, we should pay particular attention to the means that the language
provides for combining simple ideas to form more complex ideas.  Every powerful
language has three mechanisms for accomplishing this:

\begin{itemize}

\item \textbf{primitive expressions},
which represent the simplest entities the language is concerned with,

\item \textbf{means of combination},
by which compound elements are built from simpler ones, and

\item \textbf{means of abstraction},
by which compound elements can be named and manipulated as units.

\end{itemize}

\noindent
In programming, we deal with two kinds of elements: procedures and data. (Later
we will discover that they are really not so distinct.)  Informally, data is
``stuff\( \kern0.1em \)'' that we want to manipulate, and procedures are descriptions of the
rules for manipulating the data.  Thus, any powerful programming language
should be able to describe primitive data and primitive procedures and should
have methods for combining and abstracting procedures and data.

In this chapter we will deal only with simple numerical data so that we can
focus on the rules for building procedures.\footnote{The characterization of
numbers as ``simple data'' is a barefaced bluff.  In fact, the treatment of
numbers is one of the trickiest and most confusing aspects of any programming
language.  Some typical issues involved are these: Some computer systems
distinguish \newterm{integers}, such as 2, from \newterm{real numbers}, such as
2.71.  Is the real number 2.00 different from the integer 2?  Are the
arithmetic operations used for integers the same as the operations used for
real numbers?  Does 6 divided by 2 produce 3, or 3.0?  How large a number can
we represent?  How many decimal places of accuracy can we represent?  Is the
range of integers the same as the range of real numbers?  Above and beyond
these questions, of course, lies a collection of issues concerning roundoff and
truncation errors---the entire science of numerical analysis.  Since our focus
in this book is on large-scale program design rather than on numerical
techniques, we are going to ignore these problems.  The numerical examples in
this chapter will exhibit the usual roundoff behavior that one observes when
using arithmetic operations that preserve a limited number of decimal places of
accuracy in noninteger operations.} In later chapters we will see that these
same rules allow us to build procedures to manipulate compound data as well.



\subsection{Expressions}
\label{Section 1.1.1}

One easy way to get started at programming is to examine some typical
interactions with an interpreter for the Scheme dialect of Lisp.  Imagine that
you are sitting at a computer terminal.  You type an \newterm{expression}, and
the interpreter responds by displaying the result of its \newterm{evaluating}
that expression.

One kind of primitive expression you might type is a number.  (More precisely,
the expression that you type consists of the numerals that represent the number
in base 10.)  If you present Lisp with a number

\begin{scheme}
486
\end{scheme}

\noindent
the interpreter will respond by printing\footnote{Throughout this book, when
we wish to emphasize the distinction between the input typed by the user and
the response printed by the interpreter, we will show the latter in slanted
characters.}

\begin{scheme}
~\textit{486}~
\end{scheme}

\noindent
Expressions representing numbers may be combined with an expression
representing a primitive procedure (such as \code{+} or \code{*}) to form a
compound expression that represents the application of the procedure to those
numbers.  For example:

\begin{scheme}
(+ 137 349)
~\textit{486}~
\end{scheme}

\begin{scheme}
(- 1000 334)
~\textit{666}~
\end{scheme}

\begin{scheme}
(* 5 99)
~\textit{495}~
\end{scheme}

\begin{scheme}
(/ 10 5)
~\textit{2}~
\end{scheme}

\begin{scheme}
(+ 2.7 10)
~\textit{12.7}~
\end{scheme}

\noindent
Expressions such as these, formed by delimiting a list of expressions within
parentheses in order to denote procedure application, are called
\newterm{combinations}.  The leftmost element in the list is called the
\newterm{operator}, and the other elements are called \newterm{operands}.  The
value of a combination is obtained by applying the procedure specified by the
operator to the \newterm{arguments} that are the values of the operands.

The convention of placing the operator to the left of the operands is known as
\newterm{prefix notation}, and it may be somewhat confusing at first because it
departs significantly from the customary mathematical convention.  Prefix
notation has several advantages, however.  One of them is that it can
accommodate procedures that may take an arbitrary number of arguments, as in
the following examples:

\begin{scheme}
(+ 21 35 12 7)
~\textit{75}~
\end{scheme}

\begin{scheme}
(* 25 4 12)
~\textit{1200}~
\end{scheme}

\noindent
No ambiguity can arise, because the operator is always the leftmost element and
the entire combination is delimited by the parentheses.

A second advantage of prefix notation is that it extends in a straightforward
way to allow combinations to be \textit{nested}, that is, to have combinations whose
elements are themselves combinations:

\begin{scheme}
(+ (* 3 5) (- 10 6))
~\textit{19}~
\end{scheme}

\noindent
There is no limit (in principle) to the depth of such nesting and to the
overall complexity of the expressions that the Lisp interpreter can evaluate.
It is we humans who get confused by still relatively simple expressions such as

\begin{scheme}
(+ (* 3 (+ (* 2 4) (+ 3 5))) (+ (- 10 7) 6))
\end{scheme}

\noindent
which the interpreter would readily evaluate to be 57.  We can help ourselves
by writing such an expression in the form

\begin{scheme}
(+ (* 3
      (+ (* 2 4)
         (+ 3 5)))
   (+ (- 10 7)
      6))
\end{scheme}

\noindent
following a formatting convention known as \newterm{pretty-printing}, in which
each long combination is written so that the operands are aligned vertically.
The resulting indentations display clearly the structure of the
expression.\footnote{Lisp systems typically provide features to aid the user in
formatting expressions.  Two especially useful features are one that
automatically indents to the proper pretty-print position whenever a new line
is started and one that highlights the matching left parenthesis whenever a
right parenthesis is typed.}

Even with complex expressions, the interpreter always operates in the same
basic cycle: It reads an expression from the terminal, evaluates the
expression, and prints the result.  This mode of operation is often expressed
by saying that the interpreter runs in a \newterm{read-eval-print loop}.
Observe in particular that it is not necessary to explicitly instruct the
interpreter to print the value of the expression.\footnote{Lisp obeys the
convention that every expression has a value. This convention, together with
the old reputation of Lisp as an inefficient language, is the source of the
quip by Alan Perlis (paraphrasing Oscar Wilde) that ``Lisp programmers know the
value of everything but the cost of nothing.''}

\subsection{Naming and the Environment}
\label{Section 1.1.2}

A critical aspect of a programming language is the means it provides for using
names to refer to computational objects.  We say that the name identifies a
\newterm{variable} whose \newterm{value} is the object.

In the Scheme dialect of Lisp, we name things with \code{define}.  Typing

\begin{scheme}
(define size 2)
\end{scheme}

\noindent
causes the interpreter to associate the value 2 with the name
\code{size}.\footnote{In this book, we do not show the interpreter's response
to evaluating definitions, since this is highly implementation-dependent.} Once
the name \code{size} has been associated with the number 2, we can refer to the
value 2 by name:

\begin{scheme}
size
~\textit{2}~
\end{scheme}

\begin{scheme}
(* 5 size)
~\textit{10}~
\end{scheme}

\noindent
Here are further examples of the use of \code{define}:

\begin{scheme}
(define pi 3.14159)
(define radius 10)
(* pi (* radius radius))
~\textit{314.159}~
(define circumference (* 2 pi radius))
circumference
~\textit{62.8318}~
\end{scheme}

\noindent
\code{Define} is our language's simplest means of abstraction, for it allows us
to use simple names to refer to the results of compound operations, such as the
\code{circumference} computed above.  In general, computational objects may
have very complex structures, and it would be extremely inconvenient to have to
remember and repeat their details each time we want to use them.  Indeed,
complex programs are constructed by building, step by step, computational
objects of increasing complexity. The interpreter makes this step-by-step
program construction particularly convenient because name-object associations
can be created incrementally in successive interactions.  This feature
encourages the incremental development and testing of programs and is largely
responsible for the fact that a Lisp program usually consists of a large number
of relatively simple procedures.

It should be clear that the possibility of associating values with symbols and
later retrieving them means that the interpreter must maintain some sort of
memory that keeps track of the name-object pairs.  This memory is called the
\newterm{environment} (more precisely the \newterm{global environment}, since
we will see later that a computation may involve a number of different
environments).\footnote{\link{Chapter 3} will show that this notion of
environment is crucial, both for understanding how the interpreter works and
for implementing interpreters.}

\subsection{Evaluating Combinations}
\label{Section 1.1.3}

One of our goals in this chapter is to isolate issues about thinking
procedurally.  As a case in point, let us consider that, in evaluating
combinations, the interpreter is itself following a procedure.

To evaluate a combination, do the following:

\begin{enumerate}

\item
Evaluate the subexpressions of the combination.

\item
Apply the procedure that is the value of the leftmost subexpression (the
operator) to the arguments that are the values of the other subexpressions (the
operands).

\end{enumerate}

\noindent
Even this simple rule illustrates some important points about processes in
general.  First, observe that the first step dictates that in order to
accomplish the evaluation process for a combination we must first perform the
evaluation process on each element of the combination.  Thus, the evaluation
rule is \newterm{recursive} in nature; that is, it includes, as one of its
steps, the need to invoke the rule itself.\footnote{It may seem strange that
the evaluation rule says, as part of the first step, that we should evaluate
the leftmost element of a combination, since at this point that can only be an
operator such as \code{+} or \code{*} representing a built-in primitive
procedure such as addition or multiplication.  We will see later that it is
useful to be able to work with combinations whose operators are themselves
compound expressions.}

Notice how succinctly the idea of recursion can be used to express what, in the
case of a deeply nested combination, would otherwise be viewed as a rather
complicated process.  For example, evaluating

\begin{scheme}
(* (+ 2 (* 4 6))
   (+ 3 5 7))
\end{scheme}

\noindent
requires that the evaluation rule be applied to four different combinations.
We can obtain a picture of this process by representing the combination in the
form of a tree, as shown in \link{Figure 1.1}.  Each combination is represented
by a node with branches corresponding to the operator and the operands of the
combination stemming from it.  The terminal nodes (that is, nodes with no
branches stemming from them) represent either operators or numbers.  Viewing
evaluation in terms of the tree, we can imagine that the values of the operands
percolate upward, starting from the terminal nodes and then combining at higher
and higher levels.  In general, we shall see that recursion is a very powerful
technique for dealing with hierarchical, treelike objects.  In fact, the
``percolate values upward'' form of the evaluation rule is an example of a
general kind of process known as \newterm{tree accumulation}.

\begin{figure}[tb]
\phantomsection\label{Figure 1.1}
\centering
\begin{comment}
\heading{Figure 1.1:} Tree representation, showing the value of each subcombination.

\begin{example}
   390
   /|\____________
  / |             \
 *  26            15
    /|\           /|\
   / | \         // \\
  +  2  24      / | | \
        /|\    +  3 5  7
       / | \
      *  4  6
\end{example}
\end{comment}
\includegraphics[width=31mm]{fig/chap1/Fig1.1g.pdf}
\begin{quote}
\heading{Figure 1.1:} Tree representation, showing the value of each subcombination.
\end{quote}
\end{figure}

Next, observe that the repeated application of the first step brings us to the
point where we need to evaluate, not combinations, but primitive expressions
such as numerals, built-in operators, or other names.  We take care of the
primitive cases by stipulating that

\begin{itemize}

\item
the values of numerals are the numbers that they name,

\item
the values of built-in operators are the machine instruction sequences that
carry out the corresponding operations, and

\item
the values of other names are the objects associated with those names in the
environment.

\end{itemize}

\noindent
We may regard the second rule as a special case of the third one by stipulating
that symbols such as \code{+} and \code{*} are also included in the global
environment, and are associated with the sequences of machine instructions that
are their ``values.''  The key point to notice is the role of the environment
in determining the meaning of the symbols in expressions.  In an interactive
language such as Lisp, it is meaningless to speak of the value of an expression
such as \code{(+ x 1)} without specifying any information about the environment
that would provide a meaning for the symbol \code{x} (or even for the symbol
\code{+}).  As we shall see in \link{Chapter 3}, the general notion of the
environment as providing a context in which evaluation takes place will play an
important role in our understanding of program execution.

Notice that the evaluation rule given above does not handle definitions.  For
instance, evaluating \code{(define x 3)} does not apply \code{define} to two
arguments, one of which is the value of the symbol \code{x} and the other of
which is 3, since the purpose of the \code{define} is precisely to associate
\code{x} with a value.  (That is, \code{(define x 3)} is not a combination.)

Such exceptions to the general evaluation rule are called \newterm{special
forms}.  \code{Define} is the only example of a special form that we have seen
so far, but we will meet others shortly.  Each special form has its own
evaluation rule. The various kinds of expressions (each with its associated
evaluation rule) constitute the syntax of the programming language.  In
comparison with most other programming languages, Lisp has a very simple
syntax; that is, the evaluation rule for expressions can be described by a
simple general rule together with specialized rules for a small number of
special forms.\footnote{Special syntactic forms that are simply convenient
alternative surface structures for things that can be written in more uniform
ways are sometimes called \newterm{syntactic sugar}, to use a phrase coined by
Peter Landin.  In comparison with users of other languages, Lisp programmers,
as a rule, are less concerned with matters of syntax.  (By contrast, examine
any Pascal manual and notice how much of it is devoted to descriptions of
syntax.)  This disdain for syntax is due partly to the flexibility of Lisp,
which makes it easy to change surface syntax, and partly to the observation
that many ``convenient'' syntactic constructs, which make the language less
uniform, end up causing more trouble than they are worth when programs become
large and complex.  In the words of Alan Perlis, ``Syntactic sugar causes
cancer of the semicolon.''}

\subsection{Compound Procedures}
\label{Section 1.1.4}

We have identified in Lisp some of the elements that must appear in any
powerful programming language:

\begin{itemize}

\item
Numbers and arithmetic operations are primitive data and procedures.

\item
Nesting of combinations provides a means of combining operations.

\item
Definitions that associate names with values provide a limited means of
abstraction.

\end{itemize}

\noindent
Now we will learn about \newterm{procedure definitions}, a much more powerful
abstraction technique by which a compound operation can be given a name and
then referred to as a unit.

We begin by examining how to express the idea of ``squaring.''  We might say,
``To square something, multiply it by itself.''  This is expressed in our
language as

\begin{scheme}
(define (square x) (* x x))
\end{scheme}

\noindent
We can understand this in the following way:

\begin{scheme}
(define (square    x)         (*      x         x))
  |        |       |           |      |         |
 To     square  something,  multiply  it  by  itself.
\end{scheme}

\noindent
We have here a \newterm{compound procedure}, which has been given the name
\code{square}.  The procedure represents the operation of multiplying something
by itself.  The thing to be multiplied is given a local name, \code{x}, which
plays the same role that a pronoun plays in natural language.  Evaluating the
definition creates this compound procedure and associates it with the name
\code{square}.\footnote{Observe that there are two different operations being
combined here: we are creating the procedure, and we are giving it the name
\code{square}.  It is possible, indeed important, to be able to separate these
two notions---to create procedures without naming them, and to give names to
procedures that have already been created.  We will see how to do this in
\link{Section 1.3.2}.}

The general form of a procedure definition is

\begin{scheme}
(define (~\( \dark \langle \)~~\var{\dark name}~~\( \dark \kern0.03em\rangle \)~ ~\( \dark \langle \)~~\var{\dark formal parameters}~~\( \dark \kern0.02em\rangle \)~) 
  ~\( \dark \langle\kern0.08em \)~~\var{\dark body}~~\( \dark \rangle \)~)
\end{scheme}

\noindent
The \( \langle\hbox{\sl name}\kern0.08em\rangle \) is a symbol to be associated with the procedure definition in
the environment.\footnote{Throughout this book, we will describe the general
syntax of expressions by using italic symbols delimited by angle
brackets---e.g., \( \langle \)\var{name}\( \kern0.08em\rangle \)---to denote the ``slots'' in the expression to be
filled in when such an expression is actually used.} The \( \langle\hbox{\sl formal parameters}\kern0.08em\rangle \) are the names used within the body of the procedure to refer to
the corresponding arguments of the procedure.  The \( \langle\hbox{\sl body}\kern0.08em\rangle \) is an
expression that will yield the value of the procedure application when the
formal parameters are replaced by the actual arguments to which the procedure
is applied.\footnote{More generally, the body of the procedure can be a
sequence of expressions.  In this case, the interpreter evaluates each
expression in the sequence in turn and returns the value of the final
expression as the value of the procedure application.}  The \( \langle \)\var{name}\( \kern0.08em\rangle \) and
the \( \langle \)\var{formal parameters}\( \kern0.08em\rangle \) are grouped within parentheses, just as they
would be in an actual call to the procedure being defined.

Having defined \code{square}, we can now use it:

\begin{scheme}
(square 21)
~\textit{441}~
(square (+ 2 5))
~\textit{49}~
(square (square 3))
~\textit{81}~
\end{scheme}

\noindent
We can also use \code{square} as a building block in defining other procedures.
For example, \( x^2 + y^2 \) can be expressed as

\begin{scheme}
(+ (square x) (square y))
\end{scheme}

\noindent
We can easily define a procedure \code{sum\-/of\-/squares} that, given any two
numbers as arguments, produces the sum of their squares:

\begin{scheme}
(define (sum-of-squares x y)
  (+ (square x) (square y)))
(sum-of-squares 3 4)
~\textit{25}~
\end{scheme}

\noindent
Now we can use \code{sum\-/of\-/squares} as a building block in constructing
further procedures:

\begin{scheme}
(define (f a)
  (sum-of-squares (+ a 1) (* a 2)))
(f 5)
~\textit{136}~
\end{scheme}

\noindent
Compound procedures are used in exactly the same way as primitive procedures.
Indeed, one could not tell by looking at the definition of
\code{sum\-/of\-/squares} given above whether \code{square} was built into the
interpreter, like \code{+} and \code{*}, or defined as a compound procedure.

\subsection{The Substitution Model for Procedure Application}
\label{Section 1.1.5}

To evaluate a combination whose operator names a compound procedure, the
interpreter follows much the same process as for combinations whose operators
name primitive procedures, which we described in \link{Section 1.1.3}.  That is,
the interpreter evaluates the elements of the combination and applies the
procedure (which is the value of the operator of the combination) to the
arguments (which are the values of the operands of the combination).

We can assume that the mechanism for applying primitive procedures to arguments
is built into the interpreter.  For compound procedures, the application
process is as follows:

\begin{quote}
To apply a compound procedure to arguments, evaluate the body of the procedure
with each formal parameter replaced by the corresponding argument.
\end{quote}

\noindent
To illustrate this process, let's evaluate the combination

\begin{scheme}
(f 5)
\end{scheme}

\noindent
where \code{f} is the procedure defined in \link{Section 1.1.4}.  We begin by
retrieving the body of \code{f}:

\begin{scheme}
(sum-of-squares (+ a 1) (* a 2))
\end{scheme}

\noindent
Then we replace the formal parameter \code{a} by the argument 5:

\begin{scheme}
(sum-of-squares (+ 5 1) (* 5 2))
\end{scheme}

\noindent
Thus the problem reduces to the evaluation of a combination with two operands
and an operator \code{sum\-/of\-/squares}.  Evaluating this combination involves
three subproblems.  We must evaluate the operator to get the procedure to be
applied, and we must evaluate the operands to get the arguments.  Now \code{(+
5 1)} produces 6 and \code{(* 5 2)} produces 10, so we must apply the
\code{sum\-/of\-/squares} procedure to 6 and 10.  These values are substituted for
the formal parameters \code{x} and \code{y} in the body of
\code{sum\-/of\-/squares}, reducing the expression to

\begin{scheme}
(+ (square 6) (square 10))
\end{scheme}

\noindent
If we use the definition of \code{square}, this reduces to

\begin{scheme}
(+ (* 6 6) (* 10 10))
\end{scheme}

\noindent
which reduces by multiplication to

\begin{scheme}
(+ 36 100)
\end{scheme}

\noindent
and finally to

\begin{scheme}
136
\end{scheme}

\noindent
The process we have just described is called the \newterm{substitution model}
for procedure application.  It can be taken as a model that determines the
``meaning'' of procedure application, insofar as the procedures in this chapter
are concerned.  However, there are two points that should be stressed:

\begin{itemize}

\item
The purpose of the substitution is to help us think about procedure
application, not to provide a description of how the interpreter really works.
Typical interpreters do not evaluate procedure applications by manipulating the
text of a procedure to substitute values for the formal parameters.  In
practice, the ``substitution'' is accomplished by using a local environment for
the formal parameters.  We will discuss this more fully in \link{Chapter 3} and
\link{Chapter 4} when we examine the implementation of an interpreter in detail.

\item
Over the course of this book, we will present a sequence of increasingly
elaborate models of how interpreters work, culminating with a complete
implementation of an interpreter and compiler in \link{Chapter 5}.  The
substitution model is only the first of these models---a way to get started
thinking formally about the evaluation process.  In general, when modeling
phenomena in science and engineering, we begin with simplified, incomplete
models.  As we examine things in greater detail, these simple models become
inadequate and must be replaced by more refined models.  The substitution model
is no exception.  In particular, when we address in \link{Chapter 3} the use of
procedures with ``mutable data,'' we will see that the substitution model
breaks down and must be replaced by a more complicated model of procedure
application.\footnote{Despite the simplicity of the substitution idea, it turns
out to be surprisingly complicated to give a rigorous mathematical definition
of the substitution process.  The problem arises from the possibility of
confusion between the names used for the formal parameters of a procedure and
the (possibly identical) names used in the expressions to which the procedure
may be applied.  Indeed, there is a long history of erroneous definitions of
\newterm{substitution} in the literature of logic and programming semantics.
See \link{Stoy 1977} for a careful discussion of substitution.}

\end{itemize}

\subsubsection*{Applicative order versus normal order}

According to the description of evaluation given in \link{Section 1.1.3}, the
interpreter first evaluates the operator and operands and then applies the
resulting procedure to the resulting arguments.  This is not the only way to
perform evaluation.  An alternative evaluation model would not evaluate the
operands until their values were needed.  Instead it would first substitute
operand expressions for parameters until it obtained an expression involving
only primitive operators, and would then perform the evaluation.  If we used
this method, the evaluation of \code{(f 5)} would proceed according to the
sequence of expansions

\begin{scheme}
(sum-of-squares (+ 5 1) (* 5 2))
(+   (square (+ 5 1))      (square (* 5 2))  )
(+   (* (+ 5 1) (+ 5 1))   (* (* 5 2) (* 5 2)))
\end{scheme}

\noindent
followed by the reductions

\begin{scheme}
(+      (* 6 6)      (* 10 10))
(+         36           100)
                136
\end{scheme}

\noindent
This gives the same answer as our previous evaluation model, but the process is
different.  In particular, the evaluations of \code{(+ 5 1)} and \code{(* 5 2)}
are each performed twice here, corresponding to the reduction of the expression
\code{(* x x)} with \code{x} replaced respectively by \code{(+ 5 1)} and
\code{(* 5 2)}.

This alternative ``fully expand and then reduce'' evaluation method is known as
\newterm{normal-order evaluation}, in contrast to the ``evaluate the arguments
and then apply'' method that the interpreter actually uses, which is called
\newterm{applicative-order evaluation}.  It can be shown that, for procedure
applications that can be modeled using substitution (including all the
procedures in the first two chapters of this book) and that yield legitimate
values, normal-order and applicative-order evaluation produce the same value.
(See \link{Exercise 1.5} for an instance of an ``illegitimate'' value where
normal-order and applicative-order evaluation do not give the same result.)

Lisp uses applicative-order evaluation, partly because of the additional
efficiency obtained from avoiding multiple evaluations of expressions such as
those illustrated with \code{(+ 5 1)} and \code{(* 5 2)} above and, more
significantly, because normal-order evaluation becomes much more complicated to
deal with when we leave the realm of procedures that can be modeled by
substitution.  On the other hand, normal-order evaluation can be an extremely
valuable tool, and we will investigate some of its implications in \link{Chapter
3} and \link{Chapter 4}.\footnote{In \link{Chapter 3} we will introduce
\newterm{stream processing}, which is a way of handling apparently ``infinite''
data structures by incorporating a limited form of normal-order evaluation.  In
\link{Section 4.2} we will modify the Scheme interpreter to produce a
normal-order variant of Scheme.}

\subsection{Conditional Expressions and Predicates}
\label{Section 1.1.6}

The expressive power of the class of procedures that we can define at this
point is very limited, because we have no way to make tests and to perform
different operations depending on the result of a test.  For instance, we
cannot define a procedure that computes the absolute value of a number by
testing whether the number is positive, negative, or zero and taking different
actions in the different cases according to the rule
\begin{comment}

\begin{example}
      /
      |   x  if x > 0
|x| = <   0  if x = 0
      |  -x  if x < 0
      \
\end{example}

\end{comment}
\begin{displaymath}
 |x| = \left\{ \begin{array}{r@{\quad \mathrm{if} \quad}l}	 	 
        x  &  x > 0, \\
	0  &  x = 0, \\
  \!\! -x  &  x < 0. \end{array} \right. 
\end{displaymath}
This construct is called a \newterm{case analysis}, and there is a special form
in Lisp for notating such a case analysis.  It is called \code{cond} (which
stands for ``conditional''), and it is used as follows:

\begin{scheme}
(define (abs x)
  (cond ((> x 0) x)
        ((= x 0) 0)
        ((< x 0) (- x))))
\end{scheme}

\noindent
The general form of a conditional expression is

\begin{scheme}
(cond (~\( \dark \langle{p_1}\kern0.08em\rangle \)~ ~\( \dark \langle{e_1}\kern0.08em\rangle \)~)
      (~\( \dark \langle{p_2}\kern0.08em\rangle \)~ ~\( \dark \langle{e_2}\kern0.08em\rangle \)~)
      ~\( \dots \)~
      (~\( \dark \langle{p_n}\kern0.08em\rangle \)~ ~\( \dark \langle{e_n}\kern0.08em\rangle \)~))
\end{scheme}

\noindent
consisting of the symbol \code{cond} followed by parenthesized pairs of
expressions

\begin{scheme}
(~\( \dark \langle{p}\kern0.08em\rangle \)~ ~\( \dark \langle{e}\kern0.08em\rangle \)~)
\end{scheme}

\noindent
called \newterm{clauses}. The first expression in each pair is a
\newterm{predicate}---that is, an expression whose value is interpreted as
either true or false.\footnote{``Interpreted as either true or false'' means
this: In Scheme, there are two distinguished values that are denoted by the
constants \code{\#t} and \code{\#f}.  When the interpreter checks a predicate's
value, it interprets \code{\#f} as false.  Any other value is treated as true.
(Thus, providing \code{\#t} is logically unnecessary, but it is convenient.)  In
this book we will use names \code{true} and \code{false}, which are associated
with the values \code{\#t} and \code{\#f} respectively.}

Conditional expressions are evaluated as follows.  The predicate \( \langle{p_1}\rangle \) is
evaluated first.  If its value is false, then \( \langle{p_2}\rangle \) is evaluated.  If
\( \langle{p_2}\rangle \)'s value is also false, then \( \langle{p_3}\rangle \) is evaluated.  This process
continues until a predicate is found whose value is true, in which case the
interpreter returns the value of the corresponding \newterm{consequent
expression} \( \langle{e}\rangle \) of the clause as the value of the conditional expression.
If none of the \( \langle{p}\rangle \)'s is found to be true, the value of the \code{cond} is
undefined.

The word \newterm{predicate} is used for procedures that return true or false,
as well as for expressions that evaluate to true or false.  The absolute-value
procedure \code{abs} makes use of the primitive predicates \code{>}, \code{<},
and \code{=}.\footnote{\code{Abs} also uses the ``minus'' operator \code{-},
which, when used with a single operand, as in \code{(- x)}, indicates
negation.} These take two numbers as arguments and test whether the first
number is, respectively, greater than, less than, or equal to the second
number, returning true or false accordingly.

Another way to write the absolute-value procedure is

\begin{scheme}
(define (abs x)
  (cond ((< x 0) (- x))
        (else x)))
\end{scheme}

\noindent
which could be expressed in English as ``If \( x \) is less than zero return 
\( -x; \) otherwise return \( x \).''  \code{Else} is a special symbol that can be
used in place of the \( \langle{p}\rangle \) in the final clause of a \code{cond}.  This
causes the \code{cond} to return as its value the value of the corresponding
\( \langle{e}\rangle \) whenever all previous clauses have been bypassed.  In fact, any
expression that always evaluates to a true value could be used as the \( \langle{p}\rangle \)
here.

Here is yet another way to write the absolute-value procedure:

\begin{scheme}
(define (abs x)
  (if (< x 0)
      (- x)
      x))
\end{scheme}

\noindent
This uses the special form \code{if}, a restricted type of conditional that can
be used when there are precisely two cases in the case analysis.  The general
form of an \code{if} expression is

\begin{scheme}
(if ~\( \dark \langle\kern0.07em \)~~\var{\dark predicate}~~\( \dark \kern0.06em\rangle \)~ ~\( \dark \langle\kern0.07em \)~~\var{\dark consequent}~~\( \dark \kern0.05em\rangle \)~ ~\( \dark \langle\kern0.06em \)~~\var{\dark alternative}~~\( \dark \kern0.06em\rangle \)~)
\end{scheme}

\noindent
To evaluate an \code{if} expression, the interpreter starts by evaluating the
\( \langle \)\var{predicate}\( \kern0.08em\rangle \) part of the expression.  If the \( \langle \)\var{predicate}\( \kern0.08em\rangle \) evaluates
to a true value, the interpreter then evaluates the \( \langle \)\var{consequent}\( \kern0.08em\rangle \) and
returns its value.  Otherwise it evaluates the \( \langle \)\var{alternative}\( \kern0.08em\rangle \) and returns
its value.\footnote{A minor difference between \code{if} and \code{cond} is
that the \( \langle{e}\kern0.08em\rangle \) part of each \code{cond} clause may be a sequence of
expressions.  If the corresponding \( \langle{p}\kern0.08em\rangle \) is found to be true, the
expressions \( \langle{e}\kern0.08em\rangle \) are evaluated in sequence and the value of the final
expression in the sequence is returned as the value of the \code{cond}.  In an
\code{if} expression, however, the \( \langle \)\var{consequent}\( \kern0.08em\rangle \) and \( \langle \)\var{alternative}\( \kern0.08em\rangle \)
must be single expressions.}

In addition to primitive predicates such as \code{<}, \code{=}, and \code{>},
there are logical composition operations, which enable us to construct compound
predicates.  The three most frequently used are these:

\begin{itemize}

\item
\( \hbox{\tt(and }\langle{e_1}\rangle\;\;\dots\;\;\langle{e_n}\rangle\hbox{\tt)} \)

The interpreter evaluates the expressions \( \langle{e}\kern0.08em\rangle \) one at a time, in
left-to-right order.  If any \( \langle{e}\kern0.08em\rangle \) evaluates to false, the value of the
\code{and} expression is false, and the rest of the \( \langle{e}\kern0.08em\rangle \)'s are not
evaluated.  If all \( \langle{e}\kern0.08em\rangle \)'s evaluate to true values, the value of the
\code{and} expression is the value of the last one.

\item
\( \hbox{\tt(or }\langle{e_1}\rangle\;\;\dots\;\;\langle{e_n}\rangle\hbox{\tt)} \)

The interpreter evaluates the expressions \( \langle{e}\kern0.08em\rangle \) one at a time, in
left-to-right order.  If any \( \langle{e}\kern0.08em\rangle \) evaluates to a true value, that value is
returned as the value of the \code{or} expression, and the rest of the
\( \langle{e}\kern0.08em\rangle \)'s are not evaluated.  If all \( \langle{e}\kern0.08em\rangle \)'s evaluate to false, the value
of the \code{or} expression is false.

\item
\( \hbox{\tt(not }\langle{e}\rangle\hbox{\tt)} \)

The value of a \code{not} expression is true when the expression \( \langle{e}\kern0.08em\rangle \)
evaluates to false, and false otherwise.

\end{itemize}

\noindent
Notice that \code{and} and \code{or} are special forms, not procedures, because
the subexpressions are not necessarily all evaluated.  \code{Not} is an
ordinary procedure.

As an example of how these are used, the condition that a number \( x \) be in
the range \( 5 < x < 10 \) may be expressed as

\begin{scheme}
(and (> x 5) (< x 10))
\end{scheme}

\noindent
As another example, we can define a predicate to test whether one number is
greater than or equal to another as

\begin{scheme}
(define (>= x y) (or (> x y) (= x y)))
\end{scheme}

\noindent
or alternatively as

\begin{scheme}
(define (>= x y) (not (< x y)))
\end{scheme}

\begin{quote}
\heading{\phantomsection\label{Exercise 1.1}Exercise 1.1:} Below is a sequence of expressions.
What is the result printed by the interpreter in response to each expression?
Assume that the sequence is to be evaluated in the order in which it is
presented.

\begin{scheme}
10
(+ 5 3 4)
(- 9 1)
(/ 6 2)
(+ (* 2 4) (- 4 6))
(define a 3)
(define b (+ a 1))
(+ a b (* a b))
(= a b)
(if (and (> b a) (< b (* a b)))
    b
    a)
\end{scheme}

\begin{scheme}
(cond ((= a 4) 6)
      ((= b 4) (+ 6 7 a))
      (else 25))
\end{scheme}

\begin{scheme}
(+ 2 (if (> b a) b a))
\end{scheme}

\begin{scheme}
(* (cond ((> a b) a)
         ((< a b) b)
         (else -1))
   (+ a 1))
\end{scheme}
\end{quote}

\begin{quote}
\heading{\phantomsection\label{Exercise 1.2}Exercise 1.2:} Translate the following expression
into prefix form:
\begin{comment}

\begin{example}
5 + 4 + (2 - (3 - (6 + 4/5)))
-----------------------------
       3(6 - 2)(2 - 7)
\end{example}

\end{comment}
\begin{displaymath}
{5 + 4 + (2 - (3 - (6 + {4\over5})))\over3(6 - 2)(2 - 7)}.
\end{displaymath}
\end{quote}

\begin{quote}
\heading{\phantomsection\label{Exercise 1.3}Exercise 1.3:} Define a procedure that takes three
numbers as arguments and returns the sum of the squares of the two larger
numbers.
\end{quote}

\begin{quote}
\heading{\phantomsection\label{Exercise 1.4}Exercise 1.4:} Observe that our model of
evaluation allows for combinations whose operators are compound expressions.
Use this observation to describe the behavior of the following procedure:

\begin{scheme}
(define (a-plus-abs-b a b)
  ((if (> b 0) + -) a b))
\end{scheme}
\end{quote}

\begin{quote}
\heading{\phantomsection\label{Exercise 1.5}Exercise 1.5:} Ben Bitdiddle has invented a test
to determine whether the interpreter he is faced with is using
applicative-order evaluation or normal-order evaluation.  He defines the
following two procedures:

\begin{scheme}
(define (p) (p))
(define (test x y)
  (if (= x 0) 0 y))
\end{scheme}

Then he evaluates the expression

\begin{scheme}
(test 0 (p))
\end{scheme}

What behavior will Ben observe with an interpreter that uses applicative-order
evaluation?  What behavior will he observe with an interpreter that uses
normal-order evaluation?  Explain your answer.  (Assume that the evaluation
rule for the special form \code{if} is the same whether the interpreter is
using normal or applicative order: The predicate expression is evaluated first,
and the result determines whether to evaluate the consequent or the alternative
expression.)
\end{quote}

\subsection{Example: Square Roots by Newton's Method}
\label{Section 1.1.7}

Procedures, as introduced above, are much like ordinary mathematical functions.
They specify a value that is determined by one or more parameters.  But there
is an important difference between mathematical functions and computer
procedures.  Procedures must be effective.

As a case in point, consider the problem of computing square roots.  We can
define the square-root function as
\begin{comment}

\begin{example}
sqrt(x) = the y such that y >= 0 and y^2 = x
\end{example}

\end{comment}
\begin{displaymath}
\sqrt{x}\;\; = {\rm\;\; the\;\;} y 
{\rm\;\; such\;\; that\;\;} y \ge 0 {\rm\;\; and\;\;} y^2 = x.
\end{displaymath}
This describes a perfectly legitimate mathematical function.  We could use it
to recognize whether one number is the square root of another, or to derive
facts about square roots in general.  On the other hand, the definition does
not describe a procedure.  Indeed, it tells us almost nothing about how to
actually find the square root of a given number.  It will not help matters to
rephrase this definition in pseudo-Lisp:

\begin{scheme}
(define (sqrt x)
  (the y (and (>= y 0) 
              (= (square y) x))))
\end{scheme}

\noindent
This only begs the question.

The contrast between function and procedure is a reflection of the general
distinction between describing properties of things and describing how to do
things, or, as it is sometimes referred to, the distinction between declarative
knowledge and imperative knowledge.  In mathematics we are usually concerned
with declarative (what is) descriptions, whereas in computer science we are
usually concerned with imperative (how to) descriptions.\footnote{Declarative
and imperative descriptions are intimately related, as indeed are mathematics
and computer science.  For instance, to say that the answer produced by a
program is ``correct'' is to make a declarative statement about the program.
There is a large amount of research aimed at establishing techniques for
proving that programs are correct, and much of the technical difficulty of this
subject has to do with negotiating the transition between imperative statements
(from which programs are constructed) and declarative statements (which can be
used to deduce things).  In a related vein, an important current area in
programming-language design is the exploration of so-called very high-level
languages, in which one actually programs in terms of declarative statements.
The idea is to make interpreters sophisticated enough so that, given ``what
is'' knowledge specified by the programmer, they can generate ``how to''
knowledge automatically.  This cannot be done in general, but there are
important areas where progress has been made.  We shall revisit this idea in
\link{Chapter 4}.}

How does one compute square roots?  The most common way is to use Newton's
method of successive approximations, which says that whenever we have a guess
\( y \) for the value of the square root of a number \( x \), we can perform a
simple manipulation to get a better guess (one closer to the actual square
root) by averaging \( y \) with \( x / y \).\footnote{This square-root algorithm
is actually a special case of Newton's method, which is a general technique for
finding roots of equations.  The square-root algorithm itself was developed by
Heron of Alexandria in the first century \acronym{A.D.}  We will see how to
express the general Newton's method as a Lisp procedure in \link{Section 1.3.4}.} 
For example, we can compute the square root of 2 as follows.
Suppose our initial guess is 1:

\vspace{-0.8em}
\begin{smallexample}
Guess       Quotient                 Average
1           (2/1) = 2                ((2 + 1)/2) = 1.5
1.5         (2/1.5) = 1.3333         ((1.3333 + 1.5)/2) = 1.4167
1.4167      (2/1.4167) = 1.4118      ((1.4167 + 1.4118)/2) = 1.4142
1.4142      ...                      ...
\end{smallexample}

\noindent
Continuing this process, we obtain better and better approximations to the
square root.

Now let's formalize the process in terms of procedures.  We start with a value
for the radicand (the number whose square root we are trying to compute) and a
value for the guess.  If the guess is good enough for our purposes, we are
done; if not, we must repeat the process with an improved guess.  We write this
basic strategy as a procedure:

\begin{scheme}
(define (sqrt-iter guess x)
  (if (good-enough? guess x)
      guess
      (sqrt-iter (improve guess x) x)))
\end{scheme}

\noindent
A guess is improved by averaging it with the quotient of the radicand and the
old guess:

\begin{scheme}
(define (improve guess x)
  (average guess (/ x guess)))
\end{scheme}

\noindent
where

\begin{scheme}
(define (average x y)
  (/ (+ x y) 2))
\end{scheme}

\noindent
We also have to say what we mean by ``good enough.''  The following will do for
illustration, but it is not really a very good test.  (See \link{Exercise 1.7}.)  
The idea is to improve the answer until it is close
enough so that its square differs from the radicand by less than a
predetermined tolerance (here 0.001):\footnote{We will usually give predicates
names ending with question marks, to help us remember that they are predicates.
This is just a stylistic convention.  As far as the interpreter is concerned,
the question mark is just an ordinary character.}

\begin{scheme}
(define (good-enough? guess x)
  (< (abs (- (square guess) x)) 0.001))
\end{scheme}

\noindent
Finally, we need a way to get started.  For instance, we can always guess that
the square root of any number is 1:\footnote{Observe that we express our
initial guess as 1.0 rather than 1.  This would not make any difference in many
Lisp implementations.  \acronym{MIT} Scheme, however, distinguishes between
exact integers and decimal values, and dividing two integers produces a
rational number rather than a decimal.  For example, dividing 10 by 6 yields
5/3, while dividing 10.0 by 6.0 yields 1.6666666666666667.  (We will learn how
to implement arithmetic on rational numbers in \link{Section 2.1.1}.)  If we
start with an initial guess of 1 in our square-root program, and \( x \) is an
exact integer, all subsequent values produced in the square-root computation
will be rational numbers rather than decimals.  Mixed operations on rational
numbers and decimals always yield decimals, so starting with an initial guess
of 1.0 forces all subsequent values to be decimals.}

\begin{scheme}
(define (sqrt x)
  (sqrt-iter 1.0 x))
\end{scheme}

\noindent
If we type these definitions to the interpreter, we can use \code{sqrt} just as
we can use any procedure:

\begin{scheme}
(sqrt 9)
~\textit{3.00009155413138}~

(sqrt (+ 100 37))
~\textit{11.704699917758145}~

(sqrt (+ (sqrt 2) (sqrt 3)))
~\textit{1.7739279023207892}~

(square (sqrt 1000))
~\textit{1000.000369924366}~
\end{scheme}

\noindent
The \code{sqrt} program also illustrates that the simple procedural language we
have introduced so far is sufficient for writing any purely numerical program
that one could write in, say, C or Pascal.  This might seem surprising, since
we have not included in our language any iterative (looping) constructs that
direct the computer to do something over and over again.  \code{Sqrt\-/iter}, on
the other hand, demonstrates how iteration can be accomplished using no special
construct other than the ordinary ability to call a procedure.\footnote{Readers
who are worried about the efficiency issues involved in using procedure calls
to implement iteration should note the remarks on ``tail recursion'' in 
\link{Section 1.2.1}.}

\begin{quote}
\heading{\phantomsection\label{Exercise 1.6}Exercise 1.6:} Alyssa P. Hacker doesn't see why
\code{if} needs to be provided as a special form.  ``Why can't I just define it
as an ordinary procedure in terms of \code{cond}?'' she asks.  Alyssa's friend
Eva Lu Ator claims this can indeed be done, and she defines a new version of
\code{if}:

\begin{scheme}
(define (new-if predicate then-clause else-clause)
  (cond (predicate then-clause)
        (else else-clause)))
\end{scheme}

Eva demonstrates the program for Alyssa:

\begin{scheme}
(new-if (= 2 3) 0 5)
~\textit{5}~
(new-if (= 1 1) 0 5)
~\textit{}~
\end{scheme}

Delighted, Alyssa uses \code{new\-/if} to rewrite the square-root program:

\begin{scheme}
(define (sqrt-iter guess x)
  (new-if (good-enough? guess x)
          guess
          (sqrt-iter (improve guess x) x)))
\end{scheme}

What happens when Alyssa attempts to use this to compute square roots?
Explain.
\end{quote}

\begin{quote}
\heading{\phantomsection\label{Exercise 1.7}Exercise 1.7:} The \code{good\-/enough?} test used
in computing square roots will not be very effective for finding the square
roots of very small numbers.  Also, in real computers, arithmetic operations
are almost always performed with limited precision.  This makes our test
inadequate for very large numbers.  Explain these statements, with examples
showing how the test fails for small and large numbers.  An alternative
strategy for implementing \code{good\-/enough?} is to watch how \code{guess}
changes from one iteration to the next and to stop when the change is a very
small fraction of the guess.  Design a square-root procedure that uses this
kind of end test.  Does this work better for small and large numbers?
\end{quote}

\begin{quote}
\heading{\phantomsection\label{Exercise 1.8}Exercise 1.8:} Newton's method for cube roots is
based on the fact that if \( y \) is an approximation to the cube root of \( x \),
then a better approximation is given by the value
\begin{comment}

\begin{example}
x/y^2 + 2y
----------
    3
\end{example}

\end{comment}
\begin{displaymath}
{{x / y^2} + 2y \over 3}.
\end{displaymath}
\noindent
Use this formula to implement a cube-root procedure analogous to the
square-root procedure.  (In \link{Section 1.3.4} we will see how to implement
Newton's method in general as an abstraction of these square-root and cube-root
procedures.)
\end{quote}

\subsection{Procedures as Black-Box Abstractions}
\label{Section 1.1.8}

\code{Sqrt} is our first example of a process defined by a set of mutually
defined procedures.  Notice that the definition of \code{sqrt\-/iter} is
\newterm{recursive}; that is, the procedure is defined in terms of itself.  The
idea of being able to define a procedure in terms of itself may be disturbing;
it may seem unclear how such a ``circular'' definition could make sense at all,
much less specify a well-defined process to be carried out by a computer.  This
will be addressed more carefully in \link{Section 1.2}.  But first let's
consider some other important points illustrated by the \code{sqrt} example.

Observe that the problem of computing square roots breaks up naturally into a
number of subproblems: how to tell whether a guess is good enough, how to
improve a guess, and so on.  Each of these tasks is accomplished by a separate
procedure.  The entire \code{sqrt} program can be viewed as a cluster of
procedures (shown in \link{Figure 1.2}) that mirrors the decomposition of the
problem into subproblems.

\begin{figure}[tb]
\phantomsection\label{Figure 1.2}
\centering
\begin{comment}
\heading{Figure 1.2:} Procedural decomposition of the \code{sqrt} program.

\begin{example}
                       sqrt
                        |
                    sqrt-iter
                    /       \
            good-enough    improve
              /     \          \
          square    abs      average
\end{example}
\end{comment}
\includegraphics[width=44mm]{fig/chap1/Fig1.2.pdf}
\begin{quote}
\heading{Figure 1.2:} Procedural decomposition of the \code{sqrt} program.
\end{quote}
\end{figure}

The importance of this decomposition strategy is not simply that one is
dividing the program into parts.  After all, we could take any large program
and divide it into parts---the first ten lines, the next ten lines, the next
ten lines, and so on.  Rather, it is crucial that each procedure accomplishes
an identifiable task that can be used as a module in defining other procedures.
For example, when we define the \code{good\-/enough?} procedure in terms of
\code{square}, we are able to regard the \code{square} procedure as a ``black
box.''  We are not at that moment concerned with \emph{how} the procedure
computes its result, only with the fact that it computes the square.  The
details of how the square is computed can be suppressed, to be considered at a
later time.  Indeed, as far as the \code{good\-/enough?} procedure is concerned,
\code{square} is not quite a procedure but rather an abstraction of a
procedure, a so-called \newterm{procedural abstraction}.  At this level of
abstraction, any procedure that computes the square is equally good.

Thus, considering only the values they return, the following two procedures for
squaring a number should be indistinguishable.  Each takes a numerical argument
and produces the square of that number as the value.\footnote{It is not even
clear which of these procedures is a more efficient implementation.  This
depends upon the hardware available.  There are machines for which the
``obvious'' implementation is the less efficient one.  Consider a machine that
has extensive tables of logarithms and antilogarithms stored in a very
efficient manner.}

\begin{scheme}
(define (square x) (* x x))
(define (square x) (exp (double (log x))))
(define (double x) (+ x x))
\end{scheme}

\noindent
So a procedure definition should be able to suppress detail.  The users of the
procedure may not have written the procedure themselves, but may have obtained
it from another programmer as a black box.  A user should not need to know how
the procedure is implemented in order to use it.

\subsubsection*{Local names}

One detail of a procedure's implementation that should not matter to the user
of the procedure is the implementer's choice of names for the procedure's
formal parameters.  Thus, the following procedures should not be
distinguishable:

\begin{scheme}
(define (square x) (* x x))
(define (square y) (* y y))
\end{scheme}

\noindent
This principle---that the meaning of a procedure should be independent of the
parameter names used by its author---seems on the surface to be self-evident,
but its consequences are profound.  The simplest consequence is that the
parameter names of a procedure must be local to the body of the procedure.  For
example, we used \code{square} in the definition of \code{good\-/enough?} in our
square-root procedure:

\begin{scheme}
(define (good-enough? guess x)
  (< (abs (- (square guess) x))
     0.001))
\end{scheme}

\noindent
The intention of the author of \code{good\-/enough?} is to determine if the
square of the first argument is within a given tolerance of the second
argument.  We see that the author of \code{good\-/enough?} used the name
\code{guess} to refer to the first argument and \code{x} to refer to the second
argument.  The argument of \code{square} is \code{guess}.  If the author of
\code{square} used \code{x} (as above) to refer to that argument, we see that
the \code{x} in \code{good\-/enough?} must be a different \code{x} than the one
in \code{square}.  Running the procedure \code{square} must not affect the
value of \code{x} that is used by \code{good\-/enough?}, because that value of
\code{x} may be needed by \code{good\-/enough?} after \code{square} is done
computing.

If the parameters were not local to the bodies of their respective procedures,
then the parameter \code{x} in \code{square} could be confused with the
parameter \code{x} in \code{good\-/enough?}, and the behavior of
\code{good\-/enough?} would depend upon which version of \code{square} we used.
Thus, \code{square} would not be the black box we desired.

A formal parameter of a procedure has a very special role in the procedure
definition, in that it doesn't matter what name the formal parameter has.  Such
a name is called a \newterm{bound variable}, and we say that the procedure
definition \newterm{binds} its formal parameters.  The meaning of a procedure
definition is unchanged if a bound variable is consistently renamed throughout
the definition.\footnote{The concept of consistent renaming is actually subtle
and difficult to define formally.  Famous logicians have made embarrassing
errors here.}  If a variable is not bound, we say that it is \newterm{free}.
The set of expressions for which a binding defines a name is called the
\newterm{scope} of that name.  In a procedure definition, the bound variables
declared as the formal parameters of the procedure have the body of the
procedure as their scope.

In the definition of \code{good\-/enough?} above, \code{guess} and \code{x} are
bound variables but \code{<}, \code{-}, \code{abs}, and \code{square} are free.
The meaning of \code{good\-/enough?} should be independent of the names we choose
for \code{guess} and \code{x} so long as they are distinct and different from
\code{<}, \code{-}, \code{abs}, and \code{square}.  (If we renamed \code{guess}
to \code{abs} we would have introduced a bug by \newterm{capturing} the
variable \code{abs}.  It would have changed from free to bound.)  The meaning
of \code{good\-/enough?} is not independent of the names of its free variables,
however.  It surely depends upon the fact (external to this definition) that
the symbol \code{abs} names a procedure for computing the absolute value of a
number.  \code{Good\-/enough?} will compute a different function if we substitute
\code{cos} for \code{abs} in its definition.

\subsubsection*{Internal definitions and block structure}

We have one kind of name isolation available to us so far: The formal
parameters of a procedure are local to the body of the procedure.  The
square-root program illustrates another way in which we would like to control
the use of names.  The existing program consists of separate procedures:

\begin{scheme}
(define (sqrt x)
  (sqrt-iter 1.0 x))
(define (sqrt-iter guess x)
  (if (good-enough? guess x)
      guess
      (sqrt-iter (improve guess x) x)))
(define (good-enough? guess x)
  (< (abs (- (square guess) x)) 0.001))
(define (improve guess x)
  (average guess (/ x guess)))
\end{scheme}

\noindent
The problem with this program is that the only procedure that is important to
users of \code{sqrt} is \code{sqrt}.  The other procedures (\code{sqrt\-/iter},
\code{good\-/enough?}, and \code{improve}) only clutter up their minds.  They may
not define any other procedure called \code{good\-/enough?} as part of another
program to work together with the square-root program, because \code{sqrt}
needs it.  The problem is especially severe in the construction of large
systems by many separate programmers.  For example, in the construction of a
large library of numerical procedures, many numerical functions are computed as
successive approximations and thus might have procedures named
\code{good\-/enough?} and \code{improve} as auxiliary procedures.  We would like
to localize the subprocedures, hiding them inside \code{sqrt} so that
\code{sqrt} could coexist with other successive approximations, each having its
own private \code{good\-/enough?} procedure.  To make this possible, we allow a
procedure to have internal definitions that are local to that procedure.  For
example, in the square-root problem we can write

\begin{scheme}
(define (sqrt x)
  (define (good-enough? guess x)
    (< (abs (- (square guess) x)) 0.001))
  (define (improve guess x) (average guess (/ x guess)))
  (define (sqrt-iter guess x)
    (if (good-enough? guess x)
        guess
        (sqrt-iter (improve guess x) x)))
  (sqrt-iter 1.0 x))
\end{scheme}

\noindent
Such nesting of definitions, called \newterm{block structure}, is basically the
right solution to the simplest name-packaging problem.  But there is a better
idea lurking here.  In addition to internalizing the definitions of the
auxiliary procedures, we can simplify them.  Since \code{x} is bound in the
definition of \code{sqrt}, the procedures \code{good\-/enough?}, \code{improve},
and \code{sqrt\-/iter}, which are defined internally to \code{sqrt}, are in the
scope of \code{x}.  Thus, it is not necessary to pass \code{x} explicitly to
each of these procedures.  Instead, we allow \code{x} to be a free variable in
the internal definitions, as shown below. Then \code{x} gets its value from the
argument with which the enclosing procedure \code{sqrt} is called.  This
discipline is called \newterm{lexical scoping}.\footnote{Lexical scoping dictates 
that free variables in a procedure are taken to refer to bindings made by enclosing 
procedure definitions; that is, they are looked up in the environment in which the
procedure was defined. We will see how this works in detail in chapter 3 when we study 
environments and the detailed behavior of the interpreter.\label{Footnote 28}}

\begin{scheme}
(define (sqrt x)
  (define (good-enough? guess)
    (< (abs (- (square guess) x)) 0.001))
  (define (improve guess)
    (average guess (/ x guess)))
  (define (sqrt-iter guess)
    (if (good-enough? guess)
        guess
        (sqrt-iter (improve guess))))
  (sqrt-iter 1.0))
\end{scheme}

\enlargethispage{\baselineskip}

\noindent
We will use block structure extensively to help us break up large
programs into tractable pieces.\footnote{Embedded definitions must come
first in a procedure body. The management is not responsible for the
consequences of running programs that intertwine definition and use.}
The idea of block structure originated with the programming language
Algol 60. It appears in most advanced programming languages and is an
important tool for helping to organize the construction of large
programs.

\section{Procedures and the Processes They \mbox{Generate}}
\label{Section 1.2}

We have now considered the elements of programming: We have used primitive
arithmetic operations, we have combined these operations, and we have
abstracted these composite operations by defining them as compound procedures.
But that is not enough to enable us to say that we know how to program.  Our
situation is analogous to that of someone who has learned the rules for how the
pieces move in chess but knows nothing of typical openings, tactics, or
strategy.  Like the novice chess player, we don't yet know the common patterns
of usage in the domain.  We lack the knowledge of which moves are worth making
(which procedures are worth defining).  We lack the experience to predict the
consequences of making a move (executing a procedure).

The ability to visualize the consequences of the actions under consideration is
crucial to becoming an expert programmer, just as it is in any synthetic,
creative activity.  In becoming an expert photographer, for example, one must
learn how to look at a scene and know how dark each region will appear on a
print for each possible choice of exposure and development conditions.  Only
then can one reason backward, planning framing, lighting, exposure, and
development to obtain the desired effects.  So it is with programming, where we
are planning the course of action to be taken by a process and where we control
the process by means of a program.  To become experts, we must learn to
visualize the processes generated by various types of procedures.  Only after
we have developed such a skill can we learn to reliably construct programs that
exhibit the desired behavior.

A procedure is a pattern for the \newterm{local evolution} of a computational
process.  It specifies how each stage of the process is built upon the previous
stage.  We would like to be able to make statements about the overall, or
\newterm{global}, behavior of a process whose local evolution has been
specified by a procedure.  This is very difficult to do in general, but we can
at least try to describe some typical patterns of process evolution.

In this section we will examine some common ``shapes'' for processes generated
by simple procedures.  We will also investigate the rates at which these
processes consume the important computational resources of time and space.  The
procedures we will consider are very simple.  Their role is like that played by
test patterns in photography: as oversimplified prototypical patterns, rather
than practical examples in their own right.



\subsection{Linear Recursion and Iteration}
\label{Section 1.2.1}

We begin by considering the factorial function, defined by
\begin{comment}

\begin{example}
n! = n * (n - 1) * (n - 2) ... 3 * 2 * 1
\end{example}

\end{comment}
\begin{displaymath}
n! = n \cdot (n - 1) \cdot (n - 2) \cdots 3 \cdot 2 \cdot 1.
\end{displaymath}
There are many ways to compute factorials.  One way is to make use of the
observation that \( n! \) is equal to \( n \) times \( (n - 1)! \) for any positive
integer \( n \):
\begin{comment}

\begin{example}
n! = n * [(n - 1) * (n - 2) ... 3 * 2 * 1] = n * (n - 1)!
\end{example}

\end{comment}
\begin{displaymath}
n! = n \cdot [(n - 1) \cdot (n - 2) \cdots 3 \cdot 2 \cdot 1] = n \cdot (n - 1)!.
\end{displaymath}
Thus, we can compute \( n! \) by computing \( (n - 1)! \) and multiplying the
result by \( n \).  If we add the stipulation that 1! is equal to 1, this
observation translates directly into a procedure:

\begin{scheme}
(define (factorial n)
  (if (= n 1)
      1
      (* n (factorial (- n 1)))))
\end{scheme}

\noindent
We can use the substitution model of \link{Section 1.1.5} to watch this
procedure in action computing 6!, as shown in \link{Figure 1.3}.

\begin{figure}[tb]
\phantomsection\label{Figure 1.3}
\centering
\begin{comment}
\heading{Figure 1.3:} A linear recursive process for computing 6!.

\begin{example}
(factorial 6)        ----------------
(* 6 (factorial 5))                   \
(* 6 (* 5 (factorial 4)))               \
(* 6 (* 5 (* 4 (factorial 3))))           \
(* 6 (* 5 (* 4 (* 3 (factorial 2)))))       \
(* 6 (* 5 (* 4 (* 3 (* 2 (factorial 1))))))  |
(* 6 (* 5 (* 4 (* 3 (* 2 1)))))             /
(* 6 (* 5 (* 4 (* 3 2))))                 /
(* 6 (* 5 (* 4 6)))                     /
(* 6 (* 5 24))                        /
(* 6 120)                           /
720          <---------------------
\end{example}
\end{comment}
\includegraphics[width=82mm]{fig/chap1/Fig1.3c.pdf}
\par\bigskip
\noindent
\heading{Figure 1.3:} A linear recursive process for computing 6!.
\end{figure}

Now let's take a different perspective on computing factorials.  We could
describe a rule for computing \( n! \) by specifying that we first multiply 1 by
2, then multiply the result by 3, then by 4, and so on until we reach \( n \).
More formally, we maintain a running product, together with a counter that
counts from 1 up to \( n \).  We can describe the computation by saying that the
counter and the product simultaneously change from one step to the next
according to the rule

\begin{scheme}
product ~\( \dark \gets \)~ counter * product
counter ~\( \dark \gets \)~ counter + 1
\end{scheme}

\noindent
and stipulating that \( n! \) is the value of the product when the counter
exceeds \( n \).

\enlargethispage{\baselineskip}

\begin{figure}[tb]
\phantomsection\label{Figure 1.4}
\centering
\begin{comment}
\heading{Figure 1.4:} A linear iterative process for computing 6!.

\begin{example}
(factorial 6)   -----.
(fact-iter   1 1 6)  |
(fact-iter   1 2 6)  |
(fact-iter   2 3 6)  |
(fact-iter   6 4 6)  |
(fact-iter  24 5 6)  |
(fact-iter 120 6 6)  |
(fact-iter 720 7 6)  V
720
\end{example}
\end{comment}
\includegraphics[width=36mm]{fig/chap1/Fig1.4c.pdf}
\par\bigskip
\noindent
\heading{Figure 1.4:} A linear iterative process for computing 6!.
\end{figure}

Once again, we can recast our description as a procedure for computing
factorials:\footnote{In a real program we would probably use the block
structure introduced in the last section to hide the definition of
\code{fact\-/iter}:

\begin{smallscheme}
(define (factorial n)
  (define (iter product counter)
    (if (> counter n)
        product
        (iter (* counter product)
              (+ counter 1))))
  (iter 1 1))
\end{smallscheme}

We avoided doing this here so as to minimize the number of things to think
about at once.}

\begin{scheme}
(define (factorial n)
  (fact-iter 1 1 n))
(define (fact-iter product counter max-count)
  (if (> counter max-count)
      product
      (fact-iter (* counter product)
                 (+ counter 1)
                 max-count)))
\end{scheme}

\noindent
As before, we can use the substitution model to visualize the process of
computing 6!, as shown in \link{Figure 1.4}.

Compare the two processes.  From one point of view, they seem hardly different
at all.  Both compute the same mathematical function on the same domain, and
each requires a number of steps proportional to \( n \) to compute \( n! \).
Indeed, both processes even carry out the same sequence of multiplications,
obtaining the same sequence of partial products.  On the other hand, when we
consider the ``shapes'' of the two processes, we find that they evolve quite
differently.

Consider the first process.  The substitution model reveals a shape of
expansion followed by contraction, indicated by the arrow in \link{Figure 1.3}.
The expansion occurs as the process builds up a chain of \newterm{deferred
operations} (in this case, a chain of multiplications).  The contraction occurs
as the operations are actually performed.  This type of process, characterized
by a chain of deferred operations, is called a \newterm{recursive process}.
Carrying out this process requires that the interpreter keep track of the
operations to be performed later on.  In the computation of \( n! \), the length
of the chain of deferred multiplications, and hence the amount of information
needed to keep track of it, grows linearly with \( n \) (is proportional to
\( n \)), just like the number of steps.  Such a process is called a
\newterm{linear recursive process}.

By contrast, the second process does not grow and shrink.  At each step, all we
need to keep track of, for any \( n \), are the current values of the variables
\code{product}, \code{counter}, and \code{max\-/count}.  We call this an
\newterm{iterative process}.  In general, an iterative process is one whose
state can be summarized by a fixed number of \newterm{state variables},
together with a fixed rule that describes how the state variables should be
updated as the process moves from state to state and an (optional) end test
that specifies conditions under which the process should terminate.  In
computing \( n! \), the number of steps required grows linearly with \( n \).  Such
a process is called a \newterm{linear iterative process}.

The contrast between the two processes can be seen in another way.  In the
iterative case, the program variables provide a complete description of the
state of the process at any point.  If we stopped the computation between
steps, all we would need to do to resume the computation is to supply the
interpreter with the values of the three program variables.  Not so with the
recursive process.  In this case there is some additional ``hidden''
information, maintained by the interpreter and not contained in the program
variables, which indicates ``where the process is'' in negotiating the chain of
deferred operations.  The longer the chain, the more information must be
maintained.\footnote{When we discuss the implementation of procedures on
register machines in \link{Chapter 5}, we will see that any iterative process
can be realized ``in hardware'' as a machine that has a fixed set of registers
and no auxiliary memory.  In contrast, realizing a recursive process requires a
machine that uses an auxiliary data structure known as a \newterm{stack}.}

In contrasting iteration and recursion, we must be careful not to confuse the
notion of a recursive \newterm{process} with the notion of a recursive
\newterm{procedure}.  When we describe a procedure as recursive, we are
referring to the syntactic fact that the procedure definition refers (either
directly or indirectly) to the procedure itself.  But when we describe a
process as following a pattern that is, say, linearly recursive, we are
speaking about how the process evolves, not about the syntax of how a procedure
is written.  It may seem disturbing that we refer to a recursive procedure such
as \code{fact\-/iter} as generating an iterative process.  However, the process
really is iterative: Its state is captured completely by its three state
variables, and an interpreter need keep track of only three variables in order
to execute the process.

\enlargethispage{\baselineskip}

One reason that the distinction between process and procedure may be confusing
is that most implementations of common languages (including Ada, Pascal, and C)
are designed in such a way that the interpretation of any recursive procedure
consumes an amount of memory that grows with the number of procedure calls,
even when the process described is, in principle, iterative.  As a consequence,
these languages can describe iterative processes only by resorting to
special-purpose ``looping constructs'' such as \code{do}, \code{repeat},
\code{until}, \code{for}, and \code{while}.  The implementation of Scheme we
shall consider in \link{Chapter 5} does not share this defect.  It will execute
an iterative process in constant space, even if the iterative process is
described by a recursive procedure.  An implementation with this property is
called \newterm{tail-recursive}.  With a tail-recursive implementation,
iteration can be expressed using the ordinary procedure call mechanism, so that
special iteration constructs are useful only as syntactic sugar.\footnote{Tail
recursion has long been known as a compiler optimization trick.  A coherent
semantic basis for tail recursion was provided by Carl \link{Hewitt (1977)}, who
explained it in terms of the ``message-passing'' model of computation that we
shall discuss in \link{Chapter 3}.  Inspired by this, Gerald Jay Sussman and Guy
Lewis Steele Jr. (see \link{Steele and Sussman 1975}) constructed a tail-recursive interpreter for
Scheme.  Steele later showed how tail recursion is a consequence of the natural
way to compile procedure calls (\link{Steele 1977}).  The \acronym{IEEE} standard for
Scheme requires that Scheme implementations be tail-recursive.}

\begin{quote}
\heading{\phantomsection\label{Exercise 1.9}Exercise 1.9:} Each of the following two
procedures defines a method for adding two positive integers in terms of the
procedures \code{inc}, which increments its argument by 1, and \code{dec},
which decrements its argument by 1.

\begin{scheme}
(define (+ a b)
  (if (= a 0) b (inc (+ (dec a) b))))
(define (+ a b)
  (if (= a 0) b (+ (dec a) (inc b))))
\end{scheme}

Using the substitution model, illustrate the process generated by each
procedure in evaluating \code{(+ 4 5)}.  Are these processes iterative or
recursive?
\end{quote}

\begin{quote}
\heading{\phantomsection\label{Exercise 1.10}Exercise 1.10:} The following procedure computes
a mathematical function called Ackermann's function.

\begin{scheme}
(define (A x y)
  (cond ((= y 0) 0)
        ((= x 0) (* 2 y))
        ((= y 1) 2)
        (else (A (- x 1) (A x (- y 1))))))
\end{scheme}

What are the values of the following expressions?

\begin{scheme}
(A 1 10)
(A 2 4)
(A 3 3)
\end{scheme}

Consider the following procedures, where \code{A} is the procedure
defined above:

\begin{scheme}
(define (f n) (A 0 n))
(define (g n) (A 1 n))
(define (h n) (A 2 n))
(define (k n) (* 5 n n))
\end{scheme}

Give concise mathematical definitions for the functions computed by the
procedures \code{f}, \code{g}, and \code{h} for positive integer values of
\( n \).  For example, \code{(k n)} computes \( 5n^2 \).
\end{quote}

\subsection{Tree Recursion}
\label{Section 1.2.2}

Another common pattern of computation is called \newterm{tree recursion}.  As
an example, consider computing the sequence of Fibonacci numbers, in which each
number is the sum of the preceding two:
\begin{comment}
0, 1, 1, 2, 3, 5, 8, 13, 21, \( \dots \)
\end{comment}
\begin{displaymath}
 0,\; 1,\; 1,\; 2,\; 3,\; 5,\; 8,\; 13,\; 21,\; \dots. 
\end{displaymath}
In general, the Fibonacci numbers can be defined by the rule
\begin{comment}

\begin{example}
         /
         |  0                        if n = 0
Fib(n) = <  1                        if n = 1
         |  Fib(n - 1) + Fib(n - 2)  otherwise
         \
\end{example}

\end{comment}
\begin{displaymath}
 {\rm Fib}(n) = 
\begin{cases}	
        \; 0 & {\rm if} \;\; n=0, \\
	\; 1 & {\rm if} \;\; n=1, \\
	\; {\rm Fib}(n-1) + {\rm Fib}(n-2) \quad & {\rm otherwise}. 
\end{cases} 
\end{displaymath}
We can immediately translate this definition into a recursive procedure for
computing Fibonacci numbers:

\begin{scheme}
(define (fib n)
  (cond ((= n 0) 0)
        ((= n 1) 1)
        (else (+ (fib (- n 1))
                 (fib (- n 2))))))
\end{scheme}

\noindent
Consider the pattern of this computation.  To compute \code{(fib 5)}, we
compute \code{(fib 4)} and \code{(fib 3)}.  To compute \code{(fib 4)}, we
compute \code{(fib 3)} and \code{(fib 2)}.  In general, the evolved process
looks like a tree, as shown in \link{Figure 1.5}.  Notice that the branches
split into two at each level (except at the bottom); this reflects the fact
that the \code{fib} procedure calls itself twice each time it is invoked.

This procedure is instructive as a prototypical tree recursion, but it is a
terrible way to compute Fibonacci numbers because it does so much redundant
computation.  Notice in \link{Figure 1.5} that the entire computation of
\code{(fib 3)}---almost half the work---is duplicated.  In fact, it is not hard
to show that the number of times the procedure will compute \code{(fib 1)} or
\code{(fib 0)} (the number of leaves in the above tree, in general) is
precisely Fib(\( n+1 \)).  To get an idea of how bad this is, one can
show that the value of Fib(\( n \)) grows exponentially with \( n \).  More
precisely (see \link{Exercise 1.13}), Fib(\( n \)) is the closest integer
to \( \varphi^n / \sqrt{5} \), where
\begin{comment}

\begin{example}
[phi] = (1 + [sqrt]5)/2 ~= 1.6180
\end{example}

\end{comment}
\begin{displaymath}
\varphi = {1 + \sqrt{5}\over2} \approx 1.6180 
\end{displaymath}
\noindent
is the \newterm{golden ratio}, which satisfies the equation
\begin{comment}

\begin{example}
[phi]^2 = [phi] + 1
\end{example}

\end{comment}
\begin{displaymath}
\varphi^2 = \varphi + 1. 
\end{displaymath}

\begin{figure}[tb]
\phantomsection\label{Figure 1.5}
\centering
\begin{comment}
\heading{Figure 1.5:} The tree-recursive process generated in computing \code{(fib 5)}.

\begin{example}

                   ..<............ fib5   <.......... 
                ...     ___________/  \___________   .  
             ...       /       . .....            \    . 
           ..       fib4     .        . . . .     fib3  .  
         ..     ____/. \____  ..             .  __/  \__  .  
       ..      /  . .  ..   \    .        ..   /  . .   \   . 
     ..     fib3 .       .  fib2 .        . fib2 .   .  fib1 .
   ..      / . \  .     .   /  \  .      .  /  \ ...  .  |  .
 ..       / . . \   .  .   /  . \   .  .   / .  \   .  . 1 .
.      fib2 . . fib1.  .fib1 .  fib0 . .fib1. . fib0 .  .  .
.      /  \  . . |  .  . |  .  . |   . . |   . . |   .   .>
V     /  . \   . 1  .  . 1  .  . 0  .  . 1  .  . 0  ..
.  fib1 .. fib0..  .   .   .   .   .   V   .   ..  . 
.   |  .  . |  . .>     .>.     . .    ..>.      .>
.   1 .   . 0  .      
 .   .     .  .       
  .>.       ..        

\end{example}
\end{comment}
\includegraphics[width=90mm]{fig/chap1/Fig1.5c.pdf}
\begin{quote}
\heading{Figure 1.5:} The tree-recursive process generated in computing \code{(fib 5)}.
\end{quote}
\end{figure}

\noindent
Thus, the process uses a number of steps that grows exponentially with the
input.  On the other hand, the space required grows only linearly with the
input, because we need keep track only of which nodes are above us in the tree
at any point in the computation.  In general, the number of steps required by a
tree-recursive process will be proportional to the number of nodes in the tree,
while the space required will be proportional to the maximum depth of the tree.

We can also formulate an iterative process for computing the Fibonacci numbers.
The idea is to use a pair of integers \( a \) and \( b \), initialized to
Fib(1) = 1 and Fib(0) = 0, and to repeatedly apply the
simultaneous transformations
\begin{comment}

\begin{example}
a <- a + b
b <- a
\end{example}

\end{comment}
\begin{displaymath}
% \eqalign{a &\gets a + b, \cr 
% b &\gets a. \cr}
\begin{array}{l@{\quad\gets\quad}l}
  a & a + b, \\ 
  b & a.
\end{array}
\end{displaymath}
\noindent
It is not hard to show that, after applying this transformation \( n \) times,
\( a \) and \( b \) will be equal, respectively, to Fib(\( n+1 \)) and
Fib(\( n \)).  Thus, we can compute Fibonacci numbers iteratively using
the procedure

\begin{scheme}
(define (fib n)
  (fib-iter 1 0 n))
(define (fib-iter a b count)
  (if (= count 0)
      b
      (fib-iter (+ a b) a (- count 1))))
\end{scheme}

\noindent
This second method for computing Fib(\( n \)) is a linear iteration.  The
difference in number of steps required by the two methods---one linear in
\( n \), one growing as fast as Fib(\( n \)) itself---is enormous, even for
small inputs.

One should not conclude from this that tree-recursive processes are useless.
When we consider processes that operate on hierarchically structured data
rather than numbers, we will find that tree recursion is a natural and powerful
tool.\footnote{An example of this was hinted at in \link{Section 1.1.3}. The
interpreter itself evaluates expressions using a tree-recursive process.} But
even in numerical operations, tree-recursive processes can be useful in helping
us to understand and design programs.  For instance, although the first
\code{fib} procedure is much less efficient than the second one, it is more
straightforward, being little more than a translation into Lisp of the
definition of the Fibonacci sequence.  To formulate the iterative algorithm
required noticing that the computation could be recast as an iteration with
three state variables.

\subsubsection*{Example: Counting change}

It takes only a bit of cleverness to come up with the iterative Fibonacci
algorithm.  In contrast, consider the following problem: How many different
ways can we make change of \$1.00, given half-dollars, quarters, dimes,
nickels, and pennies?  More generally, can we write a procedure to compute the
number of ways to change any given amount of money?

This problem has a simple solution as a recursive procedure.  Suppose we think
of the types of coins available as arranged in some order.  Then the following
relation holds:

The number of ways to change amount \( a \) using \( n \) kinds of coins equals

\begin{itemize}

\item
the number of ways to change amount \( a \) using all but the first kind of coin,
plus

\item
the number of ways to change amount \( a - d \) using all \( n \) kinds of
coins, where \( d \) is the denomination of the first kind of coin.

\end{itemize}

\noindent
To see why this is true, observe that the ways to make change can be divided
into two groups: those that do not use any of the first kind of coin, and those
that do.  Therefore, the total number of ways to make change for some amount is
equal to the number of ways to make change for the amount without using any of
the first kind of coin, plus the number of ways to make change assuming that we
do use the first kind of coin.  But the latter number is equal to the number of
ways to make change for the amount that remains after using a coin of the first
kind.

Thus, we can recursively reduce the problem of changing a given amount to the
problem of changing smaller amounts using fewer kinds of coins.  Consider this
reduction rule carefully, and convince yourself that we can use it to describe
an algorithm if we specify the following degenerate cases:\footnote{For
example, work through in detail how the reduction rule applies to the problem
of making change for 10 cents using pennies and nickels.}

\begin{itemize}

\item
If \( a \) is exactly 0, we should count that as 1 way to make change.

\item
If \( a \) is less than 0, we should count that as 0 ways to make change.

\item
If \( n \) is 0, we should count that as 0 ways to make change.

\end{itemize}

\noindent
We can easily translate this description into a recursive procedure:

\begin{scheme}
(define (count-change amount) (cc amount 5))
(define (cc amount kinds-of-coins)
  (cond ((= amount 0) 1)
        ((or (< amount 0) (= kinds-of-coins 0)) 0)
        (else (+ (cc amount
                     (- kinds-of-coins 1))
                 (cc (- amount
                        (first-denomination 
                         kinds-of-coins))
                     kinds-of-coins)))))
(define (first-denomination kinds-of-coins)
  (cond ((= kinds-of-coins 1) 1)
        ((= kinds-of-coins 2) 5)
        ((= kinds-of-coins 3) 10)
        ((= kinds-of-coins 4) 25)
        ((= kinds-of-coins 5) 50)))
\end{scheme}

\noindent
(The \code{first\-/denomination} procedure takes as input the number of kinds of
coins available and returns the denomination of the first kind.  Here we are
thinking of the coins as arranged in order from largest to smallest, but any
order would do as well.)  We can now answer our original question about
changing a dollar:

\begin{scheme}
(count-change 100)
~\textit{292}~
\end{scheme}

\noindent
\code{Count\-/change} generates a tree-recursive process with redundancies
similar to those in our first implementation of \code{fib}.  (It will take
quite a while for that 292 to be computed.)  On the other hand, it is not
obvious how to design a better algorithm for computing the result, and we leave
this problem as a challenge.  The observation that a tree-recursive process may
be highly inefficient but often easy to specify and understand has led people
to propose that one could get the best of both worlds by designing a ``smart
compiler'' that could transform tree-recursive procedures into more efficient
procedures that compute the same result.\footnote{One approach to coping with
redundant computations is to arrange matters so that we automatically construct
a table of values as they are computed.  Each time we are asked to apply the
procedure to some argument, we first look to see if the value is already stored
in the table, in which case we avoid performing the redundant computation.
This strategy, known as \newterm{tabulation} or \newterm{memoization}, can be
implemented in a straightforward way.  Tabulation can sometimes be used to
transform processes that require an exponential number of steps (such as
\code{count\-/change}) into processes whose space and time requirements grow
linearly with the input.  See \link{Exercise 3.27}.}

\begin{quote}
\heading{\phantomsection\label{Exercise 1.11}Exercise 1.11:} A function \( f \) is defined by
the rule that 
\begin{displaymath}
f(n) = 
\begin{cases} 
\;\; n \quad \text{if \; \( n < 3 \),} \\ 
\;\; f(n-1) + 2\kern-0.08em f(n-2) + 3\kern-0.08em f(n-3) \quad \text{if \; \( n \ge 3 \).}
\end{cases}
\end{displaymath}
Write a procedure that computes \( f \) by means of a recursive process.  Write a procedure that
computes \( f \) by means of an iterative process.
\end{quote}

\begin{quote}
\heading{\phantomsection\label{Exercise 1.12}Exercise 1.12:} The following pattern of numbers
is called \newterm{Pascal's triangle}.

\begin{example}
        1
      1   1
    1   2   1
  1   3   3   1
1   4   6   4   1
      . . .
\end{example}

The numbers at the edge of the triangle are all 1, and each number inside the
triangle is the sum of the two numbers above it.\footnote{The elements of
Pascal's triangle are called the \newterm{binomial coefficients}, because the
\( n^{\mathrm{th}} \) row consists of the coefficients of the terms in the expansion of
\( (x + y)^n \).  This pattern for computing the coefficients appeared in
Blaise Pascal's 1653 seminal work on probability theory, \textit{Trait\'e du
triangle arithm\'etique}.  According to \link{Knuth (1973)}, the same pattern appears
in the \textit{Szu-yuen Y\"u-chien} (``The Precious Mirror of the Four
Elements''), published by the Chinese mathematician Chu Shih-chieh in 1303, in
the works of the twelfth-century Persian poet and mathematician Omar Khayyam,
and in the works of the twelfth-century Hindu mathematician Bh\'ascara
\'Ach\'arya.} Write a procedure that computes elements of Pascal's triangle by
means of a recursive process.
\end{quote}

\begin{quote}
\heading{\phantomsection\label{Exercise 1.13}Exercise 1.13:} Prove that Fib(\( n \)) is
the closest integer to \( \varphi^n / \sqrt{5} \), where \( \varphi = (1 +
\sqrt{5}) / 2 \).  Hint: Let \( \psi = (1 - \sqrt{5}) / 2 \).
Use induction and the definition of the Fibonacci numbers (see \link{Section 1.2.2}) 
to prove that Fib(\( n \)) = \( (\varphi^n - \psi^n) / \sqrt{5} \).
\end{quote}

\subsection{Orders of Growth}
\label{Section 1.2.3}

The previous examples illustrate that processes can differ considerably in the
rates at which they consume computational resources.  One convenient way to
describe this difference is to use the notion of \newterm{order of growth} to
obtain a gross measure of the resources required by a process as the inputs
become larger.

Let \( n \) be a parameter that measures the size of the problem, and let
\( R(n) \) be the amount of resources the process requires for a problem of
size \( n \).  In our previous examples we took \( n \) to be the number for which
a given function is to be computed, but there are other possibilities.  For
instance, if our goal is to compute an approximation to the square root of a
number, we might take \( n \) to be the number of digits accuracy required.  For
matrix multiplication we might take \( n \) to be the number of rows in the
matrices.  In general there are a number of properties of the problem with
respect to which it will be desirable to analyze a given process.  Similarly,
\( R(n) \) might measure the number of internal storage registers used, the
number of elementary machine operations performed, and so on.  In computers
that do only a fixed number of operations at a time, the time required will be
proportional to the number of elementary machine operations performed.

We say that \( R(n) \) has order of growth \( \Theta(f(n)) \), written
\( R(n) \) = \( \Theta(f(n)) \) (pronounced ``theta of
\( f(n) \)''), if there are positive constants \( k_1 \) and \( k_2 \)
independent of \( n \) such that \( k_1f(n) \le R(n) \le k_2f(n) \) 
for any sufficiently large value of \( n \).  (In other words, for large \( n \),
the value \( R(n) \) is sandwiched between \( k_1f(n) \) and
\( k_2f(n) \).)

\enlargethispage{\baselineskip}

For instance, with the linear recursive process for computing factorial
described in \link{Section 1.2.1} the number of steps grows proportionally to
the input \( n \).  Thus, the steps required for this process grows as
\( \Theta(n) \).  We also saw that the space required grows as
\( \Theta(n) \).  For the iterative factorial, the number of steps is still
\( \Theta(n) \) but the space is \( \Theta(1) \)---that is,
constant.\footnote{These statements mask a great deal of oversimplification.
For instance, if we count process steps as ``machine operations'' we are making
the assumption that the number of machine operations needed to perform, say, a
multiplication is independent of the size of the numbers to be multiplied,
which is false if the numbers are sufficiently large.  Similar remarks hold for
the estimates of space.  Like the design and description of a process, the
analysis of a process can be carried out at various levels of abstraction.} The
tree-recursive Fibonacci computation requires \( \Theta(\varphi^n) \)
steps and space \( \Theta(n) \), where \( \varphi \) is the golden ratio
described in \link{Section 1.2.2}.

Orders of growth provide only a crude description of the behavior of a process.
For example, a process requiring \( n^2 \) steps and a process requiring
\( 1000n^2 \) steps and a process requiring \( 3n^2 + 10n + 17 \) steps all
have \( \Theta(n^2) \) order of growth.  On the other hand, order of growth
provides a useful indication of how we may expect the behavior of the process
to change as we change the size of the problem.  For a \( \Theta(n) \)
(linear) process, doubling the size will roughly double the amount of resources
used.  For an exponential process, each increment in problem size will multiply
the resource utilization by a constant factor.  In the remainder of 
\link{Section 1.2} we will examine two algorithms whose order of growth is logarithmic,
so that doubling the problem size increases the resource requirement by a
constant amount.

\begin{quote}
\heading{\phantomsection\label{Exercise 1.14}Exercise 1.14:} Draw the tree illustrating the
process generated by the \code{count\-/change} procedure of \link{Section 1.2.2}
in making change for 11 cents.  What are the orders of growth of the space and
number of steps used by this process as the amount to be changed increases?
\end{quote}

\begin{quote}
\heading{\phantomsection\label{Exercise 1.15}Exercise 1.15:} The sine of an angle (specified
in radians) can be computed by making use of the approximation 
\mbox{\( \sin x \approx x \)} if \( x \) is sufficiently small, and the trigonometric
identity
\begin{comment}

\begin{example}
               x             x
sin x = 3 sin --- - 4 sin^3 ---
               3             3
\end{example}

\end{comment}
\begin{displaymath}
\sin x = 3\sin {x\over3} - 4\sin^3 {x\over3} 
\end{displaymath}
\noindent
to reduce the size of the argument of sin.  (For purposes of this
exercise an angle is considered ``sufficiently small'' if its magnitude is not
greater than 0.1 radians.) These ideas are incorporated in the following
procedures:

\begin{scheme}
(define (cube x) (* x x x))
(define (p x) (- (* 3 x) (* 4 (cube x))))
(define (sine angle)
   (if (not (> (abs angle) 0.1))
       angle
       (p (sine (/ angle 3.0)))))
\end{scheme}

\begin{enumerate}[a.]

\item
How many times is the procedure \code{p} applied when \code{(sine 12.15)} is
evaluated?

\item
What is the order of growth in space and number of steps (as a function of
\( a \)) used by the process generated by the \code{sine} procedure when
\code{(sine a)} is evaluated?

\end{enumerate}
\end{quote}

\subsection{Exponentiation}
\label{Section 1.2.4}

Consider the problem of computing the exponential of a given number.  We would
like a procedure that takes as arguments a base \( b \) and a positive integer
exponent \( n \) and computes \( b^n \).  One way to do this is via the
recursive definition
\begin{comment}

\begin{example}
b^n = b * b^(n - 1)
b^0 = 1
\end{example}

\end{comment}
\begin{displaymath}
% \eqalign{b^n &= b\cdot b^{n-1}, \cr 
% b^0 &= 1, \cr} 
\begin{array}{l@{{}={}}l}
  b^n & b\cdot b^{n-1}, \\ 
  b^0 & 1,
\end{array}
\end{displaymath}
which translates readily into the procedure

\begin{scheme}
(define (expt b n)
  (if (= n 0)
      1
      (* b (expt b (- n 1)))))
\end{scheme}

\noindent
This is a linear recursive process, which requires \( \Theta(n) \) steps and
\( \Theta(n) \) space.  Just as with factorial, we can readily formulate an
equivalent linear iteration:

\begin{scheme}
(define (expt b n)
  (expt-iter b n 1))
(define (expt-iter b counter product)
  (if (= counter 0)
      product
      (expt-iter b
                 (- counter 1)
                 (* b product))))
\end{scheme}

\noindent
This version requires \( \Theta(n) \) steps and \( \Theta(1) \) space.

We can compute exponentials in fewer steps by using successive squaring.  For
instance, rather than computing \( b^8 \) as
\begin{comment}

\begin{example}
b * (b * (b * (b * (b * (b * (b * b))))))
\end{example}

\end{comment}
\begin{displaymath}
 b\cdot (b\cdot (b\cdot (b\cdot (b\cdot (b\cdot (b\cdot b))))))\,, 
\end{displaymath}
we can compute it using three multiplications:
\begin{comment}

\begin{example}
b^2 = b * b
b^4 = b^2 * b^2
b^8 = b^4 * b^4
\end{example}

\end{comment}
\begin{displaymath}
% \eqalign{	b^2 &= b\cdot b, \cr
%		b^4 &= b^2\cdot b^2, \cr
%		b^8 &= b^4\cdot b^4. \cr} 
\begin{array}{l@{{}={}}l}
  b^2 & b\cdot b, \\
  b^4 & b^2\cdot b^2, \\
  b^8 & b^4\cdot b^4.
\end{array}
\end{displaymath}
This method works fine for exponents that are powers of 2.  We can also take
advantage of successive squaring in computing exponentials in general if we use
the rule
\begin{comment}

\begin{example}
b^n = (b^(n/2))^2    if n is even
b^n = b * b^(n - 1)  if n is odd
\end{example}

\end{comment}
\begin{displaymath}
% \eqalign{b^n &= (b^{n / 2})^2 \qquad \;\, {\rm if\;} n {\rm\; is\; even}, \cr
% b^n &= b\cdot b^{n-1} \qquad {\rm if\;} n {\rm\; is\; odd}. \cr} 
\begin{array}{l@{{}={}}lr@{\ n\ }l}
  b^n & (b^{n / 2})^2  \;\; & \mbox{if\,} & \mbox{\,is\, even}, \\
  b^n & b\cdot b^{n-1} \;\; & \mbox{if\,} & \mbox{\,is\, odd}.
\end{array}
\end{displaymath}
We can express this method as a procedure:

\begin{scheme}
(define (fast-expt b n)
  (cond ((= n 0) 1)
        ((even? n) (square (fast-expt b (/ n 2))))
        (else (* b (fast-expt b (- n 1))))))
\end{scheme}

\noindent
where the predicate to test whether an integer is even is defined in terms of
the primitive procedure \code{remainder} by

\begin{scheme}
(define (even? n)
  (= (remainder n 2) 0))
\end{scheme}

\noindent
The process evolved by \code{fast\-/expt} grows logarithmically with \( n \) in
both space and number of steps.  To see this, observe that computing
\( b^{2n} \) using \code{fast\-/expt} requires only one more multiplication
than computing \( b^n \).  The size of the exponent we can compute therefore
doubles (approximately) with every new multiplication we are allowed.  Thus,
the number of multiplications required for an exponent of \( n \) grows about as
fast as the logarithm of \( n \) to the base 2.  The process has
\( \Theta(\log n) \) growth.\footnote{More precisely, the number of
multiplications required is equal to 1 less than the log base 2 of \( n \) plus
the number of ones in the binary representation of \( n \).  This total is always
less than twice the log base 2 of \( n \).  The arbitrary constants \( k_1 \) and
\( k_2 \) in the definition of order notation imply that, for a logarithmic
process, the base to which logarithms are taken does not matter, so all such
processes are described as \( \Theta(\log n) \).}

The difference between \( \Theta(\log n) \) growth and
\( \Theta(n) \) growth becomes striking as \( n \) becomes large.  For
example, \code{fast\-/expt} for \( n \) = 1000 requires only 14
multiplications.\footnote{You may wonder why anyone would care about raising
numbers to the 1000th power.  See \link{Section 1.2.6}.} It is also possible to
use the idea of successive squaring to devise an iterative algorithm that
computes exponentials with a logarithmic number of steps (see \link{Exercise 1.16}), 
although, as is often the case with iterative algorithms, this is not
written down so straightforwardly as the recursive algorithm.\footnote{This
iterative algorithm is ancient.  It appears in the \textit{Chandah-sutra} by
\'Ach\'arya Pingala, written before 200 \acronym{B.C.} See \link{Knuth 1981}, section
4.6.3, for a full discussion and analysis of this and other methods of
exponentiation.}

\begin{quote}
\heading{\phantomsection\label{Exercise 1.16}Exercise 1.16:} Design a procedure that evolves
an iterative exponentiation process that uses successive squaring and uses a
logarithmic number of steps, as does \code{fast\-/expt}.  (Hint: Using the
observation that \( (b^{n / 2})^2 = (b^2)^{n / 2} \), keep, along with
the exponent \( n \) and the base \( b \), an additional state variable \( a \), and
define the state transformation in such a way that the product \( ab^n \) 
is unchanged from state to state.  At the beginning of the process
\( a \) is taken to be 1, and the answer is given by the value of \( a \) at the
end of the process.  In general, the technique of defining an
\newterm{invariant quantity} that remains unchanged from state to state is a
powerful way to think about the design of iterative algorithms.)
\end{quote}

\begin{quote}
\heading{\phantomsection\label{Exercise 1.17}Exercise 1.17:} The exponentiation algorithms in
this section are based on performing exponentiation by means of repeated
multiplication.  In a similar way, one can perform integer multiplication by
means of repeated addition.  The following multiplication procedure (in which
it is assumed that our language can only add, not multiply) is analogous to the
\code{expt} procedure:

\begin{scheme}
(define (* a b)
  (if (= b 0)
      0
      (+ a (* a (- b 1)))))
\end{scheme}

This algorithm takes a number of steps that is linear in \code{b}.  Now suppose
we include, together with addition, operations \code{double}, which doubles an
integer, and \code{halve}, which divides an (even) integer by 2.  Using these,
design a multiplication procedure analogous to \code{fast\-/expt} that uses a
logarithmic number of steps.
\end{quote}

\begin{quote}
\heading{\phantomsection\label{Exercise 1.18}Exercise 1.18:} Using the results of
\link{Exercise 1.16} and \link{Exercise 1.17}, devise a procedure that generates
an iterative process for multiplying two integers in terms of adding, doubling,
and halving and uses a logarithmic number of steps.\footnote{This algorithm,
which is sometimes known as the ``Russian peasant method'' of multiplication,
is ancient.  Examples of its use are found in the Rhind Papyrus, one of the two
oldest mathematical documents in existence, written about 1700 \acronym{B.C.}
(and copied from an even older document) by an Egyptian scribe named A\( \! \)'h-mose.}
\end{quote}

\enlargethispage{\baselineskip}

\begin{quote}
\heading{\phantomsection\label{Exercise 1.19}Exercise 1.19:} There is a clever algorithm for
computing the Fibonacci numbers in a logarithmic number of steps.  Recall the
transformation of the state variables \( a \) and \( b \) in the \code{fib\-/iter}
process of \link{Section 1.2.2}: \( a \gets a + b \) and \( b \gets a \).
Call this transformation \( T \), and observe that applying \( T \) over and over
again \( n \) times, starting with 1 and 0, produces the pair Fib(\( n+1 \)) and 
Fib(\( n \)).  In other words, the Fibonacci numbers are produced
by applying \( T^n \), the \( n^{\mathrm{th}} \) power of the transformation \( T \),
starting with the pair (1, 0).  Now consider \( T \) to be the special case of
\( p=0 \) and \( q=1 \) in a family of transformations \( T_{pq} \),
where \( T_{pq} \) transforms the pair \( (a, b) \) according to 
\( a \gets bq + aq + ap \) and \( b \gets bp + aq \).
Show that if we apply such a transformation \( T_{pq} \) twice, the
effect is the same as using a single transformation \( T_{p'\!q'} \) of the
same form, and compute \( p'\! \) and \( q'\! \) in terms of \( p \) and \( q \).  This
gives us an explicit way to square these transformations, and thus we can
compute \( T^n \) using successive squaring, as in the \code{fast\-/expt}
procedure.  Put this all together to complete the following procedure, which
runs in a logarithmic number of steps:\footnote{This exercise was suggested to
us by Joe Stoy, based on an example in \link{Kaldewaij 1990}.}

\begin{scheme}
(define (fib n)
  (fib-iter 1 0 0 1 n))
(define (fib-iter a b p q count)
  (cond ((= count 0) b)
        ((even? count)
         (fib-iter a
                   b
                   ~\( \dark \langle \)~??~\( \dark \rangle \)~   ~\textrm{; compute \( p' \)}~
                   ~\( \dark \langle \)~??~\( \dark \rangle \)~   ~\textrm{; compute \( q' \)}~
                   (/ count 2)))
        (else (fib-iter (+ (* b q) (* a q) (* a p))
                        (+ (* b p) (* a q))
                        p
                        q
                        (- count 1)))))
\end{scheme}
\end{quote}

\subsection{Greatest Common Divisors}
\label{Section 1.2.5}

The greatest common divisor (\acronym{GCD}) of two integers \( a \) and \( b \) is
defined to be the largest integer that divides both \( a \) and \( b \) with no
remainder.  For example, the \acronym{GCD} of 16 and 28 is 4.  In \link{Chapter
2}, when we investigate how to implement rational-number arithmetic, we will
need to be able to compute \acronym{GCD}s in order to reduce rational numbers
to lowest terms.  (To reduce a rational number to lowest terms, we must divide
both the numerator and the denominator by their \acronym{GCD}.  For example,
16/28 reduces to 4/7.)  One way to find the \acronym{GCD} of two integers is to
factor them and search for common factors, but there is a famous algorithm that
is much more efficient.

The idea of the algorithm is based on the observation that, if \( r \) is the
remainder when \( a \) is divided by \( b \), then the common divisors of \( a \) and
\( b \) are precisely the same as the common divisors of \( b \) and \( r \).  Thus,
we can use the equation

\begin{example}
GCD(a,b) = GCD(b,r)
\end{example}

\noindent
to successively reduce the problem of computing a \acronym{GCD} to the problem
of computing the \acronym{GCD} of smaller and smaller pairs of integers.  For
example,

\begin{example}
GCD(206,40) = GCD(40,6)
            = GCD(6,4)
            = GCD(4,2)
            = GCD(2,0)
            = 2
\end{example}

\noindent
reduces \acronym{GCD}(206, 40) to \acronym{GCD}(2, 0), which is 2.  It is
possible to show that starting with any two positive integers and performing
repeated reductions will always eventually produce a pair where the second
number is 0.  Then the \acronym{GCD} is the other number in the pair.  This
method for computing the \acronym{GCD} is known as \newterm{Euclid's
Algorithm}.\footnote{Euclid's Algorithm is so called because it appears in
Euclid's \textit{Elements} (Book 7, ca. 300 \acronym{B.C.}).  According to \link{Knuth (1973)}, 
it can be considered the oldest known nontrivial algorithm.  The
ancient Egyptian method of multiplication (\link{Exercise 1.18}) is surely
older, but, as Knuth explains, Euclid's algorithm is the oldest known to have
been presented as a general algorithm, rather than as a set of illustrative
examples.}

It is easy to express Euclid's Algorithm as a procedure:

\begin{scheme}
(define (gcd a b)
  (if (= b 0)
      a
      (gcd b (remainder a b))))
\end{scheme}

\noindent
This generates an iterative process, whose number of steps grows as the
logarithm of the numbers involved.

The fact that the number of steps required by Euclid's Algorithm has
logarithmic growth bears an interesting relation to the Fibonacci numbers:

\begin{quote}
\heading{Lam\'e's Theorem:} If Euclid's Algorithm requires \( k \) steps to
compute the \acronym{GCD} of some pair, then the smaller number in the pair
must be greater than or equal to the \( k^{\mathrm{th}} \) Fibonacci number.\footnote{This
theorem was proved in 1845 by Gabriel Lam\'e, a French mathematician and
engineer known chiefly for his contributions to mathematical physics.  To prove
the theorem, we consider pairs (\( a_k, b_k \)), where \( a_k \ge
b_k \), for which Euclid's Algorithm terminates in \( k \) steps.  The proof
is based on the claim that, if \( (a_{k+1}, b_{k+1}) \to
(a_k, b_k) \to (a_{k-1}, b_{k-1}) \) are three
successive pairs in the reduction process, then we must have \( b_{k+1} \ge
b_k + b_{k-1} \).  To verify the claim, consider that a reduction
step is defined by applying the transformation \( a_{k-1} = b_k,
b_{k-1} = \) remainder of \( a_k \) divided by \( b_k \).  The second
equation means that \( a_k = qb_k + b_{k-1} \) for some
positive integer \( q \).  And since \( q \) must be at least 1 we have \( a_k
= qb_k + b_{k-1} \ge b_k + b_{k-1} \).  But in
the previous reduction step we have \( b_{k+1} = a_k \).  Therefore,
\( b_{k+1} = a_k \ge b_k + b_{k-1} \).  This verifies
the claim.  Now we can prove the theorem by induction on \( k \), the number of
steps that the algorithm requires to terminate.  The result is true for \( k =
1 \), since this merely requires that \( b \) be at least as large as Fib(1)
= 1.  Now, assume that the result is true for all integers less than or equal
to \( k \) and establish the result for \( k + 1 \).  Let \( (a_{k+1},
b_{k+1}) \to (a_k, b_k) \to (a_{k-1}, b_{k-1}) \) be successive pairs in the 
reduction process.  By our
induction hypotheses, we have \( b_{k-1} \ge {\rm Fib}(k - 1) \) and
\( b_k \ge {\rm Fib}(k) \).  Thus, applying the claim we just proved
together with the definition of the Fibonacci numbers gives \( b_{k+1} \ge
b_k + b_{k-1} \ge {\rm Fib}(k) + {\rm Fib}(k-1) =
{\rm Fib}(k+1) \), which completes the proof of Lam\'e's Theorem.}
\end{quote}

\noindent
We can use this theorem to get an order-of-growth estimate for Euclid's
Algorithm.  Let \( n \) be the smaller of the two inputs to the procedure.  If
the process takes \( k \) steps, then we must have \( n \ge {\rm Fib}(k) 
\approx \varphi^k / \sqrt{5} \).  Therefore the number of steps \( k \)
grows as the logarithm (to the base \( \varphi \)) of \( n \).  Hence, the order of
growth is \( \Theta(\log n) \).

\begin{quote}
\heading{\phantomsection\label{Exercise 1.20}Exercise 1.20:} The process that a procedure
generates is of course dependent on the rules used by the interpreter.  As an
example, consider the iterative \code{gcd} procedure given above.  Suppose we
were to interpret this procedure using normal-order evaluation, as discussed in
\link{Section 1.1.5}.  (The normal-order-evaluation rule for \code{if} is
described in \link{Exercise 1.5}.)  Using the substitution method (for normal
order), illustrate the process generated in evaluating \code{(gcd 206 40)} and
indicate the \code{remainder} operations that are actually performed.  How many
\code{remainder} operations are actually performed in the normal-order
evaluation of \code{(gcd 206 40)}?  In the applicative-order evaluation?
\end{quote}

\subsection{Example: Testing for Primality}
\label{Section 1.2.6}

This section describes two methods for checking the primality of an integer
\( n \), one with order of growth \( \Theta(\sqrt{n}) \), and a
``probabilistic'' algorithm with order of growth \( \Theta(\log n) \).
The exercises at the end of this section suggest programming projects based on
these algorithms.

\subsubsection*{Searching for divisors}

Since ancient times, mathematicians have been fascinated by problems concerning
prime numbers, and many people have worked on the problem of determining ways
to test if numbers are prime.  One way to test if a number is prime is to find
the number's divisors.  The following program finds the smallest integral
divisor (greater than 1) of a given number \( n \).  It does this in a
straightforward way, by testing \( n \) for divisibility by successive integers
starting with 2.

\begin{scheme}
(define (smallest-divisor n) (find-divisor n 2))
(define (find-divisor n test-divisor)
  (cond ((> (square test-divisor) n) n)
        ((divides? test-divisor n) test-divisor)
        (else (find-divisor n (+ test-divisor 1)))))
(define (divides? a b) (= (remainder b a) 0))
\end{scheme}

\noindent
We can test whether a number is prime as follows: \( n \) is prime if and only if
\( n \) is its own smallest divisor.

\begin{scheme}
(define (prime? n)
  (= n (smallest-divisor n)))
\end{scheme}

\noindent
The end test for \code{find\-/divisor} is based on the fact that if \( n \) is not
prime it must have a divisor less than or equal to
\( \sqrt{n} \).\footnote{If \( d \) is a divisor of \( n \), then so is
\( n / d \).  But \( d \) and \( n / d \) cannot both be greater than
\( \sqrt{n} \).}  This means that the algorithm need only test divisors
between 1 and \( \sqrt{n} \).  Consequently, the number of steps required
to identify \( n \) as prime will have order of growth
\( \Theta(\sqrt{n}) \).

\subsubsection*{The Fermat test}

The \( \Theta(\log n) \) primality test is based on a result from
number theory known as Fermat's Little Theorem.\footnote{Pierre de Fermat
(1601-1665) is considered to be the founder of modern number theory.  He
obtained many important number-theoretic results, but he usually announced just
the results, without providing his proofs.  Fermat's Little Theorem was stated
in a letter he wrote in 1640.  The first published proof was given by Euler in
1736 (and an earlier, identical proof was discovered in the unpublished
manuscripts of Leibniz).  The most famous of Fermat's results---known as
Fermat's Last Theorem---was jotted down in 1637 in his copy of the book
\textit{Arithmetic} (by the third-century Greek mathematician Diophantus) with
the remark ``I have discovered a truly remarkable proof, but this margin is too
small to contain it.''  Finding a proof of Fermat's Last Theorem became one of
the most famous challenges in number theory.  A complete solution was finally
given in 1995 by Andrew Wiles of Princeton University.}

\begin{quote}
\heading{Fermat's Little Theorem:} If \( n \) is a prime number and \( a \) is any
positive integer less than \( n \), then \( a \) raised to the \( n^{\mathrm{th}} \) power is
congruent to \( a \) modulo \( n \).
\end{quote}

\noindent
(Two numbers are said to be \newterm{congruent modulo} \( n \) if they both have
the same remainder when divided by \( n \).  The remainder of a number \( a \) when
divided by \( n \) is also referred to as the \newterm{remainder of} \( a \)
\newterm{modulo} \( n \), or simply as \( a \) \newterm{modulo} \( n \).)

If \( n \) is not prime, then, in general, most of the numbers \( a < n \) will
not satisfy the above relation.  This leads to the following algorithm for
testing primality: Given a number \( n \), pick a random number \( a < n \) and
compute the remainder of \( a^n \) modulo \( n \).  If the result is not equal
to \( a \), then \( n \) is certainly not prime.  If it is \( a \), then chances are
good that \( n \) is prime.  Now pick another random number \( a \) and test it
with the same method.  If it also satisfies the equation, then we can be even
more confident that \( n \) is prime.  By trying more and more values of \( a \),
we can increase our confidence in the result.  This algorithm is known as the
Fermat test.

To implement the Fermat test, we need a procedure that computes the exponential
of a number modulo another number:

\begin{scheme}
(define (expmod base exp m)
  (cond ((= exp 0) 1)
        ((even? exp)
         (remainder
          (square (expmod base (/ exp 2) m))
          m))
        (else
         (remainder
          (* base (expmod base (- exp 1) m))
          m))))
\end{scheme}

\noindent
This is very similar to the \code{fast\-/expt} procedure of \link{Section 1.2.4}.
It uses successive squaring, so that the number of steps grows logarithmically
with the exponent.\footnote{The reduction steps in the cases where the exponent
\( e \) is greater than 1 are based on the fact that, for any integers \( x \),
\( y \), and \( m \), we can find the remainder of \( x \) times \( y \) modulo \( m \)
by computing separately the remainders of \( x \) modulo \( m \) and \( y \) modulo
\( m \), multiplying these, and then taking the remainder of the result modulo
\( m \).  For instance, in the case where \( e \) is even, we compute the remainder
of \( b^{e / 2} \) modulo \( m \), square this, and take the remainder modulo
\( m \).  This technique is useful because it means we can perform our
computation without ever having to deal with numbers much larger than \( m \).
(Compare \link{Exercise 1.25}.)}

The Fermat test is performed by choosing at random a number \( a \) between 1 and
\( n-1 \) inclusive and checking whether the remainder modulo \( n \) of the
\( n^{\mathrm{th}} \) power of \( a \) is equal to \( a \).  The random number \( a \) is chosen
using the procedure \code{random}, which we assume is included as a primitive
in Scheme. \code{Random} returns a nonnegative integer less than its integer
input.  Hence, to obtain a random number between 1 and \( n-1 \), we call
\code{random} with an input of \( n-1 \) and add 1 to the result:

\begin{scheme}
(define (fermat-test n)
  (define (try-it a)
    (= (expmod a n n) a))
  (try-it (+ 1 (random (- n 1)))))
\end{scheme}

\noindent
The following procedure runs the test a given number of times, as specified by
a parameter.  Its value is true if the test succeeds every time, and false
otherwise.

\begin{scheme}
(define (fast-prime? n times)
  (cond ((= times 0) true)
        ((fermat-test n) (fast-prime? n (- times 1)))
        (else false)))
\end{scheme}

\subsubsection*{Probabilistic methods}

The Fermat test differs in character from most familiar algorithms, in which
one computes an answer that is guaranteed to be correct.  Here, the answer
obtained is only probably correct.  More precisely, if \( n \) ever fails the
Fermat test, we can be certain that \( n \) is not prime.  But the fact that
\( n \) passes the test, while an extremely strong indication, is still not a
guarantee that \( n \) is prime.  What we would like to say is that for any
number \( n \), if we perform the test enough times and find that \( n \) always
passes the test, then the probability of error in our primality test can be
made as small as we like.

Unfortunately, this assertion is not quite correct.  There do exist numbers
that fool the Fermat test: numbers \( n \) that are not prime and yet have the
property that \( a^n \) is congruent to \( a \) modulo \( n \) for all integers
\( a < n \).  Such numbers are extremely rare, so the Fermat test is quite
reliable in practice.\footnote{\label{Footnote 1.47} Numbers
that fool the Fermat test are called \newterm{Carmichael numbers}, and little
is known about them other than that they are extremely rare.  There are 255
Carmichael numbers below 100,000,000.  The smallest few are 561, 1105, 1729,
2465, 2821, and 6601.  In testing primality of very large numbers chosen at
random, the chance of stumbling upon a value that fools the Fermat test is less
than the chance that cosmic radiation will cause the computer to make an error
in carrying out a ``correct'' algorithm.  Considering an algorithm to be
inadequate for the first reason but not for the second illustrates the
difference between mathematics and engineering.}

There are variations of the Fermat test that cannot be fooled.  In these tests,
as with the Fermat method, one tests the primality of an integer \( n \) by
choosing a random integer \( a < n \) and checking some condition that depends
upon \( n \) and \( a \).  (See \link{Exercise 1.28} for an example of such a test.)
On the other hand, in contrast to the Fermat test, one can prove that, for any
\( n \), the condition does not hold for most of the integers \( a < n \) unless
\( n \) is prime.  Thus, if \( n \) passes the test for some random choice of
\( a \), the chances are better than even that \( n \) is prime.  If \( n \) passes
the test for two random choices of \( a \), the chances are better than 3 out of
4 that \( n \) is prime. By running the test with more and more randomly chosen
values of \( a \) we can make the probability of error as small as we like.

The existence of tests for which one can prove that the chance of error becomes
arbitrarily small has sparked interest in algorithms of this type, which have
come to be known as \newterm{probabilistic algorithms}.  There is a great deal
of research activity in this area, and probabilistic algorithms have been
fruitfully applied to many fields.\footnote{One of the most striking
applications of probabilistic prime testing has been to the field of
cryptography.  Although it is now computationally infeasible to factor an
arbitrary 200-digit number, the primality of such a number can be checked in a
few seconds with the Fermat test.  This fact forms the basis of a technique for
constructing ``unbreakable codes'' suggested by \link{Rivest et al. (1977)}.  
The resulting \newterm{RSA algorithm} has become a widely used
technique for enhancing the security of electronic communications.  Because of
this and related developments, the study of prime numbers, once considered the
epitome of a topic in ``pure'' mathematics to be studied only for its own sake,
now turns out to have important practical applications to cryptography,
electronic funds transfer, and information retrieval.}

\begin{quote}
\heading{\phantomsection\label{Exercise 1.21}Exercise 1.21:} Use the \code{smallest\-/divisor}
procedure to find the smallest divisor of each of the following numbers: 199,
1999, 19999.
\end{quote}

\begin{quote}
\heading{\phantomsection\label{Exercise 1.22}Exercise 1.22:} Most Lisp implementations include
a primitive called \code{runtime} that returns an integer that specifies the
amount of time the system has been running (measured, for example, in
microseconds).  The following \code{timed\-/prime\-/test} procedure, when called
with an integer \( n \), prints \( n \) and checks to see if \( n \) is prime.  If
\( n \) is prime, the procedure prints three asterisks followed by the amount of
time used in performing the test.

\begin{scheme}
(define (timed-prime-test n)
  (newline)
  (display n) 
  (start-prime-test n (runtime)))
(define (start-prime-test n start-time)
  (if (prime? n) 
      (report-prime (- (runtime) start-time))))
(define (report-prime elapsed-time)
  (display " *** ")
  (display elapsed-time))
\end{scheme}

Using this procedure, write a procedure \code{search\-/for\-/primes} that checks
the primality of consecutive odd integers in a specified range.  Use your
procedure to find the three smallest primes larger than 1000; larger than
10,000; larger than 100,000; larger than 1,000,000.  Note the time needed to
test each prime.  Since the testing algorithm has order of growth of
\( \Theta(\sqrt{n}) \), you should expect that testing for primes
around 10,000 should take about \( \sqrt{10} \) times as long as testing for
primes around 1000.  Do your timing data bear this out?  How well do the data
for 100,000 and 1,000,000 support the \( \Theta(\sqrt{n}) \) prediction?  Is your
result compatible with the notion that programs on your machine run in time
proportional to the number of steps required for the computation?
\end{quote}

\begin{quote}
\heading{\phantomsection\label{Exercise 1.23}Exercise 1.23:} The \code{smallest\-/divisor}
procedure shown at the start of this section does lots of needless testing:
After it checks to see if the number is divisible by 2 there is no point in
checking to see if it is divisible by any larger even numbers.  This suggests
that the values used for \code{test\-/divisor} should not be 2, 3, 4, 5, 6,
\( \dots \), but rather 2, 3, 5, 7, 9, \( \dots \).  To implement this change, define a
procedure \code{next} that returns 3 if its input is equal to 2 and otherwise
returns its input plus 2.  Modify the \code{smallest\-/divisor} procedure to use
\code{(next test\-/divisor)} instead of \code{(+ test\-/divisor 1)}.  With
\code{timed\-/prime\-/test} incorporating this modified version of
\code{smallest\-/divisor}, run the test for each of the 12 primes found in
\link{Exercise 1.22}.  Since this modification halves the number of test steps,
you should expect it to run about twice as fast.  Is this expectation
confirmed?  If not, what is the observed ratio of the speeds of the two
algorithms, and how do you explain the fact that it is different from 2?
\end{quote}

\begin{quote}
\heading{\phantomsection\label{Exercise 1.24}Exercise 1.24:} Modify the
\code{timed\-/prime\-/test} procedure of \link{Exercise 1.22} to use
\code{fast\-/prime?} (the Fermat method), and test each of the 12 primes you
found in that exercise.  Since the Fermat test has \( \Theta(\log n) \)
growth, how would you expect the time to test primes near 1,000,000 to
compare with the time needed to test primes near 1000?  Do your data bear this
out?  Can you explain any discrepancy you find?
\end{quote}

\begin{quote}
\heading{\phantomsection\label{Exercise 1.25}Exercise 1.25:} Alyssa P. Hacker complains that
we went to a lot of extra work in writing \code{expmod}.  After all, she says,
since we already know how to compute exponentials, we could have simply written

\begin{scheme}
(define (expmod base exp m)
  (remainder (fast-expt base exp) m))
\end{scheme}

Is she correct?  Would this procedure serve as well for our fast prime tester?
Explain.
\end{quote}

\begin{quote}
\heading{\phantomsection\label{Exercise 1.26}Exercise 1.26:} Louis Reasoner is having great
difficulty doing \link{Exercise 1.24}.  His \code{fast\-/prime?} test seems to run
more slowly than his \code{prime?} test.  Louis calls his friend Eva Lu Ator
over to help.  When they examine Louis's code, they find that he has rewritten
the \code{expmod} procedure to use an explicit multiplication, rather than
calling \code{square}:

\begin{scheme}
(define (expmod base exp m)
  (cond ((= exp 0) 1)
        ((even? exp)
         (remainder (* (expmod base (/ exp 2) m)
                       (expmod base (/ exp 2) m))
                    m))
        (else
         (remainder (* base 
                       (expmod base (- exp 1) m))
                    m))))
\end{scheme}

``I don't see what difference that could make,'' says Louis.  ``I do.''  says
Eva.  ``By writing the procedure like that, you have transformed the
\( \Theta(\log n) \) process into a \( \Theta(n) \) process.''
Explain.
\end{quote}

\begin{quote}
\heading{\phantomsection\label{Exercise 1.27}Exercise 1.27:} Demonstrate that the Carmichael
numbers listed in \link{Footnote 1.47} really do fool the Fermat test.  That is,
write a procedure that takes an integer \( n \) and tests whether \( a^n \) is
congruent to \( a \) modulo \( n \) for every \( a < n \), and try your procedure
on the given Carmichael numbers.
\end{quote}

\begin{quote}
\heading{\phantomsection\label{Exercise 1.28}Exercise 1.28:} One variant of the Fermat test
that cannot be fooled is called the \newterm{Miller-Rabin test} (\link{Miller 1976};
\link{Rabin 1980}).  This starts from an alternate form of Fermat's Little Theorem,
which states that if \( n \) is a prime number and \( a \) is any positive integer
less than \( n \), then \( a \) raised to the (\( n-1 \))-st power is congruent to 1
modulo \( n \).  To test the primality of a number \( n \) by the Miller-Rabin
test, we pick a random number \( a < n \) and raise \( a \) to the (\( n-1 \))-st
power modulo \( n \) using the \code{expmod} procedure.  However, whenever we
perform the squaring step in \code{expmod}, we check to see if we have
discovered a ``nontrivial square root of 1 modulo \( n \),'' that is, a number
not equal to 1 or \( n-1 \) whose square is equal to 1 modulo \( n \).  It is
possible to prove that if such a nontrivial square root of 1 exists, then \( n \)
is not prime.  It is also possible to prove that if \( n \) is an odd number that
is not prime, then, for at least half the numbers \( a < n \), computing
\( a^{n-1} \) in this way will reveal a nontrivial square root of 1 modulo
\( n \).  (This is why the Miller-Rabin test cannot be fooled.)  Modify the
\code{expmod} procedure to signal if it discovers a nontrivial square root of
1, and use this to implement the Miller-Rabin test with a procedure analogous
to \code{fermat\-/test}.  Check your procedure by testing various known primes
and non-primes.  Hint: One convenient way to make \code{expmod} signal is to
have it return 0.
\end{quote}

\vspace{0.5em}
\section{Formulating Abstractions\\ with Higher-Order Procedures}
\label{Section 1.3}

We have seen that procedures are, in effect, abstractions that describe
compound operations on numbers independent of the particular numbers.  For
example, when we

\begin{scheme}
(define (cube x) (* x x x))
\end{scheme}

\noindent
we are not talking about the cube of a particular number, but rather about a
method for obtaining the cube of any number.  Of course we could get along
without ever defining this procedure, by always writing expressions such as

\begin{scheme}
(* 3 3 3)
(* x x x)
(* y y y)
\end{scheme}

\noindent
and never mentioning \code{cube} explicitly.  This would place us at a serious
disadvantage, forcing us to work always at the level of the particular
operations that happen to be primitives in the language (multiplication, in
this case) rather than in terms of higher-level operations.  Our programs would
be able to compute cubes, but our language would lack the ability to express
the concept of cubing.  One of the things we should demand from a powerful
programming language is the ability to build abstractions by assigning names to
common patterns and then to work in terms of the abstractions directly.
Procedures provide this ability.  This is why all but the most primitive
programming languages include mechanisms for defining procedures.

Yet even in numerical processing we will be severely limited in our ability to
create abstractions if we are restricted to procedures whose parameters must be
numbers.  Often the same programming pattern will be used with a number of
different procedures.  To express such patterns as concepts, we will need to
construct procedures that can accept procedures as arguments or return
procedures as values.  Procedures that manipulate procedures are called
\newterm{higher-order procedures}.  This section shows how higher-order
procedures can serve as powerful abstraction mechanisms, vastly increasing the
expressive power of our language.



\subsection{Procedures as Arguments}
\label{Section 1.3.1}

Consider the following three procedures.  The first computes the sum of the
integers from \code{a} through \code{b}:

\begin{scheme}
(define (sum-integers a b)
  (if (> a b) 
      0
      (+ a (sum-integers (+ a 1) b))))
\end{scheme}

\noindent
The second computes the sum of the cubes of the integers in the given range:

\begin{scheme}
(define (sum-cubes a b)
  (if (> a b)
      0
      (+ (cube a)
         (sum-cubes (+ a 1) b))))
\end{scheme}

\noindent
The third computes the sum of a sequence of terms in the series
\begin{comment}

\begin{example}
  1       1       1
----- + ----- + ------ + ...
1 * 3   5 * 7   9 * 11
\end{example}

\end{comment}
\begin{displaymath}
 {1\over1\cdot 3} +  {1\over5\cdot 7} + {1\over9\cdot 11} + \dots, 
\end{displaymath}
\noindent
which converges to \( \pi / 8 \) (very slowly):\footnote{This series, usually
written in the equivalent form 
\( {\pi\over4} = 1 - {1\over3} + {1\over5} - {1\over7} + \dots \), 
is due to Leibniz.  We'll see how to use this as the basis for some
fancy numerical tricks in \link{Section 3.5.3}.}

\begin{scheme}
(define (pi-sum a b)
  (if (> a b)
      0
      (+ (/ 1.0 (* a (+ a 2)))
         (pi-sum (+ a 4) b))))
\end{scheme}

\noindent
These three procedures clearly share a common underlying pattern.  They are for
the most part identical, differing only in the name of the procedure, the
function of \code{a} used to compute the term to be added, and the function
that provides the next value of \code{a}.  We could generate each of the
procedures by filling in slots in the same template:

\begin{scheme}
(define (~\( \dark \langle \)~~\var{\dark name}~~\( \dark \rangle \)~ a b)
  (if (> a b)
      0
      (+ (~\( \dark \langle \)~~\var{\dark term}~~\( \dark \rangle \)~ a)
         (~\( \dark \langle \)~~\var{\dark name}~~\( \dark \rangle \)~ (~\( \dark \langle \)~~\var{\dark next}~~\( \dark \rangle \)~ a) b))))
\end{scheme}

\noindent
The presence of such a common pattern is strong evidence that there is a useful
abstraction waiting to be brought to the surface.  Indeed, mathematicians long
ago identified the abstraction of \newterm{summation of a series} and invented
``sigma notation,'' for example
\begin{comment}

\begin{example}
  b
 ---
 >    f(n) = f(a) + ... + f(b)
 ---
 n=a
\end{example}

\end{comment}
\begin{displaymath}
 \sum\limits_{n=a}^b f(n) = f(a) + \dots + f(b), 
\end{displaymath}
\noindent
to express this concept.  The power of sigma notation is that it allows
mathematicians to deal with the concept of summation itself rather than only
with particular sums---for example, to formulate general results about sums
that are independent of the particular series being summed.

Similarly, as program designers, we would like our language to be powerful
enough so that we can write a procedure that expresses the concept of summation
itself rather than only procedures that compute particular sums.  We can do so
readily in our procedural language by taking the common template shown above
and transforming the ``slots'' into formal parameters:

\begin{scheme}
(define (sum term a next b)
  (if (> a b)
      0
      (+ (term a)
         (sum term (next a) next b))))
\end{scheme}

\noindent
Notice that \code{sum} takes as its arguments the lower and upper bounds
\code{a} and \code{b} together with the procedures \code{term} and \code{next}.
We can use \code{sum} just as we would any procedure.  For example, we can use
it (along with a procedure \code{inc} that increments its argument by 1) to
define \code{sum\-/cubes}:

\begin{scheme}
(define (inc n) (+ n 1))
(define (sum-cubes a b)
  (sum cube a inc b))
\end{scheme}

\noindent
Using this, we can compute the sum of the cubes of the integers from 1 to 10:

\begin{scheme}
(sum-cubes 1 10)
~\textit{3025}~
\end{scheme}

\noindent
With the aid of an identity procedure to compute the term, we can define
\code{sum\-/integers} in terms of \code{sum}:

\begin{scheme}
(define (identity x) x)
(define (sum-integers a b)
  (sum identity a inc b))
\end{scheme}

\noindent
Then we can add up the integers from 1 to 10:

\begin{scheme}
(sum-integers 1 10)
~\textit{55}~
\end{scheme}

\noindent
We can also define \code{pi\-/sum} in the same way:\footnote{Notice that we have
used block structure (\link{Section 1.1.8}) to embed the definitions of
\code{pi\-/next} and \code{pi\-/term} within \code{pi\-/sum}, since these procedures
are unlikely to be useful for any other purpose.  We will see how to get rid of
them altogether in \link{Section 1.3.2}.}

\begin{scheme}
(define (pi-sum a b)
  (define (pi-term x)
    (/ 1.0 (* x (+ x 2))))
  (define (pi-next x)
    (+ x 4))
  (sum pi-term a pi-next b))
\end{scheme}

\noindent
Using these procedures, we can compute an approximation to \( \pi \):

\begin{scheme}
(* 8 (pi-sum 1 1000))
~\textit{3.139592655589783}~
\end{scheme}

\noindent
Once we have \code{sum}, we can use it as a building block in formulating
further concepts.  For instance, the definite integral of a function \( f \)
between the limits \( a \) and \( b \) can be approximated numerically using the
formula
\begin{comment}

\begin{example}
/b     /  /     dx \    /          dx \    /           dx \      \
|  f = | f| a + -- | + f| a + dx + -- | + f| a + 2dx + -- | + ...| dx
/a     \  \     2  /    \          2  /    \           2  /      /
\end{example}

\end{comment}
\begin{displaymath}
{\int_a^b \!\!\! f} = {\left[\;f\! \left(a + {dx \over 2}\right) 
		+ f\! \left(a + dx + {dx \over 2}\right) 
		+ f\! \left(a + 2dx + {dx \over 2}\right) + \,\dots \;\right]\! dx} 
\end{displaymath}
\noindent
for small values of \( dx \).  We can express this directly as a procedure:

\begin{scheme}
(define (integral f a b dx)
  (define (add-dx x) 
    (+ x dx))
  (* (sum f (+ a (/ dx 2.0)) add-dx b) 
     dx))

(integral cube 0 1 0.01)
~\textit{.24998750000000042}~

(integral cube 0 1 0.001)
~\textit{.249999875000001}~
\end{scheme}

\noindent
(The exact value of the integral of \code{cube} between 0 and 1 is 1/4.)

\begin{quote}
\heading{\phantomsection\label{Exercise 1.29}Exercise 1.29:} Simpson's Rule is a more accurate
method of numerical integration than the method illustrated above.  Using
Simpson's Rule, the integral of a function \( f \) between \( a \) and \( b \) is
approximated as
\begin{comment}

\begin{example}
h
- (y_0 + 4y_1 + 2y_2 + 4y_3 + 2y_4 + ... + 2y_(n-2) + 4y_(n-1) + y_n)
3
\end{example}

\end{comment}
\begin{displaymath}
 {h\over 3}(y_0 + 4y_1 + 2y_2 + 4y_3 + 2y_4 + \dots + 2y_{n-2} + 4y_{n-1} + y_n), 
\end{displaymath}
\noindent
where \( h = (b - a) / n \), for some even integer \( n \), and
\( y_k = f(a + kh) \).  (Increasing \( n \) increases the
accuracy of the approximation.)  Define a procedure that takes as arguments
\( f \), \( a \), \( b \), and \( n \) and returns the value of the integral, computed
using Simpson's Rule.  Use your procedure to integrate \code{cube} between 0
and 1 (with \( n = 100 \) and \( n = 1000 \)), and compare the results to those of
the \code{integral} procedure shown above.
\end{quote}

\begin{quote}
\heading{\phantomsection\label{Exercise 1.30}Exercise 1.30:} The \code{sum} procedure above
generates a linear recursion.  The procedure can be rewritten so that the sum
is performed iteratively.  Show how to do this by filling in the missing
expressions in the following definition:

\begin{scheme}
(define (sum term a next b)
  (define (iter a result)
    (if ~\( \dark \langle \)~??~\( \dark \rangle \)~
        ~\( \dark \langle \)~??~\( \dark \rangle \)~
        (iter ~\( \dark \langle \)~??~\( \dark \rangle \)~ ~\( \dark \langle \)~??~\( \dark \rangle \)~)))
  (iter ~\( \dark \langle \)~??~\( \dark \rangle \)~ ~\( \dark \langle \)~??~\( \dark \rangle \)~))
\end{scheme}
\end{quote}

\begin{quote}
\heading{\phantomsection\label{Exercise 1.31}Exercise 1.31:} \begin{enumerate}[a.]

\item
The \code{sum} procedure is only the simplest of a vast number of similar
abstractions that can be captured as higher-order procedures.\footnote{The
intent of \link{Exercise 1.31} through \link{Exercise 1.33} is to demonstrate the
expressive power that is attained by using an appropriate abstraction to
consolidate many seemingly disparate operations.  However, though accumulation
and filtering are elegant ideas, our hands are somewhat tied in using them at
this point since we do not yet have data structures to provide suitable means
of combination for these abstractions.  We will return to these ideas in
\link{Section 2.2.3} when we show how to use \newterm{sequences} as interfaces
for combining filters and accumulators to build even more powerful
abstractions.  We will see there how these methods really come into their own
as a powerful and elegant approach to designing programs.}  Write an analogous
procedure called \code{product} that returns the product of the values of a
function at points over a given range.  Show how to define \code{factorial} in
terms of \code{product}.  Also use \code{product} to compute approximations to
\( \pi \) using the formula\footnote{This formula was discovered by the
seventeenth-century English mathematician John Wallis.}
\begin{comment}

\begin{example}
pi   2 * 4 * 4 * 6 * 6 * 8 ...
-- = -------------------------
 4   3 * 3 * 5 * 5 * 7 * 7 ...
\end{example}

\end{comment}
\begin{displaymath}
 {\pi\over 4} = {2\cdot 4\cdot 4\cdot 6\cdot 6\cdot 8\cdots\over 
		   3\cdot 3\cdot 5\cdot 5\cdot 7\cdot 7\cdots}\,. 
\end{displaymath}
\item
If your \code{product} procedure generates a recursive process, write one that
generates an iterative process.  If it generates an iterative process, write
one that generates a recursive process.

\end{enumerate}
\end{quote}

\begin{quote}
\heading{\phantomsection\label{Exercise 1.32}Exercise 1.32:} \begin{enumerate}[a.]

\item
Show that \code{sum} and \code{product} (\link{Exercise 1.31}) are both special
cases of a still more general notion called \code{accumulate} that combines a
collection of terms, using some general accumulation function:

\begin{scheme}
(accumulate combiner null-value term a next b)
\end{scheme}

\code{Accumulate} takes as arguments the same term and range specifications as
\code{sum} and \code{product}, together with a \code{combiner} procedure (of
two arguments) that specifies how the current term is to be combined with the
accumulation of the preceding terms and a \code{null\-/value} that specifies what
base value to use when the terms run out.  Write \code{accumulate} and show how
\code{sum} and \code{product} can both be defined as simple calls to
\code{accumulate}.

\item
If your \code{accumulate} procedure generates a recursive process, write one
that generates an iterative process.  If it generates an iterative process,
write one that generates a recursive process.

\end{enumerate}
\end{quote}

\begin{quote}
\heading{\phantomsection\label{Exercise 1.33}Exercise 1.33:} You can obtain an even more
general version of \code{accumulate} (\link{Exercise 1.32}) by introducing the
notion of a \newterm{filter} on the terms to be combined.  That is, combine
only those terms derived from values in the range that satisfy a specified
condition.  The resulting \code{filtered\-/accumulate} abstraction takes the same
arguments as accumulate, together with an additional predicate of one argument
that specifies the filter.  Write \code{filtered\-/accumulate} as a procedure.
Show how to express the following using \code{filtered\-/accumulate}:

\begin{enumerate}[a.]

\item
the sum of the squares of the prime numbers in the interval \( a \) to \( b \)
(assuming that you have a \code{prime?} predicate already written)

\item
the product of all the positive integers less than \( n \) that are relatively
prime to \( n \) (i.e., all positive integers \( i < n \) such that
\( {\rm GCD}(i, n) = 1 \)).

\end{enumerate}
\end{quote}

\subsection{Constructing Procedures Using \code{Lambda}}
\label{Section 1.3.2}

In using \code{sum} as in \link{Section 1.3.1}, it seems terribly awkward to
have to define trivial procedures such as \code{pi\-/term} and \code{pi\-/next}
just so we can use them as arguments to our higher-order procedure.  Rather
than define \code{pi\-/next} and \code{pi\-/term}, it would be more convenient to
have a way to directly specify ``the procedure that returns its input
incremented by 4'' and ``the procedure that returns the reciprocal of its input
times its input plus 2.''  We can do this by introducing the special form
\code{lambda}, which creates procedures.  Using \code{lambda} we can describe
what we want as

\begin{scheme}
(lambda (x) (+ x 4))
\end{scheme}

\noindent
and

\begin{scheme}
(lambda (x) (/ 1.0 (* x (+ x 2))))
\end{scheme}

\noindent
Then our \code{pi\-/sum} procedure can be expressed without defining any
auxiliary procedures as

\begin{scheme}
(define (pi-sum a b)
  (sum (lambda (x) (/ 1.0 (* x (+ x 2))))
       a
       (lambda (x) (+ x 4))
       b))
\end{scheme}

\noindent
Again using \code{lambda}, we can write the \code{integral} procedure without
having to define the auxiliary procedure \code{add\-/dx}:

\begin{scheme}
(define (integral f a b dx)
  (* (sum f
          (+ a (/ dx 2.0))
          (lambda (x) (+ x dx))
          b)
     dx))
\end{scheme}

\noindent
In general, \code{lambda} is used to create procedures in the same way as
\code{define}, except that no name is specified for the procedure:

\begin{scheme}
(lambda (~\( \dark \langle \)~~\var{\dark formal-parameters}~~\( \dark \rangle \)~) ~\( \dark \langle \)~~\var{\dark body}~~\( \dark \rangle \)~)
\end{scheme}

\noindent
The resulting procedure is just as much a procedure as one that is created
using \code{define}.  The only difference is that it has not been associated
with any name in the environment.  In fact,

\begin{scheme}
(define (plus4 x) (+ x 4))
\end{scheme}

\noindent
is equivalent to

\begin{scheme}
(define plus4 (lambda (x) (+ x 4)))
\end{scheme}

\noindent
We can read a \code{lambda} expression as follows:

\begin{scheme}
(lambda                     (x)     (+   x     4))
    |                        |       |   |     |
the procedure of an argument x that adds x and 4
\end{scheme}

\noindent
Like any expression that has a procedure as its value, a \code{lambda}
expression can be used as the operator in a combination such as

\begin{scheme}
((lambda (x y z) (+ x y (square z)))
 1 2 3)
~\textit{12}~
\end{scheme}

\noindent
or, more generally, in any context where we would normally use a procedure
name.\footnote{It would be clearer and less intimidating to people learning
Lisp if a name more obvious than \code{lambda}, such as \code{make\-/procedure},
were used.  But the convention is firmly entrenched.  The notation is adopted
from the λ-calculus, a mathematical formalism introduced by the
mathematical logician Alonzo \link{Church (1941)}.  Church developed the 
λ-calculus to provide a rigorous foundation for studying the 
notions of function
and function application.  The λ-calculus has become a basic tool
for mathematical investigations of the semantics of programming languages.}

\subsubsection*{Using \code{let} to create local variables}

Another use of \code{lambda} is in creating local variables.  We often need
local variables in our procedures other than those that have been bound as
formal parameters.  For example, suppose we wish to compute the function
\begin{comment}

\begin{example}
f(x,y) = x(1 + xy)^2 + y(1 - y) + (1 + xy)(1 - y)
\end{example}

\end{comment}
\begin{displaymath}
 f(x,y) = x(1 + xy)^2 + y(1 - y) + (1 + xy)(1 - y), 
\end{displaymath}
\noindent
which we could also express as
\begin{comment}

\begin{example}
     a = 1 + xy
     b = 1 - y
f(x,y) = xa^2 + yb + ab
\end{example}

\end{comment}
\begin{displaymath}
% \eqalign{	a 	&= 1 + xy, \cr
% 		b 	&= 1 - y,  \cr
% 		f(x,y) 	&= xa^2 + yb + ab. \cr}
\begin{array}{r@{{}={}}l}
  a 	  &  1 + xy, \\
  b 	  &  1 - y,  \\
  f(x,y)  &  xa^2 + yb + ab.
\end{array}
\end{displaymath}
In writing a procedure to compute \( f \), we would like to include as local
variables not only \( x \) and \( y \) but also the names of intermediate
quantities like \( a \) and \( b \).  One way to accomplish this is to use an
auxiliary procedure to bind the local variables:

\begin{scheme}
(define (f x y)
  (define (f-helper a b)
    (+ (* x (square a))
       (* y b)
       (* a b)))
  (f-helper (+ 1 (* x y))
            (- 1 y)))
\end{scheme}

\noindent
Of course, we could use a \code{lambda} expression to specify an anonymous
procedure for binding our local variables.  The body of \code{f} then becomes a
single call to that procedure:

\begin{scheme}
(define (f x y)
  ((lambda (a b)
     (+ (* x (square a))
        (* y b)
        (* a b)))
   (+ 1 (* x y))
   (- 1 y)))
\end{scheme}

\noindent
This construct is so useful that there is a special form called \code{let} to
make its use more convenient.  Using \code{let}, the \code{f} procedure could
be written as

\begin{scheme}
(define (f x y)
  (let ((a (+ 1 (* x y)))
        (b (- 1 y)))
    (+ (* x (square a))
       (* y b)
       (* a b))))
\end{scheme}

\noindent
The general form of a \code{let} expression is

\begin{scheme}
(let ((~\( \dark \langle \)~~\( \dark var_1 \)~~\( \dark \rangle \)~ ~\( \dark \langle \)~~\( \dark exp_1 \)~~\( \dark \rangle \)~)
      (~\( \dark \langle \)~~\( \dark var_2 \)~~\( \dark \rangle \)~ ~\( \dark \langle \)~~\( \dark exp_2 \)~~\( \dark \rangle \)~)
      ~\( \dots \)~
      (~\( \dark \langle \)~~\( \dark var_n \)~~\( \dark \rangle \)~ ~\( \dark \langle \)~~\( \dark exp_n \)~~\( \dark \rangle \)~))
   ~\( \dark \langle \)~~\var{\dark body}~~\( \dark \rangle \)~)
\end{scheme}

\noindent
which can be thought of as saying

\begin{scheme}
let ~\( \dark \langle \)~~\( \dark var_1 \)~~\( \dark \rangle \)~ have the value ~\( \dark \langle \)~~\( \dark exp_1 \)~~\( \dark \rangle \)~ and
    ~\( \dark \langle \)~~\( \dark var_2 \)~~\( \dark \rangle \)~ have the value ~\( \dark \langle \)~~\( \dark exp_2 \)~~\( \dark \rangle \)~ and
    ~\( \dots \)~
    ~\( \dark \langle \)~~\( \dark var_n \)~~\( \dark \rangle \)~ have the value ~\( \dark \langle \)~~\( \dark exp_n \)~~\( \dark \rangle \)~
in  ~\( \dark \langle \)~~\var{\dark body}~~\( \dark \rangle \)~
\end{scheme}

\noindent
The first part of the \code{let} expression is a list of name-expression pairs.
When the \code{let} is evaluated, each name is associated with the value of the
corresponding expression.  The body of the \code{let} is evaluated with these
names bound as local variables.  The way this happens is that the \code{let}
expression is interpreted as an alternate syntax for

\begin{scheme}
((lambda (~\( \dark \langle \)~~\( \dark var_1 \)~~\( \dark \rangle \)~ ~\( \dots \)~ ~\( \dark \langle \)~~\( \dark var_n \)~~\( \dark \rangle \)~)
    ~\( \dark \langle \)~~\var{\dark body}~~\( \dark \rangle \)~)
 ~\( \dark \langle \)~~\( \dark exp_1 \)~~\( \dark \rangle \)~
 ~\( \dots \)~
 ~\( \dark \langle \)~~\( \dark exp_n \)~~\( \dark \rangle \)~)
\end{scheme}

\noindent
No new mechanism is required in the interpreter in order to provide local
variables.  A \code{let} expression is simply syntactic sugar for the
underlying \code{lambda} application.

We can see from this equivalence that the scope of a variable specified by a
\code{let} expression is the body of the \code{let}.  This implies that:

\begin{itemize}

\item
\code{Let} allows one to bind variables as locally as possible to where they
are to be used.  For example, if the value of \code{x} is 5, the value of the
expression

\begin{scheme}
(+ (let ((x 3))
     (+ x (* x 10)))
   x)
\end{scheme}

\noindent
is 38.  Here, the \code{x} in the body of the \code{let} is 3, so the value of
the \code{let} expression is 33.  On the other hand, the \code{x} that is the
second argument to the outermost \code{+} is still 5.

\item
The variables' values are computed outside the \code{let}.  This matters when
the expressions that provide the values for the local variables depend upon
variables having the same names as the local variables themselves.  For
example, if the value of \code{x} is 2, the expression

\begin{scheme}
(let ((x 3)
      (y (+ x 2)))
  (* x y))
\end{scheme}

\noindent
will have the value 12 because, inside the body of the \code{let}, \code{x}
will be 3 and \code{y} will be 4 (which is the outer \code{x} plus 2).

\end{itemize}

\noindent
Sometimes we can use internal definitions to get the same effect as with
\code{let}.  For example, we could have defined the procedure \code{f} above as

\begin{scheme}
(define (f x y)
  (define a (+ 1 (* x y)))
  (define b (- 1 y))
  (+ (* x (square a))
     (* y b)
     (* a b)))
\end{scheme}

\noindent
We prefer, however, to use \code{let} in situations like this and to use
internal \code{define} only for internal procedures.\footnote{Understanding
internal definitions well enough to be sure a program means what we intend it
to mean requires a more elaborate model of the evaluation process than we have
presented in this chapter.  The subtleties do not arise with internal
definitions of procedures, however.  We will return to this issue in 
\link{Section 4.1.6}, after we learn more about evaluation.}

\begin{quote}
\heading{\phantomsection\label{Exercise 1.34}Exercise 1.34:} Suppose we define the procedure

\begin{scheme}
(define (f g) (g 2))
\end{scheme}

Then we have

\begin{scheme}
(f square)
~\textit{4}~
(f (lambda (z) (* z (+ z 1))))
~\textit{6}~
\end{scheme}

What happens if we (perversely) ask the interpreter to evaluate the combination
\code{(f f)}?  Explain.
\end{quote}

\subsection{Procedures as General Methods}
\label{Section 1.3.3}

We introduced compound procedures in \link{Section 1.1.4} as a mechanism for
abstracting patterns of numerical operations so as to make them independent of
the particular numbers involved.  With higher-order procedures, such as the
\code{integral} procedure of \link{Section 1.3.1}, we began to see a more
powerful kind of abstraction: procedures used to express general methods of
computation, independent of the particular functions involved.  In this section
we discuss two more elaborate examples---general methods for finding zeros and
fixed points of functions---and show how these methods can be expressed
directly as procedures.

\subsubsection*{Finding roots of equations by the half-interval method}

The \newterm{half-interval method} is a simple but powerful technique for
finding roots of an equation \( f(x) = 0 \), where \( f \) is a continuous
function.  The idea is that, if we are given points \( a \) and \( b \) such that
\( f(a) < 0 < f(b) \), then \( f \) must have at least one zero between
\( a \) and \( b \).  To locate a zero, let \( x \) be the average of \( a \) and \( b \),
and compute \( f(x) \).  If \( f(x) > 0 \), then \( f \) must have a zero
between \( a \) and \( x \).  If \( f(x) < 0 \), then \( f \) must have a zero
between \( x \) and \( b \).  Continuing in this way, we can identify smaller and
smaller intervals on which \( f \) must have a zero.  When we reach a point where
the interval is small enough, the process stops.  Since the interval of
uncertainty is reduced by half at each step of the process, the number of steps
required grows as \( \Theta(\log(L / T)) \), where \( L \) is the
length of the original interval and \( T \) is the error tolerance (that is, the
size of the interval we will consider ``small enough'').  Here is a procedure
that implements this strategy:

\begin{scheme}
(define (search f neg-point pos-point)
  (let ((midpoint (average neg-point pos-point)))
    (if (close-enough? neg-point pos-point)
        midpoint
        (let ((test-value (f midpoint)))
          (cond ((positive? test-value)
                 (search f neg-point midpoint))
                ((negative? test-value)
                 (search f midpoint pos-point))
                (else midpoint))))))
\end{scheme}

\noindent
We assume that we are initially given the function \( f \) together with points
at which its values are negative and positive.  We first compute the midpoint
of the two given points.  Next we check to see if the given interval is small
enough, and if so we simply return the midpoint as our answer.  Otherwise, we
compute as a test value the value of \( f \) at the midpoint.  If the test value
is positive, then we continue the process with a new interval running from the
original negative point to the midpoint.  If the test value is negative, we
continue with the interval from the midpoint to the positive point.  Finally,
there is the possibility that the test value is 0, in which case the midpoint
is itself the root we are searching for.

To test whether the endpoints are ``close enough'' we can use a procedure
similar to the one used in \link{Section 1.1.7} for computing square
roots:\footnote{We have used 0.001 as a representative ``small'' number to
indicate a tolerance for the acceptable error in a calculation.  The
appropriate tolerance for a real calculation depends upon the problem to be
solved and the limitations of the computer and the algorithm.  This is often a
very subtle consideration, requiring help from a numerical analyst or some
other kind of magician.}

\begin{scheme}
(define (close-enough? x y) (< (abs (- x y)) 0.001))
\end{scheme}

\noindent
\code{Search} is awkward to use directly, because we can accidentally give it
points at which \( f \)'s values do not have the required sign, in which case we
get a wrong answer.  Instead we will use \code{search} via the following
procedure, which checks to see which of the endpoints has a negative function
value and which has a positive value, and calls the \code{search} procedure
accordingly.  If the function has the same sign on the two given points, the
half-interval method cannot be used, in which case the procedure signals an
error.\footnote{This can be accomplished using \code{error}, which takes as
arguments a number of items that are printed as error messages.}

\begin{scheme}
(define (half-interval-method f a b)
  (let ((a-value (f a))
        (b-value (f b)))
    (cond ((and (negative? a-value) (positive? b-value))
           (search f a b))
          ((and (negative? b-value) (positive? a-value))
           (search f b a))
          (else
           (error "Values are not of opposite sign" a b)))))
\end{scheme}

\noindent
The following example uses the half-interval method to approximate \( \pi \) as
the root between 2 and 4 of \( \sin x = 0 \):

\begin{scheme}
(half-interval-method sin 2.0 4.0)
~\textit{3.14111328125}~
\end{scheme}

\noindent
Here is another example, using the half-interval method to search for a root of
the equation \( x^3 - 2x - 3 = 0 \) between 1 and 2:

\begin{scheme}
(half-interval-method (lambda (x) (- (* x x x) (* 2 x) 3))
                      1.0
                      2.0)
~\textit{1.89306640625}~
\end{scheme}

\subsubsection*{Finding fixed points of functions}

A number \( x \) is called a \newterm{fixed point} of a function \( f \) if \( x \)
satisfies the equation \( f(x) = x \).  For some functions \( f \) we can
locate a fixed point by beginning with an initial guess and applying \( f \)
repeatedly,
\begin{comment}

\begin{example}
f(x), f(f(x)), f(f(f(x))), ...
\end{example}

\end{comment}
\begin{displaymath}
 f(x),\quad f(f(x)),\quad f(f(f(x))), \quad\dots, 
\end{displaymath}
\noindent
until the value does not change very much.  Using this idea, we can devise a
procedure \code{fixed\-/point} that takes as inputs a function and an initial
guess and produces an approximation to a fixed point of the function.  We apply
the function repeatedly until we find two successive values whose difference is
less than some prescribed tolerance:

\begin{scheme}
(define tolerance 0.00001)
(define (fixed-point f first-guess)
  (define (close-enough? v1 v2)
    (< (abs (- v1 v2)) 
       tolerance))
  (define (try guess)
    (let ((next (f guess)))
      (if (close-enough? guess next)
          next
          (try next))))
  (try first-guess))
\end{scheme}

\noindent
For example, we can use this method to approximate the fixed point of the
cosine function, starting with 1 as an initial approximation:\footnote{Try this
during a boring lecture: Set your calculator to radians mode and then
repeatedly press the \code{cos} button until you obtain the fixed point.}

\begin{scheme}
(fixed-point cos 1.0)
~\textit{.7390822985224023}~
\end{scheme}

\noindent
Similarly, we can find a solution to the equation 
\( y = \sin y + \cos y \):

\begin{scheme}
(fixed-point (lambda (y) (+ (sin y) (cos y)))
             1.0)
~\textit{1.2587315962971173}~
\end{scheme}

\noindent
The fixed-point process is reminiscent of the process we used for finding
square roots in \link{Section 1.1.7}.  Both are based on the idea of repeatedly
improving a guess until the result satisfies some criterion.  In fact, we can
readily formulate the square-root computation as a fixed-point search.
Computing the square root of some number \( x \) requires finding a \( y \) such
that \( y^2 = x \).  Putting this equation into the equivalent form 
\( y = x / y \), we recognize that we are looking for a fixed point of the
function\footnote{\( \mapsto \) (pronounced ``maps to'') is the mathematician's way of
writing \code{lambda}.  \( y \mapsto x / y \) means \code{(lambda (y) (/ x y))},
that is, the function whose value at \( y \) is \( x / y \).} \( y \mapsto x / y \), 
and we can therefore try to compute square roots as

\begin{scheme}
(define (sqrt x)
  (fixed-point (lambda (y) (/ x y))
               1.0))
\end{scheme}

\noindent
Unfortunately, this fixed-point search does not converge.  Consider an initial
guess \( y_1 \).  The next guess is \( y_2 = x / y_1 \) and the next guess is
\( y_3 = x / y_2 = x / (x / y_1) = y_1 \).  This results in an
infinite loop in which the two guesses \( y_1 \) and \( y_2 \) repeat over and
over, oscillating about the answer.

One way to control such oscillations is to prevent the guesses from changing so
much.  Since the answer is always between our guess \( y \) and \( x / y \), we
can make a new guess that is not as far from \( y \) as \( x / y \) by averaging
\( y \) with \( x / y \), so that the next guess after \( y \) is 
\( {1\over2}(y + x / y) \) 
instead of \( x / y \).  The process of making such a sequence of
guesses is simply the process of looking for a fixed point of 
\( y \mapsto {1\over2}(y + x / y) \): 

\begin{scheme}
(define (sqrt x)
  (fixed-point (lambda (y) (average y (/ x y)))
               1.0))
\end{scheme}

\noindent
(Note that \( y = {1\over2}(y + x / y) \) is a simple transformation of the
equation \( y = x / y; \) to derive it, add \( y \) to both sides of the
equation and divide by 2.)

With this modification, the square-root procedure works.  In fact, if we
unravel the definitions, we can see that the sequence of approximations to the
square root generated here is precisely the same as the one generated by our
original square-root procedure of \link{Section 1.1.7}.  This approach of
averaging successive approximations to a solution, a technique that we call
\newterm{average damping}, often aids the convergence of fixed-point searches.

\begin{quote}
\heading{\phantomsection\label{Exercise 1.35}Exercise 1.35:} Show that the golden ratio
\( \varphi \) (\link{Section 1.2.2}) is a fixed point of the transformation 
\( x \mapsto 1 + 1 / x \), and use this fact to compute \( \varphi \) by means 
of the \code{fixed\-/point} procedure.
\end{quote}

\begin{quote}
\heading{\phantomsection\label{Exercise 1.36}Exercise 1.36:} Modify \code{fixed\-/point} so that
it prints the sequence of approximations it generates, using the \code{newline}
and \code{display} primitives shown in \link{Exercise 1.22}.  Then find a
solution to \( x^x = 1000 \) by finding a fixed point of \( x \mapsto
\log(1000) / \log(x) \).  (Use Scheme's primitive \code{log}
procedure, which computes natural logarithms.)  Compare the number of steps
this takes with and without average damping.  (Note that you cannot start
\code{fixed\-/point} with a guess of 1, as this would cause division by
\( \log(1) = 0 \).)
\end{quote}

\begin{quote}
\heading{\phantomsection\label{Exercise 1.37}Exercise 1.37:} \begin{enumerate}[a.]

\item
An infinite \newterm{continued fraction} is an expression of the form
\begin{comment}

\begin{example}
           N_1
f = ---------------------
               N_2
    D_1 + ---------------
                   N_3
          D_2 + ---------
                D_3 + ...
\end{example}

\end{comment}
\begin{displaymath}
 {f} = \cfrac{N_1}{D_1 + \cfrac{N_2}{D_2 + \cfrac{N_3}{D_3 + \dots}}}\,. 
\end{displaymath}
As an example, one can show that the infinite continued fraction expansion with
the \( N_i \) and the \( D_i \) all equal to 1 produces \( 1 / \varphi \), where
\( \varphi \) is the golden ratio (described in \link{Section 1.2.2}).  One way to
approximate an infinite continued fraction is to truncate the expansion after a
given number of terms.  Such a truncation---a so-called \newterm{\textit{k}-term
finite continued fraction}---has the form
\begin{comment}

\begin{example}
       N_1
-----------------
          N_2
D_1 + -----------
      ...    N_K
          + -----
             D_K
\end{example}

\end{comment}
\begin{displaymath}
 \cfrac{N_1}{D_1 + \cfrac{N_2}{\ddots + \cfrac{N_k}{D_k}}}\,. 
\end{displaymath}
Suppose that \code{n} and \code{d} are procedures of one argument (the term
index \( i \)) that return the \( N_i \) and \( D_i \) of the terms of the
continued fraction.  Define a procedure \code{cont\-/frac} such that evaluating
\code{(cont\-/frac n d k)} computes the value of the \( k \)-term finite continued
fraction.  Check your procedure by approximating \( 1 / \varphi \) using

\begin{scheme}
(cont-frac (lambda (i) 1.0)
           (lambda (i) 1.0)
           k)
\end{scheme}

\noindent
for successive values of \code{k}.  How large must you make \code{k} in order
to get an approximation that is accurate to 4 decimal places?

\item
If your \code{cont\-/frac} procedure generates a recursive process, write one
that generates an iterative process.  If it generates an iterative process,
write one that generates a recursive process.

\end{enumerate}
\end{quote}

\begin{quote}
\heading{\phantomsection\label{Exercise 1.38}Exercise 1.38:} In 1737, the Swiss mathematician
Leonhard Euler published a memoir \textit{De Fractionibus Continuis}, which
included a continued fraction expansion for \( e - 2 \), where \( e \) is the base
of the natural logarithms.  In this fraction, the \( N_i \) are all 1, and
the \( D_i \) are successively 1, 2, 1, 1, 4, 1, 1, 6, 1, 1, 8, \( \dots \).
Write a program that uses your \code{cont\-/frac} procedure from \link{Exercise 1.37} 
to approximate \( e \), based on Euler's expansion.
\end{quote}

\begin{quote}
\heading{\phantomsection\label{Exercise 1.39}Exercise 1.39:} A continued fraction
representation of the tangent function was published in 1770 by the German
mathematician J.H. Lambert:
\begin{comment}

\begin{example}
              x
tan x = ---------------
                x^2
        1 - -----------
                  x^2
            3 - -------
                5 - ...
\end{example}

\end{comment}
\begin{displaymath}
 {\tan x} = \cfrac{x}{1 - \cfrac{x^2}{3 - \cfrac{x^2}{5 - \dots}}}\,, 
\end{displaymath}
\noindent
where \( x \) is in radians.  Define a procedure \code{(tan\-/cf x k)} that
computes an approximation to the tangent function based on Lambert's formula.
\code{k} specifies the number of terms to compute, as in \link{Exercise 1.37}.
\end{quote}

\subsection{Procedures as Returned Values}
\label{Section 1.3.4}

The above examples demonstrate how the ability to pass procedures as arguments
significantly enhances the expressive power of our programming language.  We
can achieve even more expressive power by creating procedures whose returned
values are themselves procedures.

We can illustrate this idea by looking again at the fixed-point example
described at the end of \link{Section 1.3.3}.  We formulated a new version of
the square-root procedure as a fixed-point search, starting with the
observation that \( \sqrt{x} \) is a fixed-point of the function \( y \mapsto
x / y \).  Then we used average damping to make the approximations converge.
Average damping is a useful general technique in itself.  Namely, given a
function \( f \), we consider the function whose value at \( x \) is equal to the
average of \( x \) and \( f(x) \).

We can express the idea of average damping by means of the following procedure:

\begin{scheme}
(define (average-damp f)
  (lambda (x) (average x (f x))))
\end{scheme}

\noindent
\code{Average\-/damp} is a procedure that takes as its argument a procedure
\code{f} and returns as its value a procedure (produced by the \code{lambda})
that, when applied to a number \code{x}, produces the average of \code{x} and
\code{(f x)}.  For example, applying \code{average\-/damp} to the \code{square}
procedure produces a procedure whose value at some number \( x \) is the average
of \( x \) and \( x^2 \).  Applying this resulting procedure to 10 returns the
average of 10 and 100, or 55:\footnote{Observe that this is a combination whose
operator is itself a combination.  \link{Exercise 1.4} already demonstrated the
ability to form such combinations, but that was only a toy example.  Here we
begin to see the real need for such combinations---when applying a procedure
that is obtained as the value returned by a higher-order procedure.}

\begin{scheme}
((average-damp square) 10)
~\textit{55}~
\end{scheme}

\noindent
Using \code{average\-/damp}, we can reformulate the square-root procedure as
follows:

\begin{scheme}
(define (sqrt x)
  (fixed-point (average-damp (lambda (y) (/ x y)))
               1.0))
\end{scheme}

\noindent
Notice how this formulation makes explicit the three ideas in the method:
fixed-point search, average damping, and the function \( y \mapsto x / y \).
It is instructive to compare this formulation of the square-root method with
the original version given in \link{Section 1.1.7}.  Bear in mind that these
procedures express the same process, and notice how much clearer the idea
becomes when we express the process in terms of these abstractions.  In
general, there are many ways to formulate a process as a procedure.
Experienced programmers know how to choose procedural formulations that are
particularly perspicuous, and where useful elements of the process are exposed
as separate entities that can be reused in other applications.  As a simple
example of reuse, notice that the cube root of \( x \) is a fixed point of the
function \( y \mapsto x / y^2 \), so we can immediately generalize our
square-root procedure to one that extracts cube roots:\footnote{See
\link{Exercise 1.45} for a further generalization.}

\begin{scheme}
(define (cube-root x)
  (fixed-point (average-damp (lambda (y) (/ x (square y))))
               1.0))
\end{scheme}

\subsubsection*{Newton's method}

When we first introduced the square-root procedure, in \link{Section 1.1.7}, we
mentioned that this was a special case of \newterm{Newton's method}.  If \( x
\mapsto g(x) \) is a differentiable function, then a solution of the equation
\( g(x) = 0 \) is a fixed point of the function \( x \mapsto f(x) \), where
\begin{comment}

\begin{example}
           g(x)
f(x) = x - -----
           Dg(x)
\end{example}

\end{comment}
\begin{displaymath}
 {f(x) = x} - {g(x)\over Dg(x)} 
\end{displaymath}
\noindent
and \( Dg(x) \) is the derivative of \( g \) evaluated at \( x \).  Newton's
method is the use of the fixed-point method we saw above to approximate a
solution of the equation by finding a fixed point of the function
\( f\!. \)\footnote{Elementary calculus books usually describe Newton's method in
terms of the sequence of approximations \( x_{n+1} = x_n -
g(x_n) / Dg(x_n) \).  Having language for talking about
processes and using the idea of fixed points simplifies the description of the
method.}

For many functions \( g \) and for sufficiently good initial guesses for \( x \),
Newton's method converges very rapidly to a solution of \( g(x) = 0. \)\footnote{Newton's method does not always converge to an answer, but it can
be shown that in favorable cases each iteration doubles the number-of-digits
accuracy of the approximation to the solution.  In such cases, Newton's method
will converge much more rapidly than the half-interval method.}

In order to implement Newton's method as a procedure, we must first express the
idea of derivative.  Note that ``derivative,'' like average damping, is
something that transforms a function into another function.  For instance, the
derivative of the function \( x \mapsto x^3 \) is the function \( x \mapsto 3x^2\!. \)  
In general, if \( g \) is a function and \( dx \) is a small number,
then the derivative \( Dg \) of \( g \) is the function whose value at any
number \( x \) is given (in the limit of small \( dx \)) by
\begin{comment}

\begin{example}
        g(x + dx) - g(x)
Dg(x) = ----------------
               dx
\end{example}

\end{comment}
\begin{displaymath}
 {Dg(x)} = {g(x + {\it dx}) - g(x) \over {\it dx}}\,. 
\end{displaymath}
\noindent
Thus, we can express the idea of derivative (taking \( dx \) to be, say,
0.00001) as the procedure

\begin{scheme}
(define (deriv g)
  (lambda (x) (/ (- (g (+ x dx)) (g x)) dx)))
\end{scheme}

\noindent
along with the definition

\begin{scheme}
(define dx 0.00001)
\end{scheme}

\noindent
Like \code{average\-/damp}, \code{deriv} is a procedure that takes a procedure as
argument and returns a procedure as value.  For example, to approximate the
derivative of \( x \mapsto x^3 \) at 5 (whose exact value is 75) we can evaluate

\begin{scheme}
(define (cube x) (* x x x))
((deriv cube) 5)
~\textit{75.00014999664018}~
\end{scheme}

\noindent
With the aid of \code{deriv}, we can express Newton's method as a fixed-point
process:

\begin{scheme}
(define (newton-transform g)
  (lambda (x) (- x (/ (g x) ((deriv g) x)))))
(define (newtons-method g guess)
  (fixed-point (newton-transform g) guess))
\end{scheme}

\noindent
The \code{newton\-/transform} procedure expresses the formula at the beginning of
this section, and \code{newtons\-/method} is readily defined in terms of this.
It takes as arguments a procedure that computes the function for which we want
to find a zero, together with an initial guess.  For instance, to find the
square root of \( x \), we can use Newton's method to find a zero of the function
\( y \mapsto y^2 - x \) starting with an initial guess of 1.\footnote{For
finding square roots, Newton's method converges rapidly to the correct solution
from any starting point.}

This provides yet another form of the square-root procedure:

\begin{scheme}
(define (sqrt x)
  (newtons-method
   (lambda (y) (- (square y) x)) 1.0))
\end{scheme}

\subsubsection*{Abstractions and first-class procedures}

We've seen two ways to express the square-root computation as an instance of a
more general method, once as a fixed-point search and once using Newton's
method.  Since Newton's method was itself expressed as a fixed-point process,
we actually saw two ways to compute square roots as fixed points.  Each method
begins with a function and finds a fixed point of some transformation of the
function.  We can express this general idea itself as a procedure:

\begin{scheme}
(define (fixed-point-of-transform g transform guess)
  (fixed-point (transform g) guess))
\end{scheme}

\noindent
This very general procedure takes as its arguments a procedure \code{g} that
computes some function, a procedure that transforms \code{g}, and an initial
guess.  The returned result is a fixed point of the transformed function.

Using this abstraction, we can recast the first square-root computation from
this section (where we look for a fixed point of the average-damped version of
\( y \mapsto x / y \)) as an instance of this general method:

\begin{scheme}
(define (sqrt x)
  (fixed-point-of-transform
   (lambda (y) (/ x y)) average-damp 1.0))
\end{scheme}

\noindent
Similarly, we can express the second square-root computation from this section
(an instance of Newton's method that finds a fixed point of the Newton
transform of \( y \mapsto y^2 - x \)) as

\begin{scheme}
(define (sqrt x)
  (fixed-point-of-transform
   (lambda (y) (- (square y) x)) newton-transform 1.0))
\end{scheme}

\noindent
We began \link{Section 1.3} with the observation that compound procedures are a
crucial abstraction mechanism, because they permit us to express general
methods of computing as explicit elements in our programming language.  Now
we've seen how higher-order procedures permit us to manipulate these general
methods to create further abstractions.

As programmers, we should be alert to opportunities to identify the underlying
abstractions in our programs and to build upon them and generalize them to
create more powerful abstractions.  This is not to say that one should always
write programs in the most abstract way possible; expert programmers know how
to choose the level of abstraction appropriate to their task.  But it is
important to be able to think in terms of these abstractions, so that we can be
ready to apply them in new contexts.  The significance of higher-order
procedures is that they enable us to represent these abstractions explicitly as
elements in our programming language, so that they can be handled just like
other computational elements.

In general, programming languages impose restrictions on the ways in which
computational elements can be manipulated.  Elements with the fewest
restrictions are said to have \newterm{first-class} status.  Some of the
``rights and privileges'' of first-class elements are:\footnote{The notion of
first-class status of programming-language elements is due to the British
computer scientist Christopher Strachey (1916-1975).}

\begin{itemize}

\item
They may be named by variables.

\item
They may be passed as arguments to procedures.

\item
They may be returned as the results of procedures.

\item
They may be included in data structures.\footnote{We'll see examples of this
after we introduce data structures in \link{Chapter 2}.}

\end{itemize}

\noindent
Lisp, unlike other common programming languages, awards procedures full
first-class status.  This poses challenges for efficient implementation, but
the resulting gain in expressive power is enormous.\footnote{The major
implementation cost of first-class procedures is that allowing procedures to be
returned as values requires reserving storage for a procedure's free variables
even while the procedure is not executing.  In the Scheme implementation we
will study in \link{Section 4.1}, these variables are stored in the procedure's
environment.}

\begin{quote}
\heading{\phantomsection\label{Exercise 1.40}Exercise 1.40:} Define a procedure \code{cubic}
that can be used together with the \code{newtons\-/method} procedure in
expressions of the form

\begin{scheme}
(newtons-method (cubic a b c) 1)
\end{scheme}

\noindent
to approximate zeros of the cubic \( x^3 + ax^2 + bx + c \).
\end{quote}

\begin{quote}
\heading{\phantomsection\label{Exercise 1.41}Exercise 1.41:} Define a procedure \code{double}
that takes a procedure of one argument as argument and returns a procedure that
applies the original procedure twice.  For example, if \code{inc} is a
procedure that adds 1 to its argument, then \code{(double inc)} should be a
procedure that adds 2.  What value is returned by

\begin{scheme}
(((double (double double)) inc) 5)
\end{scheme}
\end{quote}

\begin{quote}
\heading{\phantomsection\label{Exercise 1.42}Exercise 1.42:} Let \( f \) and \( g \) be two
one-argument functions.  The \newterm{composition} \( f \) after \( g \) is defined
to be the function \( x \mapsto f(g(x)) \).  Define a procedure
\code{compose} that implements composition.  For example, if \code{inc} is a
procedure that adds 1 to its argument,

\begin{scheme}
((compose square inc) 6)
~\textit{49}~
\end{scheme}
\end{quote}

\begin{quote}
\heading{\phantomsection\label{Exercise 1.43}Exercise 1.43:} If \( f \) is a numerical function
and \( n \) is a positive integer, then we can form the \( n^{\mathrm{th}} \) repeated
application of \( f \), which is defined to be the function whose value at \( x \)
is \( f(f(\dots (f(x))\dots )) \).  For example, if \( f \) is the
function \( x \mapsto x + 1 \), then the \( n^{\mathrm{th}} \) repeated application of \( f \) is
the function \( x \mapsto x + n \).  If \( f \) is the operation of squaring a
number, then the \( n^{\mathrm{th}} \) repeated application of \( f \) is the function that
raises its argument to the \( 2^n \)-th power.  Write a procedure that takes as
inputs a procedure that computes \( f \) and a positive integer \( n \) and returns
the procedure that computes the \( n^{\mathrm{th}} \) repeated application of \( f \).  Your
procedure should be able to be used as follows:

\begin{scheme}
((repeated square 2) 5)
~\textit{625}~
\end{scheme}

Hint: You may find it convenient to use \code{compose} from \link{Exercise 1.42}.
\end{quote}

\begin{quote}
\heading{\phantomsection\label{Exercise 1.44}Exercise 1.44:} The idea of \newterm{smoothing} a
function is an important concept in signal processing.  If \( f \) is a function
and \( dx \) is some small number, then the smoothed version of \( f \) is the
function whose value at a point \( x \) is the average of \( f(x - dx) \), 
\( f(x) \), and \( f(x + dx) \).  Write a procedure
\code{smooth} that takes as input a procedure that computes \( f \) and returns a
procedure that computes the smoothed \( f \).  It is sometimes valuable to
repeatedly smooth a function (that is, smooth the smoothed function, and so on)
to obtain the \newterm{\textit{n}-fold smoothed function}.  Show how to generate
the \textit{n}-fold smoothed function of any given function using \code{smooth} and
\code{repeated} from \link{Exercise 1.43}.
\end{quote}

\begin{quote}
\heading{\phantomsection\label{Exercise 1.45}Exercise 1.45:} We saw in \link{Section 1.3.3}
that attempting to compute square roots by naively finding a fixed point of
\( y \mapsto x / y \) does not converge, and that this can be fixed by average
damping.  The same method works for finding cube roots as fixed points of the
average-damped \( y \mapsto x / y^2 \).  Unfortunately, the process does not
work for fourth roots---a single average damp is not enough to make a
fixed-point search for \( y \mapsto x / y^3 \) converge.  On the other hand, if
we average damp twice (i.e., use the average damp of the average damp of 
\( y \mapsto x / y^3 \)) the fixed-point search does converge.  Do some experiments
to determine how many average damps are required to compute \( n^{\mathrm{th}} \) roots as a
fixed-point search based upon repeated average damping of \( y \mapsto x / y^{n-1} \).  
Use this to implement a simple procedure for computing
\( n^{\mathrm{th}} \) roots using \code{fixed\-/point}, \code{average\-/damp}, and the
\code{repeated} procedure of \link{Exercise 1.43}.  Assume that any arithmetic
operations you need are available as primitives.
\end{quote}

\begin{quote}
\heading{\phantomsection\label{Exercise 1.46}Exercise 1.46:} Several of the numerical methods
described in this chapter are instances of an extremely general computational
strategy known as \newterm{iterative improvement}.  Iterative improvement says
that, to compute something, we start with an initial guess for the answer, test
if the guess is good enough, and otherwise improve the guess and continue the
process using the improved guess as the new guess.  Write a procedure
\code{iterative\-/improve} that takes two procedures as arguments: a method for
telling whether a guess is good enough and a method for improving a guess.
\code{Iterative\-/improve} should return as its value a procedure that takes a
guess as argument and keeps improving the guess until it is good enough.
Rewrite the \code{sqrt} procedure of \link{Section 1.1.7} and the
\code{fixed\-/point} procedure of \link{Section 1.3.3} in terms of
\code{iterative\-/improve}.
\end{quote}

\chapter{Building Abstractions with Data}
\label{Chapter 2}

\begin{quote}
We now come to the decisive step of mathematical abstraction: we forget about
what the symbols stand for. \( \dots \)[The mathematician] need not be idle; there
are many operations which he may carry out with these symbols, without ever
having to look at the things they stand for.

---Hermann Weyl, \textit{The Mathematical Way of Thinking}
\end{quote}

% \vspace{1.0em}

\noindent
\lettrine[findent=0pt]{W}{e concentrated in} \link{Chapter 1} on computational processes and on the role
of procedures in program design.  We saw how to use primitive data (numbers)
and primitive operations (arithmetic operations), how to combine procedures to
form compound procedures through composition, conditionals, and the use of
parameters, and how to abstract procedures by using \code{define}.  We saw that
a procedure can be regarded as a pattern for the local evolution of a process,
and we classified, reasoned about, and performed simple algorithmic analyses of
some common patterns for processes as embodied in procedures.  We also saw that
higher-order procedures enhance the power of our language by enabling us to
manipulate, and thereby to reason in terms of, general methods of computation.
This is much of the essence of programming.

In this chapter we are going to look at more complex data.  All the procedures
in chapter 1 operate on simple numerical data, and simple data are not
sufficient for many of the problems we wish to address using computation.
Programs are typically designed to model complex phenomena, and more often than
not one must construct computational objects that have several parts in order
to model real-world phenomena that have several aspects.  Thus, whereas our
focus in chapter 1 was on building abstractions by combining procedures
to form compound procedures, we turn in this chapter to another key aspect of
any programming language: the means it provides for building abstractions by
combining data objects to form \newterm{compound data}.

Why do we want compound data in a programming language?  For the same reasons
that we want compound procedures: to elevate the conceptual level at which we
can design our programs, to increase the modularity of our designs, and to
enhance the expressive power of our language.  Just as the ability to define
procedures enables us to deal with processes at a higher conceptual level than
that of the primitive operations of the language, the ability to construct
compound data objects enables us to deal with data at a higher conceptual level
than that of the primitive data objects of the language.

\enlargethispage{\baselineskip}

Consider the task of designing a system to perform arithmetic with rational
numbers.  We could imagine an operation \code{add\-/rat} that takes two rational
numbers and produces their sum.  In terms of simple data, a rational number can
be thought of as two integers: a numerator and a denominator.  Thus, we could
design a program in which each rational number would be represented by two
integers (a numerator and a denominator) and where \code{add\-/rat} would be
implemented by two procedures (one producing the numerator of the sum and one
producing the denominator).  But this would be awkward, because we would then
need to explicitly keep track of which numerators corresponded to which
denominators.  In a system intended to perform many operations on many rational
numbers, such bookkeeping details would clutter the programs substantially, to
say nothing of what they would do to our minds.  It would be much better if we
could ``glue together'' a numerator and denominator to form a pair---a
\newterm{compound data object}---that our programs could manipulate in a way
that would be consistent with regarding a rational number as a single
conceptual unit.

The use of compound data also enables us to increase the modularity of our
programs.  If we can manipulate rational numbers directly as objects in their
own right, then we can separate the part of our program that deals with
rational numbers per se from the details of how rational numbers may be
represented as pairs of integers.  The general technique of isolating the parts
of a program that deal with how data objects are represented from the parts of
a program that deal with how data objects are used is a powerful design
methodology called \newterm{data abstraction}.  We will see how data
abstraction makes programs much easier to design, maintain, and modify.

The use of compound data leads to a real increase in the expressive
power of our programming language.  Consider the idea of forming a
``linear combination'' \( ax + by \).  We might like to write
a procedure that would accept \( a \), \( b \), \( x \), and \( y \) as
arguments and return the value of \( ax + by \).  This
presents no difficulty if the arguments are to be numbers, because we
can readily define the procedure

\begin{scheme}
(define (linear-combination a b x y)
  (+ (* a x) (* b y)))
\end{scheme}

\noindent
But suppose we are not concerned only with numbers.  Suppose we would like to
express, in procedural terms, the idea that one can form linear combinations
whenever addition and multiplication are defined---for rational numbers,
complex numbers, polynomials, or whatever.  We could express this as a
procedure of the form

\begin{scheme}
(define (linear-combination a b x y)
  (add (mul a x) (mul b y)))
\end{scheme}

\noindent
where \code{add} and \code{mul} are not the primitive procedures \code{+} and
\code{*} but rather more complex things that will perform the appropriate
operations for whatever kinds of data we pass in as the arguments \code{a},
\code{b}, \code{x}, and \code{y}. The key point is that the only thing
\code{linear\-/combination} should need to know about \code{a}, \code{b},
\code{x}, and \code{y} is that the procedures \code{add} and \code{mul} will
perform the appropriate manipulations.  From the perspective of the procedure
\code{linear\-/combination}, it is irrelevant what \code{a}, \code{b}, \code{x},
and \code{y} are and even more irrelevant how they might happen to be
represented in terms of more primitive data.  This same example shows why it is
important that our programming language provide the ability to manipulate
compound objects directly: Without this, there is no way for a procedure such
as \code{linear\-/combination} to pass its arguments along to \code{add} and
\code{mul} without having to know their detailed structure.\footnote{The
ability to directly manipulate procedures provides an analogous increase in the
expressive power of a programming language.  For example, in 
\link{Section 1.3.1} we introduced the \code{sum} procedure, which takes a procedure
\code{term} as an argument and computes the sum of the values of \code{term}
over some specified interval.  In order to define \code{sum}, it is crucial
that we be able to speak of a procedure such as \code{term} as an entity in its
own right, without regard for how \code{term} might be expressed with more
primitive operations.  Indeed, if we did not have the notion of ``a
procedure,'' it is doubtful that we would ever even think of the possibility of
defining an operation such as \code{sum}.  Moreover, insofar as performing the
summation is concerned, the details of how \code{term} may be constructed from
more primitive operations are irrelevant.}

We begin this chapter by implementing the rational-number arithmetic system
mentioned above.  This will form the background for our discussion of compound
data and data abstraction.  As with compound procedures, the main issue to be
addressed is that of abstraction as a technique for coping with complexity, and
we will see how data abstraction enables us to erect suitable
\newterm{abstraction barriers} between different parts of a program.

We will see that the key to forming compound data is that a programming
language should provide some kind of ``glue'' so that data objects can be
combined to form more complex data objects.  There are many possible kinds of
glue.  Indeed, we will discover how to form compound data using no special
``data'' operations at all, only procedures.  This will further blur the
distinction between ``procedure'' and ``data,'' which was already becoming
tenuous toward the end of chapter 1.  We will also explore some
conventional techniques for representing sequences and trees.  One key idea in
dealing with compound data is the notion of \newterm{closure}---that the glue
we use for combining data objects should allow us to combine not only primitive
data objects, but compound data objects as well.  Another key idea is that
compound data objects can serve as \newterm{conventional interfaces} for
combining program modules in mix-and-match ways.  We illustrate some of these
ideas by presenting a simple graphics language that exploits closure.

\enlargethispage{\baselineskip}

We will then augment the representational power of our language by introducing
\newterm{symbolic expressions}---data whose elementary parts can be arbitrary
symbols rather than only numbers.  We explore various alternatives for
representing sets of objects.  We will find that, just as a given numerical
function can be computed by many different computational processes, there are
many ways in which a given data structure can be represented in terms of
simpler objects, and the choice of representation can have significant impact
on the time and space requirements of processes that manipulate the data.  We
will investigate these ideas in the context of symbolic differentiation, the
representation of sets, and the encoding of information.

Next we will take up the problem of working with data that may be represented
differently by different parts of a program.  This leads to the need to
implement \newterm{generic operations}, which must handle many different types
of data.  Maintaining modularity in the presence of generic operations requires
more powerful abstraction barriers than can be erected with simple data
abstraction alone.  In particular, we introduce \newterm{data-directed
programming} as a technique that allows individual data representations to be
designed in isolation and then combined \newterm{additively} (i.e., without
modification).  To illustrate the power of this approach to system design, we
close the chapter by applying what we have learned to the implementation of a
package for performing symbolic arithmetic on polynomials, in which the
coefficients of the polynomials can be integers, rational numbers, complex
numbers, and even other polynomials.



\section{Introduction to Data Abstraction}
\label{Section 2.1}

In \link{Section 1.1.8}, we noted that a procedure used as an element in
creating a more complex procedure could be regarded not only as a collection of
particular operations but also as a procedural abstraction.  That is, the
details of how the procedure was implemented could be suppressed, and the
particular procedure itself could be replaced by any other procedure with the
same overall behavior.  In other words, we could make an abstraction that would
separate the way the procedure would be used from the details of how the
procedure would be implemented in terms of more primitive procedures.  The
analogous notion for compound data is called \newterm{data abstraction}.  Data
abstraction is a methodology that enables us to isolate how a compound data
object is used from the details of how it is constructed from more primitive
data objects.

The basic idea of data abstraction is to structure the programs that are to use
compound data objects so that they operate on ``abstract data.'' That is, our
programs should use data in such a way as to make no assumptions about the data
that are not strictly necessary for performing the task at hand.  At the same
time, a ``concrete'' data representation is defined independent of the programs
that use the data.  The interface between these two parts of our system will be
a set of procedures, called \newterm{selectors} and \newterm{constructors},
that implement the abstract data in terms of the concrete representation.  To
illustrate this technique, we will consider how to design a set of procedures
for manipulating rational numbers.



\subsection{Example: Arithmetic Operations\\ for Rational Numbers}
\label{Section 2.1.1}

Suppose we want to do arithmetic with rational numbers.  We want to be able to
add, subtract, multiply, and divide them and to test whether two rational
numbers are equal.

Let us begin by assuming that we already have a way of constructing a rational
number from a numerator and a denominator.  We also assume that, given a
rational number, we have a way of extracting (or selecting) its numerator and
its denominator.  Let us further assume that the constructor and selectors are
available as procedures:

\begin{itemize}

\item
\( \hbox{\tt(make-rat}\;\langle{n}\kern0.08em\rangle\;\langle{d}\kern0.06em\rangle\hbox{\tt)} \) returns the rational number whose numerator is the integer \( \langle{n}\kern0.08em\rangle \) and whose denominator is the integer
\( \langle{d}\kern0.06em\rangle \).

\item
\( \hbox{\tt(numer}\;\;\langle{x}\kern0.08em\rangle\hbox{\tt)} \) returns the numerator of the rational number
\( \langle{x}\kern0.08em\rangle \).

\item
\( \hbox{\tt(denom}\;\;\langle{x}\kern0.08em\rangle\hbox{\tt)} \) returns the denominator of the rational number
\( \langle{x}\kern0.08em\rangle \).

\end{itemize}

\noindent
We are using here a powerful strategy of synthesis: \newterm{wishful thinking}.
We haven't yet said how a rational number is represented, or how the procedures
\code{numer}, \code{denom}, and \code{make\-/rat} should be implemented.  Even
so, if we did have these three procedures, we could then add, subtract,
multiply, divide, and test equality by using the following relations:
\begin{comment}

\begin{example}
n_1   n_2   n_1 d_2 + n_2 d_1
--- + --- = -----------------
d_1   d_2        d_1 d_2

n_1   n_2   n_1 d_2 - n_2 d_1
--- - --- = -----------------
d_1   d_2        d_1 d_2

n_1   n_2   n_1 n_2
--- * --- = -------
d_1   d_2   d_1 d_2

n_1 / d_1   n_1 d_2
--------- = -------
n_2 / d_2   d_1 n_2

n_1   n_2
--- = ---  if and only if n_1 d_2 = n_2 d_1
d_1   d_2
\end{example}
\end{comment}
\begin{align*}
  {n_1 \over d_1} + {n_2 \over d_2} 	&= {n_1 d_2 + n_2 d_1 \over d_1 d_2}, \\
  {n_1 \over d_1} - {n_2 \over d_2} 	&= {n_1 d_2 - n_2 d_1 \over d_1 d_2}, \\
  {n_1 \over d_1} \cdot {n_2 \over d_2}	&= {n_1 n_2 \over d_1 d_2}, \\
  {n_1 / d_1} \over {n_2 / d_2} 	&= {n_1 d_2 \over d_1 n_2}, \\
  {n_1 \over d_1} 			&= {n_2 \over d_2} \quad 
						{\rm\ if\ and\ only\ if\quad} 
						n_1 d_2 = n_2 d_1.
\end{align*}
\noindent
We can express these rules as procedures:

\begin{scheme}
(define (add-rat x y)
  (make-rat (+ (* (numer x) (denom y))
               (* (numer y) (denom x)))
            (* (denom x) (denom y))))
(define (sub-rat x y)
  (make-rat (- (* (numer x) (denom y))
               (* (numer y) (denom x)))
            (* (denom x) (denom y))))
(define (mul-rat x y)
  (make-rat (* (numer x) (numer y))
            (* (denom x) (denom y))))
(define (div-rat x y)
  (make-rat (* (numer x) (denom y))
            (* (denom x) (numer y))))

(define (equal-rat? x y)
  (= (* (numer x) (denom y))
     (* (numer y) (denom x))))
\end{scheme}

Now we have the operations on rational numbers defined in terms of the selector
and constructor procedures \code{numer}, \code{denom}, and \code{make\-/rat}.
But we haven't yet defined these.  What we need is some way to glue together a
numerator and a denominator to form a rational number.

\subsubsection*{Pairs}

To enable us to implement the concrete level of our data abstraction, our
language provides a compound structure called a \newterm{pair}, which can be
constructed with the primitive procedure \code{cons}.  This procedure takes two
arguments and returns a compound data object that contains the two arguments as
parts.  Given a pair, we can extract the parts using the primitive procedures
\code{car} and \code{cdr}.\footnote{The name \code{cons} stands for
``construct.''  The names \code{car} and \code{cdr} derive from the original
implementation of Lisp on the IBM 704.  That machine had an addressing scheme
that allowed one to reference the ``address'' and ``decrement'' parts of a
memory location.  \code{Car} stands for ``Contents of Address part of
Register'' and \code{cdr} (pronounced ``could-er'') stands for ``Contents of
Decrement part of Register.''} Thus, we can use \code{cons}, \code{car}, and
\code{cdr} as follows:

\begin{scheme}
(define x (cons 1 2))
(car x)
~\textit{1}~
(cdr x)
~\textit{2}~
\end{scheme}

\noindent
Notice that a pair is a data object that can be given a name and manipulated,
just like a primitive data object.  Moreover, \code{cons} can be used to form
pairs whose elements are pairs, and so on:

\begin{scheme}
(define x (cons 1 2))
(define y (cons 3 4))
(define z (cons x y))
(car (car z))
~\textit{1}~
(car (cdr z))
~\textit{3}~
\end{scheme}

\noindent
In \link{Section 2.2} we will see how this ability to combine pairs means that
pairs can be used as general-purpose building blocks to create all sorts of
complex data structures.  The single compound-data primitive \newterm{pair},
implemented by the procedures \code{cons}, \code{car}, and \code{cdr}, is the
only glue we need.  Data objects constructed from pairs are called
\newterm{list-structured} data.

\enlargethispage{\baselineskip}

\subsubsection*{Representing rational numbers}

Pairs offer a natural way to complete the rational-number system.  Simply
represent a rational number as a pair of two integers: a numerator and a
denominator.  Then \code{make\-/rat}, \code{numer}, and \code{denom} are readily
implemented as follows:\footnote{Another way to define the selectors and
constructor is

\begin{smallscheme}
(define make-rat cons)
(define numer car)
(define denom cdr)
\end{smallscheme}

The first definition associates the name \code{make\-/rat} with the value of the
expression \code{cons}, which is the primitive procedure that constructs pairs.
Thus \code{make\-/rat} and \code{cons} are names for the same primitive
constructor.

Defining selectors and constructors in this way is efficient: Instead of
\code{make\-/rat} \emph{calling} \code{cons}, \code{make\-/rat} \emph{is}
\code{cons}, so there is only one procedure called, not two, when
\code{make\-/rat} is called.  On the other hand, doing this defeats debugging
aids that trace procedure calls or put breakpoints on procedure calls: You may
want to watch \code{make\-/rat} being called, but you certainly don't want to
watch every call to \code{cons}.

We have chosen not to use this style of definition in this book.}

\begin{scheme}
(define (make-rat n d) (cons n d))
(define (numer x) (car x))
(define (denom x) (cdr x))
\end{scheme}

\noindent
Also, in order to display the results of our computations, we can print
rational numbers by printing the numerator, a slash, and the
denominator:\footnote{\code{Display} is the Scheme primitive for printing data.
The Scheme primitive \code{newline} starts a new line for printing.  Neither of
these procedures returns a useful value, so in the uses of \code{print\-/rat}
below, we show only what \code{print\-/rat} prints, not what the interpreter
prints as the value returned by \code{print\-/rat}.}

\begin{scheme}
(define (print-rat x)
  (newline)
  (display (numer x))
  (display "/")
  (display (denom x)))
\end{scheme}

\noindent
Now we can try our rational-number procedures:

\begin{scheme}
(define one-half (make-rat 1 2))
(print-rat one-half)
~\textit{1/2}~
(define one-third (make-rat 1 3))
(print-rat (add-rat one-half one-third))
~\textit{5/6}~
(print-rat (mul-rat one-half one-third))
~\textit{1/6}~
(print-rat (add-rat one-third one-third))
~\textit{6/9}~
\end{scheme}

\noindent
As the final example shows, our rational-number implementation does not reduce
rational numbers to lowest terms.  We can remedy this by changing
\code{make\-/rat}. If we have a \code{gcd} procedure like the one in 
\link{Section 1.2.5} that produces the greatest common divisor of two integers, we can
use \code{gcd} to reduce the numerator and the denominator to lowest terms
before constructing the pair:

\begin{scheme}
(define (make-rat n d)
  (let ((g (gcd n d)))
    (cons (/ n g) (/ d g))))
\end{scheme}

\noindent
Now we have

\begin{scheme}
(print-rat (add-rat one-third one-third))
~\textit{2/3}~
\end{scheme}

\noindent
as desired.  This modification was accomplished by changing the constructor
\code{make\-/rat} without changing any of the procedures (such as \code{add\-/rat}
and \code{mul\-/rat}) that implement the actual operations.

\begin{quote}
\heading{\phantomsection\label{Exercise 2.1}Exercise 2.1:} Define a better version of
\code{make\-/rat} that handles both positive and negative arguments.
\code{Make\-/rat} should normalize the sign so that if the rational number is
positive, both the numerator and denominator are positive, and if the rational
number is negative, only the numerator is negative.
\end{quote}

\subsection{Abstraction Barriers}
\label{Section 2.1.2}

Before continuing with more examples of compound data and data abstraction, let
us consider some of the issues raised by the rational-number example.  We
defined the rational-number operations in terms of a constructor
\code{make\-/rat} and selectors \code{numer} and \code{denom}.  In general, the
underlying idea of data abstraction is to identify for each type of data object
a basic set of operations in terms of which all manipulations of data objects
of that type will be expressed, and then to use only those operations in
manipulating the data.

\begin{figure}[tb]
\phantomsection\label{Figure 2.1}
\centering
\begin{comment}
\begin{quote}
\heading{Figure 2.1:} Data-abstraction barriers in the rational-number package.

\begin{example}
        +------------------------------------+
--------| Programs that use rational numbers |--------
        +------------------------------------+
          Rational numbers in problem domain
            +---------------------------+
------------|   add-rat  sub-rat  ...   |-------------
            +---------------------------+
   Rational numbers as numerators and denominators
              +------------------------+
--------------| make-rat  numer  denom |--------------
              +------------------------+
              Rational numbers as pairs
                  +----------------+
------------------| cons  car  cdr |------------------
                  +----------------+
            However pairs are implemented
\end{example}
\end{quote}
\end{comment}
\includegraphics[width=91mm]{fig/chap2/Fig2.1c.pdf}
\begin{quote}
\heading{Figure 2.1:} Data-abstraction barriers in the rational-number package.
\end{quote}
\end{figure}

We can envision the structure of the rational-number system as shown in 
\link{Figure 2.1}.  The horizontal lines represent \newterm{abstraction
barriers} that isolate different ``levels'' of the system.  At each level, the
barrier separates the programs (above) that use the data abstraction from the
programs (below) that implement the data abstraction.  Programs that use
rational numbers manipulate them solely in terms of the procedures supplied
``for public use'' by the rational-number package: \code{add\-/rat},
\code{sub\-/rat}, \code{mul\-/rat}, \code{div\-/rat}, and \code{equal\-/rat?}. These,
in turn, are implemented solely in terms of the constructor and selectors
\code{make\-/rat}, \code{numer}, and \code{denom}, which themselves are
implemented in terms of pairs.  The details of how pairs are implemented are
irrelevant to the rest of the rational-number package so long as pairs can be
manipulated by the use of \code{cons}, \code{car}, and \code{cdr}.  In effect,
procedures at each level are the interfaces that define the abstraction
barriers and connect the different levels.

This simple idea has many advantages.  One advantage is that it makes programs
much easier to maintain and to modify.  Any complex data structure can be
represented in a variety of ways with the primitive data structures provided by
a programming language.  Of course, the choice of representation influences the
programs that operate on it; thus, if the representation were to be changed at
some later time, all such programs might have to be modified accordingly.  This
task could be time-consuming and expensive in the case of large programs unless
the dependence on the representation were to be confined by design to a very
few program modules.

For example, an alternate way to address the problem of reducing rational
numbers to lowest terms is to perform the reduction whenever we access the
parts of a rational number, rather than when we construct it.  This leads to
different constructor and selector procedures:

\begin{scheme}
(define (make-rat n d) (cons n d))
(define (numer x)
  (let ((g (gcd (car x) (cdr x))))
    (/ (car x) g)))
(define (denom x)
  (let ((g (gcd (car x) (cdr x))))
    (/ (cdr x) g)))
\end{scheme}

\noindent
The difference between this implementation and the previous one lies in when we
compute the \code{gcd}.  If in our typical use of rational numbers we access
the numerators and denominators of the same rational numbers many times, it
would be preferable to compute the \code{gcd} when the rational numbers are
constructed.  If not, we may be better off waiting until access time to compute
the \code{gcd}.  In any case, when we change from one representation to the
other, the procedures \code{add\-/rat}, \code{sub\-/rat}, and so on do not have to
be modified at all.

Constraining the dependence on the representation to a few interface procedures
helps us design programs as well as modify them, because it allows us to
maintain the flexibility to consider alternate implementations.  To continue
with our simple example, suppose we are designing a rational-number package and
we can't decide initially whether to perform the \code{gcd} at construction
time or at selection time.  The data-abstraction methodology gives us a way to
defer that decision without losing the ability to make progress on the rest of
the system.

\begin{quote}
\heading{\phantomsection\label{Exercise 2.2}Exercise 2.2:} Consider the problem of
representing line segments in a plane.  Each segment is represented as a pair
of points: a starting point and an ending point.  Define a constructor
\code{make\-/segment} and selectors \code{start\-/segment} and \code{end\-/segment}
that define the representation of segments in terms of points.  Furthermore, a
point can be represented as a pair of numbers: the \( x \) coordinate and the
\( y \) coordinate.  Accordingly, specify a constructor \code{make\-/point} and
selectors \code{x\-/point} and \code{y\-/point} that define this representation.
Finally, using your selectors and constructors, define a procedure
\code{midpoint\-/segment} that takes a line segment as argument and returns its
midpoint (the point whose coordinates are the average of the coordinates of the
endpoints).  To try your procedures, you'll need a way to print points:

\begin{scheme}
(define (print-point p)
  (newline)
  (display "(")
  (display (x-point p))
  (display ",")
  (display (y-point p))
  (display ")"))
\end{scheme}
\end{quote}

\begin{quote}
\heading{\phantomsection\label{Exercise 2.3}Exercise 2.3:} Implement a representation for
rectangles in a plane.  (Hint: You may want to make use of \link{Exercise 2.2}.)
In terms of your constructors and selectors, create procedures that compute the
perimeter and the area of a given rectangle.  Now implement a different
representation for rectangles.  Can you design your system with suitable
abstraction barriers, so that the same perimeter and area procedures will work
using either representation?
\end{quote}

\subsection{What Is Meant by Data?}
\label{Section 2.1.3}

We began the rational-number implementation in \link{Section 2.1.1} by
implementing the rational-number operations \code{add\-/rat}, \code{sub\-/rat}, and
so on in terms of three unspecified procedures: \code{make\-/rat}, \code{numer},
and \code{denom}.  At that point, we could think of the operations as being
defined in terms of data objects---numerators, denominators, and rational
numbers---whose behavior was specified by the latter three procedures.

But exactly what is meant by \newterm{data}?  It is not enough to say
``whatever is implemented by the given selectors and constructors.''  Clearly,
not every arbitrary set of three procedures can serve as an appropriate basis
for the rational-number implementation.  We need to guarantee that, if we
construct a rational number \code{x} from a pair of integers \code{n} and
\code{d}, then extracting the \code{numer} and the \code{denom} of \code{x} and
dividing them should yield the same result as dividing \code{n} by \code{d}.
In other words, \code{make\-/rat}, \code{numer}, and \code{denom} must satisfy
the condition that, for any integer \code{n} and any non-zero integer \code{d},
if \code{x} is \code{(make\-/rat n d)}, then
\begin{comment}

\begin{example}
(numer x)    n
--------- = ---
(denom x)    d
\end{example}

\end{comment}
\begin{displaymath}
 {\hbox{\tt(numer x)} \over \hbox{\tt(denom x)}} = {{\tt n} \over {\tt d}}\,.  
\end{displaymath}
In fact, this is the only condition \code{make\-/rat}, \code{numer}, and
\code{denom} must fulfill in order to form a suitable basis for a
rational-number representation.  In general, we can think of data as defined by
some collection of selectors and constructors, together with specified
conditions that these procedures must fulfill in order to be a valid
representation.\footnote{Surprisingly, this idea is very difficult to formulate
rigorously. There are two approaches to giving such a formulation.  One,
pioneered by C. A. R. \link{Hoare (1972)}, is known as the method of \newterm{abstract
models}.  It formalizes the ``procedures plus conditions'' specification as
outlined in the rational-number example above.  Note that the condition on the
rational-number representation was stated in terms of facts about integers
(equality and division).  In general, abstract models define new kinds of data
objects in terms of previously defined types of data objects.  Assertions about
data objects can therefore be checked by reducing them to assertions about
previously defined data objects.  Another approach, introduced by Zilles at
\acronym{MIT}, by Goguen, Thatcher, Wagner, and Wright at IBM 
(see \link{Thatcher et al. 1978}), 
and by Guttag at Toronto (see \link{Guttag 1977}), is called
\newterm{algebraic specification}.  It regards the ``procedures'' as elements
of an abstract algebraic system whose behavior is specified by axioms that
correspond to our ``conditions,'' and uses the techniques of abstract algebra
to check assertions about data objects.  Both methods are surveyed in the paper
by \link{Liskov and Zilles (1975)}.}

This point of view can serve to define not only ``high-level'' data objects,
such as rational numbers, but lower-level objects as well.  Consider the notion
of a pair, which we used in order to define our rational numbers.  We never
actually said what a pair was, only that the language supplied procedures
\code{cons}, \code{car}, and \code{cdr} for operating on pairs.  But the only
thing we need to know about these three operations is that if we glue two
objects together using \code{cons} we can retrieve the objects using \code{car}
and \code{cdr}.  That is, the operations satisfy the condition that, for any
objects \code{x} and \code{y}, if \code{z} is \code{(cons x y)} then \code{(car
z)} is \code{x} and \code{(cdr z)} is \code{y}.  Indeed, we mentioned that
these three procedures are included as primitives in our language.  However,
any triple of procedures that satisfies the above condition can be used as the
basis for implementing pairs.  This point is illustrated strikingly by the fact
that we could implement \code{cons}, \code{car}, and \code{cdr} without using
any data structures at all but only using procedures.  Here are the
definitions:

\begin{scheme}
(define (cons x y)
  (define (dispatch m)
    (cond ((= m 0) x)
          ((= m 1) y)
          (else (error "Argument not 0 or 1: CONS" m))))
  dispatch)
(define (car z) (z 0))
(define (cdr z) (z 1))
\end{scheme}

\noindent
This use of procedures corresponds to nothing like our intuitive notion of what
data should be.  Nevertheless, all we need to do to show that this is a valid
way to represent pairs is to verify that these procedures satisfy the condition
given above.

The subtle point to notice is that the value returned by \code{(cons x y)} is a
procedure---namely the internally defined procedure \code{dispatch}, which
takes one argument and returns either \code{x} or \code{y} depending on whether
the argument is 0 or 1.  Correspondingly, \code{(car z)} is defined to apply
\code{z} to 0.  Hence, if \code{z} is the procedure formed by \code{(cons x
y)}, then \code{z} applied to 0 will yield \code{x}. Thus, we have shown that
\code{(car (cons x y))} yields \code{x}, as desired.  Similarly, \code{(cdr
(cons x y))} applies the procedure returned by \code{(cons x y)} to 1, which
returns \code{y}.  Therefore, this procedural implementation of pairs is a
valid implementation, and if we access pairs using only \code{cons},
\code{car}, and \code{cdr} we cannot distinguish this implementation from one
that uses ``real'' data structures.

The point of exhibiting the procedural representation of pairs is not that our
language works this way (Scheme, and Lisp systems in general, implement pairs
directly, for efficiency reasons) but that it could work this way.  The
procedural representation, although obscure, is a perfectly adequate way to
represent pairs, since it fulfills the only conditions that pairs need to
fulfill.  This example also demonstrates that the ability to manipulate
procedures as objects automatically provides the ability to represent compound
data.  This may seem a curiosity now, but procedural representations of data
will play a central role in our programming repertoire.  This style of
programming is often called \newterm{message passing}, and we will be using it
as a basic tool in \link{Chapter 3} when we address the issues of modeling and
simulation.

\begin{quote}
\heading{\phantomsection\label{Exercise 2.4}Exercise 2.4:} Here is an alternative procedural
representation of pairs.  For this representation, verify that \code{(car (cons
x y))} yields \code{x} for any objects \code{x} and \code{y}.

\begin{scheme}
(define (cons x y)
  (lambda (m) (m x y)))
(define (car z)
  (z (lambda (p q) p)))
\end{scheme}

What is the corresponding definition of \code{cdr}? (Hint: To verify that this
works, make use of the substitution model of \link{Section 1.1.5}.)
\end{quote}

\begin{quote}
\heading{\phantomsection\label{Exercise 2.5}Exercise 2.5:} Show that we can represent pairs of
nonnegative integers using only numbers and arithmetic operations if we
represent the pair \( a \) and \( b \) as the integer that is the product \( 2^a 3^b \).  
Give the corresponding definitions of the procedures \code{cons},
\code{car}, and \code{cdr}.
\end{quote}

\begin{quote}
\heading{\phantomsection\label{Exercise 2.6}Exercise 2.6:} In case representing pairs as
procedures wasn't mind-boggling enough, consider that, in a language that can
manipulate procedures, we can get by without numbers (at least insofar as
nonnegative integers are concerned) by implementing 0 and the operation of
adding 1 as

\begin{scheme}
(define zero (lambda (f) (lambda (x) x)))
(define (add-1 n)
  (lambda (f) (lambda (x) (f ((n f) x)))))
\end{scheme}

This representation is known as \newterm{Church numerals}, after its inventor,
Alonzo Church, the logician who invented the λ-calculus.

Define \code{one} and \code{two} directly (not in terms of \code{zero} and
\code{add\-/1}).  (Hint: Use substitution to evaluate \code{(add\-/1 zero)}).  Give
a direct definition of the addition procedure \code{+} (not in terms of
repeated application of \code{add\-/1}).
\end{quote}

\subsection{Extended Exercise: Interval Arithmetic}
\label{Section 2.1.4}

Alyssa P. Hacker is designing a system to help people solve engineering
problems.  One feature she wants to provide in her system is the ability to
manipulate inexact quantities (such as measured parameters of physical devices)
with known precision, so that when computations are done with such approximate
quantities the results will be numbers of known precision.

Electrical engineers will be using Alyssa's system to compute electrical
quantities.  It is sometimes necessary for them to compute the value of a
parallel equivalent resistance \( R_p \) of two resistors \( R_1 \), \( R_2 \)
using the formula
\begin{comment}

\begin{example}
            1
R_p = -------------
      1/R_1 + 1/R_2
\end{example}

\end{comment}
\begin{displaymath}
 R_p = {1 \over 1 / R_1 + 1 / R_2}.  
\end{displaymath}
Resistance values are usually known only up to some tolerance guaranteed by the
manufacturer of the resistor.  For example, if you buy a resistor labeled ``6.8
ohms with 10\% tolerance'' you can only be sure that the resistor has a
resistance between \( 6.8 - 0.68 = 6.12 \) and \( 6.8 + 0.68 = 7.48 \) ohms.  Thus, if you
have a 6.8-ohm 10\% resistor in parallel with a 4.7-ohm 5\% resistor, the
resistance of the combination can range from about 2.58 ohms (if the two
resistors are at the lower bounds) to about 2.97 ohms (if the two resistors are
at the upper bounds).

Alyssa's idea is to implement ``interval arithmetic'' as a set of arithmetic
operations for combining ``intervals'' (objects that represent the range of
possible values of an inexact quantity).  The result of adding, subtracting,
multiplying, or dividing two intervals is itself an interval, representing the
range of the result.

Alyssa postulates the existence of an abstract object called an ``interval''
that has two endpoints: a lower bound and an upper bound.  She also presumes
that, given the endpoints of an interval, she can construct the interval using
the data constructor \code{make\-/interval}.  Alyssa first writes a procedure for
adding two intervals.  She reasons that the minimum value the sum could be is
the sum of the two lower bounds and the maximum value it could be is the sum of
the two upper bounds:

\begin{scheme}
(define (add-interval x y)
  (make-interval (+ (lower-bound x) (lower-bound y))
                 (+ (upper-bound x) (upper-bound y))))
\end{scheme}

\noindent
Alyssa also works out the product of two intervals by finding the minimum and
the maximum of the products of the bounds and using them as the bounds of the
resulting interval.  (\code{Min} and \code{max} are primitives that find the
minimum or maximum of any number of arguments.)

\begin{scheme}
(define (mul-interval x y)
  (let ((p1 (* (lower-bound x) (lower-bound y)))
        (p2 (* (lower-bound x) (upper-bound y)))
        (p3 (* (upper-bound x) (lower-bound y)))
        (p4 (* (upper-bound x) (upper-bound y))))
    (make-interval (min p1 p2 p3 p4)
                   (max p1 p2 p3 p4))))
\end{scheme}

\noindent
To divide two intervals, Alyssa multiplies the first by the reciprocal of the
second.  Note that the bounds of the reciprocal interval are the reciprocal of
the upper bound and the reciprocal of the lower bound, in that order.

\begin{scheme}
(define (div-interval x y)
  (mul-interval 
   x
   (make-interval (/ 1.0 (upper-bound y))
                  (/ 1.0 (lower-bound y)))))
\end{scheme}

\begin{quote}
\heading{\phantomsection\label{Exercise 2.7}Exercise 2.7:} Alyssa's program is incomplete
because she has not specified the implementation of the interval abstraction.
Here is a definition of the interval constructor:

\begin{scheme}
(define (make-interval a b) (cons a b))
\end{scheme}

Define selectors \code{upper\-/bound} and \code{lower\-/bound} to complete the
implementation.
\end{quote}

\begin{quote}
\heading{\phantomsection\label{Exercise 2.8}Exercise 2.8:} Using reasoning analogous to
Alyssa's, describe how the difference of two intervals may be computed.  Define
a corresponding subtraction procedure, called \code{sub\-/interval}.
\end{quote}

\begin{quote}
\heading{\phantomsection\label{Exercise 2.9}Exercise 2.9:} The \newterm{width} of an interval
is half of the difference between its upper and lower bounds.  The width is a
measure of the uncertainty of the number specified by the interval.  For some
arithmetic operations the width of the result of combining two intervals is a
function only of the widths of the argument intervals, whereas for others the
width of the combination is not a function of the widths of the argument
intervals.  Show that the width of the sum (or difference) of two intervals is
a function only of the widths of the intervals being added (or subtracted).
Give examples to show that this is not true for multiplication or division.
\end{quote}

\begin{quote}
\heading{\phantomsection\label{Exercise 2.10}Exercise 2.10:} Ben Bitdiddle, an expert systems
programmer, looks over Alyssa's shoulder and comments that it is not clear what
it means to divide by an interval that spans zero.  Modify Alyssa's code to
check for this condition and to signal an error if it occurs.
\end{quote}

\begin{quote}
\heading{\phantomsection\label{Exercise 2.11}Exercise 2.11:} In passing, Ben also cryptically
comments: ``By testing the signs of the endpoints of the intervals, it is
possible to break \code{mul\-/interval} into nine cases, only one of which
requires more than two multiplications.''  Rewrite this procedure using Ben's
suggestion.

After debugging her program, Alyssa shows it to a potential user, who complains
that her program solves the wrong problem.  He wants a program that can deal
with numbers represented as a center value and an additive tolerance; for
example, he wants to work with intervals such as \( 3.5 \pm 0.15 \) rather than
[3.35, 3.65].  Alyssa returns to her desk and fixes this problem by supplying
an alternate constructor and alternate selectors:

\begin{scheme}
(define (make-center-width c w)
  (make-interval (- c w) (+ c w)))
(define (center i)
  (/ (+ (lower-bound i) (upper-bound i)) 2))
(define (width i)
  (/ (- (upper-bound i) (lower-bound i)) 2))
\end{scheme}

Unfortunately, most of Alyssa's users are engineers.  Real engineering
situations usually involve measurements with only a small uncertainty, measured
as the ratio of the width of the interval to the midpoint of the interval.
Engineers usually specify percentage tolerances on the parameters of devices,
as in the resistor specifications given earlier.
\end{quote}

\begin{quote}
\heading{\phantomsection\label{Exercise 2.12}Exercise 2.12:} Define a constructor
\code{make\-/center\-/percent} that takes a center and a percentage tolerance and
produces the desired interval.  You must also define a selector \code{percent}
that produces the percentage tolerance for a given interval.  The \code{center}
selector is the same as the one shown above.
\end{quote}

\begin{quote}
\heading{\phantomsection\label{Exercise 2.13}Exercise 2.13:} Show that under the assumption of
small percentage tolerances there is a simple formula for the approximate
percentage tolerance of the product of two intervals in terms of the tolerances
of the factors.  You may simplify the problem by assuming that all numbers are
positive.

After considerable work, Alyssa P. Hacker delivers her finished system.
Several years later, after she has forgotten all about it, she gets a frenzied
call from an irate user, Lem E. Tweakit.  It seems that Lem has noticed that
the formula for parallel resistors can be written in two algebraically
equivalent ways:
\begin{comment}

\begin{example}
 R_1 R_2
---------
R_1 + R_2
\end{example}

\end{comment}
\begin{displaymath}
 R_1 R_2 \over R_1 + R_2 
\end{displaymath}
\noindent
and
\begin{comment}

\begin{example}
      1
-------------
1/R_1 + 1/R_2
\end{example}

\end{comment}
\begin{displaymath}
 {1 \over 1 / R_1 + 1 / R_2}. 
\end{displaymath}
He has written the following two programs, each of which computes the
parallel-resistors formula differently:

\begin{scheme}
(define (par1 r1 r2)
  (div-interval (mul-interval r1 r2)
                (add-interval r1 r2)))
\end{scheme}

\begin{scheme}
(define (par2 r1 r2)
  (let ((one (make-interval 1 1)))
    (div-interval 
     one (add-interval (div-interval one r1)
                       (div-interval one r2)))))
\end{scheme}
\vspace{0.3em}

Lem complains that Alyssa's program gives different answers for the two ways of
computing. This is a serious complaint.
\end{quote}

\begin{quote}
\heading{\phantomsection\label{Exercise 2.14}Exercise 2.14:} Demonstrate that Lem is right.
Investigate the behavior of the system on a variety of arithmetic
expressions. Make some intervals \( A \) and \( B \), and use them in computing the
expressions \( A / A \) and \( A / B \).  You will get the most insight by
using intervals whose width is a small percentage of the center value. Examine
the results of the computation in center-percent form (see \link{Exercise 2.12}).
\end{quote}

\begin{quote}
\heading{\phantomsection\label{Exercise 2.15}Exercise 2.15:} Eva Lu Ator, another user, has
also noticed the different intervals computed by different but algebraically
equivalent expressions. She says that a formula to compute with intervals using
Alyssa's system will produce tighter error bounds if it can be written in such
a form that no variable that represents an uncertain number is repeated.  Thus,
she says, \code{par2} is a ``better'' program for parallel resistances than
\code{par1}.  Is she right?  Why?
\end{quote}

\begin{quote}
\heading{\phantomsection\label{Exercise 2.16}Exercise 2.16:} Explain, in general, why
equivalent algebraic expressions may lead to different answers.  Can you devise
an interval-arithmetic package that does not have this shortcoming, or is this
task impossible?  (Warning: This problem is very difficult.)
\end{quote}


\section{Hierarchical Data and the Closure Property}
\label{Section 2.2}

As we have seen, pairs provide a primitive ``glue'' that we can use to
construct compound data objects.  \link{Figure 2.2} shows a standard way to
visualize a pair---in this case, the pair formed by \code{(cons 1 2)}.  In this
representation, which is called \newterm{box-and-pointer notation}, each object
is shown as a \newterm{pointer} to a box.  The box for a primitive object
contains a representation of the object.  For example, the box for a number
contains a numeral.  The box for a pair is actually a double box, the left part
containing (a pointer to) the \code{car} of the pair and the right part
containing the \code{cdr}.

We have already seen that \code{cons} can be used to combine not only numbers
but pairs as well.  (You made use of this fact, or should have, in doing
\link{Exercise 2.2} and \link{Exercise 2.3}.)  As a consequence, pairs provide a
universal building block from which we can construct all sorts of data
structures.  \link{Figure 2.3} shows two ways to use pairs to combine the
numbers 1, 2, 3, and 4.

\begin{figure}[tb]
\phantomsection\label{Figure 2.2}
\centering
\begin{comment}
\heading{Figure 2.2:} Box-and-pointer representation of \code{(cons 1 2)}.

\begin{example}
     +---+---+     +---+
---->| * | *-+---->| 2 |
     +-|-+---+     +---+
       |
       V
     +---+
     | 1 |
     +---+
\end{example}
\end{comment}
\includegraphics[width=34mm]{fig/chap2/Fig2.2c.pdf}
\begin{quote}
\heading{Figure 2.2:} Box-and-pointer representation of \code{(cons 1 2)}.
\end{quote}
\end{figure}

\begin{figure}[tb]
\phantomsection\label{Figure 2.3}
\centering
\begin{comment}
\heading{Figure 2.3:} Two ways to combine 1, 2, 3, and 4 using pairs.

\begin{example}
     +---+---+     +---+---+         +---+---+     +---+
---->| * | *-+---->| * | * |    ---->| * | *-+---->| 4 |
     +-|-+---+     +-|-+-|-+         +-|-+---+     +---+
       |             |   |             |
       V             V   V             V
   +---+---+      +---+ +---+      +---+---+     +---+---+
   | * | * |      | 3 | | 4 |      | * | *-+---->| * | * |
   +-|-+-|-+      +---+ +---+      +-|-+---+     +-|-+-|-+
     |   |                           |             |   |
     V   V                           V             V   V
  +---+ +---+                      +---+        +---+ +---+
  | 1 | | 2 |                      | 1 |        | 2 | | 3 |
  +---+ +---+                      +---+        +---+ +---+

  (cons (cons 1 2)                 (cons (cons 1
        (cons 3 4))                            (cons 2 3))
                                         4)
\end{example}
\end{comment}
\includegraphics[width=96mm]{fig/chap2/Fig2.3c.pdf}
\begin{quote}
\heading{Figure 2.3:} Two ways to combine 1, 2, 3, and 4 using pairs.
\end{quote}
\end{figure}

The ability to create pairs whose elements are pairs is the essence of list
structure's importance as a representational tool.  We refer to this ability as
the \newterm{closure property} of \code{cons}.  In general, an operation for
combining data objects satisfies the closure property if the results of
combining things with that operation can themselves be combined using the same
operation.\footnote{The use of the word ``closure'' here comes from abstract
algebra, where a set of elements is said to be closed under an operation if
applying the operation to elements in the set produces an element that is again
an element of the set.  The Lisp community also (unfortunately) uses the word
``closure'' to describe a totally unrelated concept: A closure is an
implementation technique for representing procedures with free variables.  We
do not use the word ``closure'' in this second sense in this book.  } Closure
is the key to power in any means of combination because it permits us to create
\newterm{hierarchical} structures---structures made up of parts, which
themselves are made up of parts, and so on.

From the outset of \link{Chapter 1}, we've made essential use of closure in
dealing with procedures, because all but the very simplest programs rely on the
fact that the elements of a combination can themselves be combinations.  In
this section, we take up the consequences of closure for compound data.  We
describe some conventional techniques for using pairs to represent sequences
and trees, and we exhibit a graphics language that illustrates closure in a
vivid way.\footnote{The notion that a means of combination should satisfy
closure is a straightforward idea.  Unfortunately, the data combiners provided
in many popular programming languages do not satisfy closure, or make closure
cumbersome to exploit.  In Fortran or Basic, one typically combines data
elements by assembling them into arrays---but one cannot form arrays whose
elements are themselves arrays.  Pascal and C admit structures whose elements
are structures.  However, this requires that the programmer manipulate pointers
explicitly, and adhere to the restriction that each field of a structure can
contain only elements of a prespecified form.  Unlike Lisp with its pairs,
these languages have no built-in general-purpose glue that makes it easy to
manipulate compound data in a uniform way.  This limitation lies behind Alan
Perlis's comment in his foreword to this book: ``In Pascal the plethora of
declarable data structures induces a specialization within functions that
inhibits and penalizes casual cooperation.  It is better to have 100 functions
operate on one data structure than to have 10 functions operate on 10 data
structures.''}



\subsection{Representing Sequences}
\label{Section 2.2.1}

One of the useful structures we can build with pairs is a
\newterm{sequence}---an ordered collection of data objects.  There are, of
course, many ways to represent sequences in terms of pairs.  One particularly
straightforward representation is illustrated in \link{Figure 2.4}, where the
sequence 1, 2, 3, 4 is represented as a chain of pairs.  The \code{car} of each
pair is the corresponding item in the chain, and the \code{cdr} of the pair is
the next pair in the chain.  The \code{cdr} of the final pair signals the end
of the sequence by pointing to a distinguished value that is not a pair,
represented in box-and-pointer diagrams as a diagonal line and in programs as
the value of the variable \code{nil}.  The entire sequence is constructed by
nested \code{cons} operations:

\begin{scheme}
(cons 1
      (cons 2
            (cons 3
                  (cons 4 nil))))
\end{scheme}

\begin{figure}[tb]
\phantomsection\label{Figure 2.4}
\centering
\begin{comment}
\heading{Figure 2.4:} The sequence 1, 2, 3, 4 represented as a chain of pairs.

\begin{example}
     +---+---+     +---+---+     +---+---+     +---+---+
---->| * | *-+---->| * | *-+---->| * | *-+---->| * | / |
     +-|-+---+     +-|-+---+     +-|-+---+     +-|-+---+
       |             |             |             |
       V             V             V             V
     +---+         +---+         +---+         +---+
     | 1 |         | 2 |         | 3 |         | 4 |
     +---+         +---+         +---+         +---+
\end{example}
\end{comment}
\includegraphics[width=76mm]{fig/chap2/Fig2.4c.pdf}
\begin{quote}
\heading{Figure 2.4:} The sequence 1, 2, 3, 4 represented as a chain of pairs.
\end{quote}
\end{figure}

Such a sequence of pairs, formed by nested \code{cons}es, is called a
\newterm{list}, and Scheme provides a primitive called \code{list} to help in
constructing lists.\footnote{In this book, we use \newterm{list} to mean a
chain of pairs terminated by the end-of-list marker.  In contrast, the term
\newterm{list structure} refers to any data structure made out of pairs, not
just to lists.}  The above sequence could be produced by \code{(list 1 2 3 4)}.
In general,

\begin{scheme}
(list ~\( \dark \langle \)~~\( \dark a_1 \)~~\( \dark \rangle \)~ ~\( \dark \langle \)~~\( \dark a_2 \)~~\( \dark \rangle \)~ ~\( \dots \)~ ~\( \dark \langle \)~~\( \dark a_n \)~~\( \dark \rangle \)~)
\end{scheme}

\noindent
is equivalent to

\begin{scheme}
(cons ~\( \dark \langle \)~~\( \dark a_1 \)~~\( \dark \rangle \)~
      (cons ~\( \dark \langle \)~~\( \dark a_2 \)~~\( \dark \rangle \)~
            (cons ~\( \dots \)~
                  (cons ~\( \dark \langle \)~~\( \dark a_n \)~~\( \dark \rangle \)~
                        nil)~\( \dots \)~)))
\end{scheme}

\noindent
Lisp systems conventionally print lists by printing the sequence of elements,
enclosed in parentheses.  Thus, the data object in \link{Figure 2.4} is printed
as \code{(1 2 3 4)}:

\begin{scheme}
(define one-through-four (list 1 2 3 4))
one-through-four
~\textit{(1 2 3 4)}~
\end{scheme}

\noindent
Be careful not to confuse the expression \code{(list 1 2 3 4)} with the list
\mbox{\code{(1 2 3 4)}}, which is the result obtained when the expression is
evaluated.  Attempting to evaluate the expression \code{(1 2 3 4)} will signal
an error when the interpreter tries to apply the procedure \code{1} to
arguments \code{2}, \code{3}, and \code{4}.

We can think of \code{car} as selecting the first item in the list, and of
\code{cdr} as selecting the sublist consisting of all but the first item.
Nested applications of \code{car} and \code{cdr} can be used to extract the
second, third, and subsequent items in the list.\footnote{Since nested
applications of \code{car} and \code{cdr} are cumbersome to write, Lisp
dialects provide abbreviations for them---for instance,

\begin{smallscheme}
(cadr ~\( \dark \langle \)~~\( \dark ar\!g \)~~\( \dark \rangle \)~) = (car (cdr ~\( \dark \langle \)~~\( \dark ar\!g \)~~\( \dark \rangle \)~))
\end{smallscheme}

The names of all such procedures start with \code{c} and end with \code{r}.
Each \code{a} between them stands for a \code{car} operation and each \code{d}
for a \code{cdr} operation, to be applied in the same order in which they
appear in the name.  The names \code{car} and \code{cdr} persist because simple
combinations like \code{cadr} are pronounceable.} The constructor \code{cons}
makes a list like the original one, but with an additional item at the
beginning.

\begin{scheme}
(car one-through-four)
~\textit{1}~
(cdr one-through-four)
~\textit{(2 3 4)}~
(car (cdr one-through-four))
~\textit{2}~
(cons 10 one-through-four)
~\textit{(10 1 2 3 4)}~
(cons 5 one-through-four)
~\textit{(5 1 2 3 4)}~
\end{scheme}

\noindent
The value of \code{nil}, used to terminate the chain of pairs, can be thought
of as a sequence of no elements, the \newterm{empty list}.  The word
\newterm{nil} is a contraction of the Latin word \emph{nihil}, which means
``nothing.''\footnote{It's remarkable how much energy in the standardization of
Lisp dialects has been dissipated in arguments that are literally over nothing:
Should \code{nil} be an ordinary name?  Should the value of \code{nil} be a
symbol?  Should it be a list?  Should it be a pair?  In Scheme, \code{nil} is
an ordinary name, which we use in this section as a variable whose value is the
end-of-list marker (just as \code{true} is an ordinary variable that has a true
value).  Other dialects of Lisp, including Common Lisp, treat \code{nil} as a
special symbol.  The authors of this book, who have endured too many language
standardization brawls, would like to avoid the entire issue.  Once we have
introduced quotation in \link{Section 2.3}, we will denote the empty list as
\code{'()} and dispense with the variable \code{nil} entirely.}

\subsubsection*{List operations}

The use of pairs to represent sequences of elements as lists is accompanied by
conventional programming techniques for manipulating lists by successively
``\code{cdr}ing down'' the lists.  For example, the procedure \code{list\-/ref}
takes as arguments a list and a number \( n \) and returns the \( n^{\mathrm{th}} \) item of
the list.  It is customary to number the elements of the list beginning with 0.
The method for computing \code{list\-/ref} is the following:

\begin{itemize}

\item
For \( n = 0 \), \code{list\-/ref} should return the \code{car} of the list.

\item
Otherwise, \code{list\-/ref} should return  the \( (n - 1) \)-st item of the
\code{cdr} of the list.

\end{itemize}

\begin{scheme}
(define (list-ref items n)
  (if (= n 0)
      (car items)
      (list-ref (cdr items) (- n 1))))
(define squares (list 1 4 9 16 25))
(list-ref squares 3)
~\textit{16}~
\end{scheme}

\noindent
Often we \code{cdr} down the whole list.  To aid in this, Scheme includes a
primitive predicate \code{null?}, which tests whether its argument is the empty
list.  The procedure \code{length}, which returns the number of items in a
list, illustrates this typical pattern of use:

\begin{scheme}
(define (length items)
  (if (null? items)
      0
      (+ 1 (length (cdr items)))))
(define odds (list 1 3 5 7))
(length odds)
~\textit{4}~
\end{scheme}

\noindent
The \code{length} procedure implements a simple recursive plan. The reduction
step is:

\begin{itemize}

\item
The \code{length} of any list is 1 plus the \code{length} of the \code{cdr} of
the list.

\end{itemize}

\noindent
This is applied successively until we reach the base case:

\begin{itemize}

\item
The \code{length} of the empty list is 0.

\end{itemize}

\noindent
We could also compute \code{length} in an iterative style:

\begin{scheme}
(define (length items)
  (define (length-iter a count)
    (if (null? a)
        count
        (length-iter (cdr a) (+ 1 count))))
  (length-iter items 0))
\end{scheme}

\noindent
Another conventional programming technique is to ``\code{cons} up'' an answer
list while \code{cdr}ing down a list, as in the procedure \code{append}, which
takes two lists as arguments and combines their elements to make a new list:

\begin{scheme}
(append squares odds)
~\textit{(1 4 9 16 25 1 3 5 7)}~
(append odds squares)
~\textit{(1 3 5 7 1 4 9 16 25)}~
\end{scheme}

\noindent
\code{Append} is also implemented using a recursive plan.  To \code{append}
lists \code{list1} and \code{list2}, do the following:

\begin{itemize}

\item
If \code{list1} is the empty list, then the result is just \code{list2}.

\item
Otherwise, \code{append} the \code{cdr} of \code{list1} and \code{list2}, and
\code{cons} the \code{car} of \code{list1} onto the result:

\end{itemize}

\begin{scheme}
(define (append list1 list2)
  (if (null? list1)
      list2
      (cons (car list1) (append (cdr list1) list2))))
\end{scheme}

\begin{quote}
\heading{\phantomsection\label{Exercise 2.17}Exercise 2.17:} Define a procedure
\code{last\-/pair} that returns the list that contains only the last element of a
given (nonempty) list:

\begin{scheme}
(last-pair (list 23 72 149 34))
~\textit{(34)}~
\end{scheme}
\end{quote}

\begin{quote}
\heading{\phantomsection\label{Exercise 2.18}Exercise 2.18:} Define a procedure \code{reverse}
that takes a list as argument and returns a list of the same elements in
reverse order:

\begin{scheme}
(reverse (list 1 4 9 16 25))
~\textit{(25 16 9 4 1)}~
\end{scheme}
\end{quote}

\begin{quote}
\heading{\phantomsection\label{Exercise 2.19}Exercise 2.19:} Consider the change-counting
program of \link{Section 1.2.2}.  It would be nice to be able to easily change
the currency used by the program, so that we could compute the number of ways
to change a British pound, for example.  As the program is written, the
knowledge of the currency is distributed partly into the procedure
\code{first\-/denomination} and partly into the procedure \code{count\-/change}
(which knows that there are five kinds of U.S. coins).  It would be nicer to be
able to supply a list of coins to be used for making change.

We want to rewrite the procedure \code{cc} so that its second argument is a
list of the values of the coins to use rather than an integer specifying which
coins to use.  We could then have lists that defined each kind of currency:

\begin{scheme}
(define us-coins (list 50 25 10 5 1))
(define uk-coins (list 100 50 20 10 5 2 1 0.5))
\end{scheme}

We could then call \code{cc} as follows:

\begin{scheme}
(cc 100 us-coins)
~\textit{292}~
\end{scheme}

To do this will require changing the program \code{cc} somewhat.  It will still
have the same form, but it will access its second argument differently, as
follows:

\begin{scheme}
(define (cc amount coin-values)
  (cond ((= amount 0) 1)
        ((or (< amount 0) (no-more? coin-values)) 0)
        (else
         (+ (cc amount
                (except-first-denomination 
                 coin-values))
            (cc (- amount
                   (first-denomination 
                    coin-values))
                coin-values)))))
\end{scheme}

Define the procedures \code{first\-/denomination},
\code{except\-/first\-/de- nomination}, and \code{no\-/more?} in terms of primitive
operations on list structures.  Does the order of the list \code{coin\-/values}
affect the answer produced by \code{cc}?  Why or why not?
\end{quote}

\begin{quote}
\heading{\phantomsection\label{Exercise 2.20}Exercise 2.20:} The procedures \code{+},
\code{*}, and \code{list} take arbitrary numbers of arguments. One way to
define such procedures is to use \code{define} with \newterm{dotted-tail
notation}.  In a procedure definition, a parameter list that has a dot before
the last parameter name indicates that, when the procedure is called, the
initial parameters (if any) will have as values the initial arguments, as
usual, but the final parameter's value will be a \newterm{list} of any
remaining arguments.  For instance, given the definition

\begin{scheme}
(define (f x y . z) ~\( \dark \langle \)~~\var{\dark body}~~\( \dark \rangle \)~)
\end{scheme}

\noindent
the procedure \code{f} can be called with two or more arguments.  If we
evaluate

\begin{scheme}
(f 1 2 3 4 5 6)
\end{scheme}

\noindent
then in the body of \code{f}, \code{x} will be 1, \code{y} will be 2, and
\code{z} will be the list \mbox{\code{(3 4 5 6)}}.  Given the definition

\begin{scheme}
(define (g . w) ~\( \dark \langle \)~~\var{\dark body}~~\( \dark \rangle \)~)
\end{scheme}

\noindent
the procedure \code{g} can be called with zero or more arguments.  If we
evaluate

\begin{scheme}
(g 1 2 3 4 5 6)
\end{scheme}

\noindent
then in the body of \code{g}, \code{w} will be the list \code{(1 2 3 4 5
6)}.\footnote{To define \code{f} and \code{g} using \code{lambda} we would
write

\begin{smallscheme}
(define f (lambda (x y . z) ~\( \dark \langle \)~~\var{\dark body}~~\( \dark \rangle \)~))
(define g (lambda w ~\( \dark \langle \)~~\var{\dark body}~~\( \dark \rangle \)~))
\end{smallscheme}
}

Use this notation to write a procedure \code{same\-/parity} that takes one or
more integers and returns a list of all the arguments that have the same
even-odd parity as the first argument.  For example,

\begin{scheme}
(same-parity 1 2 3 4 5 6 7)
~\textit{(1 3 5 7)}~
(same-parity 2 3 4 5 6 7)
~\textit{(2 4 6)}~
\end{scheme}
\end{quote}

\subsubsection*{Mapping over lists}

One extremely useful operation is to apply some transformation to each element
in a list and generate the list of results.  For instance, the following
procedure scales each number in a list by a given factor:

\begin{scheme}
(define (scale-list items factor)
  (if (null? items)
      nil
      (cons (* (car items) factor)
            (scale-list (cdr items) 
                        factor))))
(scale-list (list 1 2 3 4 5) 10)
~\textit{(10 20 30 40 50)}~
\end{scheme}

\noindent
We can abstract this general idea and capture it as a common pattern expressed
as a higher-order procedure, just as in \link{Section 1.3}.  The higher-order
procedure here is called \code{map}.  \code{Map} takes as arguments a procedure
of one argument and a list, and returns a list of the results produced by
applying the procedure to each element in the list:\footnote{\label{Footnote 12}
Scheme standardly provides a \code{map} procedure that
is more general than the one described here.  This more general \code{map}
takes a procedure of \( n \) arguments, together with \( n \) lists, and applies
the procedure to all the first elements of the lists, all the second elements
of the lists, and so on, returning a list of the results.  For example:

\begin{smallscheme}
(map + (list 1 2 3) (list 40 50 60) (list 700 800 900))
~\textit{(741 852 963)}~
(map (lambda (x y) (+ x (* 2 y)))
     (list 1 2 3)
     (list 4 5 6))
~\textit{(9 12 15)}~
\end{smallscheme}
}

\begin{scheme}
(define (map proc items)
  (if (null? items)
      nil
      (cons (proc (car items))
            (map proc (cdr items)))))
(map abs (list -10 2.5 -11.6 17))
~\textit{(10 2.5 11.6 17)}~
(map (lambda (x) (* x x)) (list 1 2 3 4))
~\textit{(1 4 9 16)}~
\end{scheme}

\noindent
Now we can give a new definition of \code{scale\-/list} in terms of \code{map}:

\begin{scheme}
(define (scale-list items factor)
  (map (lambda (x) (* x factor))
       items))
\end{scheme}

\noindent
\code{Map} is an important construct, not only because it captures a common
pattern, but because it establishes a higher level of abstraction in dealing
with lists.  In the original definition of \code{scale\-/list}, the recursive
structure of the program draws attention to the element-by-element processing
of the list.  Defining \code{scale\-/list} in terms of \code{map} suppresses that
level of detail and emphasizes that scaling transforms a list of elements to a
list of results.  The difference between the two definitions is not that the
computer is performing a different process (it isn't) but that we think about
the process differently.  In effect, \code{map} helps establish an abstraction
barrier that isolates the implementation of procedures that transform lists
from the details of how the elements of the list are extracted and combined.
Like the barriers shown in \link{Figure 2.1}, this abstraction gives us the
flexibility to change the low-level details of how sequences are implemented,
while preserving the conceptual framework of operations that transform
sequences to sequences.  \link{Section 2.2.3} expands on this use of sequences
as a framework for organizing programs.

\begin{quote}
\heading{\phantomsection\label{Exercise 2.21}Exercise 2.21:} The procedure \code{square\-/list}
takes a list of numbers as argument and returns a list of the squares of those
numbers.

\begin{scheme}
(square-list (list 1 2 3 4))
~\textit{(1 4 9 16)}~
\end{scheme}

Here are two different definitions of \code{square\-/list}.  Complete both of
them by filling in the missing expressions:

\begin{scheme}
(define (square-list items)
  (if (null? items)
      nil
      (cons ~\( \dark \langle \)~??~\( \dark \rangle \)~ ~\( \dark \langle \)~??~\( \dark \rangle \)~)))
(define (square-list items)
  (map ~\( \dark \langle \)~??~\( \dark \rangle \)~ ~\( \dark \langle \)~??~\( \dark \rangle \)~))
\end{scheme}
\end{quote}

\begin{quote}
\heading{\phantomsection\label{Exercise 2.22}Exercise 2.22:} Louis Reasoner tries to rewrite
the first \code{square\-/list} procedure of \link{Exercise 2.21} so that it
evolves an iterative process:

\begin{scheme}
(define (square-list items)
  (define (iter things answer)
    (if (null? things)
        answer
        (iter (cdr things)
              (cons (square (car things))
                    answer))))
  (iter items nil))
\end{scheme}

Unfortunately, defining \code{square\-/list} this way produces the answer list in
the reverse order of the one desired.  Why?

Louis then tries to fix his bug by interchanging the arguments to \code{cons}:

\begin{scheme}
(define (square-list items)
  (define (iter things answer)
    (if (null? things)
        answer
        (iter (cdr things)
              (cons answer
                    (square (car things))))))
  (iter items nil))
\end{scheme}

This doesn't work either.  Explain.
\end{quote}

\begin{quote}
\heading{\phantomsection\label{Exercise 2.23}Exercise 2.23:} The procedure \code{for\-/each} is
similar to \code{map}.  It takes as arguments a procedure and a list of
elements.  However, rather than forming a list of the results, \code{for\-/each}
just applies the procedure to each of the elements in turn, from left to right.
The values returned by applying the procedure to the elements are not used at
all---\code{for\-/each} is used with procedures that perform an action, such as
printing.  For example,

\begin{scheme}
(for-each (lambda (x)
            (newline)
            (display x))
          (list 57 321 88))
~\textit{57}~
~\textit{321}~
~\textit{88}~
\end{scheme}

The value returned by the call to \code{for\-/each} (not illustrated above) can
be something arbitrary, such as true.  Give an implementation of
\code{for\-/each}.
\end{quote}

\subsection{Hierarchical Structures}
\label{Section 2.2.2}

The representation of sequences in terms of lists generalizes naturally to
represent sequences whose elements may themselves be sequences.  For example,
we can regard the object \code{((1 2) 3 4)} constructed by

\begin{scheme}
(cons (list 1 2) (list 3 4))
\end{scheme}

\noindent
as a list of three items, the first of which is itself a list, \code{(1 2)}.
Indeed, this is suggested by the form in which the result is printed by the
interpreter.  \link{Figure 2.5} shows the representation of
this structure in terms of pairs.

\begin{figure}[tb]
\phantomsection\label{Figure 2.5}
\centering
\begin{comment}
\heading{Figure 2.5:} Structure formed by \code{(cons (list 1 2) (list 3 4))}.

\begin{example}
                                          (3 4)
                                            |
                                            V
((1 2) 3 4)  +---+---+                  +---+---+     +---+---+
        ---->| * | *-+----------------->| * | *-+---->| * | / |
             +-|-+---+                  +-|-+---+     +-|-+---+
               |                          |             |
               V                          V             V
      (1 2)  +---+---+     +---+---+    +---+         +---+
        ---->| * | *-+---->| * | / |    | 3 |         | 4 |
             +-|-+---+     +-|-+---+    +---+         +---+
               |             |
               V             V
             +---+         +---+
             | 1 |         | 2 |
             +---+         +---+
\end{example}
\end{comment}
\includegraphics[width=91mm]{fig/chap2/Fig2.5c.pdf}
\begin{quote}
\heading{Figure 2.5:} Structure formed by \code{(cons (list 1 2) (list 3 4))}.
\end{quote}
\end{figure}

Another way to think of sequences whose elements are sequences is as
\newterm{trees}.  The elements of the sequence are the branches of the tree,
and \mbox{elements} that are themselves sequences are subtrees.  \link{Figure 2.6}
shows \mbox{the structure} in \link{Figure 2.5} viewed as a tree.

\begin{figure}[tb]
\phantomsection\label{Figure 2.6}
\centering
\begin{comment}
\heading{Figure 2.6:} The list structure in \link{Figure 2.5} viewed as a tree.

\begin{example}
 ((1 2) 3 4)
     /\\
    /  | \
(1 2)  3 4
 / \
 1 2
\end{example}
\end{comment}
\includegraphics[width=22mm]{fig/chap2/Fig2.6a.pdf}
\begin{quote}
\heading{Figure 2.6:} The list structure in \link{Figure 2.5} viewed as a tree.
\end{quote}
\end{figure}

Recursion is a natural tool for dealing with tree structures, since we can
often reduce operations on trees to operations on their branches, which reduce
in turn to operations on the branches of the branches, and so on, until we
reach the leaves of the tree.  As an example, compare the \code{length}
procedure of \link{Section 2.2.1} with the \code{count\-/leaves} procedure, which
returns the total number of leaves of a tree:

\begin{scheme}
(define x (cons (list 1 2) (list 3 4)))
(length x)
~\textit{3}~
(count-leaves x)
~\textit{4}~
(list x x)
~\textit{(((1 2) 3 4) ((1 2) 3 4))}~
(length (list x x))
~\textit{2}~
(count-leaves (list x x))
~\textit{8}~
\end{scheme}

\noindent
To implement \code{count\-/leaves}, recall the recursive plan for computing
\code{length}:

\begin{itemize}

\item
\code{Length} of a list \code{x} is 1 plus \code{length} of the
\code{cdr} of \code{x}.

\item
\code{Length} of the empty list is 0.

\end{itemize}

\noindent
\code{Count\-/leaves} is similar.  The value for the empty list is the same:

\begin{itemize}

\item
\code{Count\-/leaves} of the empty list is 0.

\end{itemize}

\noindent
But in the reduction step, where we strip off the \code{car} of the list, we
must take into account that the \code{car} may itself be a tree whose leaves we
need to count.  Thus, the appropriate reduction step is

\begin{itemize}

\item
\code{Count\-/leaves} of a tree \code{x} is \code{count\-/leaves} of the \code{car}
of \code{x} plus \code{count\-/leaves} of the \code{cdr} of \code{x}.

\end{itemize}

\noindent
Finally, by taking \code{car}s we reach actual leaves, so we need another base
case:

\begin{itemize}

\item
\code{Count\-/leaves} of a leaf is 1.

\end{itemize}

\noindent
To aid in writing recursive procedures on trees, Scheme provides the primitive
predicate \code{pair?}, which tests whether its argument is a pair.  Here is
the complete procedure:\footnote{The order of the first two clauses in the
\code{cond} matters, since the empty list satisfies \code{null?} and also is
not a pair.}

\begin{scheme}
(define (count-leaves x)
  (cond ((null? x) 0)
        ((not (pair? x)) 1)
        (else (+ (count-leaves (car x))
                 (count-leaves (cdr x))))))
\end{scheme}

\begin{quote}
\heading{\phantomsection\label{Exercise 2.24}Exercise 2.24:} Suppose we evaluate the
expression \code{(list 1 (list 2 (list 3 4)))}.  Give the result printed by the
interpreter, the corresponding box-and-pointer structure, and the
interpretation of this as a tree (as in \link{Figure 2.6}).
\end{quote}

\begin{quote}
\heading{\phantomsection\label{Exercise 2.25}Exercise 2.25:} Give combinations of \code{car}s
and \code{cdr}s that will pick 7 from each of the following lists:

\begin{scheme}
(1 3 (5 7) 9)
((7))
(1 (2 (3 (4 (5 (6 7))))))
\end{scheme}
\end{quote}

\begin{quote}
\heading{\phantomsection\label{Exercise 2.26}Exercise 2.26:} Suppose we define \code{x} and
\code{y} to be two lists:

\begin{scheme}
(define x (list 1 2 3))
(define y (list 4 5 6))
\end{scheme}

What result is printed by the interpreter in response to evaluating each of the
following expressions:

\begin{scheme}
(append x y)
(cons x y)
(list x y)
\end{scheme}
\end{quote}

\begin{quote}
\heading{\phantomsection\label{Exercise 2.27}Exercise 2.27:} Modify your \code{reverse}
procedure of \link{Exercise 2.18} to produce a \code{deep\-/reverse} procedure
that takes a list as argument and returns as its value the list with its
elements reversed and with all sublists deep-reversed as well.  For example,

\begin{scheme}
(define x (list (list 1 2) (list 3 4)))
x
~\textit{((1 2) (3 4))}~
(reverse x)
~\textit{((3 4) (1 2))}~
(deep-reverse x)
~\textit{((4 3) (2 1))}~
\end{scheme}
\end{quote}

\begin{quote}
\heading{\phantomsection\label{Exercise 2.28}Exercise 2.28:} Write a procedure \code{fringe}
that takes as argument a tree (represented as a list) and returns a list whose
elements are all the leaves of the tree arranged in left-to-right order.  For
example,

\begin{scheme}
(define x (list (list 1 2) (list 3 4)))
(fringe x)
~\textit{(1 2 3 4)}~
(fringe (list x x))
~\textit{(1 2 3 4 1 2 3 4)}~
\end{scheme}
\end{quote}

\begin{quote}
\heading{\phantomsection\label{Exercise 2.29}Exercise 2.29:} A binary mobile consists of two
branches, a left branch and a right branch.  Each branch is a rod of a certain
length, from which hangs either a weight or another binary mobile.  We can
represent a binary mobile using compound data by constructing it from two
branches (for example, using \code{list}):

\begin{scheme}
(define (make-mobile left right)
  (list left right))
\end{scheme}

A branch is constructed from a \code{length} (which must be a number) together
with a \code{structure}, which may be either a number (representing a simple
weight) or another mobile:

\begin{scheme}
(define (make-branch length structure)
  (list length structure))
\end{scheme}

\begin{enumerate}[a.]

\item
Write the corresponding selectors \code{left\-/branch} and \code{right\-/branch},
which return the branches of a mobile, and \code{branch\-/length} and
\code{branch\-/structure}, which return the components of a branch.

\item
Using your selectors, define a procedure \code{total\-/weight} that returns the
total weight of a mobile.

\item
A mobile is said to be \newterm{balanced} if the torque applied by its top-left
branch is equal to that applied by its top-right branch (that is, if the length
of the left rod multiplied by the weight hanging from that rod is equal to the
corresponding product for the right side) and if each of the submobiles hanging
off its branches is balanced. Design a predicate that tests whether a binary
mobile is balanced.

\item
Suppose we change the representation of mobiles so that the constructors are

\begin{scheme}
(define (make-mobile left right) (cons left right))
(define (make-branch length structure)
  (cons length structure))
\end{scheme}

How much do you need to change your programs to convert to the new
representation?

\end{enumerate}
\end{quote}

\subsubsection*{Mapping over trees}

Just as \code{map} is a powerful abstraction for dealing with sequences,
\code{map} together with recursion is a powerful abstraction for dealing with
trees.  For instance, the \code{scale\-/tree} procedure, analogous to
\code{scale\-/list} of \link{Section 2.2.1}, takes as arguments a numeric factor
and a tree whose leaves are numbers.  It returns a tree of the same shape,
where each number is multiplied by the factor.  The recursive plan for
\code{scale\-/tree} is similar to the one for \code{count\-/leaves}:

\begin{scheme}
(define (scale-tree tree factor)
  (cond ((null? tree) nil)
        ((not (pair? tree)) (* tree factor))
        (else (cons (scale-tree (car tree) factor)
                    (scale-tree (cdr tree) factor)))))
(scale-tree (list 1 (list 2 (list 3 4) 5) (list 6 7)) 10)
~\textit{(10 (20 (30 40) 50) (60 70))}~
\end{scheme}

\noindent
Another way to implement \code{scale\-/tree} is to regard the tree as a sequence
of sub-trees and use \code{map}.  We map over the sequence, scaling each
sub-tree in turn, and return the list of results.  In the base case, where the
tree is a leaf, we simply multiply by the factor:

\begin{scheme}
(define (scale-tree tree factor)
  (map (lambda (sub-tree)
         (if (pair? sub-tree)
             (scale-tree sub-tree factor)
             (* sub-tree factor)))
       tree))
\end{scheme}

\noindent
Many tree operations can be implemented by similar combinations of sequence
operations and recursion.

\begin{quote}
\heading{\phantomsection\label{Exercise 2.30}Exercise 2.30:} Define a procedure
\code{square\-/tree} analogous to the \code{square\-/list} procedure of
\link{Exercise 2.21}.  That is, \code{square\-/tree} should behave as follows:

\begin{scheme}
(square-tree
 (list 1
       (list 2 (list 3 4) 5)
       (list 6 7)))
~\textit{(1 (4 (9 16) 25) (36 49))}~
\end{scheme}

Define \code{square\-/tree} both directly (i.e., without using any higher-order
procedures) and also by using \code{map} and recursion.
\end{quote}

\begin{quote}
\heading{\phantomsection\label{Exercise 2.31}Exercise 2.31:} Abstract your answer to
\link{Exercise 2.30} to produce a procedure \code{tree\-/map} with the property
that \code{square\-/tree} could be defined as

\begin{scheme}
(define (square-tree tree) (tree-map square tree))
\end{scheme}
\end{quote}

\begin{quote}
\heading{\phantomsection\label{Exercise 2.32}Exercise 2.32:} We can represent a set as a list
of distinct elements, and we can represent the set of all subsets of the set as
a list of lists.  For example, if the set is \code{(1 2 3)}, then the set of
all subsets is \code{(() (3) (2) (2 3) (1) (1 3) (1 2) (1 2 3))}.  Complete the
following definition of a procedure that generates the set of subsets of a set
and give a clear explanation of why it works:

\begin{scheme}
(define (subsets s)
  (if (null? s)
      (list nil)
      (let ((rest (subsets (cdr s))))
        (append rest (map ~\( \dark \langle \)~??~\( \dark \rangle \)~ rest)))))
\end{scheme}
\end{quote}

\subsection{Sequences as Conventional Interfaces}
\label{Section 2.2.3}

In working with compound data, we've stressed how data abstraction permits us
to design programs without becoming enmeshed in the details of data
representations, and how abstraction preserves for us the flexibility to
experiment with alternative representations.  In this section, we introduce
another powerful design principle for working with data structures---the use of
\newterm{conventional interfaces}.

In \link{Section 1.3} we saw how program abstractions, implemented as
higher-order procedures, can capture common patterns in programs that deal with
numerical data.  Our ability to formulate analogous operations for working with
compound data depends crucially on the style in which we manipulate our data
structures.  Consider, for example, the following procedure, analogous to the
\code{count\-/leaves} procedure of \link{Section 2.2.2}, which takes a tree as
argument and computes the sum of the squares of the leaves that are odd:

\begin{scheme}
(define (sum-odd-squares tree)
  (cond ((null? tree) 0)
        ((not (pair? tree))
         (if (odd? tree) (square tree) 0))
        (else (+ (sum-odd-squares (car tree))
                 (sum-odd-squares (cdr tree))))))
\end{scheme}

\noindent
On the surface, this procedure is very different from the following one, which
constructs a list of all the even Fibonacci numbers \( {\rm Fib}(k) \), where
\( k \) is less than or equal to a given integer \( n \):

\begin{scheme}
(define (even-fibs n)
  (define (next k)
    (if (> k n)
        nil
        (let ((f (fib k)))
          (if (even? f)
              (cons f (next (+ k 1)))
              (next (+ k 1))))))
  (next 0))
\end{scheme}

\noindent
Despite the fact that these two procedures are structurally very different, a
more abstract description of the two computations reveals a great deal of
similarity.  The first program

\begin{itemize}

\item
enumerates the leaves of a tree;

\item
filters them, selecting the odd ones;

\item
squares each of the selected ones; and

\item
accumulates the results using \code{+}, starting with 0.

\end{itemize}

\noindent
The second program

\begin{itemize}

\item
enumerates the integers from 0 to \( n \);

\item
computes the Fibonacci number for each integer;

\item
filters them, selecting the even ones; and

\item
accumulates the results using \code{cons},  starting with the
empty list.

\end{itemize}

\begin{figure}[tb]
\phantomsection\label{Figure 2.7}
\centering
\begin{comment}
\begin{quote}
\heading{Figure 2.7:} The signal-flow plans for the
procedures \code{sum\-/odd\-/squares} (top) and \code{even\-/fibs} (bottom) reveal
the commonality between the two programs.

\begin{example}
+-------------+   +-------------+   +-------------+   +-------------+
| enumerate:  |-->| filter:     |-->| map:        |-->| accumulate: |
| tree leaves |   | odd?        |   | square      |   | +, 0        |
+-------------+   +-------------+   +-------------+   +-------------+

+-------------+   +-------------+   +-------------+   +-------------+
| enumerate:  |-->| map:        |-->| filter:     |-->| accumulate: |
| integers    |   | fib         |   | even?       |   | cons, ()    |
+-------------+   +-------------+   +-------------+   +-------------+
\end{example}
\end{quote}
\end{comment}
\includegraphics[width=111mm]{fig/chap2/Fig2.7d.pdf} 
\begin{quote}
\heading{Figure 2.7:} The signal-flow plans for the procedures \code{sum\-/odd\-/squares} (top) and \code{even\-/fibs} (bottom) reveal the commonality between the two programs.
\end{quote}
\end{figure}

\noindent
A signal-processing engineer would find it natural to conceptualize these
processes in terms of signals flowing through a cascade of stages, each of
which implements part of the program plan, as shown in \link{Figure 2.7}.  In
\code{sum\-/odd\-/squares}, we begin with an \newterm{enumerator}, which generates
a ``signal'' consisting of the leaves of a given tree.  This signal is passed
through a \newterm{filter}, which eliminates all but the odd elements.  The
resulting signal is in turn passed through a \newterm{map}, which is a
``transducer'' that applies the \code{square} procedure to each element.  The
output of the map is then fed to an \newterm{accumulator}, which combines the
elements using \code{+}, starting from an initial 0.  The plan for
\code{even\-/fibs} is analogous.

Unfortunately, the two procedure definitions above fail to exhibit this
signal-flow structure.  For instance, if we examine the \code{sum\-/odd\-/squares}
procedure, we find that the enumeration is implemented partly by the
\code{null?} and \code{pair?} tests and partly by the tree-recursive structure
of the procedure.  Similarly, the accumulation is found partly in the tests and
partly in the addition used in the recursion.  In general, there are no
distinct parts of either procedure that correspond to the elements in the
signal-flow description.  Our two procedures decompose the computations in a
different way, spreading the enumeration over the program and mingling it with
the map, the filter, and the accumulation.  If we could organize our programs
to make the signal-flow structure manifest in the procedures we write, this
would increase the conceptual clarity of the resulting code.

\subsubsection*{Sequence Operations}

The key to organizing programs so as to more clearly reflect the signal-flow
structure is to concentrate on the ``signals'' that flow from one stage in the
process to the next.  If we represent these signals as lists, then we can use
list operations to implement the processing at each of the stages.  For
instance, we can implement the mapping stages of the signal-flow diagrams using
the \code{map} procedure from \link{Section 2.2.1}:

\begin{scheme}
(map square (list 1 2 3 4 5))
~\textit{(1 4 9 16 25)}~
\end{scheme}

\noindent
Filtering a sequence to select only those elements that satisfy a given
predicate is accomplished by

\begin{scheme}
(define (filter predicate sequence)
  (cond ((null? sequence) nil)
        ((predicate (car sequence))
         (cons (car sequence)
               (filter predicate (cdr sequence))))
        (else (filter predicate (cdr sequence)))))
\end{scheme}

\noindent
For example,

\begin{scheme}
(filter odd? (list 1 2 3 4 5))
~\textit{(1 3 5)}~
\end{scheme}

\noindent
Accumulations can be implemented by

\begin{scheme}
(define (accumulate op initial sequence)
  (if (null? sequence)
      initial
      (op (car sequence)
          (accumulate op initial (cdr sequence)))))
(accumulate + 0 (list 1 2 3 4 5))
~\textit{15}~
(accumulate * 1 (list 1 2 3 4 5))
~\textit{120}~
(accumulate cons nil (list 1 2 3 4 5))
~\textit{(1 2 3 4 5)}~
\end{scheme}

\noindent
All that remains to implement signal-flow diagrams is to enumerate the sequence
of elements to be processed.  For \code{even\-/fibs}, we need to generate the
sequence of integers in a given range, which we can do as follows:

\begin{scheme}
(define (enumerate-interval low high)
  (if (> low high)
      nil
      (cons low (enumerate-interval (+ low 1) high))))
(enumerate-interval 2 7)
~\textit{(2 3 4 5 6 7)}~
\end{scheme}

\noindent
To enumerate the leaves of a tree, we can use\footnote{This is, in fact,
precisely the \code{fringe} procedure from \link{Exercise 2.28}.  Here we've
renamed it to emphasize that it is part of a family of general
sequence-manipulation procedures.}

\begin{scheme}
(define (enumerate-tree tree)
  (cond ((null? tree) nil)
        ((not (pair? tree)) (list tree))
        (else (append (enumerate-tree (car tree))
                      (enumerate-tree (cdr tree))))))
(enumerate-tree (list 1 (list 2 (list 3 4)) 5))
~\textit{(1 2 3 4 5)}~
\end{scheme}

\noindent
Now we can reformulate \code{sum\-/odd\-/squares} and \code{even\-/fibs} as in the
signal-flow diagrams.  For \code{sum\-/odd\-/squares}, we enumerate the sequence of
leaves of the tree, filter this to keep only the odd numbers in the sequence,
square each element, and sum the results:

\begin{scheme}
(define (sum-odd-squares tree)
  (accumulate
   + 0 (map square (filter odd? (enumerate-tree tree)))))
\end{scheme}

\noindent
For \code{even\-/fibs}, we enumerate the integers from 0 to \( n \), generate the
Fibonacci number for each of these integers, filter the resulting sequence to
keep only the even elements, and accumulate the results into a list:

\begin{scheme}
(define (even-fibs n)
  (accumulate
   cons
   nil
   (filter even? (map fib (enumerate-interval 0 n)))))
\end{scheme}

\noindent
The value of expressing programs as sequence operations is that this helps us
make program designs that are modular, that is, designs that are constructed by
combining relatively independent pieces.  We can encourage modular design by
providing a library of standard components together with a conventional
interface for connecting the components in flexible ways.

Modular construction is a powerful strategy for controlling complexity in
engineering design.  In real signal-processing applications, for example,
designers regularly build systems by cascading elements selected from
standardized families of filters and transducers.  Similarly, sequence
operations provide a library of standard program elements that we can mix and
match.  For instance, we can reuse pieces from the \code{sum\-/odd\-/squares} and
\code{even\-/fibs} procedures in a program that constructs a list of the squares
of the first \( n + 1 \) Fibonacci numbers:

\begin{scheme}
(define (list-fib-squares n)
  (accumulate
   cons
   nil
   (map square (map fib (enumerate-interval 0 n)))))
(list-fib-squares 10)
~\textit{(0 1 1 4 9 25 64 169 441 1156 3025)}~
\end{scheme}

\noindent
We can rearrange the pieces and use them in computing the product of the squares of the odd
integers in a sequence:

\begin{scheme}
(define (product-of-squares-of-odd-elements sequence)
  (accumulate * 1 (map square (filter odd? sequence))))
(product-of-squares-of-odd-elements (list 1 2 3 4 5))
~\textit{225}~
\end{scheme}

\noindent
We can also formulate conventional data-processing applications in terms of
sequence operations.  Suppose we have a sequence of personnel records and we
want to find the salary of the highest-paid programmer.  Assume that we have a
selector \code{salary} that returns the salary of a record, and a predicate
\code{programmer?} that tests if a record is for a programmer.  Then we can
write

\begin{scheme}
(define (salary-of-highest-paid-programmer records)
  (accumulate max 0 (map salary (filter programmer? records))))
\end{scheme}

\noindent
These examples give just a hint of the vast range of operations that can be
expressed as sequence operations.\footnote{Richard \link{Waters (1979)} developed a
program that automatically analyzes traditional Fortran programs, viewing them
in terms of maps, filters, and accumulations.  He found that fully 90 percent
of the code in the Fortran Scientific Subroutine Package fits neatly into this
paradigm.  One of the reasons for the success of Lisp as a programming language
is that lists provide a standard medium for expressing ordered collections so
that they can be manipulated using higher-order operations.  The programming
language APL owes much of its power and appeal to a similar choice. In APL all
data are represented as arrays, and there is a universal and convenient set of
generic operators for all sorts of array operations.}

Sequences, implemented here as lists, serve as a conventional interface that
permits us to combine processing modules.  Additionally, when we uniformly
represent structures as sequences, we have localized the data-structure
dependencies in our programs to a small number of sequence operations.  By
changing these, we can experiment with alternative representations of
sequences, while leaving the overall design of our programs intact.  We will
exploit this capability in \link{Section 3.5}, when we generalize the
sequence-processing paradigm to admit infinite sequences.

\begin{quote}
\heading{\phantomsection\label{Exercise 2.33}Exercise 2.33:} Fill in the missing expressions
to complete the following definitions of some basic list-manipulation
operations as accumulations:

\begin{scheme}
(define (map p sequence)
  (accumulate (lambda (x y) ~\( \dark \langle \)~??~\( \dark \rangle \)~) nil sequence))
(define (append seq1 seq2)
  (accumulate cons ~\( \dark \langle \)~??~\( \dark \rangle \)~ ~\( \dark \langle \)~??~\( \dark \rangle \)~))
(define (length sequence)
  (accumulate ~\( \dark \langle \)~??~\( \dark \rangle \)~ 0 sequence))
\end{scheme}
\end{quote}

\begin{quote}
\heading{\phantomsection\label{Exercise 2.34}Exercise 2.34:} Evaluating a polynomial in \( x \)
at a given value of \( x \) can be formulated as an accumulation.  We evaluate
the polynomial
\begin{comment}

\begin{example}
a_n x^n + a_(n-1) x^(n-1) + ... + a_1 x + a_0
\end{example}

\end{comment}
\begin{displaymath}
 a_n x^n + a_{n-1} x^{n-1} + \dots + a_1 x + a_0 
\end{displaymath}
\noindent
using a well-known algorithm called \newterm{Horner's rule}, which structures
the computation as
\begin{comment}

\begin{example}
(... (a_n x + a_(n-1)) x + ... + a_1) x + a_0
\end{example}

\end{comment}
\begin{displaymath}
 (\dots (a_n x + a_{n-1}) x + \dots + a_1) x + a_0. 
\end{displaymath}
\noindent
In other words, we start with \( a_n \), multiply by \( x \), add
\( a_{n-1} \), multiply by \( x \), and so on, until we reach
\( a_0 \).\footnote{According to \link{Knuth 1981}, this rule was formulated by
W. G. Horner early in the nineteenth century, but the method was actually used
by Newton over a hundred years earlier.  Horner's rule evaluates the polynomial
using fewer additions and multiplications than does the straightforward method
of first computing \( a_n x^n \), then adding
\( a_{n-1}x^{n-1} \), and so on.  In fact, it is possible to prove
that any algorithm for evaluating arbitrary polynomials must use at least as
many additions and multiplications as does Horner's rule, and thus Horner's
rule is an optimal algorithm for polynomial evaluation.  This was proved (for
the number of additions) by A. M. Ostrowski in a 1954 paper that essentially
founded the modern study of optimal algorithms.  The analogous statement for
multiplications was proved by V. Y. Pan in 1966.  The book by \link{Borodin and Munro (1975)} 
provides an overview of these and other results about optimal
algorithms.}

Fill in the following template to produce a procedure that evaluates a
polynomial using Horner's rule.  Assume that the coefficients of the polynomial
are arranged in a sequence, from \( a_0 \) through \( a_n \).

\begin{scheme}
(define (horner-eval x coefficient-sequence)
  (accumulate (lambda (this-coeff higher-terms) ~\( \dark \langle \)~??~\( \dark \rangle \)~)
              0
              coefficient-sequence))
\end{scheme}

For example, to compute \( 1 + 3x + 5x^3 + x^5 \) at \( x = 2 \) you
would evaluate

\begin{scheme}
(horner-eval 2 (list 1 3 0 5 0 1))
\end{scheme}
\end{quote}

\begin{quote}
\heading{\phantomsection\label{Exercise 2.35}Exercise 2.35:} Redefine \code{count\-/leaves} from
\link{Section 2.2.2} as an accumulation:

\begin{scheme}
(define (count-leaves t)
  (accumulate ~\( \dark \langle \)~??~\( \dark \rangle \)~ ~\( \dark \langle \)~??~\( \dark \rangle \)~ (map ~\( \dark \langle \)~??~\( \dark \rangle \)~ ~\( \dark \langle \)~??~\( \dark \rangle \)~)))
\end{scheme}
\end{quote}

\begin{quote}
\heading{\phantomsection\label{Exercise 2.36}Exercise 2.36:} The procedure \code{accumulate\-/n}
is similar to \code{accumu\-/late} except that it takes as its third argument a
sequence of sequences, which are all assumed to have the same number of
elements.  It applies the designated accumulation procedure to combine all the
first elements of the sequences, all the second elements of the sequences, and
so on, and returns a sequence of the results.  For instance, if \code{s} is a
sequence containing four sequences, \code{((1 2 3) (4 5 6) (7 8 9) (10 11
12)),} then the value of \code{(accumulate\-/n + 0 s)} should be the sequence
\code{(22 26 30)}.  Fill in the missing expressions in the following definition
of \code{accumulate\-/n}:

\begin{scheme}
(define (accumulate-n op init seqs)
  (if (null? (car seqs))
      nil
      (cons (accumulate op init ~\( \dark \langle \)~??~\( \dark \rangle \)~)
            (accumulate-n op init ~\( \dark \langle \)~??~\( \dark \rangle \)~))))
\end{scheme}
\end{quote}

\begin{quote}
\heading{\phantomsection\label{Exercise 2.37}Exercise 2.37:}
Suppose we represent vectors \( \hbox{\bf v} = (v_i) \) as sequences of numbers, and
matrices \( \hbox{\bf m} = (m_{ij}) \) as sequences of vectors (the rows of the
matrix).  For example, the matrix
\begin{comment}

\begin{example}
+-         -+
|  1 2 3 4  |
|  4 5 6 6  |
|  6 7 8 9  |
+-         -+
\end{example}

\end{comment}
\begin{displaymath}
%  
% \left(\matrix{	1 & 2 & 3 & 4 \cr
% 		4 & 5 & 6 & 6 \cr
% 		6 & 7 & 8 & 9 \cr }\right) 
\left(
\begin{array}{cccc}
  1 & 2 & 3 & 4 \\
  4 & 5 & 6 & 6 \\
  6 & 7 & 8 & 9 
\end{array}
\right) 
\end{displaymath}
\noindent
is represented as the sequence \code{((1 2 3 4) (4 5 6 6) (6 7 8 9))}.  With
this representation, we can use sequence operations to concisely express the
basic matrix and vector operations.  These operations (which are described in
any book on matrix algebra) are the following:
\begin{comment}

\begin{example}
                                       __
(dot-product v w)      returns the sum >_i v_i w_i

(matrix-*-vector m v)  returns the vector t,
                                   __
                       where t_i = >_j m_(ij) v_j

(matrix-*-matrix m n)  returns the matrix p,
                                      __
                       where p_(ij) = >_k m_(ik) n_(kj)

(transpose m)          returns the matrix n,
                       where n_(ij) = m_(ji)
\end{example}

\end{comment}
\begin{displaymath}
%  
% \eqalign{ 
% 	\hbox{\tt (dot-product v w)} 		&\; {\rm returns\;the\;sum\;} 
% 		\Sigma_i v_i w_i, \cr
% 	\hbox{\tt (matrix-*-vector m v)} 	&\; {\rm returns\;the\;vector\;} {\bf t}, 
% 		\; {\rm where\;} t_i = \Sigma_j m_{ij} v_j, \cr
% 	\hbox{\tt (matrix-*-matrix m n)} 	&\; {\rm returns\;the\;matrix\;} {\bf p},
% 		\; {\rm where\;} p_{ij} = \Sigma_k m_{ik} n_{kj}, \cr
% 	\hbox{\tt (transpose m)} 			&\; {\rm returns\;the\;matrix\;} {\bf n}, 
% 		\; {\rm where\;} n_{ij} = m_{ji}. \cr
% } 
\begin{array}{rl}
	\hbox{\tt (dot-product v w)} 		&\; {\rm returns\;the\;sum\;} 
		\Sigma_i v_i w_i, \\
	\hbox{\tt (matrix-*-vector m v)} 	&\; {\rm returns\;the\;vector\;} \hbox{\bf t}, \\ 
		&\; {\rm where\;} t_i = \Sigma_j m_{ij} v_j, \\
	\hbox{\tt (matrix-*-matrix m n)} 	&\; {\rm returns\;the\;matrix\;} \hbox{\bf p}, \\
		&\; {\rm where\;} p_{ij} = \Sigma_k m_{ik} n_{kj}, \\
	\hbox{\tt (transpose m)} 			&\; {\rm returns\;the\;matrix\;} \hbox{\bf n}, \\
		&\; {\rm where\;} n_{ij} = m_{ji}.
\end{array}
\end{displaymath}
We can define the dot product as\footnote{This definition uses the extended
version of \code{map} described in \link{Footnote 12}.}

\begin{scheme}
(define (dot-product v w)
  (accumulate + 0 (map * v w)))
\end{scheme}

Fill in the missing expressions in the following procedures for computing the
other matrix operations.  (The procedure \code{accumulate\-/n} is defined in
\link{Exercise 2.36}.)

\begin{scheme}
(define (matrix-*-vector m v)
  (map ~\( \dark \langle \)~??~\( \dark \rangle \)~ m))
(define (transpose mat)
  (accumulate-n ~\( \dark \langle \)~??~\( \dark \rangle \)~ ~\( \dark \langle \)~??~\( \dark \rangle \)~ mat))
(define (matrix-*-matrix m n)
  (let ((cols (transpose n)))
    (map ~\( \dark \langle \)~??~\( \dark \rangle \)~ m)))
\end{scheme}
\end{quote}

\begin{quote}
\heading{\phantomsection\label{Exercise 2.38}Exercise 2.38:} The \code{accumulate} procedure
is also known as \code{fold\-/right}, because it combines the first element of
the sequence with the result of combining all the elements to the right.  There
is also a \code{fold\-/left}, which is similar to \code{fold\-/right}, except that
it combines elements working in the opposite direction:

\begin{scheme}
(define (fold-left op initial sequence)
  (define (iter result rest)
    (if (null? rest)
        result
        (iter (op result (car rest))
              (cdr rest))))
  (iter initial sequence))
\end{scheme}

What are the values of

\begin{scheme}
(fold-right / 1 (list 1 2 3))
(fold-left / 1 (list 1 2 3))
(fold-right list nil (list 1 2 3))
(fold-left list nil (list 1 2 3))
\end{scheme}

Give a property that \code{op} should satisfy to guarantee that
\code{fold\-/right} and \code{fold\-/left} will produce the same values for any
sequence.
\end{quote}

\begin{quote}
\heading{\phantomsection\label{Exercise 2.39}Exercise 2.39:} Complete the following
definitions of \code{reverse} (\link{Exercise 2.18}) in terms of
\code{fold\-/right} and \code{fold\-/left} from \link{Exercise 2.38}:

\begin{scheme}
(define (reverse sequence)
  (fold-right (lambda (x y) ~\( \dark \langle \)~??~\( \dark \rangle \)~) nil sequence))
(define (reverse sequence)
  (fold-left (lambda (x y) ~\( \dark \langle \)~??~\( \dark \rangle \)~) nil sequence))
\end{scheme}
\end{quote}

\subsubsection*{Nested Mappings}

We can extend the sequence paradigm to include many computations that are
commonly expressed using nested loops.\footnote{This approach to nested
mappings was shown to us by David Turner, whose languages KRC and Miranda
provide elegant formalisms for dealing with these constructs.  The examples in
this section (see also \link{Exercise 2.42}) are adapted from \link{Turner 1981}.  In
\link{Section 3.5.3}, we'll see how this approach generalizes to infinite
sequences.} Consider this problem: Given a positive integer \( n \), find all
ordered pairs of distinct positive integers \( i \) and \( j \), where 
\( 1 \le j < i \le n \), such that \( i + j \) is prime.  For example, if \( n \) is 6,
then the pairs are the following:
\begin{comment}

\begin{example}
  i   | 2 3 4 4 5 6 6
  j   | 1 2 1 3 2 1 5
------+---------------
i + j | 3 5 5 7 7 7 11
\end{example}

\end{comment}
\begin{displaymath}
\vbox{
\offinterlineskip
\halign{
\strut \hfil \quad #\quad \hfil & \vrule 
	\hfil \quad #\quad \hfil &
	\hfil \quad #\quad \hfil &
	\hfil \quad #\quad \hfil &
	\hfil \quad #\quad \hfil &
	\hfil \quad #\quad \hfil &
	\hfil \quad #\quad \hfil &
	\hfil \quad #\quad \hfil \cr

$i$ 	& 2 & 3 & 4 & 4 & 5 & 6 & 6 \cr
$j$ 	& 1 & 2 & 1 & 3 & 2 & 1 & 5 \cr
\noalign{\hrule}
$i + j$	& 3 & 5 & 5 & 7 & 7 & 7 & 11 \cr}
}
\end{displaymath}
A natural way to organize this computation is to generate the sequence of all
ordered pairs of positive integers less than or equal to \( n \), filter to
select those pairs whose sum is prime, and then, for each pair \( (i, j) \)
that passes through the filter, produce the triple \( (i, j, i + j) \).

Here is a way to generate the sequence of pairs: For each integer \( i \le n \), 
enumerate the integers \( j < i \), and for each such \( i \) and \( j \)
generate the pair \( (i, j) \).  In terms of sequence operations, we map along
the sequence \code{(enumerate\-/interval 1 n)}.  For each \( i \) in this sequence,
we map along the sequence \code{(enumerate\-/interval 1 (- i 1))}.  For each
\( j \) in this latter sequence, we generate the pair \code{(list i j)}.  This
gives us a sequence of pairs for each \( i \).  Combining all the sequences for
all the \( i \) (by accumulating with \code{append}) produces the required
sequence of pairs:\footnote{We're representing a pair here as a list of two
elements rather than as a Lisp pair.  Thus, the ``pair'' \( (i, j) \) is
represented as \code{(list i j)}, not \code{(cons i j)}.}

\begin{scheme}
(accumulate
 append nil (map (lambda (i)
                   (map (lambda (j) (list i j))
                        (enumerate-interval 1 (- i 1))))
                 (enumerate-interval 1 n)))
\end{scheme}

\noindent
The combination of mapping and accumulating with \code{append} is so common in
this sort of program that we will isolate it as a separate procedure:

\begin{scheme}
(define (flatmap proc seq)
  (accumulate append nil (map proc seq)))
\end{scheme}

\noindent
Now filter this sequence of pairs to find those whose sum is prime. The filter
predicate is called for each element of the sequence; its argument is a pair
and it must extract the integers from the pair.  Thus, the predicate to apply
to each element in the sequence is

\begin{scheme}
(define (prime-sum? pair)
  (prime? (+ (car pair) (cadr pair))))
\end{scheme}

\noindent
Finally, generate the sequence of results by mapping over the filtered pairs
using the following procedure, which constructs a triple consisting of the two
elements of the pair along with their sum:

\begin{scheme}
(define (make-pair-sum pair)
  (list (car pair) (cadr pair) (+ (car pair) (cadr pair))))
\end{scheme}

\noindent
Combining all these steps yields the complete procedure:

\begin{smallscheme}
(define (prime-sum-pairs n)
  (map make-pair-sum
       (filter prime-sum? (flatmap
                           (lambda (i)
                             (map (lambda (j) (list i j))
                                  (enumerate-interval 1 (- i 1))))
                           (enumerate-interval 1 n)))))
\end{smallscheme}

\noindent
Nested mappings are also useful for sequences other than those that enumerate
intervals.  Suppose we wish to generate all the permutations of a set \( S; \)
that is, all the ways of ordering the items in the set.  For instance, the
permutations of \( \{1, 2, 3\} \) are \( \{1, 2, 3\} \), \( \{1, 3, 2\} \), \( \{2, 1, 3\} \), \( \{2, 3, 1\} \),
\( \{3, 1, 2\} \), and \( \{3, 2, 1\} \).  Here is a plan for generating the permutations of
\( S \): For each item \( x \) in \( S \), recursively generate the sequence of
permutations of \( S - x \),\footnote{The set \( S - x \) is the set of all
elements of \( S \), excluding \( x \).} and adjoin \( x \) to the front of each one.
This yields, for each \( x \) in \( S \), the sequence of permutations of \( S \)
that begin with \( x \).  Combining these sequences for all \( x \) gives all the
permutations of \( S \):\footnote{Semicolons in Scheme code are used to introduce
\newterm{comments}.  Everything from the semicolon to the end of the line is
ignored by the interpreter.  In this book we don't use many comments; we try to
make our programs self-documenting by using descriptive names.}

\begin{scheme}
(define (permutations s)
  (if (null? s)              ~\textrm{; empty set?}~
      (list nil)             ~\textrm{; sequence containing empty set}~
      (flatmap (lambda (x)
                 (map (lambda (p) (cons x p))
                      (permutations (remove x s))))
               s)))
\end{scheme}

\noindent
Notice how this strategy reduces the problem of generating permutations of
\( S \) to the problem of generating the permutations of sets with fewer elements
than \( S \).  In the terminal case, we work our way down to the empty list,
which represents a set of no elements.  For this, we generate \code{(list
nil)}, which is a sequence with one item, namely the set with no elements.  The
\code{remove} procedure used in \code{permutations} returns all the items in a
given sequence except for a given item.  This can be expressed as a simple
filter:

\begin{scheme}
(define (remove item sequence)
  (filter (lambda (x) (not (= x item)))
          sequence))
\end{scheme}

\begin{quote}
\heading{\phantomsection\label{Exercise 2.40}Exercise 2.40:} Define a procedure
\code{unique\-/pairs} that, given an integer \( n \), generates the sequence of
pairs \( (i, j) \) with \( 1 \le j < i \le n \).  Use \code{unique\-/pairs}
to simplify the definition of \code{prime\-/sum\-/pairs} given above.
\end{quote}

\begin{quote}
\heading{\phantomsection\label{Exercise 2.41}Exercise 2.41:} Write a procedure to find all
ordered triples of distinct positive integers \( i \), \( j \), and \( k \) less than
or equal to a given integer \( n \) that sum to a given integer \( s \).
\end{quote}

\begin{quote}
\heading{\phantomsection\label{Exercise 2.42}Exercise 2.42:} The ``eight-queens puzzle'' asks
how to place eight queens on a chessboard so that no queen is in check from any
other (i.e., no two queens are in the same row, column, or diagonal).  One
possible solution is shown in \link{Figure 2.8}.  One way to solve the puzzle is
to work across the board, placing a queen in each column.  Once we have placed
\( k - 1 \) queens, we must place the \( k^{\mathrm{th}} \) queen in a position where it does
not check any of the queens already on the board.  We can formulate this
approach recursively: Assume that we have already generated the sequence of all
possible ways to place \( k - 1 \) queens in the first \( k - 1 \) columns of the
board.  For each of these ways, generate an extended set of positions by
placing a queen in each row of the \( k^{\mathrm{th}} \) column.  Now filter these, keeping
only the positions for which the queen in the \( k^{\mathrm{th}} \) column is safe with
respect to the other queens.  This produces the sequence of all ways to place
\( k \) queens in the first \( k \) columns.  By continuing this process, we will
produce not only one solution, but all solutions to the puzzle.

\begin{figure}[tb]
\phantomsection\label{Figure 2.8}
\centering
\begin{comment}
\heading{Figure 2.8:} A solution to the eight-queens puzzle.

\begin{example}
+---+---+---+---+---+---+---+---+
|   |   |   |   |   | Q |   |   |
+---+---+---+---+---+---+---+---+
|   |   | Q |   |   |   |   |   |
+---+---+---+---+---+---+---+---+
| Q |   |   |   |   |   |   |   |
+---+---+---+---+---+---+---+---+
|   |   |   |   |   |   | Q |   |
+---+---+---+---+---+---+---+---+
|   |   |   |   | Q |   |   |   |
+---+---+---+---+---+---+---+---+
|   |   |   |   |   |   |   | Q |
+---+---+---+---+---+---+---+---+
|   | Q |   |   |   |   |   |   |
+---+---+---+---+---+---+---+---+
|   |   |   | Q |   |   |   |   |
+---+---+---+---+---+---+---+---+
\end{example}
\end{comment}
\includegraphics[width=48mm]{fig/chap2/Fig2.8c.pdf}      % 73mm
\par\bigskip
\noindent
\heading{Figure 2.8:} A solution to the eight-queens puzzle.
\end{figure}

We implement this solution as a procedure \code{queens}, which returns a
sequence of all solutions to the problem of placing \( n \) queens on an
\( n \times n \) chessboard.  \code{Queens} has an internal procedure
\code{queen\-/cols} that returns the sequence of all ways to place queens in the
first \( k \) columns of the board.

\begin{scheme}

(define (queens board-size)
  (define (queen-cols k)
    (if (= k 0)
        (list empty-board)
        (filter
         (lambda (positions) (safe? k positions))
         (flatmap
          (lambda (rest-of-queens)
            (map (lambda (new-row)
                   (adjoin-position
                    new-row k rest-of-queens))
                 (enumerate-interval 1 board-size)))
          (queen-cols (- k 1))))))
  (queen-cols board-size))
\end{scheme}

In this procedure \code{rest\-/of\-/queens} is a way to place \( k - 1 \) queens in
the first \( k - 1 \) columns, and \code{new\-/row} is a proposed row in which to
place the queen for the \( k^{\mathrm{th}} \) column.  Complete the program by implementing
the representation for sets of board positions, including the procedure
\code{adjoin\-/position}, which adjoins a new row-column position to a set of
positions, and \code{empty\-/board}, which represents an empty set of positions.
You must also write the procedure \code{safe?}, which determines for a set of
positions, whether the queen in the \( k^{\mathrm{th}} \) column is safe with respect to the
others.  (Note that we need only check whether the new queen is safe---the
other queens are already guaranteed safe with respect to each other.)
\end{quote}

\begin{quote}
\heading{\phantomsection\label{Exercise 2.43}Exercise 2.43:} Louis Reasoner is having a
terrible time doing \link{Exercise 2.42}.  His \code{queens} procedure seems to
work, but it runs extremely slowly.  (Louis never does manage to wait long
enough for it to solve even the \( 6\times6 \) case.)  When Louis asks Eva Lu Ator for
help, she points out that he has interchanged the order of the nested mappings
in the \code{flatmap}, writing it as

\begin{scheme}
(flatmap
 (lambda (new-row)
   (map (lambda (rest-of-queens)
          (adjoin-position new-row k rest-of-queens))
        (queen-cols (- k 1))))
 (enumerate-interval 1 board-size))
\end{scheme}

Explain why this interchange makes the program run slowly.  Estimate how long
it will take Louis's program to solve the eight-queens puzzle, assuming that
the program in \link{Exercise 2.42} solves the puzzle in time \( T \).
\end{quote}

\subsection{Example: A Picture Language}
\label{Section 2.2.4}

This section presents a simple language for drawing pictures that illustrates
the power of data abstraction and closure, and also exploits higher-order
procedures in an essential way.  The language is designed to make it easy to
experiment with patterns such as the ones in \link{Figure 2.9}, which are
composed of repeated elements that are shifted and scaled.\footnote{The picture
language is based on the language Peter Henderson created to construct images
like M.C. Escher's ``Square Limit'' woodcut (see \link{Henderson 1982}).  The woodcut
incorporates a repeated scaled pattern, similar to the arrangements drawn using
the \code{square\-/limit} procedure in this section.} In this language, the data
objects being combined are represented as procedures rather than as list
structure.  Just as \code{cons}, which satisfies the closure property, allowed
us to easily build arbitrarily complicated list structure, the operations in
this language, which also satisfy the closure property, allow us to easily
build arbitrarily complicated patterns.


\subsubsection*{The picture language}

When we began our study of programming in \link{Section 1.1}, we emphasized the
importance of describing a language by focusing on the language's primitives,
its means of combination, and its means of abstraction.  We'll follow that
framework here.

\begin{figure}[tb]
\phantomsection\label{Figure 2.9}
\centering
\begin{comment}
\heading{Figure 2.9:} Designs generated with the picture language.

[two graphic images not included]
\end{comment}
\includegraphics[width=111mm]{fig/chap2/Fig2.9-bigger.png}
\par\bigskip
\noindent
\heading{Figure 2.9:} Designs generated with the picture language.
\end{figure}

\begin{figure}[tb]
\phantomsection\label{Figure 2.10}
\centering
\begin{comment}
\heading{Figure 2.10:} Images produced by the \code{wave}
painter, with respect to four different frames.  The frames, shown with dotted
lines, are not part of the images.

[four graphic images not included]
\end{comment}
\includegraphics[width=50mm]{fig/chap2/Fig2.10.pdf}
\begin{quote}
\heading{Figure 2.10:} Images produced by the \code{wave} painter, with respect to four different frames.  The frames, shown with dotted lines, are not part of the images.
\end{quote}
\end{figure}

Part of the elegance of this picture language is that there is only one kind of
element, called a \newterm{painter}.  A painter draws an image that is shifted
and scaled to fit within a designated parallelogram-shaped frame.  For example,
there's a primitive painter we'll call \code{wave} that makes a crude line
drawing, as shown in \link{Figure 2.10}.  The actual shape of the drawing
depends on the frame---all four images in figure 2.10 are produced by the
same \code{wave} painter, but with respect to four different frames.  Painters
can be more elaborate than this: The primitive painter called \code{rogers}
paints a picture of \acronym{MIT}'s founder, William Barton Rogers, as shown in
\link{Figure 2.11}.\footnote{William Barton Rogers (1804-1882) was the founder
and first president of \acronym{MIT}.  A geologist and talented teacher, he
taught at William and Mary College and at the University of Virginia.  In 1859
he moved to Boston, where he had more time for research, worked on a plan for
establishing a ``polytechnic institute,'' and served as Massachusetts's first
State Inspector of Gas Meters.

When \acronym{MIT} was established in 1861, Rogers was elected its first
president.  Rogers espoused an ideal of ``useful learning'' that was different
from the university education of the time, with its overemphasis on the
classics, which, as he wrote, ``stand in the way of the broader, higher and
more practical instruction and discipline of the natural and social sciences.''
This education was likewise to be different from narrow trade-school education.
In Rogers's words:

\begin{quote}
The world-enforced distinction between the practical and the scientific worker
is utterly futile, and the whole experience of modern times has demonstrated
its utter worthlessness.
\end{quote}

Rogers served as president of \acronym{MIT} until 1870, when he resigned due to
ill health.  In 1878 the second president of \acronym{MIT}, John Runkle,
resigned under the pressure of a financial crisis brought on by the Panic of
1873 and strain of fighting off attempts by Harvard to take over \acronym{MIT}.
Rogers returned to hold the office of president until 1881.

Rogers collapsed and died while addressing \acronym{MIT}'s graduating class at
the commencement exercises of 1882.  Runkle quoted Rogers's last words in a
memorial address delivered that same year:

\begin{quote}
``As I stand here today and see what the Institute is, \( \dots \) I call to mind
the beginnings of science.  I remember one hundred and fifty years ago Stephen
Hales published a pamphlet on the subject of illuminating gas, in which he
stated that his researches had demonstrated that 128 grains of bituminous coal
-- '' ``Bituminous coal,'' these were his last words on earth.  Here he bent
forward, as if consulting some notes on the table before him, then slowly
regaining an erect position, threw up his hands, and was translated from the
scene of his earthly labors and triumphs to ``the tomorrow of death,'' where
the mysteries of life are solved, and the disembodied spirit finds unending
satisfaction in contemplating the new and still unfathomable mysteries of the
infinite future.
\end{quote}

In the words of Francis A. Walker (\acronym{MIT}'s third president):

\begin{quote}
All his life he had borne himself most faithfully and heroically, and he died
as so good a knight would surely have wished, in harness, at his post, and in
the very part and act of public duty.
\end{quote}
} The four images in figure 2.11 are drawn with respect to the same four
frames as the \code{wave} images in figure 2.10.

\begin{figure}[tb]
\phantomsection\label{Figure 2.11}
\centering
\begin{comment}
\heading{Figure 2.11:} Images of William Barton Rogers,
founder and first president of \acronym{MIT}, painted with respect to the same
four frames as in \link{Figure 2.10} (original image reprinted with the
permission of the \acronym{MIT} Museum).

[four graphic images not included]
\end{comment}
\includegraphics[width=48mm]{fig/chap2/Fig2.11.pdf}
\begin{quote}
\heading{Figure 2.11:} Images of William Barton Rogers, founder and first president of \acronym{MIT}, painted with respect to the same four frames as in \link{Figure 2.10} (original image from Wikimedia Commons).
\end{quote}
\end{figure}

To combine images, we use various operations that construct new painters from
given painters.  For example, the \code{beside} operation takes two painters
and produces a new, compound painter that draws the first painter's image in
the left half of the frame and the second painter's image in the right half of
the frame.  Similarly, \code{below} takes two painters and produces a compound
painter that draws the first painter's image below the second painter's image.
Some operations transform a single painter to produce a new painter.  For
example, \code{flip\-/vert} takes a painter and produces a painter that draws its
image upside-down, and \code{flip\-/horiz} produces a painter that draws the
original painter's image left-to-right reversed.

\link{Figure 2.12} shows the drawing of a painter called
\code{wave4} that is built up in two stages starting from \code{wave}:

\begin{scheme}
(define wave2 (beside wave (flip-vert wave)))
(define wave4 (below wave2 wave2))
\end{scheme}

\begin{figure}[tb]
\phantomsection\label{Figure 2.12}
\centering
\begin{comment}
\heading{Figure 2.12:} Creating a complex figure, starting
from the \code{wave} painter of \link{Figure 2.10}.

[two graphic images not included]

\begin{scheme}
(define wave2                      (define wave4
  (beside wave (flip-vert wave)))    (below wave2 wave2))
\end{scheme}
\end{comment}
\includegraphics[width=50mm]{fig/chap2/Fig2.12.pdf}
\begin{quote}
\heading{Figure 2.12:} Creating a complex figure, starting from the \code{wave} painter of \link{Figure 2.10}.
\end{quote}
\end{figure}

In building up a complex image in this manner we are exploiting the fact that
painters are closed under the language's means of combination.  The
\code{beside} or \code{below} of two painters is itself a painter; therefore,
we can use it as an element in making more complex painters.  As with building
up list structure using \code{cons}, the closure of our data under the means of
combination is crucial to the ability to create complex structures while using
only a few operations.

Once we can combine painters, we would like to be able to abstract typical
patterns of combining painters.  We will implement the painter operations as
Scheme procedures.  This means that we don't need a special abstraction
mechanism in the picture language: Since the means of combination are ordinary
Scheme procedures, we automatically have the capability to do anything with
painter operations that we can do with procedures.  For example, we can
abstract the pattern in \code{wave4} as

\begin{scheme}
(define (flipped-pairs painter)
  (let ((painter2 (beside painter (flip-vert painter))))
    (below painter2 painter2)))
\end{scheme}

\noindent
and define \code{wave4} as an instance of this pattern:

\begin{scheme}
(define wave4 (flipped-pairs wave))
\end{scheme}

\begin{figure}[tb]
\phantomsection\label{Figure 2.13}
\centering
\begin{comment}
\begin{quote}
\heading{Figure 2.13:} Recursive plans for \code{right\-/split} and \code{corner\-/split}.

\begin{example}
+-------------+-------------+    +------+------+-------------+
|             |             |    | up-  | up-  |             |
|             | right-split |    | split| split| corner-split|
|             |             |    |      |      |             |
|             |     n-1     |    |  n-1 |  n-1 |     n-1     |
|             |             |    |      |      |             |
|  identity   +-------------+    +------+------+-------------+
|             |             |    |             | right-split |
|             | right-split |    |             |     n-1     |
|             |             |    |  identity   +-------------+
|             |     n-1     |    |             | right-split |
|             |             |    |             |     n-1     |
+-------------+-------------+    +-------------+-------------+

       right-split n                    corner-split n
\end{example}
\end{quote}
\end{comment}
\includegraphics[width=111mm]{fig/chap2/Fig2.13a.pdf}
\par\bigskip
\noindent
\heading{Figure 2.13:} Recursive plans for \code{right\-/split} and \code{corner\-/split}. 
\end{figure}

\noindent
We can also define recursive operations.  Here's one that makes painters split
and branch towards the right as shown in \link{Figure 2.13}
and \link{Figure 2.14}:

\begin{scheme}
(define (right-split painter n)
  (if (= n 0)
      painter
      (let ((smaller (right-split painter (- n 1))))
        (beside painter (below smaller smaller)))))
\end{scheme}

\noindent
We can produce balanced patterns by branching upwards as well as towards the
right (see exercise \link{Exercise 2.44} and figures \link{Figure 2.13} and \link{Figure 2.14}):

\begin{scheme}
(define (corner-split painter n)
  (if (= n 0)
      painter
      (let ((up (up-split painter (- n 1)))
            (right (right-split painter (- n 1))))
        (let ((top-left (beside up up))
              (bottom-right (below right right))
              (corner (corner-split painter (- n 1))))
          (beside (below painter top-left)
                  (below bottom-right corner))))))
\end{scheme}

\noindent
By placing four copies of a \code{corner\-/split} appropriately, we obtain a
pattern called \code{square\-/limit}, whose application to \code{wave} and
\code{rogers} is shown in \link{Figure 2.9}:

\begin{scheme}
(define (square-limit painter n)
  (let ((quarter (corner-split painter n)))
    (let ((half (beside (flip-horiz quarter) quarter)))
      (below (flip-vert half) half))))
\end{scheme}

\begin{quote}
\heading{\phantomsection\label{Exercise 2.44}Exercise 2.44:} Define the procedure
\code{up\-/split} used by \code{corner\-/split}.  It is similar to
\code{right\-/split}, except that it switches the roles of \code{below} and
\code{beside}.
\end{quote}

\begin{figure}[tbp]
\phantomsection\label{Figure 2.14}
\centering
\begin{comment}
\heading{Figure 2.14:} The recursive operations \code{right\-/split} and \code{corner\-/split} applied to the painters \code{wave} and \code{rogers}.  Combining four \code{corner\-/split} figures produces symmetric \code{square\-/limit} designs as shown in \link{Figure 2.9}.

[two graphic images not included]

\begin{scheme}
(right-split wave 4)         (right-split rogers 4)
\end{scheme}

[two graphic images not included]

\begin{scheme}
(corner-split wave 4)        (corner-split rogers 4)
\end{scheme}
\end{comment}
\includegraphics[width=91mm]{fig/chap2/Fig2.14b.pdf}
\begin{quote}
\heading{Figure 2.14:} The recursive operations \code{right\-/split} and \code{corner\-/split} applied to the painters \code{wave} and \code{rogers}.  Combining four \code{corner\-/split} figures produces symmetric \code{square\-/limit} designs as shown in \link{Figure 2.9}.
\end{quote}
\end{figure}

\subsubsection*{Higher-order operations}

In addition to abstracting patterns of combining painters, we can work at a
higher level, abstracting patterns of combining painter operations.  That is,
we can view the painter operations as elements to manipulate and can write
means of combination for these elements---procedures that take painter
operations as arguments and create new painter operations.

For example, \code{flipped\-/pairs} and \code{square\-/limit} each arrange four
copies of a painter's image in a square pattern; they differ only in how they
orient the copies.  One way to abstract this pattern of painter combination is
with the following procedure, which takes four one-argument painter operations
and produces a painter operation that transforms a given painter with those
four operations and arranges the results in a square.  \code{Tl}, \code{tr},
\code{bl}, and \code{br} are the transformations to apply to the top left copy,
the top right copy, the bottom left copy, and the bottom right copy,
respectively.

\begin{scheme}
(define (square-of-four tl tr bl br)
  (lambda (painter)
    (let ((top (beside (tl painter) (tr painter)))
          (bottom (beside (bl painter) (br painter))))
      (below bottom top))))
\end{scheme}

\noindent
Then \code{flipped\-/pairs} can be defined in terms of \code{square\-/of\-/four} as
follows:\footnote{Equivalently, we could write

\begin{smallscheme}
(define flipped-pairs
  (square-of-four identity flip-vert identity flip-vert))
\end{smallscheme}
}

\begin{scheme}
(define (flipped-pairs painter)
  (let ((combine4 (square-of-four identity flip-vert
                                  identity flip-vert)))
    (combine4 painter)))
\end{scheme}

\noindent
and \code{square\-/limit} can be expressed as\footnote{\code{Rotate180} rotates a
painter by 180 degrees (see \link{Exercise 2.50}).  Instead of \code{rotate180}
we could say \code{(compose flip\-/vert flip\-/horiz)}, using the \code{compose}
procedure from \link{Exercise 1.42}.}

\begin{scheme}
(define (square-limit painter n)
  (let ((combine4 (square-of-four flip-horiz identity
                                  rotate180 flip-vert)))
    (combine4 (corner-split painter n))))
\end{scheme}

\begin{quote}
\heading{\phantomsection\label{Exercise 2.45}Exercise 2.45:} \code{Right\-/split} and
\code{up\-/split} can be expressed as instances of a general splitting operation.
Define a procedure \code{split} with the property that evaluating

\begin{scheme}
(define right-split (split beside below))
(define up-split (split below beside))
\end{scheme}

\noindent
produces procedures \code{right\-/split} and \code{up\-/split} with the same
behaviors as the ones already defined.
\end{quote}

\subsubsection*{Frames}

Before we can show how to implement painters and their means of combination, we
must first consider frames.  A frame can be described by three vectors---an
origin vector and two edge vectors.  The origin vector specifies the offset of
the frame's origin from some absolute origin in the plane, and the edge vectors
specify the offsets of the frame's corners from its origin.  If the edges are
perpendicular, the frame will be rectangular.  Otherwise the frame will be a
more general parallelogram.

\link{Figure 2.15} shows a frame and its associated vectors.
In accordance with data abstraction, we need not be specific yet about how
frames are represented, other than to say that there is a constructor
\code{make\-/frame}, which takes three vectors and produces a frame, and three
corresponding selectors \code{origin\-/frame}, \code{edge1\-/frame}, and
\code{edge2\-/frame} (see \link{Exercise 2.47}).

\begin{figure}[tb]
\phantomsection\label{Figure 2.15}
\centering
\begin{comment}
\heading{Figure 2.15:} A frame is described by three vectors
-- an origin and two edges.

\begin{example}
                         __
                     __--  \
                 __--       \
      __     __--            \   __
     |\  __--                 \__-|
       \-                  __--
frame   \              __--
edge2    \         __--    frame
vector    \    __--        edge1
           \_--            vector
            -   <--+
          frame    |
          origin   +-- (0, 0) point
          vector       on display screen
\end{example}
\end{comment}
\includegraphics[width=51mm]{fig/chap2/Fig2.15a.pdf}
\begin{quote}
\heading{Figure 2.15:} A frame is described by three vectors --- an origin and two edges.
\end{quote}
\end{figure}

We will use coordinates in the unit square \( (0 \le x, y \le 1) \) to specify
images.  With each frame, we associate a \newterm{frame coordinate map}, which
will be used to shift and scale images to fit the frame.  The map transforms
the unit square into the frame by mapping the vector \( \hbox{\bf v} = (x, y) \) to
the vector sum
\begin{comment}

\begin{example}
Origin(Frame) + x * Edge_1(Frame) + y * Edge_2(Frame)
\end{example}

\end{comment}
\begin{displaymath}
 {\rm Origin(Frame)} + x \cdot {\rm Edge_1(Frame)} + y \cdot {\rm Edge_2(Frame)}. 
\end{displaymath}
\noindent
For example, (0, 0) is mapped to the origin of the frame, (1, 1) to the vertex
diagonally opposite the origin, and (0.5, 0.5) to the center of the frame.  We
can create a frame's coordinate map with the following
procedure:\footnote{\code{Frame\-/coord\-/map} uses the vector operations described
in \link{Exercise 2.46} below, which we assume have been implemented using some
representation for vectors.  Because of data abstraction, it doesn't matter
what this vector representation is, so long as the vector operations behave
correctly.}

\begin{scheme}
(define (frame-coord-map frame)
  (lambda (v)
    (add-vect
     (origin-frame frame)
     (add-vect (scale-vect (xcor-vect v) (edge1-frame frame))
               (scale-vect (ycor-vect v) (edge2-frame frame))))))
\end{scheme}

\noindent
Observe that applying \code{frame\-/coord\-/map} to a frame returns a procedure
that, given a vector, returns a vector.  If the argument vector is in the unit
square, the result vector will be in the frame.  For example,

\begin{scheme}
((frame-coord-map a-frame) (make-vect 0 0))
\end{scheme}

\noindent
returns the same vector as

\begin{scheme}
(origin-frame a-frame)
\end{scheme}

\begin{quote}
\heading{\phantomsection\label{Exercise 2.46}Exercise 2.46:} A two-dimensional vector \( \hbox{\bf v} \)
running from the origin to a point can be represented as a pair consisting of
an \( x \)-coordinate and a \( y \)-coordinate.  Implement a data abstraction for
vectors by giving a constructor \code{make\-/vect} and corresponding selectors
\code{xcor\-/vect} and \code{ycor\-/vect}.  In terms of your selectors and
constructor, implement procedures \code{add\-/vect}, \code{sub\-/vect}, and
\code{scale\-/vect} that perform the operations vector addition, vector
subtraction, and multiplying a vector by a scalar:
\begin{comment}

\begin{example}
(x_1, y_1) + (x_2, y_2) = (x_1 + x_2, y_1 + y_2)
(x_1, y_1) - (x_2, y_2) = (x_1 - x_2, y_1 - y_2)
             s * (x, y) = (sx, sy)
\end{example}
\end{comment}
\begin{displaymath}
%  
% \eqalign{
% 	(x_1, y_1) + (x_2, y_2) 	&= (x_1 + x_2, y_1 + y_2), \cr
% 	(x_1, y_1) - (x_2, y_2) 	&= (x_1 - x_2, y_1 - y_2), \cr
% 	s \cdot (x, y) 			&= (sx, sy). \cr }
% 
\begin{array}{r@{{}={}}l}
	(x_1, y_1) + (x_2, y_2) 	& (x_1 + x_2, y_1 + y_2), \\
	(x_1, y_1) - (x_2, y_2) 	& (x_1 - x_2, y_1 - y_2), \\
	s \cdot (x, y) 			& (sx, sy). 
\end{array}
\end{displaymath}
\end{quote}

\begin{quote}
\heading{\phantomsection\label{Exercise 2.47}Exercise 2.47:} Here are two possible
constructors for frames:

\begin{scheme}
(define (make-frame origin edge1 edge2)
  (list origin edge1 edge2))
(define (make-frame origin edge1 edge2)
  (cons origin (cons edge1 edge2)))
\end{scheme}

For each constructor supply the appropriate selectors to produce an
implementation for frames.
\end{quote}

\subsubsection*{Painters}

A painter is represented as a procedure that, given a frame as argument, draws
a particular image shifted and scaled to fit the frame.  That is to say, if
\code{p} is a painter and \code{f} is a frame, then we produce \code{p}'s image
in \code{f} by calling \code{p} with \code{f} as argument.

The details of how primitive painters are implemented depend on the particular
characteristics of the graphics system and the type of image to be drawn.  For
instance, suppose we have a procedure \code{draw\-/line} that draws a line on the
screen between two specified points.  Then we can create painters for line
drawings, such as the \code{wave} painter in \link{Figure 2.10}, from lists of
line segments as follows:\footnote{\code{Segments\-/>painter} uses the
representation for line segments described in \link{Exercise 2.48} below.  It
also uses the \code{for\-/each} procedure described in \link{Exercise 2.23}.}

\begin{scheme}
(define (segments->painter segment-list)
  (lambda (frame)
    (for-each
     (lambda (segment)
       (draw-line
        ((frame-coord-map frame) 
         (start-segment segment))
        ((frame-coord-map frame)
         (end-segment segment))))
     segment-list)))
\end{scheme}

\noindent
The segments are given using coordinates with respect to the unit square.  For
each segment in the list, the painter transforms the segment endpoints with the
frame coordinate map and draws a line between the transformed points.

Representing painters as procedures erects a powerful abstraction barrier in
the picture language.  We can create and intermix all sorts of primitive
painters, based on a variety of graphics capabilities. The details of their
implementation do not matter.  Any procedure can serve as a painter, provided
that it takes a frame as argument and draws something scaled to fit the
frame.\footnote{For example, the \code{rogers} painter of \link{Figure 2.11} was
constructed from a gray-level image.  For each point in a given frame, the
\code{rogers} painter determines the point in the image that is mapped to it
under the frame coordinate map, and shades it accordingly.  By allowing
different types of painters, we are capitalizing on the abstract data idea
discussed in \link{Section 2.1.3}, where we argued that a rational-number
representation could be anything at all that satisfies an appropriate
condition.  Here we're using the fact that a painter can be implemented in any
way at all, so long as it draws something in the designated frame.  
\link{Section 2.1.3} also showed how pairs could be implemented as procedures.  Painters
are our second example of a procedural representation for data.}

\begin{quote}
\heading{\phantomsection\label{Exercise 2.48}Exercise 2.48:} A directed line segment in the
plane can be represented as a pair of vectors---the vector running from the
origin to the start-point of the segment, and the vector running from the
origin to the end-point of the segment.  Use your vector representation from
\link{Exercise 2.46} to define a representation for segments with a constructor
\code{make\-/segment} and selectors \code{start\-/segment} and \code{end\-/segment}.
\end{quote}

\begin{quote}
\heading{\phantomsection\label{Exercise 2.49}Exercise 2.49:} Use \code{segments\-/>painter}
to define the following primitive painters:

\begin{enumerate}[a.]

\item
The painter that draws the outline of the designated frame.

\item
The painter that draws an ``X'' by connecting opposite corners of the frame.

\item
The painter that draws a diamond shape by connecting the midpoints of the sides
of the frame.

\item
The \code{wave} painter.

\end{enumerate}
\end{quote}

\subsubsection*{Transforming and combining painters}

An operation on painters (such as \code{flip\-/vert} or \code{beside}) works by
creating a painter that invokes the original painters with respect to frames
derived from the argument frame.  Thus, for example, \code{flip\-/vert} doesn't
have to know how a painter works in order to flip it---it just has to know how
to turn a frame upside down: The flipped painter just uses the original
painter, but in the inverted frame.

Painter operations are based on the procedure \code{transform\-/painter}, which
takes as arguments a painter and information on how to transform a frame and
produces a new painter.  The transformed painter, when called on a frame,
transforms the frame and calls the original painter on the transformed frame.
The arguments to \code{transform\-/painter} are points (represented as vectors)
that specify the corners of the new frame: When mapped into the frame, the
first point specifies the new frame's origin and the other two specify the ends
of its edge vectors.  Thus, arguments within the unit square specify a frame
contained within the original frame.

\begin{scheme}
(define (transform-painter painter origin corner1 corner2)
  (lambda (frame)
    (let ((m (frame-coord-map frame)))
      (let ((new-origin (m origin)))
        (painter (make-frame 
                  new-origin
                  (sub-vect (m corner1) new-origin)
                  (sub-vect (m corner2) new-origin)))))))
\end{scheme}

\noindent
Here's how to flip painter images vertically:

\begin{scheme}
(define (flip-vert painter)
  (transform-painter painter
                     (make-vect 0.0 1.0)   ~\textrm{; new \code{origin}}~
                     (make-vect 1.0 1.0)   ~\textrm{; new end of \code{edge1}}~
                     (make-vect 0.0 0.0))) ~\textrm{; new end of \code{edge2}}~
\end{scheme}

\noindent
Using \code{transform\-/painter}, we can easily define new transformations.
For example, we can define a painter that shrinks its image to the
upper-right quarter of the frame it is given:

\begin{scheme}
(define (shrink-to-upper-right painter)
  (transform-painter
   painter (make-vect 0.5 0.5)
   (make-vect 1.0 0.5) (make-vect 0.5 1.0)))
\end{scheme}

\noindent
Other transformations rotate images counterclockwise by 90
degrees\footnote{\code{Rotate90} is a pure rotation only for square frames,
because it also stretches and shrinks the image to fit into the rotated frame.}

\begin{scheme}
(define (rotate90 painter)
  (transform-painter painter 
                     (make-vect 1.0 0.0)
                     (make-vect 1.0 1.0)
                     (make-vect 0.0 0.0)))
\end{scheme}

\noindent
or squash images towards the center of the frame:\footnote{The diamond-shaped
images in \link{Figure 2.10} and \link{Figure 2.11} were created with
\code{squash\-/inwards} applied to \code{wave} and \code{rogers}.}

\begin{scheme}
(define (squash-inwards painter)
  (transform-painter painter
                     (make-vect 0.0 0.0)
                     (make-vect 0.65 0.35)
                     (make-vect 0.35 0.65)))
\end{scheme}

\noindent
Frame transformation is also the key to defining means of combining two or more
painters.  The \code{beside} procedure, for example, takes two painters,
transforms them to paint in the left and right halves of an argument frame
respectively, and produces a new, compound painter.  When the compound painter
is given a frame, it calls the first transformed painter to paint in the left
half of the frame and calls the second transformed painter to paint in the
right half of the frame:

\begin{scheme}
(define (beside painter1 painter2)
  (let ((split-point (make-vect 0.5 0.0)))
    (let ((paint-left
           (transform-painter 
            painter1
            (make-vect 0.0 0.0)
            split-point
            (make-vect 0.0 1.0)))
          (paint-right
           (transform-painter
            painter2
            split-point
            (make-vect 1.0 0.0)
            (make-vect 0.5 1.0))))
      (lambda (frame)
        (paint-left frame)
        (paint-right frame)))))
\end{scheme}

\noindent
Observe how the painter data abstraction, and in particular the representation
of painters as procedures, makes \code{beside} easy to implement.  The
\code{beside} procedure need not know anything about the details of the
component painters other than that each painter will draw something in its
designated frame.

\begin{quote}
\heading{\phantomsection\label{Exercise 2.50}Exercise 2.50:} Define the transformation
\code{flip\-/horiz}, which flips painters horizontally, and transformations that
rotate painters counterclockwise by 180 degrees and 270 degrees.
\end{quote}

\begin{quote}
\heading{\phantomsection\label{Exercise 2.51}Exercise 2.51:} Define the \code{below} operation
for painters.  \code{Below} takes two painters as arguments.  The resulting
painter, given a frame, draws with the first painter in the bottom of the frame
and with the second painter in the top.  Define \code{below} in two different
ways---first by writing a procedure that is analogous to the \code{beside}
procedure given above, and again in terms of \code{beside} and suitable
rotation operations (from \link{Exercise 2.50}).
\end{quote}

\subsubsection*{Levels of language for robust design}

The picture language exercises some of the critical ideas we've introduced
about abstraction with procedures and data.  The fundamental data abstractions,
painters, are implemented using procedural representations, which enables the
language to handle different basic drawing capabilities in a uniform way.  The
means of combination satisfy the closure property, which permits us to easily
build up complex designs.  Finally, all the tools for abstracting procedures
are available to us for abstracting means of combination for painters.

We have also obtained a glimpse of another crucial idea about languages and
program design.  This is the approach of \newterm{stratified design}, the
notion that a complex system should be structured as a sequence of levels that
are described using a sequence of languages.  Each level is constructed by
combining parts that are regarded as primitive at that level, and the parts
constructed at each level are used as primitives at the next level.  The
language used at each level of a stratified design has primitives, means of
combination, and means of abstraction appropriate to that level of detail.

Stratified design pervades the engineering of complex systems.  For example, in
computer engineering, resistors and transistors are combined (and described
using a language of analog circuits) to produce parts such as and-gates and
or-gates, which form the primitives of a language for digital-circuit
design.\footnote{\link{Section 3.3.4} describes one such language.} These parts
are combined to build processors, bus structures, and memory systems, which are
in turn combined to form computers, using languages appropriate to computer
architecture.  Computers are combined to form distributed systems, using
languages appropriate for describing network interconnections, and so on.

As a tiny example of stratification, our picture language uses primitive
elements (primitive painters) that are created using a language that specifies
points and lines to provide the lists of line segments for
\code{segments\-/>painter}, or the shading details for a painter like
\code{rogers}.  The bulk of our description of the picture language focused on
combining these primitives, using geometric combiners such as \code{beside} and
\code{below}.  We also worked at a higher level, regarding \code{beside} and
\code{below} as primitives to be manipulated in a language whose operations,
such as \code{square\-/of\-/four}, capture common patterns of combining geometric
combiners.

Stratified design helps make programs \newterm{robust}, that is, it makes it
likely that small changes in a specification will require correspondingly small
changes in the program.  For instance, suppose we wanted to change the image
based on \code{wave} shown in \link{Figure 2.9}.  We could work at the lowest
level to change the detailed appearance of the \code{wave} element; we could
work at the middle level to change the way \code{corner\-/split} replicates the
\code{wave}; we could work at the highest level to change how
\code{square\-/limit} arranges the four copies of the corner.  In general, each
level of a stratified design provides a different vocabulary for expressing the
characteristics of the system, and a different kind of ability to change it.

\begin{quote}
\heading{\phantomsection\label{Exercise 2.52}Exercise 2.52:} Make changes to the square limit
of \code{wave} shown in \link{Figure 2.9} by working at each of the levels
described above.  In particular:

\begin{enumerate}[a.]

\item
Add some segments to the primitive \code{wave} painter of \link{Exercise 2.49}
(to add a smile, for example).

\item
Change the pattern constructed by \code{corner\-/split} (for example, by using
only one copy of the \code{up\-/split} and \code{right\-/split} images instead of
two).

\item
Modify the version of \code{square\-/limit} that uses \code{square\-/of\-/four} so as
to assemble the corners in a different pattern.  (For example, you might make
the big Mr. Rogers look outward from each corner of the square.)

\end{enumerate}
\end{quote}

\section{Symbolic Data}
\label{Section 2.3}

All the compound data objects we have used so far were constructed ultimately
from numbers.  In this section we extend the representational capability of our
language by introducing the ability to work with arbitrary symbols as data.



\subsection{Quotation}
\label{Section 2.3.1}

If we can form compound data using symbols, we can have lists such as

\begin{scheme}
(a b c d)
(23 45 17)
((Norah 12) (Molly 9) (Anna 7) (Lauren 6) (Charlotte 4))
\end{scheme}

\noindent
Lists containing symbols can look just like the expressions of our
language:

\begin{scheme}
(* (+ 23 45) 
   (+ x 9))
(define (fact n) 
  (if (= n 1) 1 (* n (fact (- n 1)))))
\end{scheme}

\noindent
In order to manipulate symbols we need a new element in our language: the
ability to \newterm{quote} a data object.  Suppose we want to construct the
list \code{(a b)}.  We can't accomplish this with \code{(list a b)}, because
this expression constructs a list of the \newterm{values} of \code{a} and
\code{b} rather than the symbols themselves.  This issue is well known in the
context of natural languages, where words and sentences may be regarded either
as semantic entities or as character strings (syntactic entities).  The common
practice in natural languages is to use quotation marks to indicate that a word
or a sentence is to be treated literally as a string of characters.  For
instance, the first letter of ``John'' is clearly ``J.''  If we tell somebody
``say your name aloud,'' we expect to hear that person's name.  However, if we
tell somebody ``say `your name' aloud,'' we expect to hear the words ``your
name.''  Note that we are forced to nest quotation marks to describe what
somebody else might say.\footnote{Allowing quotation in a language wreaks havoc
with the ability to reason about the language in simple terms, because it
destroys the notion that equals can be substituted for equals.  For example,
three is one plus two, but the word ``three'' is not the phrase ``one plus
two.''  Quotation is powerful because it gives us a way to build expressions
that manipulate other expressions (as we will see when we write an interpreter
in \link{Chapter 4}). But allowing statements in a language that talk about
other statements in that language makes it very difficult to maintain any
coherent principle of what ``equals can be substituted for equals'' should
mean.  For example, if we know that the evening star is the morning star, then
from the statement ``the evening star is Venus'' we can deduce ``the morning
star is Venus.''  However, given that ``John knows that the evening star is
Venus'' we cannot infer that ``John knows that the morning star is Venus.''}

We can follow this same practice to identify lists and symbols that are to be
treated as data objects rather than as expressions to be evaluated.  However,
our format for quoting differs from that of natural languages in that we place
a quotation mark (traditionally, the single quote symbol \code{'}) only at the
beginning of the object to be quoted.  We can get away with this in Scheme
syntax because we rely on blanks and parentheses to delimit objects.  Thus, the
meaning of the single quote character is to quote the next object.\footnote{The
single quote is different from the double quote we have been using to enclose
character strings to be printed.  Whereas the single quote can be used to
denote lists or symbols, the double quote is used only with character strings.
In this book, the only use for character strings is as items to be printed.}

Now we can distinguish between symbols and their values:

\begin{scheme}
(define a 1)
(define b 2)
(list a b)
~\textit{(1 2)}~
(list 'a 'b)
~\textit{(a b)}~
(list 'a b)
~\textit{(a 2)}~
\end{scheme}

\noindent
Quotation also allows us to type in compound objects, using the conventional
printed representation for lists:\footnote{Strictly, our use of the quotation
mark violates the general rule that all compound expressions in our language
should be delimited by parentheses and look like lists.  We can recover this
consistency by introducing a special form \code{quote}, which serves the same
purpose as the quotation mark.  Thus, we would type \code{(quote a)} instead of
\code{'a}, and we would type \code{(quote (a b c))} instead of \code{'(a b c)}.
This is precisely how the interpreter works.  The quotation mark is just a
single-character abbreviation for wrapping the next complete expression with
\code{quote} to form \( \hbox{\ttfamily(quote}\;\langle\kern0.06em\hbox{\ttfamily\slshape expression}\kern0.08em\rangle\hbox{\ttfamily)} \).  This is
important because it maintains the principle that any expression seen by the
interpreter can be manipulated as a data object.  For instance, we could
construct the expression \code{(car '(a b c))}, which is the same as \code{(car
(quote (a b c)))}, by evaluating \code{(list 'car (list 'quote '(a b c)))}.}

\begin{scheme}
(car '(a b c))
~\textit{a}~
(cdr '(a b c))
~\textit{(b c)}~
\end{scheme}

\noindent
In keeping with this, we can obtain the empty list by evaluating \code{'()},
and thus dispense with the variable \code{nil}.

One additional primitive used in manipulating symbols is \code{eq?}, which
takes two symbols as arguments and tests whether they are the same.\footnote{We
can consider two symbols to be ``the same'' if they consist of the same
characters in the same order.  Such a definition skirts a deep issue that we
are not yet ready to address: the meaning of ``sameness'' in a programming
language.  We will return to this in \link{Chapter 3} (\link{Section 3.1.3}).}
Using \code{eq?}, we can implement a useful procedure called \code{memq}.  This
takes two arguments, a symbol and a list.  If the symbol is not contained in
the list (i.e., is not \code{eq?} to any item in the list), then \code{memq}
returns false.  Otherwise, it returns the sublist of the list beginning with
the first occurrence of the symbol:

\begin{scheme}
(define (memq item x)
  (cond ((null? x) false)
        ((eq? item (car x)) x)
        (else (memq item (cdr x)))))
\end{scheme}

\noindent
For example, the value of

\begin{scheme}
(memq 'apple '(pear banana prune))
\end{scheme}

\noindent
is false, whereas the value of

\begin{scheme}
(memq 'apple '(x (apple sauce) y apple pear))
\end{scheme}

\noindent
is \code{(apple pear)}.

\begin{quote}
\heading{\phantomsection\label{Exercise 2.53}Exercise 2.53:} What would the interpreter print
in response to evaluating each of the following expressions?

\begin{scheme}
(list 'a 'b 'c)
(list (list 'george))
(cdr '((x1 x2) (y1 y2)))
(cadr '((x1 x2) (y1 y2)))
(pair? (car '(a short list)))
(memq 'red '((red shoes) (blue socks)))
(memq 'red '(red shoes blue socks))
\end{scheme}
\end{quote}

\begin{quote}
\heading{\phantomsection\label{Exercise 2.54}Exercise 2.54:} Two lists are said to be
\code{equal?} if they contain equal elements arranged in the same order.  For
example,

\begin{scheme}
(equal? '(this is a list) '(this is a list))
\end{scheme}

\noindent
is true, but

\begin{scheme}
(equal? '(this is a list) '(this (is a) list))
\end{scheme}

\noindent
is false.  To be more precise, we can define \code{equal?}  recursively in
terms of the basic \code{eq?} equality of symbols by saying that \code{a} and
\code{b} are \code{equal?} if they are both symbols and the symbols are
\code{eq?}, or if they are both lists such that \code{(car a)} is \code{equal?}
to \code{(car b)} and \code{(cdr a)} is \code{equal?} to \code{(cdr b)}.  Using
this idea, implement \code{equal?} as a procedure.\footnote{In practice,
programmers use \code{equal?} to compare lists that contain numbers as well as
symbols.  Numbers are not considered to be symbols.  The question of whether
two numerically equal numbers (as tested by \code{=}) are also \code{eq?} is
highly implementation-dependent.  A better definition of \code{equal?} (such as
the one that comes as a primitive in Scheme) would also stipulate that if
\code{a} and \code{b} are both numbers, then \code{a} and \code{b} are
\code{equal?} if they are numerically equal.}
\end{quote}

\begin{quote}
\heading{\phantomsection\label{Exercise 2.55}Exercise 2.55:} Eva Lu Ator types to the
interpreter the expression

\begin{scheme}
(car ''abracadabra)
\end{scheme}

To her surprise, the interpreter prints back \code{quote}.  Explain.
\end{quote}

\subsection{Example: Symbolic Differentiation}
\label{Section 2.3.2}

As an illustration of symbol manipulation and a further illustration of data
abstraction, consider the design of a procedure that performs symbolic
differentiation of algebraic expressions.  We would like the procedure to take
as arguments an algebraic expression and a variable and to return the
derivative of the expression with respect to the variable.  For example, if the
arguments to the procedure are \( ax^2 + bx + c \) and \( x \), the
procedure should return \( 2ax + b \).  Symbolic differentiation is of
special historical significance in Lisp.  It was one of the motivating examples
behind the development of a computer language for symbol manipulation.
Furthermore, it marked the beginning of the line of research that led to the
development of powerful systems for symbolic mathematical work, which are
currently being used by a growing number of applied mathematicians and
physicists.

In developing the symbolic-differentiation program, we will follow the same
strategy of data abstraction that we followed in developing the rational-number
system of \link{Section 2.1.1}.  That is, we will first define a differentiation
algorithm that operates on abstract objects such as ``sums,'' ``products,'' and
``variables'' without worrying about how these are to be represented.  Only
afterward will we address the representation problem.

\subsubsection*{The differentiation program with abstract data}

In order to keep things simple, we will consider a very simple
symbolic-differentiation program that handles expressions that are built up
using only the operations of addition and multiplication with two arguments.
Differentiation of any such expression can be carried out by applying the
following reduction rules:
\begin{comment}

\begin{example}
dc
-- = 0  for c a constant, or a variable different from x
dx

dx
-- = 1
dx

d(u + v)   du   dv
-------- = -- + --
   dx      dx   dx

d(uv)     / dv \     / du \
----- = u | -- | + v | -- |
 dx       \ dx /     \ dx /
\end{example}

\end{comment}
\begin{displaymath}
{{\it dc} \over {\it dx}} 		= 0, 
	\quad {\rm for\ } c\ {\rm a\ constant\ or\ a\ variable\ different\ from\ } x, 
\end{displaymath}
\begin{displaymath}
{{\it dx} \over {\it dx}} 		= 1, 
\end{displaymath}
\begin{displaymath}
{{\it d\,(u + v\,)} \over {\it dx}} 	= {{\it du} \over {\it dx}} + {{\it dv} \over {\it dx}}, 
\end{displaymath}
\begin{displaymath}
{{\it d\,(uv\,)} \over {\it dx}} 	= u {{\it dv} \over {\it dx}} + v {{\it du} \over {\it dx}}. 
\end{displaymath}
\noindent
Observe that the latter two rules are recursive in nature.  That is, to obtain
the derivative of a sum we first find the derivatives of the terms and add
them.  Each of the terms may in turn be an expression that needs to be
decomposed.  Decomposing into smaller and smaller pieces will eventually
produce pieces that are either constants or variables, whose derivatives will
be either 0 or 1.

To embody these rules in a procedure we indulge in a little wishful thinking,
as we did in designing the rational-number implementation.  If we had a means
for representing algebraic expressions, we should be able to tell whether an
expression is a sum, a product, a constant, or a variable.  We should be able
to extract the parts of an expression.  For a sum, for example we want to be
able to extract the addend (first term) and the augend (second term).  We
should also be able to construct expressions from parts.  Let us assume that we
already have procedures to implement the following selectors, constructors, and
predicates:

\begin{scheme}
(variable? e)            ~\textrm{Is \code{e} a variable?}~
(same-variable? v1 v2)   ~\textrm{Are \code{v1} and \code{v2} the same variable?}~
(sum? e)                 ~\textrm{Is \code{e} a sum?}~
(addend e)               ~\textrm{Addend of the sum \code{e}.}~
(augend e)               ~\textrm{Augend of the sum \code{e}.}~
(make-sum a1 a2)         ~\textrm{Construct the sum of \code{a1} and \code{a2}.}~
(product? e)             ~\textrm{Is \code{e} a product?}~
(multiplier e)           ~\textrm{Multiplier of the product \code{e}.}~
(multiplicand e)         ~\textrm{Multiplicand of the product \code{e}.}~
(make-product m1 m2)     ~\textrm{Construct the product of \code{m1} and \code{m2}.}~
\end{scheme}

\noindent
Using these, and the primitive predicate \code{number?}, which identifies
numbers, we can express the differentiation rules as the following procedure:

\begin{scheme}
(define (deriv exp var)
  (cond ((number? exp) 0)
        ((variable? exp) (if (same-variable? exp var) 1 0))
        ((sum? exp) (make-sum (deriv (addend exp) var)
                              (deriv (augend exp) var)))
        ((product? exp)
         (make-sum
           (make-product (multiplier exp)
                         (deriv (multiplicand exp) var))
           (make-product (deriv (multiplier exp) var)
                         (multiplicand exp))))
        (else
         (error "unknown expression type: DERIV" exp))))
\end{scheme}

\noindent
This \code{deriv} procedure incorporates the complete differentiation
algorithm.  Since it is expressed in terms of abstract data, it will work no
matter how we choose to represent algebraic expressions, as long as we design a
proper set of selectors and constructors.  This is the issue we must address
next.

\subsubsection*{Representing algebraic expressions}

We can imagine many ways to use list structure to represent algebraic
expressions.  For example, we could use lists of symbols that mirror the usual
algebraic notation, representing \( ax + b \) as the list \code{(a * x +
b)}.  However, one especially straightforward choice is to use the same
parenthesized prefix notation that Lisp uses for combinations; that is, to
represent \( ax + b \) as \code{(+ (* a x) b)}.  Then our data
representation for the differentiation problem is as follows:

\begin{itemize}

\item
The variables are symbols.  They are identified by the primitive predicate
\code{symbol?}:

\begin{scheme}
(define (variable? x) (symbol? x))
\end{scheme}

\item
Two variables are the same if the symbols representing them are \code{eq?}:

\begin{scheme}
(define (same-variable? v1 v2)
  (and (variable? v1) (variable? v2) (eq? v1 v2)))
\end{scheme}

\item
Sums and products are constructed as lists:

\begin{scheme}
(define (make-sum a1 a2) (list '+ a1 a2))
(define (make-product m1 m2) (list '* m1 m2))
\end{scheme}

\item
A sum is a list whose first element is the symbol \code{+}:

\begin{scheme}
(define (sum? x) (and (pair? x) (eq? (car x) '+)))
\end{scheme}

\item
The addend is the second item of the sum list:

\begin{scheme}
(define (addend s) (cadr s))
\end{scheme}

\item
The augend is the third item of the sum list:

\begin{scheme}
(define (augend s) (caddr s))
\end{scheme}

\item
A product is a list whose first element is the symbol \code{*}:

\begin{scheme}
(define (product? x) (and (pair? x) (eq? (car x) '*)))
\end{scheme}

\item
The multiplier is the second item of the product list:

\begin{scheme}
(define (multiplier p) (cadr p))
\end{scheme}

\item
The multiplicand is the third item of the product list:

\begin{scheme}
(define (multiplicand p) (caddr p))
\end{scheme}

\end{itemize}

\noindent
Thus, we need only combine these with the algorithm as embodied by \code{deriv}
in order to have a working symbolic-differentiation program.  Let us look at
some examples of its behavior:

\begin{scheme}
(deriv '(+ x 3) 'x)
~\textit{(+ 1 0)}~
(deriv '(* x y) 'x)
~\textit{(+ (* x 0) (* 1 y))}~
(deriv '(* (* x y) (+ x 3)) 'x)
~\textit{(+ (* (* x y) (+ 1 0))}~
   ~\textit{(* (+ (* x 0) (* 1 y))}~
      ~\textit{(+  x 3)))}~
\end{scheme}

\noindent
The program produces answers that are correct; however, they are unsimplified.
It is true that
\begin{comment}

\begin{example}
d(xy)
----- = x * 0 + 1 * y
 dx
\end{example}

\end{comment}
\begin{displaymath}
 {{\it d\,(xy\,)} \over {\it dx}} = x \cdot 0 + 1 \cdot y, 
\end{displaymath}
\noindent
but we would like the program to know that \( x \cdot 0 = 0 \), \( 1 \cdot y = y \),
and \( 0 + y = y \).  The answer for the second example should have been
simply \code{y}.  As the third example shows, this becomes a serious issue when
the expressions are complex.

Our difficulty is much like the one we encountered with the rational-number
implementation: we haven't reduced answers to simplest form.  To accomplish the
rational-number reduction, we needed to change only the constructors and the
selectors of the implementation.  We can adopt a similar strategy here.  We
won't change \code{deriv} at all.  Instead, we will change \code{make\-/sum} so
that if both summands are numbers, \code{make\-/sum} will add them and return
their sum.  Also, if one of the summands is 0, then \code{make\-/sum} will return
the other summand.

\begin{scheme}
(define (make-sum a1 a2)
  (cond ((=number? a1 0) a2)
        ((=number? a2 0) a1)
        ((and (number? a1) (number? a2))
         (+ a1 a2))
        (else (list '+ a1 a2))))
\end{scheme}

\noindent
This uses the procedure \code{=number?}, which checks whether an expression is
equal to a given number:

\begin{scheme}
(define (=number? exp num) (and (number? exp) (= exp num)))
\end{scheme}

\noindent
Similarly, we will change \code{make\-/product} to build in the rules that 0
times anything is 0 and 1 times anything is the thing itself:

\begin{scheme}
(define (make-product m1 m2)
  (cond ((or (=number? m1 0) (=number? m2 0)) 0)
        ((=number? m1 1) m2)
        ((=number? m2 1) m1)
        ((and (number? m1) (number? m2)) (* m1 m2))
        (else (list '* m1 m2))))
\end{scheme}

\noindent
Here is how this version works on our three examples:

\begin{scheme}
(deriv '(+ x 3) 'x)
~\textit{1}~
(deriv '(* x y) 'x)
~\textit{y}~
(deriv '(* (* x y) (+ x 3)) 'x)
~\textit{(+ (* x y) (* y (+ x 3)))}~
\end{scheme}

\noindent
Although this is quite an improvement, the third example shows that there is
still a long way to go before we get a program that puts expressions into a
form that we might agree is ``simplest.''  The problem of algebraic
simplification is complex because, among other reasons, a form that may be
simplest for one purpose may not be for another.

\begin{quote}
\heading{\phantomsection\label{Exercise 2.56}Exercise 2.56:} Show how to extend the basic
differentiator to handle more kinds of expressions.  For instance, implement
the differentiation rule
\begin{comment}

\begin{example}
d(u^n)            du
------ = nu^(n-1) --  
  dx              dx
\end{example}

\end{comment}
\begin{displaymath}
 {{\it d\,(u^n\,)} \over {\it dx}} = nu^{n-1} {{\it du} \over {\it dx}} 
\end{displaymath}
\noindent
by adding a new clause to the \code{deriv} program and defining appropriate
procedures \code{exponentiation?}, \code{base}, \code{exponent}, and
\code{make\-/exponentiation}.  (You may use the symbol \code{**} to denote
exponentiation.)  Build in the rules that anything raised to the power 0 is 1
and anything raised to the power 1 is the thing itself.
\end{quote}

\begin{quote}
\heading{\phantomsection\label{Exercise 2.57}Exercise 2.57:} Extend the differentiation
program to handle sums and products of arbitrary numbers of (two or more)
terms.  Then the last example above could be expressed as

\begin{scheme}
(deriv '(* x y (+ x 3)) 'x)
\end{scheme}

Try to do this by changing only the representation for sums and products,
without changing the \code{deriv} procedure at all.  For example, the
\code{addend} of a sum would be the first term, and the \code{augend} would be
the sum of the rest of the terms.
\end{quote}

\begin{quote}
\heading{\phantomsection\label{Exercise 2.58}Exercise 2.58:} Suppose we want to modify the
differentiation program so that it works with ordinary mathematical notation,
in which \code{+} and \code{*} are infix rather than prefix operators.  Since
the differentiation program is defined in terms of abstract data, we can modify
it to work with different representations of expressions solely by changing the
predicates, selectors, and constructors that define the representation of the
algebraic expressions on which the differentiator is to operate.

\begin{enumerate}[a.]

\item
Show how to do this in order to differentiate algebraic expressions presented
in infix form, such as \code{(x + (3 * (x + (y + 2))))}.  To simplify the task,
assume that \code{+} and \code{*} always take two arguments and that
expressions are fully parenthesized.

\item
The problem becomes substantially harder if we allow standard algebraic
notation, such as \code{(x + 3 * (x + y + 2))}, which drops unnecessary
parentheses and assumes that multiplication is done before addition.  Can you
design appropriate predicates, selectors, and constructors for this notation
such that our derivative program still works?

\end{enumerate}
\end{quote}

\subsection{Example: Representing Sets}
\label{Section 2.3.3}

In the previous examples we built representations for two kinds of compound
data objects: rational numbers and algebraic expressions.  In one of these
examples we had the choice of simplifying (reducing) the expressions at either
construction time or selection time, but other than that the choice of a
representation for these structures in terms of lists was straightforward. When
we turn to the representation of sets, the choice of a representation is not so
obvious.  Indeed, there are a number of possible representations, and they
differ significantly from one another in several ways.

Informally, a set is simply a collection of distinct objects.  To give a more
precise definition we can employ the method of data abstraction.  That is, we
define ``set'' by specifying the operations that are to be used on sets.  These
are \code{union\-/set}, \code{intersection\-/set}, \code{element\-/of\-/set?}, and
\code{adjoin\-/set}.  \code{Element\-/of\-/set?} is a predicate that determines
whether a given element is a member of a set.  \code{Adjoin\-/set} takes an
object and a set as arguments and returns a set that contains the elements of
the original set and also the adjoined element.  \code{Union\-/set} computes the
union of two sets, which is the set containing each element that appears in
either argument.  \code{Intersection\-/set} computes the intersection of two
sets, which is the set containing only elements that appear in both arguments.
From the viewpoint of data abstraction, we are free to design any
representation that implements these operations in a way consistent with the
interpretations given above.\footnote{If we want to be more formal, we can
specify ``consistent with the interpretations given above'' to mean that the
operations satisfy a collection of rules such as these:

\noindent
\( \bullet \) For any set \code{S} and any object \code{x},
\code{(element\-/of\-/set? x (adjoin\-/set x S))}
is true (informally: ``Adjoining an object to a
set produces a set that contains the object'').

\noindent
\( \bullet \) For any sets \code{S} and \code{T} and any object \code{x},
\code{(element\-/of\-/set? x (union\-/set S T))}
is equal to
\code{(or (element\-/of\-/set? x S) (element\-/of\-/set? x T))}
(informally: ``The elements of \code{(union S T)} are the elements that
are in \code{S} or in \code{T}'').

\noindent
\( \bullet \) For any object \code{x},
\code{(element\-/of\-/set? x '())}
is false (informally: ``No object is an element of the empty set'').
}

\subsubsection*{Sets as unordered lists}

One way to represent a set is as a list of its elements in which no element
appears more than once.  The empty set is represented by the empty list.  In
this representation, \code{element\-/of\-/set?} is similar to the procedure
\code{memq} of \link{Section 2.3.1}.  It uses \code{equal?}  instead of
\code{eq?} so that the set elements need not be symbols:

\begin{scheme}
(define (element-of-set? x set)
  (cond ((null? set) false)
        ((equal? x (car set)) true)
        (else (element-of-set? x (cdr set)))))
\end{scheme}

\noindent
Using this, we can write \code{adjoin\-/set}.  If the object to be adjoined is
already in the set, we just return the set.  Otherwise, we use \code{cons} to
add the object to the list that represents the set:

\begin{scheme}
(define (adjoin-set x set)
  (if (element-of-set? x set)
      set
      (cons x set)))
\end{scheme}

\noindent
For \code{intersection\-/set} we can use a recursive strategy.  If we know how to
form the intersection of \code{set2} and the \code{cdr} of \code{set1}, we only
need to decide whether to include the \code{car} of \code{set1} in this.  But
this depends on whether \code{(car set1)} is also in \code{set2}.  Here is the
resulting procedure:

\enlargethispage{\baselineskip}

\begin{scheme}
(define (intersection-set set1 set2)
  (cond ((or (null? set1) (null? set2)) '())
        ((element-of-set? (car set1) set2)
         (cons (car set1) (intersection-set (cdr set1) set2)))
        (else (intersection-set (cdr set1) set2))))
\end{scheme}

\noindent
In designing a representation, one of the issues we should be concerned with is
efficiency.  Consider the number of steps required by our set operations.
Since they all use \code{element\-/of\-/set?}, the speed of this operation has a
major impact on the efficiency of the set implementation as a whole.  Now, in
order to check whether an object is a member of a set, \code{element\-/of\-/set?}
may have to scan the entire set. (In the worst case, the object turns out not
to be in the set.)  Hence, if the set has \( n \) elements,
\code{element\-/of\-/set?}  might take up to \( n \) steps.  Thus, the number of
steps required grows as \( \Theta(n) \).  The number of steps required by
\code{adjoin\-/set}, which uses this operation, also grows as \( \Theta(n) \).
For \code{intersection\-/set}, which does an \code{element\-/of\-/set?} check for
each element of \code{set1}, the number of steps required grows as the product
of the sizes of the sets involved, or \( \Theta(n^2) \) for two sets of size
\( n \).  The same will be true of \code{union\-/set}.

\begin{quote}
\heading{\phantomsection\label{Exercise 2.59}Exercise 2.59:} Implement the \code{union\-/set}
operation for the unordered-list representation of sets.
\end{quote}

\begin{quote}
\heading{\phantomsection\label{Exercise 2.60}Exercise 2.60:} We specified that a set would be
represented as a list with no duplicates.  Now suppose we allow duplicates.
For instance, the set \( \{1, 2, 3\} \) could be represented as the list \code{(2 3 2 1
3 2 2)}.  Design procedures \code{element\-/of\-/set?}, \code{adjoin\-/set},
\code{union\-/set}, and \code{intersection\-/set} that operate on this
representation.  How does the efficiency of each compare with the corresponding
procedure for the non-duplicate representation?  Are there applications for
which you would use this representation in preference to the non-duplicate one?
\end{quote}

\subsubsection*{Sets as ordered lists}

One way to speed up our set operations is to change the representation so that
the set elements are listed in increasing order.  To do this, we need some way
to compare two objects so that we can say which is bigger.  For example, we
could compare symbols lexicographically, or we could agree on some method for
assigning a unique number to an object and then compare the elements by
comparing the corresponding numbers.  To keep our discussion simple, we will
consider only the case where the set elements are numbers, so that we can
compare elements using \code{>} and \code{<}.  We will represent a set of
numbers by listing its elements in increasing order.  Whereas our first
representation above allowed us to represent the set \( \{1, 3, 6, 10\} \) by listing
the elements in any order, our new representation allows only the list \code{(1
3 6 10)}.

One advantage of ordering shows up in \code{element\-/of\-/set?}: In checking for
the presence of an item, we no longer have to scan the entire set.  If we reach
a set element that is larger than the item we are looking for, then we know
that the item is not in the set:

\begin{scheme}
(define (element-of-set? x set)
  (cond ((null? set) false)
        ((= x (car set)) true)
        ((< x (car set)) false)
        (else (element-of-set? x (cdr set)))))
\end{scheme}

\noindent
How many steps does this save?  In the worst case, the item we are looking for
may be the largest one in the set, so the number of steps is the same as for
the unordered representation.  On the other hand, if we search for items of
many different sizes we can expect that sometimes we will be able to stop
searching at a point near the beginning of the list and that other times we
will still need to examine most of the list.  On the average we should expect
to have to examine about half of the items in the set.  Thus, the average
number of steps required will be about \( n / 2 \).  This is still
\( \Theta(n) \) growth, but it does save us, on the average, a factor of 2
in number of steps over the previous implementation.

We obtain a more impressive speedup with \code{intersection\-/set}.  In the
unordered representation this operation required \( \Theta(n^2) \) steps,
because we performed a complete scan of \code{set2} for each element of
\code{set1}.  But with the ordered representation, we can use a more clever
method.  Begin by comparing the initial elements, \code{x1} and \code{x2}, of
the two sets.  If \code{x1} equals \code{x2}, then that gives an element of the
intersection, and the rest of the intersection is the intersection of the
\code{cdr}-s of the two sets.  Suppose, however, that \code{x1} is less than
\code{x2}.  Since \code{x2} is the smallest element in \code{set2}, we can
immediately conclude that \code{x1} cannot appear anywhere in \code{set2} and
hence is not in the intersection.  Hence, the intersection is equal to the
intersection of \code{set2} with the \code{cdr} of \code{set1}.  Similarly, if
\code{x2} is less than \code{x1}, then the intersection is given by the
intersection of \code{set1} with the \code{cdr} of \code{set2}.  Here is the
procedure:

\begin{scheme}
(define (intersection-set set1 set2)
  (if (or (null? set1) (null? set2))
      '()
      (let ((x1 (car set1)) (x2 (car set2)))
        (cond ((= x1 x2)
               (cons x1 (intersection-set (cdr set1)
                                          (cdr set2))))
              ((< x1 x2)
               (intersection-set (cdr set1) set2))
              ((< x2 x1)
               (intersection-set set1 (cdr set2)))))))
\end{scheme}

\noindent
To estimate the number of steps required by this process, observe that at each
step we reduce the intersection problem to computing intersections of smaller
sets---removing the first element from \code{set1} or \code{set2} or both.
Thus, the number of steps required is at most the sum of the sizes of
\code{set1} and \code{set2}, rather than the product of the sizes as with the
unordered representation.  This is \( \Theta(n) \) growth rather than
\( \Theta(n^2) \)---a considerable speedup, even for sets of moderate size.

\begin{quote}
\heading{\phantomsection\label{Exercise 2.61}Exercise 2.61:} Give an implementation of
\code{adjoin\-/set} using the ordered representation.  By analogy with
\code{element\-/of\-/set?} show how to take advantage of the ordering to produce a
procedure that requires on the average about half as many steps as with the
unordered representation.
\end{quote}

\begin{quote}
\heading{\phantomsection\label{Exercise 2.62}Exercise 2.62:} Give a \( \Theta(n) \)
implementation of \code{union\-/set} for sets represented as ordered lists.
\end{quote}

\subsubsection*{Sets as binary trees}

We can do better than the ordered-list representation by arranging the set
elements in the form of a tree.  Each node of the tree holds one element of the
set, called the ``entry'' at that node, and a link to each of two other
(possibly empty) nodes.  The ``left'' link points to elements smaller than the
one at the node, and the ``right'' link to elements greater than the one at the
node.  \link{Figure 2.16} shows some trees that represent the set
\( \{1, 3, 5, 7, 9, 11\} \).  The same set may be represented by a tree in a number of
different ways.  The only thing we require for a valid representation is that
all elements in the left subtree be smaller than the node entry and that all
elements in the right subtree be larger.

\begin{figure}[tb]
\phantomsection\label{Figure 2.16}
\centering
\begin{comment}
\heading{Figure 2.16:} Various binary trees that represent the set \( \{1, 3, 5, 7, 9, 11\} \).

\begin{example}
   7          3             5
   /\         /\            /\
  3  9       1  7          3  9
 /\   \         /\        /   /\
1  5  11       5  9      1   7  11
                   \
                   11
\end{example}
\end{comment}
\includegraphics[width=70mm]{fig/chap2/Fig2.16b.pdf}
\begin{quote}
\heading{Figure 2.16:} Various binary trees that represent the set \( \{1, 3, 5, 7, 9, 11\} \).
\end{quote}
\end{figure}

The advantage of the tree representation is this: Suppose we want to check
whether a number \( x \) is contained in a set.  We begin by comparing \( x \) with
the entry in the top node.  If \( x \) is less than this, we know that we need
only search the left subtree; if \( x \) is greater, we need only search the
right subtree.  Now, if the tree is ``balanced,'' each of these subtrees will
be about half the size of the original.  Thus, in one step we have reduced the
problem of searching a tree of size \( n \) to searching a tree of size \( n / 2 \).
Since the size of the tree is halved at each step, we should expect that the
number of steps needed to search a tree of size \( n \) grows as
\( \Theta(\log n) \).\footnote{Halving the size of the problem at each
step is the distinguishing characteristic of logarithmic growth, as we saw with
the fast-exponentiation algorithm of \link{Section 1.2.4} and the half-interval
search method of \link{Section 1.3.3}.} For large sets, this will be a
significant speedup over the previous representations.

We can represent trees by using lists.  Each node will be a list of three
items: the entry at the node, the left subtree, and the right subtree.  A left
or a right subtree of the empty list will indicate that there is no subtree
connected there.  We can describe this representation by the following
procedures:\footnote{We are representing sets in terms of trees, and trees in
terms of lists---in effect, a data abstraction built upon a data abstraction.
We can regard the procedures \code{entry}, \code{left\-/branch},
\code{right\-/branch}, and \code{make\-/tree} as a way of isolating the abstraction
of a ``binary tree'' from the particular way we might wish to represent such a
tree in terms of list structure.}

\begin{scheme}
(define (entry tree) (car tree))
(define (left-branch tree) (cadr tree))
(define (right-branch tree) (caddr tree))
(define (make-tree entry left right)
  (list entry left right))
\end{scheme}

\noindent
Now we can write the \code{element\-/of\-/set?} procedure using the strategy
described above:

\begin{scheme}
(define (element-of-set? x set)
  (cond ((null? set) false)
        ((= x (entry set)) true)
        ((< x (entry set))
         (element-of-set? x (left-branch set)))
        ((> x (entry set))
         (element-of-set? x (right-branch set)))))
\end{scheme}

\noindent
Adjoining an item to a set is implemented similarly and also requires
\( \Theta(\log n) \) steps.  To adjoin an item \code{x}, we compare
\code{x} with the node entry to determine whether \code{x} should be added to
the right or to the left branch, and having adjoined \code{x} to the
appropriate branch we piece this newly constructed branch together with the
original entry and the other branch.  If \code{x} is equal to the entry, we
just return the node.  If we are asked to adjoin \code{x} to an empty tree, we
generate a tree that has \code{x} as the entry and empty right and left
branches.  Here is the procedure:

\begin{scheme}
(define (adjoin-set x set)
  (cond ((null? set) (make-tree x '() '()))
        ((= x (entry set)) set)
        ((< x (entry set))
         (make-tree (entry set)
                    (adjoin-set x (left-branch set))
                    (right-branch set)))
        ((> x (entry set))
         (make-tree (entry set) (left-branch set)
                    (adjoin-set x (right-branch set))))))
\end{scheme}

\noindent
The above claim that searching the tree can be performed in a logarithmic
number of steps rests on the assumption that the tree is ``balanced,'' i.e.,
that the left and the right subtree of every tree have approximately the same
number of elements, so that each subtree contains about half the elements of
its parent.  But how can we be certain that the trees we construct will be
balanced?  Even if we start with a balanced tree, adding elements with
\code{adjoin\-/set} may produce an unbalanced result.  Since the position of a
newly adjoined element depends on how the element compares with the items
already in the set, we can expect that if we add elements ``randomly'' the tree
will tend to be balanced on the average.  But this is not a guarantee.  For
example, if we start with an empty set and adjoin the numbers 1 through 7 in
sequence we end up with the highly unbalanced tree shown in \link{Figure 2.17}.
In this tree all the left subtrees are empty, so it has no advantage over a
simple ordered list.  One way to solve this problem is to define an operation
that transforms an arbitrary tree into a balanced tree with the same elements.
Then we can perform this transformation after every few \code{adjoin\-/set}
operations to keep our set in balance.  There are also other ways to solve this
problem, most of which involve designing new data structures for which
searching and insertion both can be done in \( \Theta(\log n) \)
steps.\footnote{Examples of such structures include \newterm{B-trees} and
\newterm{red-black trees}.  There is a large literature on data structures
devoted to this problem.  See \link{Cormen et al. 1990}.}

\begin{figure}[tb]
\phantomsection\label{Figure 2.17}
\centering
\begin{comment}
\heading{Figure 2.17:} Unbalanced tree produced by adjoining 1 through 7 in sequence.

\begin{example}
1
 \
  2
   \
    4
     \
      5
       \
        6
         \
          7
\end{example}
\end{comment}
\includegraphics[width=40mm]{fig/chap2/Fig2.17a.pdf}
\begin{quote}
\heading{Figure 2.17:} Unbalanced tree produced by adjoining 1 through 7 in sequence.
\end{quote}
\end{figure}

\begin{quote}
\heading{\phantomsection\label{Exercise 2.63}Exercise 2.63:} Each of the following two
procedures converts a binary tree to a list.

\begin{scheme}
(define (tree->list-1 tree)
  (if (null? tree)
      '()
      (append (tree->list-1 (left-branch tree))
              (cons (entry tree)
                    (tree->list-1 
                      (right-branch tree))))))
(define (tree->list-2 tree)
  (define (copy-to-list tree result-list)
    (if (null? tree)
        result-list
        (copy-to-list (left-branch tree)
                      (cons (entry tree)
                            (copy-to-list 
                              (right-branch tree)
                              result-list)))))
  (copy-to-list tree '()))
\end{scheme}

\begin{enumerate}[a.]

\item
Do the two procedures produce the same result for every tree?  If not, how do
the results differ?  What lists do the two procedures produce for the trees in
\link{Figure 2.16}?

\item
Do the two procedures have the same order of growth in the number of steps
required to convert a balanced tree with \( n \) elements to a list?  If not,
which one grows more slowly?

\end{enumerate}
\end{quote}

\begin{quote}
\heading{\phantomsection\label{Exercise 2.64}Exercise 2.64:} The following procedure
\code{list\-/>tree} converts an ordered list to a balanced binary tree.  The
helper procedure \code{partial\-/tree} takes as arguments an integer \( n \) and
list of at least \( n \) elements and constructs a balanced tree containing the
first \( n \) elements of the list.  The result returned by \code{partial\-/tree}
is a pair (formed with \code{cons}) whose \code{car} is the constructed tree
and whose \code{cdr} is the list of elements not included in the tree.

\begin{scheme}
(define (list->tree elements)
  (car (partial-tree elements (length elements))))
(define (partial-tree elts n)
  (if (= n 0)
      (cons '() elts)
      (let ((left-size (quotient (- n 1) 2)))
        (let ((left-result 
               (partial-tree elts left-size)))
          (let ((left-tree (car left-result))
                (non-left-elts (cdr left-result))
                (right-size (- n (+ left-size 1))))
            (let ((this-entry (car non-left-elts))
                  (right-result 
                   (partial-tree 
                    (cdr non-left-elts)
                    right-size)))
              (let ((right-tree (car right-result))
                    (remaining-elts 
                     (cdr right-result)))
                (cons (make-tree this-entry
                                 left-tree
                                 right-tree)
                      remaining-elts))))))))
\end{scheme}

\begin{enumerate}[a.]

\item
Write a short paragraph explaining as clearly as you can how
\code{partial\-/tree} works.  Draw the tree produced by \code{list\-/>tree} for
the list \code{(1 3 5 7 9 11)}.

\item
What is the order of growth in the number of steps required by
\code{list\-/>tree} to convert a list of \( n \) elements?

\end{enumerate}
\end{quote}

\begin{quote}
\heading{\phantomsection\label{Exercise 2.65}Exercise 2.65:} Use the results of \link{Exercise 2.63} 
and \link{Exercise 2.64} to give \( \Theta(n) \) implementations of
\code{union\-/set} and \code{intersection\-/set} for sets implemented as (balanced)
binary trees.\footnote{\link{Exercise 2.63} through \link{Exercise 2.65} are due
to Paul Hilfinger.}
\end{quote}

\subsubsection*{Sets and information retrieval}

We have examined options for using lists to represent sets and have seen how
the choice of representation for a data object can have a large impact on the
performance of the programs that use the data.  Another reason for
concentrating on sets is that the techniques discussed here appear again and
again in applications involving information retrieval.

Consider a data base containing a large number of individual records, such as
the personnel files for a company or the transactions in an accounting system.
A typical data-management system spends a large amount of time accessing or
modifying the data in the records and therefore requires an efficient method
for accessing records.  This is done by identifying a part of each record to
serve as an identifying \newterm{key}.  A key can be anything that uniquely
identifies the record.  For a personnel file, it might be an employee's ID
number.  For an accounting system, it might be a transaction number.  Whatever
the key is, when we define the record as a data structure we should include a
\code{key} selector procedure that retrieves the key associated with a given
record.

Now we represent the data base as a set of records. To locate the record with a
given key we use a procedure \code{lookup}, which takes as arguments a key and
a data base and which returns the record that has that key, or false if there
is no such record.  \code{Lookup} is implemented in almost the same way as
\code{element\-/of\-/set?}.  For example, if the set of records is implemented as
an unordered list, we could use

\begin{scheme}
(define (lookup given-key set-of-records)
  (cond ((null? set-of-records) false)
        ((equal? given-key (key (car set-of-records)))
         (car set-of-records))
        (else (lookup given-key (cdr set-of-records)))))
\end{scheme}

\noindent
Of course, there are better ways to represent large sets than as unordered
lists.  Information-retrieval systems in which records have to be ``randomly
accessed'' are typically implemented by a tree-based method, such as the
binary-tree representation discussed previously.  In designing such a system
the methodology of data abstraction can be a great help.  The designer can
create an initial implementation using a simple, straightforward representation
such as unordered lists.  This will be unsuitable for the eventual system, but
it can be useful in providing a ``quick and dirty'' data base with which to
test the rest of the system.  Later on, the data representation can be modified
to be more sophisticated.  If the data base is accessed in terms of abstract
selectors and constructors, this change in representation will not require any
changes to the rest of the system.

\begin{quote}
\heading{\phantomsection\label{Exercise 2.66}Exercise 2.66:} Implement the \code{lookup}
procedure for the case where the set of records is structured as a binary tree,
ordered by the numerical values of the keys.
\end{quote}

\subsection{Example: Huffman Encoding Trees}
\label{Section 2.3.4}

This section provides practice in the use of list structure and data
abstraction to manipulate sets and trees.  The application is to methods for
representing data as sequences of ones and zeros (bits).  For example, the
ASCII standard code used to represent text in computers encodes each character
as a sequence of seven bits.  Using seven bits allows us to distinguish \( 2^7 \),
or 128, possible different characters.  In general, if we want to distinguish
\( n \) different symbols, we will need to use \( \log_2\!n \) bits per
symbol.  If all our messages are made up of the eight symbols A, B, C, D, E, F,
G, and H, we can choose a code with three bits per character, for example

\begin{example}
A 000    C 010    E 100    G 110
B 001    D 011    F 101    H 111
\end{example}

\noindent
With this code, the message

\begin{example}
BACADAEAFABBAAAGAH
\end{example}

\noindent
is encoded as the string of 54 bits

\begin{example}
001000010000011000100000101000001001000000000110000111
\end{example}

\noindent
Codes such as ASCII and the A-through-H code above are known as
\newterm{fixed-length} codes, because they represent each symbol in the message
with the same number of bits.  It is sometimes advantageous to use
\newterm{variable-length} codes, in which different symbols may be represented
by different numbers of bits.  For example, Morse code does not use the same
number of dots and dashes for each letter of the alphabet.  In particular, E,
the most frequent letter, is represented by a single dot.  In general, if our
messages are such that some symbols appear very frequently and some very
rarely, we can encode data more efficiently (i.e., using fewer bits per
message) if we assign shorter codes to the frequent symbols.  Consider the
following alternative code for the letters A through H:

\begin{example}
A 0      C 1010    E 1100    G 1110
B 100    D 1011    F 1101    H 1111
\end{example}

\noindent
With this code, the same message as above is encoded as the string

\begin{example}
100010100101101100011010100100000111001111
\end{example}

\noindent
This string contains 42 bits, so it saves more than 20\% in space in comparison
with the fixed-length code shown above.

One of the difficulties of using a variable-length code is knowing when you
have reached the end of a symbol in reading a sequence of zeros and ones.
Morse code solves this problem by using a special \newterm{separator code} (in
this case, a pause) after the sequence of dots and dashes for each letter.
Another solution is to design the code in such a way that no complete code for
any symbol is the beginning (or \newterm{prefix}) of the code for another
symbol.  Such a code is called a \newterm{prefix code}.  In the example above,
A is encoded by 0 and B is encoded by 100, so no other symbol can have a code
that begins with 0 or with 100.

In general, we can attain significant savings if we use variable-length prefix
codes that take advantage of the relative frequencies of the symbols in the
messages to be encoded.  One particular scheme for doing this is called the
Huffman encoding method, after its discoverer, David Huffman.  A Huffman code
can be represented as a binary tree whose leaves are the symbols that are
encoded.  At each non-leaf node of the tree there is a set containing all the
symbols in the leaves that lie below the node.  In addition, each symbol at a
leaf is assigned a weight (which is its relative frequency), and each non-leaf
node contains a weight that is the sum of all the weights of the leaves lying
below it.  The weights are not used in the encoding or the decoding process.
We will see below how they are used to help construct the tree.

\link{Figure 2.18} shows the Huffman tree for the A-through-H code given above.
The weights at the leaves indicate that the tree was designed for messages in
which A appears with relative frequency 8, B with relative frequency 3, and the
other letters each with relative frequency 1.

\begin{figure}[tb]
\phantomsection\label{Figure 2.18}
\centering
\begin{comment}
\heading{Figure 2.18:} A Huffman encoding tree.

\begin{example}
           {A B C D E F G H} 17
                    *
                   / \
                  /   \
                A 8    * {B C D E F G H} 9
            __________/ \_____________
           /                          \
{B C D} 5 *                            * {E F G H} 4
         / \                       ___/ \___
        /   \                     /         \
      B 3    * {C D} 2   {E F} 2 *           * {G H} 2
            / \                 / \         / \
           /   \               /   \       /   \
         C 1   D 1           E 1   F 1   G 1   H 1
\end{example}
\end{comment}
\includegraphics[width=81mm]{fig/chap2/Fig2.18a.pdf}
\par\bigskip
\noindent
\heading{Figure 2.18:} A Huffman encoding tree.
\end{figure}

Given a Huffman tree, we can find the encoding of any symbol by starting at the
root and moving down until we reach the leaf that holds the symbol.  Each time
we move down a left branch we add a 0 to the code, and each time we move down a
right branch we add a 1.  (We decide which branch to follow by testing to see
which branch either is the leaf node for the symbol or contains the symbol in
its set.)  For example, starting from the root of the tree in \link{Figure 2.18}, 
we arrive at the leaf for D by following a right branch, then a left
branch, then a right branch, then a right branch; hence, the code for D is
1011.

To decode a bit sequence using a Huffman tree, we begin at the root and use the
successive zeros and ones of the bit sequence to determine whether to move down
the left or the right branch.  Each time we come to a leaf, we have generated a
new symbol in the message, at which point we start over from the root of the
tree to find the next symbol.  For example, suppose we are given the tree above
and the sequence 10001010.  Starting at the root, we move down the right
branch, (since the first bit of the string is 1), then down the left branch
(since the second bit is 0), then down the left branch (since the third bit is
also 0).  This brings us to the leaf for B, so the first symbol of the decoded
message is B.  Now we start again at the root, and we make a left move because
the next bit in the string is 0.  This brings us to the leaf for A.  Then we
start again at the root with the rest of the string 1010, so we move right,
left, right, left and reach C.  Thus, the entire message is BAC.

\subsubsection*{Generating Huffman trees}

Given an ``alphabet'' of symbols and their relative frequencies, how do we
construct the ``best'' code?  (In other words, which tree will encode messages
with the fewest bits?)  Huffman gave an algorithm for doing this and showed
that the resulting code is indeed the best variable-length code for messages
where the relative frequency of the symbols matches the frequencies with which
the code was constructed.  We will not prove this optimality of Huffman codes
here, but we will show how Huffman trees are constructed.\footnote{See \link{Hamming 1980} 
for a discussion of the mathematical properties of Huffman codes.}

The algorithm for generating a Huffman tree is very simple. The idea is to
arrange the tree so that the symbols with the lowest frequency appear farthest
away from the root. Begin with the set of leaf nodes, containing symbols and
their frequencies, as determined by the initial data from which the code is to
be constructed. Now find two leaves with the lowest weights and merge them to
produce a node that has these two nodes as its left and right branches. The
weight of the new node is the sum of the two weights. Remove the two leaves
from the original set and replace them by this new node. Now continue this
process. At each step, merge two nodes with the smallest weights, removing them
from the set and replacing them with a node that has these two as its left and
right branches. The process stops when there is only one node left, which is
the root of the entire tree.  Here is how the Huffman tree of \link{Figure 2.18}
was generated:

\begin{example}
Initial leaves  {(A 8) (B 3) (C 1) (D 1) (E 1) (F 1) (G 1) (H 1)}
         Merge  {(A 8) (B 3) ({C D} 2) (E 1) (F 1) (G 1) (H 1)}
         Merge  {(A 8) (B 3) ({C D} 2) ({E F} 2) (G 1) (H 1)}
         Merge  {(A 8) (B 3) ({C D} 2) ({E F} 2) ({G H} 2)}
         Merge  {(A 8) (B 3) ({C D} 2) ({E F G H} 4)}
         Merge  {(A 8) ({B C D} 5) ({E F G H} 4)}
         Merge  {(A 8) ({B C D E F G H} 9)}
   Final merge  {({A B C D E F G H} 17)}
\end{example}

\noindent
The algorithm does not always specify a unique tree, because there may not be
unique smallest-weight nodes at each step.  Also, the choice of the order in
which the two nodes are merged (i.e., which will be the right branch and which
will be the left branch) is arbitrary.

\subsubsection*{Representing Huffman trees}

In the exercises below we will work with a system that uses Huffman trees to
encode and decode messages and generates Huffman trees according to the
algorithm outlined above.  We will begin by discussing how trees are
represented.

Leaves of the tree are represented by a list consisting of the symbol
\code{leaf}, the symbol at the leaf, and the weight:

\begin{scheme}
(define (make-leaf symbol weight) (list 'leaf symbol weight))
(define (leaf? object) (eq? (car object) 'leaf))
(define (symbol-leaf x) (cadr x))
(define (weight-leaf x) (caddr x))
\end{scheme}

\noindent
A general tree will be a list of a left branch, a right branch, a set of
symbols, and a weight.  The set of symbols will be simply a list of the
symbols, rather than some more sophisticated set representation.  When we make
a tree by merging two nodes, we obtain the weight of the tree as the sum of the
weights of the nodes, and the set of symbols as the union of the sets of
symbols for the nodes.  Since our symbol sets are represented as lists, we can
form the union by using the \code{append} procedure we defined in 
\link{Section 2.2.1}:

\begin{scheme}
(define (make-code-tree left right)
  (list left
        right
        (append (symbols left) (symbols right))
        (+ (weight left) (weight right))))
\end{scheme}

\noindent
If we make a tree in this way, we have the following selectors:

\begin{scheme}
(define (left-branch  tree) (car  tree))
(define (right-branch tree) (cadr tree))
(define (symbols tree)
  (if (leaf? tree)
      (list (symbol-leaf tree))
      (caddr tree)))
(define (weight tree)
  (if (leaf? tree)
      (weight-leaf tree)
      (cadddr tree)))
\end{scheme}

\noindent
The procedures \code{symbols} and \code{weight} must do something slightly
different depending on whether they are called with a leaf or a general tree.
These are simple examples of \newterm{generic procedures} (procedures that can
handle more than one kind of data), which we will have much more to say about
in \link{Section 2.4} and \link{Section 2.5}.

\enlargethispage{\baselineskip}

\subsubsection*{The decoding procedure}

The following procedure implements the decoding algorithm.  It takes as
arguments a list of zeros and ones, together with a Huffman tree.

\begin{scheme}
(define (decode bits tree)
  (define (decode-1 bits current-branch)
    (if (null? bits)
        '()
        (let ((next-branch
               (choose-branch (car bits) current-branch)))
          (if (leaf? next-branch)
              (cons (symbol-leaf next-branch)
                    (decode-1 (cdr bits) tree))
              (decode-1 (cdr bits) next-branch)))))
  (decode-1 bits tree))
(define (choose-branch bit branch)
  (cond ((= bit 0) (left-branch branch))
        ((= bit 1) (right-branch branch))
        (else (error "bad bit: CHOOSE-BRANCH" bit))))
\end{scheme}

\noindent
The procedure \code{decode\-/1} takes two arguments: the list of remaining bits
and the current position in the tree.  It keeps moving ``down'' the tree,
choosing a left or a right branch according to whether the next bit in the list
is a zero or a one.  (This is done with the procedure \code{choose\-/branch}.)
When it reaches a leaf, it returns the symbol at that leaf as the next symbol
in the message by \code{cons}ing it onto the result of decoding the rest of the
message, starting at the root of the tree.  Note the error check in the final
clause of \code{choose\-/branch}, which complains if the procedure finds
something other than a zero or a one in the input data.

\subsubsection*{Sets of weighted elements}

In our representation of trees, each non-leaf node contains a set of symbols,
which we have represented as a simple list.  However, the tree-generating
algorithm discussed above requires that we also work with sets of leaves and
trees, successively merging the two smallest items.  Since we will be required
to repeatedly find the smallest item in a set, it is convenient to use an
ordered representation for this kind of set.

We will represent a set of leaves and trees as a list of elements, arranged in
increasing order of weight.  The following \code{adjoin\-/set} procedure for
constructing sets is similar to the one described in \link{Exercise 2.61};
however, items are compared by their weights, and the element being added to
the set is never already in it.

\begin{scheme}
(define (adjoin-set x set)
  (cond ((null? set) (list x))
        ((< (weight x) (weight (car set))) (cons x set))
        (else (cons (car set)
                    (adjoin-set x (cdr set))))))
\end{scheme}

\noindent
The following procedure takes a list of symbol-frequency pairs such as
\code{((A 4) (B 2) (C 1) (D 1))} and constructs an initial ordered set of
leaves, ready to be merged according to the Huffman algorithm:

\begin{scheme}
(define (make-leaf-set pairs)
  (if (null? pairs)
      '()
      (let ((pair (car pairs)))
        (adjoin-set (make-leaf (car pair)    ~\textrm{; symbol}~
                               (cadr pair))  ~\textrm{; frequency}~
                    (make-leaf-set (cdr pairs))))))
\end{scheme}

\begin{quote}
\heading{\phantomsection\label{Exercise 2.67}Exercise 2.67:} Define an encoding tree and a
sample message:

\begin{scheme}
(define sample-tree
  (make-code-tree (make-leaf 'A 4)
                  (make-code-tree
                   (make-leaf 'B 2)
                   (make-code-tree
                    (make-leaf 'D 1)
                    (make-leaf 'C 1)))))
(define sample-message '(0 1 1 0 0 1 0 1 0 1 1 1 0))
\end{scheme}

Use the \code{decode} procedure to decode the message, and give the \mbox{result}.
\end{quote}

\begin{quote}
\heading{\phantomsection\label{Exercise 2.68}Exercise 2.68:} The \code{encode} procedure takes
as arguments a message and a tree and produces the list of bits that gives the
encoded message.

\begin{scheme}
(define (encode message tree)
  (if (null? message)
      '()
      (append (encode-symbol (car message) tree)
              (encode (cdr message) tree))))
\end{scheme}

\code{Encode\-/symbol} is a procedure, which you must write, that returns the
list of bits that encodes a given symbol according to a given tree.  You should
design \code{encode\-/symbol} so that it signals an error if the symbol is not in
the tree at all.  Test your procedure by encoding the result you obtained in
\link{Exercise 2.67} with the sample tree and seeing whether it is the same as
the original sample message.
\end{quote}

\begin{quote}
\heading{\phantomsection\label{Exercise 2.69}Exercise 2.69:} The following procedure takes as
its argument a list of symbol-frequency pairs (where no symbol appears in more
than one pair) and generates a Huffman encoding tree according to the Huffman
algorithm.

\begin{scheme}
(define (generate-huffman-tree pairs)
  (successive-merge (make-leaf-set pairs)))
\end{scheme}

\code{Make\-/leaf\-/set} is the procedure given above that transforms the list of
pairs into an ordered set of leaves.  \code{Successive\-/merge} is the procedure
you must write, using \code{make\-/code\-/tree} to successively merge the
smallest-weight elements of the set until there is only one element left, which
is the desired Huffman tree.  (This procedure is slightly tricky, but not
really complicated.  If you find yourself designing a complex procedure, then
you are almost certainly doing something wrong.  You can take significant
advantage of the fact that we are using an ordered set representation.)
\end{quote}

\begin{quote}
\heading{\phantomsection\label{Exercise 2.70}Exercise 2.70:} The following eight-symbol
alphabet with associated relative frequencies was designed to efficiently
encode the lyrics of 1950s rock songs.  (Note that the ``symbols'' of an
``alphabet'' need not be individual letters.)

\begin{example}
A    2   GET 2   SHA 3   WAH 1
BOOM 1   JOB 2   NA 16   YIP 9
\end{example}

Use \code{generate\-/huffman\-/tree} (\link{Exercise 2.69}) to generate a
corresponding Huffman tree, and use \code{encode} (\link{Exercise 2.68}) to
encode the following message:

\begin{example}
Get a job
Sha na na na na na na na na
Get a job
Sha na na na na na na na na
Wah yip yip yip yip yip yip yip yip yip
Sha boom
\end{example}

How many bits are required for the encoding?  What is the smallest number of
bits that would be needed to encode this song if we used a fixed-length code
for the eight-symbol alphabet?
\end{quote}

\begin{quote}
\heading{\phantomsection\label{Exercise 2.71}Exercise 2.71:} Suppose we have a Huffman tree
for an alphabet of \( n \) symbols, and that the relative frequencies of the
symbols are \( 1, 2, 4, \dots, 2^{n-1} \).  Sketch the tree for \( n=5 \); for
\( n=10 \).  In such a tree (for general \( n \)) how many bits are required to
encode the most frequent symbol?  The least frequent symbol?
\end{quote}

\begin{quote}
\heading{\phantomsection\label{Exercise 2.72}Exercise 2.72:} Consider the encoding procedure
that you designed in \link{Exercise 2.68}.  What is the order of growth in the
number of steps needed to encode a symbol?  Be sure to include the number of
steps needed to search the symbol list at each node encountered.  To answer
this question in general is difficult.  Consider the special case where the
relative frequencies of the \( n \) symbols are as described in \link{Exercise 2.71}, 
and give the order of growth (as a function of \( n \)) of the number of
steps needed to encode the most frequent and least frequent symbols in the
alphabet.
\end{quote}

\section{Multiple Representations for Abstract Data}
\label{Section 2.4}

We have introduced data abstraction, a methodology for structuring systems in
such a way that much of a program can be specified independent of the choices
involved in implementing the data objects that the program manipulates.  For
example, we saw in \link{Section 2.1.1} how to separate the task of designing a
program that uses rational numbers from the task of implementing rational
numbers in terms of the computer language's primitive mechanisms for
constructing compound data.  The key idea was to erect an abstraction 
barrier---in this case, the selectors and constructors for rational numbers
(\code{make\-/rat}, \code{numer}, \code{denom})---that isolates the way rational
numbers are used from their underlying representation in terms of list
structure.  A similar abstraction barrier isolates the details of the
procedures that perform rational arithmetic (\code{add\-/rat}, \code{sub\-/rat},
\code{mul\-/rat}, and \code{div\-/rat}) from the ``higher-level'' procedures that
use rational numbers.  The resulting program has the structure shown in
\link{Figure 2.1}.

These data-abstraction barriers are powerful tools for controlling complexity.
By isolating the underlying representations of data objects, we can divide the
task of designing a large program into smaller tasks that can be performed
separately.  But this kind of data abstraction is not yet powerful enough,
because it may not always make sense to speak of ``the underlying
representation'' for a data object.

For one thing, there might be more than one useful representation for a data
object, and we might like to design systems that can deal with multiple
representations.  To take a simple example, complex numbers may be represented
in two almost equivalent ways: in rectangular form (real and imaginary parts)
and in polar form (magnitude and angle).  Sometimes rectangular form is more
appropriate and sometimes polar form is more appropriate.  Indeed, it is
perfectly plausible to imagine a system in which complex numbers are
represented in both ways, and in which the procedures for manipulating complex
numbers work with either representation.

More importantly, programming systems are often designed by many people working
over extended periods of time, subject to requirements that change over time.
In such an environment, it is simply not possible for everyone to agree in
advance on choices of data representation.  So in addition to the
data-abstraction barriers that isolate representation from use, we need
abstraction barriers that isolate different design choices from each other and
permit different choices to coexist in a single program.  Furthermore, since
large programs are often created by combining pre-existing modules that were
designed in isolation, we need conventions that permit programmers to
incorporate modules into larger systems \newterm{additively}, that is, without
having to redesign or reimplement these modules.

In this section, we will learn how to cope with data that may be represented in
different ways by different parts of a program.  This requires constructing
\newterm{generic procedures}---procedures that can operate on data that may be
represented in more than one way.  Our main technique for building generic
procedures will be to work in terms of data objects that have \newterm{type
tags}, that is, data objects that include explicit information about how they
are to be processed.  We will also discuss \newterm{data-directed} programming,
a powerful and convenient implementation strategy for additively assembling
systems with generic operations.

We begin with the simple complex-number example. We will see how type tags and
data-directed style enable us to design separate rectangular and polar
representations for complex numbers while maintaining the notion of an abstract
``complex-number'' data object.  We will accomplish this by defining arithmetic
procedures for complex numbers (\code{add\-/complex}, \code{sub\-/complex},
\code{mul\-/complex}, and \code{div\-/complex}) in terms of generic selectors that
access parts of a complex number independent of how the number is represented.
The resulting complex-number system, as shown in \link{Figure 2.19}, contains
two different kinds of abstraction barriers.  The ``horizontal'' abstraction
barriers play the same role as the ones in \link{Figure 2.1}.  
They isolate``higher-level'' operations from ``lower-level'' 
representations.  In addition, there is a ``vertical'' barrier that 
gives us the ability to separately design and install alternative 
representations.

\begin{figure}[tb]
\phantomsection\label{Figure 2.19}
\centering
\begin{comment}
\begin{quote}
\heading{Figure 2.19:} Data-abstraction barriers in the complex-number system.

\begin{example}
           Programs that use complex numbers
  +-------------------------------------------------+
--| add-complex sub-complex mul-complex div-complex |--
  +-------------------------------------------------+
              Complex arithmetic package
---------------------------+---------------------------
          Rectangular      |         Polar
        representation     |     representation
---------------------------+---------------------------
    List structure and primitive machine arithmetic
\end{example}
\end{quote}
\end{comment}
\includegraphics[width=108mm]{fig/chap2/Fig2.19a.pdf}
\begin{quote}
\heading{Figure 2.19:} Data-abstraction barriers in the complex-number system.
\end{quote}
\end{figure}

In \link{Section 2.5} we will show how to use type tags and data-directed style
to develop a generic arithmetic package.  This provides procedures (\code{add},
\code{mul}, and so on) that can be used to manipulate all sorts of ``numbers''
and can be easily extended when a new kind of number is needed.  In 
\link{Section 2.5.3}, we'll show how to use generic arithmetic in a system that performs
symbolic algebra.



\subsection{Representations for Complex Numbers}
\label{Section 2.4.1}

We will develop a system that performs arithmetic operations on complex numbers
as a simple but unrealistic example of a program that uses generic operations.
We begin by discussing two plausible representations for complex numbers as
ordered pairs: rectangular form (real part and imaginary part) and polar form
(magnitude and angle).\footnote{In actual computational systems, rectangular
form is preferable to polar form most of the time because of roundoff errors in
conversion between rectangular and polar form.  This is why the complex-number
example is unrealistic.  Nevertheless, it provides a clear illustration of the
design of a system using generic operations and a good introduction to the more
substantial systems to be developed later in this chapter.}  
\link{Section 2.4.2} will show how both representations can be made to coexist in a
single system through the use of type tags and generic operations.

Like rational numbers, complex numbers are naturally represented as ordered
pairs.  The set of complex numbers can be thought of as a two-dimensional space
with two orthogonal axes, the ``real'' axis and the ``imaginary'' axis. (See
\link{Figure 2.20}.
)  From this point of view, the complex number 
\( z = x + iy \) (where \( i^2 = -1 \)) can be thought of as the point in the plane
whose real coordinate is \( x \) and whose imaginary coordinate is \( y \).
Addition of complex numbers reduces in this representation to addition of
coordinates:
\begin{comment}

\begin{example}
Real-part(z_1 + z_2) = Real-part(z_1) + Real-part(z_2)

Imaginary-part(z_1 + z_2) = Imaginary-part(z_1) + Imaginary-part(z_2)
\end{example}

\end{comment}
\begin{displaymath}
%  \eqalign{
% \hbox{Real-part} (z_1 + z_2)\; 		&= 
% 	\hbox{ Real-part} (z_1)\; + \hbox{ Real-part} (z_2), \cr
% \hbox{Imaginary-part} (z_1 + z_2)\; 	&= 
% 	\hbox{ Imaginary-part} (z_1)\; + \hbox{ Imaginary-part} (z_2). \cr 
% } 
\begin{array}{r@{{}={}}l}
  \hbox{Real-part} (z_1 + z_2)\; 	& 
	\hbox{ Real-part} (z_1)\; + \hbox{ Real-part} (z_2), \\
  \hbox{Imaginary-part} (z_1 + z_2)\; 	& 
	\hbox{ Imaginary-part} (z_1)\; + \hbox{ Imaginary-part} (z_2). 
\end{array}
\end{displaymath}

\begin{figure}[tb]
\phantomsection\label{Figure 2.20}
\centering
\begin{comment}
\heading{Figure 2.20:} Complex numbers as points in the plane.

\begin{example}
 Imaginary
    ^
    |
  y |.........................* z = x + ?y = r e^(?A)
    |                    __-- .
    |                __--     .
    |          r __--         .
    |        __--             .
    |    __-- \               .
    |__--    A |              .
----+----------+-------------------> Real
                              x
\end{example}
\end{comment}
\includegraphics[width=79mm]{fig/chap2/Fig2.20.pdf}
\par\bigskip
\noindent
\heading{Figure 2.20:} Complex numbers as points in the plane.
\end{figure}

When multiplying complex numbers, it is more natural to think in terms of
representing a complex number in polar form, as a magnitude and an angle (\( r \)
and \( A \) in \link{Figure 2.20}).  The product of two complex numbers is the
vector obtained by stretching one complex number by the length of the other and
then rotating it through the angle of the other:
\begin{comment}

\begin{example}
Magnitude(z_1 * z_2) = Magnitude(z_1) * Magnitude(z_2)

Angle(z_1 * z_2) = Angle(z_1) + Angle(z_2)
\end{example}

\end{comment}
\begin{displaymath}
%  \eqalign{
% 	\hbox{Magnitude} (z_1 \cdot z_2)\; 	&= 
% 		\hbox{ Magnitude} (z_1)\; \cdot \hbox{ Magnitude} (z_2), \cr
% 	\hbox{Angle} (z_1 \cdot z_2)\; 		&= 
% 		\hbox{ Angle} (z_1)\; + \hbox{ Angle} (z_2). \cr
% } 
\begin{array}{r@{{}={}}l}
	\hbox{Magnitude} (z_1 \cdot z_2)\; 	& 
		\hbox{ Magnitude} (z_1)\; \cdot \hbox{ Magnitude} (z_2), \\
	\hbox{Angle} (z_1 \cdot z_2)\; 		& 
		\hbox{ Angle} (z_1)\; + \hbox{ Angle} (z_2). 
\end{array}
\end{displaymath}

\noindent
Thus, there are two different representations for complex numbers, which are
appropriate for different operations.  Yet, from the viewpoint of someone
writing a program that uses complex numbers, the principle of data abstraction
suggests that all the operations for manipulating complex numbers should be
available regardless of which representation is used by the computer.  For
example, it is often useful to be able to find the magnitude of a complex
number that is specified by rectangular coordinates.  Similarly, it is often
useful to be able to determine the real part of a complex number that is
specified by polar coordinates.

To design such a system, we can follow the same data-abstraction strategy we
followed in designing the rational-number package in \link{Section 2.1.1}.
Assume that the operations on complex numbers are implemented in terms of four
selectors: \code{real\-/part}, \code{imag\-/part}, \code{magnitude} and
\code{angle}.  Also assume that we have two procedures for constructing complex
numbers: \code{make\-/from\-/real\-/imag} returns a complex number with specified
real and imaginary parts, and \code{make\-/from\-/mag\-/ang} returns a complex number
with specified magnitude and angle.  These procedures have the property that,
for any complex number \code{z}, both

\begin{scheme}
(make-from-real-imag (real-part z) (imag-part z))
\end{scheme}

\noindent
and

\begin{scheme}
(make-from-mag-ang (magnitude z) (angle z))
\end{scheme}

\noindent
produce complex numbers that are equal to \code{z}.

Using these constructors and selectors, we can implement arithmetic on complex
numbers using the ``abstract data'' specified by the constructors and
selectors, just as we did for rational numbers in \link{Section 2.1.1}.  As
shown in the formulas above, we can add and subtract complex numbers in terms
of real and imaginary parts while multiplying and dividing complex numbers in
terms of magnitudes and angles:

\begin{scheme}

(define (add-complex z1 z2)
  (make-from-real-imag (+ (real-part z1) (real-part z2))
                       (+ (imag-part z1) (imag-part z2))))
(define (sub-complex z1 z2)
  (make-from-real-imag (- (real-part z1) (real-part z2))
                       (- (imag-part z1) (imag-part z2))))
(define (mul-complex z1 z2)
  (make-from-mag-ang (* (magnitude z1) (magnitude z2))
                     (+ (angle z1) (angle z2))))
(define (div-complex z1 z2)
  (make-from-mag-ang (/ (magnitude z1) (magnitude z2))
                     (- (angle z1) (angle z2))))
\end{scheme}

\noindent
To complete the complex-number package, we must choose a representation and we
must implement the constructors and selectors in terms of primitive numbers and
primitive list structure.  There are two obvious ways to do this: We can
represent a complex number in ``rectangular form'' as a pair (real part,
imaginary part) or in ``polar form'' as a pair (magnitude, angle).  Which shall
we choose?

In order to make the different choices concrete, imagine that there are two
programmers, Ben Bitdiddle and Alyssa P. Hacker, who are independently
designing representations for the complex-number system.  Ben chooses to
represent complex numbers in rectangular form.  With this choice, selecting the
real and imaginary parts of a complex number is straightforward, as is
constructing a complex number with given real and imaginary parts.  To find the
magnitude and the angle, or to construct a complex number with a given
magnitude and angle, he uses the trigonometric relations
\begin{comment}

\begin{example}
                      __________
x = r cos A     r = ./ x^2 + y^2

y = r sin A     A = arctan(y,x)
\end{example}

\end{comment}
\begin{displaymath}
%  \eqalign{
% 	x = r \cos A, \qquad 	& r = \sqrt{x^2 + y^2}, \cr
% 	y = r \sin A, \qquad 	& A = \arctan(y, x), \cr
% } 
\begin{array}{r@{{}={}}lr@{{}={}}l}
	x & r \cos A, \qquad 	& r & \sqrt{x^2 + y^2}, \\
	y & r \sin A, \qquad 	& A & \arctan(y, x), 
\end{array}
\end{displaymath}
\noindent
which relate the real and imaginary parts \( (x, y) \) to the magnitude and
the angle \( (r, A) \).\footnote{The arctangent function referred to here,
computed by Scheme's \code{atan} procedure, is defined so as to take two
arguments \( y \) and \( x \) and to return the angle whose tangent is \( y / x \).
The signs of the arguments determine the quadrant of the angle.}  Ben's
representation is therefore given by the following selectors and constructors:

\begin{scheme}
(define (real-part z) (car z))
(define (imag-part z) (cdr z))
(define (magnitude z)
  (sqrt (+ (square (real-part z))
           (square (imag-part z)))))
(define (angle z)
  (atan (imag-part z) (real-part z)))
(define (make-from-real-imag x y) (cons x y))
(define (make-from-mag-ang r a)
  (cons (* r (cos a)) (* r (sin a))))
\end{scheme}

\noindent
Alyssa, in contrast, chooses to represent complex numbers in polar form.  For
her, selecting the magnitude and angle is straightforward, but she has to use
the trigonometric relations to obtain the real and imaginary parts.  Alyssa's
representation is:

\begin{scheme}
(define (real-part z) (* (magnitude z) (cos (angle z))))
(define (imag-part z) (* (magnitude z) (sin (angle z))))
(define (magnitude z) (car z))
(define (angle z) (cdr z))
(define (make-from-real-imag x y)
  (cons (sqrt (+ (square x) (square y)))
        (atan y x)))
(define (make-from-mag-ang r a) (cons r a))
\end{scheme}

\noindent
The discipline of data abstraction ensures that the same implementation of
\code{add\-/complex}, \code{sub\-/complex}, \code{mul\-/complex}, and
\code{div\-/complex} will work with either Ben's representation or Alyssa's
representation.

\subsection{Tagged data}
\label{Section 2.4.2}

One way to view data abstraction is as an application of the ``principle of
least commitment.''  In implementing the complex-number system in 
\link{Section 2.4.1}, we can use either Ben's rectangular representation or Alyssa's
polar representation.  The abstraction barrier formed by the selectors and
constructors permits us to defer to the last possible moment the choice of a
concrete representation for our data objects and thus retain maximum
flexibility in our system design.

The principle of least commitment can be carried to even further extremes.  If
we desire, we can maintain the ambiguity of representation even \emph{after} we
have designed the selectors and constructors, and elect to use both Ben's
representation \emph{and} Alyssa's representation.  If both representations are
included in a single system, however, we will need some way to distinguish data
in polar form from data in rectangular form.  Otherwise, if we were asked, for
instance, to find the \code{magnitude} of the pair (3, 4), we wouldn't know
whether to answer 5 (interpreting the number in rectangular form) or 3
(interpreting the number in polar form).  A straightforward way to accomplish
this distinction is to include a \newterm{type tag}---the symbol
\code{rectangular} or \code{polar}---as part of each complex number.  Then
when we need to manipulate a complex number we can use the tag to decide which
selector to apply.

In order to manipulate tagged data, we will assume that we have procedures
\code{type\-/tag} and \code{contents} that extract from a data object the tag and
the actual contents (the polar or rectangular coordinates, in the case of a
complex number).  We will also postulate a procedure \code{attach\-/tag} that
takes a tag and contents and produces a tagged data object.  A straightforward
way to implement this is to use ordinary list structure:

\begin{scheme}
(define (attach-tag type-tag contents)
  (cons type-tag contents))
(define (type-tag datum)
  (if (pair? datum)
      (car datum)
      (error "Bad tagged datum: TYPE-TAG" datum)))
(define (contents datum)
  (if (pair? datum)
      (cdr datum)
      (error "Bad tagged datum: CONTENTS" datum)))
\end{scheme}

\noindent
Using these procedures, we can define predicates \code{rectangular?}  and
\code{polar?}, which recognize rectangular and polar numbers, respectively:

\begin{scheme}
(define (rectangular? z)
  (eq? (type-tag z) 'rectangular))
(define (polar? z) (eq? (type-tag z) 'polar))
\end{scheme}

\noindent
With type tags, Ben and Alyssa can now modify their code so that their two
different representations can coexist in the same system.  Whenever Ben
constructs a complex number, he tags it as rectangular.  Whenever Alyssa
constructs a complex number, she tags it as polar.  In addition, Ben and Alyssa
must make sure that the names of their procedures do not conflict.  One way to
do this is for Ben to append the suffix \code{rectangular} to the name of each
of his representation procedures and for Alyssa to append \code{polar} to the
names of hers.  Here is Ben's revised rectangular representation from 
\link{Section 2.4.1}:

\begin{scheme}
(define (real-part-rectangular z) (car z))
(define (imag-part-rectangular z) (cdr z))
(define (magnitude-rectangular z)
  (sqrt (+ (square (real-part-rectangular z))
           (square (imag-part-rectangular z)))))
(define (angle-rectangular z)
  (atan (imag-part-rectangular z)
        (real-part-rectangular z)))
(define (make-from-real-imag-rectangular x y)
  (attach-tag 'rectangular (cons x y)))
(define (make-from-mag-ang-rectangular r a)
  (attach-tag 'rectangular
              (cons (* r (cos a)) (* r (sin a)))))
\end{scheme}

\noindent
and here is Alyssa's revised polar representation:

\begin{scheme}
(define (real-part-polar z)
  (* (magnitude-polar z) (cos (angle-polar z))))
(define (imag-part-polar z)
  (* (magnitude-polar z) (sin (angle-polar z))))
(define (magnitude-polar z) (car z))
(define (angle-polar z) (cdr z))
(define (make-from-real-imag-polar x y)
  (attach-tag 'polar
              (cons (sqrt (+ (square x) (square y)))
                    (atan y x))))
(define (make-from-mag-ang-polar r a)
  (attach-tag 'polar (cons r a)))
\end{scheme}

\noindent
Each generic selector is implemented as a procedure that checks the tag of its
argument and calls the appropriate procedure for handling data of that type.
For example, to obtain the real part of a complex number, \code{real\-/part}
examines the tag to determine whether to use Ben's \code{real\-/part\-/rectangular}
or Alyssa's \code{real\-/part\-/polar}.  In either case, we use \code{contents} to
extract the bare, untagged datum and send this to the rectangular or polar
procedure as required:

\begin{scheme}
(define (real-part z)
  (cond ((rectangular? z)
         (real-part-rectangular (contents z)))
        ((polar? z)
         (real-part-polar (contents z)))
        (else (error "Unknown type: REAL-PART" z))))
(define (imag-part z)
  (cond ((rectangular? z)
         (imag-part-rectangular (contents z)))
        ((polar? z)
         (imag-part-polar (contents z)))
        (else (error "Unknown type: IMAG-PART" z))))
(define (magnitude z)
  (cond ((rectangular? z)
         (magnitude-rectangular (contents z)))
        ((polar? z)
         (magnitude-polar (contents z)))
        (else (error "Unknown type: MAGNITUDE" z))))
(define (angle z)
  (cond ((rectangular? z)
         (angle-rectangular (contents z)))
        ((polar? z)
         (angle-polar (contents z)))
        (else (error "Unknown type: ANGLE" z))))
\end{scheme}

\noindent
To implement the complex-number arithmetic operations, we can use the same
procedures \code{add\-/complex}, \code{sub\-/complex}, \code{mul\-/complex}, and
\code{div\-/complex} from \link{Section 2.4.1}, because the selectors they call
are generic, and so will work with either representation.  For example, the
procedure \code{add\-/complex} is still

\begin{scheme}
(define (add-complex z1 z2)
  (make-from-real-imag (+ (real-part z1) (real-part z2))
                       (+ (imag-part z1) (imag-part z2))))
\end{scheme}

\noindent
Finally, we must choose whether to construct complex numbers using Ben's
representation or Alyssa's representation.  One reasonable choice is to
construct rectangular numbers whenever we have real and imaginary parts and to
construct polar numbers whenever we have magnitudes and angles:

\begin{scheme}
(define (make-from-real-imag x y)
  (make-from-real-imag-rectangular x y))
(define (make-from-mag-ang r a)
  (make-from-mag-ang-polar r a))
\end{scheme}

\noindent
The resulting complex-number system has the structure shown in \link{Figure 2.21}.  
The system has been decomposed into three relatively independent parts:
the complex-number-arithmetic operations, Alyssa's polar implementation, and
Ben's rectangular implementation.  The polar and rectangular implementations
could have been written by Ben and Alyssa working separately, and both of these
can be used as underlying representations by a third programmer implementing
the complex-arithmetic procedures in terms of the abstract constructor/selector
interface.

\begin{figure}[tb]
\phantomsection\label{Figure 2.21}
\centering
\begin{comment}
\begin{quote}
\heading{Figure 2.21:} Structure of the generic complex-arithmetic system.

\begin{example}
    +-------------------------------------------------+
----| add-complex sub-complex mul-complex div-complex |----
    +-------------------------------------------------+
                Complex arithmetic package
                 +-----------------------+
                 | real-part   imag-part |
-----------------|                       |------------------
                 | magnitude   angle     |
                 +-----------+-----------+
           Rectangular       |          Polar
          representation     |     representation
-----------------------------+------------------------------
       List structure and primitive machine arithmetic
\end{example}
\end{quote}
\end{comment}
\includegraphics[width=108mm]{fig/chap2/Fig2.21a.pdf}
\par\bigskip
\noindent
\heading{Figure 2.21:} Structure of the generic complex-arithmetic system. 
\end{figure}

Since each data object is tagged with its type, the selectors operate on the
data in a generic manner.  That is, each selector is defined to have a behavior
that depends upon the particular type of data it is applied to.  Notice the
general mechanism for interfacing the separate representations: Within a given
representation implementation (say, Alyssa's polar package) a complex number is
an untyped pair (magnitude, angle).  When a generic selector operates on a
number of \code{polar} type, it strips off the tag and passes the contents on
to Alyssa's code.  Conversely, when Alyssa constructs a number for general use,
she tags it with a type so that it can be appropriately recognized by the
higher-level procedures.  This discipline of stripping off and attaching tags
as data objects are passed from level to level can be an important
organizational strategy, as we shall see in \link{Section 2.5}.

\subsection{Data-Directed Programming and Additivity}
\label{Section 2.4.3}

The general strategy of checking the type of a datum and calling an appropriate
procedure is called \newterm{dispatching on type}.  This is a powerful strategy
for obtaining modularity in system design.  On the other hand, implementing the
dispatch as in \link{Section 2.4.2} has two significant weaknesses.  One
weakness is that the generic interface procedures (\code{real\-/part},
\code{imag\-/part}, \code{magnitude}, and \code{angle}) must know about all the
different representations.  For instance, suppose we wanted to incorporate a
new representation for complex numbers into our complex-number system.  We
would need to identify this new representation with a type, and then add a
clause to each of the generic interface procedures to check for the new type
and apply the appropriate selector for that representation.

Another weakness of the technique is that even though the individual
representations can be designed separately, we must guarantee that no two
procedures in the entire system have the same name.  This is why Ben and Alyssa
had to change the names of their original procedures from 
\link{Section 2.4.1}.

The issue underlying both of these weaknesses is that the technique for
implementing generic interfaces is not \newterm{additive}.  The person
implementing the generic selector procedures must modify those procedures each
time a new representation is installed, and the people interfacing the
individual representations must modify their code to avoid name conflicts.  In
each of these cases, the changes that must be made to the code are
straightforward, but they must be made nonetheless, and this is a source of
inconvenience and error.  This is not much of a problem for the complex-number
system as it stands, but suppose there were not two but hundreds of different
representations for complex numbers.  And suppose that there were many generic
selectors to be maintained in the abstract-data interface.  Suppose, in fact,
that no one programmer knew all the interface procedures or all the
representations.  The problem is real and must be addressed in such programs as
large-scale data-base-management systems.

What we need is a means for modularizing the system design even further.  This
is provided by the programming technique known as \newterm{data-directed
programming}.  To understand how data-directed programming works, begin with
the observation that whenever we deal with a set of generic operations that are
common to a set of different types we are, in effect, dealing with a
two-dimensional table that contains the possible operations on one axis and the
possible types on the other axis.  The entries in the table are the procedures
that implement each operation for each type of argument presented.  In the
complex-number system developed in the previous section, the correspondence
between operation name, data type, and actual procedure was spread out among
the various conditional clauses in the generic interface procedures.  But the
same information could have been organized in a table, as shown in \link{Figure 2.22}.

Data-directed programming is the technique of designing programs to work with
such a table directly.  Previously, we implemented the mechanism that
interfaces the complex-arithmetic code with the two representation packages as
a set of procedures that each perform an explicit dispatch on type.  Here we
will implement the interface as a single procedure that looks up the
combination of the operation name and argument type in the table to find the
correct procedure to apply, and then applies it to the contents of the
argument.  If we do this, then to add a new representation package to the
system we need not change any existing procedures; we need only add new entries
to the table.

\begin{figure}[tb]
\phantomsection\label{Figure 2.22}
\centering
\begin{comment}
\begin{quote}
\heading{Figure 2.22:} Table of operations for the complex-number system.

\begin{example}
           |               Types
Operations | Polar           | Rectangular
===========+=================+======================
real-part  | real-part-polar | real-part-rectangular
imag-part  | imag-part-polar | imag-part-rectangular
magnitude  | magnitude-polar | magnitude-rectangular
angle      | angle-polar     | angle-rectangular
\end{example}
\end{quote}
\end{comment}
\includegraphics[width=102mm]{fig/chap2/Fig2.22.pdf}
\par\bigskip
\noindent
\heading{Figure 2.22:} Table of operations for the complex-number system. 
\end{figure}

To implement this plan, assume that we have two procedures, \code{put} and
\code{get}, for manipulating the operation-and-type table:

\begin{itemize}

\item
\( \hbox{\tt(put}\;\langle \)\var{op}\( \kern0.1em\rangle\;\langle \)\var{type}\( \kern0.08em\rangle\;\langle \)\var{item}\( \kern0.08em\rangle\hbox{\tt)} \) installs the \( \langle \)\var{item}\( \kern0.08em\rangle \) in the table, indexed by the
\( \langle \)\var{op}\( \kern0.1em\rangle \) and the \( \langle \)\var{type}\( \kern0.08em\rangle \).

\item
\( \hbox{\tt(get}\;\langle \)\var{op}\( \kern0.1em\rangle\;\langle \)\var{type}\( \kern0.08em\rangle\hbox{\tt)} \) looks up the
\( \langle \)\var{op}\( \kern0.08em\rangle \), \( \langle \)\var{type}\( \kern0.08em\rangle \) entry in the table and
returns the item found there.  If no item is found, \code{get} returns false.

\end{itemize}

\noindent
For now, we can assume that \code{put} and \code{get} are included in our
language.  In \link{Chapter 3} (\link{Section 3.3.3}) we
will see how to implement these and other operations for manipulating tables.

Here is how data-directed programming can be used in the complex-number system.
Ben, who developed the rectangular representation, implements his code just as
he did originally.  He defines a collection of procedures, or a
\newterm{package}, and interfaces these to the rest of the system by adding
entries to the table that tell the system how to operate on rectangular
numbers.  This is accomplished by calling the following procedure:

\begin{scheme}
(define (install-rectangular-package)
  ~\textrm{;; internal procedures}~
  (define (real-part z) (car z))
  (define (imag-part z) (cdr z))
  (define (make-from-real-imag x y) (cons x y))
  (define (magnitude z)
    (sqrt (+ (square (real-part z))
             (square (imag-part z)))))
  (define (angle z)
    (atan (imag-part z) (real-part z)))
  (define (make-from-mag-ang r a)
    (cons (* r (cos a)) (* r (sin a))))

  ~\textrm{;; interface to the rest of the system}~
  (define (tag x) (attach-tag 'rectangular x))
  (put 'real-part '(rectangular) real-part)
  (put 'imag-part '(rectangular) imag-part)
  (put 'magnitude '(rectangular) magnitude)
  (put 'angle '(rectangular) angle)
  (put 'make-from-real-imag 'rectangular
       (lambda (x y) (tag (make-from-real-imag x y))))
  (put 'make-from-mag-ang 'rectangular
       (lambda (r a) (tag (make-from-mag-ang r a))))
  'done)
\end{scheme}

\noindent
Notice that the internal procedures here are the same procedures from 
\link{Section 2.4.1} that Ben wrote when he was working in isolation.  No changes are
necessary in order to interface them to the rest of the system.  Moreover,
since these procedure definitions are internal to the installation procedure,
Ben needn't worry about name conflicts with other procedures outside the
rectangular package.  To interface these to the rest of the system, Ben
installs his \code{real\-/part} procedure under the operation name
\code{real\-/part} and the type \code{(rectangular)}, and similarly for the other
selectors.\footnote{We use the list \code{(rectangular)} rather than the symbol
\code{rectangular} to allow for the possibility of operations with multiple
arguments, not all of the same type.}  The interface also defines the
constructors to be used by the external system.\footnote{The type the
constructors are installed under needn't be a list because a constructor is
always used to make an object of one particular type.}  These are identical to
Ben's internally defined constructors, except that they attach the tag.

Alyssa's polar package is analogous:

\begin{scheme}
(define (install-polar-package)
  ~\textrm{;; internal procedures}~
  (define (magnitude z) (car z))
  (define (angle z) (cdr z))
  (define (make-from-mag-ang r a) (cons r a))
  (define (real-part z) (* (magnitude z) (cos (angle z))))
  (define (imag-part z) (* (magnitude z) (sin (angle z))))
  (define (make-from-real-imag x y)
    (cons (sqrt (+ (square x) (square y)))
          (atan y x)))
  ~\textrm{;; interface to the rest of the system}~
  (define (tag x) (attach-tag 'polar x))
  (put 'real-part '(polar) real-part)
  (put 'imag-part '(polar) imag-part)
  (put 'magnitude '(polar) magnitude)
  (put 'angle '(polar) angle)
  (put 'make-from-real-imag 'polar
       (lambda (x y) (tag (make-from-real-imag x y))))
  (put 'make-from-mag-ang 'polar
       (lambda (r a) (tag (make-from-mag-ang r a))))
  'done)
\end{scheme}

\noindent
Even though Ben and Alyssa both still use their original procedures defined
with the same names as each other's (e.g., \code{real\-/part}), these definitions
are now internal to different procedures (see \link{Section 1.1.8}), so there is
no name conflict.

The complex-arithmetic selectors access the table by means of a general
``operation'' procedure called \code{apply\-/generic}, which applies a generic
operation to some arguments.  \code{Apply\-/generic} looks in the table under the
name of the operation and the types of the arguments and applies the resulting
procedure if one is present:\footnote{\code{Apply\-/generic} uses the dotted-tail
notation described in \link{Exercise 2.20}, because different generic operations
may take different numbers of arguments.  In \code{apply\-/generic}, \code{op}
has as its value the first argument to \code{apply\-/generic} and \code{args} has
as its value a list of the remaining arguments.

\code{Apply\-/generic} also uses the primitive procedure \code{apply}, which
takes two arguments, a procedure and a list.  \code{Apply} applies the
procedure, using the elements in the list as arguments.  For example,

\begin{smallscheme}
(apply + (list 1 2 3 4))
\end{smallscheme}

\noindent
returns 10.}

\begin{scheme}
(define (apply-generic op . args)
  (let ((type-tags (map type-tag args)))
    (let ((proc (get op type-tags)))
      (if proc
          (apply proc (map contents args))
          (error
            "No method for these types: APPLY-GENERIC"
            (list op type-tags))))))
\end{scheme}

\noindent
Using \code{apply\-/generic}, we can define our generic selectors as follows:

\begin{scheme}
(define (real-part z) (apply-generic 'real-part z))
(define (imag-part z) (apply-generic 'imag-part z))
(define (magnitude z) (apply-generic 'magnitude z))
(define (angle z) (apply-generic 'angle z))
\end{scheme}

\noindent
Observe that these do not change at all if a new representation is added to the
system.

We can also extract from the table the constructors to be used by the programs
external to the packages in making complex numbers from real and imaginary
parts and from magnitudes and angles.  As in \link{Section 2.4.2}, we construct
rectangular numbers whenever we have real and imaginary parts, and polar
numbers whenever we have magnitudes and angles:

\begin{scheme}
(define (make-from-real-imag x y)
  ((get 'make-from-real-imag 'rectangular) x y))
(define (make-from-mag-ang r a)
  ((get 'make-from-mag-ang 'polar) r a))
\end{scheme}

\begin{quote}
\heading{\phantomsection\label{Exercise 2.73}Exercise 2.73:} \link{Section 2.3.2} described a
program that performs symbolic differentiation:

\begin{scheme}
(define (deriv exp var)
  (cond ((number? exp) 0)
        ((variable? exp) 
         (if (same-variable? exp var) 1 0))
        ((sum? exp)
         (make-sum (deriv (addend exp) var)
                   (deriv (augend exp) var)))
        ((product? exp)
         (make-sum (make-product
                    (multiplier exp)
                    (deriv (multiplicand exp) var))
                   (make-product 
                    (deriv (multiplier exp) var)
                    (multiplicand exp))))
        ~\( \dark \langle \)~~\var{\dark more rules can be added here}~~\( \dark \rangle \)~
        (else (error "unknown expression type: 
                      DERIV" exp))))
\end{scheme}

We can regard this program as performing a dispatch on the type of the
expression to be differentiated.  In this situation the ``type tag'' of the
datum is the algebraic operator symbol (such as \code{+}) and the operation
being performed is \code{deriv}.  We can transform this program into
data-directed style by rewriting the basic derivative procedure as

\begin{scheme}
(define (deriv exp var)
  (cond ((number? exp) 0)
        ((variable? exp) (if (same-variable? exp var) 1 0))
        (else ((get 'deriv (operator exp)) 
               (operands exp) var))))
(define (operator exp) (car exp))
(define (operands exp) (cdr exp))
\end{scheme}

\begin{enumerate}[a.]

\item
Explain what was done above.  Why can't we assimilate the predicates
\code{number?} and \code{variable?} into the data-directed dispatch?

\item
Write the procedures for derivatives of sums and products, and the auxiliary
code required to install them in the table used by the program above.

\item
Choose any additional differentiation rule that you like, such as the one for
exponents (\link{Exercise 2.56}), and install it in this data-directed
system.

\item
In this simple algebraic manipulator the type of an expression is the algebraic
operator that binds it together.  Suppose, however, we indexed the procedures
in the opposite way, so that the dispatch line in \code{deriv} looked like

\begin{scheme}
((get (operator exp) 'deriv) (operands exp) var)
\end{scheme}

\noindent
What corresponding changes to the derivative system are required?

\end{enumerate}
\end{quote}

\begin{quote}
\heading{\phantomsection\label{Exercise 2.74}Exercise 2.74:} Insatiable Enterprises, Inc., is
a highly decentralized conglomerate company consisting of a large number of
independent divisions located all over the world.  The company's computer
facilities have just been interconnected by means of a clever
network-interfacing scheme that makes the entire network appear to any user to
be a single computer.  Insatiable's president, in her first attempt to exploit
the ability of the network to extract administrative information from division
files, is dismayed to discover that, although all the division files have been
implemented as data structures in Scheme, the particular data structure used
varies from division to division.  A meeting of division managers is hastily
called to search for a strategy to integrate the files that will satisfy
headquarters' needs while preserving the existing autonomy of the divisions.

Show how such a strategy can be implemented with data-directed programming.  As
an example, suppose that each division's personnel records consist of a single
file, which contains a set of records keyed on employees' names.  The structure
of the set varies from division to division.  Furthermore, each employee's
record is itself a set (structured differently from division to division) that
contains information keyed under identifiers such as \code{address} and
\code{salary}.  In particular:

\begin{enumerate}[a.]

\item
Implement for headquarters a \code{get\-/record} procedure that retrieves a
specified employee's record from a specified personnel file.  The procedure
should be applicable to any division's file.  Explain how the individual
divisions' files should be structured.  In particular, what type information
must be supplied?

\item
Implement for headquarters a \code{get\-/salary} procedure that returns the
salary information from a given employee's record from any division's personnel
file.  How should the record be structured in order to make this operation
work?

\item
Implement for headquarters a \code{find\-/employee\-/record} procedure.  This
should search all the divisions' files for the record of a given employee and
return the record.  Assume that this procedure takes as arguments an employee's
name and a list of all the divisions' files.

\item
When Insatiable takes over a new company, what changes must be made in order to
incorporate the new personnel information into the central system?

\end{enumerate}
\end{quote}

\subsubsection*{Message passing}

The key idea of data-directed programming is to handle generic operations in
programs by dealing explicitly with operation-and-type tables, such as the
table in \link{Figure 2.22}.  The style of programming we used in 
\link{Section 2.4.2} organized the required dispatching on type by having each operation
take care of its own dispatching.  In effect, this decomposes the
operation-and-type table into rows, with each generic operation procedure
representing a row of the table.

An alternative implementation strategy is to decompose the table into columns
and, instead of using ``intelligent operations'' that dispatch on data types,
to work with ``intelligent data objects'' that dispatch on operation names.  We
can do this by arranging things so that a data object, such as a rectangular
number, is represented as a procedure that takes as input the required
operation name and performs the operation indicated.  In such a discipline,
\code{make\-/from\-/real\-/imag} could be written as

\begin{scheme}
(define (make-from-real-imag x y)
  (define (dispatch op)
    (cond ((eq? op 'real-part) x)
          ((eq? op 'imag-part) y)
          ((eq? op 'magnitude) (sqrt (+ (square x) (square y))))
          ((eq? op 'angle) (atan y x))
          (else (error "Unknown op: MAKE-FROM-REAL-IMAG" op))))
  dispatch)
\end{scheme}

\noindent
The corresponding \code{apply\-/generic} procedure, which applies a generic
operation to an argument, now simply feeds the operation's name to the data
object and lets the object do the work:\footnote{One limitation of this
organization is it permits only generic procedures of one argument.}

\begin{scheme}
(define (apply-generic op arg) (arg op))
\end{scheme}

\noindent
Note that the value returned by \code{make\-/from\-/real\-/imag} is a procedure---the
internal \code{dispatch} procedure.  This is the procedure that is invoked when
\code{apply\-/generic} requests an operation to be performed.

This style of programming is called \newterm{message passing}.  The name comes
from the image that a data object is an entity that receives the requested
operation name as a ``message.''  We have already seen an example of message
passing in \link{Section 2.1.3}, where we saw how \code{cons}, \code{car}, and
\code{cdr} could be defined with no data objects but only procedures.  Here we
see that message passing is not a mathematical trick but a useful technique for
organizing systems with generic operations.  In the remainder of this chapter
we will continue to use data-directed programming, rather than message passing,
to discuss generic arithmetic operations.  In \link{Chapter 3} we will return to
message passing, and we will see that it can be a powerful tool for structuring
simulation programs.

\begin{quote}
\heading{\phantomsection\label{Exercise 2.75}Exercise 2.75:} Implement the constructor
\code{make\-/from\-/mag\-/ang} in message-passing style.  This procedure should be
analogous to the \code{make\-/from\-/real\-/imag} procedure given above.
\end{quote}

\begin{quote}
\heading{\phantomsection\label{Exercise 2.76}Exercise 2.76:} As a large system with generic
operations evolves, new types of data objects or new operations may be needed.
For each of the three strategies---generic operations with explicit dispatch,
data-directed style, and message-passing-style---describe the changes that
must be made to a system in order to add new types or new operations.  Which
organization would be most appropriate for a system in which new types must
often be added?  Which would be most appropriate for a system in which new
operations must often be added?
\end{quote}

\section{Systems with Generic Operations}
\label{Section 2.5}

In the previous section, we saw how to design systems in which data objects can
be represented in more than one way.  The key idea is to link the code that
specifies the data operations to the several representations by means of
generic interface procedures.  Now we will see how to use this same idea not
only to define operations that are generic over different representations but
also to define operations that are generic over different kinds of arguments.
We have already seen several different packages of arithmetic operations: the
primitive arithmetic (\code{+}, \code{-}, \code{*}, \code{/}) built into our
language, the rational-number arithmetic (\code{add\-/rat}, \code{sub\-/rat},
\code{mul\-/rat}, \code{div\-/rat}) of \link{Section 2.1.1}, and the complex-number
arithmetic that we implemented in \link{Section 2.4.3}.  We will now use
data-directed techniques to construct a package of arithmetic operations that
incorporates all the arithmetic packages we have already constructed.

\begin{figure}[tb]
\phantomsection\label{Figure 2.23}
\centering
\begin{comment}
\label{Figure 2.23}
\heading{Figure 2.23:} Generic arithmetic system.

\begin{example}
                        Programs that use numbers
                           +-----------------+
---------------------------| add sub mul div |-------------------
                           +-----------------+
                        Generic arithmetic package
 +-----------------+   +-------------------------+
 | add-rat sub-rat |   | add-complex sub-complex |   +---------+
-|                 |-+-|                         |-+-| + - * / |-
 | mul-rat div-rat | | | mul-complex div-complex | | +---------+
 +-----------------+ | +-------------------------+ |
      Rational       |     Complex artithmetic     |   Ordinary
     arithmetic      +--------------+--------------+  arithmetic
                     | Rectangular  |     Polar    |
---------------------+--------------+--------------+-------------
             List structure and primitive machine arithmetic
\end{example}
\end{comment}
\includegraphics[width=111mm]{fig/chap2/Fig2.23a.pdf}
\par\bigskip
\noindent
\heading{Figure 2.23:} Generic arithmetic system.
\end{figure}

\link{Figure 2.23} shows the structure of the system we shall build.  
Notice the
abstraction barriers.  From the perspective of someone using ``numbers,'' there
is a single procedure \code{add} that operates on whatever numbers are
supplied.  \code{Add} is part of a generic interface that allows the separate
ordinary-arithmetic, rational-arithmetic, and complex-arithmetic 
packages to be
accessed uniformly by programs that use numbers.  Any individual arithmetic
package (such as the complex package) may itself be accessed through generic
procedures (such as \code{add\-/complex}) that combine packages designed for
different representations (such as rectangular and polar).  Moreover, the
structure of the system is additive, so that one can design the individual
arithmetic packages separately and combine them to produce a generic arithmetic
system.



\subsection{Generic Arithmetic Operations}
\label{Section 2.5.1}

The task of designing generic arithmetic operations is analogous to that of
designing the generic complex-number operations.  We would like, for instance,
to have a generic addition procedure \code{add} that acts like ordinary
primitive addition \code{+} on ordinary numbers, like \code{add\-/rat} on
rational numbers, and like \code{add\-/complex} on complex numbers.  We can
implement \code{add}, and the other generic arithmetic operations, by following
the same strategy we used in \link{Section 2.4.3} to implement the generic
selectors for complex numbers.  We will attach a type tag to each kind of
number and cause the generic procedure to dispatch to an appropriate package
according to the data type of its arguments.

The generic arithmetic procedures are defined as follows:

\begin{scheme}
(define (add x y) (apply-generic 'add x y))
(define (sub x y) (apply-generic 'sub x y))
(define (mul x y) (apply-generic 'mul x y))
(define (div x y) (apply-generic 'div x y))
\end{scheme}

\noindent
We begin by installing a package for handling \newterm{ordinary} numbers, that
is, the primitive numbers of our language.  We will tag these with the symbol
\code{scheme\-/number}.  The arithmetic operations in this package are the
primitive arithmetic procedures (so there is no need to define extra procedures
to handle the untagged numbers).  Since these operations each take two
arguments, they are installed in the table keyed by the list
\code{(scheme\-/number scheme\-/number)}:

\begin{scheme}
(define (install-scheme-number-package)
  (define (tag x) (attach-tag 'scheme-number x))
  (put 'add '(scheme-number scheme-number)
       (lambda (x y) (tag (+ x y))))
  (put 'sub '(scheme-number scheme-number)
       (lambda (x y) (tag (- x y))))
  (put 'mul '(scheme-number scheme-number)
       (lambda (x y) (tag (* x y))))
  (put 'div '(scheme-number scheme-number)
       (lambda (x y) (tag (/ x y))))
  (put 'make 'scheme-number (lambda (x) (tag x)))
  'done)
\end{scheme}

\noindent
Users of the Scheme-number package will create (tagged) ordinary numbers by
means of the procedure:

\begin{scheme}
(define (make-scheme-number n)
  ((get 'make 'scheme-number) n))
\end{scheme}

\noindent
Now that the framework of the generic arithmetic system is in place, we can
readily include new kinds of numbers.  Here is a package that performs rational
arithmetic.  Notice that, as a benefit of additivity, we can use without
modification the rational-number code from \link{Section 2.1.1} as the internal
procedures in the package:

\begin{scheme}
(define (install-rational-package)
  ~\textrm{;; internal procedures}~
  (define (numer x) (car x))
  (define (denom x) (cdr x))
  (define (make-rat n d)
    (let ((g (gcd n d)))
      (cons (/ n g) (/ d g))))
  (define (add-rat x y)
    (make-rat (+ (* (numer x) (denom y))
                 (* (numer y) (denom x)))
              (* (denom x) (denom y))))
  (define (sub-rat x y)
    (make-rat (- (* (numer x) (denom y))
                 (* (numer y) (denom x)))
              (* (denom x) (denom y))))
  (define (mul-rat x y)
    (make-rat (* (numer x) (numer y))
              (* (denom x) (denom y))))
  (define (div-rat x y)
    (make-rat (* (numer x) (denom y))
              (* (denom x) (numer y))))
  ~\textrm{;; interface to rest of the system}~
  (define (tag x) (attach-tag 'rational x))
  (put 'add '(rational rational)
       (lambda (x y) (tag (add-rat x y))))
  (put 'sub '(rational rational)
       (lambda (x y) (tag (sub-rat x y))))
  (put 'mul '(rational rational)
       (lambda (x y) (tag (mul-rat x y))))
  (put 'div '(rational rational)
       (lambda (x y) (tag (div-rat x y))))
  (put 'make 'rational
       (lambda (n d) (tag (make-rat n d))))
  'done)
(define (make-rational n d)
  ((get 'make 'rational) n d))
\end{scheme}

\noindent
We can install a similar package to handle complex numbers, using the tag
\code{complex}.  In creating the package, we extract from the table the
operations \code{make\-/from\-/real\-/imag} and \code{make\-/from\-/mag\-/ang} that were
defined by the rectangular and polar packages.  Additivity permits us to use,
as the internal operations, the same \code{add\-/complex}, \code{sub\-/complex},
\code{mul\-/complex}, and \code{div\-/complex} procedures from \link{Section 2.4.1}.

\begin{scheme}
(define (install-complex-package)
  ~\textrm{;; imported procedures from rectangular and polar packages}~
  (define (make-from-real-imag x y)
    ((get 'make-from-real-imag 'rectangular) x y))
  (define (make-from-mag-ang r a)
    ((get 'make-from-mag-ang 'polar) r a))
  ~\textrm{;; internal procedures}~
  (define (add-complex z1 z2)
    (make-from-real-imag (+ (real-part z1) (real-part z2))
                         (+ (imag-part z1) (imag-part z2))))
  (define (sub-complex z1 z2)
    (make-from-real-imag (- (real-part z1) (real-part z2))
                         (- (imag-part z1) (imag-part z2))))
  (define (mul-complex z1 z2)
    (make-from-mag-ang (* (magnitude z1) (magnitude z2))
                       (+ (angle z1) (angle z2))))
  (define (div-complex z1 z2)
    (make-from-mag-ang (/ (magnitude z1) (magnitude z2))
                       (- (angle z1) (angle z2))))
  ~\textrm{;; interface to rest of the system}~
  (define (tag z) (attach-tag 'complex z))
  (put 'add '(complex complex)
       (lambda (z1 z2) (tag (add-complex z1 z2))))
  (put 'sub '(complex complex)
       (lambda (z1 z2) (tag (sub-complex z1 z2))))
  (put 'mul '(complex complex)
       (lambda (z1 z2) (tag (mul-complex z1 z2))))
  (put 'div '(complex complex)
       (lambda (z1 z2) (tag (div-complex z1 z2))))
  (put 'make-from-real-imag 'complex
       (lambda (x y) (tag (make-from-real-imag x y))))
  (put 'make-from-mag-ang 'complex
       (lambda (r a) (tag (make-from-mag-ang r a))))
  'done)
\end{scheme}

\noindent
Programs outside the complex-number package can construct complex numbers
either from real and imaginary parts or from magnitudes and angles.  Notice how
the underlying procedures, originally defined in the rectangular and polar
packages, are exported to the complex package, and exported from there to the
outside world.

\begin{scheme}
(define (make-complex-from-real-imag x y)
  ((get 'make-from-real-imag 'complex) x y))
(define (make-complex-from-mag-ang r a)
  ((get 'make-from-mag-ang 'complex) r a))
\end{scheme}

\noindent
What we have here is a two-level tag system.  A typical complex number, such as
\( 3 + 4i \) in rectangular form, would be represented as shown in \link{Figure 2.24}.  
The outer tag (\code{complex}) is used to direct the number to the
complex package.  Once within the complex package, the next tag
(\code{rectangular}) is used to direct the number to the rectangular package.
In a large and complicated system there might be many levels, each interfaced
with the next by means of generic operations.  As a data object is passed
``downward,'' the outer tag that is used to direct it to the appropriate
package is stripped off (by applying \code{contents}) and the next level of tag
(if any) becomes visible to be used for further dispatching.

\begin{figure}[tb]
\phantomsection\label{Figure 2.24}
\centering
\begin{comment}
\heading{Figure 2.24:} Representation of \( 3 + 4i \) in rectangular form.

\begin{example}
     +---+---+     +---+---+     +---+---+
---->| * | *-+---->| * | *-+---->| * | * |
     +-|-+---+     +-|-+---+     +-|-+-|-+
       |             |             |   |
       V             V             V   V
 +---------+   +-------------+  +---+ +---+
 | complex |   | rectangular |  | 3 | | 4 |
 +---------+   +-------------+  +---+ +---+
\end{example}
\end{comment}
\includegraphics[width=64mm]{fig/chap2/Fig2.24c.pdf}
\begin{quote}
\heading{Figure 2.24:} Representation of \( 3 + 4i \) in rectangular form.
\end{quote}
\end{figure}

In the above packages, we used \code{add\-/rat}, \code{add\-/complex}, and the
other arithmetic procedures exactly as originally written.  Once these
definitions are internal to different installation procedures, however, they no
longer need names that are distinct from each other: we could simply name them
\code{add}, \code{sub}, \code{mul}, and \code{div} in both packages.

\begin{quote}
\heading{\phantomsection\label{Exercise 2.77}Exercise 2.77:} Louis Reasoner tries to evaluate
the expression \code{(magnitude z)} where \code{z} is the object shown in
\link{Figure 2.24}.  To his surprise, instead of the answer 5 he gets an error
message from \code{apply\-/generic}, saying there is no method for the operation
\code{magnitude} on the types \code{(complex)}.  He shows this interaction to
Alyssa P. Hacker, who says ``The problem is that the complex-number selectors
were never defined for \code{complex} numbers, just for \code{polar} and
\code{rectangular} numbers.  All you have to do to make this work is add the
following to the \code{complex} package:''

\begin{scheme}
(put 'real-part '(complex) real-part)
(put 'imag-part '(complex) imag-part)
(put 'magnitude '(complex) magnitude)
(put 'angle '(complex) angle)
\end{scheme}

Describe in detail why this works.  As an example, trace through all the
procedures called in evaluating the expression \code{(magnitude z)} where
\code{z} is the object shown in \link{Figure 2.24}.  In particular, how many
times is \code{apply\-/generic} invoked?  What procedure is dispatched to in each
case?
\end{quote}

\begin{quote}
\heading{\phantomsection\label{Exercise 2.78}Exercise 2.78:} The internal procedures in the
\code{scheme\-/number} package are essentially nothing more than calls to the
primitive procedures \code{+}, \code{-}, etc.  It was not possible to use the
primitives of the language directly because our type-tag system requires that
each data object have a type attached to it.  In fact, however, all Lisp
implementations do have a type system, which they use internally.  Primitive
predicates such as \code{symbol?} and \code{number?}  determine whether data
objects have particular types.  Modify the definitions of \code{type\-/tag},
\code{contents}, and \code{attach\-/tag} from \link{Section 2.4.2} so that our
generic system takes advantage of Scheme's internal type system.  That is to
say, the system should work as before except that ordinary numbers should be
represented simply as Scheme numbers rather than as pairs whose \code{car} is
the symbol \code{scheme\-/number}.
\end{quote}

\begin{quote}
\heading{\phantomsection\label{Exercise 2.79}Exercise 2.79:} Define a generic equality
predicate \code{equ?} that tests the equality of two numbers, and install it in
the generic arithmetic package.  This operation should work for ordinary
numbers, rational numbers, and complex numbers.
\end{quote}

\begin{quote}
\heading{\phantomsection\label{Exercise 2.80}Exercise 2.80:} Define a generic predicate
\code{=zero?} that tests if its argument is zero, and install it in the generic
arithmetic package.  This operation should work for ordinary numbers, rational
numbers, and complex numbers.
\end{quote}

\subsection{Combining Data of Different Types}
\label{Section 2.5.2}

We have seen how to define a unified arithmetic system that encompasses
ordinary numbers, complex numbers, rational numbers, and any other type of
number we might decide to invent, but we have ignored an important issue.  The
operations we have defined so far treat the different data types as being
completely independent.  Thus, there are separate packages for adding, say, two
ordinary numbers, or two complex numbers.  What we have not yet considered is
the fact that it is meaningful to define operations that cross the type
boundaries, such as the addition of a complex number to an ordinary number.  We
have gone to great pains to introduce barriers between parts of our programs so
that they can be developed and understood separately.  We would like to
introduce the cross-type operations in some carefully controlled way, so that
we can support them without seriously violating our module boundaries.

One way to handle cross-type operations is to design a different procedure for
each possible combination of types for which the operation is valid.  For
example, we could extend the complex-number package so that it provides a
procedure for adding complex numbers to ordinary numbers and installs this in
the table using the tag \code{(complex scheme\-/number)}:\footnote{We also have
to supply an almost identical procedure to handle the types
\code{(scheme\-/number complex)}.}

\begin{scheme}
~\textrm{;; to be included in the complex package}~
(define (add-complex-to-schemenum z x)
  (make-from-real-imag (+ (real-part z) x) (imag-part z)))
(put 'add '(complex scheme-number)
     (lambda (z x) (tag (add-complex-to-schemenum z x))))
\end{scheme}

\noindent
This technique works, but it is cumbersome.  With such a system, the cost of
introducing a new type is not just the construction of the package of
procedures for that type but also the construction and installation of the
procedures that implement the cross-type operations.  This can easily be much
more code than is needed to define the operations on the type itself.  The
method also undermines our ability to combine separate packages additively, or
at least to limit the extent to which the implementors of the individual packages
need to take account of other packages.  For instance, in the example above, it
seems reasonable that handling mixed operations on complex numbers and ordinary
numbers should be the responsibility of the complex-number package.  Combining
rational numbers and complex numbers, however, might be done by the complex
package, by the rational package, or by some third package that uses operations
extracted from these two packages.  Formulating coherent policies on the
division of responsibility among packages can be an overwhelming task in
designing systems with many packages and many cross-type operations.

\subsubsection*{Coercion}

In the general situation of completely unrelated operations acting on
completely unrelated types, implementing explicit cross-type operations,
cumbersome though it may be, is the best that one can hope for.  Fortunately,
we can usually do better by taking advantage of additional structure that may
be latent in our type system.  Often the different data types are not
completely independent, and there may be ways by which objects of one type may
be viewed as being of another type.  This process is called \newterm{coercion}.
For example, if we are asked to arithmetically combine an ordinary number with
a complex number, we can view the ordinary number as a complex number whose
imaginary part is zero.  This transforms the problem to that of combining two
complex numbers, which can be handled in the ordinary way by the
complex-arithmetic package.

In general, we can implement this idea by designing coercion procedures that
transform an object of one type into an equivalent object of another type.
Here is a typical coercion procedure, which transforms a given ordinary number
to a complex number with that real part and zero imaginary part:

\begin{scheme}
(define (scheme-number->complex n)
  (make-complex-from-real-imag (contents n) 0))
\end{scheme}

\noindent
We install these coercion procedures in a special coercion table, indexed under
the names of the two types:

\begin{scheme}
(put-coercion 'scheme-number
              'complex 
              scheme-number->complex)
\end{scheme}

\noindent
(We assume that there are \code{put\-/coercion} and \code{get\-/coercion}
procedures available for manipulating this table.)  Generally some of the slots
in the table will be empty, because it is not generally possible to coerce an
arbitrary data object of each type into all other types.  For example, there is
no way to coerce an arbitrary complex number to an ordinary number, so there
will be no general \code{complex\-/>scheme\-/number} procedure included in the
table.

Once the coercion table has been set up, we can handle coercion in a uniform
manner by modifying the \code{apply\-/generic} procedure of \link{Section 2.4.3}.
When asked to apply an operation, we first check whether the operation is
defined for the arguments' types, just as before.  If so, we dispatch to the
procedure found in the operation-and-type table.  Otherwise, we try coercion.
For simplicity, we consider only the case where there are two
arguments.\footnote{See \link{Exercise 2.82} for generalizations.}  We check the
coercion table to see if objects of the first type can be coerced to the second
type.  If so, we coerce the first argument and try the operation again.  If
objects of the first type cannot in general be coerced to the second type, we
try the coercion the other way around to see if there is a way to coerce the
second argument to the type of the first argument.  Finally, if there is no
known way to coerce either type to the other type, we give up.  Here is the
procedure:

\begin{scheme}
(define (apply-generic op . args)
  (let ((type-tags (map type-tag args)))
    (let ((proc (get op type-tags)))
      (if proc
          (apply proc (map contents args))
          (if (= (length args) 2)
              (let ((type1 (car type-tags))
                    (type2 (cadr type-tags))
                    (a1 (car args))
                    (a2 (cadr args)))
                (let ((t1->t2 (get-coercion type1 type2))
                      (t2->t1 (get-coercion type2 type1)))
                  (cond (t1->t2 
                         (apply-generic op (t1->t2 a1) a2))
                        (t2->t1 
                         (apply-generic op a1 (t2->t1 a2)))
                        (else (error "No method for these types"
                                     (list op type-tags))))))
              (error "No method for these types"
                     (list op type-tags)))))))
\end{scheme}

\noindent
This coercion scheme has many advantages over the method of defining explicit
cross-type operations, as outlined above.  Although we still need to write
coercion procedures to relate the types (possibly \( n^2 \) procedures for a
system with \( n \) types), we need to write only one procedure for each pair of
types rather than a different procedure for each collection of types and each
generic operation.\footnote{If we are clever, we can usually get by with fewer
than \( n^2 \) coercion procedures.  For instance, if we know how to convert from
type 1 to type 2 and from type 2 to type 3, then we can use this knowledge to
convert from type 1 to type 3.  This can greatly decrease the number of
coercion procedures we need to supply explicitly when we add a new type to the
system.  If we are willing to build the required amount of sophistication into
our system, we can have it search the ``graph'' of relations among types and
automatically generate those coercion procedures that can be inferred from the
ones that are supplied explicitly.}  What we are counting on here is the fact
that the appropriate transformation between types depends only on the types
themselves, not on the operation to be applied.

On the other hand, there may be applications for which our coercion scheme is
not general enough.  Even when neither of the objects to be combined can be
converted to the type of the other it may still be possible to perform the
operation by converting both objects to a third type.  In order to deal with
such complexity and still preserve modularity in our programs, it is usually
necessary to build systems that take advantage of still further structure in
the relations among types, as we discuss next.

\subsubsection*{Hierarchies of types}

The coercion scheme presented above relied on the existence of natural
relations between pairs of types.  Often there is more ``global'' structure in
how the different types relate to each other.  For instance, suppose we are
building a generic arithmetic system to handle integers, rational numbers, real
numbers, and complex numbers.  In such a system, it is quite natural to regard
an integer as a special kind of rational number, which is in turn a special
kind of real number, which is in turn a special kind of complex number.  What
we actually have is a so-called \newterm{hierarchy of types}, in which, for
example, integers are a \newterm{subtype} of rational numbers (i.e., any
operation that can be applied to a rational number can automatically be applied
to an integer).  Conversely, we say that rational numbers form a
\newterm{supertype} of integers.  The particular hierarchy we have here is of a
very simple kind, in which each type has at most one supertype and at most one
subtype.  Such a structure, called a \newterm{tower}, is illustrated in
\link{Figure 2.25}.

\begin{figure}[tb]
\phantomsection\label{Figure 2.25}
\centering
\begin{comment}
\heading{Figure 2.25:} A tower of types.

\begin{example}
 complex
   ^
   |
  real
   ^
   |
rational
   ^
   |
integer
\end{example}
\end{comment}
\includegraphics[width=11mm]{fig/chap2/Fig2.25.pdf}
\par\bigskip
\noindent
\heading{Figure 2.25:} A tower of types.
\end{figure}

If we have a tower structure, then we can greatly simplify the problem of
adding a new type to the hierarchy, for we need only specify how the new type
is embedded in the next supertype above it and how it is the supertype of the
type below it.  For example, if we want to add an integer to a complex number,
we need not explicitly define a special coercion procedure
\code{integer\-/>complex}.  Instead, we define how an integer can be transformed
into a rational number, how a rational number is transformed into a real
number, and how a real number is transformed into a complex number.  We then
allow the system to transform the integer into a complex number through these
steps and then add the two complex numbers.

We can redesign our \code{apply\-/generic} procedure in the following way: For
each type, we need to supply a \code{raise} procedure, which ``raises'' objects
of that type one level in the tower.  Then when the system is required to
operate on objects of different types it can successively raise the lower types
until all the objects are at the same level in the tower.  (\link{Exercise 2.83}
and \link{Exercise 2.84} concern the details of implementing such a strategy.)

Another advantage of a tower is that we can easily implement the notion that
every type ``inherits'' all operations defined on a supertype.  For instance,
if we do not supply a special procedure for finding the real part of an
integer, we should nevertheless expect that \code{real\-/part} will be defined
for integers by virtue of the fact that integers are a subtype of complex
numbers.  In a tower, we can arrange for this to happen in a uniform way by
modifying \code{apply\-/generic}.  If the required operation is not directly
defined for the type of the object given, we raise the object to its supertype
and try again.  We thus crawl up the tower, transforming our argument as we go,
until we either find a level at which the desired operation can be performed or
hit the top (in which case we give up).

Yet another advantage of a tower over a more general hierarchy is that it gives
us a simple way to ``lower'' a data object to the simplest representation.  For
example, if we add \( 2 + 3i \) to \( 4 - 3i \), it would be nice to obtain the
answer as the integer 6 rather than as the complex number \( 6 + 0i \).
\link{Exercise 2.85} discusses a way to implement such a lowering operation.
(The trick is that we need a general way to distinguish those objects that can
be lowered, such as \( 6 + 0i \), from those that cannot, such as \( 6 + 2i \).)

\begin{figure}[tb]
\phantomsection\label{Figure 2.26}
\centering
\begin{comment}
\heading{Figure 2.26:} Relations among types of geometric figures.

\begin{example}
                     polygon
                    /       \
                   /         \
            triangle         quadrilateral
            /     \              /     \
           /       \            /       \
     isosceles   right      trapezoid   kite
     triangle    triangle       |         |
      |     \      |            |         |
      |      \     |            |         |
equilateral   isosceles   parallelogram   |
triangle      right          |       \    |
              triangle       |        \   |
                          rectangle  rhombus
                                \    /
                                 \  /
                                square
\end{example}
\end{comment}
\includegraphics[width=96mm]{fig/chap2/Fig2.26e.pdf}
\par\bigskip
\noindent
\heading{Figure 2.26:} Relations among types of geometric figures.
\end{figure}

\subsubsection*{Inadequacies of hierarchies}

\enlargethispage{\baselineskip}

If the data types in our system can be naturally arranged in a tower, this
greatly simplifies the problems of dealing with generic operations on different
types, as we have seen.  Unfortunately, this is usually not the case.
\link{Figure 2.26} illustrates a more complex arrangement of mixed types, this
one showing relations among different types of geometric figures.  We see that,
in general, a type may have more than one subtype.  Triangles and
quadrilaterals, for instance, are both subtypes of polygons.  In addition, a
type may have more than one supertype.  For example, an isosceles right
triangle may be regarded either as an isosceles triangle or as a right
triangle.  This multiple-supertypes issue is particularly thorny, since it
means that there is no unique way to ``raise'' a type in the hierarchy.
Finding the ``correct'' supertype in which to apply an operation to an object
may involve considerable searching through the entire type network on the part
of a procedure such as \code{apply\-/generic}.  Since there generally are
multiple subtypes for a type, there is a similar problem in coercing a value
``down'' the type hierarchy.  Dealing with large numbers of interrelated types
while still preserving modularity in the design of large systems is very
difficult, and is an area of much current research.\footnote{This statement,
which also appears in the first edition of this book, is just as true now as it
was when we wrote it twelve years ago.  Developing a useful, general framework
for expressing the relations among different types of entities (what
philosophers call ``ontology'') seems intractably difficult.  The main
difference between the confusion that existed ten years ago and the confusion
that exists now is that now a variety of inadequate ontological theories have
been embodied in a plethora of correspondingly inadequate programming
languages.  For example, much of the complexity of object-oriented programming
languages---and the subtle and confusing differences among contemporary
object-oriented languages---centers on the treatment of generic operations on
interrelated types.  Our own discussion of computational objects in
\link{Chapter 3} avoids these issues entirely.  Readers familiar with
object-oriented programming will notice that we have much to say in
chapter 3 about local state, but we do not even mention ``classes'' or
``inheritance.''  In fact, we suspect that these problems cannot be adequately
addressed in terms of computer-language design alone, without also drawing on
work in knowledge representation and automated reasoning.}

\begin{quote}
\heading{\phantomsection\label{Exercise 2.81}Exercise 2.81:} Louis Reasoner has noticed that
\code{apply\-/generic} may try to coerce the arguments to each other's type even
if they already have the same type.  Therefore, he reasons, we need to put
procedures in the coercion table to \newterm{coerce} arguments of each type to
their own type.  For example, in addition to the
\code{scheme\-/number\-/>complex} coercion shown above, he would do:

\begin{scheme}
(define (scheme-number->scheme-number n) n)
(define (complex->complex z) z)
(put-coercion 'scheme-number 
              'scheme-number
              scheme-number->scheme-number)
(put-coercion 'complex 'complex complex->complex)
\end{scheme}

\begin{enumerate}[a.]

\item
With Louis's coercion procedures installed, what happens if
\code{apply\-/generic} is called with two arguments of type \code{scheme\-/number}
or two arguments of type \code{complex} for an operation that is not found in
the table for those types?  For example, assume that we've defined a generic
exponentiation operation:

\begin{scheme}
(define (exp x y) (apply-generic 'exp x y))
\end{scheme}

\noindent
and have put a procedure for exponentiation in the Scheme-number
package but not in any other package:

\begin{scheme}
~\textrm{;; following added to Scheme-number package}~
(put 'exp '(scheme-number scheme-number)
     (lambda (x y) (tag (expt x y)))) 
     ~\textrm{; using primitive \code{expt}}~
\end{scheme}

\noindent
What happens if we call \code{exp} with two complex numbers as arguments?

\item
Is Louis correct that something had to be done about coercion with arguments of
the same type, or does \code{apply\-/generic} work correctly as is?

\item
Modify \code{apply\-/generic} so that it doesn't try coercion if the two
arguments have the same type.

\end{enumerate}
\end{quote}

\begin{quote}
\heading{\phantomsection\label{Exercise 2.82}Exercise 2.82:} Show how to generalize
\code{apply\-/generic} to handle coercion in the general case of multiple
arguments.  One strategy is to attempt to coerce all the arguments to the type
of the first argument, then to the type of the second argument, and so on.
Give an example of a situation where this strategy (and likewise the
two-argument version given above) is not sufficiently general.  (Hint: Consider
the case where there are some suitable mixed-type operations present in the
table that will not be tried.)
\end{quote}

\begin{quote}
\heading{\phantomsection\label{Exercise 2.83}Exercise 2.83:} Suppose you are designing a
generic arithmetic system for dealing with the tower of types shown in
\link{Figure 2.25}: integer, rational, real, complex.  For each type (except
complex), design a procedure that raises objects of that type one level in the
tower.  Show how to install a generic \code{raise} operation that will work for
each type (except complex).
\end{quote}

\begin{quote}
\heading{\phantomsection\label{Exercise 2.84}Exercise 2.84:} Using the \code{raise} operation
of \link{Exercise 2.83}, modify the \code{apply\-/generic} procedure so that it
coerces its arguments to have the same type by the method of successive
raising, as discussed in this section.  You will need to devise a way to test
which of two types is higher in the tower.  Do this in a manner that is
``compatible'' with the rest of the system and will not lead to problems in
adding new levels to the tower.
\end{quote}

\begin{quote}
\heading{\phantomsection\label{Exercise 2.85}Exercise 2.85:} This section mentioned a method
for ``simplifying'' a data object by lowering it in the tower of types as far
as possible.  Design a procedure \code{drop} that accomplishes this for the
tower described in \link{Exercise 2.83}.  The key is to decide, in some general
way, whether an object can be lowered.  For example, the complex number 
\( 1.5 + 0i \) can be lowered as far as \code{real}, the complex number \( 1 + 0i \) can
be lowered as far as \code{integer}, and the complex number \( 2 + 3i \) cannot
be lowered at all.  Here is a plan for determining whether an object can be
lowered: Begin by defining a generic operation \code{project} that ``pushes''
an object down in the tower.  For example, projecting a complex number would
involve throwing away the imaginary part.  Then a number can be dropped if,
when we \code{project} it and \code{raise} the result back to the type we
started with, we end up with something equal to what we started with.  Show how
to implement this idea in detail, by writing a \code{drop} procedure that drops
an object as far as possible.  You will need to design the various projection
operations\footnote{A real number can be projected to an integer using the
\code{round} primitive, which returns the closest integer to its argument.} and
install \code{project} as a generic operation in the system.  You will also
need to make use of a generic equality predicate, such as described in
\link{Exercise 2.79}.  Finally, use \code{drop} to rewrite \code{apply\-/generic}
from \link{Exercise 2.84} so that it ``simplifies'' its answers.
\end{quote}

\begin{quote}
\heading{\phantomsection\label{Exercise 2.86}Exercise 2.86:} Suppose we want to handle complex
numbers whose real parts, imaginary parts, magnitudes, and angles can be either
ordinary numbers, rational numbers, or other numbers we might wish to add to
the system.  Describe and implement the changes to the system needed to
accommodate this.  You will have to define operations such as \code{sine} and
\code{cosine} that are generic over ordinary numbers and rational numbers.
\end{quote}

\subsection{Example: Symbolic Algebra}
\label{Section 2.5.3}

The manipulation of symbolic algebraic expressions is a complex process that
illustrates many of the hardest problems that occur in the design of
large-scale systems.  An algebraic expression, in general, can be viewed as a
hierarchical structure, a tree of operators applied to operands.  We can
construct algebraic expressions by starting with a set of primitive objects,
such as constants and variables, and combining these by means of algebraic
operators, such as addition and multiplication.  As in other languages, we form
abstractions that enable us to refer to compound objects in simple terms.
Typical abstractions in symbolic algebra are ideas such as linear combination,
polynomial, rational function, or trigonometric function.  We can regard these
as compound ``types,'' which are often useful for directing the processing of
expressions.  For example, we could describe the expression
\begin{comment}

\begin{example}
x^2 sin (y^2 + 1) + x cos 2y + cos(y^3 - 2y^2)
\end{example}

\end{comment}
\begin{displaymath}
 x^2 \sin (y^2 + 1) + x \cos 2y + \cos(y^3 - 2y^2) 
\end{displaymath}
\noindent
as a polynomial in \( x \) with coefficients that are trigonometric functions of
polynomials in \( y \) whose coefficients are integers.

We will not attempt to develop a complete algebraic-manipulation system here.
Such systems are exceedingly complex programs, embodying deep algebraic
knowledge and elegant algorithms.  What we will do is look at a simple but
important part of algebraic manipulation: the arithmetic of polynomials.  We
will illustrate the kinds of decisions the designer of such a system faces, and
how to apply the ideas of abstract data and generic operations to help organize
this effort.

\subsubsection*{Arithmetic on polynomials}

Our first task in designing a system for performing arithmetic on polynomials
is to decide just what a polynomial is.  Polynomials are normally defined
relative to certain variables (the \newterm{indeterminates} of the polynomial).
For simplicity, we will restrict ourselves to polynomials having just one
indeterminate (\newterm{univariate polynomials}).\footnote{On the other hand,
we will allow polynomials whose coefficients are themselves polynomials in
other variables.  This will give us essentially the same representational power
as a full multivariate system, although it does lead to coercion problems, as
discussed below.} We will define a polynomial to be a sum of terms, each of
which is either a coefficient, a power of the indeterminate, or a product of a
coefficient and a power of the indeterminate.  A coefficient is defined as an
algebraic expression that is not dependent upon the indeterminate of the
polynomial.  For example,
\begin{comment}

\begin{example}
5x^2 + 3x + 7
\end{example}

\end{comment}
\begin{displaymath}
 5x^2 + 3x + 7 
\end{displaymath}
\noindent
is a simple polynomial in \( x \), and
\begin{comment}

\begin{example}
(y^2 + 1)x^3 + (2y)x + 1
\end{example}

\end{comment}
\begin{displaymath}
 (y^2 + 1)x^3 + (2y)x + 1 
\end{displaymath}
\noindent
is a polynomial in \( x \) whose coefficients are polynomials in \( y \).

Already we are skirting some thorny issues.  Is the first of these polynomials
the same as the polynomial \( 5y^2 + 3y + 7 \), or not?  A reasonable answer
might be ``yes, if we are considering a polynomial purely as a mathematical
function, but no, if we are considering a polynomial to be a syntactic form.''
The second polynomial is algebraically equivalent to a polynomial in \( y \)
whose coefficients are polynomials in \( x \).  Should our system recognize this,
or not?  Furthermore, there are other ways to represent a polynomial---for
example, as a product of factors, or (for a univariate polynomial) as the set
of roots, or as a listing of the values of the polynomial at a specified set of
points.\footnote{For univariate polynomials, giving the value of a polynomial
at a given set of points can be a particularly good representation.  This makes
polynomial arithmetic extremely simple.  To obtain, for example, the sum of two
polynomials represented in this way, we need only add the values of the
polynomials at corresponding points.  To transform back to a more familiar
representation, we can use the Lagrange interpolation formula, which shows how
to recover the coefficients of a polynomial of degree \( n \) given the values of
the polynomial at \( n + 1 \) points.}  We can finesse these questions by
deciding that in our algebraic-manipulation system a ``polynomial'' will be a
particular syntactic form, not its underlying mathematical meaning.

Now we must consider how to go about doing arithmetic on polynomials.  In this
simple system, we will consider only addition and multiplication.  Moreover, we
will insist that two polynomials to be combined must have the same
indeterminate.

We will approach the design of our system by following the familiar discipline
of data abstraction.  We will represent polynomials using a data structure
called a \newterm{poly}, which consists of a variable and a collection of
terms.  We assume that we have selectors \code{variable} and \code{term\-/list}
that extract those parts from a poly and a constructor \code{make\-/poly} that
assembles a poly from a given variable and a term list.  A variable will be
just a symbol, so we can use the \code{same\-/variable?}  procedure of 
\link{Section 2.3.2} to compare variables.  The following procedures define addition and
multiplication of polys:

\begin{scheme}
(define (add-poly p1 p2)
  (if (same-variable? (variable p1) (variable p2))
      (make-poly (variable p1)
                 (add-terms (term-list p1) (term-list p2)))
      (error "Polys not in same var: ADD-POLY" (list p1 p2))))
(define (mul-poly p1 p2)
  (if (same-variable? (variable p1) (variable p2))
      (make-poly (variable p1)
                 (mul-terms (term-list p1) (term-list p2)))
      (error "Polys not in same var: MUL-POLY" (list p1 p2))))
\end{scheme}

\noindent
To incorporate polynomials into our generic arithmetic system, we need to
supply them with type tags.  We'll use the tag \code{polynomial}, and install
appropriate operations on tagged polynomials in the operation table.  We'll
embed all our code in an installation procedure for the polynomial package,
similar to the ones in \link{Section 2.5.1}:

\begin{scheme}
(define (install-polynomial-package)
  ~\textrm{;; internal procedures}~
  ~\textrm{;; representation of poly}~
  (define (make-poly variable term-list) (cons variable term-list))
  (define (variable p) (car p))
  (define (term-list p) (cdr p))
  ~\( \dark \langle \)~~\emph{procedures \emph{same-variable?} and \emph{variable?} from section 2.3.2}~~\( \dark \rangle \)~
  ~\textrm{;; representation of terms and term lists}~
  ~\( \dark \langle \)~~\emph{procedures \emph{adjoin-term} \( \dots \) \emph{coeff} from text below}~~\( \dark \rangle \)~
  (define (add-poly p1 p2) ~\( \dots \)~)
  ~\( \dark \langle \)~~\emph{procedures used by \emph{add-poly}}~~\( \dark \rangle \)~
  (define (mul-poly p1 p2) ~\( \dots \)~)
  ~\( \dark \langle \)~~\emph{procedures used by \emph{mul-poly}}~~\( \dark \rangle \)~
  ~\textrm{;; interface to rest of the system}~
  (define (tag p) (attach-tag 'polynomial p))
  (put 'add '(polynomial polynomial)
       (lambda (p1 p2) (tag (add-poly p1 p2))))
  (put 'mul '(polynomial polynomial)
       (lambda (p1 p2) (tag (mul-poly p1 p2))))
  (put 'make 'polynomial
       (lambda (var terms) (tag (make-poly var terms))))
  'done)
\end{scheme}

\noindent
Polynomial addition is performed termwise.  Terms of the same order (i.e., with
the same power of the indeterminate) must be combined.  This is done by forming
a new term of the same order whose coefficient is the sum of the coefficients
of the addends.  Terms in one addend for which there are no terms of the same
order in the other addend are simply accumulated into the sum polynomial being
constructed.

In order to manipulate term lists, we will assume that we have a constructor
\code{the\-/empty\-/termlist} that returns an empty term list and a constructor
\code{adjoin\-/term} that adjoins a new term to a term list.  We will also assume
that we have a predicate \code{empty\-/termlist?} that tells if a given term list
is empty, a selector \code{first\-/term} that extracts the highest-order term
from a term list, and a selector \code{rest\-/terms} that returns all but the
highest-order term.  To manipulate terms, we will suppose that we have a
constructor \code{make\-/term} that constructs a term with given order and
coefficient, and selectors \code{order} and \code{coeff} that return,
respectively, the order and the coefficient of the term.  These operations
allow us to consider both terms and term lists as data abstractions, whose
concrete representations we can worry about separately.

Here is the procedure that constructs the term list for the sum of two
polynomials:\footnote{This operation is very much like the ordered
\code{union\-/set} operation we developed in \link{Exercise 2.62}.  In
fact, if we think of the terms of the polynomial as a set ordered according to
the power of the indeterminate, then the program that produces the term list
for a sum is almost identical to \code{union\-/set}.}

\begin{scheme}
(define (add-terms L1 L2)
  (cond ((empty-termlist? L1) L2)
        ((empty-termlist? L2) L1)
        (else
         (let ((t1 (first-term L1)) 
               (t2 (first-term L2)))
           (cond ((> (order t1) (order t2))
                  (adjoin-term
                   t1 (add-terms (rest-terms L1) L2)))
                 ((< (order t1) (order t2))
                  (adjoin-term
                   t2 (add-terms L1 (rest-terms L2))))
                 (else
                  (adjoin-term
                   (make-term (order t1)
                              (add (coeff t1) (coeff t2)))
                   (add-terms (rest-terms L1)
                              (rest-terms L2)))))))))
\end{scheme}

\noindent
The most important point to note here is that we used the generic addition
procedure \code{add} to add together the coefficients of the terms being
combined.  This has powerful consequences, as we will see below.

In order to multiply two term lists, we multiply each term of the first list by
all the terms of the other list, repeatedly using \code{mul\-/term\-/by\-/all\-/terms},
which multiplies a given term by all terms in a given term list.  The resulting
term lists (one for each term of the first list) are accumulated into a sum.
Multiplying two terms forms a term whose order is the sum of the orders of the
factors and whose coefficient is the product of the coefficients of the
factors:

\begin{scheme}
(define (mul-terms L1 L2)
  (if (empty-termlist? L1)
      (the-empty-termlist)
      (add-terms (mul-term-by-all-terms (first-term L1) L2)
                 (mul-terms (rest-terms L1) L2))))
(define (mul-term-by-all-terms t1 L)
  (if (empty-termlist? L)
      (the-empty-termlist)
      (let ((t2 (first-term L)))
        (adjoin-term
         (make-term (+ (order t1) (order t2))
                    (mul (coeff t1) (coeff t2)))
         (mul-term-by-all-terms t1 (rest-terms L))))))
\end{scheme}

\noindent
This is really all there is to polynomial addition and multiplication.  Notice
that, since we operate on terms using the generic procedures \code{add} and
\code{mul}, our polynomial package is automatically able to handle any type of
coefficient that is known about by the generic arithmetic package.  If we
include a coercion mechanism such as one of those discussed in 
\link{Section 2.5.2}, then we also are automatically able to handle operations on
polynomials of different coefficient types, such as
\begin{comment}

\begin{example}
                         /        2                 \
[3x^2 + (2 + 3i)x + 7] * | x^4 + --- x^2 + (5 + 3i) |
                         \        3                 /
\end{example}

\end{comment}
\begin{displaymath}
 [3x^2 + (2 + 3i)x + 7] \cdot \! \left[ x^4 + {2\over3} x^2 + (5 + 3i) \right]\!. 
\end{displaymath}
Because we installed the polynomial addition and multiplication procedures
\code{add\-/poly} and \code{mul\-/poly} in the generic arithmetic system as the
\code{add} and \code{mul} operations for type \code{polynomial}, our system is
also automatically able to handle polynomial operations such as
\begin{comment}

\begin{example}
[(y + 1)x^2 + (y^2 + 1)x + (y - 1)] * [(y - 2)x + (y^3 + 7)]
\end{example}

\end{comment}
\begin{displaymath}
 \Big[(y + 1)x^2 + (y^2 + 1)x + (y - 1)\Big] \cdot \Big[(y - 2)x + (y^3 + 7)\Big]\!. 
\end{displaymath}
The reason is that when the system tries to combine coefficients, it will
dispatch through \code{add} and \code{mul}.  Since the coefficients are
themselves polynomials (in \( y \)), these will be combined using \code{add\-/poly}
and \code{mul\-/poly}.  The result is a kind of ``data-directed recursion'' in
which, for example, a call to \code{mul\-/poly} will result in recursive calls to
\code{mul\-/poly} in order to multiply the coefficients.  If the coefficients of
the coefficients were themselves polynomials (as might be used to represent
polynomials in three variables), the data direction would ensure that the
system would follow through another level of recursive calls, and so on through
as many levels as the structure of the data dictates.\footnote{To make this
work completely smoothly, we should also add to our generic arithmetic system
the ability to coerce a ``number'' to a polynomial by regarding it as a
polynomial of degree zero whose coefficient is the number.  This is necessary
if we are going to perform operations such as
\begin{comment}

\begin{example}
[x^2 + (y + 1)x + 5] + [x^2 + 2x + 1]
\end{example}

\end{comment}
\begin{displaymath}
 [x^2 + (y + 1)x + 5] + [x^2 + 2x + 1], 
\end{displaymath}
\noindent
which requires adding the coefficient \( y + 1 \) to the coefficient 2.}

\subsubsection*{Representing term lists}

Finally, we must confront the job of implementing a good representation for
term lists.  A term list is, in effect, a set of coefficients keyed by the
order of the term.  Hence, any of the methods for representing sets, as
discussed in \link{Section 2.3.3}, can be applied to this task.  On the other
hand, our procedures \code{add\-/terms} and \code{mul\-/terms} always access term
lists sequentially from highest to lowest order.  Thus, we will use some kind
of ordered list representation.

How should we structure the list that represents a term list?  One
consideration is the ``density'' of the polynomials we intend to manipulate.  A
polynomial is said to be \newterm{dense} if it has nonzero coefficients in
terms of most orders.  If it has many zero terms it is said to be
\newterm{sparse}.  For example,
\begin{comment}

\begin{example}
A : x^5 + 2x^4 + 3x^2 - 2x - 5
\end{example}

\end{comment}
\begin{displaymath}
 A: \quad x^5 + 2x^4 + 3x^2 - 2x - 5 
\end{displaymath}
\noindent
is a dense polynomial, whereas
\begin{comment}

\begin{example}
B : x^100 + 2x^2 + 1
\end{example}

\end{comment}
\begin{displaymath}
 B: \quad x^{100} + 2x^2 + 1 
\end{displaymath}
\noindent
is sparse.

The term lists of dense polynomials are most efficiently represented as lists
of the coefficients.  For example, \( A \) above would be nicely represented as
\code{(1 2 0 3 -2 -5)}.  The order of a term in this representation is the
length of the sublist beginning with that term's coefficient, decremented by
1.\footnote{In these polynomial examples, we assume that we have implemented
the generic arithmetic system using the type mechanism suggested in
\link{Exercise 2.78}.  Thus, coefficients that are ordinary numbers will be
represented as the numbers themselves rather than as pairs whose \code{car} is
the symbol \code{scheme\-/number}.}  This would be a terrible representation for
a sparse polynomial such as \( B \): There would be a giant list of zeros
punctuated by a few lonely nonzero terms.  A more reasonable representation of
the term list of a sparse polynomial is as a list of the nonzero terms, where
each term is a list containing the order of the term and the coefficient for
that order.  In such a scheme, polynomial \( B \) is efficiently represented as
\code{((100 1) (2 2) (0 1))}.  As most polynomial manipulations are performed
on sparse polynomials, we will use this method.  We will assume that term lists
are represented as lists of terms, arranged from highest-order to lowest-order
term.  Once we have made this decision, implementing the selectors and
constructors for terms and term lists is straightforward:\footnote{Although we
are assuming that term lists are ordered, we have implemented
\code{adjoin\-/term} to simply \code{cons} the new term onto the existing term
list.  We can get away with this so long as we guarantee that the procedures
(such as \code{add\-/terms}) that use \code{adjoin\-/term} always call it with a
higher-order term than appears in the list.  If we did not want to make such a
guarantee, we could have implemented \code{adjoin\-/term} to be similar to the
\code{adjoin\-/set} constructor for the ordered-list representation of sets
(\link{Exercise 2.61}).}

\begin{scheme}
(define (adjoin-term term term-list)
  (if (=zero? (coeff term))
      term-list
      (cons term term-list)))
(define (the-empty-termlist) '())
(define (first-term term-list) (car term-list))
(define (rest-terms term-list) (cdr term-list))
(define (empty-termlist? term-list) (null? term-list))
(define (make-term order coeff) (list order coeff))
(define (order term) (car term))
(define (coeff term) (cadr term))
\end{scheme}

\noindent
where \code{=zero?} is as defined in \link{Exercise 2.80}.  (See also
\link{Exercise 2.87} below.)

Users of the polynomial package will create (tagged) polynomials by means of
the procedure:

\begin{scheme}
(define (make-polynomial var terms)
  ((get 'make 'polynomial) var terms))
\end{scheme}

\begin{quote}
\heading{\phantomsection\label{Exercise 2.87}Exercise 2.87:} Install \code{=zero?} for
polynomials in the generic arithmetic package.  This will allow
\code{adjoin\-/term} to work for polynomials with coefficients that are
themselves polynomials.
\end{quote}

\begin{quote}
\heading{\phantomsection\label{Exercise 2.88}Exercise 2.88:} Extend the polynomial system to
include subtraction of polynomials.  (Hint: You may find it helpful to define a
generic negation operation.)
\end{quote}

\begin{quote}
\heading{\phantomsection\label{Exercise 2.89}Exercise 2.89:} Define procedures that implement
the term-list representation described above as appropriate for dense
polynomials.
\end{quote}

\begin{quote}
\heading{\phantomsection\label{Exercise 2.90}Exercise 2.90:} Suppose we want to have a
polynomial system that is efficient for both sparse and dense polynomials.  One
way to do this is to allow both kinds of term-list representations in our
system.  The situation is analogous to the complex-number example of 
\link{Section 2.4}, where we allowed both rectangular and polar representations.  To do
this we must distinguish different types of term lists and make the operations
on term lists generic.  Redesign the polynomial system to implement this
generalization.  This is a major effort, not a local change.
\end{quote}

\begin{quote}
\heading{\phantomsection\label{Exercise 2.91}Exercise 2.91:} A univariate polynomial can be
divided by another one to produce a polynomial quotient and a polynomial
remainder.  For example,
\begin{comment}

\begin{example}
x^5 - 1
------- = x^3 + x, remainder x - 1
x^2 - 1
\end{example}

\end{comment}
\begin{displaymath}
 {x^5 - 1 \over x^2 - 1} = x^3 + x, \hbox{  remainder  } x - 1. 
\end{displaymath}
Division can be performed via long division.  That is, divide the highest-order
term of the dividend by the highest-order term of the divisor.  The result is
the first term of the quotient.  Next, multiply the result by the divisor,
subtract that from the dividend, and produce the rest of the answer by
recursively dividing the difference by the divisor.  Stop when the order of the
divisor exceeds the order of the dividend and declare the dividend to be the
remainder.  Also, if the dividend ever becomes zero, return zero as both
quotient and remainder.

We can design a \code{div\-/poly} procedure on the model of \code{add\-/poly} and
\code{mul\-/poly}. The procedure checks to see if the two polys have the same
variable.  If so, \code{div\-/poly} strips off the variable and passes the
problem to \code{div\-/terms}, which performs the division operation on term
lists. \code{Div\-/poly} finally reattaches the variable to the result supplied
by \code{div\-/terms}.  It is convenient to design \code{div\-/terms} to compute
both the quotient and the remainder of a division.  \code{Div\-/terms} can take
two term lists as arguments and return a list of the quotient term list and the
remainder term list.

Complete the following definition of \code{div\-/terms} by filling in the missing
expressions.  Use this to implement \code{div\-/poly}, which takes two polys as
arguments and returns a list of the quotient and remainder polys.

\begin{smallscheme}
(define (div-terms L1 L2)
  (if (empty-termlist? L1)
      (list (the-empty-termlist) (the-empty-termlist))
      (let ((t1 (first-term L1))
            (t2 (first-term L2)))
        (if (> (order t2) (order t1))
            (list (the-empty-termlist) L1)
            (let ((new-c (div (coeff t1) (coeff t2)))
                  (new-o (- (order t1) (order t2))))
              (let ((rest-of-result
                     ~\( \langle \)~~\var{compute rest of result recursively}~~\( \rangle \)~ ))
                ~\( \langle \)~~\var{form complete result}~~\( \rangle \)~ ))))))
\end{smallscheme}
\end{quote}

\subsubsection*{Hierarchies of types in symbolic algebra}

Our polynomial system illustrates how objects of one type (polynomials) may in
fact be complex objects that have objects of many different types as parts.
This poses no real difficulty in defining generic operations.  We need only
install appropriate generic operations for performing the necessary
manipulations of the parts of the compound types.  In fact, we saw that
polynomials form a kind of ``recursive data abstraction,'' in that parts of a
polynomial may themselves be polynomials.  Our generic operations and our
data-directed programming style can handle this complication without much
trouble.

On the other hand, polynomial algebra is a system for which the data types
cannot be naturally arranged in a tower.  For instance, it is possible to have
polynomials in \( x \) whose coefficients are polynomials in \( y \).  It is also
possible to have polynomials in \( y \) whose coefficients are polynomials in
\( x \).  Neither of these types is ``above'' the other in any natural way, yet
it is often necessary to add together elements from each set.  There are
several ways to do this.  One possibility is to convert one polynomial to the
type of the other by expanding and rearranging terms so that both polynomials
have the same principal variable.  One can impose a towerlike structure on this
by ordering the variables and thus always converting any polynomial to a
``canonical form'' with the highest-priority variable dominant and the
lower-priority variables buried in the coefficients.  This strategy works
fairly well, except that the conversion may expand a polynomial unnecessarily,
making it hard to read and perhaps less efficient to work with.  The tower
strategy is certainly not natural for this domain or for any domain where the
user can invent new types dynamically using old types in various combining
forms, such as trigonometric functions, power series, and integrals.

It should not be surprising that controlling coercion is a serious problem in
the design of large-scale algebraic-manipulation systems.  Much of the
complexity of such systems is concerned with relationships among diverse types.
Indeed, it is fair to say that we do not yet completely understand coercion.
In fact, we do not yet completely understand the concept of a data type.
Nevertheless, what we know provides us with powerful structuring and modularity
principles to support the design of large systems.

\begin{quote}
\heading{\phantomsection\label{Exercise 2.92}Exercise 2.92:} By imposing an ordering on
variables, extend the polynomial package so that addition and multiplication of
polynomials works for polynomials in different variables.  (This is not easy!)
\end{quote}

\subsubsection*{Extended exercise: Rational functions}

We can extend our generic arithmetic system to include \newterm{rational
functions}.  These are ``fractions'' whose numerator and denominator are
polynomials, such as
\begin{comment}

\begin{example}
 x + 1
-------
x^3 - 1
\end{example}

\end{comment}
\begin{displaymath}
 {x + 1 \over x^3 - 1}\,. 
\end{displaymath}
The system should be able to add, subtract, multiply, and divide rational
functions, and to perform such computations as
\begin{comment}

\begin{example}
 x + 1       x      x^3 + 2x^2 + 3x + 1
------- + ------- = -------------------
x^3 - 1   x^2 - 1    x^4 + x^3 - x - 1
\end{example}

\end{comment}
\begin{displaymath}
 {x + 1 \over x^3 - 1} + {x \over x^2 - 1} = 
	{x^3 + 2x^2 + 3x + 1 \over x^4 + x^3 - x - 1}\,. 
\end{displaymath}
\noindent
(Here the sum has been simplified by removing common factors.  Ordinary ``cross
multiplication'' would have produced a fourth-degree polynomial over a
fifth-degree polynomial.)

If we modify our rational-arithmetic package so that it uses generic
operations, then it will do what we want, except for the problem of reducing
fractions to lowest terms.

\begin{quote}
\heading{\phantomsection\label{Exercise 2.93}Exercise 2.93:} Modify the rational-arithmetic
package to use generic operations, but change \code{make\-/rat} so that it does
not attempt to reduce fractions to lowest terms.  Test your system by calling
\code{make\-/rational} on two polynomials to produce a rational function:

\begin{scheme}
(define p1 (make-polynomial 'x '((2 1) (0 1))))
(define p2 (make-polynomial 'x '((3 1) (0 1))))
(define rf (make-rational p2 p1))
\end{scheme}

Now add \code{rf} to itself, using \code{add}. You will observe that this
addition procedure does not reduce fractions to lowest terms.
\end{quote}

\noindent
We can reduce polynomial fractions to lowest terms using the same idea we used
with integers: modifying \code{make\-/rat} to divide both the numerator and the
denominator by their greatest common divisor.  The notion of ``greatest common
divisor'' makes sense for polynomials.  In fact, we can compute the
\acronym{GCD} of two polynomials using essentially the same Euclid's Algorithm
that works for integers.\footnote{The fact that Euclid's Algorithm works for
polynomials is formalized in algebra by saying that polynomials form a kind of
algebraic domain called a \newterm{Euclidean ring}.  A Euclidean ring is a
domain that admits addition, subtraction, and commutative multiplication,
together with a way of assigning to each element \( x \) of the ring a positive
integer ``measure'' \( m(x) \) with the properties that 
\( m(xy) \ge m(x) \) for any nonzero \( x \) and \( y \) and that, given any \( x \) and
\( y \), there exists a \( q \) such that \( y = qx + r \) and either
\( r = 0 \) or \( m(r) < m(x) \).  From an abstract point of view, this
is what is needed to prove that Euclid's Algorithm works.  For the domain of
integers, the measure \( m \) of an integer is the absolute value of the integer
itself.  For the domain of polynomials, the measure of a polynomial is its
degree.}  The integer version is

\begin{scheme}
(define (gcd a b)
  (if (= b 0)
      a
      (gcd b (remainder a b))))
\end{scheme}

\noindent
Using this, we could make the obvious modification to define a \acronym{GCD}
operation that works on term lists:

\begin{scheme}
(define (gcd-terms a b)
  (if (empty-termlist? b)
      a
      (gcd-terms b (remainder-terms a b))))
\end{scheme}

\noindent
where \code{remainder\-/terms} picks out the remainder component of the list
returned by the term-list division operation \code{div\-/terms} that was
implemented in \link{Exercise 2.91}.

\begin{quote}
\heading{\phantomsection\label{Exercise 2.94}Exercise 2.94:} Using \code{div\-/terms}, implement
the procedure \code{remainder\-/terms} and use this to define \code{gcd\-/terms} as
above.  Now write a procedure \code{gcd\-/poly} that computes the polynomial
\acronym{GCD} of two polys.  (The procedure should signal an error if the two
polys are not in the same variable.)  Install in the system a generic operation
\code{greatest\-/common\-/divisor} that reduces to \code{gcd\-/poly} for polynomials
and to ordinary \code{gcd} for ordinary numbers.  As a test, try

\begin{scheme}
(define p1 (make-polynomial 
            'x '((4 1) (3 -1) (2 -2) (1 2))))
(define p2 (make-polynomial 'x '((3 1) (1 -1))))
(greatest-common-divisor p1 p2)
\end{scheme}

\noindent
and check your result by hand.
\end{quote}

\begin{quote}
\heading{\phantomsection\label{Exercise 2.95}Exercise 2.95:} Define \( P_1 \), \( P_2 \), and
\( P_3 \) to be the polynomials
\begin{comment}

\begin{example}
P_1 : x^2 - 2x + 1

P_2 : 11x^2 + 7

P_3 : 13x + 5
\end{example}

\end{comment}
\begin{displaymath}
%  \eqalign{
% 	P_1 	&: \quad x^2 - 2x + 1, \cr
% 	P_2 	&: \quad 11x^2 + 7, \cr
% 	P_3 	&: \quad 13x + 5. \cr
% } 
\begin{array}{l@{{}:}l}
	P_1 	& \quad x^2 - 2x + 1, \\
	P_2 	& \quad 11x^2 + 7, \\
	P_3 	& \quad 13x + 5. 
\end{array}
\end{displaymath}
Now define \( Q_1 \) to be the product of \( P_1 \) and \( P_2 \) and \( Q_2 \) to be
the product of \( P_1 \) and \( P_3 \), and use \code{greatest\-/common\-/divisor}
(\link{Exercise 2.94}) to compute the \acronym{GCD} of \( Q_1 \) 
and \( Q_2 \).
Note that the answer is not the same as \( P_1 \).  This example introduces
noninteger operations into the computation, causing difficulties with the
\acronym{GCD} algorithm.\footnote{In an implementation like \acronym{MIT}
Scheme, this produces a polynomial that is indeed a divisor of \( Q_1 \) and
\( Q_2 \), but with rational coefficients.  In many other Scheme systems, in
which division of integers can produce limited-precision decimal numbers, we
may fail to get a valid divisor.}  To understand what is happening, try tracing
\code{gcd\-/terms} while computing the \acronym{GCD} or try performing the
division by hand.
\end{quote}

\noindent
We can solve the problem exhibited in \link{Exercise 2.95} if we use the
following modification of the \acronym{GCD} algorithm (which really works only
in the case of polynomials with integer coefficients).  Before performing any
polynomial division in the \acronym{GCD} computation, we multiply the dividend
by an integer constant factor, chosen to guarantee that no fractions will arise
during the division process.  Our answer will thus differ from the actual
\acronym{GCD} by an integer constant factor, but this does not matter in the
case of reducing rational functions to lowest terms; the \acronym{GCD} will be
used to divide both the numerator and denominator, so the integer constant
factor will cancel out.

\enlargethispage{\baselineskip}

More precisely, if \( P \) and \( Q \) are polynomials, let \( O_1 \) be the order of
\( P \) (i.e., the order of the largest term of \( P \)) and let \( O_2 \) be the
order of \( Q \).  Let \( c \) be the leading coefficient of \( Q \).  Then it can be
shown that, if we multiply \( P \) by the \newterm{integerizing factor}
\( c^{1 + O_1 - O_2} \), the resulting polynomial can be divided by \( Q \) by
using the \code{div\-/terms} algorithm without introducing any fractions.  The
operation of multiplying the dividend by this constant and then dividing is
sometimes called the \newterm{pseudodivision} of \( P \) by \( Q \).  The remainder
of the division is called the \newterm{pseudoremainder}.

\begin{quote}
\heading{\phantomsection\label{Exercise 2.96}Exercise 2.96:}
\begin{enumerate}[a.]

\item
Implement the procedure \code{pseudoremainder\-/terms}, which is just like
\code{remainder\-/terms} except that it multiplies the dividend by the
integerizing factor described above before calling \code{div\-/terms}.  Modify
\code{gcd\-/terms} to use \code{pseudoremainder\-/terms}, and verify that
\code{greatest\-/common\-/divisor} now produces an answer with integer coefficients
on the example in \link{Exercise 2.95}.

\item
The \acronym{GCD} now has integer coefficients, but they are larger than those
of \( P_1 \).  Modify \code{gcd\-/terms} so that it removes common factors from the
coefficients of the answer by dividing all the coefficients by their (integer)
greatest common divisor.

\end{enumerate}
\end{quote}

\noindent
Thus, here is how to reduce a rational function to lowest terms:

\begin{itemize}

\item
Compute the \acronym{GCD} of the numerator and denominator, using the version
of \code{gcd\-/terms} from \link{Exercise 2.96}.

\item
When you obtain the \acronym{GCD}, multiply both numerator and denominator by
the same integerizing factor before dividing through by the \acronym{GCD}, so
that division by the \acronym{GCD} will not introduce any noninteger
coefficients.  As the factor you can use the leading coefficient of the
\acronym{GCD} raised to the power \( 1 + O_1 - O_2 \), where \( O_2 \) is the
order of the \acronym{GCD} and \( O_1 \) is the maximum of the orders of the
numerator and denominator.  This will ensure that dividing the numerator and
denominator by the \acronym{GCD} will not introduce any fractions.

\item
The result of this operation will be a numerator and denominator with integer
coefficients.  The coefficients will normally be very large because of all of
the integerizing factors, so the last step is to remove the redundant factors
by computing the (integer) greatest common divisor of all the coefficients of
the numerator and the denominator and dividing through by this factor.

\end{itemize}

\begin{quote}
\heading{\phantomsection\label{Exercise 2.97}Exercise 2.97:}
\begin{enumerate}[a.]

\item
Implement this algorithm as a procedure \code{reduce\-/terms} that takes two term
lists \code{n} and \code{d} as arguments and returns a list \code{nn},
\code{dd}, which are \code{n} and \code{d} reduced to lowest terms via the
algorithm given above.  Also write a procedure \code{reduce\-/poly}, analogous to
\code{add\-/poly}, that checks to see if the two polys have the same variable.
If so, \code{reduce\-/poly} strips off the variable and passes the problem to
\code{reduce\-/terms}, then reattaches the variable to the two term lists
supplied by \code{reduce\-/terms}.

\item
Define a procedure analogous to \code{reduce\-/terms} that does what the original
\code{make\-/rat} did for integers:

\begin{scheme}
(define (reduce-integers n d)
  (let ((g (gcd n d)))
    (list (/ n g) (/ d g))))
\end{scheme}

\noindent
and define \code{reduce} as a generic operation that calls \code{apply\-/generic}
to dispatch to either \code{reduce\-/poly} (for \code{polynomial} arguments) or
\code{reduce\-/integers} (for \code{scheme\-/number} arguments).  You can now
easily make the rational-arithmetic package reduce fractions to lowest terms by
having \code{make\-/rat} call \code{reduce} before combining the given numerator
and denominator to form a rational number.  The system now handles rational
expressions in either integers or polynomials.  To test your program, try the
example at the beginning of this extended exercise:

\begin{scheme}
(define  p1 (make-polynomial 'x '((1 1) (0  1))))
(define  p2 (make-polynomial 'x '((3 1) (0 -1))))
(define  p3 (make-polynomial 'x '((1 1))))
(define  p4 (make-polynomial 'x '((2 1) (0 -1))))
(define rf1 (make-rational p1 p2))
(define rf2 (make-rational p3 p4))
(add rf1 rf2)
\end{scheme}

See if you get the correct answer, correctly reduced to lowest terms.
\end{enumerate}
\end{quote}

\noindent
The \acronym{GCD} computation is at the heart of any system that does
operations on rational functions.  The algorithm used above, although
mathematically straightforward, is extremely slow.  The slowness is due partly
to the large number of division operations and partly to the enormous size of
the intermediate coefficients generated by the pseudodivisions.  One of the
active areas in the development of algebraic-manipulation systems is the design
of better algorithms for computing polynomial \acronym{GCD}s.\footnote{One
extremely efficient and elegant method for computing polynomial \acronym{GCD}s
was discovered by Richard \link{Zippel (1979)}.  The method is a probabilistic
algorithm, as is the fast test for primality that we discussed in \link{Chapter
1}.  Zippel's book (\link{Zippel 1993}) describes this method, together with 
other ways to compute polynomial \acronym{GCD}s.}


\chapter{Modularity, Objects, and State}
\label{Chapter 3}

% \vspace{0.5em}

\begin{quote}
Mεταβάλλον ὰναπαύεται\\
(Even while it changes, it stands still.)\\
---Heraclitus
\end{quote}

\begin{quote}
Plus \c{c}a change, plus c'est la m\^{e}me chose.\\
---Alphonse Karr
\end{quote}

% \vspace{1.0em}

\noindent
\lettrine{T}{he preceding chapters} introduced the basic elements from which programs are
made.  We saw how primitive procedures and primitive data are combined to
construct compound entities, and we learned that abstraction is vital in
helping us to cope with the complexity of large systems.  But these tools are
not sufficient for designing programs.  Effective program synthesis also
requires organizational principles that can guide us in formulating the overall
design of a program.  In particular, we need strategies to help us structure
large systems so that they will be \newterm{modular}, that is, so that they can
be divided ``naturally'' into coherent parts that can be separately developed
and maintained.

One powerful design strategy, which is particularly appropriate to the
construction of programs for modeling physical systems, is to base the
structure of our programs on the structure of the system being modeled.  For
each object in the system, we construct a corresponding computational object.
For each system action, we define a symbolic operation in our computational
model.  Our hope in using this strategy is that extending the model to
accommodate new objects or new actions will require no strategic changes to the
program, only the addition of the new symbolic analogs of those objects or
actions.  If we have been successful in our system organization, then to add a
new feature or debug an old one we will have to work on only a localized part
of the system.

To a large extent, then, the way we organize a large program is dictated by our
perception of the system to be modeled.  In this chapter we will investigate
two prominent organizational strategies arising from two rather different
``world views'' of the structure of systems.  The first organizational strategy
concentrates on \newterm{objects}, viewing a large system as a collection of
distinct objects whose behaviors may change over time.  An alternative
organizational strategy concentrates on the \newterm{streams} of information
that flow in the system, much as an electrical engineer views a
signal-processing system.

Both the object-based approach and the stream-processing approach raise
significant linguistic issues in programming.  With objects, we must be
concerned with how a computational object can change and yet maintain its
identity.  This will force us to abandon our old substitution model of
computation (\link{Section 1.1.5}) in favor of a more mechanistic but less
theoretically tractable \newterm{environment model} of computation.  The
difficulties of dealing with objects, change, and identity are a fundamental
consequence of the need to grapple with time in our computational models.
These difficulties become even greater when we allow the possibility of
concurrent execution of programs.  The stream approach can be most fully
exploited when we decouple simulated time in our model from the order of the
events that take place in the computer during evaluation.  We will accomplish
this using a technique known as \newterm{delayed evaluation}.



\section{Assignment and Local State}
\label{Section 3.1}

We ordinarily view the world as populated by independent objects, each of which
has a state that changes over time.  An object is said to ``have state'' if its
behavior is influenced by its history.  A bank account, for example, has state
in that the answer to the question ``Can I withdraw \$100?''  depends upon the
history of deposit and withdrawal transactions.  We can characterize an
object's state by one or more \newterm{state variables}, which among them
maintain enough information about history to determine the object's current
behavior.  In a simple banking system, we could characterize the state of an
account by a current balance rather than by remembering the entire history of
account transactions.

In a system composed of many objects, the objects are rarely completely
independent.  Each may influence the states of others through interactions,
which serve to couple the state variables of one object to those of other
objects.  Indeed, the view that a system is composed of separate objects is
most useful when the state variables of the system can be grouped into closely
coupled subsystems that are only loosely coupled to other subsystems.

This view of a system can be a powerful framework for organizing computational
models of the system.  For such a model to be modular, it should be decomposed
into computational objects that model the actual objects in the system.  Each
computational object must have its own \newterm{local state variables}
describing the actual object's state.  Since the states of objects in the
system being modeled change over time, the state variables of the corresponding
computational objects must also change.  If we choose to model the flow of time
in the system by the elapsed time in the computer, then we must have a way to
construct computational objects whose behaviors change as our programs run.  In
particular, if we wish to model state variables by ordinary symbolic names in
the programming language, then the language must provide an \newterm{assignment
operator} to enable us to change the value associated with a name.



\subsection{Local State Variables}
\label{Section 3.1.1}

To illustrate what we mean by having a computational object with time-varying
state, let us model the situation of withdrawing money from a bank account.  We
will do this using a procedure \code{withdraw}, which takes as argument an
\code{amount} to be withdrawn.  If there is enough money in the account to
accommodate the withdrawal, then \code{withdraw} should return the balance
remaining after the withdrawal.  Otherwise, \code{withdraw} should return the
message \emph{Insufficient funds}. For example, if we begin with \$100 in the
account, we should obtain the following sequence of responses using
\code{withdraw}:

\begin{scheme}
(withdraw 25)
~\textit{75}~
(withdraw 25)
~\textit{50}~
(withdraw 60)
~\textit{"Insufficient funds"}~
(withdraw 15)
~\textit{35}~
\end{scheme}

\noindent
Observe that the expression \code{(withdraw 25)}, evaluated twice, yields
different values.  This is a new kind of behavior for a procedure.  Until now,
all our procedures could be viewed as specifications for computing mathematical
functions.  A call to a procedure computed the value of the function applied to
the given arguments, and two calls to the same procedure with the same
arguments always produced the same result.\footnote{Actually, this is not quite
true.  One exception was the random-number generator in \link{Section 1.2.6}.
Another exception involved the operation/type tables we introduced in 
\link{Section 2.4.3}, where the values of two calls to \code{get} with the same
arguments depended on intervening calls to \code{put}.  On the other hand,
until we introduce assignment, we have no way to create such procedures
ourselves.}

To implement \code{withdraw}, we can use a variable \code{balance} to indicate
the balance of money in the account and define \code{withdraw} as a procedure
that accesses \code{balance}.  The \code{withdraw} procedure checks to see if
\code{balance} is at least as large as the requested \code{amount}.  If so,
\code{withdraw} decrements \code{balance} by \code{amount} and returns the new
value of \code{balance}.  Otherwise, \code{withdraw} returns the
\emph{Insufficient funds} message.  Here are the definitions of \code{balance}
and \code{withdraw}:

\begin{scheme}
(define balance 100)
(define (withdraw amount)
  (if (>= balance amount)
      (begin (set! balance (- balance amount))
             balance)
      "Insufficient funds"))
\end{scheme}

\noindent
Decrementing \code{balance} is accomplished by the expression

\begin{scheme}
(set! balance (- balance amount))
\end{scheme}

\noindent
This uses the \code{set!} special form, whose syntax is

\begin{scheme}
(set! ~\( \dark \langle \)~~\var{\dark name}~~\( \dark \rangle \)~ ~\( \dark \langle \)~~\var{\dark new-value}~~\( \dark \rangle \)~)
\end{scheme}

\noindent
Here \( \langle \)\var{name}\( \kern0.04em\rangle \) is a symbol and \( \langle \)\var{new-value}\( \kern0.04em\rangle \) is any expression.
\code{Set!} changes \( \langle \)\var{name}\( \kern0.04em\rangle \) so that its value is the result obtained by
evaluating \( \langle \)\var{new-value}\( \kern0.04em\rangle \).  In the case at hand, we are changing
\code{balance} so that its new value will be the result of subtracting
\code{amount} from the previous value of \code{balance}.\footnote{The value of
a \code{set!} expression is implementation-dependent.  \code{Set!} should be
used only for its effect, not for its value.

The name \code{set!} reflects a naming convention used in Scheme: Operations
that change the values of variables (or that change data structures, as we will
see in \link{Section 3.3}) are given names that end with an exclamation point.
This is similar to the convention of designating predicates by names that end
with a question mark.}

\code{Withdraw} also uses the \code{begin} special form to cause two
expressions to be evaluated in the case where the \code{if} test is true: first
decrementing \code{balance} and then returning the value of \code{balance}.  In
general, evaluating the expression

\begin{scheme}
(begin ~\( \dark \langle \)~~\( \dark exp_1 \)~~\( \dark \rangle \)~ ~\( \dark \langle \)~~\( \dark exp_2 \)~~\( \dark \rangle \)~ ~\( \dots \)~ ~\( \dark \langle \)~~\( \dark exp_k \)~~\( \dark \rangle \)~)
\end{scheme}

\noindent
causes the expressions \( \langle\kern0.06em \)\( exp_1 \)\( \rangle \) through \( \langle\kern0.06em \)\( exp_k \)\( \rangle \) to be evaluated
in sequence and the value of the final expression \( \langle\kern0.06em \)\( exp_k \)\( \rangle \) to be
returned as the value of the entire \code{begin} form.\footnote{We have already
used \code{begin} implicitly in our programs, because in Scheme the body of a
procedure can be a sequence of expressions.  Also, the \( \langle \)\var{consequent}\( \kern0.06em\rangle \) part
of each clause in a \code{cond} expression can be a sequence of expressions
rather than a single expression.}

Although \code{withdraw} works as desired, the variable \code{balance} presents
a problem.  As specified above, \code{balance} is a name defined in the global
environment and is freely accessible to be examined or modified by any
procedure.  It would be much better if we could somehow make \code{balance}
internal to \code{withdraw}, so that \code{withdraw} would be the only
procedure that could access \code{balance} directly and any other procedure
could access \code{balance} only indirectly (through calls to \code{withdraw}).
This would more accurately model the notion that \code{balance} is a local
state variable used by \code{withdraw} to keep track of the state of the
account.

We can make \code{balance} internal to \code{withdraw} by rewriting the
definition as follows:

\begin{scheme}
(define new-withdraw
  (let ((balance 100))
    (lambda (amount)
      (if (>= balance amount)
          (begin (set! balance (- balance amount))
                 balance)
          "Insufficient funds"))))
\end{scheme}

\noindent
What we have done here is use \code{let} to establish an environment with a
local variable \code{balance}, bound to the initial value 100.  Within this
local environment, we use \code{lambda} to create a procedure that takes
\code{amount} as an argument and behaves like our previous \code{withdraw}
procedure.  This procedure---returned as the result of evaluating the
\code{let} expression---is \code{new\-/withdraw}, which behaves in precisely the
same way as \code{withdraw} but whose variable \code{balance} is not accessible
by any other procedure.\footnote{In programming-language jargon, the variable
\code{balance} is said to be \newterm{encapsulated} within the
\code{new\-/withdraw} procedure.  Encapsulation reflects the general
system-design principle known as the \newterm{hiding principle}: One can make a
system more modular and robust by protecting parts of the system from each
other; that is, by providing information access only to those parts of the
system that have a ``need to know.''}

Combining \code{set!} with local variables is the general programming technique
we will use for constructing computational objects with local state.
Unfortunately, using this technique raises a serious problem: When we first
introduced procedures, we also introduced the substitution model of evaluation
(\link{Section 1.1.5}) to provide an interpretation of what procedure
application means.  We said that applying a procedure should be interpreted as
evaluating the body of the procedure with the formal parameters replaced by
their values.  The trouble is that, as soon as we introduce assignment into our
language, substitution is no longer an adequate model of procedure application.
(We will see why this is so in \link{Section 3.1.3}.)  As a consequence, we
technically have at this point no way to understand why the \code{new\-/withdraw}
procedure behaves as claimed above.  In order to really understand a procedure
such as \code{new\-/withdraw}, we will need to develop a new model of procedure
application.  In \link{Section 3.2} we will introduce such a model, together
with an explanation of \code{set!} and local variables.  First, however, we
examine some variations on the theme established by \code{new\-/withdraw}.

The following procedure, \code{make\-/withdraw}, creates ``withdrawal
processors.''  The formal parameter \code{balance} in \code{make\-/withdraw}
specifies the initial amount of money in the account.\footnote{In contrast with
\code{new\-/withdraw} above, we do not have to use \code{let} to make
\code{balance} a local variable, since formal parameters are already local.
This will be clearer after the discussion of the environment model of
evaluation in \link{Section 3.2}.  (See also \link{Exercise 3.10}.)}

\begin{scheme}
(define (make-withdraw balance)
  (lambda (amount)
    (if (>= balance amount)
        (begin (set! balance (- balance amount))
               balance)
        "Insufficient funds")))
\end{scheme}

\noindent
\code{Make\-/withdraw} can be used as follows to create two objects \code{W1} and
\code{W2}:

\begin{scheme}
(define W1 (make-withdraw 100))
(define W2 (make-withdraw 100))

(W1 50)
~\textit{50}~
(W2 70)
~\textit{30}~
(W2 40)
~\textit{"Insufficient funds"}~
(W1 40)
~\textit{10}~
\end{scheme}

\noindent
Observe that \code{W1} and \code{W2} are completely independent objects, each
with its own local state variable \code{balance}.  Withdrawals from one do not
affect the other.

We can also create objects that handle deposits as well as withdrawals, and
thus we can represent simple bank accounts.  Here is a procedure that returns a
``bank-account object'' with a specified initial balance:

\begin{scheme}
(define (make-account balance)
  (define (withdraw amount)
    (if (>= balance amount)
        (begin (set! balance (- balance amount))
               balance)
        "Insufficient funds"))
  (define (deposit amount)
    (set! balance (+ balance amount))
    balance)
  (define (dispatch m)
    (cond ((eq? m 'withdraw) withdraw)
          ((eq? m 'deposit) deposit)
          (else (error "Unknown request: MAKE-ACCOUNT"
                       m))))
  dispatch)
\end{scheme}

\noindent
Each call to \code{make\-/account} sets up an environment with a local state
variable \code{balance}.  Within this environment, \code{make\-/account} defines
procedures \code{deposit} and \code{withdraw} that access \code{balance} and an
additional procedure \code{dispatch} that takes a ``message'' as input and
returns one of the two local procedures.  The \code{dispatch} procedure itself
is returned as the value that represents the bank-account object.  This is
precisely the \newterm{message-passing} style of programming that we saw in
\link{Section 2.4.3}, although here we are using it in conjunction with the
ability to modify local variables.

\code{Make\-/account} can be used as follows:

\begin{scheme}
(define acc (make-account 100))
((acc 'withdraw) 50)
~\textit{50}~
((acc 'withdraw) 60)
~\textit{"Insufficient funds"}~
((acc 'deposit) 40)
~\textit{90}~
((acc 'withdraw) 60)
~\textit{30}~
\end{scheme}

\noindent
Each call to \code{acc} returns the locally defined \code{deposit} or
\code{withdraw} procedure, which is then applied to the specified
\code{amount}.  As was the case with \code{make\-/withdraw}, another call to
\code{make\-/account}

\begin{scheme}
(define acc2 (make-account 100))
\end{scheme}

\noindent
will produce a completely separate account object, which maintains its own
local \code{balance}.

\begin{quote}
\heading{\phantomsection\label{Exercise 3.1}Exercise 3.1:} An \newterm{accumulator} is a
procedure that is called repeatedly with a single numeric argument and
accumulates its arguments into a sum.  Each time it is called, it returns the
currently accumulated sum.  Write a procedure \code{make\-/accumulator} that
generates accumulators, each maintaining an independent sum.  The input to
\code{make\-/accumulator} should specify the initial value of the sum; for
example

\begin{scheme}
(define A (make-accumulator 5))
(A 10)
~\textit{15}~
(A 10)
~\textit{25}~
\end{scheme}
\end{quote}

\begin{quote}
\heading{\phantomsection\label{Exercise 3.2}Exercise 3.2:} In software-testing applications,
it is useful to be able to count the number of times a given procedure is
called during the course of a computation.  Write a procedure
\code{make\-/monitored} that takes as input a procedure, \code{f}, that itself
takes one input.  The result returned by \code{make\-/monitored} is a third
procedure, say \code{mf}, that keeps track of the number of times it has been
called by maintaining an internal counter.  If the input to \code{mf} is the
special symbol \code{how\-/many\-/calls?}, then \code{mf} returns the value of the
counter.  If the input is the special symbol \code{reset\-/count}, then \code{mf}
resets the counter to zero.  For any other input, \code{mf} returns the result
of calling \code{f} on that input and increments the counter.  For instance, we
could make a monitored version of the \code{sqrt} procedure:

\begin{scheme}
(define s (make-monitored sqrt))
(s 100)
~\textit{10}~
(s 'how-many-calls?)
~\textit{1}~
\end{scheme}
\end{quote}

\begin{quote}
\heading{\phantomsection\label{Exercise 3.3}Exercise 3.3:} Modify the \code{make\-/account}
procedure so that it creates password-protected accounts.  That is,
\code{make\-/account} should take a symbol as an additional argument, as in

\begin{scheme}
(define acc (make-account 100 'secret-password))
\end{scheme}

The resulting account object should process a request only if it is accompanied
by the password with which the account was created, and should otherwise return
a complaint:

\begin{scheme}
((acc 'secret-password 'withdraw) 40)
~\textit{60}~
((acc 'some-other-password 'deposit) 50)
~\textit{"Incorrect password"}~
\end{scheme}
\end{quote}

\begin{quote}
\heading{\phantomsection\label{Exercise 3.4}Exercise 3.4:} Modify the \code{make\-/account}
procedure of \link{Exercise 3.3} by adding another local state variable so that,
if an account is accessed more than seven consecutive times with an incorrect
password, it invokes the procedure \code{call\-/the\-/cops}.
\end{quote}

\subsection{The Benefits of Introducing Assignment}
\label{Section 3.1.2}

As we shall see, introducing assignment into our programming language leads us
into a thicket of difficult conceptual issues.  Nevertheless, viewing systems
as collections of objects with local state is a powerful technique for
maintaining a modular design.  As a simple example, consider the design of a
procedure \code{rand} that, whenever it is called, returns an integer chosen at
random.

It is not at all clear what is meant by ``chosen at random.''  What we
presumably want is for successive calls to \code{rand} to produce a sequence of
numbers that has statistical properties of uniform distribution.  We will not
discuss methods for generating suitable sequences here.  Rather, let us assume
that we have a procedure \code{rand\-/update} that has the property that if we
start with a given number \( x_1 \) and form

\begin{scheme}
~\( \dark x_2 \)~ = (rand-update ~\( \dark x_1 \)~)
~\( \dark x_3 \)~ = (rand-update ~\( \dark x_2 \)~)
\end{scheme}

\noindent
then the sequence of values \( x_1 \), \( x_2 \), \( x_3 \), \( \dots \) will have the
desired statistical properties.\footnote{One common way to implement
\code{rand\-/update} is to use the rule that \( x \) is updated to \( ax + b \) 
modulo \( m \), where \( a \), \( b \), and \( m \) are appropriately chosen
integers.  Chapter 3 of \link{Knuth 1981} includes an extensive discussion of
techniques for generating sequences of random numbers and establishing their
statistical properties.  Notice that the \code{rand\-/update} procedure computes
a mathematical function: Given the same input twice, it produces the same
output.  Therefore, the number sequence produced by \code{rand\-/update}
certainly is not ``random,'' if by ``random'' we insist that each number in the
sequence is unrelated to the preceding number.  The relation between ``real
randomness'' and so-called \newterm{pseudo-random} sequences, which are
produced by well-determined computations and yet have suitable statistical
properties, is a complex question involving difficult issues in mathematics and
philosophy.  Kolmogorov, Solomonoff, and Chaitin have made great progress in
clarifying these issues; a discussion can be found in \link{Chaitin 1975}.}

We can implement \code{rand} as a procedure with a local state variable
\code{x} that is initialized to some fixed value \code{random\-/init}.  Each call
to \code{rand} computes \code{rand\-/update} of the current value of \code{x},
returns this as the random number, and also stores this as the new value of
\code{x}.

\begin{scheme}
(define rand (let ((x random-init))
               (lambda () 
                 (set! x (rand-update x))
                 x)))
\end{scheme}

\noindent
Of course, we could generate the same sequence of random numbers without using
assignment by simply calling \code{rand\-/update} directly.  However, this would
mean that any part of our program that used random numbers would have to
explicitly remember the current value of \code{x} to be passed as an argument
to \code{rand\-/update}.  To realize what an annoyance this would be, consider
using random numbers to implement a technique called \newterm{Monte Carlo
simulation}.

The Monte Carlo method consists of choosing sample experiments at random from a
large set and then making deductions on the basis of the probabilities
estimated from tabulating the results of those experiments.  For example, we
can approximate \( \pi \) using the fact that \( 6/\pi^2 \) is the probability
that two integers chosen at random will have no factors in common; that is,
that their greatest common divisor will be 1.\footnote{This theorem is due to
E. Ces\`aro.  See section 4.5.2 of \link{Knuth 1981} for a discussion and a proof.} To
obtain the approximation to \( \pi \), we perform a large number of experiments.
In each experiment we choose two integers at random and perform a test to see
if their \acronym{GCD} is 1.  The fraction of times that the test is passed
gives us our estimate of \( 6/\pi^2 \), and from this we obtain our
approximation to \( \pi \).

The heart of our program is a procedure \code{monte\-/carlo}, which takes as
arguments the number of times to try an experiment, together with the
experiment, represented as a no-argument procedure that will return either true
or false each time it is run.  \code{Monte\-/carlo} runs the experiment for the
designated number of trials and returns a number telling the fraction of the
trials in which the experiment was found to be true.

\begin{scheme}
(define (estimate-pi trials)
  (sqrt (/ 6 (monte-carlo trials cesaro-test))))
(define (cesaro-test)
   (= (gcd (rand) (rand)) 1))

(define (monte-carlo trials experiment)
  (define (iter trials-remaining trials-passed)
    (cond ((= trials-remaining 0)
           (/ trials-passed trials))
          ((experiment)
           (iter (- trials-remaining 1) 
                 (+ trials-passed 1)))
          (else
           (iter (- trials-remaining 1) 
                 trials-passed))))
  (iter trials 0))
\end{scheme}

\noindent
Now let us try the same computation using \code{rand\-/update} directly rather
than \code{rand}, the way we would be forced to proceed if we did not use
assignment to model local state:

\begin{scheme}
(define (estimate-pi trials)
  (sqrt (/ 6 (random-gcd-test trials random-init))))
(define (random-gcd-test trials initial-x)
  (define (iter trials-remaining trials-passed x)
    (let ((x1 (rand-update x)))
      (let ((x2 (rand-update x1)))
        (cond ((= trials-remaining 0)
               (/ trials-passed trials))
              ((= (gcd x1 x2) 1)
               (iter (- trials-remaining 1)
                     (+ trials-passed 1)
                     x2))
              (else
               (iter (- trials-remaining 1)
                     trials-passed
                     x2))))))
  (iter trials 0 initial-x))
\end{scheme}

\noindent
While the program is still simple, it betrays some painful breaches of
modularity.  In our first version of the program, using \code{rand}, we can
express the Monte Carlo method directly as a general \code{monte\-/carlo}
procedure that takes as an argument an arbitrary \code{experiment} procedure.
In our second version of the program, with no local state for the random-number
generator, \code{random\-/gcd\-/test} must explicitly manipulate the random numbers
\code{x1} and \code{x2} and recycle \code{x2} through the iterative loop as the
new input to \code{rand\-/update}.  This explicit handling of the random numbers
intertwines the structure of accumulating test results with the fact that our
particular experiment uses two random numbers, whereas other Monte Carlo
experiments might use one random number or three.  Even the top-level procedure
\code{estimate\-/pi} has to be concerned with supplying an initial random number.
The fact that the random-number generator's insides are leaking out into other
parts of the program makes it difficult for us to isolate the Monte Carlo idea
so that it can be applied to other tasks.  In the first version of the program,
assignment encapsulates the state of the random-number generator within the
\code{rand} procedure, so that the details of random-number generation remain
independent of the rest of the program.

The general phenomenon illustrated by the Monte Carlo example is this: From the
point of view of one part of a complex process, the other parts appear to
change with time.  They have hidden time-varying local state.  If we wish to
write computer programs whose structure reflects this decomposition, we make
computational objects (such as bank accounts and random-number generators)
whose behavior changes with time.  We model state with local state variables,
and we model the changes of state with assignments to those variables.

It is tempting to conclude this discussion by saying that, by introducing
assignment and the technique of hiding state in local variables, we are able to
structure systems in a more modular fashion than if all state had to be
manipulated explicitly, by passing additional parameters.  Unfortunately, as we
shall see, the story is not so simple.

\begin{quote}
\heading{\phantomsection\label{Exercise 3.5}Exercise 3.5:} \newterm{Monte Carlo integration}
is a method of estimating definite integrals by means of Monte Carlo
simulation.  Consider computing the area of a region of space described by a
predicate \( P(x, y) \) that is true for points \( (x, y) \) in the
region and false for points not in the region.  For example, the region
contained within a circle of radius 3 centered at (5, 7) is described by the
predicate that tests whether \( (x - 5)^2 + (y - 7)^2 \le 3^2 \).  To estimate
the area of the region described by such a predicate, begin by choosing a
rectangle that contains the region.  For example, a rectangle with diagonally
opposite corners at (2, 4) and (8, 10) contains the circle above.  The desired
integral is the area of that portion of the rectangle that lies in the region.
We can estimate the integral by picking, at random, points \( (x, y) \) that
lie in the rectangle, and testing \( P(x, y) \) for each point to
determine whether the point lies in the region.  If we try this with many
points, then the fraction of points that fall in the region should give an
estimate of the proportion of the rectangle that lies in the region.  Hence,
multiplying this fraction by the area of the entire rectangle should produce an
estimate of the integral.

Implement Monte Carlo integration as a procedure \code{estimate\-/integral} that
takes as arguments a predicate \code{P}, upper and lower bounds \code{x1},
\code{x2}, \code{y1}, and \code{y2} for the rectangle, and the number of trials
to perform in order to produce the estimate.  Your procedure should use the
same \code{monte\-/carlo} procedure that was used above to estimate \( \pi \).
Use your \code{estimate\-/integral} to produce an estimate of \( \pi \) by
measuring the area of a unit circle.

You will find it useful to have a procedure that returns a number chosen at
random from a given range.  The following \code{random\-/in\-/range} procedure
implements this in terms of the \code{random} procedure used in 
\link{Section 1.2.6}, which returns a nonnegative number less than its
input.\footnote{\acronym{MIT} Scheme provides such a procedure.  If
\code{random} is given an exact integer (as in \link{Section 1.2.6}) it returns
an exact integer, but if it is given a decimal value (as in this exercise) it
returns a decimal value.}

\begin{scheme}
(define (random-in-range low high)
  (let ((range (- high low)))
    (+ low (random range))))
\end{scheme}
\end{quote}

\begin{quote}
\heading{\phantomsection\label{Exercise 3.6}Exercise 3.6:} It is useful to be able to reset a
random-number generator to produce a sequence starting from a given value.
Design a new \code{rand} procedure that is called with an argument that is
either the symbol \code{generate} or the symbol \code{reset} and behaves as
follows: \code{(rand 'generate)} produces a new random number; \code{((rand
'reset)}\( \;\langle \)\var{new-value}\( \kern0.11em\rangle \)\code{)} resets the internal state variable to the
designated \( \langle \)\var{new-value}\( \kern0.08em\rangle \).  Thus, by resetting the state, one can generate
repeatable sequences.  These are very handy to have when testing and debugging
programs that use random numbers.
\end{quote}

\subsection{The Costs of Introducing Assignment}
\label{Section 3.1.3}

As we have seen, the \code{set!} operation enables us to model objects that
have local state.  However, this advantage comes at a price.  Our programming
language can no longer be interpreted in terms of the substitution model of
procedure application that we introduced in \link{Section 1.1.5}.  Moreover, no
simple model with ``nice'' mathematical properties can be an adequate framework
for dealing with objects and assignment in programming languages.

So long as we do not use assignments, two evaluations of the same procedure
with the same arguments will produce the same result, so that procedures can be
viewed as computing mathematical functions.  Programming without any use of
assignments, as we did throughout the first two chapters of this book, is
accordingly known as \newterm{functional programming}.

To understand how assignment complicates matters, consider a simplified version
of the \code{make\-/withdraw} procedure of \link{Section 3.1.1} that does not
bother to check for an insufficient amount:

\begin{scheme}
(define (make-simplified-withdraw balance)
  (lambda (amount)
    (set! balance (- balance amount))
    balance))
(define W (make-simplified-withdraw 25))
(W 20)
~\textit{5}~
(W 10)
~\textit{-5}~
\end{scheme}

\noindent
Compare this procedure with the following \code{make\-/decrementer} procedure,
which does not use \code{set!}:

\begin{scheme}
(define (make-decrementer balance)
  (lambda (amount)
    (- balance amount)))
\end{scheme}

\noindent
\code{Make\-/decrementer} returns a procedure that subtracts its input from a
designated amount \code{balance}, but there is no accumulated effect over
successive calls, as with \code{make\-/simplified\-/withdraw}:

\begin{scheme}
(define D (make-decrementer 25))
(D 20)
~\textit{5}~
(D 10)
~\textit{15}~
\end{scheme}

\noindent
We can use the substitution model to explain how \code{make\-/decrementer} works.
For instance, let us analyze the evaluation of the expression

\begin{scheme}
((make-decrementer 25) 20)
\end{scheme}

\noindent
We first simplify the operator of the combination by substituting 25 for
\code{balance} in the body of \code{make\-/decrementer}.  This reduces the
expression to

\begin{scheme}
((lambda (amount) (- 25 amount)) 20)
\end{scheme}

\noindent
Now we apply the operator by substituting 20 for \code{amount} in the body of
the \code{lambda} expression:

\begin{scheme}
(- 25 20)
\end{scheme}

\noindent
The final answer is 5.

Observe, however, what happens if we attempt a similar substitution analysis
with \code{make\-/simplified\-/withdraw}:

\begin{scheme}
((make-simplified-withdraw 25) 20)
\end{scheme}

\noindent
We first simplify the operator by substituting 25 for \code{balance} in the
body of \code{make\-/simplified\-/withdraw}.  This reduces the expression
to\footnote{We don't substitute for the occurrence of \code{balance} in the
\code{set!} expression because the \( \langle \)\var{name}\( \kern0.08em\rangle \) in a \code{set!} is not
evaluated.  If we did substitute for it, we would get \code{(set! 25 (- 25
amount))}, which makes no sense.}

\begin{scheme}
((lambda (amount) (set! balance (- 25 amount)) 25) 20)
\end{scheme}

\noindent
Now we apply the operator by substituting 20 for \code{amount} in the body of
the \code{lambda} expression:

\begin{scheme}
(set! balance (- 25 20)) 25
\end{scheme}

\noindent
If we adhered to the substitution model, we would have to say that the meaning
of the procedure application is to first set \code{balance} to 5 and then
return 25 as the value of the expression.  This gets the wrong answer.  In
order to get the correct answer, we would have to somehow distinguish the first
occurrence of \code{balance} (before the effect of the \code{set!})  from the
second occurrence of \code{balance} (after the effect of the \code{set!}), and
the substitution model cannot do this.

The trouble here is that substitution is based ultimately on the notion that
the symbols in our language are essentially names for values.  But as soon as
we introduce \code{set!} and the idea that the value of a variable can change,
a variable can no longer be simply a name.  Now a variable somehow refers to a
place where a value can be stored, and the value stored at this place can
change.  In \link{Section 3.2} we will see how environments play this role of
``place'' in our computational model.

\subsubsection*{Sameness and change}

The issue surfacing here is more profound than the mere breakdown of a
particular model of computation.  As soon as we introduce change into our
computational models, many notions that were previously straightforward become
problematical.  Consider the concept of two things being ``the same.''

Suppose we call \code{make\-/decrementer} twice with the same argument to create
two procedures:

\begin{scheme}
(define D1 (make-decrementer 25))
(define D2 (make-decrementer 25))
\end{scheme}

\noindent
Are \code{D1} and \code{D2} the same?  An acceptable answer is yes, because
\code{D1} and \code{D2} have the same computational behavior---each is a
procedure that subtracts its input from 25.  In fact, \code{D1} could be
substituted for \code{D2} in any computation without changing the result.

Contrast this with making two calls to \code{make\-/simplified\-/withdraw}:

\begin{scheme}
(define W1 (make-simplified-withdraw 25))
(define W2 (make-simplified-withdraw 25))
\end{scheme}

\noindent
Are \code{W1} and \code{W2} the same?  Surely not, because calls to \code{W1}
and \code{W2} have distinct effects, as shown by the following sequence of
interactions:

\begin{scheme}
(W1 20)
~\textit{5}~
(W1 20)
~\textit{-15}~
(W2 20)
~\textit{5}~
\end{scheme}

\noindent
Even though \code{W1} and \code{W2} are ``equal'' in the sense that they are
both created by evaluating the same expression, \code{(make\-/simplified\-/withdraw
25)}, it is not true that \code{W1} could be substituted for \code{W2} in any
expression without changing the result of evaluating the expression.

A language that supports the concept that ``equals can be substituted for
equals'' in an expression without changing the value of the expression is said
to be \newterm{referentially transparent}.  Referential transparency is
violated when we include \code{set!} in our computer language.  This makes it
tricky to determine when we can simplify expressions by substituting equivalent
expressions.  Consequently, reasoning about programs that use assignment
becomes drastically more difficult.

Once we forgo referential transparency, the notion of what it means for
computational objects to be ``the same'' becomes difficult to capture in a
formal way.  Indeed, the meaning of ``same'' in the real world that our
programs model is hardly clear in itself.  In general, we can determine that
two apparently identical objects are indeed ``the same one'' only by modifying
one object and then observing whether the other object has changed in the same
way.  But how can we tell if an object has ``changed'' other than by observing
the ``same'' object twice and seeing whether some property of the object
differs from one observation to the next?  Thus, we cannot determine ``change''
without some \emph{a priori} notion of ``sameness,'' and we cannot determine
sameness without observing the effects of change.

As an example of how this issue arises in programming, consider the situation
where Peter and Paul have a bank account with \$100 in it.  There is a
substantial difference between modeling this as

\begin{scheme}
(define peter-acc (make-account 100))
(define paul-acc (make-account 100))
\end{scheme}

\noindent
and modeling it as

\begin{scheme}
(define peter-acc (make-account 100))
(define paul-acc peter-acc)
\end{scheme}

\noindent
In the first situation, the two bank accounts are distinct.  Transactions made
by Peter will not affect Paul's account, and vice versa.  In the second
situation, however, we have defined \code{paul\-/acc} to be \emph{the same thing}
as \code{peter\-/acc}.  In effect, Peter and Paul now have a joint bank account,
and if Peter makes a withdrawal from \code{peter\-/acc} Paul will observe less
money in \code{paul\-/acc}.  These two similar but distinct situations can cause
confusion in building computational models.  With the shared account, in
particular, it can be especially confusing that there is one object (the bank
account) that has two different names (\code{peter\-/acc} and \code{paul\-/acc});
if we are searching for all the places in our program where \code{paul\-/acc} can
be changed, we must remember to look also at things that change
\code{peter\-/acc}.\footnote{The phenomenon of a single computational object
being accessed by more than one name is known as \newterm{aliasing}.  The joint
bank account situation illustrates a very simple example of an alias.  In
\link{Section 3.3} we will see much more complex examples, such as ``distinct''
compound data structures that share parts.  Bugs can occur in our programs if
we forget that a change to an object may also, as a ``side effect,'' change a
``different'' object because the two ``different'' objects are actually a
single object appearing under different aliases.  These so-called
\newterm{side-effect bugs} are so difficult to locate and to analyze that some
people have proposed that programming languages be designed in such a way as to
not allow side effects or aliasing (\link{Lampson et al. 1981}; 
\link{Morris et al. 1980}).}

With reference to the above remarks on ``sameness'' and ``change,'' observe
that if Peter and Paul could only examine their bank balances, and could not
perform operations that changed the balance, then the issue of whether the two
accounts are distinct would be moot.  In general, so long as we never modify
data objects, we can regard a compound data object to be precisely the totality
of its pieces.  For example, a rational number is determined by giving its
numerator and its denominator.  But this view is no longer valid in the
presence of change, where a compound data object has an ``identity'' that is
something different from the pieces of which it is composed.  A bank account is
still ``the same'' bank account even if we change the balance by making a
withdrawal; conversely, we could have two different bank accounts with the same
state information.  This complication is a consequence, not of our programming
language, but of our perception of a bank account as an object.  We do not, for
example, ordinarily regard a rational number as a changeable object with
identity, such that we could change the numerator and still have ``the same''
rational number.

\subsubsection*{Pitfalls of imperative programming}

In contrast to functional programming, programming that makes extensive use of
assignment is known as \newterm{imperative programming}.  In addition to
raising complications about computational models, programs written in
imperative style are susceptible to bugs that cannot occur in functional
programs.  For example, recall the iterative factorial program from 
\link{Section 1.2.1}:

\begin{scheme}
(define (factorial n)
  (define (iter product counter)
    (if (> counter n)
        product
        (iter (* counter product) (+ counter 1))))
  (iter 1 1))
\end{scheme}

\noindent
Instead of passing arguments in the internal iterative loop, we could adopt a
more imperative style by using explicit assignment to update the values of the
variables \code{product} and \code{counter}:

\begin{scheme}
(define (factorial n)
  (let ((product 1)
        (counter 1))
    (define (iter)
      (if (> counter n)
          product
          (begin (set! product (* counter product))
                 (set! counter (+ counter 1))
                 (iter))))
    (iter)))
\end{scheme}

\noindent
This does not change the results produced by the program, but it does introduce
a subtle trap.  How do we decide the order of the assignments?  As it happens,
the program is correct as written.  But writing the assignments in the opposite
order

\begin{scheme}
(set! counter (+ counter 1))
(set! product (* counter product))
\end{scheme}

\noindent
would have produced a different, incorrect result.  In general, programming
with assignment forces us to carefully consider the relative orders of the
assignments to make sure that each statement is using the correct version of
the variables that have been changed.  This issue simply does not arise in
functional programs.\footnote{In view of this, it is ironic that introductory
programming is most often taught in a highly imperative style.  This may be a
vestige of a belief, common throughout the 1960s and 1970s, that programs that
call procedures must inherently be less efficient than programs that perform
assignments.  (\link{Steele 1977} debunks this argument.)  Alternatively it may
reflect a view that step-by-step assignment is easier for beginners to
visualize than procedure call.  Whatever the reason, it often saddles beginning
programmers with ``should I set this variable before or after that one''
concerns that can complicate programming and obscure the important ideas.}

The complexity of imperative programs becomes even worse if we consider
applications in which several processes execute concurrently.  We will return
to this in \link{Section 3.4}.  First, however, we will address the issue of
providing a computational model for expressions that involve assignment, and
explore the uses of objects with local state in designing simulations.

\begin{quote}
\heading{\phantomsection\label{Exercise 3.7}Exercise 3.7:} Consider the bank account objects
created by \code{make\-/account}, with the password modification described in
\link{Exercise 3.3}.  Suppose that our banking system requires the ability to
make joint accounts.  Define a procedure \code{make\-/joint} that accomplishes
this.  \code{Make\-/joint} should take three arguments.  The first is a
password-protected account.  The second argument must match the password with
which the account was defined in order for the \code{make\-/joint} operation to
proceed.  The third argument is a new password.  \code{Make\-/joint} is to create
an additional access to the original account using the new password.  For
example, if \code{peter\-/acc} is a bank account with password
\code{open\-/sesame}, then

\begin{scheme}
(define paul-acc
  (make-joint peter-acc 'open-sesame 'rosebud))
\end{scheme}

\noindent
will allow one to make transactions on \code{peter\-/acc} using the name
\code{paul\-/acc} and the password \code{rosebud}.  You may wish to modify your
solution to \link{Exercise 3.3} to accommodate this new feature.
\end{quote}

\begin{quote}
\heading{\phantomsection\label{Exercise 3.8}Exercise 3.8:} When we defined the evaluation
model in \link{Section 1.1.3}, we said that the first step in evaluating an
expression is to evaluate its subexpressions.  But we never specified the order
in which the subexpressions should be evaluated (e.g., left to right or right
to left).  When we introduce assignment, the order in which the arguments to a
procedure are evaluated can make a difference to the result.  Define a simple
procedure \code{f} such that evaluating 

\begin{scheme}
(+ (f 0) (f 1))
\end{scheme}

will return 0 if
the arguments to \code{+} are evaluated from left to right but will return 1 if
the arguments are evaluated from right to left.
\end{quote}

\section{The Environment Model of Evaluation}
\label{Section 3.2}

When we introduced compound procedures in \link{Chapter 1}, we used the
substitution model of evaluation (\link{Section 1.1.5}) to define what is meant
by applying a procedure to arguments:

\begin{itemize}

\item
To apply a compound procedure to arguments, evaluate the body of the procedure
with each formal parameter replaced by the corresponding argument.

\end{itemize}

\noindent
Once we admit assignment into our programming language, such a definition is no
longer adequate.  In particular, \link{Section 3.1.3} argued that, in the
presence of assignment, a variable can no longer be considered to be merely a
name for a value.  Rather, a variable must somehow designate a ``place'' in
which values can be stored.  In our new model of evaluation, these places will
be maintained in structures called \newterm{environments}.

An environment is a sequence of \newterm{frames}.  Each frame is a table
(possibly empty) of \newterm{bindings}, which associate variable names with
their corresponding values.  (A single frame may contain at most one binding
for any variable.)  Each frame also has a pointer to its \newterm{enclosing
environment}, unless, for the purposes of discussion, the frame is considered
to be \newterm{global}.  The \newterm{value of a variable} with respect to an
environment is the value given by the binding of the variable in the first
frame in the environment that contains a binding for that variable.  If no
frame in the sequence specifies a binding for the variable, then the variable
is said to be \newterm{unbound} in the environment.

\link{Figure 3.1} shows a simple environment structure consisting of three
frames, labeled I, II, and III.  In the diagram, A, B, C, and D are pointers to
environments.  C and D point to the same environment.  The variables \code{z}
and \code{x} are bound in frame II, while \code{y} and \code{x} are bound in
frame I.  The value of \code{x} in environment D is 3.  The value of \code{x}
with respect to environment B is also 3.  This is determined as follows: We
examine the first frame in the sequence (frame III) and do not find a binding
for \code{x}, so we proceed to the enclosing environment D and find the binding
in frame I.  On the other hand, the value of \code{x} in environment A is 7,
because the first frame in the sequence (frame II) contains a binding of
\code{x} to 7.  With respect to environment A, the binding of \code{x} to 7 in
frame II is said to \newterm{shadow} the binding of \code{x} to 3 in frame I.

\begin{figure}[tb]
\phantomsection\label{Figure 3.1}
\centering
\begin{comment}
\heading{Figure 3.1:} A simple environment structure.

\begin{example}
           +--------+
           |      I |
           | x: 3   |
           | y: 5   |
           +--------+
              ^  ^
              |  |
            C |  | D
+---------+   |  |   +----------+
|      II |   |  |   |      III |
| z: 6    +---+  +---+ m: 1     |
| x: 7    |          | y: 2     |
+---------+          +----------+
\end{example}
\end{comment}
\includegraphics[width=48mm]{fig/chap3/Fig3.1.pdf}
\par\bigskip
\noindent
\heading{Figure 3.1:} A simple environment structure.
\end{figure}

The environment is crucial to the evaluation process, because it determines the
context in which an expression should be evaluated.  Indeed, one could say that
expressions in a programming language do not, in themselves, have any meaning.
Rather, an expression acquires a meaning only with respect to some environment
in which it is evaluated.  Even the interpretation of an expression as
straightforward as \code{(+ 1 1)} depends on an understanding that one is
operating in a context in which \code{+} is the symbol for addition.  Thus, in
our model of evaluation we will always speak of evaluating an expression with
respect to some environment.  To describe interactions with the interpreter, we
will suppose that there is a global environment, consisting of a single frame
(with no enclosing environment) that includes values for the symbols associated
with the primitive procedures.  For example, the idea that \code{+} is the
symbol for addition is captured by saying that the symbol \code{+} is bound in
the global environment to the primitive addition procedure.



\subsection{The Rules for Evaluation}
\label{Section 3.2.1}

The overall specification of how the interpreter evaluates a combination
remains the same as when we first introduced it in \link{Section 1.1.3}:

\begin{itemize}

\item
To evaluate a combination:

\end{itemize}

\begin{enumerate}

\item
Evaluate the subexpressions of the combination.\footnote{Assignment introduces a
subtlety into step 1 of the evaluation rule.  As shown in \link{Exercise 3.8},
the presence of assignment allows us to write expressions that will produce
different values depending on the order in which the subexpressions in a
combination are evaluated.  Thus, to be precise, we should specify an
evaluation order in step 1 (e.g., left to right or right to left).  However,
this order should always be considered to be an implementation detail, and one
should never write programs that depend on some particular order.  For
instance, a sophisticated compiler might optimize a program by varying the
order in which subexpressions are evaluated.}

\item
Apply the value of the operator subexpression to the values of the operand
subexpressions.

\end{enumerate}

\noindent
The environment model of evaluation replaces the substitution model in
specifying what it means to apply a compound procedure to arguments.

In the environment model of evaluation, a procedure is always a pair consisting
of some code and a pointer to an environment.  Procedures are created in one
way only: by evaluating a λ-expression.  This produces a procedure
whose code is obtained from the text of the λ-expression and whose
environment is the environment in which the λ-expression was
evaluated to produce the procedure.  For example, consider the procedure
definition

\begin{scheme}
(define (square x)
  (* x x))
\end{scheme}

\noindent
evaluated in the global environment.  The procedure definition syntax is just
syntactic sugar for an underlying implicit λ-expression.  It would
have been equivalent to have used

\begin{scheme}
(define square
  (lambda (x) (* x x)))
\end{scheme}

\noindent
which evaluates \code{(lambda (x) (* x x))} and binds \code{square} to the
resulting value, all in the global environment.

\link{Figure 3.2} shows the result of evaluating this \code{define} expression.
The procedure object is a pair whose code specifies that the procedure has one
formal parameter, namely \code{x}, and a procedure body \code{(* x x)}.  The
environment part of the procedure is a pointer to the global environment, since
that is the environment in which the λ-expression was evaluated to
produce the procedure. A new binding, which associates the procedure object
with the symbol \code{square}, has been added to the global frame.  In general,
\code{define} creates definitions by adding bindings to frames.

\begin{figure}[tb]
\phantomsection\label{Figure 3.2}
\centering
\begin{comment}
\heading{Figure 3.2:} Environment structure produced by
evaluating \code{(define (square x) (* x x))} in the global environment.

\begin{example}
           +----------------------+
           | other variables      |
global --->|                      |
env        | square: --+          |
           +-----------|----------+
                       |       ^
(define (square x)     |       |
  (* x x))             V       |
                   .---.---.   |
                   | O | O-+---+
                   `-|-^---'
                     |
                     V
                   parameters: x
                   body: (* x x)
\end{example}
\end{comment}
\includegraphics[width=49mm]{fig/chap3/Fig3.2b.pdf}
\begin{quote}
\heading{Figure 3.2:} Environment structure produced by evaluating \\ \code{(define (square x) (* x x))} in the global environment.
\end{quote}
\end{figure}

Now that we have seen how procedures are created, we can describe how
procedures are applied.  The environment model specifies: To apply a procedure
to arguments, create a new environment containing a frame that binds the
parameters to the values of the arguments.  The enclosing environment of this
frame is the environment specified by the procedure.  Now, within this new
environment, evaluate the procedure body.

To show how this rule is followed, \link{Figure 3.3} illustrates the environment
structure created by evaluating the expression \code{(square 5)} in the global
environment, where \code{square} is the procedure generated in \link{Figure 3.2}.  
Applying the procedure results in the creation of a new environment,
labeled E1 in the figure, that begins with a frame in which \code{x}, the
formal parameter for the procedure, is bound to the argument 5.  The pointer
leading upward from this frame shows that the frame's enclosing environment is
the global environment.  The global environment is chosen here, because this is
the environment that is indicated as part of the \code{square} procedure
object.  Within E1, we evaluate the body of the procedure, \code{(* x x)}.
Since the value of \code{x} in E1 is 5, the result is \code{(* 5 5)}, or 25.

\begin{figure}[tb]
\phantomsection\label{Figure 3.3}
\centering
\begin{comment}
\heading{Figure 3.3:} Environment created by evaluating
\code{(square 5)} in the global environment.

\begin{example}
          +------------------------------------+
          | other variables                    |
global -->|                                    |
env       | square: --+                        |
          +-----------|---------------------+--+
                      |       ^             ^
(square 5)            |       |             |
                      V       |             |
                  .---.---.   |         +---+--+
                  | O | O-+---+   E1 -->| x: 5 |
                  `-|-^---'             +------+
                    |
                    V
                  parameters: x
                  body: (* x x)
\end{example}
\end{comment}
\includegraphics[width=78mm]{fig/chap3/Fig3.3b.pdf}
\begin{quote}
\heading{Figure 3.3:} Environment created by evaluating \code{(square 5)} in the global environment.
\end{quote}
\end{figure}

The environment model of procedure application can be summarized by two rules:

\begin{itemize}

\item
A procedure object is applied to a set of arguments by constructing a frame,
binding the formal parameters of the procedure to the arguments of the call,
and then evaluating the body of the procedure in the context of the new
environment constructed.  The new frame has as its enclosing environment the
environment part of the procedure object being applied.

\item
A procedure is created by evaluating a λ-expression relative to a
given environment.  The resulting procedure object is a pair consisting of the
text of the λ-expression and a pointer to the environment in which
the procedure was created.

\end{itemize}

\noindent
We also specify that defining a symbol using \code{define} creates a binding in
the current environment frame and assigns to the symbol the indicated
value.\footnote{If there is already a binding for the variable in the current
frame, then the binding is changed.  This is convenient because it allows
redefinition of symbols; however, it also means that \code{define} can be used
to change values, and this brings up the issues of assignment without
explicitly using \code{set!}.  Because of this, some people prefer
redefinitions of existing symbols to signal errors or warnings.} Finally, we
specify the behavior of \code{set!}, the operation that forced us to introduce
the environment model in the first place.  Evaluating the expression
\code{(set!}\( \;\langle \)\var{variable}\( \kern0.08em\rangle \)\( \;\langle \)\var{value}\( \kern0.08em\rangle \)\code{)} in some environment locates the
binding of the variable in the environment and changes that binding to indicate
the new value.  That is, one finds the first frame in the environment that
contains a binding for the variable and modifies that frame.  If the variable
is unbound in the environment, then \code{set!} signals an error.

These evaluation rules, though considerably more complex than the substitution
model, are still reasonably straightforward.  Moreover, the evaluation model,
though abstract, provides a correct description of how the interpreter
evaluates expressions.  In \link{Chapter 4} we shall see how this model can
serve as a blueprint for implementing a working interpreter.  The following
sections elaborate the details of the model by analyzing some illustrative
programs.

\subsection{Applying Simple Procedures}
\label{Section 3.2.2}

When we introduced the substitution model in \link{Section 1.1.5} we showed how
the combination \code{(f 5)} evaluates to 136, given the following procedure
definitions:

\begin{scheme}
(define (square x)
  (* x x))
(define (sum-of-squares x y)
  (+ (square x) (square y)))
(define (f a)
  (sum-of-squares (+ a 1) (* a 2)))
\end{scheme}

\noindent
We can analyze the same example using the environment model.  \link{Figure 3.4}
shows the three procedure objects created by evaluating the definitions of
\code{f}, \code{square}, and \code{sum\-/of\-/squares} in the global environment.
Each procedure object consists of some code, together with a pointer to the
global environment.

\begin{figure}[tb]
\phantomsection\label{Figure 3.4}
\centering
\begin{comment}
\heading{Figure 3.4:} Procedure objects in the global frame.

\begin{example}
          +--------------------------------------------+
          | sum-of-squares:                            |
global -->| square:                                    |
env       | f: --+                                     |
          +------|--------------+--------------+-------+
                 |     ^        |     ^        |     ^
                 |     |        |     |        |     |
                 V     |        V     |        V     |
             .---.---. |    .---.---. |    .---.---. |
             | O | O-+-+    | O | O-+-+    | O | O-+-+
             `-|-^---'      `-|-^---'      `-|-^---'
               |              |              |
               V              V              V
   parameters: a          parameters: x  parameters: x, y
   body: (sum-of-squares  body: (* x x)  body: (+ (square x)
           (+ a 1)                                (square y))
           (* a 2))
\end{example}
\end{comment}
\includegraphics[width=106mm]{fig/chap3/Fig3.4a.pdf}
\par\bigskip
\noindent
\heading{Figure 3.4:} Procedure objects in the global frame.
\end{figure}

In \link{Figure 3.5} we see the environment structure created by evaluating the
expression \code{(f 5)}.  The call to \code{f} creates a new environment E1
beginning with a frame in which \code{a}, the formal parameter of \code{f}, is
bound to the argument 5.  In E1, we evaluate the body of \code{f}:

\begin{scheme}
(sum-of-squares (+ a 1) (* a 2))
\end{scheme}

\noindent
To evaluate this combination, we first evaluate the subexpressions.  The first
subexpression, \code{sum\-/of\-/squares}, has a value that is a procedure object.
(Notice how this value is found: We first look in the first frame of E1, which
contains no binding for \code{sum\-/of\-/squares}.  Then we proceed to the
enclosing environment, i.e. the global environment, and find the binding shown
in \link{Figure 3.4}.)  The other two subexpressions are evaluated by applying
the primitive operations \code{+} and \code{*} to evaluate the two combinations
\code{(+ a 1)} and \code{(* a 2)} to obtain 6 and 10, respectively.

Now we apply the procedure object \code{sum\-/of\-/squares} to the arguments 6 and
10.  This results in a new environment E2 in which the formal parameters
\code{x} and \code{y} are bound to the arguments.  Within E2 we evaluate the
combination \code{(+ (square x) (square y))}.  This leads us to evaluate
\code{(square x)}, where \code{square} is found in the global frame and
\code{x} is 6.  Once again, we set up a new environment, E3, in which \code{x}
is bound to 6, and within this we evaluate the body of \code{square}, which is
\code{(* x x)}.  Also as part of applying \code{sum\-/of\-/squares}, we must
evaluate the subexpression \code{(square y)}, where \code{y} is 10.  This
second call to \code{square} creates another environment, E4, in which
\code{x}, the formal parameter of \code{square}, is bound to 10.  And within E4
we must evaluate \code{(* x x)}.

\begin{figure}[tb]
\phantomsection\label{Figure 3.5}
\centering
\begin{comment}
\heading{Figure 3.5:} Environments created by evaluating
\code{(f 5)} using the procedures in \link{Figure 3.4}.

\begin{example}
          +-----------------------------------------------------+
global -->|                                                     |
env       +-----------------------------------------------------+
            ^              ^                ^               ^
(f 5)       |              |                |               |
        +------+       +-------+        +------+        +-------+
  E1 -->| a: 5 |  E2 ->| x: 6  |  E3 -->| x: 6 |  E4 -->| x: 10 |
        |      |       | y: 10 |        |      |        |       |
        +------+       +-------+        +------+        +-------+
   (sum-of-squares   (+ (square x)       (* x x)         (* x x)
     (+ a 1)            (square u))
     (+ a 2))
\end{example}
\end{comment}
\includegraphics[width=100mm]{fig/chap3/Fig3.5a.pdf}
\begin{quote}
\heading{Figure 3.5:} Environments created by evaluating \code{(f 5)} using the procedures in \link{Figure 3.4}.
\end{quote}
\end{figure}

The important point to observe is that each call to \code{square} creates a new
environment containing a binding for \code{x}.  We can see here how the
different frames serve to keep separate the different local variables all named
\code{x}.  Notice that each frame created by \code{square} points to the global
environment, since this is the environment indicated by the \code{square}
procedure object.

After the subexpressions are evaluated, the results are returned.  The values
generated by the two calls to \code{square} are added by \code{sum\-/of\-/squares},
and this result is returned by \code{f}.  Since our focus here is on the
environment structures, we will not dwell on how these returned values are
passed from call to call; however, this is also an important aspect of the
evaluation process, and we will return to it in detail in \link{Chapter 5}.

\begin{quote}
\heading{\phantomsection\label{Exercise 3.9}Exercise 3.9:} In \link{Section 1.2.1} we used the
substitution model to analyze two procedures for computing factorials, a
recursive version

\begin{scheme}
(define (factorial n)
  (if (= n 1) 1 (* n (factorial (- n 1)))))
\end{scheme}

\noindent
and an iterative version

\begin{scheme}
(define (factorial n) (fact-iter 1 1 n))
(define (fact-iter product counter max-count)
  (if (> counter max-count)
      product
      (fact-iter (* counter product)
                 (+ counter 1)
                 max-count)))
\end{scheme}

Show the environment structures created by evaluating \\ \code{(factorial 6)}
using each version of the \code{factorial} procedure.\footnote{The environment
model will not clarify our claim in \link{Section 1.2.1} that the interpreter
can execute a procedure such as \code{fact\-/iter} in a constant amount of space
using tail recursion.  We will discuss tail recursion when we deal with the
control structure of the interpreter in \link{Section 5.4}.}
\end{quote}

\subsection{Frames as the Repository of Local State}
\label{Section 3.2.3}

We can turn to the environment model to see how procedures and assignment can
be used to represent objects with local state.  As an example, consider the
``withdrawal processor'' from \link{Section 3.1.1} created by calling the
procedure

\begin{scheme}
(define (make-withdraw balance)
  (lambda (amount)
    (if (>= balance amount)
        (begin (set! balance (- balance amount))
               balance)
        "Insufficient funds")))
\end{scheme}

\noindent
Let us describe the evaluation of

\begin{scheme}
(define W1 (make-withdraw 100))
\end{scheme}

\noindent
followed by

\begin{scheme}
(W1 50)
~\textit{50}~
\end{scheme}

\noindent
\link{Figure 3.6} shows the result of defining the \code{make\-/withdraw}
procedure in the global environment.  This produces a procedure object that
contains a pointer to the global environment.  So far, this is no different
from the examples we have already seen, except that the body of the procedure
is itself a λ-expression.

\begin{figure}[tb]
\phantomsection\label{Figure 3.6}
\centering
\begin{comment}
\begin{quote}
\heading{Figure 3.6:} Result of defining \code{make\-/withdraw} in the global environment.

\begin{example}
          +---------------------------+
global -->| make-withdraw: --+        |
env       +------------------|--------+
                             |      ^
                             V      |
                         .---.---.  |
                         | O | O-+--+
                         `-|-^---'
                           |
                           V
         parameters: balance
         body: (lambda (amount)
                 (if (>= balance amount)
                     (begin (set! balance
                                  (- balance amount))
                            balance)
                     "Insufficient funds"))
\end{example}
\end{quote}
\end{comment}
\includegraphics[width=91mm]{fig/chap3/Fig3.6b.pdf}
\begin{quote}
\heading{Figure 3.6:} Result of defining \code{make\-/withdraw} in the global environment.
\end{quote}
\end{figure}

\begin{figure}[tb]
\phantomsection\label{Figure 3.7}
\centering
\begin{comment}
\heading{Figure 3.7:} Result of evaluating \code{(define W1 (make\-/withdraw 100))}.

\begin{example}
          +-----------------------------------------------+
          | make-withdraw: -----------------------+       |
global -->|                                       |       |
          | W1: --+                               |       |
          +-------|-------------------------------|-------+
                  |                ^              |     ^
                  |                |              V     |
                  |        +-------+------+   .---.---. |
                  |  E1 -->| balance: 100 |   | O | O-+-+
                  |        +--------------+   `-|-^---'
                  V                ^            |
              .---.---.            |            V
            +-+-O | O-+------------+    parameters: balance
            | `---^---'                 body: ...
            V
    parameters: amount
    body: (if (>= balance amount)
              (begin (set! balance (- balance amount))
                     balance)
              "Insufficient funds")
\end{example}
\end{comment}
\includegraphics[width=100mm]{fig/chap3/Fig3.7a.pdf}
\par\bigskip
\noindent
\heading{Figure 3.7:} Result of evaluating \code{(define W1 (make\-/withdraw 100))}. 
\end{figure}

The interesting part of the computation happens when we apply the procedure
\code{make\-/withdraw} to an argument:

\begin{scheme}
(define W1 (make-withdraw 100))
\end{scheme}

\noindent
We begin, as usual, by setting up an environment E1 in which the formal
parameter \code{balance} is bound to the argument 100.  Within this
environment, we evaluate the body of \code{make\-/withdraw}, namely the
λ-expression.  This constructs a new procedure object, whose code
is as specified by the \code{lambda} and whose environment is E1, the
environment in which the \code{lambda} was evaluated to produce the procedure.
The resulting procedure object is the value returned by the call to
\code{make\-/withdraw}.  This is bound to \code{W1} in the global environment,
since the \code{define} itself is being evaluated in the global environment.
\link{Figure 3.7} shows the resulting environment structure.

\begin{figure}[tb]
\phantomsection\label{Figure 3.8}
\centering
\begin{comment}
\heading{Figure 3.8:} Environments created by applying the procedure object \code{W1}.

\begin{example}
          +---------------------------------------------------+
          | make-withdraw: ...                                |
global -->|                                                   |
env       | W1: --+                                           |
          +-------|-------------------------------------------+
                  |               ^
                  |               |
                  |       +-------+------+ Here is the balance
                  | E1 -->| balance: 100 | that will be changed
                  |       +--------------+ by the set!.
                  V               ^   ^
              .---.---.           |   +----+
              | O | O-+-----------+        |
              `-|-^---'             +------+-----+
                |                   | amount: 50 |
                V                   +------------+
      parameters: amount   (if (>= balance amount)
      body: ...                (begin (set! balance
                                            (- balance amount))
                                      balance)
                               "Insufficient funds")
\end{example}
\end{comment}
\includegraphics[width=99mm]{fig/chap3/Fig3.8c.pdf}
\par\bigskip
\noindent
\heading{Figure 3.8:} Environments created by applying the procedure object \code{W1}. 
\end{figure}

\enlargethispage{\baselineskip}

Now we can analyze what happens when \code{W1} is applied to an argument:

\begin{scheme}
(W1 50)
~\textit{50}~
\end{scheme}

We begin by constructing a frame in which \code{amount}, the formal parameter
of \code{W1}, is bound to the argument 50.  The crucial point to observe is
that this frame has as its enclosing environment not the global environment,
but rather the environment E1, because this is the environment that is
specified by the \code{W1} procedure object.  Within this new environment, we
evaluate the body of the procedure:

\begin{scheme}
(if (>= balance amount)
    (begin (set! balance (- balance amount))
           balance)
    "Insufficient funds")
\end{scheme}

\noindent
The resulting environment structure is shown in \link{Figure 3.8}.  The
expression being evaluated references both \code{amount} and \code{balance}.
\code{Amount} will be found in the first frame in the environment, while
\code{balance} will be found by following the enclosing-environment pointer to
E1.

\begin{figure}[tb]
\phantomsection\label{Figure 3.9}
\centering
\begin{comment}
\heading{Figure 3.9:} Environments after the call to \code{W1}.

\begin{example}
           +------------------------------------+
           | make-withdraw: ...                 |
global --->|                                    |
env        | W1: --+                            |
           +-------|----------------------------+
                   |                   ^
                   |                   |
                   |            +------+------+
                   |     E1 --->| balance: 50 |
                   |            +-------------+
                   V                   ^
               .---.---.               |
               | O | O-+---------------+
               `-|-^---'
                 |
                 V
          parameters: amount
          body: ...
\end{example}
\end{comment}
\includegraphics[width=96mm]{fig/chap3/Fig3.9a.pdf}
\par\bigskip
\noindent
\heading{Figure 3.9:} Environments after the call to \code{W1}.
\end{figure}

\enlargethispage{\baselineskip}

When the \code{set!} is executed, the binding of \code{balance} in E1 is
changed.  At the completion of the call to \code{W1}, \code{balance} is 50, and
the frame that contains \code{balance} is still pointed to by the procedure
object \code{W1}.  The frame that binds \code{amount} (in which we executed the
code that changed \code{balance}) is no longer relevant, since the procedure
call that constructed it has terminated, and there are no pointers to that
frame from other parts of the environment.  The next time \code{W1} is called,
this will build a new frame that binds \code{amount} and whose enclosing
environment is E1.  We see that E1 serves as the ``place'' that holds the local
state variable for the procedure object \code{W1}.  \link{Figure 3.9} shows the
situation after the call to \code{W1}.

Observe what happens when we create a second ``withdraw'' object by making
another call to \code{make\-/withdraw}:

\begin{scheme}
(define W2 (make-withdraw 100))
\end{scheme}

\begin{figure}[tb]
\phantomsection\label{Figure 3.10}
\centering
\begin{comment}
\heading{Figure 3.10:} Using \code{(define W2 (make\-/withdraw 100))} to create a second object.

\begin{example}
         +-------------------------------------------------+
         | make-withdraw: ...                              |
global ->| W2: ---------------------------+                |
env      | W1: --+                        |                |
         +-------|------------------------|----------------+
                 |              ^         |              ^
                 |              |         |              |
                 |       +------+------+  |       +------+-------+
                 |  E1 ->| balance: 50 |  |  E2 ->| balance: 100 |
                 |       +-------------+  |       +--------------+
                 V              ^         V              ^
             .---.---.          |     .---.---.          |
             | O | O-+----------+     | O | O-+----------+
             `-|-^---'                `-|-^---'
               | +----------------------+
               V V
        parameters: amount
        body: ...
\end{example}
\end{comment}
\includegraphics[width=108mm]{fig/chap3/Fig3.10a.pdf}
\begin{quote}
\heading{Figure 3.10:} Using \code{(define W2 (make\-/withdraw 100))} to create a second object.
\end{quote}
\end{figure}

\enlargethispage{\baselineskip}

This produces the environment structure of \link{Figure 3.10}, which shows that
\code{W2} is a procedure object, that is, a pair with some code and an
environment.  The environment E2 for \code{W2} was created by the call to
\code{make\-/withdraw}.  It contains a frame with its own local binding for
\code{balance}.  On the other hand, \code{W1} and \code{W2} have the same code:
the code specified by the λ-expression in the body of
\code{make\-/withdraw}.\footnote{Whether \code{W1} and \code{W2} share the same
physical code stored in the computer, or whether they each keep a copy of the
code, is a detail of the implementation.  For the interpreter we implement in
\link{Chapter 4}, the code is in fact shared.} We see here why \code{W1} and
\code{W2} behave as independent objects.  Calls to \code{W1} reference the
state variable \code{balance} stored in E1, whereas calls to \code{W2}
reference the \code{balance} stored in E2. Thus, changes to the local state of
one object do not affect the other object.

\begin{quote}
\heading{\phantomsection\label{Exercise 3.10}Exercise 3.10:} In the \code{make\-/withdraw}
procedure, the local variable \code{balance} is created as a parameter of
\code{make\-/withdraw}.  We could also create the local state variable
explicitly, using \code{let}, as follows:

\begin{scheme}
(define (make-withdraw initial-amount)
  (let ((balance initial-amount))
    (lambda (amount)
      (if (>= balance amount)
          (begin (set! balance (- balance amount))
                 balance)
          "Insufficient funds"))))
\end{scheme}

Recall from \link{Section 1.3.2} that \code{let} is simply syntactic sugar for a
procedure call:

\begin{scheme}
(let ((~\( \dark \langle \)~~\var{\dark var}~~\( \dark \rangle \)~ ~\( \dark \langle \)~~\var{\dark exp}~~\( \dark \rangle \)~)) ~\( \dark \langle \)~~\var{\dark body}~~\( \dark \rangle \)~)
\end{scheme}

\noindent
is interpreted as an alternate syntax for

\begin{scheme}
((lambda (~\( \dark \langle \)~~\var{\dark var}~~\( \dark \rangle \)~) ~\( \dark \langle \)~~\var{\dark body}~~\( \dark \rangle \)~) ~\( \dark \langle \)~~\var{\dark exp}~~\( \dark \rangle \)~)
\end{scheme}

Use the environment model to analyze this alternate version of
\code{make\-/withdraw}, drawing figures like the ones above to illustrate the
interactions

\begin{scheme}
(define W1 (make-withdraw 100))
(W1 50)
(define W2 (make-withdraw 100))
\end{scheme}

Show that the two versions of \code{make\-/withdraw} create objects with the same
behavior.  How do the environment structures differ for the two versions?
\end{quote}

\subsection{Internal Definitions}
\label{Section 3.2.4}

\link{Section 1.1.8} introduced the idea that procedures can have internal
definitions, thus leading to a block structure as in the following procedure to
compute square roots:

\begin{scheme}
(define (sqrt x)
  (define (good-enough? guess)
    (< (abs (- (square guess) x)) 0.001))
  (define (improve guess)
    (average guess (/ x guess)))
  (define (sqrt-iter guess)
    (if (good-enough? guess)
        guess
        (sqrt-iter (improve guess))))
  (sqrt-iter 1.0))
\end{scheme}

\noindent
Now we can use the environment model to see why these internal definitions
behave as desired.  \link{Figure 3.11} shows the point in the evaluation of the
expression \code{(sqrt 2)} where the internal procedure \code{good\-/enough?} has
been called for the first time with \code{guess} equal to 1.

Observe the structure of the environment.  \code{Sqrt} is a symbol in the
global environment that is bound to a procedure object whose associated
environment is the global environment.  When \code{sqrt} was called, a new
environment E1 was formed, subordinate to the global environment, in which the
parameter \code{x} is bound to 2.  The body of \code{sqrt} was then evaluated
in E1.  Since the first expression in the body of \code{sqrt} is

\begin{scheme}
(define (good-enough? guess)
  (< (abs (- (square guess) x)) 0.001))
\end{scheme}

\noindent
evaluating this expression defined the procedure \code{good\-/enough?}  in the
environment E1.  To be more precise, the symbol \code{good\-/enough?} was added
to the first frame of E1, bound to a procedure object whose associated
environment is E1.  Similarly, \code{improve} and \code{sqrt\-/iter} were defined
as procedures in E1.  For conciseness, \link{Figure 3.11} shows only the
procedure object for \code{good\-/enough?}.

\begin{figure}[tb]
\phantomsection\label{Figure 3.11}
\centering
\begin{comment}
\heading{Figure 3.11:} \code{Sqrt} procedure with internal definitions.

\begin{example}
          +--------------------------------------------------+
global -->| sqrt: --+                                        |
env       |         |                                        |
          +---------|----------------------------------------+
                    V       ^                   ^
                .---.---.   |                   |
     +----------+-O | O-+---+        +----------+------------+
     |          `---^---'            | x: 2                  |
     V                         E1 -->| good-enough?: -+      |
parameters: x                        | improve: ...   |      |
body: (define good-enough? ...)      | sqrt-iter: ... |      |
      (define improve ...)           +----------------|------+
      (define sqrt-iter ...)          ^  ^            |     ^
      (sqrt-iter 1.0)                 |  |            V     |
                            +---------++ |        .---.---. |
                      E2 -->| guess: 1 | |        | O | O-+-+
                            +----------+ |        `-|-^---'
                      call to sqrt-iter  |          |
                                         |          V
                               +---------++    parameters: guess
                         E3 -->| guess: 1 |    body: (< (abs ...)
                               +----------+             ...)
                         call to good-enough?
\end{example}
\end{comment}
\includegraphics[width=107mm]{fig/chap3/Fig3.11a.pdf}
\begin{quote}
\heading{Figure 3.11:} \code{Sqrt} procedure with internal definitions.
\end{quote}
\end{figure}

After the local procedures were defined, the expression \code{(sqrt\-/iter 1.0)}
was evaluated, still in environment E1.  So the procedure object bound to
\code{sqrt\-/iter} in E1 was called with 1 as an argument.  This created an
environment E2 in which \code{guess}, the parameter of \code{sqrt\-/iter}, is
bound to 1.  \code{Sqrt\-/iter} in turn called \code{good\-/enough?} with the value
of \code{guess} (from E2) as the argument for \code{good\-/enough?}.  This set up
another environment, E3, in which \code{guess} (the parameter of
\code{good\-/enough?}) is bound to 1.  Although \code{sqrt\-/iter} and
\code{good\-/enough?} both have a parameter named \code{guess}, these are two
distinct local variables located in different frames.  Also, E2 and E3 both
have E1 as their enclosing environment, because the \code{sqrt\-/iter} and
\code{good\-/enough?} procedures both have E1 as their environment part.  One
consequence of this is that the symbol \code{x} that appears in the body of
\code{good\-/enough?} will reference the binding of \code{x} that appears in E1,
namely the value of \code{x} with which the original \code{sqrt} procedure was
called.

The environment model thus explains the two key properties that make local
procedure definitions a useful technique for modularizing programs:

\begin{itemize}

\item
The names of the local procedures do not interfere with names external to the
enclosing procedure, because the local procedure names will be bound in the
frame that the procedure creates when it is run, rather than being bound in the
global environment.

\item
The local procedures can access the arguments of the enclosing procedure,
simply by using parameter names as free variables.  This is because the body of
the local procedure is evaluated in an environment that is subordinate to the
evaluation environment for the enclosing procedure.

\end{itemize}

\begin{quote}
\heading{\phantomsection\label{Exercise 3.11}Exercise 3.11:} In \link{Section 3.2.3} we saw how
the environment model described the behavior of procedures with local state.
Now we have seen how internal definitions work.  A typical message-passing
procedure contains both of these aspects.  Consider the bank account procedure
of \link{Section 3.1.1}:

\begin{scheme}
(define (make-account balance)
  (define (withdraw amount)
    (if (>= balance amount)
        (begin (set! balance (- balance amount))
               balance)
        "Insufficient funds"))
  (define (deposit amount)
    (set! balance (+ balance amount))
    balance)
  (define (dispatch m)
    (cond ((eq? m 'withdraw) withdraw)
          ((eq? m 'deposit) deposit)
          (else 
           (error "Unknown request: MAKE-ACCOUNT"
                  m))))
  dispatch)
\end{scheme}

Show the environment structure generated by the sequence of interactions

\begin{scheme}
(define acc (make-account 50))
((acc 'deposit) 40)
~\textit{90}~
((acc 'withdraw) 60)
~\textit{30}~
\end{scheme}

Where is the local state for \code{acc} kept?  Suppose we define another
account

\begin{scheme}
(define acc2 (make-account 100))
\end{scheme}

How are the local states for the two accounts kept distinct?  Which parts of
the environment structure are shared between \code{acc} and \code{acc2}?
\end{quote}

\section{Modeling with Mutable Data}
\label{Section 3.3}

Chapter 2 dealt with compound data as a means for constructing computational
objects that have several parts, in order to model real-world objects that have
several aspects.  In that chapter we introduced the discipline of data
abstraction, according to which data structures are specified in terms of
constructors, which create data objects, and selectors, which access the parts
of compound data objects.  But we now know that there is another aspect of data
that \link{Chapter 2} did not address.  The desire to model systems composed of
objects that have changing state leads us to the need to modify compound data
objects, as well as to construct and select from them.  In order to model
compound objects with changing state, we will design data abstractions to
include, in addition to selectors and constructors, operations called
\newterm{mutators}, which modify data objects.  For instance, modeling a
banking system requires us to change account balances.  Thus, a data structure
for representing bank accounts might admit an operation

\begin{scheme}
(set-balance! ~\( \dark \langle \)~~\var{\dark account}~~\( \dark \rangle \)~ ~\( \dark \langle \)~~\var{\dark new-value}~~\( \dark \rangle \)~)
\end{scheme}

\noindent
that changes the balance of the designated account to the designated new value.
Data objects for which mutators are defined are known as \newterm{mutable data
objects}.

\link{Chapter 2} introduced pairs as a general-purpose ``glue'' for synthesizing
compound data.  We begin this section by defining basic mutators for pairs, so
that pairs can serve as building blocks for constructing mutable data objects.
These mutators greatly enhance the representational power of pairs, enabling us
to build data structures other than the sequences and trees that we worked with
in \link{Section 2.2}.  We also present some examples of simulations in which
complex systems are modeled as collections of objects with local state.



\subsection{Mutable List Structure}
\label{Section 3.3.1}

The basic operations on pairs---\code{cons}, \code{car}, and \code{cdr}---can
be used to construct list structure and to select parts from list structure,
but they are incapable of modifying list structure.  The same is true of the
list operations we have used so far, such as \code{append} and \code{list},
since these can be defined in terms of \code{cons}, \code{car}, and \code{cdr}.
To modify list structures we need new operations.

The primitive mutators for pairs are \code{set\-/car!} and
\code{set\-/cdr!}. \code{Set\-/car!} takes two arguments, the first of which must
be a pair.  It modifies this pair, replacing the \code{car} pointer by a
pointer to the second argument of \code{set\-/car!}.\footnote{\code{Set\-/car!} and
\code{set\-/cdr!} return implementation-dependent values.  Like \code{set!}, they
should be used only for their effect.}

As an example, suppose that \code{x} is bound to the list \code{((a b) c d)}
and \code{y} to the list \code{(e f)} as illustrated in \link{Figure 3.12}.
Evaluating the expression \code{ (set\-/car!  x y)} modifies the pair to which
\code{x} is bound, replacing its \code{car} by the value of \code{y}.  The
result of the operation is shown in \link{Figure 3.13}.  The structure \code{x}
has been modified and would now be printed as \code{((e f) c d)}.  The pairs
representing the list \code{(a b)}, identified by the pointer that was
replaced, are now detached from the original structure.\footnote{We see from
this that mutation operations on lists can create ``garbage'' that is not part
of any accessible structure.  We will see in \link{Section 5.3.2} that Lisp
memory-management systems include a \newterm{garbage collector}, which
identifies and recycles the memory space used by unneeded pairs.}

Compare \link{Figure 3.13} with \link{Figure 3.14}, which illustrates the result
of executing \code{(define z (cons y (cdr x)))} with \code{x} and \code{y}
bound to the original lists of \link{Figure 3.12}.  The variable \code{z} is now
bound to a new pair created by the \code{cons} operation; the list to which
\code{x} is bound is unchanged.

\begin{figure}[tp]
\phantomsection\label{Figure 3.12}
\centering
\begin{comment}
\heading{Figure 3.12:} Lists \code{x}: \code{((a b) c d)} and \code{y}: \code{(e f)}.

\begin{example}
     +---+---+     +---+---+     +---+---+
x -->| * | *-+---->| * | *-+---->| * | / |
     +-|-+---+     +-|-+---+     +-|-+---+
       |             V             V
       |           +---+         +---+
       |           | c |         | d |
       |           +---+         +---+
       |           +---+---+     +---+---+
       +---------->| * | *-+---->| * | / |
                   +-|-+---+     +-|-+---+
                     V             V
                   +---+         +---+
                   | a |         | b |
                   +---+         +---+
                   +---+---+     +---+---+
              y -->| * | *-+---->| * | / |
                   +-|-+---+     +-|-+---+
                     V             V
                   +---+         +---+
                   | e |         | f |
                   +---+         +---+
\end{example}
\end{comment}
\includegraphics[width=72mm]{fig/chap3/Fig3.12b.pdf}
\begin{quote}
\heading{Figure 3.12:} Lists \code{x}: \code{((a b) c d)} and \code{y}: \code{(e f)}.
\end{quote}
\end{figure}

\enlargethispage{\baselineskip}

\begin{figure}[bp]
\phantomsection\label{Figure 3.13}
\centering
\begin{comment}
\heading{Figure 3.13:} Effect of \code{(set\-/car! x y)} on the lists in \link{Figure 3.12}.

\begin{example}
     +---+---+     +---+---+     +---+---+
x -->| * | *-+---->| * | *-+---->| * | / |
     +-|-+---+     +-|-+---+     +-|-+---+
       |             V             V
       |           +---+         +---+
       |           | c |         | d |
       |           +---+         +---+
       |           +---+---+     +---+---+
       |           | * | *-+---->| * | / |
       |           +-|-+---+     +-|-+---+
       |             V             V
       |           +---+         +---+
       |           | a |         | b |
       |           +---+         +---+
       +---------->+---+---+     +---+---+
                   | * | *-+---->| * | / |
              y -->+-|-+---+     +-|-+---+
                     V             V
                   +---+         +---+
                   | e |         | f |
                   +---+         +---+
\end{example}
\end{comment}
\includegraphics[width=72mm]{fig/chap3/Fig3.13b.pdf}
\par\bigskip
\noindent
\heading{Figure 3.13:} Effect of \code{(set\-/car! x y)} on the lists in \link{Figure 3.12}. 
\end{figure}

\begin{figure}[tp]
\phantomsection\label{Figure 3.14}
\centering
\begin{comment}
\heading{Figure 3.14:} Effect of \code{(define z (cons y (cdr x)))} on the lists in \link{Figure 3.12}.

\begin{example}
     +---+---+     +---+---+     +---+---+
x -->| * | *-+---->| * | *-+---->| * | / |
     +-|-+---+ +-->+-|-+---+     +-|-+---+
       |       |     V             V
       |       |   +---+         +---+
       |       |   | c |         | d |
       |       |   +---+         +---+
       |       |   +---+---+     +---+---+
       +-------+-->| * | *-+---->| * | / |
               |   +-|-+---+     +-|-+---+
     +---+---+ |     V             V
z -->| * | *-+-+   +---+         +---+
     +-|-+---+     | a |         | b |
       |           +---+         +---+
       +---------->+---+---+     +---+---+
                   | * | *-+---->| * | / |
              y -->+-|-+---+     +-|-+---+
                     V             V
                   +---+         +---+
                   | e |         | f |
                   +---+         +---+
\end{example}
\end{comment}
\includegraphics[width=72mm]{fig/chap3/Fig3.14b.pdf}
\begin{quote}
\heading{Figure 3.14:} Effect of \code{(define z (cons y (cdr x)))} on the lists in \link{Figure 3.12}.
\end{quote}
\end{figure}

\enlargethispage{\baselineskip}

\begin{figure}[bp]
\phantomsection\label{Figure 3.15}
\centering
\begin{comment}
\heading{Figure 3.15:} Effect of \code{(set\-/cdr! x y)} on the lists in \link{Figure 3.12}.

\begin{example}
     +---+---+     +---+---+     +---+---+
x -->| * | * |     | * | *-+---->| * | / |
     +-|-+-|-+     +-|-+---+     +-|-+---+
       |   |         V             V
       |   |       +---+         +---+
       |   |       | c |         | d |
       |   |       +---+         +---+
       |   |       +---+---+     +---+---+
       +---+------>| * | *-+---->| * | / |
           |       +-|-+---+     +-|-+---+
           |         V             V
           |       +---+         +---+
           |       | a |         | b |
           |       +---+         +---+
           +------>+---+---+     +---+---+
                   | * | *-+---->| * | / |
              y -->+-|-+---+     +-|-+---+
                     V             V
                   +---+         +---+
                   | e |         | f |
                   +---+         +---+
\end{example}
\end{comment}
\includegraphics[width=72mm]{fig/chap3/Fig3.15b.pdf}
\par\bigskip
\noindent
\heading{Figure 3.15:} Effect of \code{(set\-/cdr! x y)} on the lists in \link{Figure 3.12}. 
\end{figure}

The \code{set\-/cdr!} operation is similar to \code{set\-/car!}.  The only
difference is that the \code{cdr} pointer of the pair, rather than the
\code{car} pointer, is replaced.  The effect of executing \code{(set\-/cdr! x y)}
on the lists of \link{Figure 3.12} is shown in \link{Figure 3.15}.  Here the
\code{cdr} pointer of \code{x} has been replaced by the pointer to \code{(e
f)}.  Also, the list \code{(c d)}, which used to be the \code{cdr} of \code{x},
is now detached from the structure.

\code{Cons} builds new list structure by creating new pairs, while
\code{set\-/car!} and \code{set\-/cdr!} modify existing pairs.  Indeed, we could
implement \code{cons} in terms of the two mutators, together with a procedure
\code{get\-/new\-/pair}, which returns a new pair that is not part of any existing
list structure.  We obtain the new pair, set its \code{car} and \code{cdr}
pointers to the designated objects, and return the new pair as the result of
the \code{cons}.\footnote{\code{Get\-/new\-/pair} is one of the operations that
must be implemented as part of the memory management required by a Lisp
implementation.  We will discuss this in \link{Section 5.3.1}.}

\begin{scheme}
(define (cons x y)
  (let ((new (get-new-pair)))
    (set-car! new x)
    (set-cdr! new y)
    new))
\end{scheme}

\begin{quote}
\heading{\phantomsection\label{Exercise 3.12}Exercise 3.12:} The following procedure for
appending lists was introduced in \link{Section 2.2.1}:

\begin{scheme}
(define (append x y)
  (if (null? x)
      y
      (cons (car x) (append (cdr x) y))))
\end{scheme}

\code{Append} forms a new list by successively \code{cons}ing the elements of
\code{x} onto \code{y}.  The procedure \code{append!} is similar to
\code{append}, but it is a mutator rather than a constructor.  It appends the
lists by splicing them together, modifying the final pair of \code{x} so that
its \code{cdr} is now \code{y}.  (It is an error to call \code{append!} with an
empty \code{x}.)

\begin{scheme}
(define (append! x y)
  (set-cdr! (last-pair x) y)
  x)
\end{scheme}

Here \code{last\-/pair} is a procedure that returns the last pair in its
argument:

\begin{scheme}
(define (last-pair x)
  (if (null? (cdr x)) x (last-pair (cdr x))))
\end{scheme}

Consider the interaction

\begin{scheme}
(define x (list 'a 'b))
(define y (list 'c 'd))
(define z (append x y))
z
~\textit{(a b c d)}~
(cdr x)
~\( \dark \langle \)~~\var{\dark response}~~\( \dark \rangle \)~
(define w (append! x y))
w
~\textit{(a b c d)}~
(cdr x)
~\( \dark \langle \)~~\var{\dark response}~~\( \dark \rangle \)~
\end{scheme}

What are the missing \( \langle \)\var{response}\( \rangle \)s?  
Draw box-and-pointer \\ diagrams to explain your answer.
\end{quote}

\begin{quote}
\heading{\phantomsection\label{Exercise 3.13}Exercise 3.13:} Consider the following
\code{make\-/cycle} procedure, which uses the \code{last\-/pair} procedure defined
in \link{Exercise 3.12}:

\begin{scheme}
(define (make-cycle x)
  (set-cdr! (last-pair x) x)
  x)
\end{scheme}

Draw a box-and-pointer diagram that shows the structure \code{z} created by

\begin{scheme}
(define z (make-cycle (list 'a 'b 'c)))
\end{scheme}

What happens if we try to compute \code{(last\-/pair z)}?
\end{quote}

\begin{quote}
\heading{\phantomsection\label{Exercise 3.14}Exercise 3.14:} The following procedure is quite
useful, although obscure:

\begin{scheme}
(define (mystery x)
  (define (loop x y)
    (if (null? x)
        y
        (let ((temp (cdr x)))
          (set-cdr! x y)
          (loop temp x))))
  (loop x '()))
\end{scheme}

\code{Loop} uses the ``temporary'' variable \code{temp} to hold the old value
of the \code{cdr} of \code{x}, since the \code{set\-/cdr!}  on the next line
destroys the \code{cdr}.  Explain what \code{mystery} does in general.  Suppose
\code{v} is defined by \code{(define v (list 'a 'b 'c 'd))}. Draw the
box-and-pointer diagram that represents the list to which \code{v} is bound.
Suppose that we now evaluate \code{(define w (mystery v))}. Draw
box-and-pointer diagrams that show the structures \code{v} and \code{w} after
evaluating this expression.  What would be printed as the values of \code{v}
and \code{w}?
\end{quote}

\subsubsection*{Sharing and identity}

We mentioned in \link{Section 3.1.3} the theoretical issues of ``sameness'' and
``change'' raised by the introduction of assignment.  These issues arise in
practice when individual pairs are \newterm{shared} among different data
objects.  For example, consider the structure formed by

\begin{scheme}
(define x (list 'a 'b))
(define z1 (cons x x))
\end{scheme}

\noindent
As shown in \link{Figure 3.16}, \code{z1} is a pair whose \code{car} and
\code{cdr} both point to the same pair \code{x}.  This sharing of \code{x} by
the \code{car} and \code{cdr} of \code{z1} is a consequence of the
straightforward way in which \code{cons} is implemented.  In general, using
\code{cons} to construct lists will result in an interlinked structure of pairs
in which many individual pairs are shared by many different structures.

In contrast to \link{Figure 3.16}, \link{Figure 3.17} shows the structure created
by

\begin{scheme}
(define z2 (cons (list 'a 'b) (list 'a 'b)))
\end{scheme}

\noindent
In this structure, the pairs in the two \code{(a b)} lists are distinct,
although the actual symbols are shared.\footnote{The two pairs are distinct
because each call to \code{cons} returns a new pair.  The symbols are shared;
in Scheme there is a unique symbol with any given name.  Since Scheme provides
no way to mutate a symbol, this sharing is undetectable.  Note also that the
sharing is what enables us to compare symbols using \code{eq?}, which simply
checks equality of pointers.}

\begin{figure}[tb]
\phantomsection\label{Figure 3.16}
\centering
\begin{comment}
\heading{Figure 3.16:} The list \code{z1} formed by \code{(cons x x)}.

\begin{example}
      +---+---+
z1 -->| * | * |
      +-|-+-|-+
        V   V
      +---+---+     +---+---+
 x -->| * | *-+---->| * | / |
      +-|-+---+     +-|-+---+
        V             V
      +---+         +---+
      | a |         | b |
      +---+         +---+
\end{example}
\end{comment}
\includegraphics[width=46mm]{fig/chap3/Fig3.16b.pdf}
\begin{quote}
\heading{Figure 3.16:} The list \code{z1} formed by \code{(cons x x)}.
\end{quote}
\end{figure}

\begin{figure}[tb]
\phantomsection\label{Figure 3.17}
\centering
\begin{comment}
\begin{quote}
\heading{Figure 3.17:} The list \code{z2} formed by \code{(cons (list 'a 'b) (list 'a 'b))}.

\begin{example}
      +---+---+     +---+---+     +---+---+
z2 -->| * | *-+---->| * | *-+---->| * | / |
      +-|-+---+     +-|-+---+     +-|-+---+
        |             V             V
        |           +---+         +---+
        |           | a |         | b |
        |           +---+         +---+
        |             ^             ^
        |             |             |
        |           +-|-+---+     +-|-+---+
        +---------->| * | *-+---->| * | / |
                    +---+---+     +---+---+
\end{example}
\end{quote}
\end{comment}
\includegraphics[width=71mm]{fig/chap3/Fig3.17b.pdf}
\begin{quote}
\heading{Figure 3.17:} The list \code{z2} formed by \code{(cons (list 'a 'b) (list 'a 'b))}.
\end{quote}
\end{figure}

When thought of as a list, \code{z1} and \code{z2} both represent ``the same''
list, \code{((a b) a b)}.  In general, sharing is completely undetectable if we
operate on lists using only \code{cons}, \code{car}, and \code{cdr}.  However,
if we allow mutators on list structure, sharing becomes significant.  As an
example of the difference that sharing can make, consider the following
procedure, which modifies the \code{car} of the structure to which it is
applied:

\begin{scheme}
(define (set-to-wow! x) (set-car! (car x) 'wow) x)
\end{scheme}

\noindent
Even though \code{z1} and \code{z2} are ``the same'' structure, applying
\code{set\-/to\-/wow!} to them yields different results.  With \code{z1}, altering
the \code{car} also changes the \code{cdr}, because in \code{z1} the \code{car}
and the \code{cdr} are the same pair.  With \code{z2}, the \code{car} and
\code{cdr} are distinct, so \code{set\-/to\-/wow!} modifies only the \code{car}:

\begin{scheme}
z1
~\textit{((a b) a b)}~
(set-to-wow! z1)
~\textit{((wow b) wow b)}~
z2
~\textit{((a b) a b)}~
(set-to-wow! z2)
~\textit{((wow b) a b)}~
\end{scheme}

\noindent
One way to detect sharing in list structures is to use the predicate
\code{eq?}, which we introduced in \link{Section 2.3.1} as a way to test whether
two symbols are equal.  More generally, \code{(eq?  x y)} tests whether
\code{x} and \code{y} are the same object (that is, whether \code{x} and
\code{y} are equal as pointers).  Thus, with \code{z1} and \code{z2} as defined
in \link{Figure 3.16} and \link{Figure 3.17}, \code{(eq?  (car z1) (cdr
z1))} is true and \code{(eq? (car z2) (cdr z2))} is false.

As will be seen in the following sections, we can exploit sharing to greatly
extend the repertoire of data structures that can be represented by pairs.  On
the other hand, sharing can also be dangerous, since modifications made to
structures will also affect other structures that happen to share the modified
parts.  The mutation operations \code{set\-/car!} and \code{set\-/cdr!} should be
used with care; unless we have a good understanding of how our data objects are
shared, mutation can have unanticipated results.\footnote{The subtleties of
dealing with sharing of mutable data objects reflect the underlying issues of
``sameness'' and ``change'' that were raised in \link{Section 3.1.3}.  We
mentioned there that admitting change to our language requires that a compound
object must have an ``identity'' that is something different from the pieces
from which it is composed.  In Lisp, we consider this ``identity'' to be the
quality that is tested by \code{eq?}, i.e., by equality of pointers.  Since in
most Lisp implementations a pointer is essentially a memory address, we are
``solving the problem'' of defining the identity of objects by stipulating that
a data object ``itself\( \kern0.1em \)'' is the information stored in some particular set of
memory locations in the computer.  This suffices for simple Lisp programs, but
is hardly a general way to resolve the issue of ``sameness'' in computational
models.}

\begin{quote}
\heading{\phantomsection\label{Exercise 3.15}Exercise 3.15:} Draw box-and-pointer diagrams to
explain the effect of \code{set\-/to\-/wow!} on the structures \code{z1} and
\code{z2} above.
\end{quote}

\begin{quote}
\heading{\phantomsection\label{Exercise 3.16}Exercise 3.16:} Ben Bitdiddle decides to write a
procedure to count the number of pairs in any list structure.  ``It's easy,''
he reasons.  ``The number of pairs in any structure is the number in the
\code{car} plus the number in the \code{cdr} plus one more to count the current
pair.''  So Ben writes the following procedure:

\begin{scheme}
(define (count-pairs x)
  (if (not (pair? x))
      0
      (+ (count-pairs (car x))
         (count-pairs (cdr x))
         1)))
\end{scheme}

Show that this procedure is not correct.  In particular, draw box-and-pointer
diagrams representing list structures made up of exactly three pairs for which
Ben's procedure would return 3; return 4; return 7; never return at all.
\end{quote}

\begin{quote}
\heading{\phantomsection\label{Exercise 3.17}Exercise 3.17:} Devise a correct version of the
\code{count\-/pairs} procedure of \link{Exercise 3.16} that returns the number of
distinct pairs in any structure.  (Hint: Traverse the structure, maintaining an
auxiliary data structure that is used to keep track of which pairs have already
been counted.)
\end{quote}

\begin{quote}
\heading{\phantomsection\label{Exercise 3.18}Exercise 3.18:} Write a procedure that examines a
list and determines whether it contains a cycle, that is, whether a program
that tried to find the end of the list by taking successive \code{cdr}s would
go into an infinite loop.  \link{Exercise 3.13} constructed such lists.
\end{quote}

\begin{quote}
\heading{\phantomsection\label{Exercise 3.19}Exercise 3.19:} Redo \link{Exercise 3.18} using an
algorithm that takes only a constant amount of space.  (This requires a very
clever idea.)
\end{quote}

\subsubsection*{Mutation is just assignment}

When we introduced compound data, we observed in \link{Section 2.1.3} that pairs
can be represented purely in terms of procedures:

\begin{scheme}
(define (cons x y)
  (define (dispatch m)
    (cond ((eq? m 'car) x)
          ((eq? m 'cdr) y)
          (else (error "Undefined operation: CONS" m))))
  dispatch)
(define (car z) (z 'car))
(define (cdr z) (z 'cdr))
\end{scheme}

\noindent
The same observation is true for mutable data.  We can implement mutable data
objects as procedures using assignment and local state.  For instance, we can
extend the above pair implementation to handle \code{set\-/car!} and
\code{set\-/cdr!} in a manner analogous to the way we implemented bank accounts
using \code{make\-/account} in \link{Section 3.1.1}:

\begin{scheme}
(define (cons x y)
  (define (set-x! v) (set! x v))
  (define (set-y! v) (set! y v))
  (define (dispatch m)
    (cond ((eq? m 'car) x)
          ((eq? m 'cdr) y)
          ((eq? m 'set-car!) set-x!)
          ((eq? m 'set-cdr!) set-y!)
          (else 
           (error "Undefined operation: CONS" m))))
  dispatch)
(define (car z) (z 'car))
(define (cdr z) (z 'cdr))
(define (set-car! z new-value) 
  ((z 'set-car!) new-value) z)
(define (set-cdr! z new-value)
  ((z 'set-cdr!) new-value) z)
\end{scheme}

\noindent
Assignment is all that is needed, theoretically, to account for the behavior of
mutable data.  As soon as we admit \code{set!} to our language, we raise all
the issues, not only of assignment, but of mutable data in general.\footnote{On
the other hand, from the viewpoint of implementation, assignment requires us to
modify the environment, which is itself a mutable data structure.  Thus,
assignment and mutation are equipotent: Each can be implemented in terms of the
other.}

\begin{quote}
\heading{\phantomsection\label{Exercise 3.20}Exercise 3.20:} Draw environment diagrams to
illustrate the evaluation of the sequence of expressions

\begin{scheme}
(define x (cons 1 2))
(define z (cons x x))
(set-car! (cdr z) 17)
(car x)
~\textit{17}~
\end{scheme}

\noindent
using the procedural implementation of pairs given above.  (Compare
\link{Exercise 3.11}.)
\end{quote}

\subsection{Representing Queues}
\label{Section 3.3.2}

The mutators \code{set\-/car!} and \code{set\-/cdr!} enable us to use pairs to
construct data structures that cannot be built with \code{cons}, \code{car},
and \code{cdr} alone.  This section shows how to use pairs to represent a data
structure called a queue.  \link{Section 3.3.3} will show how to represent data
structures called tables.

A \newterm{queue} is a sequence in which items are inserted at one end (called
the \newterm{rear} of the queue) and deleted from the other end (the
\newterm{front}).  \link{Figure 3.18} shows an initially empty queue in which
the items \code{a} and \code{b} are inserted.  Then \code{a} is removed,
\code{c} and \code{d} are inserted, and \code{b} is removed.  Because items are
always removed in the order in which they are inserted, a queue is sometimes
called a \newterm{FIFO} (first in, first out) buffer.

\begin{figure}[tb]
\phantomsection\label{Figure 3.18}
\centering
\begin{comment}
\heading{Figure 3.18:} Queue operations.

\begin{example}
Operation                Resulting Queue
(define q (make-queue))
(insert-queue! q 'a)     a
(insert-queue! q 'b)     a b
(delete-queue! q)        b
(insert-queue! q 'c)     b c
(insert-queue! q 'd)     b c d
(delete-queue! q)        c d
\end{example}
\end{comment}
\includegraphics[width=70mm]{fig/chap3/Fig3.18a.pdf}
\par\bigskip
\noindent
\heading{Figure 3.18:} Queue operations.
\end{figure}

In terms of data abstraction, we can regard a queue as defined by the following
set of operations:

\begin{itemize}

\item
a constructor: \code{(make\-/queue)} returns an empty queue (a queue containing
no items).

\item
two selectors: 

\code{(empty\-/queue? \( \langle \)\var{queue}\( \rangle \))}
tests if the queue is empty.

\code{(front\-/queue \( \langle \)\var{queue}\( \rangle \))}
returns the object at the front of the queue, signaling an error if the queue
is empty; it does not modify the queue.

\item
two mutators:

\code{(insert\-/queue! \( \langle \)\var{queue}\( \rangle \) \( \langle \)\var{item}\( \rangle \))}
inserts the item at the rear of the queue and returns the modified queue as its
value.

\code{(delete\-/queue! \( \langle \)\var{queue}\( \rangle \))}
removes the item at the front of the queue and returns the modified queue as
its value, signaling an error if the queue is empty before the deletion.

\end{itemize}

\noindent
Because a queue is a sequence of items, we could certainly represent it as an
ordinary list; the front of the queue would be the \code{car} of the list,
inserting an item in the queue would amount to appending a new element at the
end of the list, and deleting an item from the queue would just be taking the
\code{cdr} of the list.  However, this representation is inefficient, because
in order to insert an item we must scan the list until we reach the end.  Since
the only method we have for scanning a list is by successive \code{cdr}
operations, this scanning requires \( \Theta(n) \) steps for a list of \( n \)
items.  A simple modification to the list representation overcomes this
disadvantage by allowing the queue operations to be implemented so that they
require \( \Theta \)(1) steps; that is, so that the number of steps needed is
independent of the length of the queue.

The difficulty with the list representation arises from the need to scan to
find the end of the list.  The reason we need to scan is that, although the
standard way of representing a list as a chain of pairs readily provides us
with a pointer to the beginning of the list, it gives us no easily accessible
pointer to the end.  The modification that avoids the drawback is to represent
the queue as a list, together with an additional pointer that indicates the
final pair in the list.  That way, when we go to insert an item, we can consult
the rear pointer and so avoid scanning the list.

A queue is represented, then, as a pair of pointers, \code{front\-/ptr} and
\code{rear\-/ptr}, which indicate, respectively, the first and last pairs in an
ordinary list.  Since we would like the queue to be an identifiable object, we
can use \code{cons} to combine the two pointers.  Thus, the queue itself will
be the \code{cons} of the two pointers.  \link{Figure 3.19} illustrates this
representation.

\begin{figure}[tb]
\phantomsection\label{Figure 3.19}
\centering
\begin{comment}
\begin{quote}
\heading{Figure 3.19:} Implementation of a queue as a list with front and rear pointers.

\begin{example}
       +---+---+
  q -->| * | *-+-------------------+
       +-|-+---+                   |
         |                         |
         | front-ptr               | rear-ptr
         V                         V
     +---+---+    +---+---+    +---+---+
     | * | *-+--->| * | *-+--->| * | / |
     +-|-+---+    +-|-+---+    +-|-+---+
       V            V            V
     +---+        +---+        +---+
     | a |        | b |        | c |
     +---+        +---+        +---+
\end{example}
\end{quote}
\end{comment}
\includegraphics[width=69mm]{fig/chap3/Fig3.19b.pdf}
\begin{quote}
\heading{Figure 3.19:} Implementation of a queue as a list with front and rear pointers.
\end{quote}
\end{figure}

To define the queue operations we use the following procedures, which enable us
to select and to modify the front and rear pointers of a queue:

\begin{scheme}
(define (front-ptr queue) (car queue))
(define (rear-ptr  queue) (cdr queue))
(define (set-front-ptr! queue item) 
  (set-car! queue item))
(define (set-rear-ptr!  queue item) 
  (set-cdr! queue item))
\end{scheme}

\noindent
Now we can implement the actual queue operations.  We will consider a queue to
be empty if its front pointer is the empty list:

\begin{scheme}
(define (empty-queue? queue) 
  (null? (front-ptr queue)))
\end{scheme}

\noindent
The \code{make\-/queue} constructor returns, as an initially empty queue, a pair
whose \code{car} and \code{cdr} are both the empty list:

\begin{scheme}
(define (make-queue) (cons '() '()))
\end{scheme}

\noindent
To select the item at the front of the queue, we return the \code{car} of the
pair indicated by the front pointer:

\begin{scheme}
(define (front-queue queue)
  (if (empty-queue? queue)
      (error "FRONT called with an empty queue" queue)
      (car (front-ptr queue))))
\end{scheme}

\begin{figure}[tb]
\phantomsection\label{Figure 3.20}
\centering
\begin{comment}
\heading{Figure 3.20:} Result of using \code{(insert\-/queue! q 'd)} on the queue of \link{Figure 3.19}.

\begin{example}
       +---+---+
  q -->| * | *-+--------------------------------+
       +-|-+---+                                |
         |                                      |
         | front-ptr                            | rear-ptr
         V                                      V
     +---+---+    +---+---+    +---+---+    +---+---+
     | * | *-+--->| * | *-+--->| * | *-+--->| * | / |
     +-|-+---+    +-|-+---+    +-|-+---+    +-|-+---+
       V            V            V            V
     +---+        +---+        +---+        +---+
     | a |        | b |        | c |        | d |
     +---+        +---+        +---+        +---+
\end{example}
\end{comment}
\includegraphics[width=88mm]{fig/chap3/Fig3.20b.pdf}
\begin{quote}
\heading{Figure 3.20:} Result of using \code{(insert\-/queue! q 'd)} on the queue of \link{Figure 3.19}.
\end{quote}
\end{figure}

\noindent
To insert an item in a queue, we follow the method whose result is indicated in
\link{Figure 3.20}.  We first create a new pair whose \code{car} is the item to
be inserted and whose \code{cdr} is the empty list.  If the queue was initially
empty, we set the front and rear pointers of the queue to this new pair.
Otherwise, we modify the final pair in the queue to point to the new pair, and
also set the rear pointer to the new pair.

\begin{scheme}
(define (insert-queue! queue item)
  (let ((new-pair (cons item '())))
    (cond ((empty-queue? queue)
           (set-front-ptr! queue new-pair)
           (set-rear-ptr! queue new-pair)
           queue)
          (else
           (set-cdr! (rear-ptr queue) new-pair)
           (set-rear-ptr! queue new-pair)
           queue))))
\end{scheme}

\begin{figure}[tb]
\phantomsection\label{Figure 3.21}
\centering
\begin{comment}
\heading{Figure 3.21:} Result of using \code{(delete\-/queue!  q)} on the queue of \link{Figure 3.20}.

\begin{example}
       +---+---+
  q -->| * | *-+--------------------------------+
       +-|-+---+                                |
         +------------+                         |
            front-ptr |                         | rear-ptr
                      V                         V
     +---+---+    +---+---+    +---+---+    +---+---+
     | * | *-+--->| * | *-+--->| * | *-+--->| * | / |
     +-|-+---+    +-|-+---+    +-|-+---+    +-|-+---+
       V            V            V            V
     +---+        +---+        +---+        +---+
     | a |        | b |        | c |        | d |
     +---+        +---+        +---+        +---+
\end{example}
\end{comment}
\includegraphics[width=88mm]{fig/chap3/Fig3.21b.pdf}
\begin{quote}
\heading{Figure 3.21:} Result of using \code{(delete\-/queue!  q)} on the queue of \link{Figure 3.20}.
\end{quote}
\end{figure}

\noindent
To delete the item at the front of the queue, we merely modify the front
pointer so that it now points at the second item in the queue, which can be
found by following the \code{cdr} pointer of the first item (see 
\link{Figure 3.21}):\footnote{If the first item is the final item in the queue, the front
pointer will be the empty list after the deletion, which will mark the queue as
empty; we needn't worry about updating the rear pointer, which will still point
to the deleted item, because \code{empty\-/queue?} looks only at the front
pointer.}

\begin{scheme}
(define (delete-queue! queue)
  (cond ((empty-queue? queue)
         (error "DELETE! called with an empty queue" queue))
        (else (set-front-ptr! queue (cdr (front-ptr queue)))
              queue)))
\end{scheme}

\begin{quote}
\heading{\phantomsection\label{Exercise 3.21}Exercise 3.21:} Ben Bitdiddle decides to test the
queue implementation described above.  He types in the procedures to the Lisp
interpreter and proceeds to try them out:

\begin{scheme}
(define q1 (make-queue))
(insert-queue! q1 'a)
~\textit{((a) a)}~
(insert-queue! q1 'b)
~\textit{((a b) b)}~
(delete-queue! q1)
~\textit{((b) b)}~
(delete-queue! q1)
~\textit{(() b)}~
\end{scheme}

``It's all wrong!'' he complains.  ``The interpreter's response shows that the
last item is inserted into the queue twice.  And when I delete both items, the
second \code{b} is still there, so the queue isn't empty, even though it's
supposed to be.''  Eva Lu Ator suggests that Ben has misunderstood what is
happening.  ``It's not that the items are going into the queue twice,'' she
explains.  ``It's just that the standard Lisp printer doesn't know how to make
sense of the queue representation.  If you want to see the queue printed
correctly, you'll have to define your own print procedure for queues.'' Explain
what Eva Lu is talking about.  In particular, show why Ben's examples produce
the printed results that they do.  Define a procedure \code{print\-/queue} that
takes a queue as input and prints the sequence of items in the queue.
\end{quote}

\begin{quote}
\heading{\phantomsection\label{Exercise 3.22}Exercise 3.22:} Instead of representing a queue
as a pair of pointers, we can build a queue as a procedure with local state.
The local state will consist of pointers to the beginning and the end of an
ordinary list.  Thus, the \code{make\-/queue} procedure will have the form

\begin{scheme}
(define (make-queue)
  (let ((front-ptr ~\( \dots \)~ )
        (rear-ptr ~\( \dots \)~ ))
    ~\( \dark \langle \)~~\var{\dark definitions of internal procedures}~~\( \dark \rangle \)~
    (define (dispatch m) ~\( \dots \)~)
    dispatch))
\end{scheme}

Complete the definition of \code{make\-/queue} and provide implementations of the
queue operations using this representation.
\end{quote}

\begin{quote}
\heading{\phantomsection\label{Exercise 3.23}Exercise 3.23:} A \newterm{deque} (``double-ended
queue'') is a sequence in which items can be inserted and deleted at either the
front or the rear.  Operations on deques are the constructor \code{make\-/deque},
the predicate \code{empty\-/deque?}, selectors \code{front\-/deque} and
\code{rear\-/deque}, mutators \code{front\-/insert\-/deque!},
\code{rear\-/insert\-/deque!}, \code{front\-/delete\-/deque!}, and
\code{rear\-/delete\-/deque!}.  Show how to represent deques using pairs, and give
implementations of the operations.\footnote{Be careful not to make the
interpreter try to print a structure that contains cycles.  (See \link{Exercise 3.13}.)}  
All operations should be accomplished in \( \Theta \)(1) steps.
\end{quote}

\subsection{Representing Tables}
\label{Section 3.3.3}

When we studied various ways of representing sets in \link{Chapter 2}, we
mentioned in \link{Section 2.3.3} the task of maintaining a table of records
indexed by identifying keys.  In the implementation of data-directed
programming in \link{Section 2.4.3}, we made extensive use of two-dimensional
tables, in which information is stored and retrieved using two keys.  Here we
see how to build tables as mutable list structures.

\begin{figure}[tb]
\phantomsection\label{Figure 3.22}
\centering
\begin{comment}
\heading{Figure 3.22:} A table represented as a headed list.

\begin{example}
 +---+---+    +---+---+    +---+---+    +---+---+
 | * | *-+--->| * | *-+--->| * | *-+--->| * | / |
 +-|-+---+    +-|-+---+    +-|-+---+    +-|-+---+
   |            |            |            |
   V            V            V            V
+---------+   +---+---+   +---+---+   +---+---+
| *table* |   | * | * |   | * | * |   | * | * |
+---------+   +-|-+-|-+   +-|-+-|-+   +-|-+-|-+
                |   |       |   |       |   |
                V   V       V   V       V   V
             +---+ +---+ +---+ +---+ +---+ +---+
             | a | | 1 | | b | | 2 | | c | | 3 |
             +---+ +---+ +---+ +---+ +---+ +---+
\end{example}
\end{comment}
\includegraphics[width=81mm]{fig/chap3/Fig3.22c.pdf}
\par\bigskip
\noindent
\heading{Figure 3.22:} A table represented as a headed list.
\end{figure}

We first consider a one-dimensional table, in which each value is stored under
a single key.  We implement the table as a list of records, each of which is
implemented as a pair consisting of a key and the associated value. The records
are glued together to form a list by pairs whose \code{car}s point to
successive records.  These gluing pairs are called the \newterm{backbone} of
the table.  In order to have a place that we can change when we add a new
record to the table, we build the table as a \newterm{headed list}.  A headed
list has a special backbone pair at the beginning, which holds a dummy
``record''---in this case the arbitrarily chosen symbol \code{*table*}.
\link{Figure 3.22} shows the box-and-pointer diagram for the table

\begin{scheme}
a:  1
b:  2
c:  3
\end{scheme}

\noindent
To extract information from a table we use the \code{lookup} procedure, which
takes a key as argument and returns the associated value (or false if there is
no value stored under that key).  \code{Lookup} is defined in terms of the
\code{assoc} operation, which expects a key and a list of records as arguments.
Note that \code{assoc} never sees the dummy record.  \code{Assoc} returns the
record that has the given key as its \code{car}.\footnote{Because \code{assoc}
uses \code{equal?}, it can recognize keys that are symbols, numbers, or list
structure.}  \code{Lookup} then checks to see that the resulting record
returned by \code{assoc} is not false, and returns the value (the \code{cdr})
of the record.

\begin{scheme}
(define (lookup key table)
  (let ((record (assoc key (cdr table))))
    (if record
        (cdr record)
        false)))
(define (assoc key records)
  (cond ((null? records) false)
        ((equal? key (caar records)) (car records))
        (else (assoc key (cdr records)))))
\end{scheme}

\noindent
To insert a value in a table under a specified key, we first use \code{assoc}
to see if there is already a record in the table with this key.  If not, we
form a new record by \code{cons}ing the key with the value, and insert this at
the head of the table's list of records, after the dummy record.  If there
already is a record with this key, we set the \code{cdr} of this record to the
designated new value.  The header of the table provides us with a fixed
location to modify in order to insert the new record.\footnote{Thus, the first
backbone pair is the object that represents the table ``itself\( \kern0.1em \)''; that is, a
pointer to the table is a pointer to this pair.  This same backbone pair always
starts the table.  If we did not arrange things in this way, \code{insert!}
would have to return a new value for the start of the table when it added a new
record.}

\begin{scheme}
(define (insert! key value table)
  (let ((record (assoc key (cdr table))))
    (if record
        (set-cdr! record value)
        (set-cdr! table
                  (cons (cons key value)
                        (cdr table)))))
  'ok)
\end{scheme}

\noindent
To construct a new table, we simply create a list containing the symbol
\code{*table*}:

\begin{scheme}
(define (make-table)
  (list '*table*))
\end{scheme}

\subsubsection*{Two-dimensional tables}

\noindent
In a two-dimensional table, each value is indexed by two keys.  We can
construct such a table as a one-dimensional table in which each key identifies
a subtable.  \link{Figure 3.23} shows the box-and-pointer diagram for the table

\begin{example}
math:    +:  43        letters:    a:  97
         -:  45                    b:  98
         *:  42
\end{example}

\noindent
which has two subtables.  (The subtables don't need a special header symbol,
since the key that identifies the subtable serves this purpose.)

When we look up an item, we use the first key to identify the correct subtable.
Then we use the second key to identify the record within the subtable.

\begin{scheme}
(define (lookup key-1 key-2 table)
  (let ((subtable 
         (assoc key-1 (cdr table))))
    (if subtable
        (let ((record 
               (assoc key-2 (cdr subtable))))
          (if record
              (cdr record)
              false))
        false)))
\end{scheme}

\begin{figure}[tb]
\phantomsection\label{Figure 3.23}
\centering
\begin{comment}
\heading{Figure 3.23:} A two-dimensional table.

\begin{example}
table
  |
  V
+---+---+   +---+---+   +---+---+
| * | *-+-->| * | *-+-->| * | / |
+-|-+---+   +-|-+---+   +-|-+---+
  V           |           V
+-------+     |         +---+---+   +---+---+   +---+---+
|*table*|     |         | * | *-+-->| * | *-+-->| * | / |
+-------+     |         +-|-+---+   +-|-+---+   +-|-+---+
              |           V           V           V
              |       +-------+     +---+---+   +---+---+
              |       |letters|     | * | * |   | * | * |
              |       +-------+     +-|-+-|-+   +-|-+-|-+
              |                       V   V       V   V
              |                    +---+ +---+ +---+ +---+
              |                    | a | | 97| | b | | 98|
              |                    +---+ +---+ +---+ +---+
              V
            +---+---+   +---+---+   +---+---+   +---+---+
            | * | *-+-->| * | *-+-->| * | *-+-->| * | / |
            +-|-+---+   +-|-+---+   +-|-+---+   +-|-+---+
              V           V           V           V
          +------+      +---+---+   +---+---+   +---+---+
          | math |      | * | * |   | * | * |   | * | * |
          +------+      +-|-+-|-+   +-|-+-|-+   +-|-+-|-+
                          V   V       V   V       V   V
                       +---+ +---+ +---+ +---+ +---+ +---+
                       | + | | 43| | - | | 45| | * | | 42|
                       +---+ +---+ +---+ +---+ +---+ +---+
\end{example}
\end{comment}
\includegraphics[width=103mm]{fig/chap3/Fig3.23a.pdf}
\par\bigskip
\noindent
\heading{Figure 3.23:} A two-dimensional table.
\end{figure}

To insert a new item under a pair of keys, we use \code{assoc} to see if there
is a subtable stored under the first key.  If not, we build a new subtable
containing the single record (\code{key\-/2}, \code{value}) and insert it into
the table under the first key.  If a subtable already exists for the first key,
we insert the new record into this subtable, using the insertion method for
one-dimensional tables described above:

\begin{scheme}
(define (insert! key-1 key-2 value table)
  (let ((subtable (assoc key-1 (cdr table))))
    (if subtable
        (let ((record (assoc key-2 (cdr subtable))))
          (if record
              (set-cdr! record value)
              (set-cdr! subtable
                        (cons (cons key-2 value)
                              (cdr subtable)))))
        (set-cdr! table
                  (cons (list key-1
                              (cons key-2 value))
                        (cdr table)))))
  'ok)
\end{scheme}

\subsubsection*{Creating local tables}

The \code{lookup} and \code{insert!} operations defined above take the table as
an argument.  This enables us to use programs that access more than one table.
Another way to deal with multiple tables is to have separate \code{lookup} and
\code{insert!} procedures for each table.  We can do this by representing a
table procedurally, as an object that maintains an internal table as part of
its local state.  When sent an appropriate message, this ``table object''
supplies the procedure with which to operate on the internal table.  Here is a
generator for two-dimensional tables represented in this fashion:

\begin{scheme}
(define (make-table)
  (let ((local-table (list '*table*)))
    (define (lookup key-1 key-2)
      (let ((subtable 
             (assoc key-1 (cdr local-table))))
        (if subtable
            (let ((record 
                   (assoc key-2 (cdr subtable))))
              (if record (cdr record) false))
            false)))
    (define (insert! key-1 key-2 value)
      (let ((subtable 
             (assoc key-1 (cdr local-table))))
        (if subtable
            (let ((record 
                   (assoc key-2 (cdr subtable))))
              (if record
                  (set-cdr! record value)
                  (set-cdr! subtable
                            (cons (cons key-2 value)
                                  (cdr subtable)))))
            (set-cdr! local-table
                      (cons (list key-1 (cons key-2 value))
                            (cdr local-table)))))
      'ok)
    (define (dispatch m)
      (cond ((eq? m 'lookup-proc) lookup)
            ((eq? m 'insert-proc!) insert!)
            (else (error "Unknown operation: TABLE" m))))
    dispatch))
\end{scheme}

\noindent
Using \code{make\-/table}, we could implement the \code{get} and \code{put}
operations used in \link{Section 2.4.3} for data-directed programming, as
follows:

\begin{scheme}
(define operation-table (make-table))
(define get (operation-table 'lookup-proc))
(define put (operation-table 'insert-proc!))
\end{scheme}

\noindent
\code{Get} takes as arguments two keys, and \code{put} takes as arguments two
keys and a value.  Both operations access the same local table, which is
encapsulated within the object created by the call to \code{make\-/table}.

\begin{quote}
\heading{\phantomsection\label{Exercise 3.24}Exercise 3.24:} In the table implementations
above, the keys are tested for equality using \code{equal?} (called by
\code{assoc}).  This is not always the appropriate test.  For instance, we
might have a table with numeric keys in which we don't need an exact match to
the number we're looking up, but only a number within some tolerance of it.
Design a table constructor \code{make\-/table} that takes as an argument a
\code{same\-/key?} procedure that will be used to test ``equality'' of keys.
\code{Make\-/table} should return a \code{dispatch} procedure that can be used to
access appropriate \code{lookup} and \code{insert!} procedures for a local
table.
\end{quote}

\begin{quote}
\heading{\phantomsection\label{Exercise 3.25}Exercise 3.25:} Generalizing one- and
two-dimensional tables, show how to implement a table in which values are
stored under an arbitrary number of keys and different values may be stored
under different numbers of keys.  The \code{lookup} and \code{insert!}
proce- dures should take as input a list of keys used to access the table.
\end{quote}

\begin{quote}
\heading{\phantomsection\label{Exercise 3.26}Exercise 3.26:} To search a table as implemented
above, one needs to scan through the list of records.  This is basically the
unordered list representation of \link{Section 2.3.3}.  For large tables, it may
be more efficient to structure the table in a different manner.  Describe a
table implementation where the (key, value) records are organized using a
binary tree, assuming that keys can be ordered in some way (e.g., numerically
or alphabetically).  (Compare \link{Exercise 2.66} of \link{Chapter 2}.)
\end{quote}

\begin{quote}
\heading{\phantomsection\label{Exercise 3.27}Exercise 3.27:} \newterm{Memoization} (also
called \newterm{tabulation}) is a technique that enables a procedure to record,
in a local table, values that have previously been computed.  This technique
can make a vast difference in the performance of a program.  A memoized
procedure maintains a table in which values of previous calls are stored using
as keys the arguments that produced the values.  When the memoized procedure is
asked to compute a value, it first checks the table to see if the value is
already there and, if so, just returns that value.  Otherwise, it computes the
new value in the ordinary way and stores this in the table.  As an example of
memoization, recall from \link{Section 1.2.2} the exponential process for
computing Fibonacci numbers:

\begin{scheme}
(define (fib n)
  (cond ((= n 0) 0)
        ((= n 1) 1)
        (else (+ (fib (- n 1)) (fib (- n 2))))))
\end{scheme}

The memoized version of the same procedure is

\begin{scheme}
(define memo-fib
  (memoize
   (lambda (n)
     (cond ((= n 0) 0)
           ((= n 1) 1)
           (else (+ (memo-fib (- n 1))
                    (memo-fib (- n 2))))))))
\end{scheme}

\noindent
where the memoizer is defined as

\begin{scheme}
(define (memoize f)
  (let ((table (make-table)))
    (lambda (x)
      (let ((previously-computed-result 
             (lookup x table)))
        (or previously-computed-result
            (let ((result (f x)))
              (insert! x result table)
              result))))))
\end{scheme}

Draw an environment diagram to analyze the computation of \code{(memo\-/fib 3)}.
Explain why \code{memo\-/fib} computes the \( n^{\mathrm{th}} \) Fibonacci number in a number
of steps proportional to \( n \).  Would the scheme still work if we had simply
defined \code{memo\-/fib} to be \code{(memoize fib)}?
\end{quote}

\subsection{A Simulator for Digital Circuits}
\label{Section 3.3.4}

Designing complex digital systems, such as computers, is an important
engineering activity.  Digital systems are constructed by interconnecting
simple elements.  Although the behavior of these individual elements is simple,
networks of them can have very complex behavior.  Computer simulation of
proposed circuit designs is an important tool used by digital systems
engineers.  In this section we design a system for performing digital logic
simulations.  This system typifies a kind of program called an
\newterm{event-driven simulation}, in which actions (``events'') trigger
further events that happen at a later time, which in turn trigger more events,
and so on.

Our computational model of a circuit will be composed of objects that
correspond to the elementary components from which the circuit is constructed.
There are \newterm{wires}, which carry \newterm{digital signals}.  A digital
signal may at any moment have only one of two possible values, 0 and 1.  There
are also various types of digital \newterm{function boxes}, which connect wires
carrying input signals to other output wires.  Such boxes produce output
signals computed from their input signals.  The output signal is delayed by a
time that depends on the type of the function box.  For example, an
\newterm{inverter} is a primitive function box that inverts its input.  If the
input signal to an inverter changes to 0, then one inverter-delay later the
inverter will change its output signal to 1.  If the input signal to an
inverter changes to 1, then one inverter-delay later the inverter will change
its output signal to 0.  We draw an inverter symbolically as in \link{Figure 3.24}.  
An \newterm{and-gate}, also shown in \link{Figure 3.24}, is a primitive
function box with two inputs and one output.  It drives its output signal to a
value that is the \newterm{logical and} of the inputs.  That is, if both of its
input signals become 1, then one and-gate-delay time later the and-gate will
force its output signal to be 1; otherwise the output will be 0.  An
\newterm{or-gate} is a similar two-input primitive function box that drives its
output signal to a value that is the \newterm{logical or} of the inputs.  That
is, the output will become 1 if at least one of the input signals is 1;
otherwise the output will become 0.

\begin{figure}[tb]
\phantomsection\label{Figure 3.24}
\centering
\begin{comment}
\heading{Figure 3.24:} Primitive functions in the digital logic simulator.

\begin{example}
               __          ___
  |\        --|  \       --\  \
--| >o--      |   )--       )  >--
  |/        --|__/       --/__/

Inverter    And-gate     Or-gate
\end{example}
\end{comment}
\includegraphics[width=74mm]{fig/chap3/Fig3.24b.pdf}
\par\bigskip
\noindent
\heading{Figure 3.24:} Primitive functions in the digital logic simulator. 
\end{figure}

We can connect primitive functions together to construct more complex
functions.  To accomplish this we wire the outputs of some function boxes to
the inputs of other function boxes.  For example, the \newterm{half-adder}
circuit shown in \link{Figure 3.25} consists of an or-gate, two and-gates, and
an inverter.  It takes two input signals, A and B, and has two output signals,
S and C.  S will become 1 whenever precisely one of A and B is 1, and C will
become 1 whenever A and B are both 1.  We can see from the figure that, because
of the delays involved, the outputs may be generated at different times.  Many
of the difficulties in the design of digital circuits arise from this fact.

\begin{figure}[tb]
\phantomsection\label{Figure 3.25}
\centering
\begin{comment}
\heading{Figure 3.25:} A half-adder circuit.

\begin{example}
    +--------------------------------------+
    |         ____                         |
A --------*---\   \ D               ___    |
    |     |    >   >---------------|   \   |
    |  +--|---/___/                |    )----- S
    |  |  |              |\  E  +--|___/   |
    |  |  |           +--| >o---+          |
    |  |  |    ___    |  |/                |
    |  |  +---|   \   |                    |
    |  |      |    )--*----------------------- C
B -----*------|___/                        |
    |                                      |
    +--------------------------------------+
\end{example}
\end{comment}
\includegraphics[width=72mm]{fig/chap3/Fig3.25c.pdf}
\par\bigskip
\noindent
\heading{Figure 3.25:} A half-adder circuit.
\end{figure}

We will now build a program for modeling the digital logic circuits we wish to
study.  The program will construct computational objects modeling the wires,
which will ``hold'' the signals.  Function boxes will be modeled by procedures
that enforce the correct relationships among the signals.

One basic element of our simulation will be a procedure \code{make\-/wire}, which
constructs wires.  For example, we can construct six wires as follows:

\begin{scheme}
(define a (make-wire))
(define b (make-wire))
(define c (make-wire))
(define d (make-wire))
(define e (make-wire))
(define s (make-wire))
\end{scheme}

\noindent
We attach a function box to a set of wires by calling a procedure that
constructs that kind of box.  The arguments to the constructor procedure are
the wires to be attached to the box.  For example, given that we can construct
and-gates, or-gates, and inverters, we can wire together the half-adder shown
in \link{Figure 3.25}:

\begin{scheme}
(or-gate a b d)
~\textit{ok}~
(and-gate a b c)
~\textit{ok}~
(inverter c e)
~\textit{ok}~
(and-gate d e s)
~\textit{ok}~
\end{scheme}

\noindent
Better yet, we can explicitly name this operation by defining a procedure
\code{half\-/adder} that constructs this circuit, given the four external wires
to be attached to the half-adder:

\begin{scheme}
(define (half-adder a b s c)
  (let ((d (make-wire)) (e (make-wire)))
    (or-gate a b d)
    (and-gate a b c)
    (inverter c e)
    (and-gate d e s)
    'ok))
\end{scheme}

\noindent
The advantage of making this definition is that we can use \code{half\-/adder}
itself as a building block in creating more complex circuits.  \link{Figure 3.26}, 
for example, shows a \newterm{full-adder} composed of two half-adders
and an or-gate.\footnote{A full-adder is a basic circuit element used in adding
two binary numbers.  Here A and B are the bits at corresponding positions in
the two numbers to be added, and \( \rm C_{in} \) is the carry bit from the
addition one place to the right.  The circuit generates SUM, which is the sum
bit in the corresponding position, and \( \rm C_{out} \), which is the carry
bit to be propagated to the left.} We can construct a full-adder as follows:

\begin{scheme}
(define (full-adder a b c-in sum c-out)
  (let ((s (make-wire)) (c1 (make-wire)) (c2 (make-wire)))
    (half-adder b c-in s c1)
    (half-adder a s sum c2)
    (or-gate c1 c2 c-out)
    'ok))
\end{scheme}

\begin{figure}[tb]
\phantomsection\label{Figure 3.26}
\centering
\begin{comment}
\heading{Figure 3.26:} A full-adder circuit.

\begin{example}
    +----------------------------------+
    |              +-------+           |
A -----------------+ half- +-------------- SUM
    |  +-------+   | adder |   ____    |
B -----+ half- +---+       +---\   \   |
    |  | adder |   +-------+    >or >----- Cout
C -----+       +---------------/___/   |
    |  +-------+                       |
    +----------------------------------+
\end{example}
\end{comment}
\includegraphics[width=74mm]{fig/chap3/Fig3.26a.pdf}
\par\bigskip
\noindent
\heading{Figure 3.26:} A full-adder circuit.
\end{figure}

\noindent
Having defined \code{full\-/adder} as a procedure, we can now use it as a
building block for creating still more complex circuits.  (For example, see
\link{Exercise 3.30}.)

In essence, our simulator provides us with the tools to construct a language of
circuits.  If we adopt the general perspective on languages with which we
approached the study of Lisp in \link{Section 1.1}, we can say that the
primitive function boxes form the primitive elements of the language, that
wiring boxes together provides a means of combination, and that specifying
wiring patterns as procedures serves as a means of abstraction.

\subsubsection*{Primitive function boxes}

The primitive function boxes implement the ``forces'' by which a change in the
signal on one wire influences the signals on other wires.  To build function
boxes, we use the following operations on wires:

\begin{itemize}

\item \code{(get\-/signal}\( \;\;\langle\kern0.06em\hbox{\ttfamily\slshape wire}\kern0.08em\rangle \)\code{)}

\noindent
returns the current value of the signal on the wire.

\item \code{(set\-/signal!}\( \;\;\langle\kern0.08em\hbox{\ttfamily\slshape wire}\kern0.08em\rangle\;\;\langle\kern0.08em\hbox{\ttfamily\slshape new value}\kern0.08em\rangle \)\code{)}

\noindent
changes the value of the signal on the wire to the new value.

\item \code{(add\-/action!}\( \;\;\langle\kern0.08em\hbox{\ttfamily\slshape wire}\kern0.08em\rangle\;\;\langle\kern0.08em\hbox{\ttfamily\slshape procedure of no arguments}\kern0.02em\rangle \)\code{)}

\noindent
asserts that the designated procedure should be run whenever the signal on the
wire changes value.  Such procedures are the vehicles by which changes in the
signal value on the wire are communicated to other wires.

\end{itemize}

\noindent
In addition, we will make use of a procedure \code{after\-/delay} that takes a
time delay and a procedure to be run and executes the given procedure after the
given delay.

Using these procedures, we can define the primitive digital logic functions.
To connect an input to an output through an inverter, we use \code{add\-/action!}
to associate with the input wire a procedure that will be run whenever the
signal on the input wire changes value.  The procedure computes the
\code{logical\-/not} of the input signal, and then, after one
\code{inverter\-/delay}, sets the output signal to be this new value:

\begin{scheme}
(define (inverter input output)
  (define (invert-input)
    (let ((new-value (logical-not (get-signal input))))
      (after-delay inverter-delay
                   (lambda () (set-signal! output new-value)))))
  (add-action! input invert-input) 'ok)
(define (logical-not s)
  (cond ((= s 0) 1)
        ((= s 1) 0)
        (else (error "Invalid signal" s))))
\end{scheme}

\noindent
An and-gate is a little more complex.  The action procedure must be run if
either of the inputs to the gate changes.  It computes the \code{logical\-/and}
(using a procedure analogous to \code{logical\-/not}) of the values of the
signals on the input wires and sets up a change to the new value to occur on
the output wire after one \code{and\-/gate\-/delay}.

\begin{scheme}
(define (and-gate a1 a2 output)
  (define (and-action-procedure)
    (let ((new-value
           (logical-and (get-signal a1) (get-signal a2))))
      (after-delay 
       and-gate-delay
       (lambda () (set-signal! output new-value)))))
  (add-action! a1 and-action-procedure)
  (add-action! a2 and-action-procedure)
  'ok)
\end{scheme}

\begin{quote}
\heading{\phantomsection\label{Exercise 3.28}Exercise 3.28:} Define an or-gate as a primitive
function box.  Your \code{or\-/gate} constructor should be similar to
\code{and\-/gate}.
\end{quote}

\begin{quote}
\heading{\phantomsection\label{Exercise 3.29}Exercise 3.29:} Another way to construct an
or-gate is as a compound digital logic device, built from and-gates and
inverters.  Define a procedure \code{or\-/gate} that accomplishes this.  What is
the delay time of the or-gate in terms of \code{and\-/gate\-/delay} and
\code{inverter\-/delay}?
\end{quote}

\begin{quote}
\heading{\phantomsection\label{Exercise 3.30}Exercise 3.30:} \link{Figure 3.27} shows a
\newterm{ripple-carry adder} formed by stringing together \( n \) full-adders.
This is the simplest form of parallel adder for adding two \( n \)-bit binary
numbers.  The inputs \( A_1 \), \( A_2 \), \( A_3 \), \( \dots \), \( A_n \) and 
\( B_1 \), \( B_2 \), \( B_3 \),
\( \dots \), \( B_n \) are the two binary numbers to be added (each \( A_k \) and
\( B_k \) is a 0 or a 1).  The circuit generates \( S_1 \), \( S_2 \), 
\( S_3 \), \( \dots \), \( S_n \),
the \( n \) bits of the sum, and \( C \), the carry from the addition.  Write a
procedure \code{ripple\-/carry\-/adder} that generates this circuit.  The procedure
should take as arguments three lists of \( n \) wires each---the \( A_k \), the
\( B_k \), and the \( S_k \)---and also another wire \( C \).  The major drawback of the
ripple-carry adder is the need to wait for the carry signals to propagate.
What is the delay needed to obtain the complete output from an \( n \)-bit
ripple-carry adder, expressed in terms of the delays for and-gates, or-gates,
and inverters?
\end{quote}

\begin{figure}[tb]
\phantomsection\label{Figure 3.27}
\centering
\begin{comment}
\heading{Figure 3.27:} A ripple-carry adder for \( n \)-bit numbers.

\begin{example}
   :                                              :   :
   : A_1 B_1   C_1   A_2 B_2   C_2   A_3 B_3   C_3:   : A_n B_n C_n=0
   :  |   |   +---+   |   |   +---+   |   |   +-----  :  |   |   +-
   |  |   |   |   |   |   |   |   |   |   |   |   :   :  |   |   | 
   : ++---+---++  |  ++---+---++  |  ++---+---++  :   : ++---+---++
   : |   FA    |  |  |   FA    |  |  |   FA    |  :   : |   FA    |
   : +--+---+--+  |  +--+---+--+  |  +--+---+--+  :   : +--+---+--+
   :    |   |     |     |   |     |     |   |     :   :    |   |   
C ------+   |     +-----+   |     +-----+   |     :  ------+   |   
   :        |       C_1     |       C_2     |     :   :C_(n-1) |   
   :        |               |               |     :   :        |   
           S_1             S_2             S_3                S_n
\end{example}
\end{comment}
\includegraphics[width=96mm]{fig/chap3/Fig3.27a.pdf}
\par\bigskip
\noindent
\heading{Figure 3.27:} A ripple-carry adder for \( n \)-bit numbers.
\end{figure}

\subsubsection*{Representing wires}

A wire in our simulation will be a computational object with two local state
variables: a \code{signal\-/value} (initially taken to be 0) and a collection of
\code{action\-/procedures} to be run when the signal changes value.  We implement
the wire, using message-passing style, as a collection of local procedures
together with a \code{dispatch} procedure that selects the appropriate local
operation, just as we did with the simple bank-account object in 
\link{Section 3.1.1}:

\begin{scheme}
(define (make-wire)
  (let ((signal-value 0) (action-procedures '()))
    (define (set-my-signal! new-value)
      (if (not (= signal-value new-value))
          (begin (set! signal-value new-value)
                 (call-each action-procedures))
          'done))
    (define (accept-action-procedure! proc)
      (set! action-procedures 
            (cons proc action-procedures))
      (proc))
    (define (dispatch m)
      (cond ((eq? m 'get-signal) signal-value)
            ((eq? m 'set-signal!) set-my-signal!)
            ((eq? m 'add-action!) accept-action-procedure!)
            (else (error "Unknown operation: WIRE" m))))
    dispatch))
\end{scheme}

\noindent
The local procedure \code{set\-/my\-/signal!} tests whether the new signal value
changes the signal on the wire.  If so, it runs each of the action procedures,
using the following procedure \code{call\-/each}, which calls each of the items
in a list of no-argument procedures:

\begin{scheme}
(define (call-each procedures)
  (if (null? procedures)
      'done
      (begin ((car procedures))
             (call-each (cdr procedures)))))
\end{scheme}

\noindent
The local procedure \code{accept\-/action\-/procedure!} adds the given procedure to
the list of procedures to be run, and then runs the new procedure once.  (See
\link{Exercise 3.31}.)

With the local \code{dispatch} procedure set up as specified, we can provide
the following procedures to access the local operations on
wires:\footnote{\label{Footnote 27} These procedures are simply
syntactic sugar that allow us to use ordinary procedural syntax to access the
local procedures of objects.  It is striking that we can interchange the role
of ``procedures'' and ``data'' in such a simple way.  For example, if we write
\code{(wire 'get\-/signal)} we think of \code{wire} as a procedure that is called
with the message \code{get\-/signal} as input.  Alternatively, writing
\code{(get\-/signal wire)} encourages us to think of \code{wire} as a data object
that is the input to a procedure \code{get\-/signal}.  The truth of the matter is
that, in a language in which we can deal with procedures as objects, there is
no fundamental difference between ``procedures'' and ``data,'' and we can
choose our syntactic sugar to allow us to program in whatever style we choose.}

\begin{scheme}
(define (get-signal wire) (wire 'get-signal))
(define (set-signal! wire new-value)
  ((wire 'set-signal!) new-value))
(define (add-action! wire action-procedure)
  ((wire 'add-action!) action-procedure))
\end{scheme}

\noindent
Wires, which have time-varying signals and may be incrementally attached to
devices, are typical of mutable objects.  We have modeled them as procedures
with local state variables that are modified by assignment.  When a new wire is
created, a new set of state variables is allocated (by the \code{let}
expression in \code{make\-/wire}) and a new \code{dispatch} procedure is
constructed and returned, capturing the environment with the new state
variables.

The wires are shared among the various devices that have been connected to
them.  Thus, a change made by an interaction with one device will affect all
the other devices attached to the wire.  The wire communicates the change to
its neighbors by calling the action procedures provided to it when the
connections were established.

\subsubsection*{The agenda}

The only thing needed to complete the simulator is \code{after\-/delay}.  The
idea here is that we maintain a data structure, called an \newterm{agenda},
that contains a schedule of things to do.  The following operations are defined
for agendas:

\begin{itemize}

\item
\code{(make\-/agenda)} returns a new empty agenda.

\item
\code{(empty\-/agenda?}\( \;\;\langle\kern0.08em\hbox{\ttfamily\slshape agenda}\kern0.06em\rangle\hbox{\tt)} \) is true if the specified agenda is
empty.

\item
\code{(first\-/agenda\-/item}\( \;\;\langle\kern0.08em\hbox{\ttfamily\slshape agenda}\kern0.06em\rangle\hbox{\tt)} \) returns the first item on the
agenda.

\item
\code{(remove\-/first\-/agenda\-/item!}\( \;\langle\kern0.08em\hbox{\ttfamily\slshape agenda}\kern0.06em\rangle\hbox{\tt)} \) modifies the agenda by
removing the first item.

\item
\code{(add\-/to\-/agenda!}\( \;\;\langle\kern0.03em\hbox{\ttfamily\slshape time}\kern0.06em\rangle\;\;\langle\kern0.08em\hbox{\ttfamily\slshape action}\kern0.06em\rangle\;\;\langle\kern0.08em\hbox{\ttfamily\slshape agenda}\kern0.06em\rangle\hbox{\tt)} \) modifies the agenda by adding the given action procedure to be run at the specified time.

\item
\code{(current\-/time}\( \;\;\langle\kern0.08em\hbox{\ttfamily\slshape agenda}\kern0.04em\rangle\hbox{\tt)} \) returns the current simulation time.

\end{itemize}

\noindent
The particular agenda that we use is denoted by \code{the\-/agenda}.  The
procedure \code{after\-/delay} adds new elements to \code{the\-/agenda}:

\begin{scheme}
(define (after-delay delay action)
  (add-to-agenda! (+ delay (current-time the-agenda))
                  action
                  the-agenda))
\end{scheme}

\noindent
The simulation is driven by the procedure \code{propagate}, which operates on
\code{the\-/agenda}, executing each procedure on the agenda in sequence.  In
general, as the simulation runs, new items will be added to the agenda, and
\code{propagate} will continue the simulation as long as there are items on the
agenda:

\begin{scheme}
(define (propagate)
  (if (empty-agenda? the-agenda)
      'done
      (let ((first-item (first-agenda-item the-agenda)))
        (first-item)
        (remove-first-agenda-item! the-agenda)
        (propagate))))
\end{scheme}

\subsubsection*{A sample simulation}

The following procedure, which places a ``probe'' on a wire, shows the
simulator in action.  The probe tells the wire that, whenever its signal
changes value, it should print the new signal value, together with the current
time and a name that identifies the wire:

\begin{scheme}
(define (probe name wire)
  (add-action! wire
               (lambda ()
                 (newline)
                 (display name) (display " ")
                 (display (current-time the-agenda))
                 (display "  New-value = ")
                 (display (get-signal wire)))))
\end{scheme}

\noindent
We begin by initializing the agenda and specifying delays for the primitive
function boxes:

\begin{scheme}
(define the-agenda (make-agenda))
(define inverter-delay 2)
(define and-gate-delay 3)
(define or-gate-delay 5)
\end{scheme}

\noindent
Now we define four wires, placing probes on two of them:

\begin{scheme}
(define input-1 (make-wire))
(define input-2 (make-wire))
(define sum (make-wire))
(define carry (make-wire))

(probe 'sum sum)
~\textit{sum 0  New-value = 0}~

(probe 'carry carry)
~\textit{carry 0  New-value = 0}~
\end{scheme}

\noindent
Next we connect the wires in a half-adder circuit (as in \link{Figure 3.25}),
set the signal on \code{input\-/1} to 1, and run the simulation:

\begin{scheme}
(half-adder input-1 input-2 sum carry)
~\textit{ok}~
\end{scheme}

\begin{scheme}
(set-signal! input-1 1)
~\textit{done}~
\end{scheme}

\begin{scheme}
(propagate)
~\textit{sum 8  New-value = 1}~
~\textit{done}~
\end{scheme}

\noindent
The \code{sum} signal changes to 1 at time 8.  We are now eight time units from
the beginning of the simulation.  At this point, we can set the signal on
\code{input\-/2} to 1 and allow the values to propagate:

\begin{scheme}
(set-signal! input-2 1)
~\textit{done}~
\end{scheme}

\begin{scheme}
(propagate)
~\textit{carry 11  New-value = 1}~
~\textit{sum 16  New-value = 0}~
~\textit{done}~
\end{scheme}

\noindent
The \code{carry} changes to 1 at time 11 and the \code{sum} changes to 0 at
time 16.

\begin{quote}
\heading{\phantomsection\label{Exercise 3.31}Exercise 3.31:} The internal procedure
\code{accept\-/action\-/procedure!} defined in \code{make\-/wire} specifies that when
a new action procedure is added to a wire, the procedure is immediately run.
Explain why this initialization is necessary.  In particular, trace through the
half-adder example in the paragraphs above and say how the system's response
would differ if we had defined \code{accept\-/action\-/procedure!} as

\begin{scheme}
(define (accept-action-procedure! proc)
  (set! action-procedures 
        (cons proc action-procedures)))
\end{scheme}
\end{quote}

\subsubsection*{Implementing the agenda}

Finally, we give details of the agenda data structure, which holds the
procedures that are scheduled for future execution.

The agenda is made up of \newterm{time segments}.  Each time segment is a pair
consisting of a number (the time) and a queue (see \link{Exercise 3.32}) that
holds the procedures that are scheduled to be run during that time segment.

\begin{scheme}
(define (make-time-segment time queue)
  (cons time queue))
(define (segment-time s) (car s))
(define (segment-queue s) (cdr s))
\end{scheme}

\noindent
We will operate on the time-segment queues using the queue operations described
in \link{Section 3.3.2}.

The agenda itself is a one-dimensional table of time segments.  It differs from
the tables described in \link{Section 3.3.3} in that the segments will be sorted
in order of increasing time.  In addition, we store the \newterm{current time}
(i.e., the time of the last action that was processed) at the head of the
agenda.  A newly constructed agenda has no time segments and has a current time
of 0:\footnote{The agenda is a headed list, like the tables in 
\link{Section 3.3.3}, but since the list is headed by the time, we do not need an
additional dummy header (such as the \code{*table*} symbol used with tables).}

\begin{scheme}
(define (make-agenda) (list 0))
(define (current-time agenda) (car agenda))
(define (set-current-time! agenda time)
  (set-car! agenda time))
(define (segments agenda) (cdr agenda))
(define (set-segments! agenda segments)
  (set-cdr! agenda segments))
(define (first-segment agenda) (car (segments agenda)))
(define (rest-segments agenda) (cdr (segments agenda)))
\end{scheme}

\noindent
An agenda is empty if it has no time segments:

\begin{scheme}
(define (empty-agenda? agenda)
  (null? (segments agenda)))
\end{scheme}

\noindent
To add an action to an agenda, we first check if the agenda is empty.  If so,
we create a time segment for the action and install this in the agenda.
Otherwise, we scan the agenda, examining the time of each segment.  If we find
a segment for our appointed time, we add the action to the associated queue.
If we reach a time later than the one to which we are appointed, we insert a
new time segment into the agenda just before it.  If we reach the end of the
agenda, we must create a new time segment at the end.

\begin{scheme}
(define (add-to-agenda! time action agenda)
  (define (belongs-before? segments)
    (or (null? segments)
        (< time (segment-time (car segments)))))
  (define (make-new-time-segment time action)
    (let ((q (make-queue)))
      (insert-queue! q action)
      (make-time-segment time q)))
  (define (add-to-segments! segments)
    (if (= (segment-time (car segments)) time)
        (insert-queue! (segment-queue (car segments))
                       action)
        (let ((rest (cdr segments)))
          (if (belongs-before? rest)
              (set-cdr!
               segments
               (cons (make-new-time-segment time action)
                     (cdr segments)))
              (add-to-segments! rest)))))
  (let ((segments (segments agenda)))
    (if (belongs-before? segments)
        (set-segments!
         agenda
         (cons (make-new-time-segment time action)
               segments))
        (add-to-segments! segments))))
\end{scheme}

\noindent
The procedure that removes the first item from the agenda deletes the item at
the front of the queue in the first time segment.  If this deletion makes the
time segment empty, we remove it from the list of segments:\footnote{Observe
that the \code{if} expression in this procedure has no \( \langle \)\var{alternative}\( \kern0.08em\rangle \)
expression.  Such a ``one-armed \code{if} statement'' is used to decide whether
to do something, rather than to select between two expressions.  An \code{if}
expression returns an unspecified value if the predicate is false and there is
no \( \langle \)\var{alternative}\( \kern0.08em\rangle \).}

\begin{scheme}
(define (remove-first-agenda-item! agenda)
  (let ((q (segment-queue (first-segment agenda))))
    (delete-queue! q)
    (if (empty-queue? q)
        (set-segments! agenda (rest-segments agenda)))))
\end{scheme}

\noindent
The first agenda item is found at the head of the queue in the first time
segment.  Whenever we extract an item, we also update the current
time:\footnote{In this way, the current time will always be the time of the
action most recently processed.  Storing this time at the head of the agenda
ensures that it will still be available even if the associated time segment has
been deleted.}

\begin{scheme}
(define (first-agenda-item agenda)
  (if (empty-agenda? agenda)
      (error "Agenda is empty: FIRST-AGENDA-ITEM")
      (let ((first-seg (first-segment agenda)))
        (set-current-time! agenda 
                           (segment-time first-seg))
        (front-queue (segment-queue first-seg)))))
\end{scheme}

\begin{quote}
\heading{\phantomsection\label{Exercise 3.32}Exercise 3.32:} The procedures to be run during
each time segment of the agenda are kept in a queue.  Thus, the procedures for
each segment are called in the order in which they were added to the agenda
(first in, first out).  Explain why this order must be used.  In particular,
trace the behavior of an and-gate whose inputs change from 0, 1 to 1, 0 in the
same segment and say how the behavior would differ if we stored a segment's
procedures in an ordinary list, adding and removing procedures only at the
front (last in, first out).
\end{quote}


\subsection{Propagation of Constraints}
\label{Section 3.3.5}

Computer programs are traditionally organized as one-directional computations,
which perform operations on prespecified arguments to produce desired outputs.
On the other hand, we often model systems in terms of relations among
quantities.  For example, a mathematical model of a mechanical structure might
include the information that the deflection \( d \) of a metal rod is related to
the force \( F \) on the rod, the length \( L \) of the rod, the cross-sectional
area \( A \), and the elastic modulus \( E \) via the equation
\begin{comment}

\begin{example}
dAE = FL
\end{example}

\end{comment}
\begin{displaymath}
 dAE = FL. 
\end{displaymath}
Such an equation is not one-directional.  Given any four of the quantities, we
can use it to compute the fifth.  Yet translating the equation into a
traditional computer language would force us to choose one of the quantities to
be computed in terms of the other four.  Thus, a procedure for computing the
area \( A \) could not be used to compute the deflection \( d \), even though the
computations of \( A \) and \( d \) arise from the same
equation.\footnote{Constraint propagation first appeared in the incredibly
forward-looking \acronym{SKETCHPAD} system of Ivan \link{Sutherland (1963)}.  A
beautiful constraint-propagation system based on the Smalltalk language was
developed by Alan \link{Borning (1977)} at Xerox Palo Alto Research Center.  Sussman,
Stallman, and Steele applied constraint propagation to electrical circuit
analysis (\link{Sussman and Stallman 1975}; \link{Sussman and Steele 1980}). TK!Solver
(\link{Konopasek and Jayaraman 1984}) is an extensive modeling environment based on
constraints.}

In this section, we sketch the design of a language that enables us to work in
terms of relations themselves.  The primitive elements of the language are
\newterm{primitive constraints}, which state that certain relations hold
between quantities.  For example, \code{(adder a b c)} specifies that the
quantities \( a \), \( b \), and \( c \) must be related by the equation 
\( a + b = c \), \code{(multiplier x y z)} expresses the constraint 
\( xy = z \), and \code{(constant 3.14 x)} says that the value of \( x \) must be 3.14.

Our language provides a means of combining primitive constraints in order to
express more complex relations.  We combine constraints by constructing
\newterm{constraint networks}, in which constraints are joined by
\newterm{connectors}.  A connector is an object that ``holds'' a value that may
participate in one or more constraints.  For example, we know that the
relationship between Fahrenheit and Celsius temperatures is
\begin{comment}

\begin{example}
9C = 5(F - 32)
\end{example}

\end{comment}
\begin{displaymath}
 9C = 5(F - 32). 
\end{displaymath}
Such a constraint can be thought of as a network consisting of primitive adder,
multiplier, and constant constraints (\link{Figure 3.28}).  In the figure, we
see on the left a multiplier box with three terminals, labeled \( m \)1, \( m \)2,
and \( p \).  These connect the multiplier to the rest of the network as follows:
The \( m \)1 terminal is linked to a connector \( C \), which will hold the Celsius
temperature.  The \( m \)2 terminal is linked to a connector \( w \), which is also
linked to a constant box that holds 9.  The \( p \) terminal, which the
multiplier box constrains to be the product of \( m \)1 and \( m \)2, is linked to
the \( p \) terminal of another multiplier box, whose \( m \)2 is connected to a
constant 5 and whose \( m \)1 is connected to one of the terms in a sum.

\begin{figure}[tb]
\phantomsection\label{Figure 3.28}
\centering
\begin{comment}
\begin{quote}
\heading{Figure 3.28:} The relation \( 9C = 5(F - 32) \) expressed as a constraint network.

\begin{example}
       +---------+     +---------+   v   +---------+
C -----+ m1      |  u  |      m1 +-------+ a1      |
       |    *  p +-----+ p  *    |       |    *  s +---- F
    +--+ m2      |     |      m2 +--+ +--+ a2      |
    |  +---------+     +---------+  | |  +---------+
  w |                              x| |y
    |    +-----+        +-----+     | |     +-----+
    +----+  9  |        |  5  +-----+ +-----+  32 |
         +-----+        +-----+             +-----+
\end{example}
\end{quote}
\end{comment}
\includegraphics[width=87mm]{fig/chap3/Fig3.28.pdf}
\begin{quote}
\heading{Figure 3.28:} The relation \( 9C = 5(F - 32) \) expressed as a constraint network.
\end{quote}
\end{figure}

Computation by such a network proceeds as follows: When a connector is given a
value (by the user or by a constraint box to which it is linked), it awakens
all of its associated constraints (except for the constraint that just awakened
it) to inform them that it has a value.  Each awakened constraint box then
polls its connectors to see if there is enough information to determine a value
for a connector.  If so, the box sets that connector, which then awakens all of
its associated constraints, and so on.  For instance, in conversion between
Celsius and Fahrenheit, \( w \), \( x \), and \( y \) are immediately set by the
constant boxes to 9, 5, and 32, respectively.  The connectors awaken the
multipliers and the adder, which determine that there is not enough information
to proceed.  If the user (or some other part of the network) sets \( C \) to a
value (say 25), the leftmost multiplier will be awakened, and it will set \( u \)
to \( 25 \cdot 9 = 225 \).  Then \( u \) awakens the second multiplier, which sets \( v \) to
45, and \( v \) awakens the adder, which sets \( f \) to 77.

\subsubsection*{Using the constraint system}

To use the constraint system to carry out the temperature computation outlined
above, we first create two connectors, \code{C} and \code{F}, by calling the
constructor \code{make\-/connector}, and link \code{C} and \code{F} in an
appropriate network:

\begin{scheme}
(define C (make-connector))
(define F (make-connector))
(celsius-fahrenheit-converter C F)
~\textit{ok}~
\end{scheme}

\noindent
The procedure that creates the network is defined as follows:

\begin{scheme}
(define (celsius-fahrenheit-converter c f)
  (let ((u (make-connector))
        (v (make-connector))
        (w (make-connector))
        (x (make-connector))
        (y (make-connector)))
    (multiplier c w u)
    (multiplier v x u)
    (adder v y f)
    (constant 9 w)
    (constant 5 x)
    (constant 32 y)
    'ok))
\end{scheme}

\noindent
This procedure creates the internal connectors \code{u}, \code{v}, \code{w},
\code{x}, and \code{y}, and links them as shown in \link{Figure 3.28} using the
primitive constraint constructors \code{adder}, \code{multiplier}, and
\code{constant}.  Just as with the digital-circuit simulator of 
\link{Section 3.3.4}, expressing these combinations of primitive elements in terms of
procedures automatically provides our language with a means of abstraction for
compound objects.

To watch the network in action, we can place probes on the connectors \code{C}
and \code{F}, using a \code{probe} procedure similar to the one we used to
monitor wires in \link{Section 3.3.4}.  Placing a probe on a connector will
cause a message to be printed whenever the connector is given a value:

\begin{scheme}
(probe "Celsius temp" C)
(probe "Fahrenheit temp" F)
\end{scheme}

\noindent
Next we set the value of \code{C} to 25.  (The third argument to
\code{set\-/value!} tells \code{C} that this directive comes from the
\code{user}.)

\begin{scheme}
(set-value! C 25 'user)
~\textit{Probe: Celsius temp = 25}~
~\textit{Probe: Fahrenheit temp = 77}~
~\textit{done}~
\end{scheme}

\noindent
The probe on \code{C} awakens and reports the value.  \code{C} also propagates
its value through the network as described above.  This sets \code{F} to 77,
which is reported by the probe on \code{F}.

Now we can try to set \code{F} to a new value, say 212:

\begin{scheme}
(set-value! F 212 'user)
~\textit{Error! Contradiction (77 212)}~
\end{scheme}

\noindent
The connector complains that it has sensed a contradiction: Its value is 77,
and someone is trying to set it to 212.  If we really want to reuse the network
with new values, we can tell \code{C} to forget its old value:

\begin{scheme}
(forget-value! C 'user)
~\textit{Probe: Celsius temp = ?}~
~\textit{Probe: Fahrenheit temp = ?}~
~\textit{done}~
\end{scheme}

\noindent
\code{C} finds that the \code{user}, who set its value originally, is now
retracting that value, so \code{C} agrees to lose its value, as shown by the
probe, and informs the rest of the network of this fact.  This information
eventually propagates to \code{F}, which now finds that it has no reason for
continuing to believe that its own value is 77.  Thus, \code{F} also gives up
its value, as shown by the probe.

Now that \code{F} has no value, we are free to set it to 212:

\begin{scheme}
(set-value! F 212 'user)
~\textit{Probe: Fahrenheit temp = 212}~
~\textit{Probe: Celsius temp = 100}~
~\textit{done}~
\end{scheme}

\noindent
This new value, when propagated through the network, forces \code{C} to have a
value of 100, and this is registered by the probe on \code{C}.  Notice that the
very same network is being used to compute \code{C} given \code{F} and to
compute \code{F} given \code{C}.  This nondirectionality of computation is the
distinguishing feature of constraint-based systems.

\subsubsection*{Implementing the constraint system}

The constraint system is implemented via procedural objects with local state,
in a manner very similar to the digital-circuit simulator of 
\link{Section 3.3.4}.  Although the primitive objects of the constraint system are
somewhat more complex, the overall system is simpler, since there is no concern
about agendas and logic delays.

The basic operations on connectors are the following:

\begin{itemize}

\item
\code{(has\-/value? \( \langle \)\var{connector}\( \rangle \))} tells whether the connector has a value.

\item
\code{(get\-/value \( \langle \)\var{connector}\( \rangle \))} returns the connector's current value.

\item
\code{(set\-/value! \( \langle \)\var{connector}\( \rangle \) \( \langle \)\var{new\-/value}\( \rangle \) \( \langle \)\var{informant}\( \rangle \))}
indicates that the informant is requesting the connector to set its value to
the new value.

\item
\code{(forget\-/value! \( \langle \)\var{connector}\( \rangle \) \( \langle \)\var{retractor}\( \rangle \))} tells the connector
that the retractor is requesting it to forget its value.

\item
\code{(connect \( \langle \)\var{connector}\( \rangle \) \( \langle \)\var{new\-/constraint}\( \rangle \))} tells the connector
to participate in the new constraint.

\end{itemize}

\noindent
The connectors communicate with the constraints by means of the procedures
\code{inform\-/about\-/value}, which tells the given constraint that the connector
has a value, and \code{inform\-/about\-/no\-/value}, which tells the constraint that
the connector has lost its value.

\code{Adder} constructs an adder constraint among summand connectors \code{a1}
and \code{a2} and a \code{sum} connector.  An adder is implemented as a
procedure with local state (the procedure \code{me} below):

\begin{scheme}
(define (adder a1 a2 sum)
  (define (process-new-value)
    (cond ((and (has-value? a1) (has-value? a2))
           (set-value! sum
                       (+ (get-value a1) (get-value a2))
                       me))
          ((and (has-value? a1) (has-value? sum))
           (set-value! a2
                       (- (get-value sum) (get-value a1))
                       me))
          ((and (has-value? a2) (has-value? sum))
           (set-value! a1
                       (- (get-value sum) (get-value a2))
                       me))))
  (define (process-forget-value)
    (forget-value! sum me)
    (forget-value! a1 me)
    (forget-value! a2 me)
    (process-new-value))
  (define (me request)
    (cond ((eq? request 'I-have-a-value)  (process-new-value))
          ((eq? request 'I-lost-my-value) (process-forget-value))
          (else (error "Unknown request: ADDER" request))))
  (connect a1 me)
  (connect a2 me)
  (connect sum me)
  me)
\end{scheme}

\noindent
\code{Adder} connects the new adder to the designated connectors and returns it
as its value.  The procedure \code{me}, which represents the adder, acts as a
dispatch to the local procedures.  The following ``syntax interfaces'' (see
\link{Footnote 27} in \link{Section 3.3.4}) are used in conjunction with
the dispatch:

\begin{scheme}
(define (inform-about-value constraint)
  (constraint 'I-have-a-value))
(define (inform-about-no-value constraint)
  (constraint 'I-lost-my-value))
\end{scheme}

\noindent
The adder's local procedure \code{process\-/new\-/value} is called when the adder
is informed that one of its connectors has a value. The adder first checks to
see if both \code{a1} and \code{a2} have values. If so, it tells \code{sum} to
set its value to the sum of the two addends.  The \code{informant} argument to
\code{set\-/value!} is \code{me}, which is the adder object itself.  If \code{a1}
and \code{a2} do not both have values, then the adder checks to see if perhaps
\code{a1} and \code{sum} have values.  If so, it sets \code{a2} to the
difference of these two.  Finally, if \code{a2} and \code{sum} have values,
this gives the adder enough information to set \code{a1}.  If the adder is told
that one of its connectors has lost a value, it requests that all of its
connectors now lose their values.  (Only those values that were set by this
adder are actually lost.)  Then it runs \code{process\-/new\-/value}.  The reason
for this last step is that one or more connectors may still have a value (that
is, a connector may have had a value that was not originally set by the adder),
and these values may need to be propagated back through the adder.

A multiplier is very similar to an adder. It will set its \code{product} to 0
if either of the factors is 0, even if the other factor is not known.

\begin{scheme}
(define (multiplier m1 m2 product)
  (define (process-new-value)
    (cond ((or (and (has-value? m1) (= (get-value m1) 0))
               (and (has-value? m2) (= (get-value m2) 0)))
           (set-value! product 0 me))
          ((and (has-value? m1) (has-value? m2))
           (set-value! product
                       (* (get-value m1) (get-value m2))
                       me))
          ((and (has-value? product) (has-value? m1))
           (set-value! m2
                       (/ (get-value product) 
                          (get-value m1))
                       me))
          ((and (has-value? product) (has-value? m2))
           (set-value! m1
                       (/ (get-value product) 
                          (get-value m2))
                       me))))
  (define (process-forget-value)
    (forget-value! product me)
    (forget-value! m1 me)
    (forget-value! m2 me)
    (process-new-value))
  (define (me request)
    (cond ((eq? request 'I-have-a-value)  (process-new-value))
          ((eq? request 'I-lost-my-value) (process-forget-value))
          (else (error "Unknown request: MULTIPLIER" 
                       request))))
  (connect m1 me)
  (connect m2 me)
  (connect product me)
  me)
\end{scheme}

\noindent
A \code{constant} constructor simply sets the value of the designated
connector.  Any \code{I\-/have\-/a\-/value} or \code{I\-/lost\-/my\-/value} message sent to
the constant box will produce an error.

\begin{scheme}
(define (constant value connector)
  (define (me request)
    (error "Unknown request: CONSTANT" request))
  (connect connector me)
  (set-value! connector value me)
  me)
\end{scheme}

\noindent
Finally, a probe prints a message about the setting or unsetting of
the designated connector:

\begin{scheme}
(define (probe name connector)
  (define (print-probe value)
    (newline) (display "Probe: ") (display name)
    (display " = ") (display value))
  (define (process-new-value)
    (print-probe (get-value connector)))
  (define (process-forget-value) (print-probe "?"))
  (define (me request)
    (cond ((eq? request 'I-have-a-value)  (process-new-value))
          ((eq? request 'I-lost-my-value) (process-forget-value))
          (else (error "Unknown request: PROBE" request))))
  (connect connector me)
  me)
\end{scheme}

\subsubsection*{Representing connectors}

A connector is represented as a procedural object with local state variables
\code{value}, the current value of the connector; \code{informant}, the object
that set the connector's value; and \code{constraints}, a list of the
constraints in which the connector participates.

\begin{scheme}
(define (make-connector)
  (let ((value false) (informant false) (constraints '()))
    (define (set-my-value newval setter)
      (cond ((not (has-value? me))
             (set! value newval)
             (set! informant setter)
             (for-each-except setter
                              inform-about-value
                              constraints))
            ((not (= value newval))
             (error "Contradiction" (list value newval)))
            (else 'ignored)))
    (define (forget-my-value retractor)
      (if (eq? retractor informant)
          (begin (set! informant false)
                 (for-each-except retractor
                                  inform-about-no-value
                                  constraints))
          'ignored))
    (define (connect new-constraint)
      (if (not (memq new-constraint constraints))
          (set! constraints
                (cons new-constraint constraints)))
      (if (has-value? me)
          (inform-about-value new-constraint))
      'done)
    (define (me request)
      (cond ((eq? request 'has-value?)
             (if informant true false))
            ((eq? request 'value) value)
            ((eq? request 'set-value!) set-my-value)
            ((eq? request 'forget) forget-my-value)
            ((eq? request 'connect) connect)
            (else (error "Unknown operation: CONNECTOR"
                         request))))
    me))
\end{scheme}

\noindent
The connector's local procedure \code{set\-/my\-/value} is called when there is a
request to set the connector's value.  If the connector does not currently have
a value, it will set its value and remember as \code{informant} the constraint
that requested the value to be set.\footnote{The \code{setter} might not be a
constraint.  In our temperature example, we used \code{user} as the
\code{setter}.}  Then the connector will notify all of its participating
constraints except the constraint that requested the value to be set.  This is
accomplished using the following iterator, which applies a designated procedure
to all items in a list except a given one:

\begin{scheme}
(define (for-each-except exception procedure list)
  (define (loop items)
    (cond ((null? items) 'done)
          ((eq? (car items) exception) (loop (cdr items)))
          (else (procedure (car items))
                (loop (cdr items)))))
  (loop list))
\end{scheme}

\noindent
If a connector is asked to forget its value, it runs the local procedure
\code{forget\-/my\-/value}, which first checks to make sure that the request is
coming from the same object that set the value originally.  If so, the
connector informs its associated constraints about the loss of the value.

The local procedure \code{connect} adds the designated new constraint to the
list of constraints if it is not already in that list.  Then, if the connector
has a value, it informs the new constraint of this fact.

The connector's procedure \code{me} serves as a dispatch to the other internal
procedures and also represents the connector as an object.  The following
procedures provide a syntax interface for the dispatch:

\begin{scheme}
(define (has-value? connector) 
  (connector 'has-value?))
(define (get-value connector) 
  (connector 'value))
(define (set-value! connector new-value informant)
  ((connector 'set-value!) new-value informant))
(define (forget-value! connector retractor)
  ((connector 'forget) retractor))
(define (connect connector new-constraint)
  ((connector 'connect) new-constraint))
\end{scheme}

\begin{quote}
\heading{\phantomsection\label{Exercise 3.33}Exercise 3.33:} Using primitive multiplier,
adder, and constant constraints, define a procedure \code{averager} that takes
three connectors \code{a}, \code{b}, and \code{c} as inputs and establishes the
constraint that the value of \code{c} is the average of the values of \code{a}
and \code{b}.
\end{quote}

\begin{quote}
\heading{\phantomsection\label{Exercise 3.34}Exercise 3.34:} Louis Reasoner wants to build a
squarer, a constraint device with two terminals such that the value of
connector \code{b} on the second terminal will always be the square of the
value \code{a} on the first terminal.  He proposes the following simple device
made from a multiplier:

\begin{scheme}
(define (squarer a b) 
  (multiplier a a b))
\end{scheme}

There is a serious flaw in this idea.  Explain.
\end{quote}

\begin{quote}
\heading{\phantomsection\label{Exercise 3.35}Exercise 3.35:} Ben Bitdiddle tells Louis that
one way to avoid the trouble in \link{Exercise 3.34} is to define a squarer as a
new primitive constraint.  Fill in the missing portions in Ben's outline for a
procedure to implement such a constraint:

\begin{scheme}
(define (squarer a b)
  (define (process-new-value)
    (if (has-value? b)
        (if (< (get-value b) 0)
            (error "square less than 0: SQUARER" 
                   (get-value b))
            ~\( \dark \langle \)~~\var{\dark alternative1}~~\( \dark \rangle \)~)
        ~\( \dark \langle \)~~\var{\dark alternative2}~~\( \dark \rangle \)~))
  (define (process-forget-value) ~\( \dark \langle \)~~\var{\dark body1}~~\( \dark \rangle \)~)
  (define (me request) ~\( \dark \langle \)~~\var{\dark body2}~~\( \dark \rangle \)~)
  ~\( \dark \langle \)~~\var{\dark rest of definition}~~\( \dark \rangle \)~
  me)
\end{scheme}
\end{quote}

\begin{quote}
\heading{\phantomsection\label{Exercise 3.36}Exercise 3.36:} Suppose we evaluate the following
sequence of expressions in the global environment:

\begin{scheme}
(define a (make-connector))
(define b (make-connector))
(set-value! a 10 'user)
\end{scheme}

At some time during evaluation of the \code{set\-/value!}, the following
expression from the connector's local procedure is evaluated:

\begin{scheme}
(for-each-except 
  setter inform-about-value constraints)
\end{scheme}

Draw an environment diagram showing the environment in which the above
expression is evaluated.
\end{quote}

\begin{quote}
\heading{\phantomsection\label{Exercise 3.37}Exercise 3.37:} The
\code{celsius\-/fahrenheit\-/converter} procedure is cumbersome when compared with
a more expression-oriented style of definition, such as

\begin{scheme}
(define (celsius-fahrenheit-converter x)
  (c+ (c* (c/ (cv 9) (cv 5))
          x)
      (cv 32)))
(define C (make-connector))
(define F (celsius-fahrenheit-converter C))
\end{scheme}

Here \code{c+}, \code{c*}, etc. are the ``constraint'' versions of the
arithmetic operations.  For example, \code{c+} takes two connectors as
arguments and returns a connector that is related to these by an adder
constraint:

\begin{scheme}
(define (c+ x y)
  (let ((z (make-connector)))
    (adder x y z)
    z))
\end{scheme}

Define analogous procedures \code{c-}, \code{c*}, \code{c/}, and \code{cv}
(constant value) that enable us to define compound constraints as in the
converter example above.\footnote{The expression-oriented format is convenient
because it avoids the need to name the intermediate expressions in a
computation.  Our original formulation of the constraint language is cumbersome
in the same way that many languages are cumbersome when dealing with operations
on compound data.  For example, if we wanted to compute the product 
\( (a + b) \cdot (c + d) \), where the variables represent vectors, we could work
in ``imperative style,'' using procedures that set the values of designated
vector arguments but do not themselves return vectors as values:

\vspace{-0.8em}
\begin{smallexample}
(v-sum a b temp1)
(v-sum c d temp2)
(v-prod temp1 temp2 answer)
\end{smallexample}
\vspace{-0.8em}

\noindent
Alternatively, we could deal with expressions, using procedures that return
vectors as values, and thus avoid explicitly mentioning \code{temp1} and
\code{temp2}:

\vspace{-0.8em}
\begin{smallexample}
(define answer (v-prod (v-sum a b) (v-sum c d)))
\end{smallexample}
\vspace{-0.8em}

\noindent
Since Lisp allows us to return compound objects as values of procedures, we can
transform our imperative-style constraint language into an expression-oriented
style as shown in this exercise.  In languages that are impoverished in
handling compound objects, such as Algol, Basic, and Pascal (unless one
explicitly uses Pascal pointer variables), one is usually stuck with the
imperative style when manipulating compound objects.  Given the advantage of
the expression-oriented format, one might ask if there is any reason to have
implemented the system in imperative style, as we did in this section.  One
reason is that the non-expression-oriented constraint language provides a
handle on constraint objects (e.g., the value of the \code{adder} procedure) as
well as on connector objects.  This is useful if we wish to extend the system
with new operations that communicate with constraints directly rather than only
indirectly via operations on connectors.  Although it is easy to implement the
expression-oriented style in terms of the imperative implementation, it is very
difficult to do the converse.}
\end{quote}

\section{Concurrency: Time Is of the Essence}
\label{Section 3.4}

We've seen the power of computational objects with local state as tools for
modeling.  Yet, as \link{Section 3.1.3} warned, this power extracts a price: the
loss of referential transparency, giving rise to a thicket of questions about
sameness and change, and the need to abandon the substitution model of
evaluation in favor of the more intricate environment model.

The central issue lurking beneath the complexity of state, sameness, and change
is that by introducing assignment we are forced to admit \newterm{time} into
our computational models.  Before we introduced assignment, all our programs
were timeless, in the sense that any expression that has a value always has the
same value.  In contrast, recall the example of modeling withdrawals from a
bank account and returning the resulting balance, introduced at the beginning
of \link{Section 3.1.1}:

\begin{scheme}
(withdraw 25)
~\textit{75}~
(withdraw 25)
~\textit{50}~
\end{scheme}

\noindent
Here successive evaluations of the same expression yield different values.
This behavior arises from the fact that the execution of assignment statements
(in this case, assignments to the variable \code{balance}) delineates
\newterm{moments in time} when values change.  The result of evaluating an
expression depends not only on the expression itself, but also on whether the
evaluation occurs before or after these moments.  Building models in terms of
computational objects with local state forces us to confront time as an
essential concept in programming.

We can go further in structuring computational models to match our perception
of the physical world.  Objects in the world do not change one at a time in
sequence.  Rather we perceive them as acting \newterm{concurrently}---all at
once.  So it is often natural to model systems as collections of computational
processes that execute concurrently.  Just as we can make our programs modular
by organizing models in terms of objects with separate local state, it is often
appropriate to divide computational models into parts that evolve separately
and concurrently.  Even if the programs are to be executed on a sequential
computer, the practice of writing programs as if they were to be executed
concurrently forces the programmer to avoid inessential timing constraints and
thus makes programs more modular.

In addition to making programs more modular, concurrent computation can provide
a speed advantage over sequential computation.  Sequential computers execute
only one operation at a time, so the amount of time it takes to perform a task
is proportional to the total number of operations performed.\footnote{Most real
processors actually execute a few operations at a time, following a strategy
called \newterm{pipelining}.  Although this technique greatly improves the
effective utilization of the hardware, it is used only to speed up the
execution of a sequential instruction stream, while retaining the behavior of
the sequential program.}  However, if it is possible to decompose a problem
into pieces that are relatively independent and need to communicate only
rarely, it may be possible to allocate pieces to separate computing processors,
producing a speed advantage proportional to the number of processors available.

Unfortunately, the complexities introduced by assignment become even more
problematic in the presence of concurrency.  The fact of concurrent execution,
either because the world operates in parallel or because our computers do,
entails additional complexity in our understanding of time.



\subsection{The Nature of Time in Concurrent Systems}
\label{Section 3.4.1}

On the surface, time seems straightforward.  It is an ordering imposed on
events.\footnote{To quote some graffiti seen on a Cambridge building wall:
``Time is a device that was invented to keep everything from happening at
once.''}  For any events \( A \) and \( B \), either \( A \) occurs before \( B \),
\( A \) and \( B \) are simultaneous, or \( A \) occurs after \( B \).  For instance,
returning to the bank account example, suppose that Peter withdraws \$10 and
Paul withdraws \$25 from a joint account that initially contains \$100, leaving
\$65 in the account.  Depending on the order of the two withdrawals, the
sequence of balances in the account is either \( \,\$100 \to \$90 \to \$65\, \) or \( \,\$100 \to \$75
\to \$65\, \).  In a computer implementation of the banking system, this changing
sequence of balances could be modeled by successive assignments to a variable
\code{balance}.

In complex situations, however, such a view can be problematic.  Suppose that
Peter and Paul, and other people besides, are accessing the same bank account
through a network of banking machines distributed all over the world.  The
actual sequence of balances in the account will depend critically on the
detailed timing of the accesses and the details of the communication among the
machines.

This indeterminacy in the order of events can pose serious problems in the
design of concurrent systems.  For instance, suppose that the withdrawals made
by Peter and Paul are implemented as two separate processes sharing a common
variable \code{balance}, each process specified by the procedure given in
\link{Section 3.1.1}:

\begin{scheme}
(define (withdraw amount)
  (if (>= balance amount)
      (begin
        (set! balance (- balance amount)) 
        balance)
      "Insufficient funds"))
\end{scheme}

\noindent
If the two processes operate independently, then Peter might test the
balance and attempt to withdraw a legitimate amount.  However, Paul
might withdraw some funds in between the time that Peter checks the
balance and the time Peter completes the withdrawal, thus invalidating
Peter's test.

Things can be worse still.  Consider the expression

\begin{scheme}
(set! balance (- balance amount))
\end{scheme}

\noindent
executed as part of each withdrawal process.  This consists of three steps: (1)
accessing the value of the \code{balance} variable; (2) computing the new
balance; (3) setting \code{balance} to this new value.  If Peter and Paul's
withdrawals execute this statement concurrently, then the two withdrawals might
interleave the order in which they access \code{balance} and set it to the new
value.

\begin{figure}[tp]
\phantomsection\label{Figure 3.29}
\centering
\begin{comment}
\heading{Figure 3.29:} Timing diagram showing how interleaving the order of events 
in two banking withdrawals can lead to an incorrect final balance.

\begin{example}
 |           Peter              Bank              Paul
 |                              ____
 |                             /    \
 |             .--------------| \$100 |-------------.
 |             |               \____/              |
 |             V                                   V
 |  .----------------------.            .----------------------.           
 |  | Access balance: \$100 |            | Access balance: \$100 |
 |  `----------+-----------'            `----------+-----------'
 |             V                                   V
 |  .----------------------.            .----------------------.           
 |  | new value: 100-10=90 |            | new value: 100-25=75 |
 |  `----------+-----------'            `----------+-----------'
 |             V                                   |
 |  .----------------------.                       |
 |  | set! balance to \$90  |                       |
 |  `----------+-----------'    ____               |
 |             |               /    \              |
 |             `------------->| \$ 90 |             V
 |                             \____/   .----------------------.
 |                                      | new value: 100-25=75 |
 |                              ____    `----------+-----------'
 |                             /    \              |
 |                            | \$ 90 |<------------'
 V                             \____/
time
\end{example}
\end{comment}
\includegraphics[width=109mm]{fig/chap3/Fig3.29b.pdf}
\begin{quote}
\heading{Figure 3.29:} Timing diagram showing how interleaving the order of events in two banking withdrawals can lead to an incorrect final balance.
\end{quote}
\end{figure}

The timing diagram in \link{Figure 3.29} depicts an order of events where
\code{balance} starts at 100, Peter withdraws 10, Paul withdraws 25, and yet
the final value of \code{balance} is 75.  As shown in the diagram, the reason
for this anomaly is that Paul's assignment of 75 to \code{balance} is made
under the assumption that the value of \code{balance} to be decremented is 100.
That assumption, however, became invalid when Peter changed \code{balance} to
90.  This is a catastrophic failure for the banking system, because the total
amount of money in the system is not conserved.  Before the transactions, the
total amount of money was \$100.  Afterwards, Peter has \$10, Paul has \$25, and
the bank has \$75.\footnote{An even worse failure for this system could occur if
the two \code{set!} operations attempt to change the balance simultaneously, in
which case the actual data appearing in memory might end up being a random
combination of the information being written by the two processes.  Most
computers have interlocks on the primitive memory-write operations, which
protect against such simultaneous access.  Even this seemingly simple kind of
protection, however, raises implementation challenges in the design of
multiprocessing computers, where elaborate \newterm{cache-coherence} protocols
are required to ensure that the various processors will maintain a consistent
view of memory contents, despite the fact that data may be replicated
(``cached'') among the different processors to increase the speed of memory
access.}

The general phenomenon illustrated here is that several processes may share a
common state variable.  What makes this complicated is that more than one
process may be trying to manipulate the shared state at the same time.  For the
bank account example, during each transaction, each customer should be able to
act as if the other customers did not exist.  When a customer changes the
balance in a way that depends on the balance, he must be able to assume that,
just before the moment of change, the balance is still what he thought it was.

\subsubsection*{Correct behavior of concurrent programs}

The above example typifies the subtle bugs that can creep into concurrent
programs.  The root of this complexity lies in the assignments to variables
that are shared among the different processes.  We already know that we must be
careful in writing programs that use \code{set!}, because the results of a
computation depend on the order in which the assignments occur.\footnote{The
factorial program in \link{Section 3.1.3} illustrates this for a single
sequential process.}  With concurrent processes we must be especially careful
about assignments, because we may not be able to control the order of the
assignments made by the different processes.  If several such changes might be
made concurrently (as with two depositors accessing a joint account) we need
some way to ensure that our system behaves correctly.  For example, in the case
of withdrawals from a joint bank account, we must ensure that money is
conserved.  To make concurrent programs behave correctly, we may have to place
some restrictions on concurrent execution.

One possible restriction on concurrency would stipulate that no two operations
that change any shared state variables can occur at the same time.  This is an
extremely stringent requirement.  For distributed banking, it would require the
system designer to ensure that only one transaction could proceed at a time.
This would be both inefficient and overly conservative.  \link{Figure 3.30}
shows Peter and Paul sharing a bank account, where Paul has a private account
as well.  The diagram illustrates two withdrawals from the shared account (one
by Peter and one by Paul) and a deposit to Paul's private account.\footnote{The
columns show the contents of Peter's wallet, the joint account (in Bank1),
Paul's wallet, and Paul's private account (in Bank2), before and after each
withdrawal (W) and deposit (D).  Peter withdraws \$10 from Bank1; Paul deposits
\$5 in Bank2, then withdraws \$25 from Bank1.}  The two withdrawals from the
shared account must not be concurrent (since both access and update the same
account), and Paul's deposit and withdrawal must not be concurrent (since both
access and update the amount in Paul's wallet).  But there should be no problem
permitting Paul's deposit to his private account to proceed concurrently with
Peter's withdrawal from the shared account.

\begin{figure}[tb]
\phantomsection\label{Figure 3.30}
\centering
\begin{comment}
\heading{Figure 3.30:} Concurrent deposits and withdrawals from a joint account 
in Bank1 and a private account in Bank2.

\begin{example}
 |    Peter          Bank1          Paul           Bank2
 |    ____           ____           ____           ____  
 |   /    \         /    \         /    \         /    \ 
 |  |  \$7  |--. .--| \$100 |       |  \$5  |--. .--| \$300 |
 |   \____/   V V   \____/         \____/   V V   \____/ 
 |           +---+                         +---+         
 |           | W |                         | D |         
 |    ____   ++-++   ____           ____   ++-++   ____  
 |   /    \   | |   /    \         /    \   | |   /    \ 
 |  | \$17  |<-' `->| \$90  |--. .--|  \$0  |<-' `->| \$305 |
 |   \____/         \____/   V V   \____/         \____/ 
 |                          +---+
 |                          | W |
 |    ____           ____   ++-++   ____           ____  
 |   /    \         /    \   | |   /    \         /    \ 
 |  | \$17  |       | \$65  |<-' `->| \$25  |       | \$305 |
 |   \____/         \____/         \____/         \____/ 
 V
time
\end{example}
\end{comment}
\includegraphics[width=94mm]{fig/chap3/Fig3.30b.pdf}
\begin{quote}
\heading{Figure 3.30:} Concurrent deposits and withdrawals from a joint \mbox{account} in Bank1 and a private account in Bank2.
\end{quote}
\end{figure}

A less stringent restriction on concurrency would ensure that a concurrent
system produces the same result as if the processes had run sequentially in
some order.  There are two important aspects to this requirement.  First, it
does not require the processes to actually run sequentially, but only to
produce results that are the same \emph{as if} they had run sequentially.  For
the example in \link{Figure 3.30}, the designer of the bank account system can
safely allow Paul's deposit and Peter's withdrawal to happen concurrently,
because the net result will be the same as if the two operations had happened
sequentially.  Second, there may be more than one possible ``correct'' result
produced by a concurrent program, because we require only that the result be
the same as for \emph{some} sequential order.  For example, suppose that Peter
and Paul's joint account starts out with \$100, and Peter deposits \$40 while
Paul concurrently withdraws half the money in the account.  Then sequential
execution could result in the account balance being either \$70 or \$90 (see
\link{Exercise 3.38}).\footnote{\label{Footnote 39} A more formal
way to express this idea is to say that concurrent programs are inherently
\newterm{nondeterministic}. That is, they are described not by single-valued
functions, but by functions whose results are sets of possible values.  In
\link{Section 4.3} we will study a language for expressing nondeterministic
computations.}

There are still weaker requirements for correct execution of concurrent
programs.  A program for simulating diffusion (say, the flow of heat in an
object) might consist of a large number of processes, each one representing a
small volume of space, that update their values concurrently.  Each process
repeatedly changes its value to the average of its own value and its neighbors'
values.  This algorithm converges to the right answer independent of the order
in which the operations are done; there is no need for any restrictions on
concurrent use of the shared values.

\begin{quote}
\heading{\phantomsection\label{Exercise 3.38}Exercise 3.38:} Suppose that Peter, Paul, and
Mary share a joint bank account that initially contains \$100.  Concurrently,
Peter deposits \$10, Paul withdraws \$20, and Mary withdraws half the money in
the account, by executing the following commands:

\begin{scheme}
Peter: (set! balance (+ balance 10))
Paul:  (set! balance (- balance 20))
Mary:  (set! balance (- balance (/ balance 2)))
\end{scheme}

\begin{enumerate}[a.]

\item
List all the different possible values for \code{balance} after these three
transactions have been completed, assuming that the banking system forces the
three processes to run sequentially in some order.

\item
What are some other values that could be produced if the system allows the
processes to be interleaved?  Draw timing diagrams like the one in \link{Figure 3.29} 
to explain how these values can occur.

\end{enumerate}
\end{quote}

\subsection{Mechanisms for Controlling Concurrency}
\label{Section 3.4.2}

We've seen that the difficulty in dealing with concurrent processes is rooted
in the need to consider the interleaving of the order of events in the
different processes.  For example, suppose we have two processes, one with
three ordered events \( (a, b, c) \) and one with three ordered events
\( (x, y, z) \).  If the two processes run concurrently, with no
constraints on how their execution is interleaved, then there are 20 different
possible orderings for the events that are consistent with the individual
orderings for the two processes:

\begin{example}
(a,b,c,x,y,z)  (a,x,b,y,c,z)  (x,a,b,c,y,z)  (x,a,y,z,b,c)
(a,b,x,c,y,z)  (a,x,b,y,z,c)  (x,a,b,y,c,z)  (x,y,a,b,c,z)
(a,b,x,y,c,z)  (a,x,y,b,c,z)  (x,a,b,y,z,c)  (x,y,a,b,z,c)
(a,b,x,y,z,c)  (a,x,y,b,z,c)  (x,a,y,b,c,z)  (x,y,a,z,b,c)
(a,x,b,c,y,z)  (a,x,y,z,b,c)  (x,a,y,b,z,c)  (x,y,z,a,b,c)
\end{example}

\noindent
As programmers designing this system, we would have to consider the effects of
each of these 20 orderings and check that each behavior is acceptable.  Such an
approach rapidly becomes unwieldy as the numbers of processes and events
increase.

A more practical approach to the design of concurrent systems is to devise
general mechanisms that allow us to constrain the interleaving of concurrent
processes so that we can be sure that the program behavior is correct.  Many
mechanisms have been developed for this purpose.  In this section, we describe
one of them, the \newterm{serializer}.

\subsubsection*{Serializing access to shared state}

Serialization implements the following idea: Processes will execute
concurrently, but there will be certain collections of procedures that cannot
be executed concurrently.  More precisely, serialization creates distinguished
sets of procedures such that only one execution of a procedure in each
serialized set is permitted to happen at a time.  If some procedure in the set
is being executed, then a process that attempts to execute any procedure in the
set will be forced to wait until the first execution has finished.

We can use serialization to control access to shared variables.  For example,
if we want to update a shared variable based on the previous value of that
variable, we put the access to the previous value of the variable and the
assignment of the new value to the variable in the same procedure.  We then
ensure that no other procedure that assigns to the variable can run
concurrently with this procedure by serializing all of these procedures with
the same serializer.  This guarantees that the value of the variable cannot be
changed between an access and the corresponding assignment.

\subsubsection*{Serializers in Scheme}

To make the above mechanism more concrete, suppose that we have extended Scheme
to include a procedure called \code{parallel\-/execute}:

\begin{scheme}
(parallel-execute ~\( \dark \langle \)~~\( \dark p_1 \)~~\( \dark \rangle \)~ ~\( \dark \langle \)~~\( \dark p_2 \)~~\( \dark \rangle \)~ ~\( \dots \)~ ~\( \dark \langle \)~~\( \dark p_k \)~~\( \dark \rangle \)~)
\end{scheme}

\noindent
Each \( \langle \)\( p \)\( \kern0.08em\rangle \) must be a procedure of no arguments.  \code{Parallel\-/execute}
creates a separate process for each \( \langle \)\( p \)\( \kern0.08em\rangle \), which applies \( \langle \)\( p \)\( \kern0.08em\rangle \) (to no
arguments).  These processes all run
concurrently.\footnote{\code{Parallel\-/execute} is not part of standard Scheme,
but it can be implemented in \acronym{MIT} Scheme.  In our implementation, the
new concurrent processes also run concurrently with the original Scheme
process.  Also, in our implementation, the value returned by
\code{parallel\-/execute} is a special control object that can be used to halt
the newly created processes.}

As an example of how this is used, consider

\begin{scheme}
(define x 10)
(parallel-execute
 (lambda () (set! x (* x x)))
 (lambda () (set! x (+ x 1))))
\end{scheme}

\noindent
This creates two concurrent processes---\( P_1 \), which sets \code{x} to
\code{x} times \code{x}, and \( P_2 \), which increments \code{x}.  After
execution is complete, \code{x} will be left with one of five possible values,
depending on the interleaving of the events of \( P_1 \) and \( P_2 \):

\begin{scheme}
101: ~\textrm{\( P_1 \) sets \code{x} to 100 and then \( P_2 \) increments \code{x} to 101.}~
121: ~\textrm{\( P_2 \) increments \code{x} to 11 and then \( P_1 \) sets \code{x} to \code{x} \code{*} \code{x}.}~
110: ~\textrm{\( P_2 \) changes \code{x} from 10 to 11 between the two times that}~ 
     ~\textrm{\( P_1 \) accesses the value of \code{x} during the evaluation of \code{(* x x)}.}~
 11: ~\textrm{\( P_2 \) accesses \code{x}, then \( P_1 \) sets \code{x} to 100, then \( P_2 \) sets \code{x}.}~
100: ~\textrm{\( P_1 \) accesses \code{x} (twice), then \( P_2 \) sets \code{x} to 11, then \( P_1 \) sets \code{x}.}~
\end{scheme}

\noindent
We can constrain the concurrency by using serialized procedures, which are
created by \newterm{serializers}. Serializers are constructed by
\code{make\-/serializer}, whose implementation is given below.  A serializer
takes a procedure as argument and returns a serialized procedure that behaves
like the original procedure.  All calls to a given serializer return serialized
procedures in the same set.

Thus, in contrast to the example above, executing

\begin{scheme}
(define x 10)
(define s (make-serializer))
(parallel-execute
 (s (lambda () (set! x (* x x))))
 (s (lambda () (set! x (+ x 1)))))
\end{scheme}

\noindent
can produce only two possible values for \code{x}, 101 or 121.  The other
possibilities are eliminated, because the execution of \( P_1 \) and \( P_2 \)
cannot be interleaved.

Here is a version of the \code{make\-/account} procedure from 
\link{Section 3.1.1}, where the deposits and withdrawals have been serialized:

\begin{scheme}
(define (make-account balance)
  (define (withdraw amount)
    (if (>= balance amount)
        (begin (set! balance (- balance amount))
               balance)
        "Insufficient funds"))
  (define (deposit amount)
    (set! balance (+ balance amount))
    balance)
  (let ((protected (make-serializer)))
    (define (dispatch m)
      (cond ((eq? m 'withdraw) (protected withdraw))
            ((eq? m 'deposit) (protected deposit))
            ((eq? m 'balance) balance)
            (else (error "Unknown request: MAKE-ACCOUNT"
                         m))))
    dispatch))
\end{scheme}

\noindent
With this implementation, two processes cannot be withdrawing from or
depositing into a single account concurrently.  This eliminates the source of
the error illustrated in \link{Figure 3.29}, where Peter changes the account
balance between the times when Paul accesses the balance to compute the new
value and when Paul actually performs the assignment.  On the other hand, each
account has its own serializer, so that deposits and withdrawals for different
accounts can proceed concurrently.

\begin{quote}
\heading{\phantomsection\label{Exercise 3.39}Exercise 3.39:} Which of the five possibilities
in the parallel execution shown above remain if we instead serialize execution
as follows:

\begin{scheme}
(define x 10)
(define s (make-serializer))
(parallel-execute 
 (lambda () (set! x ((s (lambda () (* x x))))))
 (s (lambda () (set! x (+ x 1)))))
\end{scheme}
\end{quote}

\begin{quote}
\heading{\phantomsection\label{Exercise 3.40}Exercise 3.40:} Give all possible values of
\code{x} that can result from executing

\begin{scheme}
(define x 10)
(parallel-execute (lambda () (set! x (* x x)))
                  (lambda () (set! x (* x x x))))
\end{scheme}

Which of these possibilities remain if we instead use serialized procedures:

\begin{scheme}
(define x 10)
(define s (make-serializer))
(parallel-execute (s (lambda () (set! x (* x x))))
                  (s (lambda () (set! x (* x x x)))))
\end{scheme}
\end{quote}

\begin{quote}
\heading{\phantomsection\label{Exercise 3.41}Exercise 3.41:} Ben Bitdiddle worries that it
would be better to implement the bank account as follows (where the commented
line has been changed):

\begin{scheme}
(define (make-account balance)
  (define (withdraw amount)
    (if (>= balance amount)
        (begin (set! balance 
                     (- balance amount))
               balance)
        "Insufficient funds"))
  (define (deposit amount)
    (set! balance (+ balance amount))
    balance)
  (let ((protected (make-serializer)))
    (define (dispatch m)
      (cond ((eq? m 'withdraw) (protected withdraw))
            ((eq? m 'deposit) (protected deposit))
            ((eq? m 'balance)
             ((protected 
               (lambda () balance)))) ~\textrm{; serialized}~
            (else
             (error "Unknown request: MAKE-ACCOUNT"
                    m))))
    dispatch))
\end{scheme}

\noindent
because allowing unserialized access to the bank balance can result in
anomalous behavior.  Do you agree?  Is there any scenario that demonstrates
Ben's concern?
\end{quote}

\begin{quote}
\heading{\phantomsection\label{Exercise 3.42}Exercise 3.42:} Ben Bitdiddle suggests that it's
a waste of time to create a new serialized procedure in response to every
\code{withdraw} and \code{deposit} message.  He says that \code{make\-/account}
could be changed so that the calls to \code{protected} are done outside the
\code{dispatch} procedure.  That is, an account would return the same
serialized procedure (which was created at the same time as the account) each
time it is asked for a withdrawal procedure.

\begin{scheme}
(define (make-account balance)
  (define (withdraw amount)
    (if (>= balance amount)
        (begin (set! balance (- balance amount))
               balance)
        "Insufficient funds"))
  (define (deposit amount)
    (set! balance (+ balance amount))
    balance)
  (let ((protected (make-serializer)))
    (let ((protected-withdraw (protected withdraw))
          (protected-deposit (protected deposit)))
      (define (dispatch m)
        (cond ((eq? m 'withdraw) protected-withdraw)
              ((eq? m 'deposit) protected-deposit)
              ((eq? m 'balance) balance)
              (else 
               (error "Unknown request: MAKE-ACCOUNT"
                      m))))
      dispatch)))
\end{scheme}

Is this a safe change to make?  In particular, is there any difference in what
concurrency is allowed by these two versions of \code{make\-/account}?
\end{quote}

\subsubsection*{Complexity of using multiple shared resources}

Serializers provide a powerful abstraction that helps isolate the complexities
of concurrent programs so that they can be dealt with carefully and (hopefully)
correctly.  However, while using serializers is relatively straightforward when
there is only a single shared resource (such as a single bank account),
concurrent programming can be treacherously difficult when there are multiple
shared resources.

To illustrate one of the difficulties that can arise, suppose we wish to swap
the balances in two bank accounts.  We access each account to find the balance,
compute the difference between the balances, withdraw this difference from one
account, and deposit it in the other account.  We could implement this as
follows:\footnote{We have simplified \code{exchange} by exploiting the fact
that our \code{deposit} message accepts negative amounts.  (This is a serious
bug in our banking system!)}

\begin{scheme}
(define (exchange account1 account2)
  (let ((difference (- (account1 'balance)
                       (account2 'balance))))
    ((account1 'withdraw) difference)
    ((account2 'deposit) difference)))
\end{scheme}

\noindent
This procedure works well when only a single process is trying to do the
exchange.  Suppose, however, that Peter and Paul both have access to accounts
\( a \)1, \( a \)2, and \( a \)3, and that Peter exchanges \( a \)1 and \( a \)2 while
Paul concurrently exchanges \( a \)1 and \( a \)3.  Even with account deposits and
withdrawals serialized for individual accounts (as in the \code{make\-/account}
procedure shown above in this section), \code{exchange} can still produce
incorrect results.  For example, Peter might compute the difference in the
balances for \( a \)1 and \( a \)2, but then Paul might change the balance in
\( a \)1 before Peter is able to complete the exchange.\footnote{If the account
balances start out as \$10, \$20, and \$30, then after any number of concurrent
exchanges, the balances should still be \$10, \$20, and \$30 in some order.
Serializing the deposits to individual accounts is not sufficient to guarantee
this.  See \link{Exercise 3.43}.}  For correct behavior, we must arrange for the
\code{exchange} procedure to lock out any other concurrent accesses to the
accounts during the entire time of the exchange.

One way we can accomplish this is by using both accounts' serializers to
serialize the entire \code{exchange} procedure.  To do this, we will arrange
for access to an account's serializer.  Note that we are deliberately breaking
the modularity of the bank-account object by exposing the serializer.  The
following version of \code{make\-/account} is identical to the original version
given in \link{Section 3.1.1}, except that a serializer is provided to protect
the balance variable, and the serializer is exported via message passing:

\begin{scheme}
(define (make-account-and-serializer balance)
  (define (withdraw amount)
    (if (>= balance amount)
        (begin (set! balance (- balance amount))
               balance)
        "Insufficient funds"))
  (define (deposit amount)
    (set! balance (+ balance amount))
    balance)
  (let ((balance-serializer (make-serializer)))
    (define (dispatch m)
      (cond ((eq? m 'withdraw) withdraw)
            ((eq? m 'deposit) deposit)
            ((eq? m 'balance) balance)
            ((eq? m 'serializer) balance-serializer)
            (else (error "Unknown request: MAKE-ACCOUNT" m))))
    dispatch))
\end{scheme}

\enlargethispage{\baselineskip}

\noindent
We can use this to do serialized deposits and withdrawals.  However, unlike our
earlier serialized account, it is now the responsibility of each user of
bank-account objects to explicitly manage the serialization, for example as
follows:\footnote{\link{Exercise 3.45} investigates why deposits and withdrawals
are no longer automatically serialized by the account.}

\begin{scheme}
(define (deposit account amount)
  (let ((s (account 'serializer))
        (d (account 'deposit)))
    ((s d) amount)))
\end{scheme}

\noindent
Exporting the serializer in this way gives us enough flexibility to implement a
serialized exchange program.  We simply serialize the original \code{exchange}
procedure with the serializers for both accounts:

\begin{scheme}
(define (serialized-exchange account1 account2)
  (let ((serializer1 (account1 'serializer))
        (serializer2 (account2 'serializer)))
    ((serializer1 (serializer2 exchange))
     account1
     account2)))
\end{scheme}

\begin{quote}
\heading{\phantomsection\label{Exercise 3.43}Exercise 3.43:} Suppose that the balances in
three accounts start out as \$10, \$20, and \$30, and that multiple processes run,
exchanging the balances in the accounts.  Argue that if the processes are run
sequentially, after any number of concurrent exchanges, the account balances
should be \$10, \$20, and \$30 in some order.  Draw a timing diagram like the one
in \link{Figure 3.29} to show how this condition can be violated if the
exchanges are implemented using the first version of the account-exchange
program in this section.  On the other hand, argue that even with this
\code{exchange} program, the sum of the balances in the accounts will be
preserved.  Draw a timing diagram to show how even this condition would be
violated if we did not serialize the transactions on individual accounts.
\end{quote}

\begin{quote}
\heading{\phantomsection\label{Exercise 3.44}Exercise 3.44:} Consider the problem of
transferring an amount from one account to another.  Ben Bitdiddle claims that
this can be accomplished with the following procedure, even if there are
multiple people concurrently transferring money among multiple accounts, using
any account mechanism that serializes deposit and withdrawal transactions, for
example, the version of \code{make\-/account} in the text above.

\begin{scheme}
(define (transfer from-account to-account amount)
  ((from-account 'withdraw) amount)
  ((to-account 'deposit) amount))
\end{scheme}

Louis Reasoner claims that there is a problem here, and that we need to use a
more sophisticated method, such as the one required for dealing with the
exchange problem.  Is Louis right?  If not, what is the essential difference
between the transfer problem and the exchange problem?  (You should assume that
the balance in \code{from\-/account} is at least \code{amount}.)
\end{quote}

\begin{quote}
\heading{\phantomsection\label{Exercise 3.45}Exercise 3.45:} Louis Reasoner thinks our
bank-account system is unnecessarily complex and error-prone now that deposits
and withdrawals aren't automatically serialized.  He suggests that
\code{make\-/account\-/and\-/serializer} should have exported the serializer (for use
by such procedures as \code{serialized\-/exchange}) in addition to (rather than
instead of) using it to serialize accounts and deposits as \code{make\-/account}
did.  He proposes to redefine accounts as follows:

\begin{smallscheme}
(define (make-account-and-serializer balance)
  (define (withdraw amount)
    (if (>= balance amount)
        (begin (set! balance (- balance amount)) balance)
        "Insufficient funds"))
  (define (deposit amount)
    (set! balance (+ balance amount)) balance)
  (let ((balance-serializer (make-serializer)))
    (define (dispatch m)
      (cond ((eq? m 'withdraw) (balance-serializer withdraw))
            ((eq? m 'deposit) (balance-serializer deposit))
            ((eq? m 'balance) balance)
            ((eq? m 'serializer) balance-serializer)
            (else (error "Unknown request: MAKE-ACCOUNT" m))))
    dispatch))
\end{smallscheme}

Then deposits are handled as with the original \code{make\-/account}:

\begin{scheme}
(define (deposit account amount)
  ((account 'deposit) amount))
\end{scheme}

Explain what is wrong with Louis's reasoning.  In particular, consider what
happens when \code{serialized\-/exchange} is called.
\end{quote}

\subsubsection*{Implementing serializers}

We implement serializers in terms of a more primitive synchronization mechanism
called a \newterm{mutex}.  A mutex is an object that supports two
operations---the mutex can be \newterm{acquired}, and the mutex can be
\newterm{released}.  Once a mutex has been acquired, no other acquire
operations on that mutex may proceed until the mutex is released.\footnote{The
term ``mutex'' is an abbreviation for \newterm{mutual exclusion}.  The general
problem of arranging a mechanism that permits concurrent processes to safely
share resources is called the mutual exclusion problem.  Our mutex is a simple
variant of the \newterm{semaphore} mechanism (see \link{Exercise 3.47}), which
was introduced in the ``THE'' Multiprogramming System developed at the
Technological University of Eindhoven and named for the university's initials
in Dutch (\link{Dijkstra 1968a}).  The acquire and release operations were originally
called P and V, from the Dutch words \emph{passeren} (to pass) and
\emph{vrijgeven} (to release), in reference to the semaphores used on railroad
systems.  Dijkstra's classic exposition (\link{Dijkstra 1968b}) was one of the first to clearly
present the issues of concurrency control, and showed how to use semaphores to
handle a variety of concurrency problems.} In our implementation, each
serializer has an associated mutex.  Given a procedure \code{p}, the serializer
returns a procedure that acquires the mutex, runs \code{p}, and then releases
the mutex.  This ensures that only one of the procedures produced by the
serializer can be running at once, which is precisely the serialization
property that we need to guarantee.

\begin{scheme}
(define (make-serializer)
  (let ((mutex (make-mutex)))
    (lambda (p)
      (define (serialized-p . args)
        (mutex 'acquire)
        (let ((val (apply p args)))
          (mutex 'release)
          val))
      serialized-p)))
\end{scheme}

\noindent
The mutex is a mutable object (here we'll use a one-element list, which we'll
refer to as a \newterm{cell}) that can hold the value true or false.  When the
value is false, the mutex is available to be acquired.  When the value is true,
the mutex is unavailable, and any process that attempts to acquire the mutex
must wait.

Our mutex constructor \code{make\-/mutex} begins by initializing the cell
contents to false.  To acquire the mutex, we test the cell.  If the mutex is
available, we set the cell contents to true and proceed.  Otherwise, we wait in
a loop, attempting to acquire over and over again, until we find that the mutex
is available.\footnote{In most time-shared operating systems, processes that
are blocked by a mutex do not waste time ``busy-waiting'' as above.  Instead,
the system schedules another process to run while the first is waiting, and the
blocked process is awakened when the mutex becomes available.}  To release the
mutex, we set the cell contents to false.

\begin{scheme}
(define (make-mutex)
  (let ((cell (list false)))
    (define (the-mutex m)
      (cond ((eq? m 'acquire)
             (if (test-and-set! cell)
                 (the-mutex 'acquire))) ~\textrm{; retry}~
            ((eq? m 'release) (clear! cell))))
    the-mutex))
(define (clear! cell) (set-car! cell false))
\end{scheme}

\noindent
\code{Test\-/and\-/set!} tests the cell and returns the result of the test.  In
addition, if the test was false, \code{test\-/and\-/set!} sets the cell contents to
true before returning false.  We can express this behavior as the following
procedure:

\begin{scheme}
(define (test-and-set! cell)
  (if (car cell) true (begin (set-car! cell true) false)))
\end{scheme}

\noindent
However, this implementation of \code{test\-/and\-/set!} does not suffice as it
stands.  There is a crucial subtlety here, which is the essential place where
concurrency control enters the system: The \code{test\-/and\-/set!} operation must
be performed \newterm{atomically}.  That is, we must guarantee that, once a
process has tested the cell and found it to be false, the cell contents will
actually be set to true before any other process can test the cell.  If we do
not make this guarantee, then the mutex can fail in a way similar to the
bank-account failure in \link{Figure 3.29}.  (See \link{Exercise 3.46}.)

The actual implementation of \code{test\-/and\-/set!} depends on the details of how
our system runs concurrent processes.  For example, we might be executing
concurrent processes on a sequential processor using a time-slicing mechanism
that cycles through the processes, permitting each process to run for a short
time before interrupting it and moving on to the next process.  In that case,
\code{test\-/and\-/set!}  can work by disabling time slicing during the testing and
setting.\footnote{In \acronym{MIT} Scheme for a single processor, which uses a
time-slicing model, \code{test\-/and\-/set!} can be implemented as follows:

\begin{smallscheme}
(define (test-and-set! cell)
  (without-interrupts
   (lambda ()
     (if (car cell)
         true
         (begin (set-car! cell true)
                false)))))
\end{smallscheme}

\noindent
\code{Without\-/interrupts} disables time-slicing interrupts while its procedure
argument is being executed.}  Alternatively, multiprocessing computers provide
instructions that support atomic operations directly in
hardware.\footnote{There are many variants of such instructions---including
test-and-set, test-and-clear, swap, compare-and-exchange, load-reserve, and
store-conditional---whose design must be carefully matched to the machine's
processor-memory interface.  One issue that arises here is to determine what
happens if two processes attempt to acquire the same resource at exactly the
same time by using such an instruction.  This requires some mechanism for
making a decision about which process gets control.  Such a mechanism is called
an \newterm{arbiter}.  Arbiters usually boil down to some sort of hardware
device.  Unfortunately, it is possible to prove that one cannot physically
construct a fair arbiter that works 100\% of the time unless one allows the
arbiter an arbitrarily long time to make its decision.  The fundamental
phenomenon here was originally observed by the fourteenth-century French
philosopher Jean Buridan in his commentary on Aristotle's \textit{De caelo}.
Buridan argued that a perfectly rational dog placed between two equally
attractive sources of food will starve to death, because it is incapable of
deciding which to go to first.}

\begin{quote}
\heading{\phantomsection\label{Exercise 3.46}Exercise 3.46:} Suppose that we implement
\code{test\-/and\-/set!}  using an ordinary procedure as shown in the text, without
attempting to make the operation atomic.  Draw a timing diagram like the one in
\link{Figure 3.29} to demonstrate how the mutex implementation can fail by
allowing two processes to acquire the mutex at the same time.
\end{quote}

\begin{quote}
\heading{\phantomsection\label{Exercise 3.47}Exercise 3.47:} A semaphore (of size \( n \)) is a
generalization of a mutex.  Like a mutex, a semaphore supports acquire and
release operations, but it is more general in that up to \( n \) processes can
acquire it concurrently.  Additional processes that attempt to acquire the
semaphore must wait for release operations.  Give implementations of semaphores

\begin{enumerate}[a.]

\item
in terms of mutexes

\item
in terms of atomic \code{test\-/and\-/set!} operations.

\end{enumerate}
\end{quote}

\subsubsection*{Deadlock}

Now that we have seen how to implement serializers, we can see that account
exchanging still has a problem, even with the \code{serialized\-/exchange}
procedure above.  Imagine that Peter attempts to exchange \( a \)1 with \( a \)2
while Paul concurrently attempts to exchange \( a \)2 with \( a \)1.  Suppose that
Peter's process reaches the point where it has entered a serialized procedure
protecting \( a \)1 and, just after that, Paul's process enters a serialized
procedure protecting \( a \)2.  Now Peter cannot proceed (to enter a serialized
procedure protecting \( a \)2) until Paul exits the serialized procedure
protecting \( a \)2.  Similarly, Paul cannot proceed until Peter exits the
serialized procedure protecting \( a \)1.  Each process is stalled forever,
waiting for the other.  This situation is called a \newterm{deadlock}.
Deadlock is always a danger in systems that provide concurrent access to
multiple shared resources.

One way to avoid the deadlock in this situation is to give each account a
unique identification number and rewrite \code{serialized\-/exchange} so that a
process will always attempt to enter a procedure protecting the lowest-numbered
account first.  Although this method works well for the exchange problem, there
are other situations that require more sophisticated deadlock-avoidance
techniques, or where deadlock cannot be avoided at all.  (See \link{Exercise 3.48} 
and \link{Exercise 3.49}.)\footnote{The general technique for avoiding
deadlock by numbering the shared resources and acquiring them in order is due
to \link{Havender (1968)}.  Situations where deadlock cannot be avoided require
\newterm{deadlock-recovery} methods, which entail having processes ``back out''
of the deadlocked state and try again.  Deadlock-recovery mechanisms are widely
used in database management systems, a topic that is treated in detail in 
\link{Gray and Reuter 1993}.}

\begin{quote}
\heading{\phantomsection\label{Exercise 3.48}Exercise 3.48:} Explain in detail why the
deadlock-avoidance method described above, (i.e., the accounts are numbered,
and each process attempts to acquire the smaller-numbered account first) avoids
deadlock in the exchange problem.  Re\-write \code{serialized\-/exchange} to
incorporate this idea.  (You will also need to modify \code{make\-/account} so
that each account is created with a number, which can be accessed by sending an
appropriate message.)
\end{quote}

\begin{quote}
\heading{\phantomsection\label{Exercise 3.49}Exercise 3.49:} Give a scenario where the
deadlock-avoid\-ance mechanism described above does not work.  (Hint: In the
exchange problem, each process knows in advance which accounts it will need to
get access to.  Consider a situation where a process must get access to some
shared resources before it can know which additional shared resources it will
require.)
\end{quote}

\subsubsection*{Concurrency, time, and communication}

We've seen how programming concurrent systems requires controlling the ordering
of events when different processes access shared state, and we've seen how to
achieve this control through judicious use of serializers.  But the problems of
concurrency lie deeper than this, because, from a fundamental point of view,
it's not always clear what is meant by ``shared state.''

Mechanisms such as \code{test\-/and\-/set!} require processes to examine a global
shared flag at arbitrary times.  This is problematic and inefficient to
implement in modern high-speed processors, where due to optimization techniques
such as pipelining and cached memory, the contents of memory may not be in a
consistent state at every instant.  In contemporary multiprocessing systems,
therefore, the serializer paradigm is being supplanted by new approaches to
concurrency control.\footnote{One such alternative to serialization is called
\newterm{barrier synchronization}.  The programmer permits concurrent processes
to execute as they please, but establishes certain synchronization points
(``barriers'') through which no process can proceed until all the processes
have reached the barrier.  Modern processors provide machine instructions that
permit programmers to establish synchronization points at places where
consistency is required.  The \acronym{PowerPC}, for example, includes for
this purpose two instructions called \acronym{SYNC} and \acronym{EIEIO} 
(Enforced In-order Execution of Input/Output).}

The problematic aspects of shared state also arise in large, distributed
systems.  For instance, imagine a distributed banking system where individual
branch banks maintain local values for bank balances and periodically compare
these with values maintained by other branches.  In such a system the value of
``the account balance'' would be undetermined, except right after
synchronization.  If Peter deposits money in an account he holds jointly with
Paul, when should we say that the account balance has changed---when the
balance in the local branch changes, or not until after the synchronization?
And if Paul accesses the account from a different branch, what are the
reasonable constraints to place on the banking system such that the behavior is
``correct''?  The only thing that might matter for correctness is the behavior
observed by Peter and Paul individually and the ``state'' of the account
immediately after synchronization.  Questions about the ``real'' account
balance or the order of events between synchronizations may be irrelevant or
meaningless.\footnote{This may seem like a strange point of view, but there are
systems that work this way.  International charges to credit-card accounts, for
example, are normally cleared on a per-country basis, and the charges made in
different countries are periodically reconciled.  Thus the account balance may
be different in different countries.}

The basic phenomenon here is that synchronizing different processes,
establishing shared state, or imposing an order on events requires
communication among the processes.  In essence, any notion of time in
concurrency control must be intimately tied to communication.\footnote{For
distributed systems, this perspective was pursued by \link{Lamport (1978)}, who showed
how to use communication to establish ``global clocks'' that can be used to
establish orderings on events in distributed systems.}  It is intriguing that a
similar connection between time and communication also arises in the Theory of
Relativity, where the speed of light (the fastest signal that can be used to
synchronize events) is a fundamental constant relating time and space.  The
complexities we encounter in dealing with time and state in our computational
models may in fact mirror a fundamental complexity of the physical universe.

\section{Streams}
\label{Section 3.5}

We've gained a good understanding of assignment as a tool in modeling, as well
as an appreciation of the complex problems that assignment raises. It is time
to ask whether we could have gone about things in a different way, so as to
avoid some of these problems.  In this section, we explore an alternative
approach to modeling state, based on data structures called \newterm{streams}.
As we shall see, streams can mitigate some of the complexity of modeling state.

Let's step back and review where this complexity comes from.  In an attempt to
model real-world phenomena, we made some apparently reasonable decisions: We
modeled real-world objects with local state by computational objects with local
variables.  We identified time variation in the real world with time variation
in the computer.  We implemented the time variation of the states of the model
objects in the computer with assignments to the local variables of the model
objects.

Is there another approach?  Can we avoid identifying time in the computer with
time in the modeled world?  Must we make the model change with time in order to
model phenomena in a changing world?  Think about the issue in terms of
mathematical functions.  We can describe the time-varying behavior of a
quantity \( x \) as a function of time \( x(t) \).  If we concentrate on \( x \)
instant by instant, we think of it as a changing quantity.  Yet if we
concentrate on the entire time history of values, we do not emphasize
change---the function itself does not change.\footnote{Physicists sometimes
adopt this view by introducing the ``world lines'' of particles as a device for
reasoning about motion.  We've also already mentioned (\link{Section 2.2.3})
that this is the natural way to think about signal-processing systems.  We will
explore applications of streams to signal processing in \link{Section 3.5.3}.}

If time is measured in discrete steps, then we can model a time function as a
(possibly infinite) sequence.  In this section, we will see how to model change
in terms of sequences that represent the time histories of the systems being
modeled.  To accomplish this, we introduce new data structures called
\newterm{streams}.  From an abstract point of view, a stream is simply a
sequence.  However, we will find that the straightforward implementation of
streams as lists (as in \link{Section 2.2.1}) doesn't fully reveal the power of
stream processing.  As an alternative, we introduce the technique of
\newterm{delayed evaluation}, which enables us to represent very large (even
infinite) sequences as streams.

Stream processing lets us model systems that have state without ever using
assignment or mutable data.  This has important implications, both theoretical
and practical, because we can build models that avoid the drawbacks inherent in
introducing assignment.  On the other hand, the stream framework raises
difficulties of its own, and the question of which modeling technique leads to
more modular and more easily maintained systems remains open.



\subsection{Streams Are Delayed Lists}
\label{Section 3.5.1}

As we saw in \link{Section 2.2.3}, sequences can serve as standard interfaces
for combining program modules.  We formulated powerful abstractions for
manipulating sequences, such as \code{map}, \code{filter}, and
\code{accumulate}, that capture a wide variety of operations in a manner that
is both succinct and elegant.

Unfortunately, if we represent sequences as lists, this elegance is bought at
the price of severe inefficiency with respect to both the time and space
required by our computations.  When we represent manipulations on sequences as
transformations of lists, our programs must construct and copy data structures
(which may be huge) at every step of a process.

To see why this is true, let us compare two programs for computing the sum of
all the prime numbers in an interval.  The first program is written in standard
iterative style:\footnote{Assume that we have a predicate \code{prime?} (e.g.,
as in \link{Section 1.2.6}) that tests for primality.}

\begin{scheme}
(define (sum-primes a b)
  (define (iter count accum)
    (cond ((> count b) accum)
          ((prime? count) 
             (iter (+ count 1) (+ count accum)))
          (else (iter (+ count 1) accum))))
  (iter a 0))
\end{scheme}

\noindent
The second program performs the same computation using the sequence operations
of \link{Section 2.2.3}:

\begin{scheme}
(define (sum-primes a b)
  (accumulate +
              0
              (filter prime? 
                      (enumerate-interval a b))))
\end{scheme}

\noindent
In carrying out the computation, the first program needs to store only the sum
being accumulated.  In contrast, the filter in the second program cannot do any
testing until \code{enumerate\-/interval} has constructed a complete list of the
numbers in the interval.  The filter generates another list, which in turn is
passed to \code{accumulate} before being collapsed to form a sum.  Such large
intermediate storage is not needed by the first program, which we can think of
as enumerating the interval incrementally, adding each prime to the sum as it
is generated.

The inefficiency in using lists becomes painfully apparent if we use the
sequence paradigm to compute the second prime in the interval from 10,000 to
1,000,000 by evaluating the expression

\begin{scheme}
(car (cdr (filter prime?
                  (enumerate-interval 10000 1000000))))
\end{scheme}

\noindent
This expression does find the second prime, but the computational overhead is
outrageous.  We construct a list of almost a million integers, filter this list
by testing each element for primality, and then ignore almost all of the
result.  In a more traditional programming style, we would interleave the
enumeration and the filtering, and stop when we reached the second prime.

Streams are a clever idea that allows one to use sequence manipulations without
incurring the costs of manipulating sequences as lists.  With streams we can
achieve the best of both worlds: We can formulate programs elegantly as
sequence manipulations, while attaining the efficiency of incremental
computation.  The basic idea is to arrange to construct a stream only
partially, and to pass the partial construction to the program that consumes
the stream.  If the consumer attempts to access a part of the stream that has
not yet been constructed, the stream will automatically construct just enough
more of itself to produce the required part, thus preserving the illusion that
the entire stream exists.  In other words, although we will write programs as
if we were processing complete sequences, we design our stream implementation
to automatically and transparently interleave the construction of the stream
with its use.

On the surface, streams are just lists with different names for the procedures
that manipulate them.  There is a constructor, \code{cons\-/stream}, and two
selectors, \code{stream\-/car} and \code{stream\-/cdr}, which satisfy the
constraints

\begin{scheme}
(stream-car (cons-stream x y)) = x
(stream-cdr (cons-stream x y)) = y
\end{scheme}

\noindent
There is a distinguishable object, \code{the\-/empty\-/stream}, which cannot be the
result of any \code{cons\-/stream} operation, and which can be identified with
the predicate \code{stream\-/null?}.\footnote{In the \acronym{MIT}
implementation, \code{the\-/empty\-/stream} is the same as the empty list
\code{'()}, and \code{stream\-/null?} is the same as \code{null?}.}  Thus we can
make and use streams, in just the same way as we can make and use lists, to
represent aggregate data arranged in a sequence.  In particular, we can build
stream analogs of the list operations from \link{Chapter 2}, such as
\code{list\-/ref}, \code{map}, and \code{for\-/each}:\footnote{This should bother
you.  The fact that we are defining such similar procedures for streams and
lists indicates that we are missing some underlying abstraction.
Unfortunately, in order to exploit this abstraction, we will need to exert
finer control over the process of evaluation than we can at present.  We will
discuss this point further at the end of \link{Section 3.5.4}.  In 
\link{Section 4.2}, we'll develop a framework that unifies lists and streams.}

\begin{scheme}
(define (stream-ref s n)
  (if (= n 0)
      (stream-car s)
      (stream-ref (stream-cdr s) (- n 1))))
(define (stream-map proc s)
  (if (stream-null? s)
      the-empty-stream
      (cons-stream (proc (stream-car s))
                   (stream-map proc (stream-cdr s)))))
(define (stream-for-each proc s)
  (if (stream-null? s)
      'done
      (begin (proc (stream-car s))
             (stream-for-each proc (stream-cdr s)))))
\end{scheme}

\noindent
\code{Stream\-/for\-/each} is useful for viewing streams:

\begin{scheme}
(define (display-stream s) 
  (stream-for-each display-line s))
(define (display-line x) (newline) (display x))
\end{scheme}

\noindent
To make the stream implementation automatically and transparently interleave
the construction of a stream with its use, we will arrange for the \code{cdr}
of a stream to be evaluated when it is accessed by the \code{stream\-/cdr}
procedure rather than when the stream is constructed by \code{cons\-/stream}.
This implementation choice is reminiscent of our discussion of rational numbers
in \link{Section 2.1.2}, where we saw that we can choose to implement rational
numbers so that the reduction of numerator and denominator to lowest terms is
performed either at construction time or at selection time.  The two
rational-number implementations produce the same data abstraction, but the
choice has an effect on efficiency.  There is a similar relationship between
streams and ordinary lists.  As a data abstraction, streams are the same as
lists.  The difference is the time at which the elements are evaluated.  With
ordinary lists, both the \code{car} and the \code{cdr} are evaluated at
construction time.  With streams, the \code{cdr} is evaluated at selection
time.

Our implementation of streams will be based on a special form called
\code{delay}.  Evaluating \code{(delay \( \langle \)\var{exp}\( \rangle \))} does not evaluate the
expression \( \langle \)\var{exp}\( \kern0.08em\rangle \), but rather returns a so-called \newterm{delayed
object}, which we can think of as a ``promise'' to evaluate \( \langle \)\var{exp}\( \kern0.08em\rangle \) at some
future time.  As a companion to \code{delay}, there is a procedure called
\code{force} that takes a delayed object as argument and performs the
evaluation---in effect, forcing the \code{delay} to fulfill its promise.  We
will see below how \code{delay} and \code{force} can be implemented, but first
let us use these to construct streams.

\code{Cons\-/stream} is a special form defined so that

\begin{scheme}
(cons-stream ~\( \dark \langle \)~~\var{\dark a}~~\( \dark \rangle \)~ ~\( \dark \langle \)~~\var{\dark b}~~\( \dark \rangle \)~)
\end{scheme}

\noindent
is equivalent to

\begin{scheme}
(cons ~\( \dark \langle \)~~\var{\dark a}~~\( \dark \rangle \)~ (delay ~\( \dark \langle \)~~\var{\dark b}~~\( \dark \rangle \)~))
\end{scheme}

\noindent
What this means is that we will construct streams using pairs.  However, rather
than placing the value of the rest of the stream into the \code{cdr} of the
pair we will put there a promise to compute the rest if it is ever requested.
\code{Stream\-/car} and \code{stream\-/cdr} can now be defined as procedures:

\begin{scheme}
(define (stream-car stream) (car stream))
(define (stream-cdr stream) (force (cdr stream)))
\end{scheme}

\noindent
\code{Stream\-/car} selects the \code{car} of the pair; \code{stream\-/cdr} selects
the \code{cdr} of the pair and evaluates the delayed expression found there to
obtain the rest of the stream.\footnote{Although \code{stream\-/car} and
\code{stream\-/cdr} can be defined as procedures, \code{cons\-/stream} must be a
special form.  If \code{cons\-/stream} were a procedure, then, according to our
model of evaluation, evaluating \code{(cons\-/stream \( \langle \)\var{a}\( \rangle \) \( \langle \)\var{b}\( \rangle \))} would
automatically cause \( \langle \)\var{b}\( \kern0.08em\rangle \) to be evaluated, which is precisely what we do
not want to happen.  For the same reason, \code{delay} must be a special form,
though \code{force} can be an ordinary procedure.}

\subsubsection*{The stream implementation in action}

To see how this implementation behaves, let us analyze the ``outrageous'' prime
computation we saw above, reformulated in terms of streams:

\begin{scheme}
(stream-car
 (stream-cdr
  (stream-filter prime? 
                 (stream-enumerate-interval 
                  10000 1000000))))
\end{scheme}

\noindent
We will see that it does indeed work efficiently.

We begin by calling \code{stream\-/enumerate\-/interval} with the arguments 10,000
and 1,000,000.  \code{Stream\-/enumerate\-/interval} is the stream analog of
\code{enumerate\-/interval} (\link{Section 2.2.3}):

\begin{scheme}
(define (stream-enumerate-interval low high)
  (if (> low high)
      the-empty-stream
      (cons-stream
       low
       (stream-enumerate-interval (+ low 1) high))))
\end{scheme}

\noindent
and thus the result returned by \code{stream\-/enumerate\-/interval}, formed by the
\code{cons\-/stream}, is\footnote{The numbers shown here do not really appear in
the delayed expression.  What actually appears is the original expression, in
an environment in which the variables are bound to the appropriate numbers.
For example, \code{(+ low 1)} with \code{low} bound to 10,000 actually appears
where \code{10001} is shown.}

\begin{scheme}
(cons 10000
      (delay (stream-enumerate-interval 10001 1000000)))
\end{scheme}

\noindent
That is, \code{stream\-/enumerate\-/interval} returns a stream represented as a
pair whose \code{car} is 10,000 and whose \code{cdr} is a promise to enumerate
more of the interval if so requested.  This stream is now filtered for primes,
using the stream analog of the \code{filter} procedure (\link{Section 2.2.3}):

\begin{scheme}
(define (stream-filter pred stream)
  (cond ((stream-null? stream) the-empty-stream)
        ((pred (stream-car stream))
         (cons-stream (stream-car stream)
                      (stream-filter 
                       pred
                       (stream-cdr stream))))
        (else (stream-filter pred (stream-cdr stream)))))
\end{scheme}

\noindent
\code{Stream\-/filter} tests the \code{stream\-/car} of the stream (the \code{car}
of the pair, which is 10,000).  Since this is not prime, \code{stream\-/filter}
examines the \code{stream\-/cdr} of its input stream.  The call to
\code{stream\-/cdr} forces evaluation of the delayed
\code{stream\-/enumerate\-/interval}, which now returns

\begin{scheme}
(cons 10001
      (delay (stream-enumerate-interval 10002 1000000)))
\end{scheme}

\noindent
\code{Stream\-/filter} now looks at the \code{stream\-/car} of this stream, 10,001,
sees that this is not prime either, forces another \code{stream\-/cdr}, and so
on, until \code{stream\-/enumerate\-/interval} yields the prime 10,007, whereupon
\code{stream\-/filter}, according to its definition, returns

\begin{scheme}
(cons-stream (stream-car stream)
             (stream-filter pred (stream-cdr stream)))
\end{scheme}

\noindent
which in this case is

\begin{scheme}
(cons 10007
      (delay (stream-filter
              prime?
              (cons 10008
                    (delay (stream-enumerate-interval
                            10009
                            1000000))))))
\end{scheme}

\noindent
This result is now passed to \code{stream\-/cdr} in our original expression.
This forces the delayed \code{stream\-/filter}, which in turn keeps forcing the
delayed \code{stream\-/enumerate\-/interval} until it finds the next prime, which
is 10,009.  Finally, the result passed to \code{stream\-/car} in our original
expression is

\begin{scheme}
(cons 10009
      (delay (stream-filter
              prime?
              (cons 10010
                    (delay (stream-enumerate-interval
                            10011
                            1000000))))))
\end{scheme}

\noindent
\code{Stream\-/car} returns 10,009, and the computation is complete.  Only as
many integers were tested for primality as were necessary to find the second
prime, and the interval was enumerated only as far as was necessary to feed the
prime filter.

In general, we can think of delayed evaluation as ``demand-driven''
programming, whereby each stage in the stream process is activated only enough
to satisfy the next stage.  What we have done is to decouple the actual order
of events in the computation from the apparent structure of our procedures.  We
write procedures as if the streams existed ``all at once'' when, in reality,
the computation is performed incrementally, as in traditional programming
styles.

\subsubsection*{Implementing \code{delay} and \code{force}}

Although \code{delay} and \code{force} may seem like mysterious operations,
their implementation is really quite straightforward.  \code{Delay} must
package an expression so that it can be evaluated later on demand, and we can
accomplish this simply by treating the expression as the body of a procedure.
\code{Delay} can be a special form such that

\begin{scheme}
(delay ~\( \dark \langle \)~~\var{\dark exp}~~\( \dark \rangle \)~)
\end{scheme}

\noindent
is syntactic sugar for

\begin{scheme}
(lambda () ~\( \dark \langle \)~~\var{\dark exp}~~\( \dark \rangle \)~)
\end{scheme}

\noindent
\code{Force} simply calls the procedure (of no arguments) produced by
\code{delay}, so we can implement \code{force} as a procedure:

\begin{scheme}
(define (force delayed-object) (delayed-object))
\end{scheme}

\noindent
This implementation suffices for \code{delay} and \code{force} to work as
advertised, but there is an important optimization that we can include.  In
many applications, we end up forcing the same delayed object many times.  This
can lead to serious inefficiency in recursive programs involving streams.  (See
\link{Exercise 3.57}.)  The solution is to build delayed objects so that the
first time they are forced, they store the value that is computed.  Subsequent
forcings will simply return the stored value without repeating the computation.
In other words, we implement \code{delay} as a special-purpose memoized
procedure similar to the one described in \link{Exercise 3.27}.  One way to
accomplish this is to use the following procedure, which takes as argument a
procedure (of no arguments) and returns a memoized version of the procedure.
The first time the memoized procedure is run, it saves the computed result.  On
subsequent evaluations, it simply returns the result.

\begin{scheme}
(define (memo-proc proc)
  (let ((already-run? false) (result false))
    (lambda ()
      (if (not already-run?)
          (begin (set! result (proc))
                 (set! already-run? true)
                 result)
          result))))
\end{scheme}

\noindent
\code{Delay} is then defined so that \code{(delay \( \langle \)\var{exp}\( \rangle \))} is equivalent
to

\begin{scheme}
(memo-proc (lambda () ~\( \dark \langle \)~~\var{\dark exp}~~\( \dark \rangle \)~))
\end{scheme}

\noindent
and \code{force} is as defined previously.\footnote{There are many possible
implementations of streams other than the one described in this section.
Delayed evaluation, which is the key to making streams practical, was inherent
in Algol 60's \newterm{call-by-name} parameter-passing method.  The use of this
mechanism to implement streams was first described by \link{Landin (1965)}.  Delayed
evaluation for streams was introduced into Lisp by \link{Friedman and Wise (1976)}. In
their implementation, \code{cons} always delays evaluating its arguments, so
that lists automatically behave as streams.  The memoizing optimization is also
known as \newterm{call-by-need}.  The Algol community would refer to our
original delayed objects as \newterm{call-by-name thunks} and to the optimized
versions as \newterm{call-by-need thunks}.}

\begin{quote}
\heading{\phantomsection\label{Exercise 3.50}Exercise 3.50:} Complete the following
definition, which generalizes \code{stream\-/map} to allow procedures that take
multiple arguments, analogous to \code{map} in \link{Section 2.2.1}, 
\link{Footnote 12}.

\begin{scheme}
(define (stream-map proc . argstreams)
  (if (~\( \dark \langle \)~??~\( \dark \rangle \)~ (car argstreams))
      the-empty-stream
      (~\( \dark \langle \)~??~\( \dark \rangle \)~
       (apply proc (map ~\( \dark \langle \)~??~\( \dark \rangle \)~ argstreams))
       (apply stream-map
              (cons proc (map ~\( \dark \langle \)~??~\( \dark \rangle \)~ argstreams))))))
\end{scheme}
\end{quote}

\begin{quote}
\heading{\phantomsection\label{Exercise 3.51}Exercise 3.51:} In order to take a closer look at
delayed evaluation, we will use the following procedure, which simply returns
its argument after printing it:

\begin{scheme}
(define (show x)
  (display-line x)
  x)
\end{scheme}

What does the interpreter print in response to evaluating each expression in
the following sequence?\footnote{Exercises such as \link{Exercise 3.51} and
\link{Exercise 3.52} are valuable for testing our understanding of how
\code{delay} works.  On the other hand, intermixing delayed evaluation with
printing---and, even worse, with assignment---is extremely confusing, and
instructors of courses on computer languages have traditionally tormented their
students with examination questions such as the ones in this section.  Needless
to say, writing programs that depend on such subtleties is odious programming
style.  Part of the power of stream processing is that it lets us ignore the
order in which events actually happen in our programs.  Unfortunately, this is
precisely what we cannot afford to do in the presence of assignment, which
forces us to be concerned with time and change.}

\begin{scheme}
(define x 
  (stream-map show 
              (stream-enumerate-interval 0 10)))
(stream-ref x 5)
(stream-ref x 7)
\end{scheme}
\end{quote}

\begin{quote}
\heading{\phantomsection\label{Exercise 3.52}Exercise 3.52:} Consider the sequence of
expressions

\begin{scheme}
(define sum 0)
(define (accum x) (set! sum (+ x sum)) sum)
(define seq 
  (stream-map accum
              (stream-enumerate-interval 1 20)))
(define y (stream-filter even? seq))
(define z 
  (stream-filter (lambda (x) (= (remainder x 5) 0))
                 seq))
(stream-ref y 7)
(display-stream z)
\end{scheme}

What is the value of \code{sum} after each of the above expressions is
evaluated?  What is the printed response to evaluating the \code{stream\-/ref}
and \code{display\-/stream} expressions?  Would these responses differ if we had
implemented \code{(delay \( \langle \)\var{exp}\( \rangle \))} simply as \code{(lambda () \( \langle \)\var{exp}\( \rangle \))}
without using the optimization provided by \code{memo\-/proc}?  Explain
\end{quote}

\subsection{Infinite Streams}
\label{Section 3.5.2}

We have seen how to support the illusion of manipulating streams as complete
entities even though, in actuality, we compute only as much of the stream as we
need to access.  We can exploit this technique to represent sequences
efficiently as streams, even if the sequences are very long.  What is more
striking, we can use streams to represent sequences that are infinitely long.
For instance, consider the following definition of the stream of positive
integers:

\begin{scheme}
(define (integers-starting-from n)
  (cons-stream n (integers-starting-from (+ n 1))))
(define integers (integers-starting-from 1))
\end{scheme}

\noindent
This makes sense because \code{integers} will be a pair whose \code{car} is 1
and whose \code{cdr} is a promise to produce the integers beginning with 2.
This is an infinitely long stream, but in any given time we can examine only a
finite portion of it.  Thus, our programs will never know that the entire
infinite stream is not there.

Using \code{integers} we can define other infinite streams, such as the stream
of integers that are not divisible by 7:

\begin{scheme}
(define (divisible? x y) (= (remainder x y) 0))
(define no-sevens
  (stream-filter (lambda (x) (not (divisible? x 7)))
                 integers))
\end{scheme}

\noindent
Then we can find integers not divisible by 7 simply by accessing elements of
this stream:

\begin{scheme}
(stream-ref no-sevens 100)
~\textit{117}~
\end{scheme}

\noindent
In analogy with \code{integers}, we can define the infinite stream of Fibonacci
numbers:

\begin{scheme}
(define (fibgen a b) (cons-stream a (fibgen b (+ a b))))
(define fibs (fibgen 0 1))
\end{scheme}

\noindent
\code{Fibs} is a pair whose \code{car} is 0 and whose \code{cdr} is a promise
to evaluate \code{(fibgen 1 1)}.  When we evaluate this delayed \code{(fibgen 1
1)}, it will produce a pair whose \code{car} is 1 and whose \code{cdr} is a
promise to evaluate \code{(fibgen 1 2)}, and so on.

For a look at a more exciting infinite stream, we can generalize the
\code{no\-/sevens} example to construct the infinite stream of prime numbers,
using a method known as the \newterm{sieve of
Eratosthenes}.\footnote{Eratosthenes, a third-century \acronym{B.C.}
Alexandrian Greek philosopher, is famous for giving the first accurate estimate
of the circumference of the Earth, which he computed by observing shadows cast
at noon on the day of the summer solstice.  Eratosthenes's sieve method,
although ancient, has formed the basis for special-purpose hardware ``sieves''
that, until recently, were the most powerful tools in existence for locating
large primes.  Since the 70s, however, these methods have been superseded by
outgrowths of the probabilistic techniques discussed in \link{Section 1.2.6}.}
We start with the integers beginning with 2, which is the first prime.  To get
the rest of the primes, we start by filtering the multiples of 2 from the rest
of the integers.  This leaves a stream beginning with 3, which is the next
prime.  Now we filter the multiples of 3 from the rest of this stream.  This
leaves a stream beginning with 5, which is the next prime, and so on.  In other
words, we construct the primes by a sieving process, described as follows: To
sieve a stream \code{S}, form a stream whose first element is the first element
of \code{S} and the rest of which is obtained by filtering all multiples of the
first element of \code{S} out of the rest of \code{S} and sieving the
result. This process is readily described in terms of stream operations:

\begin{scheme}
(define (sieve stream)
  (cons-stream
   (stream-car stream)
   (sieve (stream-filter
           (lambda (x)
             (not (divisible? x (stream-car stream))))
           (stream-cdr stream)))))
(define primes (sieve (integers-starting-from 2)))
\end{scheme}

\noindent
Now to find a particular prime we need only ask for it:

\begin{scheme}
(stream-ref primes 50)
~\textit{233}~
\end{scheme}

\noindent
It is interesting to contemplate the signal-processing system set up by
\code{sieve}, shown in the ``Henderson diagram'' in 
\link{Figure 3.31}.\footnote{We have named these figures after Peter Henderson, who was the
first person to show us diagrams of this sort as a way of thinking about stream
processing.  Each solid line represents a stream of values being transmitted.
The dashed line from the \code{car} to the \code{cons} and the \code{filter}
indicates that this is a single value rather than a stream.}  The input stream
feeds into an ``un\code{cons}er'' that separates the first element of the
stream from the rest of the stream.  The first element is used to construct a
divisibility filter, through which the rest is passed, and the output of the
filter is fed to another sieve box.  Then the original first element is
\code{cons}ed onto the output of the internal sieve to form the output stream.
Thus, not only is the stream infinite, but the signal processor is also
infinite, because the sieve contains a sieve within it.

\begin{figure}[tb]
\phantomsection\label{Figure 3.31}
\centering
\begin{comment}
\heading{Figure 3.31:} The prime sieve viewed as a signal-processing system.

\begin{example}
  +---------------------------------------------------------------+
  | sieve                                                         |
  |                                                               |
  |        __/|                                        |\__       |
  |     __/car|........................................|   \__    |
  |   _/      |           :                            |      \_  |
----><_       |           V                            |  cons _>---->
  |    \__    |    +------------+    +------------+    |    __/   |
  |       \cdr|--->| filter:    |    | sieve      |--->| __/      |
  |          \|    |            |--->|            |    |/         |
  |                | not        |    |            |               |
  |                | divisible? |    |            |               |
  |                +------------+    +------------+               |
  +---------------------------------------------------------------+
\end{example}
\end{comment}
\includegraphics[width=111mm]{fig/chap3/Fig3.31.pdf}
\begin{quote}
\heading{Figure 3.31:} The prime sieve viewed as a signal-processing system.
\end{quote}
\end{figure}

\subsubsection*{Defining streams implicitly}

The \code{integers} and \code{fibs} streams above were defined by specifying
``generating'' procedures that explicitly compute the stream elements one by
one. An alternative way to specify streams is to take advantage of delayed
evaluation to define streams implicitly.  For example, the following expression
defines the stream \code{ones} to be an infinite stream of ones:

\begin{scheme}
(define ones (cons-stream 1 ones))
\end{scheme}

\noindent
This works much like the definition of a recursive procedure: \code{ones} is a
pair whose \code{car} is 1 and whose \code{cdr} is a promise to evaluate
\code{ones}.  Evaluating the \code{cdr} gives us again a 1 and a promise to
evaluate \code{ones}, and so on.

We can do more interesting things by manipulating streams with operations such
as \code{add\-/streams}, which produces the elementwise sum of two given
streams:\footnote{This uses the generalized version of \code{stream\-/map} from
\link{Exercise 3.50}.}

\begin{scheme}
(define (add-streams s1 s2) (stream-map + s1 s2))
\end{scheme}

\noindent
Now we can define the integers as follows:

\begin{scheme}
(define integers 
  (cons-stream 1 (add-streams ones integers)))
\end{scheme}

\noindent
This defines \code{integers} to be a stream whose first element is 1 and the
rest of which is the sum of \code{ones} and \code{integers}.  Thus, the second
element of \code{integers} is 1 plus the first element of \code{integers}, or
2; the third element of \code{integers} is 1 plus the second element of
\code{integers}, or 3; and so on.  This definition works because, at any point,
enough of the \code{integers} stream has been generated so that we can feed it
back into the definition to produce the next integer.

We can define the Fibonacci numbers in the same style:

\begin{scheme}
(define fibs 
  (cons-stream
   0
   (cons-stream 1 (add-streams (stream-cdr fibs) fibs))))
\end{scheme}

\noindent
This definition says that \code{fibs} is a stream beginning with 0 and 1, such
that the rest of the stream can be generated by adding \code{fibs} to itself
shifted by one place:

\begin{scheme}
      1  1  2  3  5  8   13  21  ~\( \dots \)~  =  ~\code{(stream\-/cdr fibs)}~
      0  1  1  2  3  5   8   13  ~\( \dots \)~  =  ~\code{fibs}~
0  1  1  2  3  5  8  13  21  34  ~\( \dots \)~  =  ~\code{fibs}~
\end{scheme}

\noindent
\code{Scale\-/stream} is another useful procedure in formulating such stream
definitions.  This multiplies each item in a stream by a given constant:

\begin{scheme}
(define (scale-stream stream factor)
  (stream-map (lambda (x) (* x factor))
              stream))
\end{scheme}

\noindent
For example,

\begin{scheme}
(define double (cons-stream 1 (scale-stream double 2)))
\end{scheme}

\noindent
produces the stream of powers of 2: 1, 2, 4, 8, 16, 32, \( \dots \).

An alternate definition of the stream of primes can be given by starting with
the integers and filtering them by testing for primality.  We will need the
first prime, 2, to get started:

\begin{scheme}
(define primes
  (cons-stream
   2
   (stream-filter prime? (integers-starting-from 3))))
\end{scheme}

\noindent
This definition is not so straightforward as it appears, because we will test
whether a number \( n \) is prime by checking whether \( n \) is divisible by a
prime (not by just any integer) less than or equal to \( \sqrt{n} \):

\begin{scheme}
(define (prime? n)
  (define (iter ps)
    (cond ((> (square (stream-car ps)) n) true)
          ((divisible? n (stream-car ps)) false)
          (else (iter (stream-cdr ps)))))
  (iter primes))
\end{scheme}

\noindent
This is a recursive definition, since \code{primes} is defined in terms of the
\code{prime?} predicate, which itself uses the \code{primes} stream.  The
reason this procedure works is that, at any point, enough of the \code{primes}
stream has been generated to test the primality of the numbers we need to check
next.  That is, for every \( n \) we test for primality, either \( n \) is not
prime (in which case there is a prime already generated that divides it) or
\( n \) is prime (in which case there is a prime already generated---i.e., a
prime less than \( n \)---that is greater than
\( \sqrt{n} \)).\footnote{This last point is very subtle and relies on
the fact that \( p_{n+1} \le p_n^2 \).  (Here, \( p_k \) denotes the
\( k^{\mathrm{th}} \) prime.)  Estimates such as these are very difficult to establish.  The
ancient proof by Euclid that there are an infinite number of primes shows that
\( p_{n+1} \le p_1 p_2 \cdots p_n + 1 \), and no substantially
better result was proved until 1851, when the Russian mathematician
P. L. Chebyshev established that \( p_{n+1} \le 2p_n \) for all \( n \).
This result, originally conjectured in 1845, is known as \newterm{Bertrand's
hypothesis}.  A proof can be found in section 22.3 of \link{Hardy and Wright 1960}.}

\begin{quote}
\heading{\phantomsection\label{Exercise 3.53}Exercise 3.53:} Without running the program,
describe the elements of the stream defined by

\begin{scheme}
(define s (cons-stream 1 (add-streams s s)))
\end{scheme}
\end{quote}

\begin{quote}
\heading{\phantomsection\label{Exercise 3.54}Exercise 3.54:} Define a procedure
\code{mul\-/streams}, analogous to \code{add\-/streams}, that produces the
elementwise product of its two input streams.  Use this together with the
stream of \code{integers} to complete the following definition of the stream
whose \( n^{\mathrm{th}} \) element (counting from 0) is \( n + 1 \) factorial:

\begin{scheme}
(define factorials 
  (cons-stream 1 (mul-streams ~\( \dark \langle \)~??~\( \dark \rangle \)~ ~\( \dark \langle \)~??~\( \dark \rangle \)~)))
\end{scheme}
% \noindent
% \code{(define factorials (cons\-/stream 1 (mul\-/streams}\( \kern0.7ex\langle \)\code{?}\( \rangle \)\( \kern0.7ex\langle \)\code{?}\( \rangle \)\code{)))}
\end{quote}

\begin{quote}
\heading{\phantomsection\label{Exercise 3.55}Exercise 3.55:} Define a procedure
\code{partial\-/sums} that takes as argument a stream \( S \) and returns the
stream whose elements are \( S_0 \), \( S_0 + S_1 \), \( S_0 + S_1 + S_2, \dots \).  
For example, \code{(partial\-/sums integers)} should be the
stream 1, 3, 6, 10, 15, \( \dots \).
\end{quote}

\begin{quote}
\heading{\phantomsection\label{Exercise 3.56}Exercise 3.56:} A famous problem, first raised by
R. Hamming, is to enumerate, in ascending order with no repetitions, all
positive integers with no prime factors other than 2, 3, or 5.  One obvious way
to do this is to simply test each integer in turn to see whether it has any
factors other than 2, 3, and 5.  But this is very inefficient, since, as the
integers get larger, fewer and fewer of them fit the requirement.  As an
alternative, let us call the required stream of numbers \code{S} and notice the
following facts about it.

\begin{itemize}

\item
\code{S} begins with 1.

\item
The elements of \code{(scale\-/stream S 2)} are also
elements of \code{S}.

\item
The same is true for \code{(scale\-/stream S 3)}
and \code{(scale\-/stream 5 S)}.

\item
These are all the elements of \code{S}.

\end{itemize}

Now all we have to do is combine elements from these sources.  For this we
define a procedure \code{merge} that combines two ordered streams into one
ordered result stream, eliminating repetitions:

\begin{scheme}
(define (merge s1 s2)
  (cond ((stream-null? s1) s2)
        ((stream-null? s2) s1)
        (else
         (let ((s1car (stream-car s1))
               (s2car (stream-car s2)))
           (cond ((< s1car s2car)
                  (cons-stream 
                   s1car 
                   (merge (stream-cdr s1) s2)))
                 ((> s1car s2car)
                  (cons-stream 
                   s2car 
                   (merge s1 (stream-cdr s2))))
                 (else
                  (cons-stream 
                   s1car
                   (merge (stream-cdr s1)
                          (stream-cdr s2)))))))))
\end{scheme}

Then the required stream may be constructed with \code{merge}, as follows:

\begin{scheme}
(define S (cons-stream 1 (merge ~\( \dark \langle \)~??~\( \dark \rangle \)~ ~\( \dark \langle \)~??~\( \dark \rangle \)~)))
\end{scheme}

Fill in the missing expressions in the places marked \( \langle \)??\( \kern0.08em\rangle \) above.
\end{quote}

\begin{quote}
\heading{\phantomsection\label{Exercise 3.57}Exercise 3.57:} How many additions are performed
when we compute the \( n^{\mathrm{th}} \) Fibonacci number using the definition of
\code{fibs} based on the \code{add\-/streams} procedure?  Show that the number of
additions would be exponentially greater if we had implemented \code{(delay
\( \langle \)\var{exp}\( \rangle \))} simply as \code{(lambda () \( \langle \)\var{exp}\( \rangle \))}, without using the
optimization provided by the \code{memo\-/proc} procedure described in 
\link{Section 3.5.1}.\footnote{This exercise shows how call-by-need is closely related
to ordinary memoization as described in \link{Exercise 3.27}.  In that exercise,
we used assignment to explicitly construct a local table.  Our call-by-need
stream optimization effectively constructs such a table automatically, storing
values in the previously forced parts of the stream.}
\end{quote}

\begin{quote}
\heading{\phantomsection\label{Exercise 3.58}Exercise 3.58:} Give an interpretation of the
stream computed by the following procedure:

\begin{scheme}
(define (expand num den radix)
  (cons-stream
   (quotient (* num radix) den)
   (expand (remainder (* num radix) den) den radix)))
\end{scheme}

(\code{Quotient} is a primitive that returns the integer quotient of two
integers.)  What are the successive elements produced by \code{(expand 1 7
10)}?  What is produced by \code{(expand 3 8 10)}?
\end{quote}

\begin{quote}
\heading{\phantomsection\label{Exercise 3.59}Exercise 3.59:} In \link{Section 2.5.3} we saw how
to implement a polynomial arithmetic system representing polynomials as lists
of terms.  In a similar way, we can work with \newterm{power series}, such as
\begin{comment}

\begin{example}
               x^2     x^3       x^4
e^x = 1 + x + ----- + ----- + --------- + ...
                2     3 * 2   4 * 3 * 2

             x^2       x^4
cos x = 1 - ----- + --------- - ...
              2     4 * 3 * 2

             x^3         x^5
sin x = x - ----- + ------------- - ...
            3 * 2   5 * 4 * 3 * 2
\end{example}

\end{comment}
\begin{displaymath}
e^x = 1 + x + \displaystyle\frac{x^2}{2} + \displaystyle\frac{x^3}{3 \cdot 2} + \displaystyle\frac{x^4}{4 \cdot 3 \cdot 2} + \dots, 
\end{displaymath}
\begin{displaymath}
\cos x = 1 - \displaystyle\frac{x^2}{2} + \displaystyle\frac{x^4}{4 \cdot 3 \cdot 2} - \dots, 
\end{displaymath}
\begin{displaymath}
\sin x = x - \displaystyle\frac{x^3}{3 \cdot 2} + \displaystyle\frac{x^5}{5 \cdot 4 \cdot 3 \cdot 2} - \dots 
\end{displaymath}
\noindent
represented as infinite streams.  We will represent the series \( a_0 +
a_1 x + a_2 x^2 + a_3 x^3 + \dots \) as the stream whose
elements are the coefficients \( a_0 \), \( a_1 \), \( a_2 \), \( a_3 \), \( \dots \).

\begin{enumerate}[a.]

\item
The integral of the series \( a_0 + a_1 x + a_2 x^2 + a_3 x^3 + \dots \) is the series
\begin{comment}

\begin{example}
             1             1             1
c + a_0 x + --- a_1 x^2 + --- a_2 x^3 + --- a_3 x^4 + ...
             2             3             4
\end{example}

\end{comment}
\begin{displaymath}
 c + a_0 x + {1\over2} a_1 x^2 + {1\over3} a_2 x^3 + {1\over4} a_3 x^4 + \dots, 
\end{displaymath}
\noindent
where \( c \) is any constant.  Define a procedure \code{integrate\-/series} that
takes as input a stream \( a_0 \), \( a_1 \), \( a_2 \), \( \dots \) representing a power
series and returns the stream \( a_0 \), \( {1\over2}a_1 \), \( {1\over3}a_2 \), \( \dots \) of
coefficients of the non-constant terms of the integral of the series.  (Since
the result has no constant term, it doesn't represent a power series; when we
use \code{integrate\-/series}, we will \code{cons} on the appropriate constant.)

\item
The function \( x \mapsto e^x \) is its own derivative.  This implies that
\( e^x \) and the integral of \( e^x \) are the same series, except for the
constant term, which is \( e^0 = 1 \).  Accordingly, we can generate the series
for \( e^x \) as

\begin{scheme}
(define exp-series
  (cons-stream 1 (integrate-series exp-series)))
\end{scheme}

Show how to generate the series for sine and cosine, starting from the facts
that the derivative of sine is cosine and the derivative of cosine is the
negative of sine:

\begin{scheme}
(define cosine-series (cons-stream 1 ~\( \dark \langle \)~??~\( \dark \rangle \)~))
(define sine-series (cons-stream 0 ~\( \dark \langle \)~??~\( \dark \rangle \)~))
\end{scheme}
\end{enumerate}
\end{quote}

\begin{quote}
\heading{\phantomsection\label{Exercise 3.60}Exercise 3.60:} With power series represented as
streams of coefficients as in \link{Exercise 3.59}, adding series is implemented
by \code{add\-/streams}.  Complete the definition of the following procedure for
multiplying series:

\begin{scheme}
(define (mul-series s1 s2)
  (cons-stream ~\( \dark \langle \)~??~\( \dark \rangle \)~ (add-streams ~\( \dark \langle \)~??~\( \dark \rangle \)~ ~\( \dark \langle \)~??~\( \dark \rangle \)~)))
\end{scheme}

You can test your procedure by verifying that \mbox{\( \sin^2\!x + \cos^2\!x = 1 \)}, 
using the series from \link{Exercise 3.59}.
\end{quote}

\begin{quote}
\heading{\phantomsection\label{Exercise 3.61}Exercise 3.61:} Let \( S \) be a power series
(\link{Exercise 3.59}) whose constant term is 1.  Suppose we want to find the
power series \( 1 / S \), that is, the series \( X \) such that \( SX = 1 \).
Write \( S = 1 + S_R \) where \( S_R \) is the part of \( S \) after the
constant term.  Then we can solve for \( X \) as follows:
\begin{comment}

\begin{example}
        S * X = 1
(1 + S_R) * X = 1
  X + S_R * X = 1
            X = 1 - S_R * X
\end{example}

\end{comment}
\begin{displaymath}
%  \eqalign{
% 	        S \cdot X 	&= 1, \cr
% 	(1 + S_R) \cdot X 	&= 1, \cr
% 	  X + S_R \cdot X 	&= 1, \cr
% 	            	X 	&= 1 - S_R \cdot X. \cr
% } 
\begin{array}{r@{{}={}}l}
	        S \cdot X 	& 1, \\
	(1 + S_R) \cdot X 	& 1, \\
	  X + S_R \cdot X 	& 1, \\
	            	X 	& 1 - S_R \cdot X. 
\end{array}
\end{displaymath}
In other words, \( X \) is the power series whose constant term is 1 and whose
higher-order terms are given by the negative of \( S_R \) times \( X \).  Use
this idea to write a procedure \code{invert\-/unit\-/series} that computes \( 1 / S \)
for a power series \( S \) with constant term 1.  You will need to use
\code{mul\-/series} from \link{Exercise 3.60}.
\end{quote}

\begin{quote}
\heading{\phantomsection\label{Exercise 3.62}Exercise 3.62:} Use the results of \link{Exercise 3.60} 
and \link{Exercise 3.61} to define a procedure \code{div\-/series} that
divides two power series.  \code{Div\-/series} should work for any two series,
provided that the denominator series begins with a nonzero constant term.  (If
the denominator has a zero constant term, then \code{div\-/series} should signal
an error.)  Show how to use \code{div\-/series} together with the result of
\link{Exercise 3.59} to generate the power series for tangent.
\end{quote}

\subsection{Exploiting the Stream Paradigm}
\label{Section 3.5.3}

Streams with delayed evaluation can be a powerful modeling tool, providing many
of the benefits of local state and assignment.  Moreover, they avoid some of
the theoretical tangles that accompany the introduction of assignment into a
programming language.

The stream approach can be illuminating because it allows us to build systems
with different module boundaries than systems organized around assignment to
state variables.  For example, we can think of an entire time series (or
signal) as a focus of interest, rather than the values of the state variables
at individual moments.  This makes it convenient to combine and compare
components of state from different moments.

\subsubsection*{Formulating iterations as stream processes}

In \link{Section 1.2.1}, we introduced iterative processes, which proceed by
updating state variables.  We know now that we can represent state as a
``timeless'' stream of values rather than as a set of variables to be updated.
Let's adopt this perspective in revisiting the square-root procedure from
\link{Section 1.1.7}.  Recall that the idea is to generate a sequence of better
and better guesses for the square root of \( x \) by applying over and over again
the procedure that improves guesses:

\begin{scheme}
(define (sqrt-improve guess x)
  (average guess (/ x guess)))
\end{scheme}

\noindent
In our original \code{sqrt} procedure, we made these guesses be the successive
values of a state variable. Instead we can generate the infinite stream of
guesses, starting with an initial guess of 1:\footnote{We can't use \code{let}
to bind the local variable \code{guesses}, because the value of \code{guesses}
depends on \code{guesses} itself.  \link{Exercise 3.63} addresses why we want a
local variable here.}

\begin{scheme}
(define (sqrt-stream x)
  (define guesses
    (cons-stream
     1.0
     (stream-map (lambda (guess) (sqrt-improve guess x))
                 guesses)))
  guesses)

(display-stream (sqrt-stream 2))
~\textit{1.}~
~\textit{1.5}~
~\textit{1.4166666666666665}~
~\textit{1.4142156862745097}~
~\textit{1.4142135623746899}~
~\( \dots \)~
\end{scheme}

\noindent
We can generate more and more terms of the stream to get better and better
guesses.  If we like, we can write a procedure that keeps generating terms
until the answer is good enough.  (See \link{Exercise 3.64}.)

Another iteration that we can treat in the same way is to generate an
approximation to \( \pi \), based upon the alternating series that we saw in
\link{Section 1.3.1}:
\begin{comment}

\begin{example}
[pi]        1     1     1
---- = 1 - --- + --- - --- + ...
  4         3     5     7
\end{example}

\end{comment}
\begin{displaymath}
 {\pi\over4} = 1 - {1\over3} + {1\over5} - {1\over7} + \dots. 
\end{displaymath}
We first generate the stream of summands of the series (the reciprocals of the
odd integers, with alternating signs).  Then we take the stream of sums of more
and more terms (using the \code{partial\-/sums} procedure of 
\link{Exercise 3.55}) and scale the result by 4:

\begin{scheme}
(define (pi-summands n)
  (cons-stream (/ 1.0 n)
               (stream-map - (pi-summands (+ n 2)))))
(define pi-stream
  (scale-stream (partial-sums (pi-summands 1)) 4))

(display-stream pi-stream)
~\textit{4.}~
~\textit{2.666666666666667}~
~\textit{3.466666666666667}~
~\textit{2.8952380952380956}~
~\textit{3.3396825396825403}~
~\textit{2.9760461760461765}~
~\textit{3.2837384837384844}~
~\textit{3.017071817071818}~
~\( \dots \)~
\end{scheme}

\noindent
This gives us a stream of better and better approximations to \( \pi \),
although the approximations converge rather slowly.  Eight terms of the
sequence bound the value of \( \pi \) between 3.284 and 3.017.

So far, our use of the stream of states approach is not much different from
updating state variables.  But streams give us an opportunity to do some
interesting tricks.  For example, we can transform a stream with a
\newterm{sequence accelerator} that converts a sequence of approximations to a
new sequence that converges to the same value as the original, only faster.

One such accelerator, due to the eighteenth-century Swiss mathematician
Leonhard Euler, works well with sequences that are partial sums of alternating
series (series of terms with alternating signs).  In Euler's technique, if
\( S_n \) is the \( n^{\mathrm{th}} \) term of the original sum sequence, then the
accelerated sequence has terms
\begin{comment}

\begin{example}
             (S_(n+1) - S_n)^2
S_(n+1) - ------------------------
          S_(n-1) - 2S_n + S_(n+1)
\end{example}

\end{comment}
\begin{displaymath}
 S_{n+1} - {(S_{n+1} - S_n)^2 \over S_{n-1} - 2S_n + S_{n+1}}\,. 
\end{displaymath}
Thus, if the original sequence is represented as a stream of values, the
transformed sequence is given by

\begin{scheme}
(define (euler-transform s)
  (let ((s0 (stream-ref s 0))     ~\textrm{; \( S_{n-1} \)}~
        (s1 (stream-ref s 1))     ~\textrm{; \( S_n \)}~
        (s2 (stream-ref s 2)))    ~\textrm{; \( S_{n+1} \)}~
    (cons-stream (- s2 (/ (square (- s2 s1))
                          (+ s0 (* -2 s1) s2)))
                 (euler-transform (stream-cdr s)))))
\end{scheme}

\noindent
We can demonstrate Euler acceleration with our sequence of approximations to
\( \pi \):

\begin{scheme}
(display-stream (euler-transform pi-stream))
~\textit{3.166666666666667}~
~\textit{3.1333333333333337}~
~\textit{3.1452380952380956}~
~\textit{3.13968253968254}~
~\textit{3.1427128427128435}~
~\textit{3.1408813408813416}~
~\textit{3.142071817071818}~
~\textit{3.1412548236077655}~
~\( \dots \)~
\end{scheme}

\noindent
Even better, we can accelerate the accelerated sequence, and recursively
accelerate that, and so on.  Namely, we create a stream of streams (a structure
we'll call a \newterm{tableau}) in which each stream is the transform of the
preceding one:

\begin{scheme}
(define (make-tableau transform s)
  (cons-stream s (make-tableau transform (transform s))))
\end{scheme}

\noindent
The tableau has the form
\begin{comment}

\begin{example}
s_00   s_01   s_02   s_03   s_04   ...
       s_10   s_11   s_12   s_13   ...
              s_20   s_21   s_22   ...
                            ...
\end{example}

\end{comment}
\begin{displaymath}
 \vbox{
\offinterlineskip
\halign{
\strut 	\hfil \  #\  \hfil & 
	\hfil \  #\  \hfil &
	\hfil \  #\  \hfil &
	\hfil \  #\  \hfil &
	\hfil \  #\  \hfil &
	\hfil \  #\  \hfil \cr

$ s_{00} $ 	&  $ s_{01} $ 	&  $ s_{02} $ 	&  $ s_{03} $ 	&  $ s_{04} $ 	&  $ \dots $ \cr
		&  $ s_{10} $ 	&  $ s_{11} $ 	&  $ s_{12} $ 	&  $ s_{13} $ 	&  $ \dots $ \cr
		& 		&  $ s_{20} $ 	&  $ s_{21} $ 	&  $ s_{22} $ 	&  $ \dots $ \cr
		& 		& 		&  $ \dots $ 	& 		&  \cr }
} 
\end{displaymath}
Finally, we form a sequence by taking the first term in each row of the
tableau:

\begin{scheme}
(define (accelerated-sequence transform s)
  (stream-map stream-car (make-tableau transform s)))
\end{scheme}

\noindent
We can demonstrate this kind of ``super-acceleration'' of the \( \pi \)
sequence:

\begin{scheme}
(display-stream
 (accelerated-sequence euler-transform pi-stream))
~\textit{4.}~
~\textit{3.166666666666667}~
~\textit{3.142105263157895}~
~\textit{3.141599357319005}~
~\textit{3.1415927140337785}~
~\textit{3.1415926539752927}~
~\textit{3.1415926535911765}~
~\textit{3.141592653589778}~
~\( \dots \)~
\end{scheme}

\noindent
The result is impressive.  Taking eight terms of the sequence yields the
correct value of \( \pi \) to 14 decimal places.  If we had used only the
original \( \pi \) sequence, we would need to compute on the order of \( 10^{13} \)
terms (i.e., expanding the series far enough so that the individual terms are
less than \( 10^{-13} \)) to get that much accuracy!

We could have implemented these acceleration techniques without using streams.
But the stream formulation is particularly elegant and convenient because the
entire sequence of states is available to us as a data structure that can be
manipulated with a uniform set of operations.

\begin{quote}
\heading{\phantomsection\label{Exercise 3.63}Exercise 3.63:} Louis Reasoner asks why the
\code{sqrt\-/stream} procedure was not written in the following more
straightforward way, without the local variable \code{guesses}:

\begin{scheme}
(define (sqrt-stream x)
  (cons-stream 1.0 (stream-map
                    (lambda (guess)
                      (sqrt-improve guess x))
                    (sqrt-stream x))))
\end{scheme}

Alyssa P. Hacker replies that this version of the procedure is considerably
less efficient because it performs redundant computation.  Explain Alyssa's
answer.  Would the two versions still differ in efficiency if our
implementation of \code{delay} used only \code{(lambda () \( \langle \)\var{exp}\( \rangle \))} without
using the optimization provided by \code{memo\-/proc} (\link{Section 3.5.1})?
\end{quote}

\begin{quote}
\heading{\phantomsection\label{Exercise 3.64}Exercise 3.64:} Write a procedure
\code{stream\-/limit} that takes as arguments a stream and a number (the
tolerance).  It should examine the stream until it finds two successive
elements that differ in absolute value by less than the tolerance, and return
the second of the two elements.  Using this, we could compute square roots up
to a given tolerance by

\begin{scheme}
(define (sqrt x tolerance)
  (stream-limit (sqrt-stream x) tolerance))
\end{scheme}
\end{quote}

\begin{quote}
\heading{\phantomsection\label{Exercise 3.65}Exercise 3.65:} Use the series
\begin{comment}

\begin{example}
            1     1     1
ln 2 = 1 - --- + --- - --- + ...
            2     3     4
\end{example}

\end{comment}
\begin{displaymath}
 \ln 2 = 1 - {1\over2} + {1\over3} - {1\over4} + \dots 
\end{displaymath}
\noindent
to compute three sequences of approximations to the natural logarithm of 2, in
the same way we did above for \( \pi \).  How rapidly do these sequences
converge?
\end{quote}

\subsubsection*{Infinite streams of pairs}

In \link{Section 2.2.3}, we saw how the sequence paradigm handles traditional
nested loops as processes defined on sequences of pairs.  If we generalize this
technique to infinite streams, then we can write programs that are not easily
represented as loops, because the ``looping'' must range over an infinite set.

For example, suppose we want to generalize the \code{prime\-/sum\-/pairs} procedure
of \link{Section 2.2.3} to produce the stream of pairs of \emph{all} integers
\( (i, j) \) with \( i \le j \) such that \( i + j \) is prime.  If
\code{int\-/pairs} is the sequence of all pairs of integers \( (i, j) \) with
\( i \le j \), then our required stream is simply\footnote{As in 
\link{Section 2.2.3}, we represent a pair of integers as a list rather than a Lisp
pair.}

\begin{scheme}
(stream-filter
 (lambda (pair) (prime? (+ (car pair) (cadr pair))))
 int-pairs)
\end{scheme}

\noindent
Our problem, then, is to produce the stream \code{int\-/pairs}.  More generally,
suppose we have two streams \( S = (S_i) \) and \( T = (T_j) \),
and imagine the infinite rectangular array
\begin{comment}

\begin{example}
(S_0, T_0)  (S_0, T_1)  (S_0, T_2)  ...
(S_1, T_0)  (S_1, T_1)  (S_1, T_2)  ...
(S_2, T_0)  (S_2, T_1)  (S_2, T_2)  ...
   ...
\end{example}

\end{comment}
\begin{displaymath}
 \vbox{
\offinterlineskip
\halign{
\strut 	\hfil \  #\  \hfil & 
	\hfil \  #\  \hfil &
	\hfil \  #\  \hfil &
	\hfil \  #\  \hfil \cr

	$ (S_0, T_0) $ & $ (S_0, T_1) $ & $ (S_0, T_2) $ & $ \dots $ \cr
	$ (S_1, T_0) $ & $ (S_1, T_1) $ & $ (S_1, T_2) $ & $ \dots $ \cr
	$ (S_2, T_0) $ & $ (S_2, T_1) $ & $ (S_2, T_2) $ & $ \dots $ \cr
	$ \dots $ & & & \cr }
} 
\end{displaymath}
We wish to generate a stream that contains all the pairs in the array that lie
on or above the diagonal, i.e., the pairs
\begin{comment}

\begin{example}
(S_0, T_0)  (S_0, T_1)  (S_0, T_2)  ...
            (S_1, T_1)  (S_1, T_2)  ...
                        (S_2, T_2)  ...
                                    ...
\end{example}

\end{comment}
\begin{displaymath}
 \vbox{
\offinterlineskip
\halign{
\strut 	\hfil \  #\  \hfil & 
	\hfil \  #\  \hfil &
	\hfil \  #\  \hfil &
	\hfil \  #\  \hfil \cr

$ (S_0, T_0) $ 	& $ (S_0, T_1) $ 	& $ (S_0, T_2) $ 	& $ \dots $ \cr
		& $ (S_1, T_1) $ 	& $ (S_1, T_2) $ 	& $ \dots $ \cr
		& 			& $ (S_2, T_2) $ 	& $ \dots $ \cr
		& 			& 			& $ \dots $ \cr }
} 
\end{displaymath}
\noindent
(If we take both \( S \) and \( T \) to be the stream of integers, then this will
be our desired stream \code{int\-/pairs}.)

Call the general stream of pairs \code{(pairs S T)}, and consider it to be
composed of three parts: the pair \( (S_0, T_0) \), the rest of the pairs in
the first row, and the remaining pairs:\footnote{See \link{Exercise 3.68} for
some insight into why we chose this decomposition.}
\begin{comment}

\begin{example}
(S_0, T_0) | (S_0, T_1)  (S_0, T_2)  ...
-----------+-----------------------------
           | (S_1, T_1)  (S_1, T_2)  ...
           |             (S_2, T_2)  ...
           |                         ...
\end{example}

\end{comment}
\begin{displaymath}
 \vbox{
\offinterlineskip
\halign{
\strut 	\hfil \  #\  \hfil & \vrule
	\hfil \  #\  \hfil &
	\hfil \  #\  \hfil &
	\hfil \  #\  \hfil \cr

$ (S_0, T_0) $ 	& $ (S_0, T_1) $ 	& $ (S_0, T_2) $ 	& $ \dots $ \cr
\noalign{\hrule}
		& $ (S_1, T_1) $ 	& $ (S_1, T_2) $ 	& $ \dots $ \cr
		& 			& $ (S_2, T_2) $ 	& $ \dots $ \cr
		& 			& 			& $ \dots $ \cr }
} 
\end{displaymath}
Observe that the third piece in this decomposition (pairs that are not in the
first row) is (recursively) the pairs formed from \code{(stream\-/cdr S)} and
\code{(stream\-/cdr T)}.  Also note that the second piece (the rest of the first
row) is

\begin{scheme}
(stream-map (lambda (x) (list (stream-car s) x))
            (stream-cdr t))
\end{scheme}

\noindent
Thus we can form our stream of pairs as follows:

\begin{scheme}
(define (pairs s t)
  (cons-stream
   (list (stream-car s) (stream-car t))
   (~\( \dark \langle \)~~\var{\dark combine-in-some-way}~~\( \dark \rangle \)~
     (stream-map (lambda (x) (list (stream-car s) x))
                 (stream-cdr t))
     (pairs (stream-cdr s) (stream-cdr t)))))
\end{scheme}

\noindent
In order to complete the procedure, we must choose some way to combine the two
inner streams.  One idea is to use the stream analog of the \code{append}
procedure from \link{Section 2.2.1}:

\begin{scheme}
(define (stream-append s1 s2)
  (if (stream-null? s1)
      s2
      (cons-stream (stream-car s1)
                   (stream-append (stream-cdr s1) s2))))
\end{scheme}

\noindent
This is unsuitable for infinite streams, however, because it takes all the
elements from the first stream before incorporating the second stream.  In
particular, if we try to generate all pairs of positive integers using

\begin{scheme}
(pairs integers integers)
\end{scheme}

\noindent
our stream of results will first try to run through all pairs with the first
integer equal to 1, and hence will never produce pairs with any other value of
the first integer.

To handle infinite streams, we need to devise an order of combination that
ensures that every element will eventually be reached if we let our program run
long enough.  An elegant way to accomplish this is with the following
\code{interleave} procedure:\footnote{The precise statement of the required
property on the order of combination is as follows: There should be a function
\( f \) of two arguments such that the pair corresponding to element \( i \) of the
first stream and element \( j \) of the second stream will appear as element
number \( f(i, j) \) of the output stream.  The trick of using
\code{interleave} to accomplish this was shown to us by David Turner, who
employed it in the language KRC (\link{Turner 1981}).}

\begin{scheme}
(define (interleave s1 s2)
  (if (stream-null? s1)
      s2
      (cons-stream (stream-car s1)
                   (interleave s2 (stream-cdr s1)))))
\end{scheme}

\noindent
Since \code{interleave} takes elements alternately from the two streams, every
element of the second stream will eventually find its way into the interleaved
stream, even if the first stream is infinite.

We can thus generate the required stream of pairs as

\begin{scheme}
(define (pairs s t)
  (cons-stream
   (list (stream-car s) (stream-car t))
   (interleave
    (stream-map (lambda (x) (list (stream-car s) x))
                (stream-cdr t))
    (pairs (stream-cdr s) (stream-cdr t)))))
\end{scheme}

\begin{quote}
\heading{\phantomsection\label{Exercise 3.66}Exercise 3.66:} Examine the stream \code{(pairs
integers integers)}. Can you make any general comments about the order in which
the pairs are placed into the stream? For example, approximately how many pairs precede
the pair (1, 100)?  the pair (99, 100)? the pair (100, 100)? (If you can make
precise mathematical statements here, all the better. But feel free to give
more qualitative answers if you find yourself getting bogged down.)
\end{quote}

\begin{quote}
\heading{\phantomsection\label{Exercise 3.67}Exercise 3.67:} Modify the \code{pairs} procedure
so that \code{(pairs integers integers)} will produce the stream of \emph{all}
pairs of integers \( (i, j) \) (without the condition \( i \le j \)).  Hint:
You will need to mix in an additional stream.
\end{quote}

\begin{quote}
\heading{\phantomsection\label{Exercise 3.68}Exercise 3.68:} Louis Reasoner thinks that
building a stream of pairs from three parts is unnecessarily complicated.
Instead of separating the pair \( (S_0, T_0) \) from the rest of the pairs in
the first row, he proposes to work with the whole first row, as follows:

\begin{scheme}
(define (pairs s t)
  (interleave
   (stream-map (lambda (x) (list (stream-car s) x))
               t)
   (pairs (stream-cdr s) (stream-cdr t))))
\end{scheme}

Does this work?  Consider what happens if we evaluate \code{(pairs integers
integers)} using Louis's definition of \code{pairs}.
\end{quote}

\begin{quote}
\heading{\phantomsection\label{Exercise 3.69}Exercise 3.69:} Write a procedure \code{triples}
that takes three infinite streams, \( S \), \( T \), and \( U \), and produces the
stream of triples \( (S_i, T_j, U_k) \) such that \( i \le j \le k \).  
Use \code{triples} to generate the stream of all Pythagorean
triples of positive integers, i.e., the triples \( (i, j, k) \) such that
\( i \le j \) and \( i^2 + j^2 = k^2 \).
\end{quote}

\begin{quote}
\heading{\phantomsection\label{Exercise 3.70}Exercise 3.70:} It would be nice to be able to
generate streams in which the pairs appear in some useful order, rather than in
the order that results from an \emph{ad hoc} interleaving process.  We can use
a technique similar to the \code{merge} procedure of \link{Exercise 3.56}, if we
define a way to say that one pair of integers is ``less than'' another.  One
way to do this is to define a ``weighting function'' \( W(i, j) \) and
stipulate that \( (i_1, j_1) \) is less than \( (i_2, j_2) \) if
\( W(i_1, j_1) < W(i_2, j_2) \).  Write a procedure
\code{merge\-/weighted} that is like \code{merge}, except that
\code{merge\-/weighted} takes an additional argument \code{weight}, which is a
procedure that computes the weight of a pair, and is used to determine the
order in which elements should appear in the resulting merged
stream.\footnote{We will require that the weighting function be such that the
weight of a pair increases as we move out along a row or down along a column of
the array of pairs.}  Using this, generalize \code{pairs} to a procedure
\code{weighted\-/pairs} that takes two streams, together with a procedure that
computes a weighting function, and generates the stream of pairs, ordered
according to weight.  Use your procedure to generate

\begin{enumerate}[a.]

\item
the stream of all pairs of positive integers \( (i, j) \) with \( i \le j \)
ordered according to the sum \( i + j \),

\item
the stream of all pairs of positive integers \( (i, j) \) with \( i \le j \),
where neither \( i \) nor \( j \) is divisible by 2, 3, or 5, and the pairs are
ordered according to the sum \( 2i + 3j + 5i\!j \).

\end{enumerate}
\end{quote}

\begin{quote}
\heading{\phantomsection\label{Exercise 3.71}Exercise 3.71:} Numbers that can be expressed as
the sum of two cubes in more than one way are sometimes called
\newterm{Ramanujan numbers}, in honor of the mathematician Srinivasa
Ramanujan.\footnote{To quote from G. H. Hardy's obituary of Ramanujan (\link{Hardy 1921}): 
``It was Mr. Littlewood (I believe) who remarked that `every positive
integer was one of his friends.'  I remember once going to see him when he was
lying ill at Putney.  I had ridden in taxi-cab No. 1729, and remarked that the
number seemed to me a rather dull one, and that I hoped it was not an
unfavorable omen.  `No,' he replied, `it is a very interesting number; it is
the smallest number expressible as the sum of two cubes in two different ways.'
'' The trick of using weighted pairs to generate the Ramanujan numbers was
shown to us by Charles Leiserson.} Ordered streams of pairs provide an elegant
solution to the problem of computing these numbers.  To find a number that can
be written as the sum of two cubes in two different ways, we need only generate
the stream of pairs of integers \( (i, j) \) weighted according to the sum
\( i^3 + j^3 \) (see \link{Exercise 3.70}), then search the stream for two
consecutive pairs with the same weight.  Write a procedure to generate the
Ramanujan numbers.  The first such number is 1,729.  What are the next five?
\end{quote}

\begin{quote}
\heading{\phantomsection\label{Exercise 3.72}Exercise 3.72:} In a similar way to \link{Exercise 3.71} 
generate a stream of all numbers that can be written as the sum of two
squares in three different ways (showing how they can be so written).
\end{quote}

\subsubsection*{Streams as signals}

We began our discussion of streams by describing them as computational analogs
of the ``signals'' in signal-processing systems.  In fact, we can use streams
to model signal-processing systems in a very direct way, representing the
values of a signal at successive time intervals as consecutive elements of a
stream.  For instance, we can implement an \newterm{integrator} or
\newterm{summer} that, for an input stream \( x = (x_i) \), an initial
value \( C \), and a small increment \( dt \), accumulates the sum
\begin{comment}

\begin{example}
           i
          ---
S_i = C + >   x_j dt
          ---
          j=1
\end{example}

\end{comment}
\begin{displaymath}
 S_i = C + \sum_{j=1}^i x_j dt 
\end{displaymath}
\noindent
and returns the stream of values \( S = (S_i) \).  The following
\code{integral} procedure is reminiscent of the ``implicit style'' definition
of the stream of integers (\link{Section 3.5.2}):

\begin{scheme}
(define (integral integrand initial-value dt)
  (define int
    (cons-stream initial-value
                 (add-streams (scale-stream integrand dt)
                              int)))
  int)
\end{scheme}

\noindent
\link{Figure 3.32} is a picture of a signal-processing system that corresponds
to the \code{integral} procedure.  The input stream is scaled by \( dt \) and
passed through an adder, whose output is passed back through the same adder.
The self-reference in the definition of \code{int} is reflected in the figure
by the feedback loop that connects the output of the adder to one of the
inputs.

\begin{figure}[tb]
\phantomsection\label{Figure 3.32}
\centering
\begin{comment}
\heading{Figure 3.32:} The \code{integral} procedure viewed as a signal-processing system.

\begin{example}
                             initial-value
                                  |
       +-----------+              |   |\__
input  |           |      |\__    +-->|   \_  integral
------>| scale: dt +----->|   \_      |cons_>--*------->
       |           |      | add_>---->| __/    |
       +-----------+  +-->| __/       |/       |
                      |   |/                   |
                      |                        |
                      +------------------------+
\end{example}
\end{comment}
\includegraphics[width=102mm]{fig/chap3/Fig3.32.pdf}
\begin{quote}
\heading{Figure 3.32:} The \code{integral} procedure viewed as a signal-processing system.
\end{quote}
\end{figure}

\begin{quote}
\heading{\phantomsection\label{Exercise 3.73}Exercise 3.73:} We can model electrical circuits
using streams to represent the values of currents or voltages at a sequence of
times.  For instance, suppose we have an \newterm{RC circuit} consisting of a
resistor of resistance \( R \) and a capacitor of capacitance \( C \) in series.
The voltage response \( v \) of the circuit to an injected current \( i \) is
determined by the formula in \link{Figure 3.33}, whose structure is shown by the
accompanying signal-flow diagram.

\begin{figure}[tb]
\phantomsection\label{Figure 3.33}
\centering
\begin{comment}
\heading{Figure 3.33:} An RC circuit and the associated signal-flow diagram.

\begin{example}
  +        v        -

 ->----'\/\/\,---| |---
  i       R         C


                  / t
               1  |
 v  =  v   +  --- |  i dt  +  R i
        0      C  |
                  / 0

         +--------------+
     +-->|   scale: R   |---------------------+   |\_
     |   +--------------+                     |   |  \_
     |                                        +-->|    \   v
  i  |   +--------------+     +------------+      | add >--->
 ----+-->|  scale: 1/C  |---->|  integral  |----->|   _/
         +--------------+     +------------+      | _/
                                    |             |/
				   v
				    0
\end{example}
\end{comment}
\includegraphics[width=94mm]{fig/chap3/Fig3.33.pdf}
\par\bigskip
\noindent
\heading{Figure 3.33:} An RC circuit and the associated signal-flow diagram. 
\end{figure}

Write a procedure \code{RC} that models this circuit.  \code{RC} should take as
inputs the values of \( R \), \( C \), and \( dt \) and should return a procedure
that takes as inputs a stream representing the current \( i \) and an initial
value for the capacitor voltage \( v_0 \) and produces as output the stream of
voltages \( v \).  For example, you should be able to use \code{RC} to model an
RC circuit with \( R \) = 5 ohms, \( C \) = 1 farad, and a 0.5-second time step by
evaluating \code{(define RC1 (RC 5 1 0.5))}.  This defines \code{RC1} as a
procedure that takes a stream representing the time sequence of currents and an
initial capacitor voltage and produces the output stream of voltages.
\end{quote}

\begin{quote}
\heading{\phantomsection\label{Exercise 3.74}Exercise 3.74:} Alyssa P. Hacker is designing a
system to process signals coming from physical sensors.  One important feature
she wishes to produce is a signal that describes the \newterm{zero crossings}
of the input signal.  That is, the resulting signal should be \( +1 \) whenever the
input signal changes from negative to positive, \( -1 \) whenever the input signal
changes from positive to negative, and 0 otherwise.  (Assume that the sign of a
0 input is positive.)  For example, a typical input signal with its associated
zero-crossing signal would be

\begin{scheme}
~\( \dots \)~ 1 2 1.5 1 0.5 -0.1 -2 -3 -2 -0.5 0.2 3 4 ~\( \dots \)~
~\( \dots \)~ 0 0  0  0  0   -1   0  0  0   0   1  0 0 ~\( \dots \)~
\end{scheme}

In Alyssa's system, the signal from the sensor is represented as a stream
\code{sense\-/data} and the stream \code{zero\-/crossings} is the corresponding
stream of zero crossings.  Alyssa first writes a procedure
\code{sign\-/change\-/detector} that takes two values as arguments and compares the
signs of the values to produce an appropriate 0, 1, or - 1.  She then
constructs her zero-crossing stream as follows:

\begin{scheme}
(define (make-zero-crossings input-stream last-value)
  (cons-stream
   (sign-change-detector
    (stream-car input-stream) 
    last-value)
   (make-zero-crossings
    (stream-cdr input-stream)
    (stream-car input-stream))))
(define zero-crossings
  (make-zero-crossings sense-data 0))
\end{scheme}

Alyssa's boss, Eva Lu Ator, walks by and suggests that this program is
approximately equivalent to the following one, which uses the generalized
version of \code{stream\-/map} from \link{Exercise 3.50}:

\begin{scheme}
(define zero-crossings
  (stream-map sign-change-detector
              sense-data
              ~\( \dark \langle \)~~\var{\dark expression}~~\( \dark \rangle \)~))
\end{scheme}

Complete the program by supplying the indicated \( \langle \)\var{expression}\( \rangle \).
\end{quote}

\begin{quote}
\heading{\phantomsection\label{Exercise 3.75}Exercise 3.75:} Unfortunately, Alyssa's
zero-crossing detector in \link{Exercise 3.74} proves to be insufficient,
because the noisy signal from the sensor leads to spurious zero crossings.  Lem
E.  Tweakit, a hardware specialist, suggests that Alyssa smooth the signal to
filter out the noise before extracting the zero crossings.  Alyssa takes his
advice and decides to extract the zero crossings from the signal constructed by
averaging each value of the sense data with the previous value.  She explains
the problem to her assistant, Louis Reasoner, who attempts to implement the
idea, altering Alyssa's program as follows:

\begin{scheme}
(define (make-zero-crossings input-stream last-value)
  (let ((avpt (/ (+ (stream-car input-stream) 
                    last-value) 
                 2)))
    (cons-stream
     (sign-change-detector avpt last-value)
     (make-zero-crossings 
      (stream-cdr input-stream) avpt))))
\end{scheme}

This does not correctly implement Alyssa's plan.  Find the bug that Louis has
installed and fix it without changing the structure of the program.  (Hint: You
will need to increase the number of arguments to \code{make\-/zero\-/crossings}.)
\end{quote}

\begin{quote}
\heading{\phantomsection\label{Exercise 3.76}Exercise 3.76:} Eva Lu Ator has a criticism of
Louis's approach in \link{Exercise 3.75}.  The program he wrote is not modular,
because it intermixes the operation of smoothing with the zero-crossing
extraction.  For example, the extractor should not have to be changed if Alyssa
finds a better way to condition her input signal.  Help Louis by writing a
procedure \code{smooth} that takes a stream as input and produces a stream in
which each element is the average of two successive input stream elements.
Then use \code{smooth} as a component to implement the zero-crossing detector
in a more modular style.
\end{quote}

\subsection{Streams and Delayed Evaluation}
\label{Section 3.5.4}

The \code{integral} procedure at the end of the preceding section shows how we
can use streams to model signal-processing systems that contain feedback loops.
The feedback loop for the adder shown in \link{Figure 3.32} is modeled by the
fact that \code{integral}'s internal stream \code{int} is defined in terms of
itself:

\begin{scheme}
(define int
  (cons-stream
   initial-value
   (add-streams (scale-stream integrand dt)
                int)))
\end{scheme}

\noindent
The interpreter's ability to deal with such an implicit definition depends on
the \code{delay} that is incorporated into \code{cons\-/stream}.  Without this
\code{delay}, the interpreter could not construct \code{int} before evaluating
both arguments to \code{cons\-/stream}, which would require that \code{int}
already be defined.  In general, \code{delay} is crucial for using streams to
model signal-processing systems that contain loops.  Without \code{delay}, our
models would have to be formulated so that the inputs to any signal-processing
component would be fully evaluated before the output could be produced.  This
would outlaw loops.

\begin{figure}[tb]
\phantomsection\label{Figure 3.34}
\centering
\begin{comment}
\heading{Figure 3.34:} An ``analog computer circuit'' that solves the equation \( dy\! / dt = f(y) \).

\begin{example}
                            y_0
                             |
                             V
    +----------+  dy   +----------+     y
+-->|  map: f  +------>| integral +--*----->
|   +----------+       +----------+  |
|                                    |
+------------------------------------+
\end{example}
\end{comment}
\includegraphics[width=67mm]{fig/chap3/Fig3.34.pdf}
\begin{quote}
\heading{Figure 3.34:} An ``analog computer circuit'' that solves the equation \( dy / dt = f(y) \).
\end{quote}
\end{figure}

Unfortunately, stream models of systems with loops may require uses of
\code{delay} beyond the ``hidden'' \code{delay} supplied by \code{cons\-/stream}.
For instance, \link{Figure 3.34} shows a signal-processing system for solving
the differential equation \( dy / dt = f(y) \) where \( f \) is a given
function.  The figure shows a mapping component, which applies \( f \) to its
input signal, linked in a feedback loop to an integrator in a manner very
similar to that of the analog computer circuits that are actually used to solve
such equations.

Assuming we are given an initial value \( y_0 \) for \( y \), we could try to model
this system using the procedure

\begin{scheme}
(define (solve f y0 dt)
  (define y (integral dy y0 dt))
  (define dy (stream-map f y))
  y)
\end{scheme}

\noindent
This procedure does not work, because in the first line of \code{solve} the
call to \code{integral} requires that the input \code{dy} be defined, which
does not happen until the second line of \code{solve}.

On the other hand, the intent of our definition does make sense, because we
can, in principle, begin to generate the \code{y} stream without knowing
\code{dy}.  Indeed, \code{integral} and many other stream operations have
properties similar to those of \code{cons\-/stream}, in that we can generate part
of the answer given only partial information about the arguments.  For
\code{integral}, the first element of the output stream is the specified
\code{initial\-/value}.  Thus, we can generate the first element of the output
stream without evaluating the integrand \code{dy}.  Once we know the first
element of \code{y}, the \code{stream\-/map} in the second line of \code{solve}
can begin working to generate the first element of \code{dy}, which will
produce the next element of \code{y}, and so on.

To take advantage of this idea, we will redefine \code{integral} to expect the
integrand stream to be a \newterm{delayed argument}.  \code{Integral} will
\code{force} the integrand to be evaluated only when it is required to generate
more than the first element of the output stream:

\begin{scheme}
(define (integral delayed-integrand initial-value dt)
  (define int
    (cons-stream
     initial-value
     (let ((integrand (force delayed-integrand)))
       (add-streams (scale-stream integrand dt) int))))
  int)
\end{scheme}

\noindent
Now we can implement our \code{solve} procedure by delaying the evaluation of
\code{dy} in the definition of \code{y}:\footnote{This procedure is not
guaranteed to work in all Scheme implementations, although for any
implementation there is a simple variation that will work.  The problem has to
do with subtle differences in the ways that Scheme implementations handle
internal definitions.  (See \link{Section 4.1.6}.)}

\begin{scheme}
(define (solve f y0 dt)
  (define y (integral (delay dy) y0 dt))
  (define dy (stream-map f y))
  y)
\end{scheme}

\noindent
In general, every caller of \code{integral} must now \code{delay} the integrand
argument.  We can demonstrate that the \code{solve} procedure works by
approximating \( e \approx 2.718 \) by computing the value at \( y = 1 \) of the
solution to the differential equation \( dy / dt = y \) with initial
condition \( y(0) = 1 \):

\begin{scheme}
(stream-ref (solve (lambda (y) y) 
                   1 
                   0.001) 
            1000)
~\textit{2.716924}~
\end{scheme}

\begin{quote}
\heading{\phantomsection\label{Exercise 3.77}Exercise 3.77:} The \code{integral} procedure
used above was analogous to the ``implicit'' definition of the infinite stream
of integers in \link{Section 3.5.2}.  Alternatively, we can give a definition of
\code{integral} that is more like \code{integers\-/starting\-/from} (also in
\link{Section 3.5.2}):

\begin{smallscheme}
(define (integral integrand initial-value dt)
  (cons-stream
   initial-value
   (if (stream-null? integrand)
       the-empty-stream
       (integral (stream-cdr integrand)
                 (+ (* dt (stream-car integrand))
                    initial-value)
                 dt))))
\end{smallscheme}

When used in systems with loops, this procedure has the same problem as does
our original version of \code{integral}.  Modify the procedure so that it
expects the \code{integrand} as a delayed argument and hence can be used in the
\code{solve} procedure shown above.
\end{quote}

\begin{figure}[tb]
\phantomsection\label{Figure 3.35}
\centering
\begin{comment}
\heading{Figure 3.35:} Signal-flow diagram for the solution to a second-order linear differential equation.

\begin{example}
               dy_0                y_0
                |                   |
                V                   V
   ddy     +----------+    dy  +----------+    y
+--------->| integral +-----*--+ integral +--*--->
|          +----------+     |  +----------+  |
|                           |                |
|            +----------+   |                |
|     __/|<--+ scale: a |<--+                |
|   _/   |   +----------+                    |
+--<_add |                                   |
     \__ |   +----------+                    |
        \|<--+ scale: b |<-------------------+
             +----------+
\end{example}
\end{comment}
\includegraphics[width=91mm]{fig/chap3/Fig3.35a.pdf}
\begin{quote}
\heading{Figure 3.35:} Signal-flow diagram for the solution to a second-order linear differential equation.
\end{quote}
\end{figure}

\begin{quote}
\heading{\phantomsection\label{Exercise 3.78}Exercise 3.78:} Consider the problem of designing
a signal-processing system to study the homogeneous second-order linear
differential equation
\begin{comment}

\begin{example}
d^2 y        d y
-----  -  a -----  -  by  =  0
d t^2        d t
\end{example}

\end{comment}
\begin{displaymath}
 {d^2\!y \over dt^2} - a {dy \over dt} - by = 0. 
\end{displaymath}
The output stream, modeling \( y \), is generated by a network that contains a
loop. This is because the value of \( d^2\!y / dt^2 \) depends upon the
values of \( y \) and \( dy / dt \) and both of these are determined by
integrating \( d^2\!y / dt^2 \).  The diagram we would like to encode is
shown in \link{Figure 3.35}.  Write a procedure \code{solve\-/2nd} that takes as
arguments the constants \( a \), \( b \), and \( dt \) and the initial values
\( y_0 \) and \( dy_0 \) for \( y \) and \( dy / dt \) and generates the
stream of successive values of \( y \).
\end{quote}

\begin{quote}
\heading{\phantomsection\label{Exercise 3.79}Exercise 3.79:} Generalize the \code{solve\-/2nd}
procedure of \link{Exercise 3.78} so that it can be used to solve general
second-order differential equations \( d^2\!y / dt^2 =
f(dy / dt, y) \).
\end{quote}

\begin{figure}[tb]
\phantomsection\label{Figure 3.36}
\centering
\begin{comment}
\heading{Figure 3.36:} A series RLC circuit.

\begin{example}
              + v_R -
        i_R
     +--->----'\/\/\,--------+
     |                       |  i_L
    \|/          R          \|/
  +  |  i_C                  |_   +
    -+-                       _)
v_C -+- C                     _)  v_L
     |                        _)
  -  |                       |    -
     +-----------------------+
\end{example}
\end{comment}
\includegraphics[width=60mm]{fig/chap3/Fig3.36.pdf}
\par\bigskip
\noindent
\heading{Figure 3.36:} A series RLC circuit.
\end{figure}

\begin{quote}
\heading{\phantomsection\label{Exercise 3.80}Exercise 3.80:} A \newterm{series RLC circuit}
consists of a resistor, a capacitor, and an inductor connected in series, as
shown in \link{Figure 3.36}.  If \( R \), \( L \), and \( C \) are the resistance,
inductance, and capacitance, then the relations between voltage (\( v \)) and
current (\( i \)) for the three components are described by the equations
\begin{comment}

\begin{example}
v_R = i_R R

         d i_L
v_L = L -------
          d t

         d v_C
i_C = C -------
          d t
\end{example}

\end{comment}
\begin{displaymath}
 	v_R 	= 	i_R R, \qquad\quad
	v_L 	= 	L {di_L \over dt}\,, \qquad\quad
	i_C 	= 	C {dv_C \over dt}\,, 
\end{displaymath}
and the circuit connections dictate the relations
\begin{comment}

\begin{example}
i_R = i_L = -i_C

v_C = v_L + v_R
\end{example}

\end{comment}
\begin{displaymath}
 	i_R 	= 	i_L = -i_C\,, \qquad\quad
	v_C 	= 	v_L +  v_R\,.  
\end{displaymath}
Combining these equations shows that the state of the circuit (summarized by
\( v_C \), the voltage across the capacitor, and \( i_L \), the current in
the inductor) is described by the pair of differential equations
\begin{comment}

\begin{example}
d v_C        i_L
-----  =  -  ---
 d t          C

d i_L      1           R
-----  =  --- v_C  -  --- i_L
 d t       L           L
\end{example}

\end{comment}
\begin{displaymath}
  {dv_C \over dt}  =  -{i_L \over C}\,, \qquad\quad
    {di_L \over dt}  =   {1   \over L} v_C - {R \over L} i_L\,. 
\end{displaymath}
The signal-flow diagram representing this system of differential equations is
shown in \link{Figure 3.37}.
\end{quote}

\begin{figure}[tb]
\phantomsection\label{Figure 3.37}
\centering
\begin{comment}
\heading{Figure 3.37:} A signal-flow diagram for the solution to a series RLC circuit.

\begin{example}
                 +-------------+
+----------------+  scale: l/L |<--+
|                +-------------+   |
|                                  |
|                +-------------+   |  v_C
|       dv_C +-->|   integral  +---*------>
|            |   +-------------+
|            |        ^
|            |        | v_(C_0)
|            |
|            |   +-------------+
|            +---+ scale: -l/C |<--+
|                +-------------+   |
|  |\__                            |
+->|   \_  di_L  +-------------+   |  i_L
   | add_>------>|   integral  +---*------>
+->| __/         +-------------+   |
|  |/                 ^            |
|                     | i_(L_0)    |
|                                  |
|                +-------------+   |
+----------------+ scale: -R/L |<--+
                 +-------------+
\end{example}
\end{comment}
\includegraphics[width=68mm]{fig/chap3/Fig3.37a.pdf}
\begin{quote}
\heading{Figure 3.37:} A signal-flow diagram for the solution to a series RLC circuit.
\end{quote}
\end{figure}

\begin{quote}
Write a procedure \code{RLC} that takes as arguments the parameters \( R \),
\( L \), and \( C \) of the circuit and the time increment \( dt \).  In a manner
similar to that of the \code{RC} procedure of \link{Exercise 3.73}, \code{RLC}
should produce a procedure that takes the initial values of the state
variables, \( v_{C_0} \) and \( i_{L_0} \), and produces a pair (using
\code{cons}) of the streams of states \( v_C \) and \( i_L \).  Using
\code{RLC}, generate the pair of streams that models the behavior of a series
RLC circuit with \( R \) = 1 ohm, \( C \) = 0.2 farad, \( L \) = 1 henry, \( dt \)
= 0.1 second, and initial values \( i_{L_0} \) = 0 amps and \( v_{C_0} \) =
10 volts.
\end{quote}

\subsubsection*{Normal-order evaluation}

The examples in this section illustrate how the explicit use of \code{delay}
and \code{force} provides great programming flexibility, but the same examples
also show how this can make our programs more complex.  Our new \code{integral}
procedure, for instance, gives us the power to model systems with loops, but we
must now remember that \code{integral} should be called with a delayed
integrand, and every procedure that uses \code{integral} must be aware of this.
In effect, we have created two classes of procedures: ordinary procedures and
procedures that take delayed arguments.  In general, creating separate classes
of procedures forces us to create separate classes of higher-order procedures
as well.\footnote{This is a small reflection, in Lisp, of the difficulties that
conventional strongly typed languages such as Pascal have in coping with
higher-order procedures.  In such languages, the programmer must specify the
data types of the arguments and the result of each procedure: number, logical
value, sequence, and so on.  Consequently, we could not express an abstraction
such as ``map a given procedure \code{proc} over all the elements in a
sequence'' by a single higher-order procedure such as \code{stream\-/map}.
Rather, we would need a different mapping procedure for each different
combination of argument and result data types that might be specified for a
\code{proc}.  Maintaining a practical notion of ``data type'' in the presence
of higher-order procedures raises many difficult issues.  One way of dealing
with this problem is illustrated by the language ML (\link{Gordon et al. 1979}), 
whose ``polymorphic data types'' include templates for
higher-order transformations between data types.  Moreover, data types for most
procedures in ML are never explicitly declared by the programmer.  Instead, ML
includes a \newterm{type-inferencing} mechanism that uses information in the
environment to deduce the data types for newly defined procedures.}

One way to avoid the need for two different classes of procedures is to make
all procedures take delayed arguments.  We could adopt a model of evaluation in
which all arguments to procedures are automatically delayed and arguments are
forced only when they are actually needed (for example, when they are required
by a primitive operation).  This would transform our language to use
normal-order evaluation, which we first described when we introduced the
substitution model for evaluation in \link{Section 1.1.5}.  Converting to
normal-order evaluation provides a uniform and elegant way to simplify the use
of delayed evaluation, and this would be a natural strategy to adopt if we were
concerned only with stream processing.  In \link{Section 4.2}, after we have
studied the evaluator, we will see how to transform our language in just this
way.  Unfortunately, including delays in procedure calls wreaks havoc with our
ability to design programs that depend on the order of events, such as programs
that use assignment, mutate data, or perform input or output.  Even the single
\code{delay} in \code{cons\-/stream} can cause great confusion, as illustrated by
\link{Exercise 3.51} and \link{Exercise 3.52}.  As far as anyone knows,
mutability and delayed evaluation do not mix well in programming languages, and
devising ways to deal with both of these at once is an active area of research.

\subsection{Modularity of Functional Programs\\ and Modularity of Objects}
\label{Section 3.5.5}

As we saw in \link{Section 3.1.2}, one of the major benefits of introducing
assignment is that we can increase the modularity of our systems by
encapsulating, or ``hiding,'' parts of the state of a large system within local
variables.  Stream models can provide an equivalent modularity without the use
of assignment.  As an illustration, we can reimplement the Monte Carlo
estimation of \( \pi \), which we examined in \link{Section 3.1.2}, from a
stream-processing point of view.

The key modularity issue was that we wished to hide the internal state of a
random-number generator from programs that used random numbers.  We began with
a procedure \code{rand\-/update}, whose successive values furnished our supply of
random numbers, and used this to produce a random-number generator:

\begin{scheme}
(define rand
  (let ((x random-init))
    (lambda ()
      (set! x (rand-update x))
      x)))
\end{scheme}

\noindent
In the stream formulation there is no random-number generator \emph{per se},
just a stream of random numbers produced by successive calls to
\code{rand\-/update}:

\begin{scheme}
(define random-numbers
  (cons-stream
   random-init
   (stream-map rand-update random-numbers)))
\end{scheme}

\noindent
We use this to construct the stream of outcomes of the Ces\`aro experiment
performed on consecutive pairs in the \code{random\-/numbers} stream:

\begin{scheme}
(define cesaro-stream
  (map-successive-pairs
   (lambda (r1 r2) (= (gcd r1 r2) 1))
   random-numbers))
(define (map-successive-pairs f s)
  (cons-stream
   (f (stream-car s) (stream-car (stream-cdr s)))
   (map-successive-pairs f (stream-cdr (stream-cdr s)))))
\end{scheme}

\noindent
The \code{cesaro\-/stream} is now fed to a \code{monte\-/carlo} procedure, which
produces a stream of estimates of probabilities.  The results are then
converted into a stream of estimates of \( \pi \).  This version of the program
doesn't need a parameter telling how many trials to perform.  Better estimates
of \( \pi \) (from performing more experiments) are obtained by looking farther
into the \code{pi} stream:

\begin{scheme}
(define (monte-carlo experiment-stream passed failed)
  (define (next passed failed)
    (cons-stream
     (/ passed (+ passed failed))
     (monte-carlo
      (stream-cdr experiment-stream) passed failed)))
  (if (stream-car experiment-stream)
      (next (+ passed 1) failed)
      (next passed (+ failed 1))))
(define pi
  (stream-map
   (lambda (p) (sqrt (/ 6 p)))
   (monte-carlo cesaro-stream 0 0)))
\end{scheme}

\noindent
There is considerable modularity in this approach, because we still can
formulate a general \code{monte\-/carlo} procedure that can deal with arbitrary
experiments.  Yet there is no assignment or local state.

\begin{quote}
\heading{\phantomsection\label{Exercise 3.81}Exercise 3.81:} \link{Exercise 3.6} discussed
generalizing the random-number generator to allow one to reset the
random-number sequence so as to produce repeatable sequences of ``random''
numbers.  Produce a stream formulation of this same generator that operates on
an input stream of requests to \code{generate} a new random number or to
\code{reset} the sequence to a specified value and that produces the desired
stream of random numbers.  Don't use assignment in your solution.
\end{quote}

\begin{quote}
\heading{\phantomsection\label{Exercise 3.82}Exercise 3.82:} Redo \link{Exercise 3.5} on Monte
Carlo integration in terms of streams.  The stream version of
\code{estimate\-/integral} will not have an argument telling how many trials to
perform.  Instead, it will produce a stream of estimates based on successively
more trials.
\end{quote}

\subsubsection*{A functional-programming view of time}

Let us now return to the issues of objects and state that were raised at the
beginning of this chapter and examine them in a new light.  We introduced
assignment and mutable objects to provide a mechanism for modular construction
of programs that model systems with state.  We constructed computational
objects with local state variables and used assignment to modify these
variables.  We modeled the temporal behavior of the objects in the world by the
temporal behavior of the corresponding computational objects.

Now we have seen that streams provide an alternative way to model objects with
local state.  We can model a changing quantity, such as the local state of some
object, using a stream that represents the time history of successive states.
In essence, we represent time explicitly, using streams, so that we decouple
time in our simulated world from the sequence of events that take place during
evaluation.  Indeed, because of the presence of \code{delay} there may be
little relation between simulated time in the model and the order of events
during the evaluation.

In order to contrast these two approaches to modeling, let us reconsider the
implementation of a ``withdrawal processor'' that monitors the balance in a
bank account.  In \link{Section 3.1.3} we implemented a simplified version of
such a processor:

\begin{scheme}
(define (make-simplified-withdraw balance)
  (lambda (amount)
    (set! balance (- balance amount))
    balance))
\end{scheme}

\noindent
Calls to \code{make\-/simplified\-/withdraw} produce computational objects, each
with a local state variable \code{balance} that is decremented by successive
calls to the object.  The object takes an \code{amount} as an argument and
returns the new balance.  We can imagine the user of a bank account typing a
sequence of inputs to such an object and observing the sequence of returned
values shown on a display screen.

Alternatively, we can model a withdrawal processor as a procedure that takes as
input a balance and a stream of amounts to withdraw and produces the stream of
successive balances in the account:

\begin{scheme}
(define (stream-withdraw balance amount-stream)
  (cons-stream
   balance
   (stream-withdraw (- balance (stream-car amount-stream))
                    (stream-cdr amount-stream))))
\end{scheme}

\noindent
\code{Stream\-/withdraw} implements a well-defined mathematical function whose
output is fully determined by its input.  Suppose, however, that the input
\code{amount\-/stream} is the stream of successive values typed by the user and
that the resulting stream of balances is displayed.  Then, from the perspective
of the user who is typing values and watching results, the stream process has
the same behavior as the object created by \code{make\-/simplified\-/withdraw}.
However, with the stream version, there is no assignment, no local state
variable, and consequently none of the theoretical difficulties that we
encountered in \link{Section 3.1.3}.  Yet the system has state!

This is really remarkable.  Even though \code{stream\-/withdraw} implements a
well-defined mathematical function whose behavior does not change, the user's
perception here is one of interacting with a system that has a changing state.
One way to resolve this paradox is to realize that it is the user's temporal
existence that imposes state on the system.  If the user could step back from
the interaction and think in terms of streams of balances rather than
individual transactions, the system would appear stateless.\footnote{Similarly
in physics, when we observe a moving particle, we say that the position (state)
of the particle is changing.  However, from the perspective of the particle's
world line in space-time there is no change involved.}

From the point of view of one part of a complex process, the other parts appear
to change with time.  They have hidden time-varying local state.  If we wish to
write programs that model this kind of natural decomposition in our world (as
we see it from our viewpoint as a part of that world) with structures in our
computer, we make computational objects that are not functional---they must
change with time.  We model state with local state variables, and we model the
changes of state with assignments to those variables.  By doing this we make
the time of execution of a computation model time in the world that we are part
of, and thus we get ``objects'' in our computer.

Modeling with objects is powerful and intuitive, largely because this matches
the perception of interacting with a world of which we are part.  However, as
we've seen repeatedly throughout this chapter, these models raise thorny
problems of constraining the order of events and of synchronizing multiple
processes.  The possibility of avoiding these problems has stimulated the
development of \newterm{functional programming languages}, which do not include
any provision for assignment or mutable data.  In such a language, all
procedures implement well-defined mathematical functions of their arguments,
whose behavior does not change.  The functional approach is extremely
attractive for dealing with concurrent systems.\footnote{John Backus, the
inventor of Fortran, gave high visibility to functional programming when he was
awarded the \acronym{ACM} Turing award in 1978.  His acceptance speech 
(\link{Backus 1978}) strongly advocated the functional approach.  A good overview of
functional programming is given in \link{Henderson 1980} and in 
\link{Darlington et al. 1982}.}

On the other hand, if we look closely, we can see time-related problems
creeping into functional models as well.  One particularly troublesome area
arises when we wish to design interactive systems, especially ones that model
interactions between independent entities.  For instance, consider once more
the implementation a banking system that permits joint bank accounts.  In a
conventional system using assignment and objects, we would model the fact that
Peter and Paul share an account by having both Peter and Paul send their
transaction requests to the same bank-account object, as we saw in 
\link{Section 3.1.3}.  From the stream point of view, where there are no ``objects''
\emph{per se}, we have already indicated that a bank account can be modeled as
a process that operates on a stream of transaction requests to produce a stream
of responses.  Accordingly, we could model the fact that Peter and Paul have a
joint bank account by merging Peter's stream of transaction requests with
Paul's stream of requests and feeding the result to the bank-account stream
process, as shown in \link{Figure 3.38}.

\begin{figure}[tb]
\phantomsection\label{Figure 3.38}
\centering
\begin{comment}
\heading{Figure 3.38:} A joint bank account, modeled by merging two streams of transaction requests.

\begin{example}
Peter's requests   +---------+     +---------+
------------------>|         |     |         |
Paul's requests    |  merge  |---->| bank    |---->
------------------>|         |     | account |
                   +---------+     +---------+
\end{example}
\end{comment}
\includegraphics[width=88mm]{fig/chap3/Fig3.38.pdf}
\begin{quote}
\heading{Figure 3.38:} A joint bank account, modeled by merging two streams of transaction requests.
\end{quote}
\end{figure}

The trouble with this formulation is in the notion of \newterm{merge}.  It will
not do to merge the two streams by simply taking alternately one request from
Peter and one request from Paul. Suppose Paul accesses the account only very
rarely.  We could hardly force Peter to wait for Paul to access the account
before he could issue a second transaction.  However such a merge is
implemented, it must interleave the two transaction streams in some way that is
constrained by ``real time'' as perceived by Peter and Paul, in the sense that,
if Peter and Paul meet, they can agree that certain transactions were processed
before the meeting, and other transactions were processed after the
meeting.\footnote{Observe that, for any two streams, there is in general more
than one acceptable order of interleaving.  Thus, technically, ``merge'' is a
relation rather than a function---the answer is not a deterministic function of
the inputs.  We already mentioned (\link{Footnote 39}) that nondeterminism is
essential when dealing with concurrency.  The merge relation illustrates the
same essential nondeterminism, from the functional perspective.  In 
\link{Section 4.3}, we will look at nondeterminism from yet another point of view.} This
is precisely the same constraint that we had to deal with in 
\link{Section 3.4.1}, where we found the need to introduce explicit synchronization to
ensure a ``correct'' order of events in concurrent processing of objects with
state.  Thus, in an attempt to support the functional style, the need to merge
inputs from different agents reintroduces the same problems that the functional
style was meant to eliminate.

We began this chapter with the goal of building computational models whose
structure matches our perception of the real world we are trying to model.  We
can model the world as a collection of separate, time-bound, interacting
objects with state, or we can model the world as a single, timeless, stateless
unity.  Each view has powerful advantages, but neither view alone is completely
satisfactory.  A grand unification has yet to emerge.\footnote{The object model
approximates the world by dividing it into separate pieces.  The functional
model does not modularize along object boundaries.  The object model is useful
when the unshared state of the ``objects'' is much larger than the state that
they share.  An example of a place where the object viewpoint fails is quantum
mechanics, where thinking of things as individual particles leads to paradoxes
and confusions.  Unifying the object view with the functional view may have
little to do with programming, but rather with fundamental epistemological
issues.}

\chapter{Metalinguistic Abstraction}
\label{Chapter 4}

\begin{quote}
\( \dots \) It's in words that the magic is---Abracadabra, Open Sesame, and the
rest---but the magic words in one story aren't magical in the next.  The real
magic is to understand which words work, and when, and for what; the trick is
to learn the trick.

\( \dots \) And those words are made from the letters of our alphabet: a
couple-dozen squiggles we can draw with the pen.  This is the key!  And the
treasure, too, if we can only get our hands on it!  It's as if---as if the key
to the treasure \emph{is} the treasure!

---John Barth, \textit{Chimera}
\end{quote}

% \vspace{0.8em}

\noindent
\lettrine{I}{n our study of program design}, we have seen that expert programmers control
the complexity of their designs with the same general techniques used by
designers of all complex systems.  They combine primitive elements to form
compound objects, they abstract compound objects to form higher-level building
blocks, and they preserve modularity by adopting appropriate large-scale views
of system structure.  In illustrating these techniques, we have used Lisp as a
language for describing processes and for constructing computational data
objects and processes to model complex phenomena in the real world.  However,
as we confront increasingly complex problems, we will find that Lisp, or indeed
any fixed programming language, is not sufficient for our needs.  We must
constantly turn to new languages in order to express our ideas more
effectively.  Establishing new languages is a powerful strategy for controlling
complexity in engineering design; we can often enhance our ability to deal with
a complex problem by adopting a new language that enables us to describe (and
hence to think about) the problem in a different way, using primitives, means
of combination, and means of abstraction that are particularly well suited to
the problem at hand.\footnote{The same idea is pervasive throughout all of
engineering.  For example, electrical engineers use many different languages
for describing circuits.  Two of these are the language of electrical
\newterm{networks} and the language of electrical \newterm{systems}.  The
network language emphasizes the physical modeling of devices in terms of
discrete electrical elements.  The primitive objects of the network language
are primitive electrical components such as resistors, capacitors, inductors,
and transistors, which are characterized in terms of physical variables called
voltage and current.  When describing circuits in the network language, the
engineer is concerned with the physical characteristics of a design.  In
contrast, the primitive objects of the system language are signal-processing
modules such as filters and amplifiers.  Only the functional behavior of the
modules is relevant, and signals are manipulated without concern for their
physical realization as voltages and currents.  The system language is erected
on the network language, in the sense that the elements of signal-processing
systems are constructed from electrical networks.  Here, however, the concerns
are with the large-scale organization of electrical devices to solve a given
application problem; the physical feasibility of the parts is assumed.  This
layered collection of languages is another example of the stratified design
technique illustrated by the picture language of \link{Section 2.2.4}.}

Programming is endowed with a multitude of languages.  There are physical
languages, such as the machine languages for particular computers.  These
languages are concerned with the representation of data and control in terms of
individual bits of storage and primitive machine instructions.  The
machine-language programmer is concerned with using the given hardware to erect
systems and utilities for the efficient implementation of resource-limited
computations.  High-level languages, erected on a machine-language substrate,
hide concerns about the representation of data as collections of bits and the
representation of programs as sequences of primitive instructions.  These
languages have means of combination and abstraction, such as procedure
definition, that are appropriate to the larger-scale organization of systems.

\newterm{Metalinguistic abstraction}---establishing new languages---plays an
important role in all branches of engineering design.  It is particularly
important to computer programming, because in programming not only can we
formulate new languages but we can also implement these languages by
constructing evaluators.  An \newterm{evaluator} (or \newterm{interpreter}) for
a programming language is a procedure that, when applied to an expression of
the language, performs the actions required to evaluate that expression.

It is no exaggeration to regard this as the most fundamental idea in
programming:

\begin{quote}
The evaluator, which determines the meaning of expressions in a programming
language, is just another program.
\end{quote}

\noindent
To appreciate this point is to change our images of ourselves as programmers.
We come to see ourselves as designers of languages, rather than only users of
languages designed by others.

In fact, we can regard almost any program as the evaluator for some language.
For instance, the polynomial manipulation system of \link{Section 2.5.3}
embodies the rules of polynomial arithmetic and implements them in terms of
operations on list-structured data.  If we augment this system with procedures
to read and print polynomial expressions, we have the core of a special-purpose
language for dealing with problems in symbolic mathematics.  The digital-logic
simulator of \link{Section 3.3.4} and the constraint propagator of 
\link{Section 3.3.5} are legitimate languages in their own right, each with its own
primitives, means of combination, and means of abstraction.  Seen from this
perspective, the technology for coping with large-scale computer systems merges
with the technology for building new computer languages, and computer science
itself becomes no more (and no less) than the discipline of constructing
appropriate descriptive languages.

We now embark on a tour of the technology by which languages are established in
terms of other languages.  In this chapter we shall use Lisp as a base,
implementing evaluators as Lisp procedures.  Lisp is particularly well suited
to this task, because of its ability to represent and manipulate symbolic
expressions.  We will take the first step in understanding how languages are
implemented by building an evaluator for Lisp itself.  The language implemented
by our evaluator will be a subset of the Scheme dialect of Lisp that we use in
this book.  Although the evaluator described in this chapter is written for a
particular dialect of Lisp, it contains the essential structure of an evaluator
for any expression-oriented language designed for writing programs for a
sequential machine.  (In fact, most language processors contain, deep within
them, a little ``Lisp'' evaluator.)  The evaluator has been simplified for the
purposes of illustration and discussion, and some features have been left out
that would be important to include in a production-quality Lisp system.
Nevertheless, this simple evaluator is adequate to execute most of the programs
in this book.\footnote{The most important features that our evaluator leaves
out are mechanisms for handling errors and supporting debugging.  For a more
extensive discussion of evaluators, see \link{Friedman et al. 1992}, which
gives an exposition of programming languages that proceeds via a sequence of
evaluators written in Scheme.}

An important advantage of making the evaluator accessible as a Lisp program is
that we can implement alternative evaluation rules by describing these as
modifications to the evaluator program.  One place where we can use this power
to good effect is to gain extra control over the ways in which computational
models embody the notion of time, which was so central to the discussion in
\link{Chapter 3}.  There, we mitigated some of the complexities of state and
assignment by using streams to decouple the representation of time in the world
from time in the computer.  Our stream programs, however, were sometimes
cumbersome, because they were constrained by the applicative-order evaluation
of Scheme.  In \link{Section 4.2}, we'll change the underlying language to
provide for a more elegant approach, by modifying the evaluator to provide for
\newterm{normal-order evaluation}.

\link{Section 4.3} implements a more ambitious linguistic change, whereby
expressions have many values, rather than just a single value.  In this
language of \newterm{nondeterministic computing}, it is natural to express
processes that generate all possible values for expressions and then search for
those values that satisfy certain constraints.  In terms of models of
computation and time, this is like having time branch into a set of ``possible
futures'' and then searching for appropriate time lines.  With our
nondeterministic evaluator, keeping track of multiple values and performing
searches are handled automatically by the underlying mechanism of the language.

In \link{Section 4.4} we implement a \newterm{logic-programming} language in
which knowledge is expressed in terms of relations, rather than in terms of
computations with inputs and outputs.  Even though this makes the language
drastically different from Lisp, or indeed from any conventional language, we
will see that the logic-programming evaluator shares the essential structure of
the Lisp evaluator.



\section{The Metacircular Evaluator}
\label{Section 4.1}

Our evaluator for Lisp will be implemented as a Lisp program.  It may seem
circular to think about evaluating Lisp programs using an evaluator that is
itself implemented in Lisp.  However, evaluation is a process, so it is
appropriate to describe the evaluation process using Lisp, which, after all, is
our tool for describing processes.\footnote{Even so, there will remain
important aspects of the evaluation process that are not elucidated by our
evaluator.  The most important of these are the detailed mechanisms by which
procedures call other procedures and return values to their callers.  We will
address these issues in \link{Chapter 5}, where we take a closer look at the
evaluation process by implementing the evaluator as a simple register machine.}
An evaluator that is written in the same language that it evaluates is said to
be \newterm{metacircular}.

The metacircular evaluator is essentially a Scheme formulation of the
environment model of evaluation described in \link{Section 3.2}.  Recall that
the model has two basic parts:

\begin{enumerate}

\item
To evaluate a combination (a compound expression other than a special form),
evaluate the subexpressions and then apply the value of the operator
subexpression to the values of the operand subexpressions.

\item
To apply a compound procedure to a set of arguments, evaluate the body of the
procedure in a new environment.  To construct this environment, extend the
environment part of the procedure object by a frame in which the formal
parameters of the procedure are bound to the arguments to which the procedure
is applied.

\end{enumerate}

\noindent
These two rules describe the essence of the evaluation process, a basic cycle
in which expressions to be evaluated in environments are reduced to procedures
to be applied to arguments, which in turn are reduced to new expressions to be
evaluated in new environments, and so on, until we get down to symbols, whose
values are looked up in the environment, and to primitive procedures, which are
applied directly (see \link{Figure 4.1}).\footnote{If we grant ourselves the
ability to apply primitives, then what remains for us to implement in the
evaluator?  The job of the evaluator is not to specify the primitives of the
language, but rather to provide the connective tissue---the means of
combination and the means of abstraction---that binds a collection of
primitives to form a language. Specifically:

\( \bullet \) The evaluator enables us to deal with nested expressions.  For example,
although simply applying primitives would suffice for evaluating the expression
\code{(+ 1 6)}, it is not adequate for handling \code{(+ 1 (* 2 3))}.  As far
as the primitive procedure \code{+} is concerned, its arguments must be
numbers, and it would choke if we passed it the expression \code{(* 2 3)} as an
argument.  One important role of the evaluator is to choreograph procedure
composition so that \code{(* 2 3)} is reduced to 6 before being passed as an
argument to \code{+}.

\( \bullet \) The evaluator allows us to use variables.  For example, the primitive procedure
for addition has no way to deal with expressions such as \code{(+ x 1)}.  We
need an evaluator to keep track of variables and obtain their values before
invoking the primitive procedures.

\( \bullet \) The evaluator allows us to define compound procedures.  This involves keeping
track of procedure definitions, knowing how to use these definitions in
evaluating expressions, and providing a mechanism that enables procedures to
accept arguments.

\( \bullet \) The evaluator provides the special forms, which must be evaluated differently
from procedure calls.} 
This evaluation cycle will be embodied by the interplay between the two
critical procedures in the evaluator, \code{eval} and \code{apply}, which are
described in \link{Section 4.1.1} (see \link{Figure 4.1}).

The implementation of the evaluator will depend upon procedures that define the
\newterm{syntax} of the expressions to be evaluated.  We will use data
abstraction to make the evaluator independent of the representation of the
language.  For example, rather than committing to a choice that an assignment
is to be represented by a list beginning with the symbol \code{set!} we use an
abstract predicate \code{assignment?} to test for an assignment, and we use
abstract selectors \code{assignment\-/variable} and \code{assignment\-/value} to
access the parts of an assignment.  Implementation of expressions will be
described in detail in \link{Section 4.1.2}.  There are also operations,
described in \link{Section 4.1.3}, that specify the representation of procedures
and environments.  For example, \code{make\-/procedure} constructs compound
procedures, \code{lookup\-/variable\-/value} accesses the values of variables, and
\code{apply\-/primitive\-/procedure} applies a primitive procedure to a given list
of arguments.

\begin{figure}[tb]
\phantomsection\label{Figure 4.1}
\centering
\begin{comment}
\heading{Figure 4.1:} The \code{eval}-\code{apply} cycle exposes the essence of a computer language.

\begin{example}
                           .,ad88888888baa,
                  _    ,d8P"""        ""9888ba.      _
                 /  .a8"          ,ad88888888888a   |\
               /   aP'          ,88888888888888888a   \
              /  ,8"           ,88888888888888888888,  \
             |  ,8'            (888888888888888888888, |
            /  ,8'             `8888888888888888888888  \
            |  8)               `888888888888888888888, |
Procedure,  |  8                  "88888 Apply 8888888) | Expression
Arguments   |  8     Eval          `888888888888888888) | Environment
            |  8)                    "8888888888888888  |
            \  (b                     "88888888888888'  /
             | `8,                     8888888888888)  |
             \  "8a                   ,888888888888)  /
              \   V8,                 d88888888888"  /
              _\| `8b,             ,d8888888888P' _/
                     `V8a,       ,ad8888888888P'
                        ""88888888888888888P"
                             """"""""""""

                               [graphic by Normand Veillux, modified]
\end{example}
\end{comment}
\includegraphics[width=100mm]{fig/chap4/Fig4.1.pdf}
\begin{quote}
\heading{Figure 4.1:} The \code{eval}-\code{apply} cycle exposes the essence of a computer language.
\end{quote}
\end{figure}



\subsection{The Core of the Evaluator}
\label{Section 4.1.1}

The evaluation process can be described as the interplay between two
procedures: \code{eval} and \code{apply}.

\subsubsection*{Eval}

\code{Eval} takes as arguments an expression and an environment.  It classifies
the expression and directs its evaluation.  \code{Eval} is structured as a case
analysis of the syntactic type of the expression to be evaluated.  In order to
keep the procedure general, we express the determination of the type of an
expression abstractly, making no commitment to any particular representation
for the various types of expressions.  Each type of expression has a predicate
that tests for it and an abstract means for selecting its parts.  This
\newterm{abstract syntax} makes it easy to see how we can change the syntax of
the language by using the same evaluator, but with a different collection of
syntax procedures.

\bigskip
\noindent
\textbf{Primitive expressions}

\begin{itemize}

\item
For self-evaluating expressions, such as numbers, \code{eval} returns the
expression itself.

\item
\code{Eval} must look up variables in the environment to find their values.

\end{itemize}

\noindent
\textbf{Special forms}

\begin{itemize}

\item
For quoted expressions, \code{eval} returns the expression that was quoted.

\item
An assignment to (or a definition of) a variable must recursively call
\code{eval} to compute the new value to be associated with the variable.  The
environment must be modified to change (or create) the binding of the variable.

\item
An \code{if} expression requires special processing of its parts, so as to
evaluate the consequent if the predicate is true, and otherwise to evaluate the
alternative.

\item
A \code{lambda} expression must be transformed into an applicable procedure by
packaging together the parameters and body specified by the \code{lambda}
expression with the environment of the evaluation.

\item
A \code{begin} expression requires evaluating its sequence of expressions in
the order in which they appear.

\item
A case analysis (\code{cond}) is transformed into a nest of \code{if}
expressions and then evaluated.

\end{itemize}

\noindent
\textbf{Combinations}

\begin{itemize}

\item
For a procedure application, \code{eval} must recursively evaluate the operator
part and the operands of the combination.  The resulting procedure and
arguments are passed to \code{apply}, which handles the actual procedure
application.

\end{itemize}

\noindent
Here is the definition of \code{eval}:

\begin{scheme}
(define (eval exp env)
  (cond ((self-evaluating? exp) exp)
        ((variable? exp) (lookup-variable-value exp env))
        ((quoted? exp) (text-of-quotation exp))
        ((assignment? exp) (eval-assignment exp env))
        ((definition? exp) (eval-definition exp env))
        ((if? exp) (eval-if exp env))
        ((lambda? exp) (make-procedure (lambda-parameters exp)
                                       (lambda-body exp)
                                       env))
        ((begin? exp)
         (eval-sequence (begin-actions exp) env))
        ((cond? exp) (eval (cond->if exp) env))
        ((application? exp)
         (apply (eval (operator exp) env)
                (list-of-values (operands exp) env)))
        (else
         (error "Unknown expression type: EVAL" exp))))
\end{scheme}

\noindent
For clarity, \code{eval} has been implemented as a case analysis using
\code{cond}.  The disadvantage of this is that our procedure handles only a few
distinguishable types of expressions, and no new ones can be defined without
editing the definition of \code{eval}.  In most Lisp implementations,
dispatching on the type of an expression is done in a data-directed style.
This allows a user to add new types of expressions that \code{eval} can
distinguish, without modifying the definition of \code{eval} itself.  (See
\link{Exercise 4.3}.)

\subsubsection*{Apply}

\code{Apply} takes two arguments, a procedure and a list of arguments to which
the procedure should be applied.  \code{Apply} classifies procedures into two
kinds: It calls \code{apply\-/primitive\-/procedure} to apply primitives; it
applies compound procedures by sequentially evaluating the expressions that
make up the body of the procedure.  The environment for the evaluation of the
body of a compound procedure is constructed by extending the base environment
carried by the procedure to include a frame that binds the parameters of the
procedure to the arguments to which the procedure is to be applied.  Here is
the definition of \code{apply}:

\begin{scheme}
(define (apply procedure arguments)
  (cond ((primitive-procedure? procedure)
         (apply-primitive-procedure procedure arguments))
        ((compound-procedure? procedure)
         (eval-sequence
           (procedure-body procedure)
           (extend-environment
             (procedure-parameters procedure)
             arguments
             (procedure-environment procedure))))
        (else
         (error
          "Unknown procedure type: APPLY" procedure))))
\end{scheme}

\subsubsection*{Procedure arguments}

When \code{eval} processes a procedure application, it uses
\code{list\-/of\-/values} to produce the list of arguments to which the procedure
is to be applied. \code{List\-/of\-/values} takes as an argument the operands of
the combination.  It evaluates each operand and returns a list of the
corresponding values:\footnote{We could have simplified the \code{application?}
clause in \code{eval} by using \code{map} (and stipulating that \code{operands}
returns a list) rather than writing an explicit \code{list\-/of\-/values}
procedure.  We chose not to use \code{map} here to emphasize the fact that the
evaluator can be implemented without any use of higher-order procedures (and
thus could be written in a language that doesn't have higher-order procedures),
even though the language that it supports will include higher-order
procedures.}

\begin{scheme}
(define (list-of-values exps env)
  (if (no-operands? exps)
      '()
      (cons (eval (first-operand exps) env)
            (list-of-values (rest-operands exps) env))))
\end{scheme}

\subsubsection*{Conditionals}

\code{Eval\-/if} evaluates the predicate part of an \code{if} expression in the
given environment.  If the result is true, \code{eval\-/if} evaluates the
consequent, otherwise it evaluates the alternative:

\begin{scheme}
(define (eval-if exp env)
  (if (true? (eval (if-predicate exp) env))
      (eval (if-consequent exp) env)
      (eval (if-alternative exp) env)))
\end{scheme}

\noindent
The use of \code{true?} in \code{eval\-/if} highlights the issue of the
connection between an implemented language and an implementation language.  The
\code{if\-/predicate} is evaluated in the language being implemented and thus
yields a value in that language.  The interpreter predicate \code{true?}
translates that value into a value that can be tested by the \code{if} in the
implementation language: The metacircular representation of truth might not be
the same as that of the underlying Scheme.\footnote{In this case, the language
being implemented and the implementation language are the same.  Contemplation
of the meaning of \code{true?} here yields expansion of consciousness without
the abuse of substance.}

\subsubsection*{Sequences}

\code{Eval\-/sequence} is used by \code{apply} to evaluate the sequence of
expressions in a procedure body and by \code{eval} to evaluate the sequence of
expressions in a \code{begin} expression.  It takes as arguments a sequence of
expressions and an environment, and evaluates the expressions in the order in
which they occur.  The value returned is the value of the final expression.

\begin{scheme}
(define (eval-sequence exps env)
  (cond ((last-exp? exps)
         (eval (first-exp exps) env))
        (else
         (eval (first-exp exps) env)
         (eval-sequence (rest-exps exps) env))))
\end{scheme}

\subsubsection*{Assignments and definitions}

The following procedure handles assignments to variables.  It calls \code{eval}
to find the value to be assigned and transmits the variable and the resulting
value to \code{set\-/variable\-/value!} to be installed in the designated
environment.

\begin{scheme}
(define (eval-assignment exp env)
  (set-variable-value! (assignment-variable exp)
                       (eval (assignment-value exp) env)
                       env)
  'ok)
\end{scheme}

\noindent
Definitions of variables are handled in a similar manner.\footnote{This
implementation of \code{define} ignores a subtle issue in the handling of
internal definitions, although it works correctly in most cases.  We will see
what the problem is and how to solve it in \link{Section 4.1.6}.}

\begin{scheme}
(define (eval-definition exp env)
  (define-variable! (definition-variable exp)
                    (eval (definition-value exp) env)
                    env)
  'ok)
\end{scheme}

\noindent
We have chosen here to return the symbol \code{ok} as the value of an
assignment or a definition.\footnote{As we said when we introduced
\code{define} and \code{set!}, these values are implementation-dependent in
Scheme---that is, the implementor can choose what value to return.}

\begin{quote}
\heading{\phantomsection\label{Exercise 4.1}Exercise 4.1:} Notice that we cannot tell whether
the metacircular evaluator evaluates operands from left to right or from right
to left.  Its evaluation order is inherited from the underlying Lisp: If the
arguments to \code{cons} in \code{list\-/of\-/values} are evaluated from left to
right, then \code{list\-/of\-/values} will evaluate operands from left to right;
and if the arguments to \code{cons} are evaluated from right to left, then
\code{list\-/of\-/values} will evaluate operands from right to left.

Write a version of \code{list\-/of\-/values} that evaluates operands from left to
right regardless of the order of evaluation in the underlying Lisp.  Also write
a version of \code{list\-/of\-/values} that evaluates operands from right to left.
\end{quote}

\subsection{Representing Expressions}
\label{Section 4.1.2}

The evaluator is reminiscent of the symbolic differentiation program discussed
in \link{Section 2.3.2}.  Both programs operate on symbolic expressions.  In
both programs, the result of operating on a compound expression is determined
by operating recursively on the pieces of the expression and combining the
results in a way that depends on the type of the expression.  In both programs
we used data abstraction to decouple the general rules of operation from the
details of how expressions are represented.  In the differentiation program
this meant that the same differentiation procedure could deal with algebraic
expressions in prefix form, in infix form, or in some other form.  For the
evaluator, this means that the syntax of the language being evaluated is
determined solely by the procedures that classify and extract pieces of
expressions.

Here is the specification of the syntax of our language:

\begin{itemize}

\item
The only self-evaluating items are numbers and strings:

\begin{scheme}
(define (self-evaluating? exp)
  (cond ((number? exp) true)
        ((string? exp) true)
        (else false)))
\end{scheme}

\item
Variables are represented by symbols:

\begin{scheme}
(define (variable? exp) (symbol? exp))
\end{scheme}

\item
Quotations have the form \code{(quote \( \langle \)\var{text\-/of\-/quotation}\( \rangle \))}:\footnote{As
mentioned in \link{Section 2.3.1}, the evaluator sees a quoted expression as a
list beginning with \code{quote}, even if the expression is typed with the
quotation mark.  For example, the expression \code{'a} would be seen by the
evaluator as \code{(quote a)}.  See \link{Exercise 2.55}.}

\begin{scheme}
(define (quoted? exp) (tagged-list? exp 'quote))
(define (text-of-quotation exp) (cadr exp))
\end{scheme}

\code{Quoted?} is defined in terms of the procedure \code{tagged\-/list?}, which
identifies lists beginning with a designated symbol:

\begin{scheme}
(define (tagged-list? exp tag)
  (if (pair? exp)
      (eq? (car exp) tag)
      false))
\end{scheme}

\item
Assignments have the form \code{(set! \( \langle \)\var{var}\( \rangle \)
\( \langle \)\var{value}\( \rangle \))}:

\begin{scheme}
(define (assignment? exp) (tagged-list? exp 'set!))
(define (assignment-variable exp) (cadr exp))
(define (assignment-value exp) (caddr exp))
\end{scheme}

\item
Definitions have the form

\begin{scheme}
(define ~\( \dark \langle \)~~\var{\dark var}~~\( \dark \rangle \)~ ~\( \dark \langle \)~~\var{\dark value}~~\( \dark \rangle \)~)
\end{scheme}

\noindent
or the form

\begin{scheme}
(define (~\( \dark \langle \)~~\var{\dark var}~~\( \dark \rangle \)~ ~\( \dark \langle \)~~\( \dark parameter_1 \)~~\( \dark \rangle \)~ ~\( \dots \)~ ~\( \dark \langle \)~~\( \dark parameter_n \)~~\( \dark \rangle \)~)
  ~\( \dark \langle \)~~\var{\dark body}~~\( \dark \rangle \)~)
\end{scheme}

The latter form (standard procedure definition) is syntactic sugar for

\begin{scheme}
(define ~\( \dark \langle \)~~\var{\dark var}~~\( \dark \rangle \)~
  (lambda (~\( \dark \langle \)~~\( \dark parameter_1 \)~~\( \dark \rangle \)~ ~\( \dots \)~ ~\( \dark \langle \)~~\( \dark parameter_n \)~~\( \dark \rangle \)~)
    ~\( \dark \langle \)~~\var{\dark body}~~\( \dark \rangle \)~))
\end{scheme}

The corresponding syntax procedures are the following:

\begin{scheme}
(define (definition? exp) (tagged-list? exp 'define))
(define (definition-variable exp)
  (if (symbol? (cadr exp))
      (cadr exp)
      (caadr exp)))
(define (definition-value exp)
  (if (symbol? (cadr exp))
      (caddr exp)
      (make-lambda (cdadr exp)     ~\textrm{; formal parameters}~
                   (cddr exp))))   ~\textrm{; body}~
\end{scheme}

\item
\code{Lambda} expressions are lists that begin with the symbol \code{lambda}:

\begin{scheme}
(define (lambda? exp) (tagged-list? exp 'lambda))
(define (lambda-parameters exp) (cadr exp))
(define (lambda-body exp) (cddr exp))
\end{scheme}

We also provide a constructor for \code{lambda} expressions, which is used by
\code{definition\-/value}, above:

\begin{scheme}
(define (make-lambda parameters body)
  (cons 'lambda (cons parameters body)))
\end{scheme}

\item
Conditionals begin with \code{if} and have a predicate, a consequent, and an
(optional) alternative.  If the expression has no alternative part, we provide
\code{false} as the alternative.\footnote{The value of an \code{if} expression
when the predicate is false and there is no alternative is unspecified in
Scheme; we have chosen here to make it false.  We will support the use of the
variables \code{true} and \code{false} in expressions to be evaluated by
binding them in the global environment.  See \link{Section 4.1.4}.}

\begin{scheme}
(define (if? exp) (tagged-list? exp 'if))
(define (if-predicate exp) (cadr exp))
(define (if-consequent exp) (caddr exp))
(define (if-alternative exp)
  (if (not (null? (cdddr exp)))
      (cadddr exp)
      'false))
\end{scheme}

We also provide a constructor for \code{if} expressions, to be used by
\code{cond\-/>if} to transform \code{cond} expressions into \code{if}
expressions:

\begin{scheme}
(define (make-if predicate consequent alternative)
  (list 'if predicate consequent alternative))
\end{scheme}

\item
\code{Begin} packages a sequence of expressions into a single expression.  We
include syntax operations on \code{begin} expressions to extract the actual
sequence from the \code{begin} expression, as well as selectors that return the
first expression and the rest of the expressions in the
sequence.\footnote{These selectors for a list of expressions---and the
corresponding ones for a list of operands---are not intended as a data
abstraction.  They are introduced as mnemonic names for the basic list
operations in order to make it easier to understand the explicit-control
evaluator in \link{Section 5.4}.}

\begin{scheme}
(define (begin? exp) (tagged-list? exp 'begin))
(define (begin-actions exp) (cdr exp))
(define (last-exp? seq) (null? (cdr seq)))
(define (first-exp seq) (car seq))
(define (rest-exps seq) (cdr seq))
\end{scheme}

We also include a constructor \code{sequence\-/>exp} (for use by \code{cond\-/>if})
that transforms a sequence into a single expression, using \code{begin} if
necessary:

\begin{scheme}
(define (sequence->exp seq)
  (cond ((null? seq) seq)
        ((last-exp? seq) (first-exp seq))
        (else (make-begin seq))))
(define (make-begin seq) (cons 'begin seq))
\end{scheme}

\item
A procedure application is any compound expression that is not one of the above
expression types.  The \code{car} of the expression is the operator, and the
\code{cdr} is the list of operands:

\begin{scheme}
(define (application? exp) (pair? exp))
(define (operator exp) (car exp))
(define (operands exp) (cdr exp))
(define (no-operands? ops) (null? ops))
(define (first-operand ops) (car ops))
(define (rest-operands ops) (cdr ops))
\end{scheme}

\end{itemize}

\subsubsection*{Derived expressions}

Some special forms in our language can be defined in terms of expressions
involving other special forms, rather than being implemented directly.  One
example is \code{cond}, which can be implemented as a nest of \code{if}
expressions.  For example, we can reduce the problem of evaluating the
expression

\begin{scheme}
(cond ((> x 0) x)
      ((= x 0) (display 'zero) 0)
      (else (- x)))
\end{scheme}

\noindent
to the problem of evaluating the following expression involving \code{if} and
\code{begin} expressions:

\begin{scheme}
(if (> x 0)
    x
    (if (= x 0)
        (begin (display 'zero) 0)
        (- x)))
\end{scheme}

\noindent
Implementing the evaluation of \code{cond} in this way simplifies the evaluator
because it reduces the number of special forms for which the evaluation process
must be explicitly specified.

We include syntax procedures that extract the parts of a \code{cond}
expression, and a procedure \code{cond\-/>if} that transforms \code{cond}
expressions into \code{if} expressions.  A case analysis begins with
\code{cond} and has a list of predicate-action clauses.  A clause is an
\code{else} clause if its predicate is the symbol \code{else}.\footnote{The
value of a \code{cond} expression when all the predicates are false and there
is no \code{else} clause is unspecified in Scheme; we have chosen here to make
it false.}

\begin{scheme}
(define (cond? exp) (tagged-list? exp 'cond))
(define (cond-clauses exp) (cdr exp))
(define (cond-else-clause? clause)
  (eq? (cond-predicate clause) 'else))
(define (cond-predicate clause) (car clause))
(define (cond-actions clause) (cdr clause))
(define (cond->if exp) (expand-clauses (cond-clauses exp)))
(define (expand-clauses clauses)
  (if (null? clauses)
      'false                        ~\textrm{; no \code{else} clause}~
      (let ((first (car clauses))
            (rest (cdr clauses)))
        (if (cond-else-clause? first)
            (if (null? rest)
                (sequence->exp (cond-actions first))
                (error "ELSE clause isn't last: COND->IF"
                       clauses))
            (make-if (cond-predicate first)
                     (sequence->exp (cond-actions first))
                     (expand-clauses rest))))))
\end{scheme}

\noindent
Expressions (such as \code{cond}) that we choose to implement as syntactic
transformations are called \newterm{derived expressions}.  \code{Let}
expressions are also derived expressions (see 
\link{Exercise 4.6}).\footnote{Practical Lisp systems provide a mechanism that allows a user
to add new derived expressions and specify their implementation as syntactic
transformations without modifying the evaluator.  Such a user-defined
transformation is called a \newterm{macro}.  Although it is easy to add an
elementary mechanism for defining macros, the resulting language has subtle
name-conflict problems.  There has been much research on mechanisms for macro
definition that do not cause these difficulties.  See, for example, \link{Kohlbecker 1986}, 
\link{Clinger and Rees 1991}, and \link{Hanson 1991}.}

\begin{quote}
\heading{\phantomsection\label{Exercise 4.2}Exercise 4.2:} Louis Reasoner plans to reorder the
\code{cond} clauses in \code{eval} so that the clause for procedure
applications appears before the clause for assignments.  He argues that this
will make the interpreter more efficient: Since programs usually contain more
applications than assignments, definitions, and so on, his modified \code{eval}
will usually check fewer clauses than the original \code{eval} before
identifying the type of an expression.

\begin{enumerate}[a.]

\item
What is wrong with Louis's plan?  (Hint: What will Louis's evaluator do with
the expression \code{(define x 3)}?)

\item
Louis is upset that his plan didn't work.  He is willing to go to any lengths
to make his evaluator recognize procedure applications before it checks for
most other kinds of expressions.  Help him by changing the syntax of the
evaluated language so that procedure applications start with \code{call}.  For
example, instead of \code{(factorial 3)} we will now have to write \code{(call
factorial 3)} and instead of \code{(+ 1 2)} we will have to write \code{(call +
1 2)}.

\end{enumerate}
\end{quote}

\begin{quote}
\heading{\phantomsection\label{Exercise 4.3}Exercise 4.3:} Rewrite \code{eval} so that the
dispatch is done in data-directed style.  Compare this with the data-directed
differentiation procedure of \link{Exercise 2.73}.  (You may use the \code{car}
of a compound expression as the type of the expression, as is appropriate for
the syntax implemented in this section.)
\end{quote}

\begin{quote}
\heading{\phantomsection\label{Exercise 4.4}Exercise 4.4:} Recall the definitions of the
special forms \code{and} and \code{or} from \link{Chapter 1}:

\begin{itemize}

\item
\code{and}: The expressions are evaluated from left to right.  If any
expression evaluates to false, false is returned; any remaining expressions are
not evaluated.  If all the expressions evaluate to true values, the value of
the last expression is returned.  If there are no expressions then true is
returned.

\item
\code{or}: The expressions are evaluated from left to right.  If any expression
evaluates to a true value, that value is returned; any remaining expressions
are not evaluated.  If all expressions evaluate to false, or if there are no
expressions, then false is returned.

\end{itemize}

Install \code{and} and \code{or} as new special forms for the evaluator by
defining appropriate syntax procedures and evaluation procedures
\code{eval\-/and} and \code{eval\-/or}.  Alternatively, show how to implement
\code{and} and \code{or} as derived expressions.
\end{quote}

\begin{quote}
\heading{\phantomsection\label{Exercise 4.5}Exercise 4.5:} Scheme allows an additional syntax
for \code{cond} clauses, \code{(\( \langle \)\var{test}\( \rangle \) => \( \langle \)\var{recipient}\( \rangle \))}.  If
\( \langle \)\var{test}\( \kern0.08em\rangle \) evaluates to a true value, then \( \langle \)\var{recipient}\( \kern0.08em\rangle \) is evaluated.
Its value must be a procedure of one argument; this procedure is then invoked
on the value of the \( \langle \)\var{test}\( \kern0.08em\rangle \), and the result is returned as the value of
the \code{cond} expression.  For example

\begin{scheme}
(cond ((assoc 'b '((a 1) (b 2))) => cadr)
      (else false))
\end{scheme}

\noindent
returns 2.  Modify the handling of \code{cond} so that it supports this
extended syntax.
\end{quote}

\begin{quote}
\heading{\phantomsection\label{Exercise 4.6}Exercise 4.6:} \code{Let} expressions are derived
expressions, because

\begin{scheme}
(let ((~\( \dark \langle \)~~\( \dark var_1 \)~~\( \dark \rangle \)~ ~\( \dark \langle \)~~\( \dark exp_1 \)~~\( \dark \rangle \)~) ~\( \dots \)~ (~\( \dark \langle \)~~\( \dark var_n \)~~\( \dark \rangle \)~ ~\( \dark \langle \)~~\( \dark exp_n \)~~\( \dark \rangle \)~))
  ~\( \dark \langle \)~~\var{\dark body}~~\( \dark \rangle \)~)
\end{scheme}

\noindent
is equivalent to

\begin{scheme}
((lambda (~\( \dark \langle \)~~\( \dark var_1 \)~~\( \dark \rangle \)~ ~\( \dots \)~ ~\( \dark \langle \)~~\( \dark var_n \)~~\( \dark \rangle \)~)
   ~\( \dark \langle \)~~\var{\dark body}~~\( \dark \rangle \)~)
 ~\( \dark \langle \)~~\( \dark exp_1 \)~~\( \dark \rangle \)~
 ~\( \dots \)~
 ~\( \dark \langle \)~~\( \dark exp_n \)~~\( \dark \rangle \)~)
\end{scheme}

Implement a syntactic transformation \code{let\-/>combination} that reduces
evaluating \code{let} expressions to evaluating combinations of the type shown
above, and add the appropriate clause to \code{eval} to handle \code{let}
expressions.
\end{quote}

\begin{quote}
\heading{\phantomsection\label{Exercise 4.7}Exercise 4.7:} \code{Let*} is similar to
\code{let}, except that the bindings of the \code{let*} variables are performed
sequentially from left to right, and each binding is made in an environment in
which all of the preceding bindings are visible.  For example

\begin{scheme}
(let* ((x 3)  (y (+ x 2))  (z (+ x y 5)))
  (* x z))
\end{scheme}

\noindent
returns 39.  Explain how a \code{let*} expression can be rewritten as a set of
nested \code{let} expressions, and write a procedure \code{let*\-/>nested\-/lets}
that performs this transformation.  If we have already implemented \code{let}
(\link{Exercise 4.6}) and we want to extend the evaluator to handle \code{let*},
is it sufficient to add a clause to \code{eval} whose action is

\begin{scheme}
(eval (let*->nested-lets exp) env)
\end{scheme}

\noindent
or must we explicitly expand \code{let*} in terms of non-derived expressions?
\end{quote}

\begin{quote}
\heading{\phantomsection\label{Exercise 4.8}Exercise 4.8:} ``Named \code{let}'' is a variant
of \code{let} that has the form

\begin{scheme}
(let ~\( \dark \langle \)~~\var{\dark var}~~\( \dark \rangle \)~ ~\( \dark \langle \)~~\var{\dark bindings}~~\( \dark \rangle \)~ ~\( \dark \langle \)~~\var{\dark body}~~\( \dark \rangle \)~)
\end{scheme}

The \( \langle \)\var{bindings}\( \kern0.08em\rangle \) and \( \langle \)\var{body}\( \kern0.08em\rangle \) are just as in ordinary \code{let},
except that \( \langle \)\var{var}\( \kern0.08em\rangle \) is bound within \( \langle \)\var{body}\( \kern0.08em\rangle \) to a procedure whose body
is \( \langle \)\var{body}\( \kern0.08em\rangle \) and whose parameters are the variables in the \( \langle \)\var{bindings}\( \kern0.08em\rangle \).
Thus, one can repeatedly execute the \( \langle \)\var{body}\( \kern0.08em\rangle \) by invoking the procedure
named \( \langle \)\var{var}\( \kern0.08em\rangle \).  For example, the iterative Fibonacci procedure 
(\link{Section 1.2.2}) can be rewritten using named \code{let} as follows:

\begin{scheme}

(define (fib n)
  (let fib-iter ((a 1)
                 (b 0)
                 (count n))
    (if (= count 0)
        b
        (fib-iter (+ a b) a (- count 1)))))
\end{scheme}

Modify \code{let\-/>combination} of \link{Exercise 4.6} to also support named
\code{let}.
\end{quote}

\begin{quote}
\heading{\phantomsection\label{Exercise 4.9}Exercise 4.9:} Many languages support a variety of
iteration constructs, such as \code{do}, \code{for}, \code{while}, and
\code{until}.  In Scheme, iterative processes can be expressed in terms of
ordinary procedure calls, so special iteration constructs provide no essential
gain in computational power.  On the other hand, such constructs are often
convenient.  Design some iteration constructs, give examples of their use, and
show how to implement them as derived expressions.
\end{quote}

\begin{quote}
\heading{\phantomsection\label{Exercise 4.10}Exercise 4.10:} By using data abstraction, we
were able to write an \code{eval} procedure that is independent of the
particular syntax of the language to be evaluated.  To illustrate this, design
and implement a new syntax for Scheme by modifying the procedures in this
section, without changing \code{eval} or \code{apply}.
\end{quote}

\subsection{Evaluator Data Structures}
\label{Section 4.1.3}

In addition to defining the external syntax of expressions, the evaluator
implementation must also define the data structures that the evaluator
manipulates internally, as part of the execution of a program, such as the
representation of procedures and environments and the representation of true
and false.

\subsubsection*{Testing of predicates}

For conditionals, we accept anything to be true that is not the explicit
\code{false} object.

\begin{scheme}
(define (true? x)  (not (eq? x false)))
(define (false? x) (eq? x false))
\end{scheme}

\subsubsection*{Representing procedures}

To handle primitives, we assume that we have available the following
procedures:

\begin{itemize}

\item
\code{(apply\-/primitive\-/procedure \( \langle \)\var{proc}\( \rangle \) \( \langle \)\var{args}\( \rangle \))}

\noindent
applies the given primitive procedure to the argument values in the list
\( \langle \)\var{args}\( \kern0.08em\rangle \) and returns the result of the application.

\item
\code{(primitive\-/procedure? \( \langle \)\var{proc}\( \rangle \))}

\noindent
tests whether \( \langle \)\var{proc}\( \kern0.08em\rangle \) is a primitive procedure.

\end{itemize}

\noindent
These mechanisms for handling primitives are further described in 
\link{Section 4.1.4}.

Compound procedures are constructed from parameters, procedure bodies, and
environments using the constructor \code{make\-/procedure}:

\begin{scheme}
(define (make-procedure parameters body env)
  (list 'procedure parameters body env))
(define (compound-procedure? p)
  (tagged-list? p 'procedure))
(define (procedure-parameters p) (cadr p))
(define (procedure-body p) (caddr p))
(define (procedure-environment p) (cadddr p))
\end{scheme}

\subsubsection*{Operations on Environments}

The evaluator needs operations for manipulating environments.  As explained in
\link{Section 3.2}, an environment is a sequence of frames, where each frame is
a table of bindings that associate variables with their corresponding values.
We use the following operations for manipulating environments:

\begin{itemize}

\item
\code{(lookup\-/variable\-/value \( \langle \)\var{var}\( \rangle \) \( \langle \)\var{env}\( \rangle \))}

returns the value that is bound to the symbol \( \langle \)\var{var}\( \kern0.08em\rangle \) in the environment
\( \langle \)\var{env}\( \kern0.08em\rangle \), or signals an error if the variable is unbound.

\item
\code{(extend\-/environment \( \langle \)\var{variables}\( \rangle \) \( \langle \)\var{values}\( \rangle \) \( \langle \)\var{base\-/env}\( \rangle \))}

returns a new environment, consisting of a new frame in which the symbols in
the list \( \langle \)\var{variables}\( \kern0.08em\rangle \) are bound to the corresponding elements in the list
\( \langle \)\var{values}\( \kern0.08em\rangle \), where the enclosing environment is the environment
\( \langle \)\var{base-env}\( \kern0.08em\rangle \).

\item
\code{(define\-/variable! \( \langle \)\var{var}\( \rangle \) \( \langle \)\var{value}\( \rangle \) \( \langle \)\var{env}\( \rangle \))}

adds to the first frame in the environment \( \langle \)\var{env}\( \kern0.08em\rangle \) a new binding that
associates the variable \( \langle \)\var{var}\( \kern0.08em\rangle \) with the value \( \langle \)\var{value}\( \kern0.08em\rangle \).

\item
\code{(set\-/variable\-/value! \( \langle \)\var{var}\( \rangle \) \( \langle \)\var{value}\( \rangle \) \( \langle \)\var{env}\( \rangle \))}

changes the binding of the variable \( \langle \)\var{var}\( \kern0.08em\rangle \) in the environment \( \langle \)\var{env}\( \kern0.08em\rangle \)
so that the variable is now bound to the value \( \langle \)\var{value}\( \kern0.08em\rangle \), or signals an
error if the variable is unbound.

\end{itemize}

\noindent
To implement these operations we represent an environment as a list of frames.
The enclosing environment of an environment is the \code{cdr} of the list.  The
empty environment is simply the empty list.

\begin{scheme}
(define (enclosing-environment env) (cdr env))
(define (first-frame env) (car env))
(define the-empty-environment '())
\end{scheme}

\noindent
Each frame of an environment is represented as a pair of lists: a list of the
variables bound in that frame and a list of the associated
values.\footnote{Frames are not really a data abstraction in the following
code: \code{Set\-/variable\-/value!} and \code{define\-/variable!} use
\code{set\-/car!}  to directly modify the values in a frame.  The purpose of the
frame procedures is to make the environment-manipulation procedures easy to
read.}

\begin{scheme}
(define (make-frame variables values)
  (cons variables values))
(define (frame-variables frame) (car frame))
(define (frame-values frame) (cdr frame))
(define (add-binding-to-frame! var val frame)
  (set-car! frame (cons var (car frame)))
  (set-cdr! frame (cons val (cdr frame))))
\end{scheme}

\noindent
To extend an environment by a new frame that associates variables with values,
we make a frame consisting of the list of variables and the list of values, and
we adjoin this to the environment.  We signal an error if the number of
variables does not match the number of values.

\begin{scheme}
(define (extend-environment vars vals base-env)
  (if (= (length vars) (length vals))
      (cons (make-frame vars vals) base-env)
      (if (< (length vars) (length vals))
          (error "Too many arguments supplied" vars vals)
          (error "Too few arguments supplied" vars vals))))
\end{scheme}

\noindent
To look up a variable in an environment, we scan the list of variables in the
first frame.  If we find the desired variable, we return the corresponding
element in the list of values.  If we do not find the variable in the current
frame, we search the enclosing environment, and so on.  If we reach the empty
environment, we signal an ``unbound variable'' error.

\begin{scheme}
(define (lookup-variable-value var env)
  (define (env-loop env)
    (define (scan vars vals)
      (cond ((null? vars)
             (env-loop (enclosing-environment env)))
            ((eq? var (car vars)) (car vals))
            (else (scan (cdr vars) (cdr vals)))))
    (if (eq? env the-empty-environment)
        (error "Unbound variable" var)
        (let ((frame (first-frame env)))
          (scan (frame-variables frame)
                (frame-values frame)))))
  (env-loop env))
\end{scheme}

\noindent
To set a variable to a new value in a specified environment, we scan for the
variable, just as in \code{lookup\-/variable\-/value}, and change the corresponding
value when we find it.

\begin{scheme}
(define (set-variable-value! var val env)
  (define (env-loop env)
    (define (scan vars vals)
      (cond ((null? vars)
             (env-loop (enclosing-environment env)))
            ((eq? var (car vars)) (set-car! vals val))
            (else (scan (cdr vars) (cdr vals)))))
    (if (eq? env the-empty-environment)
        (error "Unbound variable: SET!" var)
        (let ((frame (first-frame env)))
          (scan (frame-variables frame)
                (frame-values frame)))))
  (env-loop env))
\end{scheme}

\noindent
To define a variable, we search the first frame for a binding for the variable,
and change the binding if it exists (just as in \code{set\-/variable\-/value!}).
If no such binding exists, we adjoin one to the first frame.

\begin{scheme}
(define (define-variable! var val env)
  (let ((frame (first-frame env)))
    (define (scan vars vals)
      (cond ((null? vars)
             (add-binding-to-frame! var val frame))
            ((eq? var (car vars)) (set-car! vals val))
            (else (scan (cdr vars) (cdr vals)))))
    (scan (frame-variables frame) (frame-values frame))))
\end{scheme}

\noindent
The method described here is only one of many plausible ways to represent
environments.  Since we used data abstraction to isolate the rest of the
evaluator from the detailed choice of representation, we could change the
environment representation if we wanted to.  (See \link{Exercise 4.11}.)  In a
production-quality Lisp system, the speed of the evaluator's environment
operations---especially that of variable lookup---has a major impact on the
performance of the system.  The representation described here, although
conceptually simple, is not efficient and would not ordinarily be used in a
production system.\footnote{The drawback of this representation (as well as the
variant in \link{Exercise 4.11}) is that the evaluator may have to search
through many frames in order to find the binding for a given variable.  (Such
an approach is referred to as \newterm{deep binding}.)  One way to avoid this
inefficiency is to make use of a strategy called \newterm{lexical addressing},
which will be discussed in \link{Section 5.5.6}.}

\begin{quote}
\heading{\phantomsection\label{Exercise 4.11}Exercise 4.11:} Instead of representing a frame
as a pair of lists, we can represent a frame as a list of bindings, where each
binding is a name-value pair.  Rewrite the environment operations to use this
alternative representation.
\end{quote}

\begin{quote}
\heading{\phantomsection\label{Exercise 4.12}Exercise 4.12:} The procedures
\code{set\-/variable\-/value!}, \code{define\-/variable!} and
\code{lookup\-/variable\-/value} can be expressed in terms of more abstract
procedures for traversing the environment structure.  Define abstractions that
capture the common patterns and redefine the three procedures in terms of these
abstractions.
\end{quote}

\begin{quote}
\heading{\phantomsection\label{Exercise 4.13}Exercise 4.13:} Scheme allows us to create new
bindings for variables by means of \code{define}, but provides no way to get
rid of bindings.  Implement for the evaluator a special form
\code{make\-/unbound!} that removes the binding of a given symbol from the
environment in which the \code{make\-/unbound!} expression is evaluated.  This
problem is not completely specified.  For example, should we remove only the
binding in the first frame of the environment?  Complete the specification and
justify any choices you make.
\end{quote}

\subsection{Running the Evaluator as a Program}
\label{Section 4.1.4}

Given the evaluator, we have in our hands a description (expressed in Lisp) of
the process by which Lisp expressions are evaluated.  One advantage of
expressing the evaluator as a program is that we can run the program.  This
gives us, running within Lisp, a working model of how Lisp itself evaluates
expressions.  This can serve as a framework for experimenting with evaluation
rules, as we shall do later in this chapter.

Our evaluator program reduces expressions ultimately to the application of
primitive procedures.  Therefore, all that we need to run the evaluator is to
create a mechanism that calls on the underlying Lisp system to model the
application of primitive procedures.

There must be a binding for each primitive procedure name, so that when
\code{eval} evaluates the operator of an application of a primitive, it will
find an object to pass to \code{apply}.  We thus set up a global environment
that associates unique objects with the names of the primitive procedures that
can appear in the expressions we will be evaluating.  The global environment
also includes bindings for the symbols \code{true} and \code{false}, so that
they can be used as variables in expressions to be evaluated.

\begin{scheme}
(define (setup-environment)
  (let ((initial-env
         (extend-environment (primitive-procedure-names)
                             (primitive-procedure-objects)
                             the-empty-environment)))
    (define-variable! 'true true initial-env)
    (define-variable! 'false false initial-env)
    initial-env))
(define the-global-environment (setup-environment))
\end{scheme}

\noindent
It does not matter how we represent the primitive procedure objects, so long as
\code{apply} can identify and apply them by using the procedures
\code{primitive\-/procedure?} and \code{apply\-/primitive\-/procedure}.  We have
chosen to represent a primitive procedure as a list beginning with the symbol
\code{primitive} and containing a procedure in the underlying Lisp that
implements that primitive.

\begin{scheme}
(define (primitive-procedure? proc)
  (tagged-list? proc 'primitive))
(define (primitive-implementation proc) (cadr proc))
\end{scheme}

\noindent
\code{Setup\-/environment} will get the primitive names and implementation
procedures from a list:\footnote{Any procedure defined in the underlying Lisp
can be used as a primitive for the metacircular evaluator.  The name of a
primitive installed in the evaluator need not be the same as the name of its
implementation in the underlying Lisp; the names are the same here because the
metacircular evaluator implements Scheme itself.  Thus, for example, we could
put \code{(list 'first car)} or \code{(list 'square (lambda (x) (* x x)))} in
the list of \code{primitive\-/procedures}.}

\begin{scheme}
(define primitive-procedures
  (list (list 'car car)
        (list 'cdr cdr)
        (list 'cons cons)
        (list 'null? null?)
        ~\( \dark \langle \)~~\var{\dark more primitives}~~\( \dark \rangle \)~ ))
(define (primitive-procedure-names)
  (map car primitive-procedures))
(define (primitive-procedure-objects)
  (map (lambda (proc) (list 'primitive (cadr proc)))
       primitive-procedures))
\end{scheme}

\noindent
To apply a primitive procedure, we simply apply the implementation procedure to
the arguments, using the underlying Lisp
system:\footnote{\code{Apply\-/in\-/underlying\-/scheme} is the \code{apply}
procedure we have used in earlier chapters.  The metacircular evaluator's
\code{apply} procedure (\link{Section 4.1.1}) models the working of this
primitive.  Having two different things called \code{apply} leads to a
technical problem in running the metacircular evaluator, because defining the
metacircular evaluator's \code{apply} will mask the definition of the
primitive.  One way around this is to rename the metacircular \code{apply} to
avoid conflict with the name of the primitive procedure.  We have assumed
instead that we have saved a reference to the underlying \code{apply} by doing

\begin{smallscheme}
(define apply-in-underlying-scheme apply)
\end{smallscheme}

\noindent
before defining the metacircular \code{apply}.  This allows us to access the
original version of \code{apply} under a different name.}

\begin{scheme}
(define (apply-primitive-procedure proc args)
  (apply-in-underlying-scheme
   (primitive-implementation proc) args))
\end{scheme}

\enlargethispage{\baselineskip}

\noindent
For convenience in running the metacircular evaluator, we provide a
\newterm{driver loop} that models the read-eval-print loop of the underlying
Lisp system.  It prints a \newterm{prompt}, reads an input expression,
evaluates this expression in the global environment, and prints the result.  We
precede each printed result by an \newterm{output prompt} so as to distinguish
the value of the expression from other output that may be printed.\footnote{The
primitive procedure \code{read} waits for input from the user, and returns the
next complete expression that is typed.  For example, if the user types
\code{(+ 23 x)}, \code{read} returns a three-element list containing the symbol
\code{+}, the number 23, and the symbol \code{x}.  If the user types \code{'x},
\code{read} returns a two-element list containing the symbol \code{quote} and
the symbol \code{x}.}

\begin{scheme}
(define input-prompt  ";;; M-Eval input:")
(define output-prompt ";;; M-Eval value:")
(define (driver-loop)
  (prompt-for-input input-prompt)
  (let ((input (read)))
    (let ((output (eval input the-global-environment)))
      (announce-output output-prompt)
      (user-print output)))
  (driver-loop))
(define (prompt-for-input string)
  (newline) (newline) (display string) (newline))
(define (announce-output string)
  (newline) (display string) (newline))
\end{scheme}

\noindent
We use a special printing procedure, \code{user\-/print}, to avoid printing the
environment part of a compound procedure, which may be a very long list (or may
even contain cycles).

\begin{scheme}
(define (user-print object)
  (if (compound-procedure? object)
      (display (list 'compound-procedure
                     (procedure-parameters object)
                     (procedure-body object)
                     '<procedure-env>))
      (display object)))
\end{scheme}

\noindent
Now all we need to do to run the evaluator is to initialize the global
environment and start the driver loop.  Here is a sample interaction:

\begin{scheme}
(define the-global-environment (setup-environment))
(driver-loop)

~\textit{;;; M-Eval input:}~
(define (append x y)
  (if (null? x)
      y
      (cons (car x) (append (cdr x) y))))
~\textit{;;; M-Eval value:}~
~\textit{ok}~
~\textit{;;; M-Eval input:}~
(append '(a b c) '(d e f))
~\textit{;;; M-Eval value:}~
~\textit{(a b c d e f)}~
\end{scheme}

% \vspace{1em}
\begin{quote}
\heading{\phantomsection\label{Exercise 4.14}Exercise 4.14:} Eva Lu Ator and Louis Reasoner
are each experimenting with the metacircular evaluator.  Eva types in the
definition of \code{map}, and runs some test programs that use it.  They work
fine.  Louis, in contrast, has installed the system version of \code{map} as a
primitive for the metacircular evaluator.  When he tries it, things go terribly
wrong.  Explain why Louis's \code{map} fails even though Eva's works.
\end{quote}

\subsection{Data as Programs}
\label{Section 4.1.5}

In thinking about a Lisp program that evaluates Lisp expressions, an analogy
might be helpful.  One operational view of the meaning of a program is that a
program is a description of an abstract (perhaps infinitely large) machine.
For example, consider the familiar program to compute factorials:

\begin{scheme}
(define (factorial n)
  (if (= n 1) 1 (* (factorial (- n 1)) n)))
\end{scheme}

\noindent
We may regard this program as the description of a machine containing parts
that decrement, multiply, and test for equality, together with a two-position
switch and another factorial machine. (The factorial machine is infinite
because it contains another factorial machine within it.)  \link{Figure 4.2} is
a flow diagram for the factorial machine, showing how the parts are wired
together.

In a similar way, we can regard the evaluator as a very special machine that
takes as input a description of a machine.  Given this input, the evaluator
configures itself to emulate the machine described.  For example, if we feed
our evaluator the definition of \code{factorial}, as shown in \link{Figure 4.3},
the evaluator will be able to compute factorials.

\begin{figure}[tb]
\phantomsection\label{Figure 4.2}
\centering
\begin{comment}
\heading{Figure 4.2:} The factorial program, viewed as an abstract machine.

\begin{example}
    +-----------------------------------+
    | factorial                   |1    |
    |              |1             V     |
    |              |           +-----+  |
    |              V           | #   |  |
    |           +-----+        |     |  |
6 --------*-----|  =  |------->|   #-+-----> 720
    |     |     +-----+        |  /  |  |
    |     |                    | #   |  |
    |     |                    +-----+  |
    |     |                       ^     |
    |     |                       |     |
    |     |                    +--+--+  |
    |     *------------------->|  *  |  |
    |     |                    +-----+  |
    |     V                       ^     |
    |  +-----+    +-----------+   |     |
    |  |  -  +--->| factorial +---+     |
    |  +-----+    +-----------+         |
    |     ^                             |
    |     |1                            |
    +-----------------------------------+
\end{example}
\end{comment}
\includegraphics[width=84mm]{fig/chap4/Fig4.2.pdf}
\begin{quote}
\heading{Figure 4.2:} The factorial program, viewed as an abstract machine.
\end{quote}
\end{figure}

From this perspective, our evaluator is seen to be a \newterm{universal
machine}.  It mimics other machines when these are described as Lisp
programs.\footnote{The fact that the machines are described in Lisp is
inessential.  If we give our evaluator a Lisp program that behaves as an
evaluator for some other language, say C, the Lisp evaluator will emulate the C
evaluator, which in turn can emulate any machine described as a C program.
Similarly, writing a Lisp evaluator in C produces a C program that can execute
any Lisp program.  The deep idea here is that any evaluator can emulate any
other.  Thus, the notion of ``what can in principle be computed'' (ignoring
practicalities of time and memory required) is independent of the language or
the computer, and instead reflects an underlying notion of
\newterm{computability}.  This was first demonstrated in a clear way by Alan
M. Turing (1912-1954), whose 1936 paper laid the foundations for theoretical
computer science.  In the paper, Turing presented a simple computational
model---now known as a \newterm{Turing machine}---and argued that any
``effective process'' can be formulated as a program for such a machine.  (This
argument is known as the \newterm{Church-Turing thesis}.)  Turing then
implemented a universal machine, i.e., a Turing machine that behaves as an
evaluator for Turing-machine programs.  He used this framework to demonstrate
that there are well-posed problems that cannot be computed by Turing machines
(see \link{Exercise 4.15}), and so by implication cannot be formulated as
``effective processes.''  Turing went on to make fundamental contributions to
practical computer science as well.  For example, he invented the idea of
structuring programs using general-purpose subroutines.  See \link{Hodges 1983} for a
biography of Turing.} This is striking. Try to imagine an analogous evaluator
for electrical circuits.  This would be a circuit that takes as input a signal
encoding the plans for some other circuit, such as a filter.  Given this input,
the circuit evaluator would then behave like a filter with the same
description.  Such a universal electrical circuit is almost unimaginably
complex.  It is remarkable that the program evaluator is a rather simple
program.\footnote{Some people find it counterintuitive that an evaluator, which
is implemented by a relatively simple procedure, can emulate programs that are
more complex than the evaluator itself.  The existence of a universal evaluator
machine is a deep and wonderful property of computation.  \newterm{Recursion
theory}, a branch of mathematical logic, is concerned with logical limits of
computation.  Douglas Hofstadter's beautiful book \textit{G\"odel, Escher, Bach}
explores some of these ideas (\link{Hofstadter 1979}).}

\begin{figure}[tb]
\phantomsection\label{Figure 4.3}
\centering
\begin{comment}
\heading{Figure 4.3:} The evaluator emulating a factorial machine.

\begin{example}
                   +--------+
            6 ---->|  eval  |----> 720
                   +--------+
                       /
             . . .    /  . . .
       . . .       ../. .      .
     .                           ..
    .   (define (factorial n)      . . .
   .      (if (= n 1)                   . .
    .         1                            .
    .         (* (factorial (- n 1)) n)))   .
      . .                       . .        .
          . .  . .      . . . .     . . . .
                   . ..
\end{example}
\end{comment}
\includegraphics[width=69mm]{fig/chap4/Fig4.3.pdf}
\par\bigskip
\noindent
\heading{Figure 4.3:} The evaluator emulating a factorial machine.
\end{figure}

Another striking aspect of the evaluator is that it acts as a bridge between
the data objects that are manipulated by our programming language and the
programming language itself.  Imagine that the evaluator program (implemented
in Lisp) is running, and that a user is typing expressions to the evaluator and
observing the results.  From the perspective of the user, an input expression
such as \code{(* x x)} is an expression in the programming language, which the
evaluator should execute.  From the perspective of the evaluator, however, the
expression is simply a list (in this case, a list of three symbols: \code{*},
\code{x}, and \code{x}) that is to be manipulated according to a well-defined
set of rules.

That the user's programs are the evaluator's data need not be a source of
confusion.  In fact, it is sometimes convenient to ignore this distinction, and
to give the user the ability to explicitly evaluate a data object as a Lisp
expression, by making \code{eval} available for use in programs.  Many Lisp
dialects provide a primitive \code{eval} procedure that takes as arguments an
expression and an environment and evaluates the expression relative to the
environment.\footnote{Warning: This \code{eval} primitive is not identical to
the \code{eval} procedure we imple- mented in \link{Section 4.1.1}, because it
uses \emph{actual} Scheme environments rather than the sample environment
structures we built in \link{Section 4.1.3}.  These actual environments cannot
be manipulated by the user as ordinary lists; they must be accessed via
\code{eval} or other special operations.  Similarly, the \code{apply} primitive
we saw earlier is not identical to the metacircular \code{apply}, because it
uses actual Scheme procedures rather than the procedure objects we constructed
in \link{Section 4.1.3} and \link{Section 4.1.4}.} Thus,

\begin{scheme}
(eval '(* 5 5) user-initial-environment)
\end{scheme}

\noindent
and

\begin{scheme}
(eval (cons '* (list 5 5)) user-initial-environment)
\end{scheme}

\noindent
will both return 25.\footnote{The \acronym{MIT} implementation of Scheme
includes \code{eval}, as well as a symbol \code{user\-/initial\-/environment} that
is bound to the initial environment in which the user's input expressions are
evaluated.}

\begin{quote}
\heading{\phantomsection\label{Exercise 4.15}Exercise 4.15:} Given a one-argument procedure
\code{p} and an object \code{a}, \code{p} is said to ``halt'' on \code{a} if
evaluating the expression \code{(p a)} returns a value (as opposed to
terminating with an error message or running forever).  Show that it is
impossible to write a procedure \code{halts?} that correctly determines whether
\code{p} halts on \code{a} for any procedure \code{p} and object \code{a}.  Use
the following reasoning: If you had such a procedure \code{halts?}, you could
implement the following program:

\begin{scheme}
(define (run-forever) (run-forever))
(define (try p)
  (if (halts? p p) (run-forever) 'halted))
\end{scheme}

Now consider evaluating the expression \code{(try try)} and show that any
possible outcome (either halting or running forever) violates the intended
behavior of \code{halts?}.\footnote{Although we stipulated that \code{halts?}
is given a procedure object, notice that this reasoning still applies even if
\code{halts?} can gain access to the procedure's text and its environment.
This is Turing's celebrated \newterm{Halting Theorem}, which gave the first
clear example of a \newterm{non-computable} problem, i.e., a well-posed task
that cannot be carried out as a computational procedure.}
\end{quote}

\subsection{Internal Definitions}
\label{Section 4.1.6}

Our environment model of evaluation and our metacircular evaluator execute
definitions in sequence, extending the environment frame one definition at a
time.  This is particularly convenient for interactive program development, in
which the programmer needs to freely mix the application of procedures with the
definition of new procedures.  However, if we think carefully about the
internal definitions used to implement block structure (introduced in 
\link{Section 1.1.8}), we will find that name-by-name extension of the environment may
not be the best way to define local variables.

Consider a procedure with internal definitions, such as

\begin{scheme}
(define (f x)
  (define (even? n) (if (= n 0) true  (odd?  (- n 1))))
  (define (odd? n)  (if (= n 0) false (even? (- n 1))))
  ~\( \dark \langle \)~~\var{\dark rest of body of \code{f}}~~\( \dark \rangle \)~)
\end{scheme}

\noindent
Our intention here is that the name \code{odd?} in the body of the procedure
\code{even?} should refer to the procedure \code{odd?} that is defined after
\code{even?}.  The scope of the name \code{odd?} is the entire body of
\code{f}, not just the portion of the body of \code{f} starting at the point
where the \code{define} for \code{odd?} occurs.  Indeed, when we consider that
\code{odd?} is itself defined in terms of \code{even?}---so that \code{even?}
and \code{odd?} are mutually recursive procedures---we see that the only
satisfactory interpretation of the two \code{define}s is to regard them as if
the names \code{even?} and \code{odd?} were being added to the environment
simultaneously.  More generally, in block structure, the scope of a local name
is the entire procedure body in which the \code{define} is evaluated.

As it happens, our interpreter will evaluate calls to \code{f} correctly, but
for an ``accidental'' reason: Since the definitions of the internal procedures
come first, no calls to these procedures will be evaluated until all of them
have been defined.  Hence, \code{odd?}  will have been defined by the time
\code{even?} is executed.  In fact, our sequential evaluation mechanism will
give the same result as a mechanism that directly implements simultaneous
definition for any procedure in which the internal definitions come first in a
body and evaluation of the value expressions for the defined variables doesn't
actually use any of the defined variables.  (For an example of a procedure that
doesn't obey these restrictions, so that sequential definition isn't equivalent
to simultaneous definition, see \link{Exercise 4.19}.)\footnote{Wanting programs
to not depend on this evaluation mechanism is the reason for the ``management
is not responsible'' remark in \link{Footnote 28} of \link{Chapter 1}.  By
insisting that internal definitions come first and do not use each other while
the definitions are being evaluated, the \acronym{IEEE} standard for Scheme
leaves implementors some choice in the mechanism used to evaluate these
definitions.  The choice of one evaluation rule rather than another here may
seem like a small issue, affecting only the interpretation of ``badly formed''
programs.  However, we will see in \link{Section 5.5.6} that moving to a model
of simultaneous scoping for internal definitions avoids some nasty difficulties
that would otherwise arise in implementing a compiler.}

There is, however, a simple way to treat definitions so that internally defined
names have truly simultaneous scope---just create all local variables that will
be in the current environment before evaluating any of the value expressions.
One way to do this is by a syntax transformation on \code{lambda} expressions.
Before evaluating the body of a \code{lambda} expression, we ``scan out'' and
eliminate all the internal definitions in the body.  The internally defined
variables will be created with a \code{let} and then set to their values by
assignment.  For example, the procedure

\begin{scheme}
(lambda ~\( \dark \langle \)~~\var{\dark vars}~~\( \dark \rangle \)~
  (define u ~\( \dark \langle \)~~\var{\dark e1}~~\( \dark \rangle \)~)
  (define v ~\( \dark \langle \)~~\var{\dark e2}~~\( \dark \rangle \)~)
  ~\( \dark \langle \)~~\var{\dark e3}~~\( \dark \rangle \)~)
\end{scheme}

\noindent
would be transformed into

\begin{scheme}
(lambda ~\( \dark \langle \)~~\var{\dark vars}~~\( \dark \rangle \)~
  (let ((u '*unassigned*)
        (v '*unassigned*))
    (set! u ~\( \dark \langle \)~~\var{\dark e1}~~\( \dark \rangle \)~)
    (set! v ~\( \dark \langle \)~~\var{\dark e2}~~\( \dark \rangle \)~)
    ~\( \dark \langle \)~~\var{\dark e3}~~\( \dark \rangle \)~))
\end{scheme}

\noindent
where \code{*unassigned*} is a special symbol that causes looking up a variable
to signal an error if an attempt is made to use the value of the
not-yet-assigned variable.

An alternative strategy for scanning out internal definitions is shown in
\link{Exercise 4.18}.  Unlike the transformation shown above, this enforces the
restriction that the defined variables' values can be evaluated without using
any of the variables' values.\footnote{The \acronym{IEEE} standard for Scheme
allows for different implementation strategies by specifying that it is up to
the programmer to obey this restriction, not up to the implementation to
enforce it.  Some Scheme implementations, including \acronym{MIT} Scheme, use
the transformation shown above.  Thus, some programs that don't obey this
restriction will in fact run in such implementations.}

\begin{quote}
\heading{\phantomsection\label{Exercise 4.16}Exercise 4.16:} In this exercise we implement the
method just described for interpreting internal definitions.  We assume that
the evaluator supports \code{let} (see \link{Exercise 4.6}).

\begin{enumerate}[a.]

\item
Change \code{lookup\-/variable\-/value} (\link{Section 4.1.3}) to signal an error if
the value it finds is the symbol \code{*unassigned*}.

\item
Write a procedure \code{scan\-/out\-/defines} that takes a procedure body and
returns an equivalent one that has no internal definitions, by making the
transformation described above.

\item
Install \code{scan\-/out\-/defines} in the interpreter, either in
\code{make\-/procedure} or in \code{procedure\-/body} (see \link{Section 4.1.3}).
Which place is better?  Why?

\end{enumerate}
\end{quote}

\begin{quote}
\heading{\phantomsection\label{Exercise 4.17}Exercise 4.17:} Draw diagrams of the environment
in effect when evaluating the expression \( \langle \)\var{e3}\( \kern0.1em\rangle \) in the procedure in the
text, comparing how this will be structured when definitions are interpreted
sequentially with how it will be structured if definitions are scanned out as
described.  Why is there an extra frame in the transformed program?  Explain
why this difference in environment structure can never make a difference in the
behavior of a correct program.  Design a way to make the interpreter implement
the ``simultaneous'' scope rule for internal definitions without constructing
the extra frame.
\end{quote}

\begin{quote}
\heading{\phantomsection\label{Exercise 4.18}Exercise 4.18:} Consider an alternative strategy
for scanning out definitions that translates the example in the text to

\begin{scheme}
(lambda ~\( \dark \langle \)~~\var{\dark vars}~~\( \dark \rangle \)~
  (let ((u '*unassigned*) (v '*unassigned*))
    (let ((a ~\( \dark \langle \)~~\var{\dark e1}~~\( \dark \rangle \)~) (b ~\( \dark \langle \)~~\var{\dark e2}~~\( \dark \rangle \)~))
      (set! u a)
      (set! v b))
    ~\( \dark \langle \)~~\var{\dark e3}~~\( \dark \rangle \)~))
\end{scheme}

Here \code{a} and \code{b} are meant to represent new variable names, created
by the interpreter, that do not appear in the user's program.  Consider the
\code{solve} procedure from \link{Section 3.5.4}:

\begin{scheme}
(define (solve f y0 dt)
  (define  y (integral (delay dy) y0 dt))
  (define dy (stream-map f y))
  y)
\end{scheme}

Will this procedure work if internal definitions are scanned out as shown in
this exercise?  What if they are scanned out as shown in the text?  Explain.
\end{quote}

\begin{quote}
\heading{\phantomsection\label{Exercise 4.19}Exercise 4.19:} Ben Bitdiddle, Alyssa P. Hacker,
and Eva Lu Ator are arguing about the desired result of evaluating the
expression

\enlargethispage{\baselineskip}

\begin{scheme}
(let ((a 1))
  (define (f x)
    (define b (+ a x))
    (define a 5)
    (+ a b))
  (f 10))
\end{scheme}

Ben asserts that the result should be obtained using the sequential rule for
\code{define}: \code{b} is defined to be 11, then \code{a} is defined to be 5,
so the result is 16.  Alyssa objects that mutual recursion requires the
simultaneous scope rule for internal procedure definitions, and that it is
unreasonable to treat procedure names differently from other names.  Thus, she
argues for the mechanism implemented in \link{Exercise 4.16}.  This would lead
to \code{a} being unassigned at the time that the value for \code{b} is to be
computed.  Hence, in Alyssa's view the procedure should produce an error.  Eva
has a third opinion.  She says that if the definitions of \code{a} and \code{b}
are truly meant to be simultaneous, then the value 5 for \code{a} should be
used in evaluating \code{b}.  Hence, in Eva's view \code{a} should be 5,
\code{b} should be 15, and the result should be 20.  Which (if any) of these
viewpoints do you support?  Can you devise a way to implement internal
definitions so that they behave as Eva prefers?\footnote{The \acronym{MIT}
implementors of Scheme support Alyssa on the following grounds: Eva is in
principle correct---the definitions should be regarded as simultaneous.  But
it seems difficult to implement a general, efficient mechanism that does what
Eva requires.  In the absence of such a mechanism, it is better to generate an
error in the difficult cases of simultaneous definitions (Alyssa's notion) than
to produce an incorrect answer (as Ben would have it).}
\end{quote}

\begin{quote}
\heading{\phantomsection\label{Exercise 4.20}Exercise 4.20:} Because internal definitions look
sequential but are actually simultaneous, some people prefer to avoid them
entirely, and use the special form \code{letrec} instead.  \code{Letrec} looks
like \code{let}, so it is not surprising that the variables it binds are bound
simultaneously and have the same scope as each other.  The sample procedure
\code{f} above can be written without internal definitions, but with exactly
the same meaning, as

\begin{scheme}
(define (f x)
  (letrec
    ((even? (lambda (n)
              (if (= n 0) true  (odd?  (- n 1)))))
     (odd?  (lambda (n)
              (if (= n 0) false (even? (- n 1))))))
    ~\( \dark \langle \)~~\var{\dark rest of body of \code{f}}~~\( \dark \rangle \)~))
\end{scheme}

\code{Letrec} expressions, which have the form

\begin{scheme}
(letrec ((~\( \dark \langle \)~~\( \dark var_1 \)~~\( \dark \rangle \)~ ~\( \dark \langle \)~~\( \dark exp_1 \)~~\( \dark \rangle \)~) ~\( \dots \)~ (~\( \dark \langle \)~~\( \dark var_n \)~~\( \dark \rangle \)~ ~\( \dark \langle \)~~\( \dark exp_n \)~~\( \dark \rangle \)~))
  ~\( \dark \langle \)~~\var{\dark body}~~\( \dark \rangle \)~)
\end{scheme}

\noindent
are a variation on \code{let} in which the expressions
\( \langle \)\( exp_k \)\( \kern0.08em\rangle \) that provide the initial values for the
variables \( \langle \)\( var_k \)\( \kern0.08em\rangle \) are evaluated in an environment
that includes all the \code{letrec} bindings.  This permits recursion in the
bindings, such as the mutual recursion of \code{even?} and \code{odd?} in the
example above, or the evaluation of 10 factorial with

\begin{scheme}
(letrec
  ((fact (lambda (n)
           (if (= n 1) 1 (* n (fact (- n 1)))))))
  (fact 10))
\end{scheme}

\begin{enumerate}[a.]

\item
Implement \code{letrec} as a derived expression, by transforming a
\code{letrec} expression into a \code{let} expression as shown in the text
above or in \link{Exercise 4.18}.  That is, the \code{letrec} variables should
be created with a \code{let} and then be assigned their values with
\code{set!}.

\item
Louis Reasoner is confused by all this fuss about internal definitions.  The
way he sees it, if you don't like to use \code{define} inside a procedure, you
can just use \code{let}.  Illustrate what is loose about his reasoning by
drawing an environment diagram that shows the environment in which the
\( \langle \)\var{rest of body of \code{f}}\( \kern0.08em\rangle \) is evaluated during evaluation of the
expression \code{(f 5)}, with \code{f} defined as in this exercise.  Draw an
environment diagram for the same evaluation, but with \code{let} in place of
\code{letrec} in the definition of \code{f}.

\end{enumerate}
\end{quote}

\begin{quote}
\heading{\phantomsection\label{Exercise 4.21}Exercise 4.21:} Amazingly, Louis's intuition in
\link{Exercise 4.20} is correct.  It is indeed possible to specify recursive
procedures without using \code{letrec} (or even \code{define}), although the
method for accomplishing this is much more subtle than Louis imagined.  The
following expression computes 10 factorial by applying a recursive factorial
procedure:\footnote{This example illustrates a programming trick for
formulating recursive procedures without using \code{define}.  The most general
trick of this sort is the \( Y \) \newterm{operator}, which can be used to give a
``pure λ-calculus'' implementation of recursion.  (See \link{Stoy 1977} for
details on the λ-calculus, and \link{Gabriel 1988} for an exposition of the
\( Y \) operator in Scheme.)}

\begin{scheme}
((lambda (n)
   ((lambda (fact) (fact fact n))
    (lambda (ft k) (if (= k 1) 1 (* k (ft ft (- k 1)))))))
 10)
\end{scheme}

\begin{enumerate}[a.]

\item
Check (by evaluating the expression) that this really does compute factorials.
Devise an analogous expression for computing Fibonacci numbers.

\item
Consider the following procedure, which includes mutually recursive internal
definitions:

\begin{scheme}
(define (f x)
  (define (even? n)
    (if (= n 0) true  (odd?  (- n 1))))
  (define (odd? n)
    (if (= n 0) false (even? (- n 1))))
  (even? x))
\end{scheme}

Fill in the missing expressions to complete an alternative definition of
\code{f}, which uses neither internal definitions nor \code{letrec}:

\begin{scheme}
(define (f x)
  ((lambda (even? odd?) (even? even? odd? x))
   (lambda (ev? od? n)
     (if (= n 0) true (od? ~\( \dark \langle \)~??~\( \dark \rangle \)~ ~\( \dark \langle \)~??~\( \dark \rangle \)~ ~\( \dark \langle \)~??~\( \dark \rangle \)~)))
   (lambda (ev? od? n)
     (if (= n 0) false (ev? ~\( \dark \langle \)~??~\( \dark \rangle \)~ ~\( \dark \langle \)~??~\( \dark \rangle \)~ ~\( \dark \langle \)~??~\( \dark \rangle \)~)))))
\end{scheme}
\end{enumerate}
\end{quote}

\subsection{Separating Syntactic Analysis from Execution}
\label{Section 4.1.7}

The evaluator implemented above is simple, but it is very inefficient, because
the syntactic analysis of expressions is interleaved with their execution.
Thus if a program is executed many times, its syntax is analyzed many times.
Consider, for example, evaluating \code{(factorial 4)} using the following
definition of \code{factorial}:

\begin{scheme}
(define (factorial n)
  (if (= n 1) 1 (* (factorial (- n 1)) n)))
\end{scheme}

\noindent
Each time \code{factorial} is called, the evaluator must determine that the
body is an \code{if} expression and extract the predicate.  Only then can it
evaluate the predicate and dispatch on its value.  Each time it evaluates the
expression \code{(* (factorial (- n 1)) n)}, or the subexpressions
\code{(factorial (- n 1))} and \code{(- n 1)}, the evaluator must perform the
case analysis in \code{eval} to determine that the expression is an
application, and must extract its operator and operands.  This analysis is
expensive.  Performing it repeatedly is wasteful.

We can transform the evaluator to be significantly more efficient by arranging
things so that syntactic analysis is performed only once.\footnote{This
technique is an integral part of the compilation process, which we shall
discuss in \link{Chapter 5}.  Jonathan Rees wrote a Scheme interpreter like this
in about 1982 for the T project (\link{Rees and Adams 1982}).  Marc \link{Feeley (1986)} (see
also \link{Feeley and Lapalme 1987}) independently invented this technique in his
master's thesis.} We split \code{eval}, which takes an expression and an
environment, into two parts.  The procedure \code{analyze} takes only the
expression.  It performs the syntactic analysis and returns a new procedure,
the \newterm{execution procedure}, that encapsulates the work to be done in
executing the analyzed expression.  The execution procedure takes an
environment as its argument and completes the evaluation.  This saves work
because \code{analyze} will be called only once on an expression, while the
execution procedure may be called many times.

With the separation into analysis and execution, \code{eval} now becomes

\begin{scheme}
(define (eval exp env) ((analyze exp) env))
\end{scheme}

\noindent
The result of calling \code{analyze} is the execution procedure to be applied
to the environment.  The \code{analyze} procedure is the same case analysis as
performed by the original \code{eval} of \link{Section 4.1.1}, except that the
procedures to which we dispatch perform only analysis, not full evaluation:

\begin{scheme}
(define (analyze exp)
  (cond ((self-evaluating? exp) (analyze-self-evaluating exp))
        ((quoted? exp) (analyze-quoted exp))
        ((variable? exp) (analyze-variable exp))
        ((assignment? exp) (analyze-assignment exp))
        ((definition? exp) (analyze-definition exp))
        ((if? exp) (analyze-if exp))
        ((lambda? exp) (analyze-lambda exp))
        ((begin? exp) (analyze-sequence (begin-actions exp)))
        ((cond? exp) (analyze (cond->if exp)))
        ((application? exp) (analyze-application exp))
        (else (error "Unknown expression type: ANALYZE" exp))))
\end{scheme}

\enlargethispage{\baselineskip}

\noindent
Here is the simplest syntactic analysis procedure, which handles
self-evaluating expressions.  It returns an execution procedure that ignores
its environment argument and just returns the expression:

\begin{scheme}
(define (analyze-self-evaluating exp)
  (lambda (env) exp))
\end{scheme}

\noindent
For a quoted expression, we can gain a little efficiency by extracting the text
of the quotation only once, in the analysis phase, rather than in the execution
phase.

\begin{scheme}
(define (analyze-quoted exp)
  (let ((qval (text-of-quotation exp)))
    (lambda (env) qval)))
\end{scheme}

\noindent
Looking up a variable value must still be done in the execution phase, since
this depends upon knowing the environment.\footnote{There is, however, an
important part of the variable search that \emph{can} be done as part of the
syntactic analysis.  As we will show in \link{Section 5.5.6}, one can determine
the position in the environment structure where the value of the variable will
be found, thus obviating the need to scan the environment for the entry that
matches the variable.}

\begin{scheme}
(define (analyze-variable exp)
  (lambda (env) (lookup-variable-value exp env)))
\end{scheme}

\noindent
\code{Analyze\-/assignment} also must defer actually setting the variable until
the execution, when the environment has been supplied.  However, the fact that
the \code{assignment\-/value} expression can be analyzed (recursively) during
analysis is a major gain in efficiency, because the \code{assignment\-/value}
expression will now be analyzed only once.  The same holds true for
definitions.

\begin{scheme}
(define (analyze-assignment exp)
  (let ((var (assignment-variable exp))
        (vproc (analyze (assignment-value exp))))
    (lambda (env)
      (set-variable-value! var (vproc env) env)
      'ok)))
(define (analyze-definition exp)
  (let ((var (definition-variable exp))
        (vproc (analyze (definition-value exp))))
    (lambda (env)
      (define-variable! var (vproc env) env)
      'ok)))
\end{scheme}

\noindent
For \code{if} expressions, we extract and analyze the predicate, consequent,
and alternative at analysis time.

\begin{scheme}
(define (analyze-if exp)
  (let ((pproc (analyze (if-predicate exp)))
        (cproc (analyze (if-consequent exp)))
        (aproc (analyze (if-alternative exp))))
    (lambda (env) (if (true? (pproc env))
                      (cproc env)
                      (aproc env)))))
\end{scheme}

\noindent
Analyzing a \code{lambda} expression also achieves a major gain in efficiency:
We analyze the \code{lambda} body only once, even though procedures resulting
from evaluation of the \code{lambda} may be applied many times.

\begin{scheme}
(define (analyze-lambda exp)
  (let ((vars (lambda-parameters exp))
        (bproc (analyze-sequence (lambda-body exp))))
    (lambda (env) (make-procedure vars bproc env))))
\end{scheme}

\noindent
Analysis of a sequence of expressions (as in a \code{begin} or the body of a
\code{lambda} expression) is more involved.\footnote{See \link{Exercise 4.23}
for some insight into the processing of sequences.} Each expression in the
sequence is analyzed, yielding an execution procedure.  These execution
procedures are combined to produce an execution procedure that takes an
environment as argument and sequentially calls each individual execution
procedure with the environment as argument.

\begin{scheme}
(define (analyze-sequence exps)
  (define (sequentially proc1 proc2)
    (lambda (env) (proc1 env) (proc2 env)))
  (define (loop first-proc rest-procs)
    (if (null? rest-procs)
        first-proc
        (loop (sequentially first-proc (car rest-procs))
              (cdr rest-procs))))
  (let ((procs (map analyze exps)))
    (if (null? procs) (error "Empty sequence: ANALYZE"))
    (loop (car procs) (cdr procs))))
\end{scheme}

\noindent
To analyze an application, we analyze the operator and operands and construct
an execution procedure that calls the operator execution procedure (to obtain
the actual procedure to be applied) and the operand execution procedures (to
obtain the actual arguments).  We then pass these to
\code{execute\-/application}, which is the analog of \code{apply} in 
\link{Section 4.1.1}.  \code{Execute\-/application} differs from \code{apply} in that the
procedure body for a compound procedure has already been analyzed, so there is
no need to do further analysis.  Instead, we just call the execution procedure
for the body on the extended environment.

\begin{scheme}
(define (analyze-application exp)
  (let ((fproc (analyze (operator exp)))
        (aprocs (map analyze (operands exp))))
    (lambda (env)
      (execute-application
       (fproc env) 
       (map (lambda (aproc) (aproc env))
            aprocs)))))
(define (execute-application proc args)
  (cond ((primitive-procedure? proc)
         (apply-primitive-procedure proc args))
        ((compound-procedure? proc)
         ((procedure-body proc)
          (extend-environment
           (procedure-parameters proc)
           args
           (procedure-environment proc))))
        (else
         (error "Unknown procedure type: EXECUTE-APPLICATION"
                proc))))
\end{scheme}

\noindent
Our new evaluator uses the same data structures, syntax procedures, and
run-time support procedures as in sections \link{Section 4.1.2}, \link{Section 4.1.3}, and
\link{Section 4.1.4}.

\begin{quote}
\heading{\phantomsection\label{Exercise 4.22}Exercise 4.22:} Extend the evaluator in this
section to support the special form \code{let}.  (See \link{Exercise 4.6}.)
\end{quote}

\begin{quote}
\heading{\phantomsection\label{Exercise 4.23}Exercise 4.23:} Alyssa P. Hacker doesn't
understand why \code{analyze\-/sequence} needs to be so complicated.  All the
other analysis procedures are straightforward transformations of the
corresponding evaluation procedures (or \code{eval} clauses) in 
\link{Section 4.1.1}.  She expected \code{analyze\-/sequence} to look like this:

\begin{scheme}
(define (analyze-sequence exps)
  (define (execute-sequence procs env)
    (cond ((null? (cdr procs))
           ((car procs) env))
          (else
           ((car procs) env)
           (execute-sequence (cdr procs) env))))
  (let ((procs (map analyze exps)))
    (if (null? procs)
        (error "Empty sequence: ANALYZE"))
    (lambda (env)
      (execute-sequence procs env))))
\end{scheme}

Eva Lu Ator explains to Alyssa that the version in the text does more of the
work of evaluating a sequence at analysis time.  Alyssa's sequence-execution
procedure, rather than having the calls to the individual execution procedures
built in, loops through the procedures in order to call them: In effect,
although the individual expressions in the sequence have been analyzed, the
sequence itself has not been.

Compare the two versions of \code{analyze\-/sequence}.  For example, consider the
common case (typical of procedure bodies) where the sequence has just one
expression.  What work will the execution procedure produced by Alyssa's
program do?  What about the execution procedure produced by the program in the
text above?  How do the two versions compare for a sequence with two
expressions?
\end{quote}

\begin{quote}
\heading{\phantomsection\label{Exercise 4.24}Exercise 4.24:} Design and carry out some
experiments to compare the speed of the original metacircular evaluator with
the version in this section.  Use your results to estimate the fraction of time
that is spent in analysis versus execution for various procedures.
\end{quote}

\section{Variations on a Scheme --- Lazy Evaluation}
\label{Section 4.2}

Now that we have an evaluator expressed as a Lisp program, we can experiment
with alternative choices in language design simply by modifying the evaluator.
Indeed, new languages are often invented by first writing an evaluator that
embeds the new language within an existing high-level language.  For example,
if we wish to discuss some aspect of a proposed modification to Lisp with
another member of the Lisp community, we can supply an evaluator that embodies
the change.  The recipient can then experiment with the new evaluator and send
back comments as further modifications.  Not only does the high-level
implementation base make it easier to test and debug the evaluator; in
addition, the embedding enables the designer to snarf\,\footnote{Snarf: ``To
grab, especially a large document or file for the purpose of using it either
with or without the owner's permission.''  Snarf down: ``To snarf, sometimes
with the connotation of absorbing, processing, or understanding.''  (These
definitions were snarfed from \link{Steele et al. 1983}.  See also \link{Raymond 1993}.)}
features from the underlying language, just as our embedded Lisp evaluator uses
primitives and control structure from the underlying Lisp.  Only later (if
ever) need the designer go to the trouble of building a complete implementation
in a low-level language or in hardware.  In this section and the next we
explore some variations on Scheme that provide significant additional
expressive power.



\subsection{Normal Order and Applicative Order}
\label{Section 4.2.1}

In \link{Section 1.1}, where we began our discussion of models of evaluation, we
noted that Scheme is an \newterm{applicative-order} language, namely, that all
the arguments to Scheme procedures are evaluated when the procedure is applied.
In contrast, \newterm{normal-order} languages delay evaluation of procedure
arguments until the actual argument values are needed.  Delaying evaluation of
procedure arguments until the last possible moment (e.g., until they are
required by a primitive operation) is called \newterm{lazy
evaluation}.\footnote{The difference between the ``lazy'' terminology and the
``normal-order'' terminology is somewhat fuzzy.  Generally, ``lazy'' refers to
the mechanisms of particular evaluators, while ``normal-order'' refers to the
semantics of languages, independent of any particular evaluation strategy.  But
this is not a hard-and-fast distinction, and the two terminologies are often
used interchangeably.}  Consider the procedure

\begin{scheme}
(define (try a b) (if (= a 0) 1 b))
\end{scheme}

\noindent
Evaluating \code{(try 0 (/ 1 0))} generates an error in Scheme.  With lazy
evaluation, there would be no error.  Evaluating the expression would return 1,
because the argument \code{(/ 1 0)} would never be evaluated.

An example that exploits lazy evaluation is the definition of a procedure
\code{unless}

\begin{scheme}
(define (unless condition usual-value exceptional-value)
  (if condition exceptional-value usual-value))
\end{scheme}

\noindent
that can be used in expressions such as

\begin{scheme}
(unless (= b 0)
        (/ a b)
        (begin (display "exception: returning 0") 0))
\end{scheme}

\noindent
This won't work in an applicative-order language because both the usual value
and the exceptional value will be evaluated before \code{unless} is called
(compare \link{Exercise 1.6}).  An advantage of lazy evaluation is that some
procedures, such as \code{unless}, can do useful computation even if evaluation
of some of their arguments would produce errors or would not terminate.

If the body of a procedure is entered before an argument has been evaluated we
say that the procedure is \newterm{non-strict} in that argument.  If the
argument is evaluated before the body of the procedure is entered we say that
the procedure is \newterm{strict} in that argument.\footnote{The ``strict''
versus ``non-strict'' terminology means essentially the same thing as
``applicative-order'' versus ``normal-order,'' except that it refers to
individual procedures and arguments rather than to the language as a whole.  At
a conference on programming languages you might hear someone say, ``The
normal-order language Hassle has certain strict primitives.  Other procedures
take their arguments by lazy evaluation.''}  In a purely applicative-order
language, all procedures are strict in each argument.  In a purely normal-order
language, all compound procedures are non-strict in each argument, and
primitive procedures may be either strict or non-strict.  There are also
languages (see \link{Exercise 4.31}) that give programmers detailed control over
the strictness of the procedures they define.

A striking example of a procedure that can usefully be made non-strict is
\code{cons} (or, in general, almost any constructor for data structures).  One
can do useful computation, combining elements to form data structures and
operating on the resulting data structures, even if the values of the elements
are not known.  It makes perfect sense, for instance, to compute the length of
a list without knowing the values of the individual elements in the list.  We
will exploit this idea in \link{Section 4.2.3} to implement the streams of
\link{Chapter 3} as lists formed of non-strict \code{cons} pairs.

\begin{quote}
\heading{\phantomsection\label{Exercise 4.25}Exercise 4.25:} Suppose that (in ordinary
applicative-order Scheme) we define \code{unless} as shown above and then
define \code{factorial} in terms of \code{unless} as

\begin{scheme}
(define (factorial n)
  (unless (= n 1)
          (* n (factorial (- n 1)))
          1))
\end{scheme}

What happens if we attempt to evaluate \code{(factorial 5)}?  Will our
definitions work in a normal-order language?
\end{quote}

\begin{quote}
\heading{\phantomsection\label{Exercise 4.26}Exercise 4.26:} Ben Bitdiddle and Alyssa
P. Hacker disagree over the importance of lazy evaluation for implementing
things such as \code{unless}.  Ben points out that it's possible to implement
\code{unless} in applicative order as a special form.  Alyssa counters that, if
one did that, \code{unless} would be merely syntax, not a procedure that could
be used in conjunction with higher-order procedures.  Fill in the details on
both sides of the argument.  Show how to implement \code{unless} as a derived
expression (like \code{cond} or \code{let}), and give an example of a situation
where it might be useful to have \code{unless} available as a procedure, rather
than as a special form.
\end{quote}

\subsection{An Interpreter with Lazy Evaluation}
\label{Section 4.2.2}

In this section we will implement a normal-order language that is the same as
Scheme except that compound procedures are non-strict in each argument.
Primitive procedures will still be strict.  It is not difficult to modify the
evaluator of \link{Section 4.1.1} so that the language it interprets behaves
this way.  Almost all the required changes center around procedure application.

The basic idea is that, when applying a procedure, the interpreter must
determine which arguments are to be evaluated and which are to be delayed.  The
delayed arguments are not evaluated; instead, they are transformed into objects
called \newterm{thunks}.\footnote{The word \newterm{thunk} was invented by an
informal working group that was discussing the implementation of call-by-name
in Algol 60.  They observed that most of the analysis of (``thinking about'')
the expression could be done at compile time; thus, at run time, the expression
would already have been ``thunk'' about (\link{Ingerman et al. 1960}).} The thunk must
contain the information required to produce the value of the argument when it
is needed, as if it had been evaluated at the time of the application.  Thus,
the thunk must contain the argument expression and the environment in which the
procedure application is being evaluated.

The process of evaluating the expression in a thunk is called
\newterm{forcing}.\footnote{This is analogous to the use of \code{force} on the
delayed objects that were introduced in \link{Chapter 3} to represent streams.
The critical difference between what we are doing here and what we did in
\link{Chapter 3} is that we are building delaying and forcing into the
evaluator, and thus making this uniform and automatic throughout the language.}
In general, a thunk will be forced only when its value is needed: when it is
passed to a primitive procedure that will use the value of the thunk; when it
is the value of a predicate of a conditional; and when it is the value of an
operator that is about to be applied as a procedure.  One design choice we have
available is whether or not to \newterm{memoize} thunks, as we did with delayed
objects in \link{Section 3.5.1}.  With memoization, the first time a thunk is
forced, it stores the value that is computed.  Subsequent forcings simply
return the stored value without repeating the computation.  We'll make our
interpreter memoize, because this is more efficient for many applications.
There are tricky considerations here, however.\footnote{Lazy evaluation
combined with memoization is sometimes referred to as \newterm{call-by-need}
argument passing, in contrast to \newterm{call-by-name} argument passing.
(Call-by-name, introduced in Algol 60, is similar to non-memoized lazy
evaluation.)  As language designers, we can build our evaluator to memoize, not
to memoize, or leave this an option for programmers (\link{Exercise 4.31}).  As
you might expect from \link{Chapter 3}, these choices raise issues that become
both subtle and confusing in the presence of assignments.  (See \link{Exercise 4.27} 
and \link{Exercise 4.29}.)  An excellent article by \link{Clinger (1982)}
attempts to clarify the multiple dimensions of confusion that arise here.}

\subsubsection*{Modifying the evaluator}

The main difference between the lazy evaluator and the one in \link{Section 4.1}
is in the handling of procedure applications in \code{eval} and \code{apply}.

The \code{application?} clause of \code{eval} becomes

\begin{scheme}
((application? exp)
 (apply (actual-value (operator exp) env)
        (operands exp)
        env))
\end{scheme}

\noindent
This is almost the same as the \code{application?} clause of \code{eval} in
\link{Section 4.1.1}.  For lazy evaluation, however, we call \code{apply} with
the operand expressions, rather than the arguments produced by evaluating them.
Since we will need the environment to construct thunks if the arguments are to
be delayed, we must pass this as well.  We still evaluate the operator, because
\code{apply} needs the actual procedure to be applied in order to dispatch on
its type (primitive versus compound) and apply it.

Whenever we need the actual value of an expression, we use

\begin{scheme}
(define (actual-value exp env)
  (force-it (eval exp env)))
\end{scheme}

\noindent
instead of just \code{eval}, so that if the expression's value is a thunk, it
will be forced.

Our new version of \code{apply} is also almost the same as the version in
\link{Section 4.1.1}.  The difference is that \code{eval} has passed in
unevaluated operand expressions: For primitive procedures (which are strict),
we evaluate all the arguments before applying the primitive; for compound
procedures (which are non-strict) we delay all the arguments before applying
the procedure.

\begin{scheme}
(define (apply procedure arguments env)
  (cond ((primitive-procedure? procedure)
         (apply-primitive-procedure
          procedure
          (list-of-arg-values arguments env)))   ;~\textrm{changed}~
        ((compound-procedure? procedure)
         (eval-sequence
          (procedure-body procedure)
          (extend-environment
           (procedure-parameters procedure)
           (list-of-delayed-args arguments env)  ;~\textrm{changed}~
           (procedure-environment procedure))))
        (else (error "Unknown procedure type: APPLY"
                     procedure))))
\end{scheme}

\noindent
The procedures that process the arguments are just like \code{list\-/of\-/values}
from \link{Section 4.1.1}, except that \code{list\-/of\-/delayed\-/args} delays the
arguments instead of evaluating them, and \code{list\-/of\-/arg\-/values} uses
\code{actual\-/value} instead of \code{eval}:

\begin{scheme}
(define (list-of-arg-values exps env)
  (if (no-operands? exps)
      '()
      (cons (actual-value (first-operand exps)
                          env)
            (list-of-arg-values (rest-operands exps)
                                env))))
(define (list-of-delayed-args exps env)
  (if (no-operands? exps)
      '()
      (cons (delay-it (first-operand exps)
                      env)
            (list-of-delayed-args (rest-operands exps)
                                  env))))
\end{scheme}

\noindent
The other place we must change the evaluator is in the handling of \code{if},
where we must use \code{actual\-/value} instead of \code{eval} to get the value
of the predicate expression before testing whether it is true or false:

\begin{scheme}
(define (eval-if exp env)
  (if (true? (actual-value (if-predicate exp) env))
      (eval (if-consequent exp) env)
      (eval (if-alternative exp) env)))
\end{scheme}

\noindent
Finally, we must change the \code{driver\-/loop} procedure (\link{Section 4.1.4})
to use \code{actual\-/value} instead of \code{eval}, so that if a delayed value
is propagated back to the read-eval-print loop, it will be forced before being
printed.  We also change the prompts to indicate that this is the lazy
evaluator:

\begin{scheme}
(define input-prompt  ";;; L-Eval input:")
(define output-prompt ";;; L-Eval value:")
(define (driver-loop)
  (prompt-for-input input-prompt)
  (let ((input (read)))
    (let ((output
           (actual-value
            input the-global-environment)))
      (announce-output output-prompt)
      (user-print output)))
  (driver-loop))
\end{scheme}

\noindent
With these changes made, we can start the evaluator and test it.  The
successful evaluation of the \code{try} expression discussed in 
\link{Section 4.2.1} indicates that the interpreter is performing lazy evaluation:

\begin{scheme}
(define the-global-environment (setup-environment))
(driver-loop)
~\textit{;;; L-Eval input:}~
(define (try a b) (if (= a 0) 1 b))
~\textit{;;; L-Eval value:}~
~\textit{ok}~
~\textit{;;; L-Eval input:}~
(try 0 (/ 1 0))
~\textit{;;; L-Eval value:}~
~\textit{1}~
\end{scheme}

\subsubsection*{Representing thunks}

Our evaluator must arrange to create thunks when procedures are applied to
arguments and to force these thunks later.  A thunk must package an expression
together with the environment, so that the argument can be produced later.  To
force the thunk, we simply extract the expression and environment from the
thunk and evaluate the expression in the environment.  We use
\code{actual\-/value} rather than \code{eval} so that in case the value of the
expression is itself a thunk, we will force that, and so on, until we reach
something that is not a thunk:

\begin{scheme}
(define (force-it obj)
  (if (thunk? obj)
      (actual-value (thunk-exp obj) (thunk-env obj))
      obj))
\end{scheme}

\noindent
One easy way to package an expression with an environment is to make a list
containing the expression and the environment.  Thus, we create a thunk as
follows:

\begin{scheme}
(define (delay-it exp env)
  (list 'thunk exp env))
(define (thunk? obj)
  (tagged-list? obj 'thunk))
(define (thunk-exp thunk) (cadr  thunk))
(define (thunk-env thunk) (caddr thunk))
\end{scheme}

\noindent
Actually, what we want for our interpreter is not quite this, but rather thunks
that have been memoized.  When a thunk is forced, we will turn it into an
evaluated thunk by replacing the stored expression with its value and changing
the \code{thunk} tag so that it can be recognized as already
evaluated.\footnote{Notice that we also erase the \code{env} from the thunk
once the expression's value has been computed.  This makes no difference in the
values returned by the interpreter.  It does help save space, however, because
removing the reference from the thunk to the \code{env} once it is no longer
needed allows this structure to be \newterm{garbage-collected} and its space
recycled, as we will discuss in \link{Section 5.3}.

Similarly, we could have allowed unneeded environments in the memoized delayed
objects of \link{Section 3.5.1} to be garbage-collected, by having
\code{memo\-/proc} do something like \code{(set! proc '())} to discard the
procedure \code{proc} (which includes the environment in which the \code{delay}
was evaluated) after storing its value.}

\begin{scheme}
(define (evaluated-thunk? obj)
  (tagged-list? obj 'evaluated-thunk))
(define (thunk-value evaluated-thunk) 
  (cadr evaluated-thunk))
(define (force-it obj)
  (cond ((thunk? obj)
         (let ((result (actual-value (thunk-exp obj)
                                     (thunk-env obj))))
           (set-car! obj 'evaluated-thunk)
           (set-car! (cdr obj) 
                     result)     ;~\textrm{replace \code{exp} with its value}~
           (set-cdr! (cdr obj) 
                     '())        ;~\textrm{forget unneeded \code{env}}~
           result))
        ((evaluated-thunk? obj) (thunk-value obj))
        (else obj)))
\end{scheme}

\noindent
Notice that the same \code{delay\-/it} procedure works both with and without
memoization.

\begin{quote}
\heading{\phantomsection\label{Exercise 4.27}Exercise 4.27:} Suppose we type in the following
definitions to the lazy evaluator:

\begin{scheme}
(define count 0)
(define (id x) (set! count (+ count 1)) x)
\end{scheme}

Give the missing values in the following sequence of interactions, and explain
your answers.\footnote{This exercise demonstrates that the interaction between
lazy evaluation and side effects can be very confusing.  This is just what you
might expect from the discussion in \link{Chapter 3}.}

\begin{scheme}
(define w (id (id 10)))
~\textit{;;; L-Eval input:}~
count
~\textit{;;; L-Eval value:}~
~\( \dark \langle \)~~\var{\dark response}~~\( \dark \rangle \)~
~\textit{;;; L-Eval input:}~
w
~\textit{;;; L-Eval value:}~
~\( \dark \langle \)~~\var{\dark response}~~\( \dark \rangle \)~
~\textit{;;; L-Eval input:}~
count
~\textit{;;; L-Eval value:}~
~\( \dark \langle \)~~\var{\dark response}~~\( \dark \rangle \)~
\end{scheme}
\end{quote}

\begin{quote}
\heading{\phantomsection\label{Exercise 4.28}Exercise 4.28:} \code{Eval} uses
\code{actual\-/value} rather than \code{eval} to evaluate the operator before
passing it to \code{apply}, in order to force the value of the operator.  Give
an example that demonstrates the need for this forcing.

\noindent
\heading{\phantomsection\label{Exercise 4.29}Exercise 4.29:} Exhibit a program that you would
expect to run much more slowly without memoization than with memoization.
Also, consider the following interaction, where the \code{id} procedure is
defined as in \link{Exercise 4.27} and \code{count} starts at 0:

\begin{scheme}
(define (square x) (* x x))
~\textit{;;; L-Eval input:}~
(square (id 10))
~\textit{;;; L-Eval value:}~
~\( \dark \langle \)~~\var{\dark response}~~\( \dark \rangle \)~
~\textit{;;; L-Eval input:}~
count
~\textit{;;; L-Eval value:}~
~\( \dark \langle \)~~\var{\dark response}~~\( \dark \rangle \)~
\end{scheme}

Give the responses both when the evaluator memoizes and when it does not.
\end{quote}

\begin{quote}
\heading{\phantomsection\label{Exercise 4.30}Exercise 4.30:} Cy D. Fect, a reformed C
programmer, is worried that some side effects may never take place, because the
lazy evaluator doesn't force the expressions in a sequence.  Since the value of
an expression in a sequence other than the last one is not used (the expression
is there only for its effect, such as assigning to a variable or printing),
there can be no subsequent use of this value (e.g., as an argument to a
primitive procedure) that will cause it to be forced.  Cy thus thinks that when
evaluating sequences, we must force all expressions in the sequence except the
final one.  He proposes to modify \code{eval\-/sequence} from \link{Section 4.1.1}
to use \code{actual\-/value} rather than \code{eval}:

\begin{scheme}
(define (eval-sequence exps env)
  (cond ((last-exp? exps) (eval (first-exp exps) env))
        (else (actual-value (first-exp exps) env)
              (eval-sequence (rest-exps exps) env))))
\end{scheme}

\begin{enumerate}[a.]

\item
Ben Bitdiddle thinks Cy is wrong.  He shows Cy the \code{for\-/each} procedure
described in \link{Exercise 2.23}, which gives an important example of a
sequence with side effects:

\begin{scheme}
(define (for-each proc items)
  (if (null? items)
      'done
      (begin (proc (car items))
             (for-each proc (cdr items)))))
\end{scheme}

He claims that the evaluator in the text (with the original
\code{eval\-/sequence}) handles this correctly:

\begin{scheme}
~\textit{;;; L-Eval input:}~
(for-each (lambda (x) (newline) (display x))
          (list 57 321 88))
~\textit{57}~
~\textit{321}~
~\textit{88}~
~\textit{;;; L-Eval value:}~
~\textit{done}~
\end{scheme}

Explain why Ben is right about the behavior of \code{for\-/each}.

\item
Cy agrees that Ben is right about the \code{for\-/each} example, but says that
that's not the kind of program he was thinking about when he proposed his
change to \code{eval\-/sequence}.  He defines the following two procedures in the
lazy evaluator:

\begin{scheme}
(define (p1 x)
  (set! x (cons x '(2)))
  x)
(define (p2 x)
  (define (p e)
    e
    x)
  (p (set! x (cons x '(2)))))
\end{scheme}

What are the values of \code{(p1 1)} and \code{(p2 1)} with the original
\code{eval\-/sequence}?  What would the values be with Cy's proposed change to
\code{eval\-/sequence}?

\item
Cy also points out that changing \code{eval\-/sequence} as he proposes does not
affect the behavior of the example in part a.  Explain why this is true.

\item
How do you think sequences ought to be treated in the lazy evaluator?  Do you
like Cy's approach, the approach in the text, or some other approach?

\end{enumerate}
\end{quote}

\begin{quote}
\heading{\phantomsection\label{Exercise 4.31}Exercise 4.31:} The approach taken in this
section is somewhat unpleasant, because it makes an incompatible change to
Scheme.  It might be nicer to implement lazy evaluation as an
\newterm{upward-compatible extension}, that is, so that ordinary Scheme
programs will work as before.  We can do this by extending the syntax of
procedure declarations to let the user control whether or not arguments are to
be delayed.  While we're at it, we may as well also give the user the choice
between delaying with and without memoization.  For example, the definition

\begin{scheme}
(define (f a (b lazy) c (d lazy-memo))
  ~\( \dots \)~)
\end{scheme}

\noindent
would define \code{f} to be a procedure of four arguments, where the first and
third arguments are evaluated when the procedure is called, the second argument
is delayed, and the fourth argument is both delayed and memoized.  Thus,
ordinary procedure definitions will produce the same behavior as ordinary
Scheme, while adding the \code{lazy\-/memo} declaration to each parameter of
every compound procedure will produce the behavior of the lazy evaluator
defined in this section. Design and implement the changes required to produce
such an extension to Scheme.  You will have to implement new syntax procedures
to handle the new syntax for \code{define}.  You must also arrange for
\code{eval} or \code{apply} to determine when arguments are to be delayed, and
to force or delay arguments accordingly, and you must arrange for forcing to
memoize or not, as appropriate.
\end{quote}

\subsection{Streams as Lazy Lists}
\label{Section 4.2.3}

In \link{Section 3.5.1}, we showed how to implement streams as delayed lists.
We introduced special forms \code{delay} and \code{cons\-/stream}, which allowed
us to construct a ``promise'' to compute the \code{cdr} of a stream, without
actually fulfilling that promise until later.  We could use this general
technique of introducing special forms whenever we need more control over the
evaluation process, but this is awkward.  For one thing, a special form is not
a first-class object like a procedure, so we cannot use it together with
higher-order procedures.\footnote{This is precisely the issue with the
\code{unless} procedure, as in \link{Exercise 4.26}.}  Additionally, we were
forced to create streams as a new kind of data object similar but not identical
to lists, and this required us to reimplement many ordinary list operations
(\code{map}, \code{append}, and so on) for use with streams.

With lazy evaluation, streams and lists can be identical, so there is no need
for special forms or for separate list and stream operations.  All we need to
do is to arrange matters so that \code{cons} is non-strict.  One way to
accomplish this is to extend the lazy evaluator to allow for non-strict
primitives, and to implement \code{cons} as one of these.  An easier way is to
recall (\link{Section 2.1.3}) that there is no fundamental need to implement
\code{cons} as a primitive at all.  Instead, we can represent pairs as
procedures:\footnote{This is the procedural representation described in
\link{Exercise 2.4}.  Essentially any procedural representation (e.g., a
message-passing implementation) would do as well.  Notice that we can install
these definitions in the lazy evaluator simply by typing them at the driver
loop.  If we had originally included \code{cons}, \code{car}, and \code{cdr} as
primitives in the global environment, they will be redefined.  (Also see
\link{Exercise 4.33} and \link{Exercise 4.34}.)}

\begin{scheme}
(define (cons x y) (lambda (m) (m x y)))
(define (car z) (z (lambda (p q) p)))
(define (cdr z) (z (lambda (p q) q)))
\end{scheme}

\noindent
In terms of these basic operations, the standard definitions of the list
operations will work with infinite lists (streams) as well as finite ones, and
the stream operations can be implemented as list operations.  Here are some
examples:

\begin{scheme}
(define (list-ref items n)
  (if (= n 0)
      (car items)
      (list-ref (cdr items) (- n 1))))
(define (map proc items)
  (if (null? items)
      '()
      (cons (proc (car items)) (map proc (cdr items)))))
(define (scale-list items factor)
  (map (lambda (x) (* x factor)) items))
(define (add-lists list1 list2)
  (cond ((null? list1) list2)
        ((null? list2) list1)
        (else (cons (+ (car list1) (car list2))
                    (add-lists (cdr list1) (cdr list2))))))
(define ones (cons 1 ones))
(define integers (cons 1 (add-lists ones integers)))
~\textit{;;; L-Eval input:}~
(list-ref integers 17)
~\textit{;;; L-Eval value:}~
~\textit{18}~
\end{scheme}

\noindent
Note that these lazy lists are even lazier than the streams of \link{Chapter 3}:
The \code{car} of the list, as well as the \code{cdr}, is
delayed.\footnote{This permits us to create delayed versions of more general
kinds of list structures, not just sequences.  \link{Hughes 1990} discusses some
applications of ``lazy trees.''}  In fact, even accessing the \code{car} or
\code{cdr} of a lazy pair need not force the value of a list element.  The
value will be forced only when it is really needed---e.g., for use as the
argument of a primitive, or to be printed as an answer.

Lazy pairs also help with the problem that arose with streams in 
\link{Section 3.5.4}, where we found that formulating stream models of systems with
loops may require us to sprinkle our programs with explicit \code{delay}
operations, beyond the ones supplied by \code{cons\-/stream}.  With lazy
evaluation, all arguments to procedures are delayed uniformly.  For instance,
we can implement procedures to integrate lists and solve differential equations
as we originally intended in \link{Section 3.5.4}:

\begin{scheme}
(define (integral integrand initial-value dt)
  (define int
    (cons initial-value
          (add-lists (scale-list integrand dt) int)))
  int)
(define (solve f y0 dt)
  (define  y (integral dy y0 dt))
  (define dy (map f y))
  y)
~\textit{;;; L-Eval input:}~
(list-ref (solve (lambda (x) x) 1 0.001) 1000)
~\textit{;;; L-Eval value:}~
~\textit{2.716924}~
\end{scheme}

\begin{quote}
\heading{\phantomsection\label{Exercise 4.32}Exercise 4.32:} Give some examples that
illustrate the difference between the streams of \link{Chapter 3} and the
``lazier'' lazy lists described in this section.  How can you take advantage of
this extra laziness?
\end{quote}

\begin{quote}
\heading{\phantomsection\label{Exercise 4.33}Exercise 4.33:} Ben Bitdiddle tests the lazy list
implementation given above by evaluating the expression:

\begin{scheme}
(car '(a b c))
\end{scheme}

To his surprise, this produces an error.  After some thought, he realizes that
the ``lists'' obtained by reading in quoted expressions are different from the
lists manipulated by the new definitions of \code{cons}, \code{car}, and
\code{cdr}.  Modify the evaluator's treatment of quoted expressions so that
quoted lists typed at the driver loop will produce true lazy lists.
\end{quote}

\begin{quote}
\heading{\phantomsection\label{Exercise 4.34}Exercise 4.34:} Modify the driver loop for the
evaluator so that lazy pairs and lists will print in some reasonable way.
(What are you going to do about infinite lists?)  You may also need to modify
the representation of lazy pairs so that the evaluator can identify them in
order to print them.
\end{quote}

\section{Variations on a Scheme --- Nondeterministic Computing}
\label{Section 4.3}

In this section, we extend the Scheme evaluator to support a programming
paradigm called \newterm{nondeterministic computing} by building into the
evaluator a facility to support automatic search.  This is a much more profound
change to the language than the introduction of lazy evaluation in 
\link{Section 4.2}.

Nondeterministic computing, like stream processing, is useful for ``generate
and test'' applications.  Consider the task of starting with two lists of
positive integers and finding a pair of integers---one from the first list and
one from the second list---whose sum is prime.  We saw how to handle this with
finite sequence operations in \link{Section 2.2.3} and with infinite streams in
\link{Section 3.5.3}.  Our approach was to generate the sequence of all possible
pairs and filter these to select the pairs whose sum is prime.  Whether we
actually generate the entire sequence of pairs first as in \link{Chapter 2}, or
interleave the generating and filtering as in \link{Chapter 3}, is immaterial to
the essential image of how the computation is organized.

\enlargethispage{\baselineskip}

The nondeterministic approach evokes a different image.  Imagine simply that we
choose (in some way) a number from the first list and a number from the second
list and require (using some mechanism) that their sum be prime.  This is
expressed by following procedure:

\begin{scheme}
(define (prime-sum-pair list1 list2)
  (let ((a (an-element-of list1))
        (b (an-element-of list2)))
    (require (prime? (+ a b)))
    (list a b)))
\end{scheme}

\noindent
It might seem as if this procedure merely restates the problem, rather than
specifying a way to solve it.  Nevertheless, this is a legitimate
nondeterministic program.\footnote{We assume that we have previously defined a
procedure \code{prime?} that tests whether numbers are prime.  Even with
\code{prime?} defined, the \code{prime\-/sum\-/pair} procedure may look
suspiciously like the unhelpful ``pseudo-Lisp'' attempt to define the
square-root function, which we described at the beginning of 
\link{Section 1.1.7}.  In fact, a square-root procedure along those lines can actually
be formulated as a nondeterministic program.  By incorporating a search
mechanism into the evaluator, we are eroding the distinction between purely
declarative descriptions and imperative specifications of how to compute
answers.  We'll go even farther in this direction in \link{Section 4.4}.}

The key idea here is that expressions in a nondeterministic language can have
more than one possible value.  For instance, \code{an\-/element\-/of} might return
any element of the given list.  Our nondeterministic program evaluator will
work by automatically choosing a possible value and keeping track of the
choice.  If a subsequent requirement is not met, the evaluator will try a
different choice, and it will keep trying new choices until the evaluation
succeeds, or until we run out of choices.  Just as the lazy evaluator freed the
programmer from the details of how values are delayed and forced, the
nondeterministic program evaluator will free the programmer from the details of
how choices are made.

It is instructive to contrast the different images of time evoked by
nondeterministic evaluation and stream processing.  Stream processing uses lazy
evaluation to decouple the time when the stream of possible answers is
assembled from the time when the actual stream elements are produced.  The
evaluator supports the illusion that all the possible answers are laid out
before us in a timeless sequence.  With nondeterministic evaluation, an
expression represents the exploration of a set of possible worlds, each
determined by a set of choices.  Some of the possible worlds lead to dead ends,
while others have useful values.  The nondeterministic program evaluator
supports the illusion that time branches, and that our programs have different
possible execution histories.  When we reach a dead end, we can revisit a
previous choice point and proceed along a different branch.

The nondeterministic program evaluator implemented below is called the
\code{amb} evaluator because it is based on a new special form called
\code{amb}.  We can type the above definition of \code{prime\-/sum\-/pair} at the
\code{amb} evaluator driver loop (along with definitions of \code{prime?},
\code{an\-/element\-/of}, and \code{require}) and run the procedure as follows:

\begin{scheme}
~\textit{;;; Amb-Eval input:}~
(prime-sum-pair '(1 3 5 8) '(20 35 110))
~\textit{;;; Starting a new problem}~
~\textit{;;; Amb-Eval value:}~
~\textit{(3 20)}~
\end{scheme}

\noindent
The value returned was obtained after the evaluator repeatedly chose elements
from each of the lists, until a successful choice was made.

\link{Section 4.3.1} introduces \code{amb} and explains how it supports
nondeterminism through the evaluator's automatic search mechanism.  
\link{Section 4.3.2} presents examples of nondeterministic programs, and 
\link{Section 4.3.3} gives the details of how to implement the \code{amb} evaluator by
modifying the ordinary Scheme evaluator.



\subsection{Amb and Search}
\label{Section 4.3.1}

To extend Scheme to support nondeterminism, we introduce a new special form
called \code{amb}.\footnote{The idea of \code{amb} for nondeterministic
programming was first described in 1961 by John McCarthy (see \link{McCarthy 1963}).}
The expression

\begin{scheme}
(amb ~\( \dark \langle \)~~\( \dark e_1 \)~~\( \dark \rangle \)~ ~\( \dark \langle \)~~\( \dark e_2 \)~~\( \dark \rangle \)~ ~\( \dots \)~ ~\( \dark \langle \)~~\( \dark e_n \)~~\( \dark \rangle \)~)
\end{scheme}

\noindent
returns the value of one of the \( n \) expressions \( \langle \)\( e_i \)\( \rangle \)
``ambiguously.''  For example, the expression

\begin{scheme}
(list (amb 1 2 3) (amb 'a 'b))
\end{scheme}

\noindent
can have six possible values:

\begin{scheme}
~\code{(1 a)}~ ~\code{(1 b)}~ ~\code{(2 a)}~ ~\code{(2 b)}~ ~\code{(3 a)}~ ~\code{(3 b)}~
\end{scheme}

\noindent
\code{Amb} with a single choice produces an ordinary (single) value.

\code{Amb} with no choices---the expression \code{(amb)}---is an expression
with no acceptable values.  Operationally, we can think of \code{(amb)} as an
expression that when evaluated causes the computation to ``fail'': The
computation aborts and no value is produced.  Using this idea, we can express
the requirement that a particular predicate expression \code{p} must be true as
follows:

\begin{scheme}
(define (require p) (if (not p) (amb)))
\end{scheme}

\noindent
With \code{amb} and \code{require}, we can implement the \code{an\-/element\-/of}
procedure used above:

\begin{scheme}
(define (an-element-of items)
  (require (not (null? items)))
  (amb (car items) (an-element-of (cdr items))))
\end{scheme}

\noindent
\code{An\-/element\-/of} fails if the list is empty.  Otherwise it ambiguously
returns either the first element of the list or an element chosen from the rest
of the list.

We can also express infinite ranges of choices.  The following procedure
potentially returns any integer greater than or equal to some given \( n \):

\begin{scheme}
(define (an-integer-starting-from n)
  (amb n (an-integer-starting-from (+ n 1))))
\end{scheme}

\noindent
This is like the stream procedure \code{integers\-/starting\-/from} described in
\link{Section 3.5.2}, but with an important difference: The stream procedure
returns an object that represents the sequence of all integers beginning with
\( n \), whereas the \code{amb} procedure returns a single integer.\footnote{In
actuality, the distinction between nondeterministically returning a single
choice and returning all choices depends somewhat on our point of view.  From
the perspective of the code that uses the value, the nondeterministic choice
returns a single value.  From the perspective of the programmer designing the
code, the nondeterministic choice potentially returns all possible values, and
the computation branches so that each value is investigated separately.}

Abstractly, we can imagine that evaluating an \code{amb} expression causes time
to split into branches, where the computation continues on each branch with one
of the possible values of the expression.  We say that \code{amb} represents a
\newterm{nondeterministic choice point}.  If we had a machine with a sufficient
number of processors that could be dynamically allocated, we could implement
the search in a straightforward way.  Execution would proceed as in a
sequential machine, until an \code{amb} expression is encountered.  At this
point, more processors would be allocated and initialized to continue all of
the parallel executions implied by the choice.  Each processor would proceed
sequentially as if it were the only choice, until it either terminates by
encountering a failure, or it further subdivides, or it finishes.\footnote{One
might object that this is a hopelessly inefficient mechanism.  It might require
millions of processors to solve some easily stated problem this way, and most
of the time most of those processors would be idle.  This objection should be
taken in the context of history.  Memory used to be considered just such an
expensive commodity.  In 1964 a megabyte of RAM cost about \$400,000.  Now every
personal computer has many megabytes of RAM, and most of the time most of that
RAM is unused.  It is hard to underestimate the cost of mass-produced
electronics.}

On the other hand, if we have a machine that can execute only one process (or a
few concurrent processes), we must consider the alternatives sequentially.  One
could imagine modifying an evaluator to pick at random a branch to follow
whenever it encounters a choice point.  Random choice, however, can easily lead
to failing values.  We might try running the evaluator over and over, making
random choices and hoping to find a non-failing value, but it is better to
\newterm{systematically search} all possible execution paths.  The \code{amb}
evaluator that we will develop and work with in this section implements a
systematic search as follows: When the evaluator encounters an application of
\code{amb}, it initially selects the first alternative.  This selection may
itself lead to a further choice.  The evaluator will always initially choose
the first alternative at each choice point.  If a choice results in a failure,
then the evaluator automagically\footnote{Automagically: ``Automatically, but
in a way which, for some reason (typically because it is too complicated, or
too ugly, or perhaps even too trivial), the speaker doesn't feel like
explaining.''  (\link{Steele et al. 1983}, \link{Raymond 1993})\label{Footnote 4.47}} 
\newterm{backtracks} to the most
recent choice point and tries the next alternative.  If it runs out of
alternatives at any choice point, the evaluator will back up to the previous
choice point and resume from there.  This process leads to a search strategy
known as \newterm{depth-first search} or \newterm{chronological
backtracking}.\footnote{The integration of 
automatic search strategies into programming languages has had a long and
checkered history.  The first suggestions that nondeterministic algorithms
might be elegantly encoded in a programming language with search and automatic
backtracking came from Robert \link{Floyd (1967)}.  Carl \link{Hewitt (1969)} invented a
programming language called Planner that explicitly supported automatic
chronological backtracking, providing for a built-in depth-first search
strategy.  \link{Sussman et al. (1971)} implemented a subset of this
language, called MicroPlanner, which was used to support work in problem
solving and robot planning.  Similar ideas, arising from logic and theorem
proving, led to the genesis in Edinburgh and Marseille of the elegant language
Prolog (which we will discuss in \link{Section 4.4}).  After sufficient
frustration with automatic search, \link{McDermott and Sussman (1972)} developed a
language called Conniver, which included mechanisms for placing the search
strategy under programmer control.  This proved unwieldy, however, and \link{Sussman and Stallman 1975} 
found a more tractable approach while investigating methods
of symbolic analysis for electrical circuits.  They developed a
non-chronological backtracking scheme that was based on tracing out the logical
dependencies connecting facts, a technique that has come to be known as
\newterm{dependency-directed backtracking}.  Although their method was complex,
it produced reasonably efficient programs because it did little redundant
search.  \link{Doyle (1979)} and \link{McAllester (1978; 1980)} generalized and clarified the
methods of Stallman and Sussman, developing a new paradigm for formulating
search that is now called \newterm{truth maintenance}.  Modern problem-solving
systems all use some form of truth-maintenance system as a substrate.  See
\link{Forbus and deKleer 1993} for a discussion of elegant ways to build
truth-maintenance systems and applications using truth maintenance.  
\link{Zabih et al. 1987} describes a nondeterministic extension to Scheme
that is based on \code{amb}; it is similar to the interpreter described in this
section, but more sophisticated, because it uses dependency-directed
backtracking rather than chronological backtracking.  \link{Winston 1992} gives an
introduction to both kinds of backtracking.}

\subsubsection*{Driver loop}

The driver loop for the \code{amb} evaluator has some unusual properties.  It
reads an expression and prints the value of the first non-failing execution, as
in the \code{prime\-/sum\-/pair} example shown above.  If we want to see the value
of the next successful execution, we can ask the interpreter to backtrack and
attempt to generate a second non-failing execution.  This is signaled by typing
the symbol \code{try\-/again}.  If any expression except \code{try\-/again} is
given, the interpreter will start a new problem, discarding the unexplored
alternatives in the previous problem.  Here is a sample interaction:

\begin{scheme}
~\textit{;;; Amb-Eval input:}~
(prime-sum-pair '(1 3 5 8) '(20 35 110))
~\textit{;;; Starting a new problem}~
~\textit{;;; Amb-Eval value:}~
~\textit{(3 20)}~

~\textit{;;; Amb-Eval input:}~
try-again
~\textit{;;; Amb-Eval value:}~
~\textit{(3 110)}~

~\textit{;;; Amb-Eval input:}~
try-again
~\textit{;;; Amb-Eval value:}~
~\textit{(8 35)}~

~\textit{;;; Amb-Eval input:}~
try-again
~\textit{;;; There are no more values of}~
~\textit{(prime-sum-pair (quote (1 3 5 8)) (quote (20 35 110)))}~

~\textit{;;; Amb-Eval input:}~
(prime-sum-pair '(19 27 30) '(11 36 58))
~\textit{;;; Starting a new problem}~
~\textit{;;; Amb-Eval value:}~
~\textit{(30 11)}~
\end{scheme}

\begin{quote}
\heading{\phantomsection\label{Exercise 4.35}Exercise 4.35:} Write a procedure
\code{an\-/integer\-/between} that returns an integer between two given bounds.
This can be used to implement a procedure that finds Pythagorean triples, i.e.,
triples of integers \( (i, j, k) \) between the given bounds such that
\( i \le j \) and \( i^2 + j^2 = k^2 \), as follows:

\begin{scheme}
(define (a-pythagorean-triple-between low high)
  (let ((i (an-integer-between low high)))
    (let ((j (an-integer-between i high)))
      (let ((k (an-integer-between j high)))
        (require (= (+ (* i i) (* j j)) (* k k)))
        (list i j k)))))
\end{scheme}
\end{quote}

\begin{quote}
\heading{\phantomsection\label{Exercise 4.36}Exercise 4.36:} \link{Exercise 3.69} discussed how
to generate the stream of \emph{all} Pythagorean triples, with no upper bound
on the size of the integers to be searched.  Explain why simply replacing
\code{an\-/integer\-/between} by \code{an\-/integer\-/starting\-/from} in the procedure
in \link{Exercise 4.35} is not an adequate way to generate arbitrary Pythagorean
triples.  Write a procedure that actually will accomplish this.  (That is,
write a procedure for which repeatedly typing \code{try\-/again} would in
principle eventually generate all Pythagorean triples.)
\end{quote}

\begin{quote}
\heading{\phantomsection\label{Exercise 4.37}Exercise 4.37:} Ben Bitdiddle claims that the
following method for generating Pythagorean triples is more efficient than the
one in \link{Exercise 4.35}.  Is he correct?  (Hint: Consider the number of
possibilities that must be explored.)

\begin{scheme}
(define (a-pythagorean-triple-between low high)
  (let ((i (an-integer-between low high))
        (hsq (* high high)))
    (let ((j (an-integer-between i high)))
      (let ((ksq (+ (* i i) (* j j))))
        (require (>= hsq ksq))
        (let ((k (sqrt ksq)))
          (require (integer? k))
          (list i j k))))))
\end{scheme}
\end{quote}

\subsection{Examples of Nondeterministic Programs}
\label{Section 4.3.2}

\link{Section 4.3.3} describes the implementation of the \code{amb} evaluator.
First, however, we give some examples of how it can be used.  The advantage of
nondeterministic programming is that we can suppress the details of how search
is carried out, thereby expressing our programs at a higher level of
abstraction.

\subsubsection*{Logic Puzzles}

The following puzzle (taken from \link{Dinesman 1968}) is typical of a large class of
simple logic puzzles:

\begin{quote}
Baker, Cooper, Fletcher, Miller, and Smith live on different floors of an
apartment house that contains only five floors.  Baker does not live on the top
floor.  Cooper does not live on the bottom floor.  Fletcher does not live on
either the top or the bottom floor.  Miller lives on a higher floor than does
Cooper.  Smith does not live on a floor adjacent to Fletcher's.  Fletcher does
not live on a floor adjacent to Cooper's.  Where does everyone live?
\end{quote}

\noindent
We can determine who lives on each floor in a straightforward way by
enumerating all the possibilities and imposing the given
restrictions:\footnote{Our program uses the following procedure to determine if
the elements of a list are distinct:

\begin{smallscheme}
(define (distinct? items)
  (cond ((null? items) true)
        ((null? (cdr items)) true)
        ((member (car items) (cdr items)) false)
        (else (distinct? (cdr items)))))
\end{smallscheme}

\noindent
\code{Member} is like \code{memq} except that it uses \code{equal?} instead
of \code{eq?} to test for equality.}

\begin{scheme}
(define (multiple-dwelling)
  (let ((baker    (amb 1 2 3 4 5)) (cooper (amb 1 2 3 4 5))
        (fletcher (amb 1 2 3 4 5)) (miller (amb 1 2 3 4 5))
        (smith    (amb 1 2 3 4 5)))
    (require
     (distinct? (list baker cooper fletcher miller smith)))
    (require (not (= baker 5)))
    (require (not (= cooper 1)))
    (require (not (= fletcher 5)))
    (require (not (= fletcher 1)))
    (require (> miller cooper))
    (require (not (= (abs (- smith fletcher)) 1)))
    (require (not (= (abs (- fletcher cooper)) 1)))
    (list (list 'baker baker)       (list 'cooper cooper)
          (list 'fletcher fletcher) (list 'miller miller)
          (list 'smith smith))))
\end{scheme}

\noindent
Evaluating the expression \code{(multiple\-/dwelling)} produces the result

\begin{scheme}
((baker 3) (cooper 2) (fletcher 4) (miller 5) (smith 1))
\end{scheme}

\noindent
Although this simple procedure works, it is very slow.  \link{Exercise 4.39} and
\link{Exercise 4.40} discuss some possible improvements.

\begin{quote}
\heading{\phantomsection\label{Exercise 4.38}Exercise 4.38:} Modify the multiple-dwelling
procedure to omit the requirement that Smith and Fletcher do not live on
adjacent floors.  How many solutions are there to this modified puzzle?
\end{quote}

\begin{quote}
\heading{\phantomsection\label{Exercise 4.39}Exercise 4.39:} Does the order of the
restrictions in the multiple-dwelling procedure affect the answer? Does it
affect the time to find an answer?  If you think it matters, demonstrate a
faster program obtained from the given one by reordering the restrictions.  If
you think it does not matter, argue your case.
\end{quote}

\begin{quote}
\heading{\phantomsection\label{Exercise 4.40}Exercise 4.40:} In the multiple dwelling problem,
how many sets of assignments are there of people to floors, both before and
after the requirement that floor assignments be distinct?  It is very
inefficient to generate all possible assignments of people to floors and then
leave it to backtracking to eliminate them.  For example, most of the
restrictions depend on only one or two of the person-floor variables, and can
thus be imposed before floors have been selected for all the people.  Write and
demonstrate a much more efficient nondeterministic procedure that solves this
problem based upon generating only those possibilities that are not already
ruled out by previous restrictions.  (Hint: This will require a nest of
\code{let} expressions.)
\end{quote}

\begin{quote}
\heading{\phantomsection\label{Exercise 4.41}Exercise 4.41:} Write an ordinary Scheme program
to solve the multiple dwelling puzzle.
\end{quote}

\begin{quote}
\heading{\phantomsection\label{Exercise 4.42}Exercise 4.42:} Solve the following ``Liars''
puzzle (from \link{Phillips 1934}):

Five schoolgirls sat for an examination.  Their parents---so they
thought---showed an undue degree of interest in the result.  They therefore
agreed that, in writing home about the examination, each girl should make one
true statement and one untrue one.  The following are the relevant passages
from their letters:

\begin{itemize}

\item
Betty: ``Kitty was second in the examination.  I was only third.''

\item
Ethel: ``You'll be glad to hear that I was on top.  Joan was 2nd.''

\item
Joan: ``I was third, and poor old Ethel was bottom.''

\item
Kitty: ``I came out second.  Mary was only fourth.''

\item
Mary: ``I was fourth.  Top place was taken by Betty.''

\end{itemize}

What in fact was the order in which the five girls were placed?
\end{quote}

\begin{quote}
\heading{\phantomsection\label{Exercise 4.43}Exercise 4.43:} Use the \code{amb} evaluator to
solve the following puzzle:\footnote{This is taken from a booklet called
``Problematical Recreations,'' published in the 1960s by Litton Industries,
where it is attributed to the \textit{Kansas State Engineer}.}

Mary Ann Moore's father has a yacht and so has each of his four friends:
Colonel Downing, Mr. Hall, Sir Barnacle Hood, and Dr.  Parker.  Each of the
five also has one daughter and each has named his yacht after a daughter of one
of the others.  Sir Barnacle's yacht is the Gabrielle, Mr. Moore owns the
Lorna; Mr. Hall the Rosalind.  The Melissa, owned by Colonel Downing, is named
after Sir Barnacle's daughter.  Gabrielle's father owns the yacht that is named
after Dr.  Parker's daughter.  Who is Lorna's father?

Try to write the program so that it runs efficiently (see \link{Exercise 4.40}).
Also determine how many solutions there are if we are not told that Mary Ann's
last name is Moore.
\end{quote}

\begin{quote}
\heading{\phantomsection\label{Exercise 4.44}Exercise 4.44:} \link{Exercise 2.42} described the
``eight-queens puzzle'' of placing queens on a chessboard so that no two attack
each other.  Write a nondeterministic program to solve this puzzle.
\end{quote}

\subsubsection*{Parsing natural language}

Programs designed to accept natural language as input usually start by
attempting to \newterm{parse} the input, that is, to match the input against
some grammatical structure.  For example, we might try to recognize simple
sentences consisting of an article followed by a noun followed by a verb, such
as ``The cat eats.''  To accomplish such an analysis, we must be able to
identify the parts of speech of individual words.  We could start with some
lists that classify various words:\footnote{Here we use the convention that the
first element of each list designates the part of speech for the rest of the
words in the list.}

\begin{scheme}
(define nouns '(noun student professor cat class))
(define verbs '(verb studies lectures eats sleeps))
(define articles '(article the a))
\end{scheme}

\noindent
We also need a \newterm{grammar}, that is, a set of rules describing how
grammatical elements are composed from simpler elements.  A very simple grammar
might stipulate that a sentence always consists of two pieces---a noun phrase
followed by a verb---and that a noun phrase consists of an article followed by
a noun.  With this grammar, the sentence ``The cat eats'' is parsed as follows:

\begin{scheme}
(sentence (noun-phrase (article the) (noun cat))
          (verb eats))
\end{scheme}

\noindent
We can generate such a parse with a simple program that has separate procedures
for each of the grammatical rules.  To parse a sentence, we identify its two
constituent pieces and return a list of these two elements, tagged with the
symbol \code{sentence}:

\begin{scheme}
(define (parse-sentence)
  (list 'sentence
         (parse-noun-phrase)
         (parse-word verbs)))
\end{scheme}

\noindent
A noun phrase, similarly, is parsed by finding an article followed by a
noun:

\begin{scheme}
(define (parse-noun-phrase)
  (list 'noun-phrase
        (parse-word articles)
        (parse-word nouns)))
\end{scheme}

\noindent
At the lowest level, parsing boils down to repeatedly checking that the next
unparsed word is a member of the list of words for the required part of speech.
To implement this, we maintain a global variable \code{*unparsed*}, which is
the input that has not yet been parsed.  Each time we check a word, we require
that \code{*unparsed*} must be non-empty and that it should begin with a word
from the designated list.  If so, we remove that word from \code{*unparsed*}
and return the word together with its part of speech (which is found at the
head of the list):\footnote{Notice that \code{parse\-/word} uses \code{set!} to
modify the unparsed input list.  For this to work, our \code{amb} evaluator
must undo the effects of \code{set!} operations when it backtracks.}

\begin{scheme}
(define (parse-word word-list)
  (require (not (null? *unparsed*)))
  (require (memq (car *unparsed*) (cdr word-list)))
  (let ((found-word (car *unparsed*)))
    (set! *unparsed* (cdr *unparsed*))
    (list (car word-list) found-word)))
\end{scheme}

\noindent
To start the parsing, all we need to do is set \code{*unparsed*} to be the
entire input, try to parse a sentence, and check that nothing is left over:

\begin{scheme}
(define *unparsed* '())
(define (parse input)
  (set! *unparsed* input)
  (let ((sent (parse-sentence)))
    (require (null? *unparsed*)) sent))
\end{scheme}

\noindent
We can now try the parser and verify that it works for our simple test
sentence:

\begin{scheme}
~\textit{;;; Amb-Eval input:}~
(parse '(the cat eats))
~\textit{;;; Starting a new problem}~
~\textit{;;; Amb-Eval value:}~
\end{scheme}
\begin{smallscheme}
~\textit{(sentence (noun-phrase (article the) (noun cat)) (verb eats))}~
\end{smallscheme}

\noindent
The \code{amb} evaluator is useful here because it is convenient to express the
parsing constraints with the aid of \code{require}.  Automatic search and
backtracking really pay off, however, when we consider more complex grammars
where there are choices for how the units can be decomposed.

Let's add to our grammar a list of prepositions:

\begin{scheme}
(define prepositions '(prep for to in by with))
\end{scheme}

\noindent
and define a prepositional phrase (e.g., ``for the cat'') to be a preposition
followed by a noun phrase:

\begin{scheme}
(define (parse-prepositional-phrase)
  (list 'prep-phrase
        (parse-word prepositions)
        (parse-noun-phrase)))
\end{scheme}

\noindent
Now we can define a sentence to be a noun phrase followed by a verb phrase,
where a verb phrase can be either a verb or a verb phrase extended by a
prepositional phrase:\footnote{Observe that this definition is recursive---a
verb may be followed by any number of prepositional phrases.}

\begin{scheme}
(define (parse-sentence)
  (list 'sentence (parse-noun-phrase) (parse-verb-phrase)))
(define (parse-verb-phrase)
  (define (maybe-extend verb-phrase)
    (amb verb-phrase
         (maybe-extend
          (list 'verb-phrase
                verb-phrase
                (parse-prepositional-phrase)))))
  (maybe-extend (parse-word verbs)))
\end{scheme}

\noindent
While we're at it, we can also elaborate the definition of noun phrases to
permit such things as ``a cat in the class.''  What we used to call a noun
phrase, we'll now call a simple noun phrase, and a noun phrase will now be
either a simple noun phrase or a noun phrase extended by a prepositional
phrase:

\begin{scheme}
(define (parse-simple-noun-phrase)
  (list 'simple-noun-phrase
        (parse-word articles)
        (parse-word nouns)))
(define (parse-noun-phrase)
  (define (maybe-extend noun-phrase)
    (amb noun-phrase
         (maybe-extend
          (list 'noun-phrase
                noun-phrase
                (parse-prepositional-phrase)))))
  (maybe-extend (parse-simple-noun-phrase)))
\end{scheme}

\noindent
Our new grammar lets us parse more complex sentences.  For example

\begin{scheme}
(parse '(the student with the cat sleeps in the class))
\end{scheme}

\noindent
produces

\begin{scheme}
(sentence
 (noun-phrase
  (simple-noun-phrase (article the) (noun student))
  (prep-phrase
   (prep with)
   (simple-noun-phrase (article the) (noun cat))))
 (verb-phrase
  (verb sleeps)
  (prep-phrase
   (prep in)
   (simple-noun-phrase (article the) (noun class)))))
\end{scheme}

\noindent
Observe that a given input may have more than one legal parse.  In the sentence
``The professor lectures to the student with the cat,'' it may be that the
professor is lecturing with the cat, or that the student has the cat.  Our
nondeterministic program finds both possibilities:

\begin{scheme}
(parse '(the professor lectures to the student with the cat))
\end{scheme}

\noindent
produces

\begin{scheme}
(sentence
 (simple-noun-phrase (article the) (noun professor))
 (verb-phrase
  (verb-phrase
   (verb lectures)
   (prep-phrase
    (prep to)
    (simple-noun-phrase (article the) (noun student))))
  (prep-phrase
   (prep with)
   (simple-noun-phrase (article the) (noun cat)))))
\end{scheme}

\noindent
Asking the evaluator to try again yields

\begin{scheme}
(sentence
 (simple-noun-phrase (article the) (noun professor))
 (verb-phrase
  (verb lectures)
  (prep-phrase
   (prep to)
   (noun-phrase
    (simple-noun-phrase (article the) (noun student))
    (prep-phrase
     (prep with)
     (simple-noun-phrase (article the) (noun cat)))))))
\end{scheme}

\begin{quote}
\heading{\phantomsection\label{Exercise 4.45}Exercise 4.45:} With the grammar given above, the
following sentence can be parsed in five different ways: ``The professor
lectures to the student in the class with the cat.''  Give the five parses and
explain the differences in shades of meaning among them.
\end{quote}

\begin{quote}
\heading{\phantomsection\label{Exercise 4.46}Exercise 4.46:} The evaluators in 
\link{Section 4.1} and \link{Section 4.2} do not determine what order operands are evaluated in.
We will see that the \code{amb} evaluator evaluates them from left to right.
Explain why our parsing program wouldn't work if the operands were evaluated in
some other order.
\end{quote}

\begin{quote}
\heading{\phantomsection\label{Exercise 4.47}Exercise 4.47:} Louis Reasoner suggests that,
since a verb phrase is either a verb or a verb phrase followed by a
prepositional phrase, it would be much more straightforward to define the
procedure \code{parse\-/verb\-/phrase} as follows (and similarly for noun phrases):

\begin{scheme}
(define (parse-verb-phrase)
  (amb (parse-word verbs)
       (list 'verb-phrase
             (parse-verb-phrase)
             (parse-prepositional-phrase))))
\end{scheme}

Does this work?  Does the program's behavior change if we interchange the order
of expressions in the \code{amb}?
\end{quote}

\begin{quote}
\heading{\phantomsection\label{Exercise 4.48}Exercise 4.48:} Extend the grammar given above to
handle more complex sentences.  For example, you could extend noun phrases and
verb phrases to include adjectives and adverbs, or you could handle compound
sentences.\footnote{This kind of grammar can become arbitrarily complex, but it
is only a toy as far as real language understanding is concerned.  Real
natural-language understanding by computer requires an elaborate mixture of
syntactic analysis and interpretation of meaning.  On the other hand, even toy
parsers can be useful in supporting flexible command languages for programs
such as information-retrieval systems.  \link{Winston 1992} discusses computational
approaches to real language understanding and also the applications of simple
grammars to command languages.}
\end{quote}

\begin{quote}
\heading{\phantomsection\label{Exercise 4.49}Exercise 4.49:} Alyssa P. Hacker is more
interested in generating interesting sentences than in parsing them.  She
reasons that by simply changing the procedure \code{parse\-/word} so that it
ignores the ``input sentence'' and instead always succeeds and generates an
appropriate word, we can use the programs we had built for parsing to do
generation instead.  Implement Alyssa's idea, and show the first half-dozen or
so sentences generated.\footnote{Although Alyssa's idea works just fine (and is
surprisingly simple), the sentences that it generates are a bit boring---they
don't sample the possible sentences of this language in a very interesting way.
In fact, the grammar is highly recursive in many places, and Alyssa's technique
``falls into'' one of these recursions and gets stuck.  See \link{Exercise 4.50}
for a way to deal with this.}
\end{quote}

\subsection{Implementing the \code{Amb} Evaluator}
\label{Section 4.3.3}

The evaluation of an ordinary Scheme expression may return a value, may never
terminate, or may signal an error.  In nondeterministic Scheme the evaluation
of an expression may in addition result in the discovery of a dead end, in
which case evaluation must backtrack to a previous choice point.  The
interpretation of nondeterministic Scheme is complicated by this extra case.

We will construct the \code{amb} evaluator for nondeterministic Scheme by
modifying the analyzing evaluator of \link{Section 4.1.7}.\footnote{We chose to
implement the lazy evaluator in \link{Section 4.2} as a modification of the
ordinary metacircular evaluator of \link{Section 4.1.1}.  In contrast, we will
base the \code{amb} evaluator on the analyzing evaluator of 
\link{Section 4.1.7}, because the execution procedures in that evaluator provide a
convenient framework for implementing backtracking.}  As in the analyzing
evaluator, evaluation of an expression is accomplished by calling an execution
procedure produced by analysis of that expression.  The difference between the
interpretation of ordinary Scheme and the interpretation of nondeterministic
Scheme will be entirely in the execution procedures.

\subsubsection*{Execution procedures and continuations}

Recall that the execution procedures for the ordinary evaluator take one
argument: the environment of execution.  In contrast, the execution procedures
in the \code{amb} evaluator take three arguments: the environment, and two
procedures called \newterm{continuation procedures}.  The evaluation of an
expression will finish by calling one of these two continuations: If the
evaluation results in a value, the \newterm{success continuation} is called
with that value; if the evaluation results in the discovery of a dead end, the
\newterm{failure continuation} is called.  Constructing and calling appropriate
continuations is the mechanism by which the nondeterministic evaluator
implements backtracking.

It is the job of the success continuation to receive a value and proceed with
the computation.  Along with that value, the success continuation is passed
another failure continuation, which is to be called subsequently if the use of
that value leads to a dead end.

It is the job of the failure continuation to try another branch of the
nondeterministic process.  The essence of the nondeterministic language is in
the fact that expressions may represent choices among alternatives.  The
evaluation of such an expression must proceed with one of the indicated
alternative choices, even though it is not known in advance which choices will
lead to acceptable results.  To deal with this, the evaluator picks one of the
alternatives and passes this value to the success continuation.  Together with
this value, the evaluator constructs and passes along a failure continuation
that can be called later to choose a different alternative.

A failure is triggered during evaluation (that is, a failure continuation is
called) when a user program explicitly rejects the current line of attack (for
example, a call to \code{require} may result in execution of \code{(amb)}, an
expression that always fails---see \link{Section 4.3.1}).  The failure
continuation in hand at that point will cause the most recent choice point to
choose another alternative.  If there are no more alternatives to be considered
at that choice point, a failure at an earlier choice point is triggered, and so
on.  Failure continuations are also invoked by the driver loop in response to a
\code{try\-/again} request, to find another value of the expression.

In addition, if a side-effect operation (such as assignment to a variable)
occurs on a branch of the process resulting from a choice, it may be necessary,
when the process finds a dead end, to undo the side effect before making a new
choice.  This is accomplished by having the side-effect operation produce a
failure continuation that undoes the side effect and propagates the failure.

\enlargethispage{\baselineskip}

In summary, failure continuations are constructed by

\begin{itemize}

\item
\code{amb} expressions---to provide a mechanism to make alternative choices if
the current choice made by the \code{amb} expression leads to a dead end;

\item
the top-level driver---to provide a mechanism to report failure when the
choices are exhausted;

\item
assignments---to intercept failures and undo assignments during backtracking.

\end{itemize}

\noindent
Failures are initiated only when a dead end is encountered.  This occurs

\begin{itemize}

\item
if the user program executes \code{(amb)};

\item
if the user types \code{try\-/again} at the top-level driver.

\end{itemize}

\noindent
Failure continuations are also called during processing of a failure:

\begin{itemize}

\item
When the failure continuation created by an assignment finishes undoing a side
effect, it calls the failure continuation it intercepted, in order to propagate
the failure back to the choice point that led to this assignment or to the top
level.

\item
When the failure continuation for an \code{amb} runs out of choices, it calls
the failure continuation that was originally given to the \code{amb}, in order
to propagate the failure back to the previous choice point or to the top level.

\end{itemize}

\subsubsection*{Structure of the evaluator}

The syntax- and data-representation procedures for the \code{amb} evaluator,
and also the basic \code{analyze} procedure, are identical to those in the
evaluator of \link{Section 4.1.7}, except for the fact that we need additional
syntax procedures to recognize the \code{amb} special form:\footnote{We assume
that the evaluator supports \code{let} (see \link{Exercise 4.22}), which we have
used in our nondeterministic programs.}

\begin{scheme}
(define (amb? exp) (tagged-list? exp 'amb))
(define (amb-choices exp) (cdr exp))
\end{scheme}

\noindent
We must also add to the dispatch in \code{analyze} a clause that will recognize
this special form and generate an appropriate execution procedure:

\begin{scheme}
((amb? exp) (analyze-amb exp))
\end{scheme}

\noindent
The top-level procedure \code{ambeval} (similar to the version of \code{eval}
given in \link{Section 4.1.7}) analyzes the given expression and applies the
resulting execution procedure to the given environment, together with two given
continuations:

\begin{scheme}
(define (ambeval exp env succeed fail)
  ((analyze exp) env succeed fail))
\end{scheme}

\noindent
A success continuation is a procedure of two arguments: the value just obtained
and another failure continuation to be used if that value leads to a subsequent
failure. A failure continuation is a procedure of no arguments.  So the general
form of an execution procedure is

\begin{scheme}
(lambda (env succeed fail)
  ~\textrm{;; \code{succeed} is \code{(lambda (value fail) \( \dots \))}}~
  ~\textrm{;; \code{fail} is \code{(lambda () \( \dots \))}}~
  ~\( \dots \)~)
\end{scheme}

\noindent
For example, executing

\begin{scheme}
(ambeval ~\( \dark \langle \)~~\var{\dark exp}~~\( \dark \rangle \)~
         the-global-environment
         (lambda (value fail) value)
         (lambda () 'failed))
\end{scheme}

\noindent
will attempt to evaluate the given expression and will return either the
expression's value (if the evaluation succeeds) or the symbol \code{failed} (if
the evaluation fails).  The call to \code{ambeval} in the driver loop shown
below uses much more complicated continuation procedures, which continue the
loop and support the \code{try\-/again} request.

Most of the complexity of the \code{amb} evaluator results from the mechanics
of passing the continuations around as the execution procedures call each
other.  In going through the following code, you should compare each of the
execution procedures with the corresponding procedure for the ordinary
evaluator given in \link{Section 4.1.7}.

\subsubsection*{Simple expressions}

The execution procedures for the simplest kinds of expressions are essentially
the same as those for the ordinary evaluator, except for the need to manage the
continuations.  The execution procedures simply succeed with the value of the
expression, passing along the failure continuation that was passed to them.

\begin{scheme}
(define (analyze-self-evaluating exp)
  (lambda (env succeed fail)
    (succeed exp fail)))
(define (analyze-quoted exp)
  (let ((qval (text-of-quotation exp)))
    (lambda (env succeed fail)
      (succeed qval fail))))
(define (analyze-variable exp)
  (lambda (env succeed fail)
    (succeed (lookup-variable-value exp env) fail)))
(define (analyze-lambda exp)
  (let ((vars (lambda-parameters exp))
        (bproc (analyze-sequence (lambda-body exp))))
    (lambda (env succeed fail)
      (succeed (make-procedure vars bproc env) fail))))
\end{scheme}

\noindent
Notice that looking up a variable always `succeeds.'  If
\code{lookup\-/variable\-/value} fails to find the variable, it signals an error,
as usual.  Such a ``failure'' indicates a program bug---a reference to an
unbound variable; it is not an indication that we should try another
nondeterministic choice instead of the one that is currently being tried.

\subsubsection*{Conditionals and sequences}

Conditionals are also handled in a similar way as in the ordinary evaluator.
The execution procedure generated by \code{analyze\-/if} invokes the predicate
execution procedure \code{pproc} with a success continuation that checks
whether the predicate value is true and goes on to execute either the
consequent or the alternative.  If the execution of \code{pproc} fails, the
original failure continuation for the \code{if} expression is called.

\begin{scheme}
(define (analyze-if exp)
  (let ((pproc (analyze (if-predicate exp)))
        (cproc (analyze (if-consequent exp)))
        (aproc (analyze (if-alternative exp))))
    (lambda (env succeed fail)
      (pproc env
             ~\textrm{;; success continuation for evaluating the predicate}~
             ~\textrm{;; to obtain \code{pred\-/value}}~
             (lambda (pred-value fail2)
               (if (true? pred-value)
                   (cproc env succeed fail2)
                   (aproc env succeed fail2)))
             ~\textrm{;; failure continuation for evaluating the predicate}~
             fail))))
\end{scheme}

\noindent
Sequences are also handled in the same way as in the previous evaluator, except
for the machinations in the subprocedure \code{sequentially} that are required
for passing the continuations.  Namely, to sequentially execute \code{a} and
then \code{b}, we call \code{a} with a success continuation that calls
\code{b}.

\begin{scheme}
(define (analyze-sequence exps)
  (define (sequentially a b)
    (lambda (env succeed fail)
      (a env
         ~\textrm{;; success continuation for calling \code{a}}~
         (lambda (a-value fail2)
           (b env succeed fail2))
         ~\textrm{;; failure continuation for calling \code{a}}~
         fail)))
  (define (loop first-proc rest-procs)
    (if (null? rest-procs)
        first-proc
        (loop (sequentially first-proc
                            (car rest-procs))
              (cdr rest-procs))))
  (let ((procs (map analyze exps)))
    (if (null? procs)
        (error "Empty sequence: ANALYZE"))
    (loop (car procs) (cdr procs))))
\end{scheme}

\subsubsection*{Definitions and assignments}

Definitions are another case where we must go to some trouble to manage the
continuations, because it is necessary to evaluate the definition-value
expression before actually defining the new variable.  To accomplish this, the
definition-value execution procedure \code{vproc} is called with the
environment, a success continuation, and the failure continuation.  If the
execution of \code{vproc} succeeds, obtaining a value \code{val} for the
defined variable, the variable is defined and the success is propagated:

\begin{scheme}
(define (analyze-definition exp)
  (let ((var (definition-variable exp))
        (vproc (analyze (definition-value exp))))
    (lambda (env succeed fail)
      (vproc env
             (lambda (val fail2)
               (define-variable! var val env)
               (succeed 'ok fail2))
             fail))))
\end{scheme}

\noindent
Assignments are more interesting.  This is the first place where we really use
the continuations, rather than just passing them around.  The execution
procedure for assignments starts out like the one for definitions.  It first
attempts to obtain the new value to be assigned to the variable. If this
evaluation of \code{vproc} fails, the assignment fails.

If \code{vproc} succeeds, however, and we go on to make the assignment, we must
consider the possibility that this branch of the computation might later fail,
which will require us to backtrack out of the assignment.  Thus, we must
arrange to undo the assignment as part of the backtracking process.\footnote{We
didn't worry about undoing definitions, since we can assume that internal
definitions are scanned out (\link{Section 4.1.6}).}

This is accomplished by giving \code{vproc} a success continuation (marked with
the comment ``*1*'' below) that saves the old value of the variable before
assigning the new value to the variable and proceeding from the assignment.
The failure continuation that is passed along with the value of the assignment
(marked with the comment ``*2*'' below) restores the old value of the variable
before continuing the failure.  That is, a successful assignment provides a
failure continuation that will intercept a subsequent failure; whatever failure
would otherwise have called \code{fail2} calls this procedure instead, to undo
the assignment before actually calling \code{fail2}.

\begin{scheme}
(define (analyze-assignment exp)
  (let ((var (assignment-variable exp))
        (vproc (analyze (assignment-value exp))))
    (lambda (env succeed fail)
      (vproc env
             (lambda (val fail2)        ~\textrm{; *1*}~
               (let ((old-value
                      (lookup-variable-value var env)))
                 (set-variable-value! var val env)
                 (succeed 'ok
                          (lambda ()    ~\textrm{; *2*}~
                            (set-variable-value!
                             var old-value env)
                            (fail2)))))
             fail))))
\end{scheme}

\subsubsection*{Procedure applications}

The execution procedure for applications contains no new ideas except for the
technical complexity of managing the continuations.  This complexity arises in
\code{analyze\-/application}, due to the need to keep track of the success and
failure continuations as we evaluate the operands.  We use a procedure
\code{get\-/args} to evaluate the list of operands, rather than a simple
\code{map} as in the ordinary evaluator.

\begin{scheme}
(define (analyze-application exp)
  (let ((fproc (analyze (operator exp)))
        (aprocs (map analyze (operands exp))))
    (lambda (env succeed fail)
      (fproc env
             (lambda (proc fail2)
               (get-args aprocs
                         env
                         (lambda (args fail3)
                           (execute-application
                            proc args succeed fail3))
                         fail2))
             fail))))
\end{scheme}

\noindent
In \code{get\-/args}, notice how \code{cdr}-ing down the list of \code{aproc}
execution procedures and \code{cons}ing up the resulting list of \code{args} is
accomplished by calling each \code{aproc} in the list with a success
continuation that recursively calls \code{get\-/args}.  Each of these recursive
calls to \code{get\-/args} has a success continuation whose value is the
\code{cons} of the newly obtained argument onto the list of accumulated
arguments:

\begin{scheme}
(define (get-args aprocs env succeed fail)
  (if (null? aprocs)
      (succeed '() fail)
      ((car aprocs)
       env
       ;;~\textrm{success continuation for this \code{aproc}}~
       (lambda (arg fail2)
         (get-args
          (cdr aprocs)
          env
          ;;~\textrm{success continuation for}~
          ;;~\textrm{recursive call to \code{get\-/args}}~
          (lambda (args fail3)
            (succeed (cons arg args) fail3))
          fail2))
       fail)))
\end{scheme}

\noindent
The actual procedure application, which is performed by
\code{execute\-/appli\-/cation}, is accomplished in the same way as for the ordinary
evaluator, except for the need to manage the continuations.

\begin{scheme}
(define (execute-application proc args succeed fail)
  (cond ((primitive-procedure? proc)
         (succeed (apply-primitive-procedure proc args)
                  fail))
        ((compound-procedure? proc)
         ((procedure-body proc)
          (extend-environment
           (procedure-parameters proc)
           args
           (procedure-environment proc))
          succeed
          fail))
        (else
         (error "Unknown procedure type: EXECUTE-APPLICATION"
                proc))))
\end{scheme}

\subsubsection*{Evaluating \code{amb} expressions}

The \code{amb} special form is the key element in the nondeterministic
language.  Here we see the essence of the interpretation process and the reason
for keeping track of the continuations.  The execution procedure for \code{amb}
defines a loop \code{try\-/next} that cycles through the execution procedures for
all the possible values of the \code{amb} expression.  Each execution procedure
is called with a failure continuation that will try the next one.  When there
are no more alternatives to try, the entire \code{amb} expression fails.

\begin{scheme}
(define (analyze-amb exp)
  (let ((cprocs (map analyze (amb-choices exp))))
    (lambda (env succeed fail)
      (define (try-next choices)
        (if (null? choices)
            (fail)
            ((car choices)
             env
             succeed
             (lambda () (try-next (cdr choices))))))
      (try-next cprocs))))
\end{scheme}

\subsubsection*{Driver loop}

The driver loop for the \code{amb} evaluator is complex, due to the mechanism
that permits the user to try again in evaluating an expression.  The driver
uses a procedure called \code{internal\-/loop}, which takes as argument a
procedure \code{try\-/again}.  The intent is that calling \code{try\-/again} should
go on to the next untried alternative in the nondeterministic evaluation.
\code{Internal\-/loop} either calls \code{try\-/again} in response to the user
typing \code{try\-/again} at the driver loop, or else starts a new evaluation by
calling \code{ambeval}.

The failure continuation for this call to \code{ambeval} informs the user that
there are no more values and re-invokes the driver loop.

The success continuation for the call to \code{ambeval} is more subtle.  We
print the obtained value and then invoke the internal loop again with a
\code{try\-/again} procedure that will be able to try the next alternative.  This
\code{next\-/alternative} procedure is the second argument that was passed to the
success continuation.  Ordinarily, we think of this second argument as a
failure continuation to be used if the current evaluation branch later fails.
In this case, however, we have completed a successful evaluation, so we can
invoke the ``failure'' alternative branch in order to search for additional
successful evaluations.

\begin{scheme}
(define input-prompt  ";;; Amb-Eval input:")
(define output-prompt ";;; Amb-Eval value:")

(define (driver-loop)
  (define (internal-loop try-again)
    (prompt-for-input input-prompt)
    (let ((input (read)))
      (if (eq? input 'try-again)
          (try-again)
          (begin
            (newline) (display ";;; Starting a new problem ")
            (ambeval
             input
             the-global-environment
             ;; ~\textrm{\code{ambeval} success}~
             (lambda (val next-alternative)
               (announce-output output-prompt)
               (user-print val)
               (internal-loop next-alternative))
             ;; ~\textrm{\code{ambeval} failure}~
             (lambda ()
               (announce-output
                ";;; There are no more values of")
               (user-print input)
               (driver-loop)))))))
  (internal-loop
   (lambda ()
     (newline) (display ";;; There is no current problem")
     (driver-loop))))
\end{scheme}

\noindent
The initial call to \code{internal\-/loop} uses a \code{try\-/again} procedure that
complains that there is no current problem and restarts the driver loop.  This
is the behavior that will happen if the user types \code{try\-/again} when there
is no evaluation in progress.

\begin{quote}
\heading{\phantomsection\label{Exercise 4.50}Exercise 4.50:} Implement a new special form
\code{ramb} that is like \code{amb} except that it searches alternatives in a
random order, rather than from left to right.  Show how this can help with
Alyssa's problem in \link{Exercise 4.49}.
\end{quote}

\begin{quote}
\heading{\phantomsection\label{Exercise 4.51}Exercise 4.51:} Implement a new kind of
assignment called \code{permanent\-/set!} that is not undone upon failure.  For
example, we can choose two distinct elements from a list and count the number
of trials required to make a successful choice as follows:

\begin{scheme}
(define count 0)
(let ((x (an-element-of '(a b c)))
      (y (an-element-of '(a b c))))
  (permanent-set! count (+ count 1))
  (require (not (eq? x y)))
  (list x y count))
~\textit{;;; Starting a new problem}~
~\textit{;;; Amb-Eval value:}~
~\textit{(a b 2)}~
~\textit{;;; Amb-Eval input:}~
try-again
~\textit{;;; Amb-Eval value:}~
~\textit{(a c 3)}~
\end{scheme}

What values would have been displayed if we had used \code{set!} here
rather than \code{permanent\-/set!} ?
\end{quote}

\begin{quote}
\heading{\phantomsection\label{Exercise 4.52}Exercise 4.52:} Implement a new construct called
\code{if\-/fail} that permits the user to catch the failure of an expression.
\code{If\-/fail} takes two expressions.  It evaluates the first expression as
usual and returns as usual if the evaluation succeeds.  If the evaluation
fails, however, the value of the second expression is returned, as in the
following example:

\begin{scheme}
~\textit{;;; Amb-Eval input:}~
(if-fail (let ((x (an-element-of '(1 3 5))))
           (require (even? x))
           x)
         'all-odd)
~\textit{;;; Starting a new problem}~
~\textit{;;; Amb-Eval value:}~
~\textit{all-odd}~

~\textit{;;; Amb-Eval input:}~
(if-fail (let ((x (an-element-of '(1 3 5 8))))
           (require (even? x))
           x)
         'all-odd)
~\textit{;;; Starting a new problem}~
~\textit{;;; Amb-Eval value:}~
~\textit{8}~
\end{scheme}
\end{quote}

\begin{quote}
\heading{\phantomsection\label{Exercise 4.53}Exercise 4.53:} With \code{permanent\-/set!} as
described in \link{Exercise 4.51} and \code{if\-/fail} as in 
\link{Exercise 4.52}, what will be the result of evaluating

\begin{scheme}
(let ((pairs '()))
  (if-fail 
   (let ((p (prime-sum-pair '(1 3 5 8)
                            '(20 35 110))))
     (permanent-set! pairs (cons p pairs))
     (amb))
   pairs))
\end{scheme}
\end{quote}

\begin{quote}
\heading{\phantomsection\label{Exercise 4.54}Exercise 4.54:} If we had not realized that
\code{require} could be implemented as an ordinary procedure that uses
\code{amb}, to be defined by the user as part of a nondeterministic program, we
would have had to implement it as a special form.  This would require syntax
procedures

\begin{scheme}
(define (require? exp)
  (tagged-list? exp 'require))
(define (require-predicate exp) 
  (cadr exp))
\end{scheme}

\noindent
and a new clause in the dispatch in \code{analyze}

\begin{scheme}
((require? exp) (analyze-require exp))
\end{scheme}

\noindent
as well the procedure \code{analyze\-/require} that handles \code{require}
expressions.  Complete the following definition of \code{analyze\-/require}.

\begin{scheme}
(define (analyze-require exp)
  (let ((pproc (analyze (require-predicate exp))))
    (lambda (env succeed fail)
      (pproc env
             (lambda (pred-value fail2)
               (if ~\( \dark \langle \)~??~\( \dark \rangle \)~
                   ~\( \dark \langle \)~??~\( \dark \rangle \)~
                   (succeed 'ok fail2)))
             fail))))

\end{scheme}
\end{quote}


\section{Logic Programming}
\label{Section 4.4}

In \link{Chapter 1} we stressed that computer science deals with imperative (how
to) knowledge, whereas mathematics deals with declarative (what is) knowledge.
Indeed, programming languages require that the programmer express knowledge in
a form that indicates the step-by-step methods for solving particular problems.
On the other hand, high-level languages provide, as part of the language
implementation, a substantial amount of methodological knowledge that frees the
user from concern with numerous details of how a specified computation will
progress.

Most programming languages, including Lisp, are organized around computing the
values of mathematical functions.  Expression-oriented languages (such as Lisp,
Fortran, and Algol) capitalize on the ``pun'' that an expression that describes
the value of a function may also be interpreted as a means of computing that
value.  Because of this, most programming languages are strongly biased toward
unidirectional computations (computations with well-defined inputs and
outputs).  There are, however, radically different programming languages that
relax this bias.  We saw one such example in \link{Section 3.3.5}, where the
objects of computation were arithmetic constraints.  In a constraint system the
direction and the order of computation are not so well specified; in carrying
out a computation the system must therefore provide more detailed ``how to''
knowledge than would be the case with an ordinary arithmetic computation.  This
does not mean, however, that the user is released altogether from the
responsibility of providing imperative knowledge.  There are many constraint
networks that implement the same set of constraints, and the user must choose
from the set of mathematically equivalent networks a suitable network to
specify a particular computation.

The nondeterministic program evaluator of \link{Section 4.3} also moves away
from the view that programming is about constructing algorithms for computing
unidirectional functions.  In a nondeterministic language, expressions can have
more than one value, and, as a result, the computation is dealing with
relations rather than with single-valued functions.  Logic programming extends
this idea by combining a relational vision of programming with a powerful kind
of symbolic pattern matching called \newterm{unification}.\footnote{Logic
programming has grown out of a long history of research in automatic theorem
proving.  Early theorem-proving programs could accomplish very little, because
they exhaustively searched the space of possible proofs.  The major
breakthrough that made such a search plausible was the discovery in the early
1960s of the \newterm{unification algorithm} and the \newterm{resolution
principle} (\link{Robinson 1965}).  Resolution was used, for example, by 
\link{Green and Raphael (1968)} (see also \link{Green 1969}) as the basis for a deductive
question-answering system.  During most of this period, researchers
concentrated on algorithms that are guaranteed to find a proof if one exists.
Such algorithms were difficult to control and to direct toward a proof.  
\link{Hewitt (1969)} recognized the possibility of merging the control structure of a
programming language with the operations of a logic-manipulation system,
leading to the work in automatic search mentioned in \link{Section 4.3.1}
(\link{Footnote 4.47}).  At the same time that this was being done,
Colmerauer, in Marseille, was developing rule-based systems for manipulating
natural language (see \link{Colmerauer et al. 1973}).  He invented a programming
language called Prolog for representing those rules.  \link{Kowalski (1973; 1979)}, in
Edinburgh, recognized that execution of a Prolog program could be interpreted
as proving theorems (using a proof technique called linear Horn-clause
resolution).  The merging of the last two strands led to the logic-programming
movement.  Thus, in assigning credit for the development of logic programming,
the French can point to Prolog's genesis at the University of Marseille, while
the British can highlight the work at the University of Edinburgh.  According
to people at \acronym{MIT}, logic programming was developed by these groups in
an attempt to figure out what Hewitt was talking about in his brilliant but
impenetrable Ph.D. thesis.  For a history of logic programming, see 
\link{Robinson 1983}.}

This approach, when it works, can be a very powerful way to write programs.
Part of the power comes from the fact that a single ``what is'' fact can be
used to solve a number of different problems that would have different ``how
to'' components.  As an example, consider the \code{append} operation, which
takes two lists as arguments and combines their elements to form a single list.
In a procedural language such as Lisp, we could define \code{append} in terms
of the basic list constructor \code{cons}, as we did in \link{Section 2.2.1}:

\begin{scheme}

(define (append x y)
  (if (null? x) y (cons (car x) (append (cdr x) y))))
\end{scheme}

\noindent
This procedure can be regarded as a translation into Lisp of the following two
rules, the first of which covers the case where the first list is empty and the
second of which handles the case of a nonempty list, which is a \code{cons} of
two parts:

\begin{itemize}

\item
For any list \code{y}, the empty list and \code{y} \code{append} to form
\code{y}.

\item
For any \code{u}, \code{v}, \code{y}, and \code{z}, \code{(cons u v)} and
\code{y} \code{append} to form \code{(cons u z)} if \code{v} and \code{y}
\code{append} to form \code{z}.\footnote{To see the correspondence between the
rules and the procedure, let \code{x} in the procedure (where \code{x} is
nonempty) correspond to \code{(cons u v)} in the rule.  Then \code{z} in the
rule corresponds to the \code{append} of \code{(cdr x)} and \code{y}.}

\end{itemize}

\noindent
Using the \code{append} procedure, we can answer questions such as

\begin{quote}
Find the \code{append} of \code{(a b)} and \code{(c d)}.
\end{quote}

\noindent
But the same two rules are also sufficient for answering the following sorts of
questions, which the procedure can't answer:

\begin{quote}
Find a list \code{y} that \code{append}s with \code{(a b)} to produce \code{(a
b c d)}.

Find all \code{x} and \code{y} that \code{append} to form \code{(a b c d)}.
\end{quote}

\noindent
In a logic programming language, the programmer writes an \code{append}
``procedure'' by stating the two rules about \code{append} given above.  ``How
to'' knowledge is provided automatically by the interpreter to allow this
single pair of rules to be used to answer all three types of questions about
\code{append}.\footnote{This certainly does not relieve the user of the entire
problem of how to compute the answer.  There are many different mathematically
equivalent sets of rules for formulating the \code{append} relation, only some
of which can be turned into effective devices for computing in any direction.
In addition, sometimes ``what is'' information gives no clue ``how to'' compute
an answer.  For example, consider the problem of computing the \( y \) such that
\( y^2 = x \).}

Contemporary logic programming languages (including the one we implement here)
have substantial deficiencies, in that their general ``how to'' methods can
lead them into spurious infinite loops or other undesirable behavior.  Logic
programming is an active field of research in computer
science.\footnote{Interest in logic programming peaked during the early 80s
when the Japanese government began an ambitious project aimed at building
superfast computers optimized to run logic programming languages.  The speed of
such computers was to be measured in LIPS (Logical Inferences Per Second)
rather than the usual FLOPS (FLoating-point Operations Per Second).  Although
the project succeeded in developing hardware and software as originally
planned, the international computer industry moved in a different direction.
See \link{Feigenbaum and Shrobe 1993} for an overview evaluation of the Japanese
project.  The logic programming community has also moved on to consider
relational programming based on techniques other than simple pattern matching,
such as the ability to deal with numerical constraints such as the ones
illustrated in the constraint-propagation system of \link{Section 3.3.5}.}

Earlier in this chapter we explored the technology of implementing interpreters
and described the elements that are essential to an interpreter for a Lisp-like
language (indeed, to an interpreter for any conventional language).  Now we
will apply these ideas to discuss an interpreter for a logic programming
language.  We call this language the \newterm{query language}, because it is
very useful for retrieving information from data bases by formulating
\newterm{queries}, or questions, expressed in the language.  Even though the
query language is very different from Lisp, we will find it convenient to
describe the language in terms of the same general framework we have been using
all along: as a collection of primitive elements, together with means of
combination that enable us to combine simple elements to create more complex
elements and means of abstraction that enable us to regard complex elements as
single conceptual units.  An interpreter for a logic programming language is
considerably more complex than an interpreter for a language like Lisp.
Nevertheless, we will see that our query-language interpreter contains many of
the same elements found in the interpreter of \link{Section 4.1}.  In
particular, there will be an ``eval'' part that classifies expressions
according to type and an ``apply'' part that implements the language's
abstraction mechanism (procedures in the case of Lisp, and \newterm{rules} in
the case of logic programming).  Also, a central role is played in the
implementation by a frame data structure, which determines the correspondence
between symbols and their associated values.  One additional interesting aspect
of our query-language implementation is that we make substantial use of
streams, which were introduced in \link{Chapter 3}.



\subsection{Deductive Information Retrieval}
\label{Section 4.4.1}

Logic programming excels in providing interfaces to data bases for information
retrieval.  The query language we shall implement in this chapter is designed
to be used in this way.

In order to illustrate what the query system does, we will show how it can be
used to manage the data base of personnel records for Microshaft, a thriving
high-technology company in the Boston area.  The language provides
pattern-directed access to personnel information and can also take advantage of
general rules in order to make logical deductions.

\subsubsection*{A sample data base}

The personnel data base for Microshaft contains \newterm{assertions} about
company personnel.  Here is the information about Ben Bitdiddle, the resident
computer wizard:

\begin{scheme}
(address (Bitdiddle Ben) (Slumerville (Ridge Road) 10))
(job (Bitdiddle Ben) (computer wizard))
(salary (Bitdiddle Ben) 60000)
\end{scheme}

\noindent
Each assertion is a list (in this case a triple) whose elements can themselves
be lists.

As resident wizard, Ben is in charge of the company's computer division, and he
supervises two programmers and one technician.  Here is the information about
them:

\begin{scheme}
(address (Hacker Alyssa P) (Cambridge (Mass Ave) 78))
(job (Hacker Alyssa P) (computer programmer))
(salary (Hacker Alyssa P) 40000)
(supervisor (Hacker Alyssa P) (Bitdiddle Ben))

(address (Fect Cy D) (Cambridge (Ames Street) 3))
(job (Fect Cy D) (computer programmer))
(salary (Fect Cy D) 35000)
(supervisor (Fect Cy D) (Bitdiddle Ben))

(address (Tweakit Lem E) (Boston (Bay State Road) 22))
(job (Tweakit Lem E) (computer technician))
(salary (Tweakit Lem E) 25000)
(supervisor (Tweakit Lem E) (Bitdiddle Ben))
\end{scheme}

\noindent
There is also a programmer trainee, who is supervised by Alyssa:

\begin{scheme}
(address (Reasoner Louis) (Slumerville (Pine Tree Road) 80))
(job (Reasoner Louis) (computer programmer trainee))
(salary (Reasoner Louis) 30000)
(supervisor (Reasoner Louis) (Hacker Alyssa P))
\end{scheme}

\noindent
All of these people are in the computer division, as indicated by the word
\code{computer} as the first item in their job descriptions.

Ben is a high-level employee.  His supervisor is the company's big wheel
himself:

\begin{scheme}
(supervisor (Bitdiddle Ben) (Warbucks Oliver))
(address (Warbucks Oliver) (Swellesley (Top Heap Road)))
(job (Warbucks Oliver) (administration big wheel))
(salary (Warbucks Oliver) 150000)
\end{scheme}

\noindent
Besides the computer division supervised by Ben, the company has an accounting
division, consisting of a chief accountant and his assistant:

\begin{scheme}
(address (Scrooge Eben) (Weston (Shady Lane) 10))
(job (Scrooge Eben) (accounting chief accountant))
(salary (Scrooge Eben) 75000)
(supervisor (Scrooge Eben) (Warbucks Oliver))

(address (Cratchet Robert) (Allston (N Harvard Street) 16))
(job (Cratchet Robert) (accounting scrivener))
(salary (Cratchet Robert) 18000)
(supervisor (Cratchet Robert) (Scrooge Eben))
\end{scheme}

\noindent
There is also a secretary for the big wheel:

\begin{scheme}
(address (Aull DeWitt) (Slumerville (Onion Square) 5))
(job (Aull DeWitt) (administration secretary))
(salary (Aull DeWitt) 25000)
(supervisor (Aull DeWitt) (Warbucks Oliver))
\end{scheme}

\noindent
The data base also contains assertions about which kinds of jobs can be done by
people holding other kinds of jobs.  For instance, a computer wizard can do the
jobs of both a computer programmer and a computer technician:

\begin{scheme}
(can-do-job (computer wizard) (computer programmer))
(can-do-job (computer wizard) (computer technician))
\end{scheme}

\noindent
A computer programmer could fill in for a trainee:

\begin{scheme}
(can-do-job (computer programmer)
            (computer programmer trainee))
\end{scheme}

\noindent
Also, as is well known,

\begin{scheme}
(can-do-job (administration secretary)
            (administration big wheel))
\end{scheme}

\subsubsection*{Simple queries}

The query language allows users to retrieve information from the data base by
posing queries in response to the system's prompt.  For example, to find all
computer programmers one can say

\begin{scheme}
~\textit{;;; Query input:}~
(job ?x (computer programmer))
\end{scheme}

\noindent
The system will respond with the following items:

\begin{scheme}
~\textit{;;; Query results:}~
(job (Hacker Alyssa P) (computer programmer))
(job (Fect Cy D) (computer programmer))
\end{scheme}

\noindent
The input query specifies that we are looking for entries in the data base that
match a certain \newterm{pattern}.  In this example, the pattern specifies
entries consisting of three items, of which the first is the literal symbol
\code{job}, the second can be anything, and the third is the literal list
\code{(computer programmer)}.  The ``anything'' that can be the second item in
the matching list is specified by a \newterm{pattern variable}, \code{?x}.  The
general form of a pattern variable is a symbol, taken to be the name of the
variable, preceded by a question mark.  We will see below why it is useful to
specify names for pattern variables rather than just putting \code{?} into
patterns to represent ``anything.''  The system responds to a simple query by
showing all entries in the data base that match the specified pattern.

A pattern can have more than one variable.  For example, the query

\begin{scheme}
(address ?x ?y)
\end{scheme}

\noindent
will list all the employees' addresses.

A pattern can have no variables, in which case the query simply determines
whether that pattern is an entry in the data base.  If so, there will be one
match; if not, there will be no matches.

The same pattern variable can appear more than once in a query, specifying that
the same ``anything'' must appear in each position.  This is why variables have
names.  For example,

\begin{scheme}
(supervisor ?x ?x)
\end{scheme}

\noindent
finds all people who supervise themselves (though there are no such assertions
in our sample data base).

The query

\begin{scheme}
(job ?x (computer ?type))
\end{scheme}

\noindent
matches all job entries whose third item is a two-element list whose first item
is \code{computer}:

\begin{scheme}
(job (Bitdiddle Ben) (computer wizard))
(job (Hacker Alyssa P) (computer programmer))
(job (Fect Cy D) (computer programmer))
(job (Tweakit Lem E) (computer technician))
\end{scheme}

\noindent
This same pattern does \emph{not} match

\begin{scheme}
(job (Reasoner Louis) (computer programmer trainee))
\end{scheme}

\noindent
because the third item in the entry is a list of three elements, and the
pattern's third item specifies that there should be two elements.  If we wanted
to change the pattern so that the third item could be any list beginning with
\code{computer}, we could specify\footnote{This uses the dotted-tail notation
introduced in \link{Exercise 2.20}.}

\begin{scheme}
(job ?x (computer . ?type))
\end{scheme}

\noindent
For example,

\begin{scheme}
(computer . ?type)
\end{scheme}

\noindent
matches the data

\begin{scheme}
(computer programmer trainee)
\end{scheme}

\noindent
with \code{?type} as the list \code{(programmer trainee)}.  It also
matches the data

\begin{scheme}
(computer programmer)
\end{scheme}

\noindent
with \code{?type} as the list \code{(programmer)}, and matches the data

\begin{scheme}
(computer)
\end{scheme}

\noindent
with \code{?type} as the empty list \code{()}.

We can describe the query language's processing of simple queries as follows:

\begin{itemize}

\item
The system finds all assignments to variables in the query pattern that
\newterm{satisfy} the pattern---that is, all sets of values for the variables
such that if the pattern variables are \newterm{instantiated with} (replaced
by) the values, the result is in the data base.

\item
The system responds to the query by listing all instantiations of the query
pattern with the variable assignments that satisfy it.

\end{itemize}

\noindent
Note that if the pattern has no variables, the query reduces to a determination
of whether that pattern is in the data base.  If so, the empty assignment,
which assigns no values to variables, satisfies that pattern for that data
base.

\begin{quote}
\heading{\phantomsection\label{Exercise 4.55}Exercise 4.55:} Give simple queries that retrieve
the following information from the data base:

\begin{enumerate}

\item
all people supervised by Ben Bitdiddle;

\item
the names and jobs of all people in the accounting division;

\item
the names and addresses of all people who live in Slumerville.

\end{enumerate}
\end{quote}

\subsubsection*{Compound queries}

Simple queries form the primitive operations of the query language.  In order
to form compound operations, the query language provides means of combination.
One thing that makes the query language a logic programming language is that
the means of combination mirror the means of combination used in forming
logical expressions: \code{and}, \code{or}, and \code{not}.  (Here \code{and},
\code{or}, and \code{not} are not the Lisp primitives, but rather operations
built into the query language.)

We can use \code{and} as follows to find the addresses of all the computer
programmers:

\begin{scheme}
(and (job ?person (computer programmer))
     (address ?person ?where))
\end{scheme}

\noindent
The resulting output is

\begin{scheme}
(and (job (Hacker Alyssa P) (computer programmer))
     (address (Hacker Alyssa P) (Cambridge (Mass Ave) 78)))
(and (job (Fect Cy D) (computer programmer))
     (address (Fect Cy D) (Cambridge (Ames Street) 3)))
\end{scheme}

\noindent
In general,

\begin{scheme}
(and ~\( \dark \langle \)~~\( \dark query_1 \)~~\( \dark \rangle \)~ ~\( \dark \langle \)~~\( \dark query_2 \)~~\( \dark \rangle \)~ ~\( \dots \)~ ~\( \dark \langle \)~~\( \dark query_n \)~~\( \dark \rangle \)~)
\end{scheme}

\noindent
is satisfied by all sets of values for the pattern variables that
simultaneously satisfy \( \langle \)\( query_1 \)\( \rangle \) \( \dots \) \( \langle \)\( query_n \)\( \rangle \).

As for simple queries, the system processes a compound query by finding all
assignments to the pattern variables that satisfy the query, then displaying
instantiations of the query with those values.

Another means of constructing compound queries is through \code{or}.  For
example,

\begin{scheme}
(or (supervisor ?x (Bitdiddle Ben))
    (supervisor ?x (Hacker Alyssa P)))
\end{scheme}

\noindent
will find all employees supervised by Ben Bitdiddle or Alyssa P.  Hacker:

\begin{scheme}
(or (supervisor (Hacker Alyssa P) (Bitdiddle Ben))
    (supervisor (Hacker Alyssa P) (Hacker Alyssa P)))
(or (supervisor (Fect Cy D) (Bitdiddle Ben))
    (supervisor (Fect Cy D) (Hacker Alyssa P)))
(or (supervisor (Tweakit Lem E) (Bitdiddle Ben))
    (supervisor (Tweakit Lem E) (Hacker Alyssa P)))
(or (supervisor (Reasoner Louis) (Bitdiddle Ben))
    (supervisor (Reasoner Louis) (Hacker Alyssa P)))
\end{scheme}

\noindent
In general,

\begin{scheme}
(or ~\( \dark \langle \)~~\( \dark query_1 \)~~\( \dark \rangle \)~ ~\( \dark \langle \)~~\( \dark query_2 \)~~\( \dark \rangle \)~ ~\( \dots \)~ ~\( \dark \langle \)~~\( \dark query_n \)~~\( \dark \rangle \)~)
\end{scheme}

\noindent
is satisfied by all sets of values for the pattern variables that satisfy at
least one of \( \langle \)\( query_1 \)\( \rangle \) \( \dots \) \( \langle \)\( query_n \)\( \rangle \).

Compound queries can also be formed with \code{not}. For example,

\begin{scheme}
(and (supervisor ?x (Bitdiddle Ben))
     (not (job ?x (computer programmer))))
\end{scheme}

\noindent
finds all people supervised by Ben Bitdiddle who are not computer programmers.
In general,

\begin{scheme}
(not ~\( \dark \langle \)~~\( \dark query_1 \)~~\( \dark \rangle \)~)
\end{scheme}

\noindent
is satisfied by all assignments to the pattern variables that do not satisfy
\( \langle \)\( query_1 \)\( \rangle \).\footnote{Actually, this description of \code{not} is valid
only for simple cases.  The real behavior of \code{not} is more complex.  We
will examine \code{not}'s peculiarities in sections \link{Section 4.4.2} and
\link{Section 4.4.3}.}

The final combining form is called \code{lisp\-/value}.  When \code{lisp\-/value}
is the first element of a pattern, it specifies that the next element is a Lisp
predicate to be applied to the rest of the (instantiated) elements as
arguments.  In general,

\begin{scheme}
(lisp-value ~\( \dark \langle \)~~\var{\dark predicate}~~\( \dark \rangle \)~ ~\( \dark \langle \)~~\( \dark arg_1 \)~~\( \dark \rangle \)~ ~\( \dots \)~ ~\( \dark \langle \)~~\( \dark arg_n \)~~\( \dark \rangle \)~)
\end{scheme}

\noindent
will be satisfied by assignments to the pattern variables for which the
\( \langle \)\var{predicate}\( \rangle \) applied to the instantiated \( \langle \)\( arg_1 \)\( \rangle \) \( \dots \)
\( \langle \)\( arg_n \)\( \rangle \) is true.  For example, to find all people whose salary is
greater than \$30,000 we could write\footnote{\code{Lisp\-/value} should be used
only to perform an operation not provided in the query language.  In
particular, it should not be used to test equality (since that is what the
matching in the query language is designed to do) or inequality (since that can
be done with the \code{same} rule shown below).}

\begin{scheme}
(and (salary ?person ?amount) (lisp-value > ?amount 30000))
\end{scheme}

\begin{quote}
\heading{\phantomsection\label{Exercise 4.56}Exercise 4.56:} Formulate compound queries that
retrieve the following information:

\begin{enumerate}[a.]

\item
the names of all people who are supervised by Ben Bitdiddle, together with
their addresses;

\item
all people whose salary is less than Ben Bitdiddle's, together with their
salary and Ben Bitdiddle's salary;

\item
all people who are supervised by someone who is not in the computer division,
together with the supervisor's name and job.

\end{enumerate}
\end{quote}

\subsubsection*{Rules}

In addition to primitive queries and compound queries, the query language
provides means for abstracting queries.  These are given by \newterm{rules}.
The rule

\begin{scheme}
(rule (lives-near ?person-1 ?person-2)
      (and (address ?person-1 (?town . ?rest-1))
           (address ?person-2 (?town . ?rest-2))
           (not (same ?person-1 ?person-2))))
\end{scheme}

\noindent
specifies that two people live near each other if they live in the same town.
The final \code{not} clause prevents the rule from saying that all people live
near themselves.  The \code{same} relation is defined by a very simple
rule:\footnote{Notice that we do not need \code{same} in order to make two
things be the same: We just use the same pattern variable for each---in effect,
we have one thing instead of two things in the first place.  For example, see
\code{?town} in the \code{lives\-/near} rule and \code{?middle\-/manager} in the
\code{wheel} rule below.  \code{Same} is useful when we want to force two
things to be different, such as \code{?person\-/1} and \code{?person\-/2} in the
\code{lives\-/near} rule.  Although using the same pattern variable in two parts
of a query forces the same value to appear in both places, using different
pattern variables does not force different values to appear.  (The values
assigned to different pattern variables may be the same or different.)}

\begin{scheme}
(rule (same ?x ?x))
\end{scheme}

\noindent
The following rule declares that a person is a ``wheel'' in an organization if
he supervises someone who is in turn a supervisor:

\begin{scheme}
(rule (wheel ?person)
      (and (supervisor ?middle-manager ?person)
           (supervisor ?x ?middle-manager)))
\end{scheme}

\noindent
The general form of a rule is

\begin{scheme}
(rule ~\( \dark \langle \)~~\var{\dark conclusion}~~\( \dark \rangle \)~ ~\( \dark \langle \)~~\var{\dark body}~~\( \dark \rangle \)~)
\end{scheme}

\noindent
where \( \langle \)\var{conclusion}\( \rangle \) is a pattern and \( \langle \)\var{body}\( \rangle \) is any
query.\footnote{We will also allow rules without bodies, as in \code{same}, and
we will interpret such a rule to mean that the rule conclusion is satisfied by
any values of the variables.} We can think of a rule as representing a large
(even infinite) set of assertions, namely all instantiations of the rule
conclusion with variable assignments that satisfy the rule body.  When we
described simple queries (patterns), we said that an assignment to variables
satisfies a pattern if the instantiated pattern is in the data base.  But the
pattern needn't be explicitly in the data base as an assertion.  It can be an
implicit assertion implied by a rule.  For example, the query

\begin{scheme}
(lives-near ?x (Bitdiddle Ben))
\end{scheme}

\noindent
results in

\begin{scheme}
(lives-near (Reasoner Louis) (Bitdiddle Ben))
(lives-near (Aull DeWitt) (Bitdiddle Ben))
\end{scheme}

\noindent
To find all computer programmers who live near Ben Bitdiddle, we can ask

\begin{scheme}
(and (job ?x (computer programmer))
     (lives-near ?x (Bitdiddle Ben)))
\end{scheme}

\noindent
As in the case of compound procedures, rules can be used as parts of other
rules (as we saw with the \code{lives\-/near} rule above) or even be defined
recursively.  For instance, the rule

\begin{scheme}
(rule (outranked-by ?staff-person ?boss)
      (or (supervisor ?staff-person ?boss)
          (and (supervisor ?staff-person ?middle-manager)
               (outranked-by ?middle-manager ?boss))))
\end{scheme}

\noindent
says that a staff person is outranked by a boss in the organization if the boss
is the person's supervisor or (recursively) if the person's supervisor is
outranked by the boss.

\begin{quote}
\heading{\phantomsection\label{Exercise 4.57}Exercise 4.57:} Define a rule that says that
person 1 can replace person 2 if either person 1 does the same job as person 2
or someone who does person 1's job can also do person 2's job, and if person 1
and person 2 are not the same person. Using your rule, give queries that find
the following:

\begin{enumerate}[a.]

\item
all people who can replace Cy D. Fect;

\item
all people who can replace someone who is being paid more than they are,
together with the two salaries.

\end{enumerate}
\end{quote}

\begin{quote}
\heading{\phantomsection\label{Exercise 4.58}Exercise 4.58:} Define a rule that says that a
person is a ``big shot'' in a division if the person works in the division but
does not have a supervisor who works in the division.
\end{quote}

\begin{quote}
\heading{\phantomsection\label{Exercise 4.59}Exercise 4.59:} Ben Bitdiddle has missed one
meeting too many.  Fearing that his habit of forgetting meetings could cost him
his job, Ben decides to do something about it.  He adds all the weekly meetings
of the firm to the Microshaft data base by asserting the following:

\begin{scheme}
(meeting accounting (Monday 9am))
(meeting administration (Monday 10am))
(meeting computer (Wednesday 3pm))
(meeting administration (Friday 1pm))
\end{scheme}

Each of the above assertions is for a meeting of an entire division.  Ben also
adds an entry for the company-wide meeting that spans all the divisions.  All
of the company's employees attend this meeting.

\begin{scheme}
(meeting whole-company (Wednesday 4pm))
\end{scheme}

\begin{enumerate}[a.]

\item
On Friday morning, Ben wants to query the data base for all the meetings that
occur that day.  What query should he use?

\item
Alyssa P. Hacker is unimpressed.  She thinks it would be much more useful to be
able to ask for her meetings by specifying her name.  So she designs a rule
that says that a person's meetings include all \code{whole\-/company} meetings
plus all meetings of that person's division.  Fill in the body of Alyssa's
rule.

\begin{scheme}
(rule (meeting-time ?person ?day-and-time)
      ~\( \dark \langle \)~~\var{\dark rule-body}~~\( \dark \rangle \)~)
\end{scheme}

\item
Alyssa arrives at work on Wednesday morning and wonders what meetings she has
to attend that day.  Having defined the above rule, what query should she make
to find this out?

\end{enumerate}
\end{quote}

\begin{quote}
\heading{\phantomsection\label{Exercise 4.60}Exercise 4.60:} By giving the query

\begin{scheme}
(lives-near ?person (Hacker Alyssa P))
\end{scheme}

Alyssa P. Hacker is able to find people who live near her, with whom she can
ride to work.  On the other hand, when she tries to find all pairs of people
who live near each other by querying

\begin{scheme}
(lives-near ?person-1 ?person-2)
\end{scheme}

\noindent
she notices that each pair of people who live near each other is listed twice;
for example,

\begin{scheme}
(lives-near (Hacker Alyssa P) (Fect Cy D))
(lives-near (Fect Cy D) (Hacker Alyssa P))
\end{scheme}

Why does this happen?  Is there a way to find a list of people who live near
each other, in which each pair appears only once?  Explain.
\end{quote}

\subsubsection*{Logic as programs}

We can regard a rule as a kind of logical implication: \emph{If} an assignment
of values to pattern variables satisfies the body, \emph{then} it satisfies the
conclusion.  Consequently, we can regard the query language as having the
ability to perform \newterm{logical deductions} based upon the rules.  As an
example, consider the \code{append} operation described at the beginning of
\link{Section 4.4}.  As we said, \code{append} can be characterized by the
following two rules:

\begin{itemize}

\item
For any list \code{y}, the empty list and \code{y} \code{append} to form
\code{y}.

\item
For any \code{u}, \code{v}, \code{y}, and \code{z}, \code{(cons u v)} and
\code{y} \code{append} to form \code{(cons u z)} if \code{v} and \code{y}
\code{append} to form \code{z}.

\end{itemize}

\noindent
To express this in our query language, we define two rules for a relation

\begin{scheme}
(append-to-form x y z)
\end{scheme}

\noindent
which we can interpret to mean ``\code{x} and \code{y} \code{append} to form
\code{z}'':

\begin{scheme}
(rule (append-to-form () ?y ?y))
(rule (append-to-form (?u . ?v) ?y (?u . ?z))
      (append-to-form ?v ?y ?z))
\end{scheme}

\noindent
The first rule has no body, which means that the conclusion holds for any value
of \code{?y}.  Note how the second rule makes use of dotted-tail notation to
name the \code{car} and \code{cdr} of a list.

Given these two rules, we can formulate queries that compute the \code{append}
of two lists:

\begin{scheme}
~\textit{;;; Query input:}~
(append-to-form (a b) (c d) ?z)
~\textit{;;; Query results:}~
(append-to-form (a b) (c d) (a b c d))
\end{scheme}

\noindent
What is more striking, we can use the same rules to ask the question ``Which
list, when \code{append}ed to \code{(a b)}, yields \code{(a b c d)}?''  This is
done as follows:

\begin{scheme}
~\textit{;;; Query input:}~
(append-to-form (a b) ?y (a b c d))
~\textit{;;; Query results:}~
(append-to-form (a b) (c d) (a b c d))
\end{scheme}

\noindent
We can also ask for all pairs of lists that \code{append} to form \code{(a b c
d)}:

\begin{scheme}
~\textit{;;; Query input:}~
(append-to-form ?x ?y (a b c d))
~\textit{;;; Query results:}~
(append-to-form () (a b c d) (a b c d))
(append-to-form (a) (b c d) (a b c d))
(append-to-form (a b) (c d) (a b c d))
(append-to-form (a b c) (d) (a b c d))
(append-to-form (a b c d) () (a b c d))
\end{scheme}

\noindent
The query system may seem to exhibit quite a bit of intelligence in using the
rules to deduce the answers to the queries above.  Actually, as we will see in
the next section, the system is following a well-determined algorithm in
unraveling the rules.  Unfortunately, although the system works impressively in
the \code{append} case, the general methods may break down in more complex
cases, as we will see in \link{Section 4.4.3}.

\begin{quote}
\heading{\phantomsection\label{Exercise 4.61}Exercise 4.61:} The following rules implement a
\code{next\-/to} relation that finds adjacent elements of a list:

\begin{scheme}
(rule (?x next-to ?y in (?x ?y . ?u)))
(rule (?x next-to ?y in (?v . ?z))
      (?x next-to ?y in ?z))
\end{scheme}

What will the response be to the following queries?

\begin{scheme}
(?x next-to ?y in (1 (2 3) 4))
(?x next-to  1 in (2 1 3 1))
\end{scheme}
\end{quote}

\begin{quote}
\heading{\phantomsection\label{Exercise 4.62}Exercise 4.62:} Define rules to implement the
\code{last\-/pair} operation of \link{Exercise 2.17}, which returns a list
containing the last element of a nonempty list.  Check your rules on queries
such as \code{(last\-/pair (3) ?x)}, \code{(last\-/pair (1 2 3) ?x)} and
\code{(last\-/pair (2 ?x) (3))}.  Do your rules work correctly on queries such as
\code{(last\-/pair ?x (3))} ?
\end{quote}

\begin{quote}
\heading{\phantomsection\label{Exercise 4.63}Exercise 4.63:} The following data base (see
Genesis 4) traces the genealogy of the descendants of Ada back to Adam, by way
of Cain:

\begin{scheme}
(son Adam Cain)
(son Cain Enoch)
(son Enoch Irad)
(son Irad Mehujael)
(son Mehujael Methushael)
(son Methushael Lamech)
(wife Lamech Ada)
(son Ada Jabal)
(son Ada Jubal)
\end{scheme}

Formulate rules such as ``If \( S \) is the son of \( f \), and \( f \) is the son of
\( G \), then \( S \) is the grandson of \( G \)'' and ``If \( W \) is the wife of
\( M \), and \( S \) is the son of \( W \), then \( S \) is the son of \( M \)'' (which
was supposedly more true in biblical times than today) that will enable the
query system to find the grandson of Cain; the sons of Lamech; the grandsons of
Methushael.  (See \link{Exercise 4.69} for some rules to deduce more complicated
relationships.)
\end{quote}

\subsection{How the Query System Works}
\label{Section 4.4.2}

In \link{Section 4.4.4} we will present an implementation of the query
interpreter as a collection of procedures.  In this section we give an overview
that explains the general structure of the system independent of low-level
implementation details.  After describing the implementation of the
interpreter, we will be in a position to understand some of its limitations and
some of the subtle ways in which the query language's logical operations differ
from the operations of mathematical logic.

It should be apparent that the query evaluator must perform some kind of search
in order to match queries against facts and rules in the data base.  One way to
do this would be to implement the query system as a nondeterministic program,
using the \code{amb} evaluator of \link{Section 4.3} (see \link{Exercise 4.78}).
Another possibility is to manage the search with the aid of streams.  Our
implementation follows this second approach.

The query system is organized around two central operations called
\newterm{pattern matching} and \newterm{unification}.  We first describe
pattern matching and explain how this operation, together with the organization
of information in terms of streams of frames, enables us to implement both
simple and compound queries.  We next discuss unification, a generalization of
pattern matching needed to implement rules.  Finally, we show how the entire
query interpreter fits together through a procedure that classifies expressions
in a manner analogous to the way \code{eval} classifies expressions for the
interpreter described in \link{Section 4.1}.

\subsubsection*{Pattern matching}

A \newterm{pattern matcher} is a program that tests whether some datum fits a
specified pattern.  For example, the data list \code{((a b) c (a b))} matches
the pattern \code{(?x c ?x)} with the pattern variable \code{?x} bound to
\code{(a b)}.  The same data list matches the pattern \code{(?x ?y ?z)} with
\code{?x} and \code{?z} both bound to \code{(a b)} and \code{?y} bound to
\code{c}.  It also matches the pattern \code{((?x ?y) c (?x ?y))} with
\code{?x} bound to \code{a} and \code{?y} bound to \code{b}.  However, it does
not match the pattern \code{(?x a ?y)}, since that pattern specifies a list
whose second element is the symbol \code{a}.

The pattern matcher used by the query system takes as inputs a pattern, a
datum, and a \newterm{frame} that specifies bindings for various pattern
variables.  It checks whether the datum matches the pattern in a way that is
consistent with the bindings already in the frame.  If so, it returns the given
frame augmented by any bindings that may have been determined by the match.
Otherwise, it indicates that the match has failed.

For example, using the pattern \code{(?x ?y ?x)} to match \code{(a b a)} given
an empty frame will return a frame specifying that \code{?x} is bound to
\code{a} and \code{?y} is bound to \code{b}.  Trying the match with the same
pattern, the same datum, and a frame specifying that \code{?y} is bound to
\code{a} will fail.  Trying the match with the same pattern, the same datum,
and a frame in which \code{?y} is bound to \code{b} and \code{?x} is unbound
will return the given frame augmented by a binding of \code{?x} to \code{a}.

The pattern matcher is all the mechanism that is needed to process simple
queries that don't involve rules.  For instance, to process the query

\begin{scheme}
(job ?x (computer programmer))
\end{scheme}

\noindent
we scan through all assertions in the data base and select those that match the
pattern with respect to an initially empty frame.  For each match we find, we
use the frame returned by the match to instantiate the pattern with a value for
\code{?x}.

\subsubsection*{Streams of frames}

The testing of patterns against frames is organized through the use of streams.
Given a single frame, the matching process runs through the data-base entries
one by one.  For each data-base entry, the matcher generates either a special
symbol indicating that the match has failed or an extension to the frame.  The
results for all the data-base entries are collected into a stream, which is
passed through a filter to weed out the failures.  The result is a stream of
all the frames that extend the given frame via a match to some assertion in the
data base.\footnote{Because matching is generally very expensive, we would like
to avoid applying the full matcher to every element of the data base.  This is
usually arranged by breaking up the process into a fast, coarse match and the
final match.  The coarse match filters the data base to produce a small set of
candidates for the final match.  With care, we can arrange our data base so
that some of the work of coarse matching can be done when the data base is
constructed rather then when we want to select the candidates.  This is called
\newterm{indexing} the data base.  There is a vast technology built around
data-base-indexing schemes.  Our implementation, described in 
\link{Section 4.4.4}, contains a simple-minded form of such an optimization.}

In our system, a query takes an input stream of frames and performs the above
matching operation for every frame in the stream, as indicated in \link{Figure 4.4}.  
That is, for each frame in the input stream, the query generates a new
stream consisting of all extensions to that frame by matches to assertions in
the data base.  All these streams are then combined to form one huge stream,
which contains all possible extensions of every frame in the input stream.
This stream is the output of the query.

To answer a simple query, we use the query with an input stream consisting of a
single empty frame.  The resulting output stream contains all extensions to the
empty frame (that is, all answers to our query).  This stream of frames is then
used to generate a stream of copies of the original query pattern with the
variables instantiated by the values in each frame, and this is the stream that
is finally printed.

\begin{figure}[tb]
\phantomsection\label{Figure 4.4}
\centering
\begin{comment}
\heading{Figure 4.4:} A query processes a stream of frames.

\begin{example}
                                  output stream
  input stream   +-------------+  of frames,
  of frames      |    query    |  filtered and extended
---------------->|             +------------------------->
                 | (job ?x ?y) |
                 +-------------+
                        ^
                        |
               stream of assertions
                  from data base
\end{example}
\end{comment}
\includegraphics[width=102mm]{fig/chap4/Fig4.4.pdf}
\par\bigskip
\noindent
\heading{Figure 4.4:} A query processes a stream of frames.
\end{figure}

\subsubsection*{Compound queries}

The real elegance of the stream-of-frames implementation is evident when we
deal with compound queries.  The processing of compound queries makes use of
the ability of our matcher to demand that a match be consistent with a
specified frame.  For example, to handle the \code{and} of two queries, such as

\begin{scheme}
(and (can-do-job ?x (computer programmer trainee))
     (job ?person ?x))
\end{scheme}

\noindent
(informally, ``Find all people who can do the job of a computer programmer
trainee''), we first find all entries that match the pattern

\begin{scheme}
(can-do-job ?x (computer programmer trainee))
\end{scheme}

\begin{figure}[tb]
\phantomsection\label{Figure 4.5}
\centering
\begin{comment}
\heading{Figure 4.5:} The \code{and} combination of two queries is produced 
by operating on the stream of frames in series.

\begin{example}
                +----------------------+
                |       (and A B)      |
  input stream  |                      |  output stream
  of frames     |   +---+       +---+  |  of frames
------------------->| A +------>| B +-------------------->
                |   +---+       +---+  |
                |     ^           ^    |
                |     |           |    |
                |     +-----*-----+    |
                +-----------|----------+
                            |
                        data base
\end{example}
\end{comment}
\includegraphics[width=93mm]{fig/chap4/Fig4.5.pdf}
\begin{quote}
\heading{Figure 4.5:} The \code{and} combination of two queries is produced by operating on the stream of frames in series.
\end{quote}
\end{figure}

\noindent
This produces a stream of frames, each of which contains a binding for
\code{?x}.  Then for each frame in the stream we find all entries that match

\begin{scheme}
(job ?person ?x)
\end{scheme}

\noindent
in a way that is consistent with the given binding for \code{?x}.  Each such
match will produce a frame containing bindings for \code{?x} and
\code{?person}.  The \code{and} of two queries can be viewed as a series
combination of the two component queries, as shown in \link{Figure 4.5}.  The
frames that pass through the first query filter are filtered and further
extended by the second query.

\begin{figure}[tb]
\phantomsection\label{Figure 4.6}
\centering
\begin{comment}
\heading{Figure 4.6:} The \code{or} combination of two queries is produced 
by operating on the stream of frames in parallel and merging the results.

\begin{example}
           +---------------------------+
           |          (or A B)         |
           |    +---+                  |
input      | +->| A |------------+     |  output
stream of  | |  +---+            V     |  stream of
frames     | |    ^          +-------+ |  frames
-------------*    |          | merge +--------------->
           | |    |          +-------+ |
           | |    |              ^     |
           | |    |   +---+      |     |
           | +------->| B +------+     |
           |      |   +---+            |
           |      |     ^              |
           |      |     |              |
           |      +--*--+              |
           +---------|-----------------+
                     |
                 data base
\end{example}
\end{comment}
\includegraphics[width=107mm]{fig/chap4/Fig4.6.pdf}
\begin{quote}
\heading{Figure 4.6:} The \code{or} combination of two queries is produced by operating on the stream of frames in parallel and merging the results.
\end{quote}
\end{figure}

\link{Figure 4.6} shows the analogous method for computing the 
\code{or} of two
queries as a parallel combination of the two component queries.  The input
stream of frames is extended separately by each query.  The two resulting
streams are then merged to produce the final output stream.

Even from this high-level description, it is apparent that the processing of
compound queries can be slow.  For example, since a query may produce more than
one output frame for each input frame, and each query in an \code{and} gets its
input frames from the previous query, an \code{and} query could, in the worst
case, have to perform a number of matches that is exponential in the number of
queries (see \link{Exercise 4.76}).\footnote{But this kind of exponential
explosion is not common in \code{and} queries because the added conditions tend
to reduce rather than expand the number of frames produced.} Though systems for
handling only simple queries are quite practical, dealing with complex queries
is extremely difficult.\footnote{There is a large literature on
data-base-management systems that is concerned with how to handle complex
queries efficiently.}

From the stream-of-frames viewpoint, the \code{not} of some query acts as a
filter that removes all frames for which the query can be satisfied.  For
instance, given the pattern

\begin{scheme}
(not (job ?x (computer programmer)))
\end{scheme}

\noindent
we attempt, for each frame in the input stream, to produce extension frames
that satisfy \code{(job ?x (computer programmer))}.  We remove from the input
stream all frames for which such extensions exist.  The result is a stream
consisting of only those frames in which the binding for \code{?x} does not
satisfy \code{(job ?x (computer programmer))}.  For example, in processing the
query

\begin{scheme}
(and (supervisor ?x ?y)
     (not (job ?x (computer programmer))))
\end{scheme}

\noindent
the first clause will generate frames with bindings for \code{?x} and
\code{?y}.  The \code{not} clause will then filter these by removing all frames
in which the binding for \code{?x} satisfies the restriction that \code{?x} is
a computer programmer.\footnote{There is a subtle difference between this
filter implementation of \code{not} and the usual meaning of \code{not} in
mathematical logic.  See \link{Section 4.4.3}.}

The \code{lisp\-/value} special form is implemented as a similar filter on frame
streams.  We use each frame in the stream to instantiate any variables in the
pattern, then apply the Lisp predicate.  We remove from the input stream all
frames for which the predicate fails.

\subsubsection*{Unification}

In order to handle rules in the query language, we must be able to find the
rules whose conclusions match a given query pattern.  Rule conclusions are like
assertions except that they can contain variables, so we will need a
generalization of pattern matching---called \newterm{unification}---in which
both the ``pattern'' and the ``datum'' may contain variables.

A unifier takes two patterns, each containing constants and variables, and
determines whether it is possible to assign values to the variables that will
make the two patterns equal.  If so, it returns a frame containing these
bindings.  For example, unifying \code{(?x a ?y)} and \code{(?y ?z a)} will
specify a frame in which \code{?x}, \code{?y}, and \code{?z} must all be bound
to \code{a}.  On the other hand, unifying \code{(?x ?y a)} and \code{(?x b ?y)}
will fail, because there is no value for \code{?y} that can make the two
patterns equal.  (For the second elements of the patterns to be equal,
\code{?y} would have to be \code{b}; however, for the third elements to be
equal, \code{?y} would have to be \code{a}.)  The unifier used in the query
system, like the pattern matcher, takes a frame as input and performs
unifications that are consistent with this frame.

The unification algorithm is the most technically difficult part of the query
system.  With complex patterns, performing unification may seem to require
deduction.  To unify \code{(?x ?x)} and \code{((a ?y c) (a b ?z))}, for
example, the algorithm must infer that \code{?x} should be \code{(a b c)},
\code{?y} should be \code{b}, and \code{?z} should be \code{c}.  We may think
of this process as solving a set of equations among the pattern components.  In
general, these are simultaneous equations, which may require substantial
manipulation to solve.\footnote{In one-sided pattern matching, all the
equations that contain pattern variables are explicit and already solved for
the unknown (the pattern variable).} For example, unifying \code{(?x ?x)} and
\code{((a ?y c) (a b ?z))} may be thought of as specifying the simultaneous
equations

\begin{scheme}
?x  =  (a ?y c)
?x  =  (a b ?z)
\end{scheme}

\noindent
These equations imply that

\begin{scheme}
(a ?y c)  =  (a b ?z)
\end{scheme}

\noindent
which in turn implies that

\begin{scheme}
 a  =  a, 
?y  =  b, 
 c  =  ?z,
\end{scheme}

\noindent
and hence that

\begin{scheme}
?x  =  (a b c)
\end{scheme}

\noindent
In a successful pattern match, all pattern variables become bound, and the
values to which they are bound contain only constants.  This is also true of
all the examples of unification we have seen so far.  In general, however, a
successful unification may not completely determine the variable values; some
variables may remain unbound and others may be bound to values that contain
variables.

Consider the unification of \code{(?x a)} and \code{((b ?y) ?z)}.  We can
deduce that \code{?x = (b ?y)} and \code{a = ?z}, but we cannot further solve
for \code{?x} or \code{?y}.  The unification doesn't fail, since it is
certainly possible to make the two patterns equal by assigning values to
\code{?x} and \code{?y}.  Since this match in no way restricts the values
\code{?y} can take on, no binding for \code{?y} is put into the result frame.
The match does, however, restrict the value of \code{?x}.  Whatever value
\code{?y} has, \code{?x} must be \code{(b ?y)}.  A binding of \code{?x} to the
pattern \code{(b ?y)} is thus put into the frame.  If a value for \code{?y} is
later determined and added to the frame (by a pattern match or unification that
is required to be consistent with this frame), the previously bound \code{?x}
will refer to this value.\footnote{Another way to think of unification is that
it generates the most general pattern that is a specialization of the two input
patterns.  That is, the unification of \code{(?x a)} and \code{((b ?y) ?z)} is
\code{((b ?y) a)}, and the unification of \code{(?x a ?y)} and \code{(?y ?z
a)}, discussed above, is \code{(a a a)}.  For our implementation, it is more
convenient to think of the result of unification as a frame rather than a
pattern.}

\subsubsection*{Applying rules}

Unification is the key to the component of the query system that makes
inferences from rules. To see how this is accomplished, consider processing a
query that involves applying a rule, such as

\begin{scheme}
(lives-near ?x (Hacker Alyssa P))
\end{scheme}

\noindent
To process this query, we first use the ordinary pattern-match procedure
described above to see if there are any assertions in the data base that match
this pattern.  (There will not be any in this case, since our data base
includes no direct assertions about who lives near whom.)  The next step is to
attempt to unify the query pattern with the conclusion of each rule.  We find
that the pattern unifies with the conclusion of the rule

\begin{scheme}
(rule (lives-near ?person-1 ?person-2)
      (and (address ?person-1 (?town . ?rest-1))
           (address ?person-2 (?town . ?rest-2))
           (not (same ?person-1 ?person-2))))
\end{scheme}

\noindent
resulting in a frame specifying that \code{?person\-/2} is bound to \code{(Hacker
Alyssa P)} and that \code{?x} should be bound to (have the same value as)
\code{?person\-/1}.  Now, relative to this frame, we evaluate the compound query
given by the body of the rule.  Successful matches will extend this frame by
providing a binding for \code{?person\-/1}, and consequently a value for
\code{?x}, which we can use to instantiate the original query pattern.

In general, the query evaluator uses the following method to apply a rule when
trying to establish a query pattern in a frame that specifies bindings for some
of the pattern variables:

\begin{itemize}

\item
Unify the query with the conclusion of the rule to form, if successful, an
extension of the original frame.

\item
Relative to the extended frame, evaluate the query formed by the body of the
rule.

\end{itemize}

\noindent
Notice how similar this is to the method for applying a procedure in the
\code{eval}/\code{apply} evaluator for Lisp:

\begin{itemize}

\item
Bind the procedure's parameters to its arguments to form a frame that extends
the original procedure environment.

\item
Relative to the extended environment, evaluate the expression formed by the
body of the procedure.

\end{itemize}

\noindent
The similarity between the two evaluators should come as no surprise.  Just as
procedure definitions are the means of abstraction in Lisp, rule definitions
are the means of abstraction in the query language.  In each case, we unwind
the abstraction by creating appropriate bindings and evaluating the rule or
procedure body relative to these.

\subsubsection*{Simple queries}

We saw earlier in this section how to evaluate simple queries in the absence of
rules.  Now that we have seen how to apply rules, we can describe how to
evaluate simple queries by using both rules and assertions.

Given the query pattern and a stream of frames, we produce, for each frame in
the input stream, two streams:

\begin{itemize}

\item
a stream of extended frames obtained by matching the pattern against all
assertions in the data base (using the pattern matcher), and

\item
a stream of extended frames obtained by applying all possible rules (using the
unifier).\footnote{Since unification is a generalization of matching, we could
simplify the system by using the unifier to produce both streams.  Treating the
easy case with the simple matcher, however, illustrates how matching (as
opposed to full-blown unification) can be useful in its own right.}

\end{itemize}

\noindent
Appending these two streams produces a stream that consists of all the ways
that the given pattern can be satisfied consistent with the original frame.
These streams (one for each frame in the input stream) are now all combined to
form one large stream, which therefore consists of all the ways that any of the
frames in the original input stream can be extended to produce a match with the
given pattern.

\subsubsection*{The query evaluator and the driver loop}

Despite the complexity of the underlying matching operations, the system is
organized much like an evaluator for any language.  The procedure that
coordinates the matching operations is called \code{qeval}, and it plays a role
analogous to that of the \code{eval} procedure for Lisp.  \code{Qeval} takes as
inputs a query and a stream of frames.  Its output is a stream of frames,
corresponding to successful matches to the query pattern, that extend some
frame in the input stream, as indicated in \link{Figure 4.4}.  Like \code{eval},
\code{qeval} classifies the different types of expressions (queries) and
dispatches to an appropriate procedure for each.  There is a procedure for each
special form (\code{and}, \code{or}, \code{not}, and \code{lisp\-/value}) and one
for simple queries.

The driver loop, which is analogous to the \code{driver\-/loop} procedure for the
other evaluators in this chapter, reads queries from the terminal.  For each
query, it calls \code{qeval} with the query and a stream that consists of a
single empty frame.  This will produce the stream of all possible matches (all
possible extensions to the empty frame).  For each frame in the resulting
stream, it instantiates the original query using the values of the variables
found in the frame.  This stream of instantiated queries is then
printed.\footnote{The reason we use streams (rather than lists) of frames is
that the recursive application of rules can generate infinite numbers of values
that satisfy a query.  The delayed evaluation embodied in streams is crucial
here: The system will print responses one by one as they are generated,
regardless of whether there are a finite or infinite number of responses.}

The driver also checks for the special command \code{assert!}, which signals
that the input is not a query but rather an assertion or rule to be added to
the data base.  For instance,

\begin{scheme}
(assert! (job (Bitdiddle Ben)
              (computer wizard)))
(assert! (rule (wheel ?person)
               (and (supervisor ?middle-manager ?person)
                    (supervisor ?x ?middle-manager))))
\end{scheme}


\subsection{Is Logic Programming Mathematical Logic?}
\label{Section 4.4.3}

The means of combination used in the query language may at first seem identical
to the operations \code{and}, \code{or}, and \code{not} of mathematical logic,
and the application of query-language rules is in fact accomplished through a
legitimate method of inference.\footnote{That a particular method of inference
is legitimate is not a trivial assertion.  One must prove that if one starts
with true premises, only true conclusions can be derived.  The method of
inference represented by rule applications is \newterm{modus ponens}, the
familiar method of inference that says that if \( A \) is true and \emph{A
implies B} is true, then we may conclude that \( B \) is true.} This
identification of the query language with mathematical logic is not really
valid, though, because the query language provides a \newterm{control
structure} that interprets the logical statements procedurally.  We can often
take advantage of this control structure.  For example, to find all of the
supervisors of programmers we could formulate a query in either of two
logically equivalent forms:

\begin{scheme}
(and (job ?x (computer programmer)) (supervisor ?x ?y))
\end{scheme}

\noindent
or

\begin{scheme}
(and (supervisor ?x ?y) (job ?x (computer programmer)))
\end{scheme}

\noindent
If a company has many more supervisors than programmers (the usual case), it is
better to use the first form rather than the second because the data base must
be scanned for each intermediate result (frame) produced by the first clause of
the \code{and}.

The aim of logic programming is to provide the programmer with techniques for
decomposing a computational problem into two separate problems: ``what'' is to
be computed, and ``how'' this should be computed.  This is accomplished by
selecting a subset of the statements of mathematical logic that is powerful
enough to be able to describe anything one might want to compute, yet weak
enough to have a controllable procedural interpretation.  The intention here is
that, on the one hand, a program specified in a logic programming language
should be an effective program that can be carried out by a computer.  Control
(``how'' to compute) is effected by using the order of evaluation of the
language.  We should be able to arrange the order of clauses and the order of
subgoals within each clause so that the computation is done in an order deemed
to be effective and efficient.  At the same time, we should be able to view the
result of the computation (``what'' to compute) as a simple consequence of the
laws of logic.

Our query language can be regarded as just such a procedurally interpretable
subset of mathematical logic.  An assertion represents a simple fact (an atomic
proposition).  A rule represents the implication that the rule conclusion holds
for those cases where the rule body holds.  A rule has a natural procedural
interpretation: To establish the conclusion of the rule, establish the body of
the rule.  Rules, therefore, specify computations.  However, because rules can
also be regarded as statements of mathematical logic, we can justify any
``inference'' accomplished by a logic program by asserting that the same result
could be obtained by working entirely within mathematical logic.\footnote{We
must qualify this statement by agreeing that, in speaking of the ``inference''
accomplished by a logic program, we assume that the computation terminates.
Unfortunately, even this qualified statement is false for our implementation of
the query language (and also false for programs in Prolog and most other
current logic programming languages) because of our use of \code{not} and
\code{lisp\-/value}.  As we will describe below, the \code{not} implemented in
the query language is not always consistent with the \code{not} of mathematical
logic, and \code{lisp\-/value} introduces additional complications.  We could
implement a language consistent with mathematical logic by simply removing
\code{not} and \code{lisp\-/value} from the language and agreeing to write
programs using only simple queries, \code{and}, and \code{or}.  However, this
would greatly restrict the expressive power of the language.  One of the major
concerns of research in logic programming is to find ways to achieve more
consistency with mathematical logic without unduly sacrificing expressive
power.}

\subsubsection*{Infinite loops}

A consequence of the procedural interpretation of logic programs is that it is
possible to construct hopelessly inefficient programs for solving certain
problems.  An extreme case of inefficiency occurs when the system falls into
infinite loops in making deductions.  As a simple example, suppose we are
setting up a data base of famous marriages, including

\begin{scheme}
(assert! (married Minnie Mickey))
\end{scheme}

\noindent
If we now ask

\begin{scheme}
(married Mickey ?who)
\end{scheme}

\noindent
we will get no response, because the system doesn't know that if \( A \) is
married to \( B \), then \( B \) is married to \( A \).  So we assert the rule

\begin{scheme}
(assert! (rule (married ?x ?y) (married ?y ?x)))
\end{scheme}

\noindent
and again query

\begin{scheme}
(married Mickey ?who)
\end{scheme}

\noindent
Unfortunately, this will drive the system into an infinite loop, as follows:

\begin{itemize}

\item
The system finds that the \code{married} rule is applicable; that is, the rule
conclusion \code{(married ?x ?y)} successfully unifies with the query pattern
\code{(married Mickey ?who)} to produce a frame in which \code{?x} is bound to
\code{Mickey} and \code{?y} is bound to \code{?who}.  So the interpreter
proceeds to evaluate the rule body \code{(married ?y ?x)} in this frame---in
effect, to process the query \code{(married ?who Mickey)}.

\item
One answer appears directly as an assertion in the data base: \code{(married
Minnie Mickey)}.

\item
The \code{married} rule is also applicable, so the interpreter again evaluates
the rule body, which this time is equivalent to \code{(married Mickey ?who)}.

\end{itemize}

\noindent
The system is now in an infinite loop.  Indeed, whether the system will find
the simple answer \code{(married Minnie Mickey)} before it goes into the loop
depends on implementation details concerning the order in which the system
checks the items in the data base.  This is a very simple example of the kinds
of loops that can occur.  Collections of interrelated rules can lead to loops
that are much harder to anticipate, and the appearance of a loop can depend on
the order of clauses in an \code{and} (see \link{Exercise 4.64}) or on low-level
details concerning the order in which the system processes
queries.\footnote{This is not a problem of the logic but one of the procedural
interpretation of the logic provided by our interpreter.  We could write an
interpreter that would not fall into a loop here.  For example, we could
enumerate all the proofs derivable from our assertions and our rules in a
breadth-first rather than a depth-first order.  However, such a system makes it
more difficult to take advantage of the order of deductions in our programs.
One attempt to build sophisticated control into such a program is described in
\link{deKleer et al. 1977}.  Another technique, which does not lead to such serious
control problems, is to put in special knowledge, such as detectors for
particular kinds of loops (\link{Exercise 4.67}).  However, there can be no
general scheme for reliably preventing a system from going down infinite paths
in performing deductions.  Imagine a diabolical rule of the form ``To show
\( P(x) \) is true, show that \( P(f(x)) \) is true,'' for some
suitably chosen function \( f \).}

\subsubsection*{Problems with \code{not}}

Another quirk in the query system concerns \code{not}.  Given the data base of
\link{Section 4.4.1}, consider the following two queries:

\begin{scheme}
(and (supervisor ?x ?y)
     (not (job ?x (computer programmer))))
(and (not (job ?x (computer programmer)))
     (supervisor ?x ?y))
\end{scheme}

\noindent
These two queries do not produce the same result.  The first query begins by
finding all entries in the data base that match \code{(supervisor ?x ?y)}, and
then filters the resulting frames by removing the ones in which the value of
\code{?x} satisfies \code{(job ?x (computer programmer))}.  The second query
begins by filtering the incoming frames to remove those that can satisfy
\code{(job ?x (computer programmer))}.  Since the only incoming frame is empty,
it checks the data base to see if there are any patterns that satisfy
\code{(job ?x (computer programmer))}.  Since there generally are entries of
this form, the \code{not} clause filters out the empty frame and returns an
empty stream of frames.  Consequently, the entire compound query returns an
empty stream.

The trouble is that our implementation of \code{not} really is meant to serve
as a filter on values for the variables.  If a \code{not} clause is processed
with a frame in which some of the variables remain unbound (as does \code{?x}
in the example above), the system will produce unexpected results. Similar
problems occur with the use of \code{lisp\-/value}---the Lisp predicate can't
work if some of its arguments are unbound.  See \link{Exercise 4.77}.

There is also a much more serious way in which the \code{not} of the query
language differs from the \code{not} of mathematical logic.  In logic, we
interpret the statement ``not \( P \)'' to mean that \( P \) is not true.  In the
query system, however, ``not \( P \)'' means that \( P \) is not deducible from the
knowledge in the data base.  For example, given the personnel data base of
\link{Section 4.4.1}, the system would happily deduce all sorts of \code{not}
statements, such as that Ben Bitdiddle is not a baseball fan, that it is not
raining outside, and that 2 + 2 is not 4.\footnote{Consider the query
\code{(not (baseball\-/fan (Bitdiddle Ben)))}.  The system finds that
\code{(baseball\-/fan (Bitdiddle Ben))} is not in the data base, so the empty
frame does not satisfy the pattern and is not filtered out of the initial
stream of frames.  The result of the query is thus the empty frame, which is
used to instantiate the input query to produce \code{(not (baseball\-/fan
(Bitdiddle Ben)))}.} In other words, the \code{not} of logic programming
languages reflects the so-called \newterm{closed world assumption} that all
relevant information has been included in the data base.\footnote{A discussion
and justification of this treatment of \code{not} can be found in the article
by \link{Clark (1978)}.}

\begin{quote}
\heading{\phantomsection\label{Exercise 4.64}Exercise 4.64:} Louis Reasoner mistakenly deletes
the \code{outranked\-/by} rule (\link{Section 4.4.1}) from the data base.  When he
realizes this, he quickly reinstalls it.  Unfortunately, he makes a slight
change in the rule, and types it in as

\begin{scheme}
(rule (outranked-by ?staff-person ?boss)
      (or (supervisor ?staff-person ?boss)
          (and (outranked-by ?middle-manager ?boss)
               (supervisor ?staff-person 
                           ?middle-manager))))
\end{scheme}

Just after Louis types this information into the system, DeWitt Aull comes by
to find out who outranks Ben Bitdiddle. He issues the query

\begin{scheme}
(outranked-by (Bitdiddle Ben) ?who)
\end{scheme}

After answering, the system goes into an infinite loop.  Explain why.
\end{quote}

\begin{quote}
\heading{\phantomsection\label{Exercise 4.65}Exercise 4.65:} Cy D. Fect, looking forward to
the day when he will rise in the organization, gives a query to find all the
wheels (using the \code{wheel} rule of \link{Section 4.4.1}):

\begin{scheme}
(wheel ?who)
\end{scheme}

To his surprise, the system responds

\begin{scheme}
~\textit{;;; Query results:}~
(wheel (Warbucks Oliver))
(wheel (Bitdiddle Ben))
(wheel (Warbucks Oliver))
(wheel (Warbucks Oliver))
(wheel (Warbucks Oliver))
\end{scheme}

Why is Oliver Warbucks listed four times?
\end{quote}

\begin{quote}
\heading{\phantomsection\label{Exercise 4.66}Exercise 4.66:} Ben has been generalizing the
query system to provide statistics about the company.  For example, to find the
total salaries of all the computer programmers one will be able to say

\begin{scheme}
(sum ?amount (and (job ?x (computer programmer))
                  (salary ?x ?amount)))
\end{scheme}

In general, Ben's new system allows expressions of the form

\begin{scheme}
(accumulation-function ~\( \dark \langle \)~~\var{\dark variable}~~\( \dark \rangle \)~ ~\( \dark \langle \)~~\var{\dark query pattern}~~\( \dark \rangle \)~)
\end{scheme}

\noindent
where \code{accumulation\-/function} can be things like \code{sum},
\code{average}, or \code{maximum}.  Ben reasons that it should be a cinch to
implement this.  He will simply feed the query pattern to \code{qeval}.  This
will produce a stream of frames.  He will then pass this stream through a
mapping function that extracts the value of the designated variable from each
frame in the stream and feed the resulting stream of values to the accumulation
function.  Just as Ben completes the implementation and is about to try it out,
Cy walks by, still puzzling over the \code{wheel} query result in 
\link{Exercise 4.65}.  When Cy shows Ben the system's response, Ben groans,
``Oh, no, my simple accumulation scheme won't work!''

What has Ben just realized?  Outline a method he can use to salvage the
situation.
\end{quote}

\begin{quote}
\heading{\phantomsection\label{Exercise 4.67}Exercise 4.67:} Devise a way to install a loop
detector in the query system so as to avoid the kinds of simple loops
illustrated in the text and in \link{Exercise 4.64}.  
The general idea is that
the system should maintain some sort of history of its current chain of
deductions and should not begin processing a query that it is already working
on.  Describe what kind of information (patterns and frames) is included in
this history, and how the check should be made.  (After you study the details
of the query-system implementation in \link{Section 4.4.4}, you may want to
modify the system to include your loop detector.)
\end{quote}

\begin{quote}
\heading{\phantomsection\label{Exercise 4.68}Exercise 4.68:} Define rules to implement the
\code{reverse} operation of \link{Exercise 2.18}, which returns a list
containing the same elements as a given list in reverse order.  (Hint: Use
\code{append\-/to\-/form}.)  Can your rules answer both \code{(reverse (1 2 3) ?x)}
and \code{(reverse ?x (1 2 3))} ?
\end{quote}

\begin{quote}
\heading{\phantomsection\label{Exercise 4.69}Exercise 4.69:} Beginning with the data base and
the rules you formulated in \link{Exercise 4.63}, devise a rule for adding
``greats'' to a grandson relationship. This should enable the system to deduce
that Irad is the great-grandson of Adam, or that Jabal and Jubal are the
great-great-great-great-great-grandsons of Adam.  (Hint: Represent the fact
about Irad, for example, as \code{((great grandson) Adam Irad)}.  Write rules
that determine if a list ends in the word \code{grandson}.  Use this to express
a rule that allows one to derive the relationship \code{((great .  ?rel) ?x
?y)}, where \code{?rel} is a list ending in \code{grandson}.)  Check your rules
on queries such as \code{((great grandson) ?g ?ggs)} and \code{(?relationship
Adam Irad)}.
\end{quote}

\subsection{Implementing the Query System}
\label{Section 4.4.4}

\link{Section 4.4.2} described how the query system works. Now we fill in the
details by presenting a complete implementation of the system.



\subsubsection{The Driver Loop and Instantiation}
\label{Section 4.4.4.1}

The driver loop for the query system repeatedly reads input expressions.  If
the expression is a rule or assertion to be added to the data base, then the
information is added.  Otherwise the expression is assumed to be a query.  The
driver passes this query to the evaluator \code{qeval} together with an initial
frame stream consisting of a single empty frame.  The result of the evaluation
is a stream of frames generated by satisfying the query with variable values
found in the data base.  These frames are used to form a new stream consisting
of copies of the original query in which the variables are instantiated with
values supplied by the stream of frames, and this final stream is printed at
the terminal:

\begin{scheme}
(define input-prompt  ";;; Query input:")
(define output-prompt ";;; Query results:")

(define (query-driver-loop)
  (prompt-for-input input-prompt)
  (let ((q (query-syntax-process (read))))
    (cond ((assertion-to-be-added? q)
           (add-rule-or-assertion! (add-assertion-body q))
           (newline)
           (display "Assertion added to data base.")
           (query-driver-loop))
          (else
           (newline)
           (display output-prompt)
           (display-stream
            (stream-map
             (lambda (frame)
               (instantiate
                q
                frame
                (lambda (v f)
                  (contract-question-mark v))))
             (qeval q (singleton-stream '()))))
           (query-driver-loop)))))
\end{scheme}

\noindent
Here, as in the other evaluators in this chapter, we use an abstract syntax for
the expressions of the query language.  The implementation of the expression
syntax, including the predicate \code{assertion\-/to\-/be\-/added?} and the selector
\code{add\-/assertion\-/body}, is given in \link{Section 4.4.4.7}.
\code{Add\-/rule\-/or\-/assertion!} is defined in \link{Section 4.4.4.5}.

Before doing any processing on an input expression, the driver loop transforms
it syntactically into a form that makes the processing more efficient.  This
involves changing the representation of pattern variables.  When the query is
instantiated, any variables that remain unbound are transformed back to the
input representation before being printed.  These transformations are performed
by the two procedures \code{query\-/syntax\-/process} and
\code{contract\-/question\-/mark} (\link{Section 4.4.4.7}).

To instantiate an expression, we copy it, replacing any variables in the
expression by their values in a given frame.  The values are themselves
instantiated, since they could contain variables (for example, if \code{?x} in
\code{exp} is bound to \code{?y} as the result of unification and \code{?y} is
in turn bound to 5).  The action to take if a variable cannot be instantiated
is given by a procedural argument to \code{instantiate}.

\begin{scheme}
(define (instantiate exp frame unbound-var-handler)
  (define (copy exp)
    (cond ((var? exp)
           (let ((binding (binding-in-frame exp frame)))
             (if binding
                 (copy (binding-value binding))
                 (unbound-var-handler exp frame))))
          ((pair? exp)
           (cons (copy (car exp)) (copy (cdr exp))))
          (else exp)))
  (copy exp))
\end{scheme}

\noindent
The procedures that manipulate bindings are defined in \link{Section 4.4.4.8}.

\subsubsection{The Evaluator}
\label{Section 4.4.4.2}

The \code{qeval} procedure, called by the \code{query\-/driver\-/loop}, is the
basic evaluator of the query system.  It takes as inputs a query and a stream
of frames, and it returns a stream of extended frames.  It identifies special
forms by a data-directed dispatch using \code{get} and \code{put}, just as we
did in implementing generic operations in \link{Chapter 2}.  Any query that is
not identified as a special form is assumed to be a simple query, to be
processed by \code{simple\-/query}.

\begin{scheme}
(define (qeval query frame-stream)
  (let ((qproc (get (type query) 'qeval)))
    (if qproc
        (qproc (contents query) frame-stream)
        (simple-query query frame-stream))))
\end{scheme}

\noindent
\code{Type} and \code{contents}, defined in \link{Section 4.4.4.7}, implement
the abstract syntax of the special forms.

\subsubsection*{Simple queries}

The \code{simple\-/query} procedure handles simple queries.  It takes as
arguments a simple query (a pattern) together with a stream of frames, and it
returns the stream formed by extending each frame by all data-base matches of
the query.

\begin{scheme}
(define (simple-query query-pattern frame-stream)
  (stream-flatmap
   (lambda (frame)
     (stream-append-delayed
      (find-assertions query-pattern frame)
      (delay (apply-rules query-pattern frame))))
   frame-stream))
\end{scheme}

\noindent
For each frame in the input stream, we use \code{find\-/assertions} 
(\link{Section 4.4.4.3}) to match the pattern against all assertions in the data base,
producing a stream of extended frames, and we use \code{apply\-/rules} 
(\link{Section 4.4.4.4}) to apply all possible rules, producing another stream of
extended frames.  These two streams are combined (using
\code{stream\-/append\-/delayed}, \link{Section 4.4.4.6}) to make a stream of all
the ways that the given pattern can be satisfied consistent with the original
frame (see \link{Exercise 4.71}).  The streams for the individual input frames
are combined using \code{stream\-/flatmap} (\link{Section 4.4.4.6}) to form one
large stream of all the ways that any of the frames in the original input
stream can be extended to produce a match with the given pattern.

\subsubsection*{Compound queries}

\code{And} queries are handled as illustrated in \link{Figure 4.5} by the
\code{conjoin} procedure.  \code{Conjoin} takes as inputs the conjuncts and the
frame stream and returns the stream of extended frames.  First, \code{conjoin}
processes the stream of frames to find the stream of all possible frame
extensions that satisfy the first query in the conjunction.  Then, using this
as the new frame stream, it recursively applies \code{conjoin} to the rest of
the queries.

\begin{scheme}
(define (conjoin conjuncts frame-stream)
  (if (empty-conjunction? conjuncts)
      frame-stream
      (conjoin (rest-conjuncts conjuncts)
               (qeval (first-conjunct conjuncts) frame-stream))))
\end{scheme}

\noindent
The expression

\begin{scheme}
(put 'and 'qeval conjoin)
\end{scheme}

\noindent
sets up \code{qeval} to dispatch to \code{conjoin} when an \code{and} form is
encountered.

\code{Or} queries are handled similarly, as shown in \link{Figure 4.6}.  The
output streams for the various disjuncts of the \code{or} are computed
separately and merged using the \code{interleave\-/delayed} procedure from
\link{Section 4.4.4.6}.  (See \link{Exercise 4.71} and \link{Exercise 4.72}.)

\begin{scheme}
(define (disjoin disjuncts frame-stream)
  (if (empty-disjunction? disjuncts)
      the-empty-stream
      (interleave-delayed
       (qeval (first-disjunct disjuncts) frame-stream)
       (delay (disjoin (rest-disjuncts disjuncts) frame-stream)))))
(put 'or 'qeval disjoin)
\end{scheme}

\noindent
The predicates and selectors for the syntax of conjuncts and disjuncts are
given in \link{Section 4.4.4.7}.

\subsubsection*{Filters}

\code{Not} is handled by the method outlined in \link{Section 4.4.2}.  We
attempt to extend each frame in the input stream to satisfy the query being
negated, and we include a given frame in the output stream only if it cannot be
extended.

\begin{scheme}
(define (negate operands frame-stream)
  (stream-flatmap
   (lambda (frame)
     (if (stream-null?
          (qeval (negated-query operands)
                 (singleton-stream frame)))
         (singleton-stream frame)
         the-empty-stream))
   frame-stream))
(put 'not 'qeval negate)
\end{scheme}

\noindent
\code{Lisp\-/value} is a filter similar to \code{not}.  Each frame in the stream
is used to instantiate the variables in the pattern, the indicated predicate is
applied, and the frames for which the predicate returns false are filtered out
of the input stream.  An error results if there are unbound pattern variables.

\begin{scheme}
(define (lisp-value call frame-stream)
  (stream-flatmap
   (lambda (frame)
     (if (execute
          (instantiate
           call
           frame
           (lambda (v f)
             (error "Unknown pat var: LISP-VALUE" v))))
         (singleton-stream frame)
         the-empty-stream))
   frame-stream))
(put 'lisp-value 'qeval lisp-value)
\end{scheme}

\noindent
\code{Execute}, which applies the predicate to the arguments, must \code{eval}
the predicate expression to get the procedure to apply.  However, it must not
evaluate the arguments, since they are already the actual arguments, not
expressions whose evaluation (in Lisp) will produce the arguments.  Note that
\code{execute} is implemented using \code{eval} and \code{apply} from the
underlying Lisp system.

\begin{scheme}
(define (execute exp)
  (apply (eval (predicate exp) user-initial-environment)
         (args exp)))
\end{scheme}

\noindent
The \code{always\-/true} special form provides for a query that is always
satisfied.  It ignores its contents (normally empty) and simply passes through
all the frames in the input stream.  \code{Always\-/true} is used by the
\code{rule\-/body} selector (\link{Section 4.4.4.7}) to provide bodies for rules
that were defined without bodies (that is, rules whose conclusions are always
satisfied).

\begin{scheme}
(define (always-true ignore frame-stream) frame-stream)
(put 'always-true 'qeval always-true)
\end{scheme}

\noindent
The selectors that define the syntax of \code{not} and \code{lisp\-/value} are
given in \link{Section 4.4.4.7}.

\subsubsection{Finding Assertions\\ by Pattern Matching}
\label{Section 4.4.4.3}

\code{Find\-/assertions}, called by \code{simple\-/query} (\link{Section 4.4.4.2}),
takes as input a pattern and a frame.  It returns a stream of frames, each
extending the given one by a data-base match of the given pattern.  It uses
\code{fetch\-/assertions} (\link{Section 4.4.4.5}) to get a stream of all the
assertions in the data base that should be checked for a match against the
pattern and the frame.  The reason for \code{fetch\-/assertions} here is that we
can often apply simple tests that will eliminate many of the entries in the
data base from the pool of candidates for a successful match.  The system would
still work if we eliminated \code{fetch\-/assertions} and simply checked a stream
of all assertions in the data base, but the computation would be less efficient
because we would need to make many more calls to the matcher.

\begin{scheme}
(define (find-assertions pattern frame)
  (stream-flatmap 
   (lambda (datum) (check-an-assertion datum pattern frame))
   (fetch-assertions pattern frame)))
\end{scheme}

\noindent
\code{Check\-/an\-/assertion} takes as arguments a pattern, a data object
(assertion), and a frame and returns either a one-element stream containing the
extended frame or \code{the\-/empty\-/stream} if the match fails.

\begin{scheme}
(define (check-an-assertion assertion query-pat query-frame)
  (let ((match-result
         (pattern-match query-pat assertion query-frame)))
    (if (eq? match-result 'failed)
        the-empty-stream
        (singleton-stream match-result))))
\end{scheme}

\noindent
The basic pattern matcher returns either the symbol \code{failed} or an
extension of the given frame.  The basic idea of the matcher is to check the
pattern against the data, element by element, accumulating bindings for the
pattern variables.  If the pattern and the data object are the same, the match
succeeds and we return the frame of bindings accumulated so far.  Otherwise, if
the pattern is a variable we extend the current frame by binding the variable
to the data, so long as this is consistent with the bindings already in the
frame.  If the pattern and the data are both pairs, we (recursively) match the
\code{car} of the pattern against the \code{car} of the data to produce a
frame; in this frame we then match the \code{cdr} of the pattern against the
\code{cdr} of the data.  If none of these cases are applicable, the match fails
and we return the symbol \code{failed}.

\begin{scheme}
(define (pattern-match pat dat frame)
  (cond ((eq? frame 'failed) 'failed)
        ((equal? pat dat) frame)
        ((var? pat) (extend-if-consistent pat dat frame))
        ((and (pair? pat) (pair? dat))
         (pattern-match 
          (cdr pat)
          (cdr dat)
          (pattern-match (car pat) (car dat) frame)))
        (else 'failed)))
\end{scheme}

\noindent
Here is the procedure that extends a frame by adding a new binding, if this is
consistent with the bindings already in the frame:

\begin{scheme}
(define (extend-if-consistent var dat frame)
  (let ((binding (binding-in-frame var frame)))
    (if binding
        (pattern-match (binding-value binding) dat frame)
        (extend var dat frame))))
\end{scheme}

\noindent
If there is no binding for the variable in the frame, we simply add the binding
of the variable to the data.  Otherwise we match, in the frame, the data
against the value of the variable in the frame.  If the stored value contains
only constants, as it must if it was stored during pattern matching by
\code{extend\-/if\-/consistent}, then the match simply tests whether the stored and
new values are the same.  If so, it returns the unmodified frame; if not, it
returns a failure indication.  The stored value may, however, contain pattern
variables if it was stored during unification (see \link{Section 4.4.4.4}).  The
recursive match of the stored pattern against the new data will add or check
bindings for the variables in this pattern.  For example, suppose we have a
frame in which \code{?x} is bound to \code{(f ?y)} and \code{?y} is unbound,
and we wish to augment this frame by a binding of \code{?x} to \code{(f b)}.
We look up \code{?x} and find that it is bound to \code{(f ?y)}.  This leads us
to match \code{(f ?y)} against the proposed new value \code{(f b)} in the same
frame.  Eventually this match extends the frame by adding a binding of
\code{?y} to \code{b}.  \code{?X} remains bound to \code{(f ?y)}.  We never
modify a stored binding and we never store more than one binding for a given
variable.

The procedures used by \code{extend\-/if\-/consistent} to manipulate bindings are
defined in \link{Section 4.4.4.8}.

\subsubsection*{Patterns with dotted tails}

If a pattern contains a dot followed by a pattern variable, the pattern
variable matches the rest of the data list (rather than the next element of the
data list), just as one would expect with the dotted-tail notation described in
\link{Exercise 2.20}.  Although the pattern matcher we have just implemented
doesn't look for dots, it does behave as we want.  This is because the Lisp
\code{read} primitive, which is used by \code{query\-/driver\-/loop} to read the
query and represent it as a list structure, treats dots in a special way.

When \code{read} sees a dot, instead of making the next item be the next
element of a list (the \code{car} of a \code{cons} whose \code{cdr} will be the
rest of the list) it makes the next item be the \code{cdr} of the list
structure.  For example, the list structure produced by \code{read} for the
pattern \code{(computer ?type)} could be constructed by evaluating the
expression \code{(cons 'computer (cons '?type '()))}, and that for
\code{(computer . ?type)} could be constructed by evaluating the expression
\code{(cons 'computer '?type)}.

Thus, as \code{pattern\-/match} recursively compares \code{car}s and \code{cdr}s
of a data list and a pattern that had a dot, it eventually matches the variable
after the dot (which is a \code{cdr} of the pattern) against a sublist of the
data list, binding the variable to that list.  For example, matching the
pattern \code{(computer . ?type)} against \code{(computer programmer trainee)}
will match \code{?type} against the list \code{(programmer trainee)}.

\subsubsection{Rules and Unification}
\label{Section 4.4.4.4}

\code{Apply\-/rules} is the rule analog of \code{find\-/assertions} 
(\link{Section 4.4.4.3}).  It takes as input a pattern and a frame, and it forms a stream
of extension frames by applying rules from the data base.
\code{Stream\-/flatmap} maps \code{apply\-/a\-/rule} down the stream of possibly
applicable rules (selected by \code{fetch\-/rules}, \link{Section 4.4.4.5}) and
combines the resulting streams of frames.

\begin{scheme}
(define (apply-rules pattern frame)
  (stream-flatmap (lambda (rule)
                    (apply-a-rule rule pattern frame))
                  (fetch-rules pattern frame)))
\end{scheme}

\noindent
\code{Apply\-/a\-/rule} applies rules using the method outlined in 
\link{Section 4.4.2}.  It first augments its argument frame by unifying the rule
conclusion with the pattern in the given frame.  If this succeeds, it evaluates
the rule body in this new frame.

Before any of this happens, however, the program renames all the variables in
the rule with unique new names.  The reason for this is to prevent the
variables for different rule applications from becoming confused with each
other.  For instance, if two rules both use a variable named \code{?x}, then
each one may add a binding for \code{?x} to the frame when it is applied.
These two \code{?x}'s have nothing to do with each other, and we should not be
fooled into thinking that the two bindings must be consistent.  Rather than
rename variables, we could devise a more clever environment structure; however,
the renaming approach we have chosen here is the most straightforward, even if
not the most efficient.  (See \link{Exercise 4.79}.)  Here is the
\code{apply\-/a\-/rule} procedure:

\begin{scheme}
(define (apply-a-rule rule query-pattern query-frame)
  (let ((clean-rule (rename-variables-in rule)))
    (let ((unify-result (unify-match query-pattern
                                     (conclusion clean-rule)
                                     query-frame)))
      (if (eq? unify-result 'failed)
          the-empty-stream
          (qeval (rule-body clean-rule)
                 (singleton-stream unify-result))))))
\end{scheme}

\noindent
The selectors \code{rule\-/body} and \code{conclusion} that extract parts of a
rule are defined in \link{Section 4.4.4.7}.

We generate unique variable names by associating a unique identifier (such as a
number) with each rule application and combining this identifier with the
original variable names.  For example, if the rule-application identifier is 7,
we might change each \code{?x} in the rule to \code{?x\-/7} and each \code{?y} in
the rule to \code{?y\-/7}.  (\code{Make\-/new\-/variable} and
\code{new\-/rule\-/application\-/id} are included with the syntax procedures in
\link{Section 4.4.4.7}.)

\begin{scheme}
(define (rename-variables-in rule)
  (let ((rule-application-id (new-rule-application-id)))
    (define (tree-walk exp)
      (cond ((var? exp)
             (make-new-variable exp rule-application-id))
            ((pair? exp)
             (cons (tree-walk (car exp))
                   (tree-walk (cdr exp))))
            (else exp)))
    (tree-walk rule)))
\end{scheme}

\noindent
The unification algorithm is implemented as a procedure that takes as inputs
two patterns and a frame and returns either the extended frame or the symbol
\code{failed}.  The unifier is like the pattern matcher except that it is
symmetrical---variables are allowed on both sides of the match.
\code{Unify\-/match} is basically the same as \code{pattern\-/match}, except that
there is extra code (marked ``\code{***}'' below) to handle the case where the
object on the right side of the match is a variable.

\begin{scheme}
(define (unify-match p1 p2 frame)
  (cond ((eq? frame 'failed) 'failed)
        ((equal? p1 p2) frame)
        ((var? p1) (extend-if-possible p1 p2 frame))
        ((var? p2) (extend-if-possible p2 p1 frame))  ~\textrm{; ***}~
        ((and (pair? p1) (pair? p2))
         (unify-match (cdr p1)
                      (cdr p2)
                      (unify-match (car p1)
                                   (car p2)
                                   frame)))
        (else 'failed)))
\end{scheme}

\noindent
In unification, as in one-sided pattern matching, we want to accept a proposed
extension of the frame only if it is consistent with existing bindings.  The
procedure \code{extend\-/if\-/possible} used in unification is the same as the
\code{extend\-/if\-/consistent} used in pattern matching except for two special
checks, marked ``\code{***}'' in the program below.  In the first case, if the
variable we are trying to match is not bound, but the value we are trying to
match it with is itself a (different) variable, it is necessary to check to see
if the value is bound, and if so, to match its value.  If both parties to the
match are unbound, we may bind either to the other.

The second check deals with attempts to bind a variable to a pattern that
includes that variable.  Such a situation can occur whenever a variable is
repeated in both patterns.  Consider, for example, unifying the two patterns
\code{(?x ?x)} and \code{(?y \( \langle \)\var{expression involving \code{?y}}\( \rangle \))} in a
frame where both \code{?x} and \code{?y} are unbound.  First \code{?x} is
matched against \code{?y}, making a binding of \code{?x} to \code{?y}.  Next,
the same \code{?x} is matched against the given expression involving \code{?y}.
Since \code{?x} is already bound to \code{?y}, this results in matching
\code{?y} against the expression.  If we think of the unifier as finding a set
of values for the pattern variables that make the patterns the same, then these
patterns imply instructions to find a \code{?y} such that \code{?y} is equal to
the expression involving \code{?y}.  There is no general method for solving
such equations, so we reject such bindings; these cases are recognized by the
predicate \code{depends\-/on?}.\footnote{In general, unifying \code{?y} with an
expression involving \code{?y} would require our being able to find a fixed
point of the equation \code{?y} = \( \langle \)\var{expression involving \code{?y}}\( \rangle \).  It
is sometimes possible to syntactically form an expression that appears to be
the solution.  For example, \code{?y} = \code{(f ?y)} seems to have the fixed
point \code{(f (f (f \( \dots \) )))}, which we can produce by beginning with the
expression \code{(f ?y)} and repeatedly substituting \code{(f ?y)} for
\code{?y}.  Unfortunately, not every such equation has a meaningful fixed
point.  The issues that arise here are similar to the issues of manipulating
infinite series in mathematics.  For example, we know that 2 is the solution to
the equation \( y = 1 + y / 2 \).  Beginning with the expression \( 1 + y / 2 \)
and repeatedly substituting \( 1 + y / 2 \) for \( y \) gives
\begin{comment}

\begin{example}
2 = y = 1 + y/2 = 1 + (1 + y/2)/2 = 1 + 1/2 + y/4 = ...
\end{example}

\end{comment}
\begin{displaymath}
 2 = y = 1 + {y \over 2} = 1 + {1\over2}\left(1 + {y \over 2}\right) = 
	1 + {1\over2} + {y \over 4} = \dots , 
\end{displaymath}
\noindent
which leads to
\begin{comment}

\begin{example}
2 = 1 + 1/2 + 1/4 + 1/8 + ...
\end{example}

\end{comment}
\begin{displaymath}
 2 = 1 + {1\over2} + {1\over4} + {1\over8} + \dots. 
\end{displaymath}
\noindent
However, if we try the same manipulation beginning with the observation that -1 
is the solution to the equation \( y = 1 + 2y \), we obtain
\begin{comment}

\begin{example}
-1 = y = 1 + 2y = 1 + 2(1 + 2y) = 1 + 2 + 4y = ...
\end{example}

\end{comment}
\begin{displaymath}
 -1 = y = 1 + 2y = 1 + 2(1 + 2y) = 1 + 2 + 4y = \dots, 
\end{displaymath}
\noindent
which leads to
\begin{comment}

\begin{example}
-1 = 1 + 2 + 4 + 8 + ...
\end{example}

\end{comment}
\begin{displaymath}
 -1 = 1 + 2 + 4 + 8 + \dots. 
\end{displaymath}
\noindent
Although the formal manipulations used in deriving these two equations are
identical, the first result is a valid assertion about infinite series but the
second is not.  Similarly, for our unification results, reasoning with an
arbitrary syntactically constructed expression may lead to errors.}  On the
other hand, we do not want to reject attempts to bind a variable to itself.
For example, consider unifying \code{(?x ?x)} and \code{(?y ?y)}.  The second
attempt to bind \code{?x} to \code{?y} matches \code{?y} (the stored value of
\code{?x}) against \code{?y} (the new value of \code{?x}).  This is taken care
of by the \code{equal?}  clause of \code{unify\-/match}.

\begin{scheme}
(define (extend-if-possible var val frame)
  (let ((binding (binding-in-frame var frame)))
    (cond (binding
           (unify-match (binding-value binding) val frame))
          ((var? val)                      ~\textrm{; ***}~
           (let ((binding (binding-in-frame val frame)))
             (if binding
                 (unify-match
                  var (binding-value binding) frame)
                 (extend var val frame))))
          ((depends-on? val var frame)     ~\textrm{; ***}~
           'failed)
          (else (extend var val frame)))))
\end{scheme}

\noindent
\code{Depends\-/on?} is a predicate that tests whether an expression proposed to
be the value of a pattern variable depends on the variable.  This must be done
relative to the current frame because the expression may contain occurrences of
a variable that already has a value that depends on our test variable.  The
structure of \code{depends\-/on?} is a simple recursive tree walk in which we
substitute for the values of variables whenever necessary.

\begin{scheme}
(define (depends-on? exp var frame)
  (define (tree-walk e)
    (cond ((var? e)
           (if (equal? var e)
               true
               (let ((b (binding-in-frame e frame)))
                 (if b
                     (tree-walk (binding-value b))
                     false))))
          ((pair? e)
           (or (tree-walk (car e))
               (tree-walk (cdr e))))
          (else false)))
  (tree-walk exp))
\end{scheme}

\subsubsection{Maintaining the Data Base}
\label{Section 4.4.4.5}

One important problem in designing logic programming languages is that of
arranging things so that as few irrelevant data-base entries as possible will
be examined in checking a given pattern.  In our system, in addition to storing
all assertions in one big stream, we store all assertions whose \code{car}s are
constant symbols in separate streams, in a table indexed by the symbol.  To
fetch an assertion that may match a pattern, we first check to see if the
\code{car} of the pattern is a constant symbol.  If so, we return (to be tested
using the matcher) all the stored assertions that have the same \code{car}.  If
the pattern's \code{car} is not a constant symbol, we return all the stored
assertions.  Cleverer methods could also take advantage of information in the
frame, or try also to optimize the case where the \code{car} of the pattern is
not a constant symbol.  We avoid building our criteria for indexing (using the
\code{car}, handling only the case of constant symbols) into the program;
instead we call on predicates and selectors that embody our criteria.

\begin{scheme}
(define THE-ASSERTIONS the-empty-stream)
(define (fetch-assertions pattern frame)
  (if (use-index? pattern)
      (get-indexed-assertions pattern)
      (get-all-assertions)))
(define (get-all-assertions) THE-ASSERTIONS)
(define (get-indexed-assertions pattern)
  (get-stream (index-key-of pattern) 'assertion-stream))
\end{scheme}

\noindent
\code{Get\-/stream} looks up a stream in the table and returns an empty stream if
nothing is stored there.

\begin{scheme}
(define (get-stream key1 key2)
  (let ((s (get key1 key2)))
    (if s s the-empty-stream)))
\end{scheme}

\noindent
Rules are stored similarly, using the \code{car} of the rule conclusion.  Rule
conclusions are arbitrary patterns, however, so they differ from assertions in
that they can contain variables.  A pattern whose \code{car} is a constant
symbol can match rules whose conclusions start with a variable as well as rules
whose conclusions have the same \code{car}.  Thus, when fetching rules that
might match a pattern whose \code{car} is a constant symbol we fetch all rules
whose conclusions start with a variable as well as those whose conclusions have
the same \code{car} as the pattern.  For this purpose we store all rules whose
conclusions start with a variable in a separate stream in our table, indexed by
the symbol \code{?}.

\begin{scheme}
(define THE-RULES the-empty-stream)
(define (fetch-rules pattern frame)
  (if (use-index? pattern)
      (get-indexed-rules pattern)
      (get-all-rules)))
(define (get-all-rules) THE-RULES)
(define (get-indexed-rules pattern)
  (stream-append
   (get-stream (index-key-of pattern) 'rule-stream)
   (get-stream '? 'rule-stream)))
\end{scheme}

\noindent
\code{Add\-/rule\-/or\-/assertion!} is used by \code{query\-/driver\-/loop} to add
assertions and rules to the data base.  Each item is stored in the index, if
appropriate, and in a stream of all assertions or rules in the data base.

\begin{scheme}
(define (add-rule-or-assertion! assertion)
  (if (rule? assertion)
      (add-rule! assertion)
      (add-assertion! assertion)))
(define (add-assertion! assertion)
  (store-assertion-in-index assertion)
  (let ((old-assertions THE-ASSERTIONS))
    (set! THE-ASSERTIONS
          (cons-stream assertion old-assertions))
    'ok))
(define (add-rule! rule)
  (store-rule-in-index rule)
  (let ((old-rules THE-RULES))
    (set! THE-RULES (cons-stream rule old-rules))
    'ok))
\end{scheme}

\noindent
To actually store an assertion or a rule, we check to see if it can be indexed.
If so, we store it in the appropriate stream.

\begin{scheme}
(define (store-assertion-in-index assertion)
  (if (indexable? assertion)
      (let ((key (index-key-of assertion)))
        (let ((current-assertion-stream
               (get-stream key 'assertion-stream)))
          (put key
               'assertion-stream
               (cons-stream
                assertion
                current-assertion-stream))))))
(define (store-rule-in-index rule)
  (let ((pattern (conclusion rule)))
    (if (indexable? pattern)
        (let ((key (index-key-of pattern)))
          (let ((current-rule-stream
                 (get-stream key 'rule-stream)))
            (put key
                 'rule-stream
                 (cons-stream rule
                              current-rule-stream)))))))
\end{scheme}

\noindent
The following procedures define how the data-base index is used.  A pattern (an
assertion or a rule conclusion) will be stored in the table if it starts with a
variable or a constant symbol.

\begin{scheme}
(define (indexable? pat)
  (or (constant-symbol? (car pat))
      (var? (car pat))))
\end{scheme}

\noindent
The key under which a pattern is stored in the table is either \code{?} (if it
starts with a variable) or the constant symbol with which it starts.

\begin{scheme}
(define (index-key-of pat)
  (let ((key (car pat)))
    (if (var? key) '? key)))
\end{scheme}

\noindent
The index will be used to retrieve items that might match a pattern if the
pattern starts with a constant symbol.

\begin{scheme}
(define (use-index? pat) (constant-symbol? (car pat)))
\end{scheme}

\begin{quote}
\heading{\phantomsection\label{Exercise 4.70}Exercise 4.70:} What is the purpose of the
\code{let} bindings in the procedures \code{add\-/assertion!} and
\code{add\-/rule!} ?  What would be wrong with the following implementation of
\code{add\-/assertion!} ?  Hint: Recall the definition of the infinite stream of
ones in \link{Section 3.5.2}: \code{(define ones (cons\-/stream 1 ones))}.

\begin{scheme}
(define (add-assertion! assertion)
  (store-assertion-in-index assertion)
  (set! THE-ASSERTIONS
        (cons-stream assertion THE-ASSERTIONS))
  'ok)
\end{scheme}
\end{quote}

\subsubsection{Stream Operations}
\label{Section 4.4.4.6}

The query system uses a few stream operations that were not presented in
\link{Chapter 3}.

\code{Stream\-/append\-/delayed} and \code{interleave\-/delayed} are just like
\code{stream\-/append} and \code{interleave} (\link{Section 3.5.3}), except that
they take a delayed argument (like the \code{integral} procedure in 
\link{Section 3.5.4}).  This postpones looping in some cases (see \link{Exercise 4.71}).

\begin{scheme}
(define (stream-append-delayed s1 delayed-s2)
  (if (stream-null? s1)
      (force delayed-s2)
      (cons-stream
       (stream-car s1)
       (stream-append-delayed
        (stream-cdr s1)
        delayed-s2))))
(define (interleave-delayed s1 delayed-s2)
  (if (stream-null? s1)
      (force delayed-s2)
      (cons-stream
       (stream-car s1)
       (interleave-delayed
        (force delayed-s2)
        (delay (stream-cdr s1))))))
\end{scheme}

\noindent
\code{Stream\-/flatmap}, which is used throughout the query evaluator to map a
procedure over a stream of frames and combine the resulting streams of frames,
is the stream analog of the \code{flatmap} procedure introduced for ordinary
lists in \link{Section 2.2.3}.  Unlike ordinary \code{flatmap}, however, we
accumulate the streams with an interleaving process, rather than simply
appending them (see \link{Exercise 4.72} and \link{Exercise 4.73}).

\begin{scheme}
(define (stream-flatmap proc s)
  (flatten-stream (stream-map proc s)))

(define (flatten-stream stream)
  (if (stream-null? stream)
      the-empty-stream
      (interleave-delayed
       (stream-car stream)
       (delay (flatten-stream (stream-cdr stream))))))
\end{scheme}

\noindent
The evaluator also uses the following simple procedure to generate a stream
consisting of a single element:

\begin{scheme}
(define (singleton-stream x)
  (cons-stream x the-empty-stream))
\end{scheme}

\subsubsection{Query Syntax Procedures}
\label{Section 4.4.4.7}

\code{Type} and \code{contents}, used by \code{qeval} (\link{Section 4.4.4.2}),
specify that a special form is identified by the symbol in its \code{car}.
They are the same as the \code{type\-/tag} and \code{contents} procedures in
\link{Section 2.4.2}, except for the error message.

\begin{scheme}
(define (type exp)
  (if (pair? exp)
      (car exp)
      (error "Unknown expression TYPE" exp)))
(define (contents exp)
  (if (pair? exp)
      (cdr exp)
      (error "Unknown expression CONTENTS" exp)))
\end{scheme}

\noindent
The following procedures, used by \code{query\-/driver\-/loop} (in 
\link{Section 4.4.4.1}), specify that rules and assertions are added to the data base by
expressions of the form \code{(assert! \( \langle \)\var{rule\-/or\-/assertion}\( \rangle \))}:

\begin{scheme}
(define (assertion-to-be-added? exp)
  (eq? (type exp) 'assert!))
(define (add-assertion-body exp) (car (contents exp)))
\end{scheme}

\noindent
Here are the syntax definitions for the \code{and}, \code{or}, \code{not}, and
\code{lisp\-/value} special forms (\link{Section 4.4.4.2}):

\begin{scheme}
(define (empty-conjunction? exps) (null? exps))
(define (first-conjunct exps) (car exps))
(define (rest-conjuncts exps) (cdr exps))
(define (empty-disjunction? exps) (null? exps))
(define (first-disjunct exps) (car exps))
(define (rest-disjuncts exps) (cdr exps))
(define (negated-query exps) (car exps))
(define (predicate exps) (car exps))
(define (args exps) (cdr exps))
\end{scheme}

\noindent
The following three procedures define the syntax of rules:

\begin{scheme}
(define (rule? statement)
  (tagged-list? statement 'rule))
(define (conclusion rule) (cadr rule))
(define (rule-body rule)
  (if (null? (cddr rule)) '(always-true) (caddr rule)))
\end{scheme}

\noindent
\code{Query\-/driver\-/loop} (\link{Section 4.4.4.1}) calls
\code{query\-/syntax\-/process} to transform pattern variables in the expression,
which have the form \code{?symbol}, into the internal format \code{(? symbol)}.
That is to say, a pattern such as \code{(job ?x ?y)} is actually represented
internally by the system as \code{(job (? x) (? y))}.  This increases the
efficiency of query processing, since it means that the system can check to see
if an expression is a pattern variable by checking whether the \code{car} of
the expression is the symbol \code{?}, rather than having to extract characters
from the symbol.  The syntax transformation is accomplished by the following
procedure:\footnote{Most Lisp systems give the user the ability to modify the
ordinary \code{read} procedure to perform such transformations by defining
\newterm{reader macro characters}.  Quoted expressions are already handled in
this way: The reader automatically translates \code{'expression} into
\code{(quote expression)} before the evaluator sees it.  We could arrange for
\code{?expression} to be transformed into \code{(? expression)} in the same
way; however, for the sake of clarity we have included the transformation
procedure here explicitly.

\code{Expand\-/question\-/mark} and \code{contract\-/question\-/mark} use several
procedures with \code{string} in their names.  These are Scheme primitives.}

\begin{scheme}
(define (query-syntax-process exp)
  (map-over-symbols expand-question-mark exp))
(define (map-over-symbols proc exp)
  (cond ((pair? exp)
         (cons (map-over-symbols proc (car exp))
               (map-over-symbols proc (cdr exp))))
        ((symbol? exp) (proc exp))
        (else exp)))
(define (expand-question-mark symbol)
  (let ((chars (symbol->string symbol)))
    (if (string=? (substring chars 0 1) "?")
        (list '?
              (string->symbol
               (substring chars 1 (string-length chars))))
        symbol)))
\end{scheme}

\noindent
Once the variables are transformed in this way, the variables in a pattern are
lists starting with \code{?}, and the constant symbols (which need to be
recognized for data-base indexing, \link{Section 4.4.4.5}) are just the symbols.

\begin{scheme}
(define (var? exp) (tagged-list? exp '?))
(define (constant-symbol? exp) (symbol? exp))
\end{scheme}

\noindent
Unique variables are constructed during rule application (in 
\link{Section 4.4.4.4}) by means of the following procedures.  
The unique identifier for
a rule application is a number, which is incremented each time a rule is
applied.

\begin{scheme}
(define rule-counter 0)
(define (new-rule-application-id)
  (set! rule-counter (+ 1 rule-counter))
  rule-counter)
(define (make-new-variable var rule-application-id)
  (cons '? (cons rule-application-id (cdr var))))
\end{scheme}

\noindent
When \code{query\-/driver\-/loop} instantiates the query to print the answer, it
converts any unbound pattern variables back to the right form for printing,
using

\begin{scheme}
(define (contract-question-mark variable)
  (string->symbol
   (string-append "?"
     (if (number? (cadr variable))
         (string-append (symbol->string (caddr variable))
                        "-"
                        (number->string (cadr variable)))
         (symbol->string (cadr variable))))))
\end{scheme}

\subsubsection{Frames and Bindings}
\label{Section 4.4.4.8}

Frames are represented as lists of bindings, which are variable-value pairs:

\begin{scheme}
(define (make-binding variable value)
  (cons variable value))
(define (binding-variable binding) (car binding))
(define (binding-value binding) (cdr binding))
(define (binding-in-frame variable frame)
  (assoc variable frame))
(define (extend variable value frame)
  (cons (make-binding variable value) frame))
\end{scheme}

\begin{quote}
\heading{\phantomsection\label{Exercise 4.71}Exercise 4.71:} Louis Reasoner wonders why the
\code{simple\-/query} and \code{disjoin} procedures (\link{Section 4.4.4.2}) are
implemented using explicit \code{delay} operations, rather than being defined
as follows:

\begin{scheme}
(define (simple-query query-pattern frame-stream)
  (stream-flatmap
   (lambda (frame)
     (stream-append
      (find-assertions query-pattern frame)
      (apply-rules query-pattern frame)))
   frame-stream))
(define (disjoin disjuncts frame-stream)
  (if (empty-disjunction? disjuncts)
      the-empty-stream
      (interleave
       (qeval (first-disjunct disjuncts)
              frame-stream)
       (disjoin (rest-disjuncts disjuncts)
                frame-stream))))
\end{scheme}

Can you give examples of queries where these simpler definitions would lead to
undesirable behavior?
\end{quote}

\begin{quote}
\heading{\phantomsection\label{Exercise 4.72}Exercise 4.72:} Why do \code{disjoin} and
\code{stream\-/flatmap} interleave the streams rather than simply append them?
Give examples that illustrate why interleaving works better.  (Hint: Why did we
use \code{interleave} in \link{Section 3.5.3}?)
\end{quote}

\begin{quote}
\heading{\phantomsection\label{Exercise 4.73}Exercise 4.73:} Why does \code{flatten\-/stream}
use \code{delay} explicitly?  What would be wrong with defining it as follows:

\begin{scheme}
(define (flatten-stream stream)
  (if (stream-null? stream)
      the-empty-stream
      (interleave
       (stream-car stream)
       (flatten-stream (stream-cdr stream)))))
\end{scheme}
\end{quote}

\begin{quote}
\heading{\phantomsection\label{Exercise 4.74}Exercise 4.74:} Alyssa P. Hacker proposes to use
a simpler version of \code{stream\-/flatmap} in \code{negate}, \code{lisp\-/value},
and \code{find\-/assertions}.  She observes that the procedure that is mapped
over the frame stream in these cases always produces either the empty stream or
a singleton stream, so no interleaving is needed when combining these streams.

\begin{enumerate}[a.]

\item
Fill in the missing expressions in Alyssa's program.

\begin{scheme}
(define (simple-stream-flatmap proc s)
  (simple-flatten (stream-map proc s)))
(define (simple-flatten stream)
  (stream-map ~\( \dark \langle \)~??~\( \dark \rangle \)~
              (stream-filter ~\( \dark \langle \)~??~\( \dark \rangle \)~ stream)))
\end{scheme}

\item
Does the query system's behavior change if we change it in this way?

\end{enumerate}
\end{quote}

\begin{quote}
\heading{\phantomsection\label{Exercise 4.75}Exercise 4.75:} Implement for the query language
a new special form called \code{unique}.  \code{Unique} should succeed if there
is precisely one item in the data base satisfying a specified query.  For
example,

\begin{scheme}
(unique (job ?x (computer wizard)))
\end{scheme}

\noindent
should print the one-item stream

\begin{scheme}
(unique (job (Bitdiddle Ben) (computer wizard)))
\end{scheme}

\noindent
since Ben is the only computer wizard, and

\begin{scheme}
(unique (job ?x (computer programmer)))
\end{scheme}

\noindent
should print the empty stream, since there is more than one computer
programmer.  Moreover,

\begin{scheme}
(and (job ?x ?j) (unique (job ?anyone ?j)))
\end{scheme}

\noindent
should list all the jobs that are filled by only one person, and the people who
fill them.

There are two parts to implementing \code{unique}.  The first is to write a
procedure that handles this special form, and the second is to make
\code{qeval} dispatch to that procedure.  The second part is trivial, since
\code{qeval} does its dispatching in a data-directed way.  If your procedure is
called \code{uniquely\-/asserted}, all you need to do is

\begin{scheme}
(put 'unique 'qeval uniquely-asserted)
\end{scheme}

\noindent
and \code{qeval} will dispatch to this procedure for every query whose
\code{type} (\code{car}) is the symbol \code{unique}.

The real problem is to write the procedure \code{uniquely\-/asserted}.  This
should take as input the \code{contents} (\code{cdr}) of the \code{unique}
query, together with a stream of frames.  For each frame in the stream, it
should use \code{qeval} to find the stream of all extensions to the frame that
satisfy the given query.  Any stream that does not have exactly one item in it
should be eliminated.  The remaining streams should be passed back to be
accumulated into one big stream that is the result of the \code{unique} query.
This is similar to the implementation of the \code{not} special form.

Test your implementation by forming a query that lists all people who supervise
precisely one person.
\end{quote}

\begin{quote}
\heading{\phantomsection\label{Exercise 4.76}Exercise 4.76:} Our implementation of \code{and}
as a series combination of queries (\link{Figure 4.5}) is elegant, but it is
inefficient because in processing the second query of the \code{and} we must
scan the data base for each frame produced by the first query.  If the data
base has \( n \) elements, and a typical query produces a number of output frames
proportional to \( n \) (say \( n / k \)), then scanning the data base for each
frame produced by the first query will require \( n^2 / k \) calls to the
pattern matcher.  Another approach would be to process the two clauses of the
\code{and} separately, then look for all pairs of output frames that are
compatible.  If each query produces \( n / k \) output frames, then this means
that we must perform \( n^2 / k^2 \) compatibility checks---a factor of \( k \)
fewer than the number of matches required in our current method.

Devise an implementation of \code{and} that uses this strategy.  You must
implement a procedure that takes two frames as inputs, checks whether the
bindings in the frames are compatible, and, if so, produces a frame that merges
the two sets of bindings.  This operation is similar to unification.
\end{quote}

\begin{quote}
\heading{\phantomsection\label{Exercise 4.77}Exercise 4.77:} In \link{Section 4.4.3} we saw
that \code{not} and \code{lisp\-/value} can cause the query language to give
``wrong'' answers if these filtering operations are applied to frames in which
variables are unbound.  Devise a way to fix this shortcoming.  One idea is to
perform the filtering in a ``delayed'' manner by appending to the frame a
``promise'' to filter that is fulfilled only when enough variables have been
bound to make the operation possible.  We could wait to perform filtering until
all other operations have been performed.  However, for efficiency's sake, we
would like to perform filtering as soon as possible so as to cut down on the
number of intermediate frames generated.
\end{quote}

\begin{quote}
\heading{\phantomsection\label{Exercise 4.78}Exercise 4.78:} Redesign the query language as a
nondeterministic program to be implemented using the evaluator of 
\link{Section 4.3}, rather than as a stream process.  In this approach, each query will
produce a single answer (rather than the stream of all answers) and the user
can type \code{try\-/again} to see more answers.  You should find that much of
the mechanism we built in this section is subsumed by nondeterministic search
and backtracking.  You will probably also find, however, that your new query
language has subtle differences in behavior from the one implemented here.  Can
you find examples that illustrate this difference?
\end{quote}

\begin{quote}
\heading{\phantomsection\label{Exercise 4.79}Exercise 4.79:} When we implemented the Lisp
evaluator in \link{Section 4.1}, we saw how to use local environments to avoid
name conflicts between the parameters of procedures.  For example, in
evaluating

\begin{scheme}
(define (square x) (* x x))
(define (sum-of-squares x y)
  (+ (square x) (square y)))
(sum-of-squares 3 4)
\end{scheme}

\noindent
there is no confusion between the \code{x} in \code{square} and the \code{x} in
\code{sum\-/of\-/squares}, because we evaluate the body of each procedure in an
environment that is specially constructed to contain bindings for the local
variables.  In the query system, we used a different strategy to avoid name
conflicts in applying rules.  Each time we apply a rule we rename the variables
with new names that are guaranteed to be unique.  The analogous strategy for
the Lisp evaluator would be to do away with local environments and simply
rename the variables in the body of a procedure each time we apply the
procedure.

Implement for the query language a rule-application method that uses
environments rather than renaming.  See if you can build on your environment
structure to create constructs in the query language for dealing with large
systems, such as the rule analog of block-structured procedures.  Can you
relate any of this to the problem of making deductions in a context (e.g., ``If
I supposed that \( P \) were true, then I would be able to deduce \( A \) and
\( B \).'') as a method of problem solving?  (This problem is open-ended.  A good
answer is probably worth a Ph.D.)
\end{quote}

\chapter{Computing with Register Machines}
\label{Chapter 5}

% \vspace{0.2em}

\begin{quote}
My aim is to show that the heavenly machine is not a kind of divine, live
being, but a kind of clockwork (and he who believes that a clock has soul
attributes the maker's glory to the work), insofar as nearly all the manifold
motions are caused by a most simple and material force, just as all motions of
the clock are caused by a single weight.

---Johannes Kepler (letter to Herwart von Hohenburg, 1605)
\end{quote}

% \vspace{1.0em}

\noindent
\lettrine[findent=0pt]{W}{e began this book} by studying processes and by describing processes in terms
of procedures written in Lisp.  To explain the meanings of these procedures, we
used a succession of models of evaluation: the substitution model of
\link{Chapter 1}, the environment model of \link{Chapter 3}, and the metacircular
evaluator of \link{Chapter 4}.  Our examination of the metacircular evaluator,
in particular, dispelled much of the mystery of how Lisp-like languages are
interpreted.  But even the metacircular evaluator leaves important questions
unanswered, because it fails to elucidate the mechanisms of control in a Lisp
system.  For instance, the evaluator does not explain how the evaluation of a
subexpression manages to return a value to the expression that uses this value,
nor does the evaluator explain how some recursive procedures generate iterative
processes (that is, are evaluated using constant space) whereas other recursive
procedures generate recursive processes.  These questions remain unanswered
because the metacircular evaluator is itself a Lisp program and hence inherits
the control structure of the underlying Lisp system.  In order to provide a
more complete description of the control structure of the Lisp evaluator, we
must work at a more primitive level than Lisp itself.

In this chapter we will describe processes in terms of the step-by-step
operation of a traditional computer.  Such a computer, or \newterm{register
machine}, sequentially executes \newterm{instructions} that manipulate the
contents of a fixed set of storage elements called \newterm{registers}.  A
typical register-machine instruction applies a primitive operation to the
contents of some registers and assigns the result to another register.  Our
descriptions of processes executed by register machines will look very much
like ``machine-language'' programs for traditional computers.  However, instead
of focusing on the machine language of any particular computer, we will examine
several Lisp procedures and design a specific register machine to execute each
procedure.  Thus, we will approach our task from the perspective of a hardware
architect rather than that of a machine-language computer programmer.  In
designing register machines, we will develop mechanisms for implementing
important programming constructs such as recursion.  We will also present a
language for describing designs for register machines.  In \link{Section 5.2} we
will implement a Lisp program that uses these descriptions to simulate the
machines we design.

Most of the primitive operations of our register machines are very simple.  For
example, an operation might add the numbers fetched from two registers,
producing a result to be stored into a third register.  Such an operation can
be performed by easily described hardware.  In order to deal with list
structure, however, we will also use the memory operations \code{car},
\code{cdr}, and \code{cons}, which require an elaborate storage-allocation
mechanism.  In \link{Section 5.3} we study their implementation in terms of more
elementary operations.

In \link{Section 5.4}, after we have accumulated experience formulating simple
procedures as register machines, we will design a machine that carries out the
algorithm described by the metacircular evaluator of \link{Section 4.1}.  This
will fill in the gap in our understanding of how Scheme expressions are
interpreted, by providing an explicit model for the mechanisms of control in
the evaluator.  In \link{Section 5.5} we will study a simple compiler that
translates Scheme programs into sequences of instructions that can be executed
directly with the registers and operations of the evaluator register machine.



\section{Designing Register Machines}
\label{Section 5.1}

To design a register machine, we must design its \newterm{data paths}
(registers and operations) and the \newterm{controller} that sequences these
operations.  To illustrate the design of a simple register machine, let us
examine Euclid's Algorithm, which is used to compute the greatest common
divisor (\acronym{GCD}) of two integers.  As we saw in \link{Section 1.2.5},
Euclid's Algorithm can be carried out by an iterative process, as specified by
the following procedure:

\begin{scheme}

(define (gcd a b)
  (if (= b 0)
      a
      (gcd b (remainder a b))))
\end{scheme}

\noindent
A machine to carry out this algorithm must keep track of two numbers, \( a \) and
\( b \), so let us assume that these numbers are stored in two registers with
those names.  The basic operations required are testing whether the contents of
register \code{b} is zero and computing the remainder of the contents of
register \code{a} divided by the contents of register \code{b}.  The remainder
operation is a complex process, but assume for the moment that we have a
primitive device that computes remainders.  On each cycle of the \acronym{GCD}
algorithm, the contents of register \code{a} must be replaced by the contents
of register \code{b}, and the contents of \code{b} must be replaced by the
remainder of the old contents of \code{a} divided by the old contents of
\code{b}.  It would be convenient if these replacements could be done
simultaneously, but in our model of register machines we will assume that only
one register can be assigned a new value at each step.  To accomplish the
replacements, our machine will use a third ``temporary'' register, which we
call \code{t}.  (First the remainder will be placed in \code{t}, then the
contents of \code{b} will be placed in \code{a}, and finally the remainder
stored in \code{t} will be placed in \code{b}.)

We can illustrate the registers and operations required for this machine by
using the data-path diagram shown in \link{Figure 5.1}.  In this diagram, the
registers (\code{a}, \code{b}, and \code{t}) are represented by rectangles.
Each way to assign a value to a register is indicated by an arrow with an
\code{X} behind the head, pointing from the source of data to the register.  We
can think of the \code{X} as a button that, when pushed, allows the value at
the source to ``flow'' into the designated register.  The label next to each
button is the name we will use to refer to the button.  The names are
arbitrary, and can be chosen to have mnemonic value (for example, \code{a←b}
denotes pushing the button that assigns the contents of register \code{b} to
register \code{a}).  The source of data for a register can be another register
(as in the \code{a←b} assignment), an operation result (as in the \code{t←r}
assignment), or a constant (a built-in value that cannot be changed,
represented in a data-path diagram by a triangle containing the constant).

\begin{figure}[tb]
\phantomsection\label{Figure 5.1}
\centering
\begin{comment}
\heading{Figure 5.1:} Data paths for a \acronym{GCD} machine.

\begin{example}
                              ___
+-----+          +-----+     /   \
|  a  |<--(X)----|  b  +--->|  =  |
+--+--+   a<-b   +-+---+     \___/
   |               |  ^        ^
   +------+   +----+  |        |
          |   |      (X) b<-t  |
       .--+---+--.    |       / \
        \  rem  /     |      / O \
         \_____/      |     +-----+
            |         |
           (X) t<-r   |
            |         |
            V         |
         +-----+      |
         |  t  +------+
         +-----+
\end{example}
\end{comment}
\includegraphics[width=58mm]{fig/chap5/Fig5.1a.pdf}
\par\bigskip
\noindent
\heading{Figure 5.1:} Data paths for a \acronym{GCD} machine.
\end{figure}

An operation that computes a value from constants and the contents of registers
is represented in a data-path diagram by a trapezoid containing a name for the
operation.  For example, the box marked \code{rem} in \link{Figure 5.1}
represents an operation that computes the remainder of the contents of the
registers \code{a} and \code{b} to which it is attached.  Arrows (without
buttons) point from the input registers and constants to the box, and arrows
connect the operation's output value to registers.  A test is represented by a
circle containing a name for the test.  For example, our \acronym{GCD} machine
has an operation that tests whether the contents of register \code{b} is zero.
A test also has arrows from its input registers and constants, but it has no
output arrows; its value is used by the controller rather than by the data
paths.  Overall, the data-path diagram shows the registers and operations that
are required for the machine and how they must be connected.  If we view the
arrows as wires and the \code{X} buttons as switches, the data-path diagram is
very like the wiring diagram for a machine that could be constructed from
electrical components.

\begin{figure}[tb]
\phantomsection\label{Figure 5.2}
\centering
\begin{comment}
\heading{Figure 5.2:} Controller for a \acronym{GCD} machine.

\begin{example}
     start
       |
       V
      / \ yes
+--->< = >-----> done
|     \ /
|      | no
|      V
|  +------+
|  | t<-r |
|  +---+--+
|      |
|      V
|  +------+
|  | a<-b |
|  +---+--+
|      |
|      V
|  +------+
+--+ b<-t |
   +------+
\end{example}
\end{comment}
\includegraphics[width=41mm]{fig/chap5/Fig5.2.pdf}
\par\bigskip
\noindent
\heading{Figure 5.2:} Controller for a \acronym{GCD} machine.
\end{figure}

In order for the data paths to actually compute \acronym{GCD}s, the buttons
must be pushed in the correct sequence.  We will describe this sequence in
terms of a controller diagram, as illustrated in \link{Figure 5.2}.  The
elements of the controller diagram indicate how the data-path components should
be operated.  The rectangular boxes in the controller diagram identify
data-path buttons to be pushed, and the arrows describe the sequencing from one
step to the next.  The diamond in the diagram represents a decision.  One of
the two sequencing arrows will be followed, depending on the value of the
data-path test identified in the diamond.  We can interpret the controller in
terms of a physical analogy: Think of the diagram as a maze in which a marble
is rolling.  When the marble rolls into a box, it pushes the data-path button
that is named by the box.  When the marble rolls into a decision node (such as
the test for \code{b} = 0), it leaves the node on the path determined by the
result of the indicated test.  Taken together, the data paths and the
controller completely describe a machine for computing \acronym{GCD}s.  We
start the controller (the rolling marble) at the place marked \code{start},
after placing numbers in registers \code{a} and \code{b}.  When the controller
reaches \code{done}, we will find the value of the \acronym{GCD} in register
\code{a}.

\begin{quote}
\heading{\phantomsection\label{Exercise 5.1}Exercise 5.1:} Design a register machine to
compute factorials using the iterative algorithm specified by the following
procedure.  Draw data-path and controller diagrams for this machine.

\begin{scheme}
(define (factorial n)
  (define (iter product counter)
    (if (> counter n)
        product
        (iter (* counter product)
              (+ counter 1))))
  (iter 1 1))
\end{scheme}
\end{quote}



\subsection{A Language for Describing Register Machines}
\label{Section 5.1.1}

Data-path and controller diagrams are adequate for representing simple machines
such as \acronym{GCD}, but they are unwieldy for describing large machines such
as a Lisp interpreter.  To make it possible to deal with complex machines, we
will create a language that presents, in textual form, all the information
given by the data-path and controller diagrams.  We will start with a notation
that directly mirrors the diagrams.

We define the data paths of a machine by describing the registers and the
operations.  To describe a register, we give it a name and specify the buttons
that control assignment to it.  We give each of these buttons a name and
specify the source of the data that enters the register under the button's
control.  (The source is a register, a constant, or an operation.)  To describe
an operation, we give it a name and specify its inputs (registers or
constants).

We define the controller of a machine as a sequence of \newterm{instructions}
together with \newterm{labels} that identify \newterm{entry points} in the
sequence. An instruction is one of the following:

\begin{itemize}

\item
The name of a data-path button to push to assign a value to a register.  (This
corresponds to a box in the controller diagram.)

\item
A \code{test} instruction, that performs a specified test.

\item
A conditional branch (\code{branch} instruction) to a location indicated by a
controller label, based on the result of the previous test.  (The test and
branch together correspond to a diamond in the controller diagram.)  If the
test is false, the controller should continue with the next instruction in the
sequence.  Otherwise, the controller should continue with the instruction after
the label.

\item
An unconditional branch (\code{goto} instruction) naming a controller label at
which to continue execution.

\end{itemize}

\noindent
The machine starts at the beginning of the controller instruction sequence and
stops when execution reaches the end of the sequence.  Except when a branch
changes the flow of control, instructions are executed in the order in which
they are listed.

\begin{quote}
\heading{\phantomsection\label{Figure 5.3}Figure 5.3:} \( \downarrow \) A specification of the \acronym{GCD}
machine.

\begin{scheme}
(data-paths
 (registers
  ((name a)
   (buttons ((name a<-b) (source (register b)))))
  ((name b)
   (buttons ((name b<-t) (source (register t)))))
  ((name t)
   (buttons ((name t<-r) (source (operation rem))))))
 (operations
  ((name rem) (inputs (register a) (register b)))
  ((name =) (inputs (register b) (constant 0)))))
(controller
 test-b                           ~\textrm{; label}~
   (test =)                       ~\textrm{; test}~
   (branch (label gcd-done))      ~\textrm{; conditional branch}~
   (t<-r)                         ~\textrm{; button push}~
   (a<-b)                         ~\textrm{; button push}~
   (b<-t)                         ~\textrm{; button push}~
   (goto (label test-b))          ~\textrm{; unconditional branch}~
 gcd-done)                        ~\textrm{; label}~
\end{scheme}

\end{quote}

\noindent
\link{Figure 5.3} shows the \acronym{GCD} machine described in this way.  This
example only hints at the generality of these descriptions, since the
\acronym{GCD} machine is a very simple case: Each register has only one button,
and each button and test is used only once in the controller.

Unfortunately, it is difficult to read such a description.  In order to
understand the controller instructions we must constantly refer back to the
definitions of the button names and the operation names, and to understand what
the buttons do we may have to refer to the definitions of the operation names.
We will thus transform our notation to combine the information from the
data-path and controller descriptions so that we see it all together.

To obtain this form of description, we will replace the arbitrary button and
operation names by the definitions of their behavior.  That is, instead of
saying (in the controller) ``Push button \code{t←r}'' and separately saying
(in the data paths) ``Button \code{t←r} assigns the value of the \code{rem}
operation to register \code{t}'' and ``The \code{rem} operation's inputs are
the contents of registers \code{a} and \code{b},'' we will say (in the
controller) ``Push the button that assigns to register \code{t} the value of
the \code{rem} operation on the contents of registers \code{a} and \code{b}.''
Similarly, instead of saying (in the controller) ``Perform the \code{=} test''
and separately saying (in the data paths) ``The \code{=} test operates on the
contents of register \code{b} and the constant 0,'' we will say ``Perform the
\code{=} test on the contents of register \code{b} and the constant 0.''  We
will omit the data-path description, leaving only the controller sequence.
Thus, the \acronym{GCD} machine is described as follows:

\begin{scheme}
(controller
 test-b
   (test (op =) (reg b) (const 0))
   (branch (label gcd-done))
   (assign t (op rem) (reg a) (reg b))
   (assign a (reg b))
   (assign b (reg t))
   (goto (label test-b))
 gcd-done)
\end{scheme}

\noindent
This form of description is easier to read than the kind illustrated in
\link{Figure 5.3}, but it also has disadvantages:

\begin{itemize}

\item
It is more verbose for large machines, because complete descriptions of the
data-path elements are repeated whenever the elements are mentioned in the
controller instruction sequence.  (This is not a problem in the \acronym{GCD}
example, because each operation and button is used only once.)  Moreover,
repeating the data-path descriptions obscures the actual data-path structure of
the machine; it is not obvious for a large machine how many registers,
operations, and buttons there are and how they are interconnected.

\item
Because the controller instructions in a machine definition look like Lisp
expressions, it is easy to forget that they are not arbitrary Lisp expressions.
They can notate only legal machine operations.  For example, operations can
operate directly only on constants and the contents of registers, not on the
results of other operations.

\end{itemize}

\noindent
In spite of these disadvantages, we will use this register-machine language
throughout this chapter, because we will be more concerned with understanding
controllers than with understanding the elements and connections in data paths.
We should keep in mind, however, that data-path design is crucial in designing
real machines.

\begin{quote}
\heading{\phantomsection\label{Exercise 5.2}Exercise 5.2:} Use the register-machine language
to describe the iterative factorial machine of \link{Exercise 5.1}.
\end{quote}

\subsubsection*{Actions}

Let us modify the \acronym{GCD} machine so that we can type in the numbers
whose \acronym{GCD} we want and get the answer printed at our terminal.  We
will not discuss how to make a machine that can read and print, but will assume
(as we do when we use \code{read} and \code{display} in Scheme) that they are
available as primitive operations.\footnote{This assumption glosses over a
great deal of complexity.  Usually a large portion of the implementation of a
Lisp system is dedicated to making reading and printing work.}

\begin{figure}[tb]
\phantomsection\label{Figure 5.4}
\centering
\begin{comment}
\heading{Figure 5.4:} A \acronym{GCD} machine that reads inputs and prints results.

\begin{example}
                   .--------.
                    \ read /
                     \____/
                       |
               +-------*------+
               |              |
        a<-rd (X)            (X) b<-rd
               |              |
               V              V           ___
            +-----+        +-----+       /   \
            |  a  |<--(X)--+  b  +----->|  =  |
            +-+-+-+  a<-b  +-+---+       \___/
              | |            |  ^          ^
           +--+ +----+    +--+  |          |
           |         |    |    (X) b<-t   / \
           V         V    V     |        / O \
      .---------.  .---------.  |       /_____\
--(X)->\ print /    \  rem  /   |
   P    \_____/      \_____/    |
                        |       |
                       (X) t<-r |
                        |       |
                        V       |
                     +-----+    |
                     |  t  +----+
                     +-----+
\end{example}

\begin{scheme}
 (controller
  gcd-loop
    (assign a (op read))
    (assign b (op read))
  test-b
    (test (op =) (reg b) (const 0))
    (branch (label gcd-done))
    (assign t (op rem) (reg a) (reg b))
    (assign a (reg b))
    (assign b (reg t))
    (goto (label test-b))
  gcd-done
    (perform (op print) (reg a))
    (goto (label gcd-loop)))
\end{scheme}

\end{comment}
\includegraphics[width=107mm]{fig/chap5/Fig5.4b.pdf}
\par\bigskip
\noindent
\heading{Figure 5.4:} A \acronym{GCD} machine that reads inputs and prints results. 
\end{figure}

\code{Read} is like the operations we have been using in that it produces a
value that can be stored in a register.  But \code{read} does not take inputs
from any registers; its value depends on something that happens outside the
parts of the machine we are designing.  We will allow our machine's operations
to have such behavior, and thus will draw and notate the use of \code{read}
just as we do any other operation that computes a value.

\code{Print}, on the other hand, differs from the operations we have been using
in a fundamental way: It does not produce an output value to be stored in a
register.  Though it has an effect, this effect is not on a part of the machine
we are designing.  We will refer to this kind of operation as an
\newterm{action}.  We will represent an action in a data-path diagram just as
we represent an operation that computes a value---as a trapezoid that contains
the name of the action.  Arrows point to the action box from any inputs
(registers or constants).  We also associate a button with the action.  Pushing
the button makes the action happen.  To make a controller push an action button
we use a new kind of instruction called \code{perform}.  Thus, the action of
printing the contents of register \code{a} is represented in a controller
sequence by the instruction

\begin{scheme}
(perform (op print) (reg a))
\end{scheme}

\noindent
\link{Figure 5.4} shows the data paths and controller for the new \acronym{GCD}
machine.  Instead of having the machine stop after printing the answer, we have
made it start over, so that it repeatedly reads a pair of numbers, computes
their \acronym{GCD}, and prints the result.  This structure is like the driver
loops we used in the interpreters of \link{Chapter 4}.

\subsection{Abstraction in Machine Design}
\label{Section 5.1.2}

We will often define a machine to include ``primitive'' operations that are
actually very complex.  For example, in \link{Section 5.4} and \link{Section 5.5} we
will treat Scheme's environment manipulations as primitive.  Such abstraction
is valuable because it allows us to ignore the details of parts of a machine so
that we can concentrate on other aspects of the design.  The fact that we have
swept a lot of complexity under the rug, however, does not mean that a machine
design is unrealistic.  We can always replace the complex ``primitives'' by
simpler primitive operations.

Consider the \acronym{GCD} machine. The machine has an instruction that
computes the remainder of the contents of registers \code{a} and \code{b} and
assigns the result to register \code{t}.  If we want to construct the
\acronym{GCD} machine without using a primitive remainder operation, we must
specify how to compute remainders in terms of simpler operations, such as
subtraction.  Indeed, we can write a Scheme procedure that finds remainders in
this way:

\begin{scheme}


(define (remainder n d)
  (if (< n d) 
      n 
      (remainder (- n d) d)))
\end{scheme}

\begin{figure}[tp]
\phantomsection\label{Figure 5.5}
\centering
\begin{comment}
\heading{Figure 5.5:} Data paths and controller for the elaborated \acronym{GCD} machine.

\begin{example}
                                    ___
+-----+         +-----+            /   \
|  a  |<--(X)---+  b  +-------*-->|  =  |
+--+--+   a<-b  +-+---+       |    \___/
   |              |  ^        |
  (X) t<-a        |  |        |
   |              | (X) b<-t  |
   V              |  |       _V_
+-----+           |  |      /   \
|  t  +-------*---|--*-----|  <  |
+-----+       |   |         \___/
   ^          V   V
   |        ---------
  (X) t<-d   \  -  /
   |          --+--
   |            |
   +------------+


   start
     |
     V
    / \ yes            +-------+
+->< = >----> done     | t<-d  |<--+
|   \ /                +---+---+   |
|    | no                  |       |
|    |                     V       |
|    |   +------+         / \ no   |
|    +-->| t<-a +------->< < >-----+
|        +------+         \ /
|                          | yes
|      +-------------------+
|      V
|  +-------+
|  | a<-b  |
|  +---+---+
|      |
|      V
|  +-------+
+--+ b<-t  |
   +-------+
\end{example}
\end{comment}
\includegraphics[width=67mm]{fig/chap5/Fig5.5a.pdf}
\begin{quote}
\heading{Figure 5.5:} Data paths and controller for the elaborated \acronym{GCD} machine.
\end{quote}
\end{figure}

We can thus replace the remainder operation in the \acronym{GCD} machine's data
paths with a subtraction operation and a comparison test.  \link{Figure 5.5}
shows the data paths and controller for the elaborated machine.  The
instruction

\begin{scheme}
(assign t (op rem) (reg a) (reg b))
\end{scheme}

\noindent
in the \acronym{GCD} controller definition is replaced by a sequence of
instructions that contains a loop, as shown in \link{Figure 5.6}.

\begin{quote}
\heading{\phantomsection\label{Figure 5.6}\mbox{Figure 5.6:}} \( \downarrow \) Controller instruction sequence for the \acronym{GCD} machine in \link{Figure 5.5}.

\begin{scheme}
(controller test-b
              (test (op =) (reg b) (const 0))
              (branch (label gcd-done))
              (assign t (reg a))
            rem-loop
              (test (op <) (reg t) (reg b))
              (branch (label rem-done))
              (assign t (op -) (reg t) (reg b))
              (goto (label rem-loop))
            rem-done
              (assign a (reg b))
              (assign b (reg t))
              (goto (label test-b))
            gcd-done)
\end{scheme}
\end{quote}

\begin{quote}
\heading{\phantomsection\label{Exercise 5.3}Exercise 5.3:} Design a machine to compute square
roots using Newton's method, as described in \link{Section 1.1.7}:

\begin{scheme}
(define (sqrt x)
  (define (good-enough? guess)
    (< (abs (- (square guess) x)) 0.001))
  (define (improve guess)
    (average guess (/ x guess)))
  (define (sqrt-iter guess)
    (if (good-enough? guess)
        guess
        (sqrt-iter (improve guess))))
  (sqrt-iter 1.0))
\end{scheme}

Begin by assuming that \code{good\-/enough?} and \code{improve} operations are
available as primitives.  Then show how to expand these in terms of arithmetic
operations.  Describe each version of the \code{sqrt} machine design by drawing
a data-path diagram and writing a controller definition in the register-machine
language.
\end{quote}

\subsection{Subroutines}
\label{Section 5.1.3}

When designing a machine to perform a computation, we would often prefer to
arrange for components to be shared by different parts of the computation
rather than duplicate the components.  Consider a machine that includes two
\acronym{GCD} computations---one that finds the \acronym{GCD} of the contents
of registers \code{a} and \code{b} and one that finds the \acronym{GCD} of the
contents of registers \code{c} and \code{d}.  We might start by assuming we
have a primitive \code{gcd} operation, then expand the two instances of
\code{gcd} in terms of more primitive operations.  \link{Figure 5.7} shows just
the \acronym{GCD} portions of the resulting machine's data paths, without
showing how they connect to the rest of the machine.  The figure also shows the
corresponding portions of the machine's controller sequence.

\begin{figure}[tb]
\phantomsection\label{Figure 5.7}
\centering
\begin{comment}
\heading{Figure 5.7:} Portions of the data paths and controller sequence for a machine with two \acronym{GCD} computations.

\begin{example}
                            ___                                 ___  
+-----+        +-----+     /   \    +-----+        +-----+     /   \ 
|  a  |<-(X)---+  b  |--->|  =  |   |  c  |<-(X)---+  d  |--->|  =  |
+--+--+  a<-b  ++----+     \___/    +--+--+  c<-d  ++----+     \___/ 
   |            |  ^         ^         |            |  ^         ^   
   `----.   .---'  |         |         `----.   .---'  |         |   
        V   V     (X) b<-t   |              V   V     (X) d<-t   |   
       -------     |        / \            -------     |        / \  
       \ rem /     |       /_0_\           \ rem /     |       /_0_\ 
        --+--      |                        --+--      |             
          |        |                          |        |             
         (X) t<-r  |                         (X) s<-r  |             
          |        |                          |        |             
          V        |                          V        |             
       +-----+     |                       +-----+     |             
       |  t  +-----'                       |  s  +-----'             
       +-----+                             +-----+                   
\end{example}

\begin{scheme}
gcd-1
 (test (op =) (reg b) (const 0))
 (branch (label after-gcd-1))
 (assign t (op rem) (reg a) (reg b))
 (assign a (reg b))
 (assign b (reg t))
 (goto (label gcd-1))
after-gcd-1
   ~\( \dots \)~
gcd-2
 (test (op =) (reg d) (const 0))
 (branch (label after-gcd-2))
 (assign s (op rem) (reg c) (reg d))
 (assign c (reg d))
 (assign d (reg s))
 (goto (label gcd-2))
after-gcd-2
\end{scheme}

\end{comment}
\includegraphics[width=105mm]{fig/chap5/Fig5.7b.pdf}
\begin{quote}
\heading{Figure 5.7:} Portions of the data paths and controller sequence for a \mbox{machine} with two \acronym{GCD} computations.
\end{quote}
\end{figure}

This machine has two remainder operation boxes and two boxes for testing
equality.  If the duplicated components are complicated, as is the remainder
box, this will not be an economical way to build the machine.  We can avoid
duplicating the data-path components by using the same components for both
\acronym{GCD} computations, provided that doing so will not affect the rest of
the larger machine's computation.  If the values in registers \code{a} and
\code{b} are not needed by the time the controller gets to \code{gcd\-/2} (or if
these values can be moved to other registers for safekeeping), we can change
the machine so that it uses registers \code{a} and \code{b}, rather than
registers \code{c} and \code{d}, in computing the second \acronym{GCD} as well
as the first.  If we do this, we obtain the controller sequence shown in
\link{Figure 5.8}.

\begin{quote}
\heading{\phantomsection\label{Figure 5.8}Figure 5.8:} \( \downarrow \) Portions of the 
controller sequence for a machine that uses the same data-path components 
for two different \acronym{GCD} computations.
\begin{scheme}
gcd-1
 (test (op =) (reg b) (const 0))
 (branch (label after-gcd-1))
 (assign t (op rem) (reg a) (reg b))
 (assign a (reg b))
 (assign b (reg t))
 (goto (label gcd-1))
after-gcd-1
  ~\( \dots \)~
gcd-2
 (test (op =) (reg b) (const 0))
 (branch (label after-gcd-2))
 (assign t (op rem) (reg a) (reg b))
 (assign a (reg b))
 (assign b (reg t))
 (goto (label gcd-2))
after-gcd-2
\end{scheme}
\end{quote}

\noindent
We have removed the duplicate data-path components (so that the data paths are
again as in \link{Figure 5.1}), but the controller now has two \acronym{GCD}
sequences that differ only in their entry-point labels.  It would be better to
replace these two sequences by branches to a single sequence---a \code{gcd}
\newterm{subroutine}---at the end of which we branch back to the correct place
in the main instruction sequence.  We can accomplish this as follows: Before
branching to \code{gcd}, we place a distinguishing value (such as 0 or 1) into
a special register, \code{continue}.  At the end of the \code{gcd} subroutine
we return either to \code{after\-/gcd\-/1} or to \code{after\-/gcd\-/2}, depending on
the value of the \code{continue} register. \link{Figure 5.9} shows the relevant
portion of the resulting controller sequence, which includes only a single copy
of the \code{gcd} instructions.

\begin{quote}
\heading{\phantomsection\label{Figure 5.9}Figure 5.9:} \( \downarrow \) Using a \code{continue} 
register to avoid the duplicate controller sequence in \link{Figure 5.8}.
\begin{scheme}
gcd
 (test (op =) (reg b) (const 0))
 (branch (label gcd-done))
 (assign t (op rem) (reg a) (reg b))
 (assign a (reg b))
 (assign b (reg t))
 (goto (label gcd))
gcd-done
 (test (op =) (reg continue) (const 0))
 (branch (label after-gcd-1))
 (goto (label after-gcd-2))
  ~\( \dots \)~
 ~\textrm{;; Before branching to \code{gcd} from the first place where}~
 ~\textrm{;; it is needed, we place 0 in the \code{continue} register}~
 (assign continue (const 0))
 (goto (label gcd))
after-gcd-1
  ~\( \dots \)~

 ~\textrm{;; Before the second use of \code{gcd}, we place 1}~
 ~\textrm{;; in the \code{continue} register}~
 (assign continue (const 1))
 (goto (label gcd))
after-gcd-2
\end{scheme}
\end{quote}

\noindent
This is a reasonable approach for handling small problems, but it would be
awkward if there were many instances of \acronym{GCD} computations in the
controller sequence.  To decide where to continue executing after the
\code{gcd} subroutine, we would need tests in the data paths and branch
instructions in the controller for all the places that use \code{gcd}.  A more
powerful method for implementing subroutines is to have the \code{continue}
register hold the label of the entry point in the controller sequence at which
execution should continue when the subroutine is finished.  Implementing this
strategy requires a new kind of connection between the data paths and the
controller of a register machine: There must be a way to assign to a register a
label in the controller sequence in such a way that this value can be fetched
from the register and used to continue execution at the designated entry point.

To reflect this ability, we will extend the \code{assign} instruction of the
register-machine language to allow a register to be assigned as value a label
from the controller sequence (as a special kind of constant).  We will also
extend the \code{goto} instruction to allow execution to continue at the entry
point described by the contents of a register rather than only at an entry
point described by a constant label.  Using these new constructs we can
terminate the \code{gcd} subroutine with a branch to the location stored in the
\code{continue} register.  This leads to the controller sequence shown in
\link{Figure 5.10}.

\begin{quote}
\heading{\phantomsection\label{Figure 5.10}Figure 5.10:} \( \downarrow \) Assigning labels to the
\code{continue} register simplifies and generalizes the strategy shown in
\link{Figure 5.9}.
\begin{scheme}
gcd
 (test (op =) (reg b) (const 0))
 (branch (label gcd-done))
 (assign t (op rem) (reg a) (reg b))
 (assign a (reg b))
 (assign b (reg t))
 (goto (label gcd))
gcd-done
 (goto (reg continue))
   ~\( \dots \)~
 ~\textrm{;; Before calling \code{gcd}, we assign to \code{continue}}~
 ~\textrm{;; the label to which \code{gcd} should return.}~
 (assign continue (label after-gcd-1))
 (goto (label gcd))
after-gcd-1
   ~\( \dots \)~
 ~\textrm{;; Here is the second call to \code{gcd},}~
 ~\textrm{;; with a different continuation.}~
 (assign continue (label after-gcd-2))
 (goto (label gcd))
after-gcd-2
\end{scheme}
\end{quote}

\noindent
A machine with more than one subroutine could use multiple continuation
registers (e.g., \code{gcd\-/continue}, \code{factorial\-/continue}) or we could
have all subroutines share a single \code{continue} register.  Sharing is more
economical, but we must be careful if we have a subroutine (\code{sub1}) that
calls another subroutine (\code{sub2}).  Unless \code{sub1} saves the contents
of \code{continue} in some other register before setting up \code{continue} for
the call to \code{sub2}, \code{sub1} will not know where to go when it is
finished.  The mechanism developed in the next section to handle recursion also
provides a better solution to this problem of nested subroutine calls.

\subsection{Using a Stack to Implement Recursion}
\label{Section 5.1.4}

With the ideas illustrated so far, we can implement any iterative process by
specifying a register machine that has a register corresponding to each state
variable of the process.  The machine repeatedly executes a controller loop,
changing the contents of the registers, until some termination condition is
satisfied.  At each point in the controller sequence, the state of the machine
(representing the state of the iterative process) is completely determined by
the contents of the registers (the values of the state variables).

Implementing recursive processes, however, requires an additional mechanism.
Consider the following recursive method for computing factorials, which we
first examined in \link{Section 1.2.1}:

\begin{scheme}
(define (factorial n)
  (if (= n 1) 1 (* (factorial (- n 1)) n)))
\end{scheme}

\noindent
As we see from the procedure, computing \( n! \) requires computing \( (n - 1)! \).
Our \acronym{GCD} machine, modeled on the procedure

\begin{scheme}
(define (gcd a b)
  (if (= b 0) a (gcd b (remainder a b))))
\end{scheme}

\noindent
similarly had to compute another \acronym{GCD}.  But there is an important
difference between the \code{gcd} procedure, which reduces the original
computation to a new \acronym{GCD} computation, and \code{factorial}, which
requires computing another factorial as a subproblem.  In \acronym{GCD}, the
answer to the new \acronym{GCD} computation is the answer to the original
problem.  To compute the next \acronym{GCD}, we simply place the new arguments
in the input registers of the \acronym{GCD} machine and reuse the machine's
data paths by executing the same controller sequence.  When the machine is
finished solving the final \acronym{GCD} problem, it has completed the entire
computation.

In the case of factorial (or any recursive process) the answer to the new
factorial subproblem is not the answer to the original problem.  The value
obtained for \( (n - 1)! \) must be multiplied by \( n \) to get the final answer.
If we try to imitate the \acronym{GCD} design, and solve the factorial
subproblem by decrementing the \code{n} register and rerunning the factorial
machine, we will no longer have available the old value of \code{n} by which to
multiply the result.  We thus need a second factorial machine to work on the
subproblem.  This second factorial computation itself has a factorial
subproblem, which requires a third factorial machine, and so on.  Since each
factorial machine contains another factorial machine within it, the total
machine contains an infinite nest of similar machines and hence cannot be
constructed from a fixed, finite number of parts.

Nevertheless, we can implement the factorial process as a register machine if
we can arrange to use the same components for each nested instance of the
machine.  Specifically, the machine that computes \( n! \)  should use the same
components to work on the subproblem of computing \( (n - 1)! \), on the
subproblem for \( (n - 2)! \), and so on.  This is plausible because, although
the factorial process dictates that an unbounded number of copies of the same
machine are needed to perform a computation, only one of these copies needs to
be active at any given time.  When the machine encounters a recursive
subproblem, it can suspend work on the main problem, reuse the same physical
parts to work on the subproblem, then continue the suspended computation.

\enlargethispage{\baselineskip}

In the subproblem, the contents of the registers will be different than they
were in the main problem. (In this case the \code{n} register is decremented.)
In order to be able to continue the suspended computation, the machine must
save the contents of any registers that will be needed after the subproblem is
solved so that these can be restored to continue the suspended computation.  In
the case of factorial, we will save the old value of \code{n}, to be restored
when we are finished computing the factorial of the decremented \code{n}
register.\footnote{One might argue that we don't need to save the old \code{n};
after we decrement it and solve the subproblem, we could simply increment it to
recover the old value.  Although this strategy works for factorial, it cannot
work in general, since the old value of a register cannot always be computed
from the new one.}

Since there is no \emph{a priori} limit on the depth of nested recursive calls,
we may need to save an arbitrary number of register values.  These values must
be restored in the reverse of the order in which they were saved, since in a
nest of recursions the last subproblem to be entered is the first to be
finished.  This dictates the use of a \newterm{stack}, or ``last in, first
out'' data structure, to save register values.  We can extend the
register-machine language to include a stack by adding two kinds of
instructions: Values are placed on the stack using a \code{save} instruction
and restored from the stack using a \code{restore} instruction.  After a
sequence of values has been \code{save}d on the stack, a sequence of
\code{restore}s will retrieve these values in reverse order.\footnote{In
\link{Section 5.3} we will see how to implement a stack in terms of more
primitive operations.}

With the aid of the stack, we can reuse a single copy of the factorial
machine's data paths for each factorial subproblem.  There is a similar design
issue in reusing the controller sequence that operates the data paths.  To
reexecute the factorial computation, the controller cannot simply loop back to
the beginning, as with an iterative process, because after solving the 
\( (n - 1)! \) subproblem the machine must still multiply the result by \( n \).  The
controller must suspend its computation of \( n! \), solve the \( (n - 1)! \)
subproblem, then continue its computation of \( n! \).  This view of the
factorial computation suggests the use of the subroutine mechanism described in
\link{Section 5.1.3}, which has the controller use a \code{continue} register to
transfer to the part of the sequence that solves a subproblem and then continue
where it left off on the main problem.  We can thus make a factorial subroutine
that returns to the entry point stored in the \code{continue} register.  Around
each subroutine call, we save and restore \code{continue} just as we do the
\code{n} register, since each ``level'' of the factorial computation will use
the same \code{continue} register.  That is, the factorial subroutine must put
a new value in \code{continue} when it calls itself for a subproblem, but it
will need the old value in order to return to the place that called it to solve
a subproblem.

\link{Figure 5.11} shows the data paths and controller for a machine that
implements the recursive \code{factorial} procedure.  The machine has a stack
and three registers, called \code{n}, \code{val}, and \code{continue}.  To
simplify the data-path diagram, we have not named the register-assignment
buttons, only the stack-operation buttons (\code{sc} and \code{sn} to save
registers, \code{rc} and \code{rn} to restore registers).  To operate the
machine, we put in register \code{n} the number whose factorial we wish to
compute and start the machine.  When the machine reaches \code{fact\-/done}, the
computation is finished and the answer will be found in the \code{val}
register.  In the controller sequence, \code{n} and \code{continue} are saved
before each recursive call and restored upon return from the call.  Returning
from a call is accomplished by branching to the location stored in
\code{continue}.  \code{Continue} is initialized when the machine starts so
that the last return will go to \code{fact\-/done}.  The \code{val} register,
which holds the result of the factorial computation, is not saved before the
recursive call, because the old contents of \code{val} is not useful after the
subroutine returns.  Only the new value, which is the value produced by the
subcomputation, is needed.

\begin{figure}[tp]
\phantomsection\label{Figure 5.11}
\centering
\begin{comment}
\heading{Figure 5.11:} A recursive factorial machine.

\begin{example}
                             ___
                            /   \
    +----------*-----------|  =  |
    |          |            \___/
   (X)         |              ^
    |          |              |
    V          |          +---+---+   sn    +-------+
+-------+      |          |       +---(X)-->|       |
|  val  |<-(X)-|----------+   n   |         | stack |
+-----+-+      |          |       |<--(X)---+       |
  ^   |        |          +-------+   rn    +-+-----+
  |   |        |            ^                 |   ^
 (X)  |        |            |                 |   |
  |   |   +----|--------*  (X)                |  (X) sc
  |   |   |    |        |   |             rc (X)  |
  |   |   |    *----.   |   |                 |   |
  |   V   V    |    V   V   |                 V   |
  |  -------   |   -------  |              +------+-+
  |  \  *  /   |   \  -  /  |              |continue+--> controller
  |   --+--    |    --+--   |              +--------+
  |     |      |      |     |               ^      ^
  +-----+      |      +-----+               |      |
               |                           (X)    (X)
               |                            |      |
              / \                   after- / \    / \  fact-
             /_1_\                  fact  /___\  /___\ done
\end{example}

\begin{smallscheme}
(controller
   (assign continue (label fact-done))     ~\textrm{; set up final return address}~
 fact-loop
   (test (op =) (reg n) (const 1))
   (branch (label base-case))
   ~\textrm{;; Set up for the recursive call by saving \code{n} and \code{continue}.}~
   ~\textrm{;; Set up \code{continue} so that the computation will continue}~
   ~\textrm{;; at \code{after\-/fact} when the subroutine returns.}~
   (save continue)
   (save n)
   (assign n (op -) (reg n) (const 1))
   (assign continue (label after-fact))
   (goto (label fact-loop))
 after-fact
   (restore n)
   (restore continue)
   (assign val (op *) (reg n) (reg val))   ~\textrm{; \code{val} now contains}~ ~\( n(n-1)! \)~
   (goto (reg continue))                   ~\textrm{; return to caller}~
 base-case
   (assign val (const 1))                  ~\textrm{; base case: }~1! = 1
   (goto (reg continue))                   ~\textrm{; return to caller}~
 fact-done)
\end{smallscheme}

\end{comment}
\includegraphics[width=106mm]{fig/chap5/Fig5.11a.pdf}
\par\bigskip
\noindent
\heading{Figure 5.11:} A recursive factorial machine.
\end{figure}

Although in principle the factorial computation requires an infinite machine,
the machine in \link{Figure 5.11} is actually finite except for the stack, which
is potentially unbounded.  Any particular physical implementation of a stack,
however, will be of finite size, and this will limit the depth of recursive
calls that can be handled by the machine.  This implementation of factorial
illustrates the general strategy for realizing recursive algorithms as ordinary
register machines augmented by stacks.  When a recursive subproblem is
encountered, we save on the stack the registers whose current values will be
required after the subproblem is solved, solve the recursive subproblem, then
restore the saved registers and continue execution on the main problem.  The
\code{continue} register must always be saved.  Whether there are other
registers that need to be saved depends on the particular machine, since not
all recursive computations need the original values of registers that are
modified during solution of the subproblem (see \link{Exercise 5.4}).

\subsubsection*{A double recursion}

Let us examine a more complex recursive process, the tree-recursive computation
of the Fibonacci numbers, which we introduced in \link{Section 1.2.2}:

\begin{scheme}
(define (fib n)
  (if (< n 2)
      n
      (+ (fib (- n 1)) (fib (- n 2)))))
\end{scheme}

\noindent
Just as with factorial, we can implement the recursive Fibonacci computation as
a register machine with registers \code{n}, \code{val}, and \code{continue}.
The machine is more complex than the one for factorial, because there are two
places in the controller sequence where we need to perform recursive
calls---once to compute \( {\rm Fib}(n - 1) \) and once to compute \( {\rm Fib}(n - 2) \).  To
set up for each of these calls, we save the registers whose values will be
needed later, set the \code{n} register to the number whose Fib we need to
compute recursively (\( n - 1 \) or \( n - 2 \)), and assign to \code{continue} the
entry point in the main sequence to which to return (\code{afterfib\-/n\-/1} or
\code{afterfib\-/n\-/2}, respectively).  We then go to \code{fib\-/loop}.  When we
return from the recursive call, the answer is in \code{val}.  \link{Figure 5.12}
shows the controller sequence for this machine.

\begin{quote}
\heading{\phantomsection\label{Figure 5.12}Figure 5.12:} \( \downarrow \) Controller for a machine to compute
Fibonacci numbers.

\begin{scheme}
(controller
   (assign continue (label fib-done))
 fib-loop
   (test (op <) (reg n) (const 2))
   (branch (label immediate-answer))
   ~\textrm{;; set up to compute Fib\( (n-1) \)}~
   (save continue)
   (assign continue (label afterfib-n-1))
   (save n)                 ~\textrm{; save old value of \code{n}}~
   (assign n (op -) (reg n) (const 1)) ~\textrm{; clobber \code{n} to \code{n\-/1}}~
   (goto (label fib-loop))  ~\textrm{; perform recursive call}~
 afterfib-n-1     ~\textrm{; upon return, \code{val} contains Fib\( (n-1) \)}~
   (restore n)
   (restore continue)
   ~\textrm{;; set up to compute Fib\( (n - 2) \)}~
   (assign n (op -) (reg n) (const 2))
   (save continue)
   (assign continue (label afterfib-n-2))
   (save val)               ~\textrm{; save Fib\( (n-1) \)}~
   (goto (label fib-loop))
 afterfib-n-2     ~\textrm{; upon return, \code{val} contains Fib\( (n-2) \)}~
   (assign n (reg val))     ~\textrm{; \code{n} now contains Fib\( (n-2) \)}~
   (restore val)            ~\textrm{; \code{val} now contains Fib\( (n-1) \)}~
   (restore continue)
   (assign val              ~\textrm{; Fib\( (n-1) \) + Fib\( (n-2) \)}~
           (op +) (reg val) (reg n))
   (goto (reg continue))    ~\textrm{; return to caller, answer is in \code{val}}~
 immediate-answer
   (assign val (reg n))     ~\textrm{; base case: Fib\( (n) = n \)}~
   (goto (reg continue))
 fib-done)
\end{scheme}

\end{quote}

\begin{quote}
\heading{\phantomsection\label{Exercise 5.4}Exercise 5.4:} Specify register machines that
implement each of the following procedures.  For each machine, write a
controller instruction sequence and draw a diagram showing the data paths.

\begin{enumerate}[a.]

\item
Recursive exponentiation:

\begin{scheme}
(define (expt b n)
  (if (= n 0)
      1
      (* b (expt b (- n 1)))))
\end{scheme}

\item
Iterative exponentiation:

\begin{scheme}
(define (expt b n)
  (define (expt-iter counter product)
    (if (= counter 0)
        product
        (expt-iter (- counter 1)
                   (* b product))))
  (expt-iter n 1))
\end{scheme}

\end{enumerate}
\end{quote}

\begin{quote}
\heading{\phantomsection\label{Exercise 5.5}Exercise 5.5:} Hand-simulate the factorial and
Fibonacci machines, using some nontrivial input (requiring execution of at
least one recursive call).  Show the contents of the stack at each significant
point in the execution.
\end{quote}

\begin{quote}
\heading{\phantomsection\label{Exercise 5.6}Exercise 5.6:} Ben Bitdiddle observes that the
Fibonacci machine's controller sequence has an extra \code{save} and an extra
\code{restore}, which can be removed to make a faster machine.  Where are these
instructions?
\end{quote}

\subsection{Instruction Summary}
\label{Section 5.1.5}

A controller instruction in our register-machine language has one of the
following forms, where each \( \langle \)\( input_i \)\( \rangle \) is either \code{(reg
\( \langle \)\var{register\-/name}\( \rangle \))} or \code{(const \( \langle \)\var{constant\-/value}\( \rangle \))}.  These
instructions were introduced in \link{Section 5.1.1}:

\begin{scheme}
(assign ~\( \dark \langle \)~~\var{\dark register-name}~~\( \dark \rangle \)~ (reg ~\( \dark \langle \)~~\var{\dark register-name}~~\( \dark \rangle \)~))
(assign ~\( \dark \langle \)~~\var{\dark register-name}~~\( \dark \rangle \)~ (const ~\( \dark \langle \)~~\var{\dark constant-value}~~\( \dark \rangle \)~))
(assign ~\( \dark \langle \)~~\var{\dark register-name}~~\( \dark \rangle \)~ 
        (op ~\( \dark \langle \)~~\var{\dark operation-name}~~\( \dark \rangle \)~) 
        ~\( \dark \langle \)~~\( \dark input_1 \)~~\( \dark \rangle \)~ ~\( \dots \)~ ~\( \dark \langle \)~~\( \dark input_n \)~~\( \dark \rangle \)~)
(perform (op ~\( \dark \langle \)~~\var{\dark operation-name}~~\( \dark \rangle \)~) ~\( \dark \langle \)~~\( \dark input_1 \)~~\( \dark \rangle \)~ ~\( \dots \)~ ~\( \dark \langle \)~~\( \dark input_n \)~~\( \dark \rangle \)~)
(test (op ~\( \dark \langle \)~~\var{\dark operation-name}~~\( \dark \rangle \)~) ~\( \dark \langle \)~~\( \dark input_1 \)~~\( \dark \rangle \)~ ~\( \dots \)~ ~\( \dark \langle \)~~\( \dark input_n \)~~\( \dark \rangle \)~)
(branch (label ~\( \dark \langle \)~~\var{\dark label-name}~~\( \dark \rangle \)~))
(goto (label ~\( \dark \langle \)~~\var{\dark label-name}~~\( \dark \rangle \)~))
\end{scheme}

\noindent
The use of registers to hold labels was introduced in \link{Section 5.1.3}:

\begin{scheme}
(assign ~\( \dark \langle \)~~\var{\dark register-name}~~\( \dark \rangle \)~ (label ~\( \dark \langle \)~~\var{\dark label-name}~~\( \dark \rangle \)~))
(goto (reg ~\( \dark \langle \)~~\var{\dark register-name}~~\( \dark \rangle \)~))
\end{scheme}

\noindent
Instructions to use the stack were introduced in \link{Section 5.1.4}:

\begin{scheme}
(save ~\( \dark \langle \)~~\var{\dark register-name}~~\( \dark \rangle \)~)
(restore ~\( \dark \langle \)~~\var{\dark register-name}~~\( \dark \rangle \)~)
\end{scheme}

\noindent
The only kind of \( \langle \)\var{constant-value}\( \rangle \) we have seen 
so far is a number, but later we will use strings, symbols, and lists.
For example,
\begin{scheme}
(const "abc") ~\textrm{is the string}~ "abc",
(const abc) ~\textrm{is the symbol}~ abc,
(const (a b c)) ~\textrm{is the list}~ (a b c),
~\textrm{and}~ (const ()) ~\textrm{is the empty list.}~
\end{scheme}

\section{A Register-Machine Simulator}
\label{Section 5.2}

In order to gain a good understanding of the design of register machines, we
must test the machines we design to see if they perform as expected.  One way
to test a design is to hand-simulate the operation of the controller, as in
\link{Exercise 5.5}.  But this is extremely tedious for all but the simplest
machines.  In this section we construct a simulator for machines described in
the register-machine language.  The simulator is a Scheme program with four
interface procedures.  The first uses a description of a register machine to
construct a model of the machine (a data structure whose parts correspond to
the parts of the machine to be simulated), and the other three allow us to
simulate the machine by manipulating the model:

\begin{quote}

\begin{scheme}
(make-machine ~\( \dark \langle \)~~\var{\dark register-names}~~\( \dark \rangle \)~ ~\( \dark \langle \)~~\var{\dark operations}~~\( \dark \rangle \)~ ~\( \dark \langle \)~~\var{\dark controller}~~\( \dark \rangle \)~)
\end{scheme}

\noindent
constructs and returns a model of the machine with the given registers,
operations, and controller.

\begin{scheme}
(set-register-contents! ~\( \dark \langle\kern0.08em \)~~\var{\dark machine-model}~~\( \dark \rangle \)~ 
                        ~\( \dark \langle \)~~\var{\dark register-name}~~\( \dark \rangle \)~ 
                        ~\( \dark \langle \)~~\var{\dark value}~~\( \dark \rangle \)~)
\end{scheme}

\noindent
stores a value in a simulated register in the given machine.

\begin{scheme}
(get-register-contents ~\( \dark \langle\kern0.08em \)~~\var{\dark machine-model}~~\( \dark \rangle \)~ ~\( \dark \langle \)~~\var{\dark register-name}~~\( \dark \rangle \)~)
\end{scheme}

\noindent
returns the contents of a simulated register in the given machine.

\begin{scheme}
(start ~\( \dark \langle\kern0.08em \)~~\var{\dark machine-model}~~\( \dark \rangle \)~)
\end{scheme}

\noindent
simulates the execution of the given machine, starting from the beginning of
the controller sequence and stopping when it reaches the end of the sequence.
\end{quote}

\noindent
As an example of how these procedures are used, we can define
\code{gcd\-/machine} to be a model of the \acronym{GCD} machine of 
\link{Section 5.1.1} as follows:

\begin{scheme}
(define gcd-machine
  (make-machine
   '(a b t)
   (list (list 'rem remainder) (list '= =))
   '(test-b (test (op =) (reg b) (const 0))
            (branch (label gcd-done))
            (assign t (op rem) (reg a) (reg b))
            (assign a (reg b))
            (assign b (reg t))
            (goto (label test-b))
            gcd-done)))
\end{scheme}

\noindent
The first argument to \code{make\-/machine} is a list of register names.  The
next argument is a table (a list of two-element lists) that pairs each
operation name with a Scheme procedure that implements the operation (that is,
produces the same output value given the same input values).  The last argument
specifies the controller as a list of labels and machine instructions, as in
\link{Section 5.1}.

To compute \acronym{GCD}s with this machine, we set the input registers, start
the machine, and examine the result when the simulation terminates:

\begin{scheme}
(set-register-contents! gcd-machine 'a 206)
~\textit{done}~
(set-register-contents! gcd-machine 'b 40)
~\textit{done}~
(start gcd-machine)
~\textit{done}~
(get-register-contents gcd-machine 'a)
~\textit{2}~
\end{scheme}

\noindent
This computation will run much more slowly than a \code{gcd} procedure written
in Scheme, because we will simulate low-level machine instructions, such as
\code{assign}, by much more complex operations.

\begin{quote}
\heading{\phantomsection\label{Exercise 5.7}Exercise 5.7:} Use the simulator to test the
machines you designed in \link{Exercise 5.4}.
\end{quote}



\subsection{The Machine Model}
\label{Section 5.2.1}

The machine model generated by \code{make\-/machine} is represented as a
procedure with local state using the message-passing techniques developed in
\link{Chapter 3}.  To build this model, \code{make\-/machine} begins by calling
the procedure \code{make\-/new\-/machine} to construct the parts of the machine
model that are common to all register machines.  This basic machine model
constructed by \code{make\-/new\-/machine} is essentially a container for some
registers and a stack, together with an execution mechanism that processes the
controller instructions one by one.

\code{Make\-/machine} then extends this basic model (by sending it messages) to
include the registers, operations, and controller of the particular machine
being defined.  First it allocates a register in the new machine for each of
the supplied register names and installs the designated operations in the
machine.  Then it uses an \newterm{assembler} (described below in 
\link{Section 5.2.2}) to transform the controller list into instructions for the new
machine and installs these as the machine's instruction sequence.
\code{Make\-/machine} returns as its value the modified machine model.

\begin{scheme}
(define (make-machine register-names ops controller-text)
  (let ((machine (make-new-machine)))
    (for-each
     (lambda (register-name)
       ((machine 'allocate-register) register-name))
     register-names)
    ((machine 'install-operations) ops)
    ((machine 'install-instruction-sequence)
     (assemble controller-text machine))
    machine))
\end{scheme}

\subsubsection*{Registers}

We will represent a register as a procedure with local state, as in
\link{Chapter 3}.  The procedure \code{make\-/register} creates a register that
holds a value that can be accessed or changed:

\begin{scheme}
(define (make-register name)
  (let ((contents '*unassigned*))
    (define (dispatch message)
      (cond ((eq? message 'get) contents)
            ((eq? message 'set)
             (lambda (value) (set! contents value)))
            (else
             (error "Unknown request: REGISTER" message))))
    dispatch))
\end{scheme}

\noindent
The following procedures are used to access registers:

\begin{scheme}
(define (get-contents register) (register 'get))
(define (set-contents! register value)
  ((register 'set) value))
\end{scheme}

\subsubsection*{The stack}

We can also represent a stack as a procedure with local state.  The procedure
\code{make\-/stack} creates a stack whose local state consists of a list of the
items on the stack.  A stack accepts requests to \code{push} an item onto the
stack, to \code{pop} the top item off the stack and return it, and to
\code{initialize} the stack to empty.

\begin{scheme}
(define (make-stack)
  (let ((s '()))
    (define (push x) (set! s (cons x s)))
    (define (pop)
      (if (null? s)
          (error "Empty stack: POP")
          (let ((top (car s)))
            (set! s (cdr s))
            top)))
    (define (initialize)
      (set! s '())
      'done)
    (define (dispatch message)
      (cond ((eq? message 'push) push)
            ((eq? message 'pop) (pop))
            ((eq? message 'initialize) (initialize))
            (else (error "Unknown request: STACK" message))))
    dispatch))
\end{scheme}

\noindent
The following procedures are used to access stacks:

\begin{scheme}
(define (pop stack) (stack 'pop))
(define (push stack value) ((stack 'push) value))
\end{scheme}

\subsubsection*{The basic machine}

The \code{make\-/new\-/machine} procedure, shown in \link{Figure 5.13}, constructs
an object whose local state consists of a stack, an initially empty instruction
sequence, a list of operations that initially contains an operation to
initialize the stack, and a \newterm{register table} that initially contains
two registers, named \code{flag} and \code{pc} (for ``program counter'').  The
internal procedure \code{allocate\-/register} adds new entries to the register
table, and the internal procedure \code{lookup\-/register} looks up registers in
the table.

The \code{flag} register is used to control branching in the simulated machine.
\code{Test} instructions set the contents of \code{flag} to the result of the
test (true or false).  \code{Branch} instructions decide whether or not to
branch by examining the contents of \code{flag}.

The \code{pc} register determines the sequencing of instructions as the machine
runs.  This sequencing is implemented by the internal procedure \code{execute}.
In the simulation model, each machine instruction is a data structure that
includes a procedure of no arguments, called the \newterm{instruction execution
procedure}, such that calling this procedure simulates executing the
instruction.  As the simulation runs, \code{pc} points to the place in the
instruction sequence beginning with the next instruction to be executed.
\code{Execute} gets that instruction, executes it by calling the instruction
execution procedure, and repeats this cycle until there are no more
instructions to execute (i.e., until \code{pc} points to the end of the
instruction sequence).

\begin{quote}
\heading{\phantomsection\label{Figure 5.13}Figure 5.13:} \( \downarrow \) The \code{make\-/new\-/machine}
procedure, which implements the basic machine model.

\begin{scheme}
(define (make-new-machine)
  (let ((pc (make-register 'pc))
        (flag (make-register 'flag))
        (stack (make-stack))
        (the-instruction-sequence '()))
    (let ((the-ops
           (list (list 'initialize-stack
                       (lambda () (stack 'initialize)))))
          (register-table
           (list (list 'pc pc) (list 'flag flag))))
      (define (allocate-register name)
        (if (assoc name register-table)
            (error "Multiply defined register: " name)
            (set! register-table
                  (cons (list name (make-register name))
                        register-table)))
        'register-allocated)
      (define (lookup-register name)
        (let ((val (assoc name register-table)))
          (if val
              (cadr val)
              (error "Unknown register:" name))))
      (define (execute)
        (let ((insts (get-contents pc)))
          (if (null? insts)
              'done
              (begin
                ((instruction-execution-proc (car insts)))
                (execute)))))
      (define (dispatch message)
        (cond ((eq? message 'start)
               (set-contents! pc the-instruction-sequence)
               (execute))
              ((eq? message 'install-instruction-sequence)
               (lambda (seq)
                 (set! the-instruction-sequence seq)))
              ((eq? message 'allocate-register) 
               allocate-register)
              ((eq? message 'get-register)
               lookup-register)
              ((eq? message 'install-operations)
               (lambda (ops)
                 (set! the-ops (append the-ops ops))))
              ((eq? message 'stack) stack)
              ((eq? message 'operations) the-ops)
              (else (error "Unknown request: MACHINE"
                           message))))
      dispatch)))
\end{scheme}
\end{quote}

\noindent
As part of its operation, each instruction execution procedure modifies
\code{pc} to indicate the next instruction to be executed.  \code{Branch} and
\code{goto} instructions change \code{pc} to point to the new destination.  All
other instructions simply advance \code{pc}, making it point to the next
instruction in the sequence.  Observe that each call to \code{execute} calls
\code{execute} again, but this does not produce an infinite loop because
running the instruction execution procedure changes the contents of \code{pc}.

\code{Make\-/new\-/machine} returns a \code{dispatch} procedure that implements
message-passing access to the internal state.  Notice that starting the machine
is accomplished by setting \code{pc} to the beginning of the instruction
sequence and calling \code{execute}.

For convenience, we provide an alternate procedural interface to a machine's
\code{start} operation, as well as procedures to set and examine register
contents, as specified at the beginning of \link{Section 5.2}:

\begin{scheme}
(define (start machine) (machine 'start))
(define (get-register-contents machine register-name)
  (get-contents (get-register machine register-name)))
(define (set-register-contents! machine register-name value)
  (set-contents! (get-register machine register-name)
                 value)
  'done)
\end{scheme}

\noindent
These procedures (and many procedures in \link{Section 5.2.2} and \link{Section 5.2.3})
use the following to look up the register with a given name in a given machine:

\begin{scheme}
(define (get-register machine reg-name)
  ((machine 'get-register) reg-name))
\end{scheme}

\subsection{The Assembler}
\label{Section 5.2.2}

The assembler transforms the sequence of controller expressions for a machine
into a corresponding list of machine instructions, each with its execution
procedure.  Overall, the assembler is much like the evaluators we studied in
\link{Chapter 4}---there is an input language (in this case, the
register-machine language) and we must perform an appropriate action for each
type of expression in the language.

The technique of producing an execution procedure for each instruction is just
what we used in \link{Section 4.1.7} to speed up the evaluator by separating
analysis from runtime execution.  As we saw in \link{Chapter 4}, much useful
analysis of Scheme expressions could be performed without knowing the actual
values of variables.  Here, analogously, much useful analysis of
register-machine-language expressions can be performed without knowing the
actual contents of machine registers.  For example, we can replace references
to registers by pointers to the register objects, and we can replace references
to labels by pointers to the place in the instruction sequence that the label
designates.

Before it can generate the instruction execution procedures, the assembler must
know what all the labels refer to, so it begins by scanning the controller text
to separate the labels from the instructions.  As it scans the text, it
constructs both a list of instructions and a table that associates each label
with a pointer into that list.  Then the assembler augments the instruction
list by inserting the execution procedure for each instruction.

The \code{assemble} procedure is the main entry to the assembler.  It takes the
controller text and the machine model as arguments and returns the instruction
sequence to be stored in the model.  \code{Assemble} calls
\code{extract\-/labels} to build the initial instruction list and label table
from the supplied controller text.  The second argument to
\code{extract\-/labels} is a procedure to be called to process these results:
This procedure uses \code{update\-/insts!} to generate the instruction execution
procedures and insert them into the instruction list, and returns the modified
list.

\begin{scheme}
(define (assemble controller-text machine)
  (extract-labels
   controller-text
   (lambda (insts labels)
     (update-insts! insts labels machine)
     insts)))
\end{scheme}

\noindent
\code{Extract\-/labels} takes as arguments a list \code{text} (the sequence of
controller instruction expressions) and a \code{receive} procedure.
\code{Receive} will be called with two values: (1) a list \code{insts} of
instruction data structures, each containing an instruction from \code{text};
and (2) a table called \code{labels}, which associates each label from
\code{text} with the position in the list \code{insts} that the label
designates.

\begin{scheme}
(define (extract-labels text receive)
  (if (null? text)
      (receive '() '())
      (extract-labels
       (cdr text)
       (lambda (insts labels)
         (let ((next-inst (car text)))
           (if (symbol? next-inst)
               (receive insts
                        (cons (make-label-entry next-inst
                                                insts)
                              labels))
               (receive (cons (make-instruction next-inst)
                              insts)
                        labels)))))))
\end{scheme}

\noindent
\code{Extract\-/labels} works by sequentially scanning the elements of the
\code{text} and accumulating the \code{insts} and the \code{labels}.  If an
element is a symbol (and thus a label) an appropriate entry is added to the
\code{labels} table.  Otherwise the element is accumulated onto the
\code{insts} list.\footnote{Using the \code{receive} procedure here is a way to
get \code{extract\-/labels} to effectively return two values---\code{labels} and
\code{insts}---without explicitly making a compound data structure to hold
them.  An alternative implementation, which returns an explicit pair of values,
is

\begin{smallscheme}
(define (extract-labels text)
  (if (null? text)
      (cons '() '())
      (let ((result (extract-labels (cdr text))))
        (let ((insts (car result)) (labels (cdr result)))
          (let ((next-inst (car text)))
            (if (symbol? next-inst)
                (cons insts
                      (cons (make-label-entry next-inst insts) 
                            labels))
                (cons (cons (make-instruction next-inst) insts)
                      labels)))))))
\end{smallscheme}

\noindent
which would be called by \code{assemble} as follows:

\begin{smallscheme}
(define (assemble controller-text machine)
  (let ((result (extract-labels controller-text)))
    (let ((insts (car result)) (labels (cdr result)))
      (update-insts! insts labels machine)
      insts)))
\end{smallscheme}

\noindent
You can consider our use of \code{receive} as demonstrating an elegant way to
return multiple values, or simply an excuse to show off a programming trick.
An argument like \code{receive} that is the next procedure to be invoked is
called a ``continuation.''  Recall that we also used continuations to implement
the backtracking control structure in the \code{amb} evaluator in 
\link{Section 4.3.3}.}

\code{Update\-/insts!} modifies the instruction list, which initially contains
only the text of the instructions, to include the corresponding execution
procedures:

\begin{scheme}
(define (update-insts! insts labels machine)
  (let ((pc (get-register machine 'pc))
        (flag (get-register machine 'flag))
        (stack (machine 'stack))
        (ops (machine 'operations)))
    (for-each
     (lambda (inst)
       (set-instruction-execution-proc!
        inst
        (make-execution-procedure
         (instruction-text inst)
         labels machine pc flag stack ops)))
     insts)))
\end{scheme}

\noindent
The machine instruction data structure simply pairs the instruction text with
the corresponding execution procedure.  The execution procedure is not yet
available when \code{extract\-/labels} constructs the instruction, and is
inserted later by \code{update\-/insts!}.

\begin{scheme}
(define (make-instruction text) (cons text '()))
(define (instruction-text inst) (car inst))
(define (instruction-execution-proc inst) (cdr inst))
(define (set-instruction-execution-proc! inst proc)
  (set-cdr! inst proc))
\end{scheme}

\noindent
The instruction text is not used by our simulator, but it is handy to keep
around for debugging (see \link{Exercise 5.16}).

Elements of the label table are pairs:

\begin{scheme}
(define (make-label-entry label-name insts)
  (cons label-name insts))
\end{scheme}

\noindent
Entries will be looked up in the table with

\begin{scheme}
(define (lookup-label labels label-name)
  (let ((val (assoc label-name labels)))
    (if val
        (cdr val)
        (error "Undefined label: ASSEMBLE"
               label-name))))
\end{scheme}

\begin{quote}
\heading{\phantomsection\label{Exercise 5.8}Exercise 5.8:} The following register-machine code
is ambiguous, because the label \code{here} is defined more than once:

\begin{scheme}
start
  (goto (label here))
here
  (assign a (const 3))
  (goto (label there))
here
  (assign a (const 4))
  (goto (label there))
there
\end{scheme}

With the simulator as written, what will the contents of register \code{a} be
when control reaches \code{there}?  Modify the \code{extract\-/labels} procedure
so that the assembler will signal an error if the same label name is used to
indicate two different locations.
\end{quote}

\subsection{Generating Execution Procedures\\ for Instructions}
\label{Section 5.2.3}

The assembler calls \code{make\-/execution\-/procedure} to generate the execution
procedure for an instruction.  Like the \code{analyze} procedure in the
evaluator of \link{Section 4.1.7}, this dispatches on the type of instruction to
generate the appropriate execution procedure.

\begin{scheme}
(define (make-execution-procedure 
         inst labels machine pc flag stack ops)
  (cond ((eq? (car inst) 'assign)
         (make-assign inst machine labels ops pc))
        ((eq? (car inst) 'test)
         (make-test inst machine labels ops flag pc))
        ((eq? (car inst) 'branch)
         (make-branch inst machine labels flag pc))
        ((eq? (car inst) 'goto)
         (make-goto inst machine labels pc))
        ((eq? (car inst) 'save)
         (make-save inst machine stack pc))
        ((eq? (car inst) 'restore)
         (make-restore inst machine stack pc))
        ((eq? (car inst) 'perform)
         (make-perform inst machine labels ops pc))
        (else
         (error "Unknown instruction type: ASSEMBLE"
                inst))))
\end{scheme}

\noindent
For each type of instruction in the register-machine language, there is a
generator that builds an appropriate execution procedure.  The details of these
procedures determine both the syntax and meaning of the individual instructions
in the register-machine language.  We use data abstraction to isolate the
detailed syntax of register-machine expressions from the general execution
mechanism, as we did for evaluators in \link{Section 4.1.2}, by using syntax
procedures to extract and classify the parts of an instruction.

\subsubsection*{\code{Assign} instructions}

The \code{make\-/assign} procedure handles \code{assign} instructions:

\begin{scheme}
(define (make-assign inst machine labels operations pc)
  (let ((target
         (get-register machine (assign-reg-name inst)))
        (value-exp (assign-value-exp inst)))
    (let ((value-proc
           (if (operation-exp? value-exp)
               (make-operation-exp
                value-exp machine labels operations)
               (make-primitive-exp
                (car value-exp) machine labels))))
      (lambda ()   ~\textrm{; execution procedure for \code{assign}}~
        (set-contents! target (value-proc))
        (advance-pc pc)))))
\end{scheme}

\noindent
\code{Make\-/assign} extracts the target register name (the second element of the
instruction) and the value expression (the rest of the list that forms the
instruction) from the \code{assign} instruction using the selectors

\begin{scheme}
(define (assign-reg-name assign-instruction)
  (cadr assign-instruction))
(define (assign-value-exp assign-instruction)
  (cddr assign-instruction))
\end{scheme}

\noindent
The register name is looked up with \code{get\-/register} to produce the target
register object.  The value expression is passed to \code{make\-/operation\-/exp}
if the value is the result of an operation, and to \code{make\-/primitive\-/exp}
otherwise.  These procedures (shown below) parse the value expression and
produce an execution procedure for the value.  This is a procedure of no
arguments, called \code{value\-/proc}, which will be evaluated during the
simulation to produce the actual value to be assigned to the register.  Notice
that the work of looking up the register name and parsing the value expression
is performed just once, at assembly time, not every time the instruction is
simulated.  This saving of work is the reason we use execution procedures, and
corresponds directly to the saving in work we obtained by separating program
analysis from execution in the evaluator of \link{Section 4.1.7}.

The result returned by \code{make\-/assign} is the execution procedure for the
\code{assign} instruction.  When this procedure is called (by the machine
model's \code{execute} procedure), it sets the contents of the target register
to the result obtained by executing \code{value\-/proc}.  Then it advances the
\code{pc} to the next instruction by running the procedure

\begin{scheme}
(define (advance-pc pc)
  (set-contents! pc (cdr (get-contents pc))))
\end{scheme}

\noindent
\code{Advance\-/pc} is the normal termination for all instructions except
\code{branch} and \code{goto}.

\subsubsection*{\code{Test}, \code{branch}, and \code{goto} instructions}

\code{Make\-/test} handles \code{test} instructions in a similar way.  It
extracts the expression that specifies the condition to be tested and generates
an execution procedure for it.  At simulation time, the procedure for the
condition is called, the result is assigned to the \code{flag} register, and
the \code{pc} is advanced:

\begin{scheme}

(define (make-test inst machine labels operations flag pc)
  (let ((condition (test-condition inst)))
    (if (operation-exp? condition)
        (let ((condition-proc
               (make-operation-exp
                condition machine labels operations)))
          (lambda ()
            (set-contents! flag (condition-proc))
            (advance-pc pc)))
        (error "Bad TEST instruction: ASSEMBLE" inst))))
(define (test-condition test-instruction)
  (cdr test-instruction))
\end{scheme}

\noindent
The execution procedure for a \code{branch} instruction checks the contents of
the \code{flag} register and either sets the contents of the \code{pc} to the
branch destination (if the branch is taken) or else just advances the \code{pc}
(if the branch is not taken).  Notice that the indicated destination in a
\code{branch} instruction must be a label, and the \code{make\-/branch} procedure
enforces this.  Notice also that the label is looked up at assembly time, not
each time the \code{branch} instruction is simulated.

\begin{scheme}
(define (make-branch inst machine labels flag pc)
  (let ((dest (branch-dest inst)))
    (if (label-exp? dest)
        (let ((insts
               (lookup-label
                labels 
                (label-exp-label dest))))
          (lambda ()
            (if (get-contents flag)
                (set-contents! pc insts)
                (advance-pc pc))))
        (error "Bad BRANCH instruction: ASSEMBLE" inst))))
(define (branch-dest branch-instruction)
  (cadr branch-instruction))
\end{scheme}

\noindent
A \code{goto} instruction is similar to a branch, except that the destination
may be specified either as a label or as a register, and there is no condition
to check---the \code{pc} is always set to the new destination.

\begin{scheme}
(define (make-goto inst machine labels pc)
  (let ((dest (goto-dest inst)))
    (cond ((label-exp? dest)
           (let ((insts (lookup-label
                         labels
                         (label-exp-label dest))))
             (lambda () (set-contents! pc insts))))
          ((register-exp? dest)
           (let ((reg (get-register
                       machine
                       (register-exp-reg dest))))
             (lambda ()
               (set-contents! pc (get-contents reg)))))
          (else (error "Bad GOTO instruction: ASSEMBLE" inst)))))
(define (goto-dest goto-instruction)
  (cadr goto-instruction))
\end{scheme}

\subsubsection*{Other instructions}

The stack instructions \code{save} and \code{restore} simply use the stack with
the designated register and advance the \code{pc}:

\begin{scheme}
(define (make-save inst machine stack pc)
  (let ((reg (get-register machine
                           (stack-inst-reg-name inst))))
    (lambda ()
      (push stack (get-contents reg))
      (advance-pc pc))))
(define (make-restore inst machine stack pc)
  (let ((reg (get-register machine
                           (stack-inst-reg-name inst))))
    (lambda ()
      (set-contents! reg (pop stack))
      (advance-pc pc))))
(define (stack-inst-reg-name stack-instruction)
  (cadr stack-instruction))
\end{scheme}

\noindent
The final instruction type, handled by \code{make\-/perform}, generates an
execution procedure for the action to be performed.  At simulation time, the
action procedure is executed and the \code{pc} advanced.

\begin{scheme}
(define (make-perform inst machine labels operations pc)
  (let ((action (perform-action inst)))
    (if (operation-exp? action)
        (let ((action-proc
               (make-operation-exp
                action machine labels operations)))
          (lambda () (action-proc) (advance-pc pc)))
        (error "Bad PERFORM instruction: ASSEMBLE" inst))))
(define (perform-action inst) (cdr inst))
\end{scheme}

\subsubsection*{Execution procedures for subexpressions}

The value of a \code{reg}, \code{label}, or \code{const} expression may be
needed for assignment to a register (\code{make\-/assign}) or for input to an
operation (\code{make\-/operation\-/exp}, below).  The following procedure
generates execution procedures to produce values for these expressions during
the simulation:

\begin{scheme}
(define (make-primitive-exp exp machine labels)
  (cond ((constant-exp? exp)
         (let ((c (constant-exp-value exp)))
           (lambda () c)))
        ((label-exp? exp)
         (let ((insts (lookup-label
                       labels
                       (label-exp-label exp))))
           (lambda () insts)))
        ((register-exp? exp)
         (let ((r (get-register machine (register-exp-reg exp))))
           (lambda () (get-contents r))))
        (else (error "Unknown expression type: ASSEMBLE" exp))))
\end{scheme}

\noindent
The syntax of \code{reg}, \code{label}, and \code{const} expressions is
determined by

\begin{scheme}
(define (register-exp? exp) (tagged-list? exp 'reg))
(define (register-exp-reg exp) (cadr exp))
(define (constant-exp? exp) (tagged-list? exp 'const))
(define (constant-exp-value exp) (cadr exp))
(define (label-exp? exp) (tagged-list? exp 'label))
(define (label-exp-label exp) (cadr exp))
\end{scheme}

\noindent
\code{Assign}, \code{perform}, and \code{test} instructions may include the
application of a machine operation (specified by an \code{op} expression) to
some operands (specified by \code{reg} and \code{const} expressions).  The
following procedure produces an execution procedure for an ``operation
expression''---a list containing the operation and operand expressions from the
instruction:

\begin{scheme}
(define (make-operation-exp exp machine labels operations)
  (let ((op (lookup-prim (operation-exp-op exp)
                         operations))
        (aprocs
         (map (lambda (e)
                (make-primitive-exp e machine labels))
              (operation-exp-operands exp))))
    (lambda ()
      (apply op (map (lambda (p) (p)) aprocs)))))
\end{scheme}

\noindent
The syntax of operation expressions is determined by

\begin{scheme}
(define (operation-exp? exp)
  (and (pair? exp) (tagged-list? (car exp) 'op)))
(define (operation-exp-op operation-exp)
  (cadr (car operation-exp)))
(define (operation-exp-operands operation-exp)
  (cdr operation-exp))
\end{scheme}

\noindent
Observe that the treatment of operation expressions is very much like the
treatment of procedure applications by the \code{analyze\-/application} procedure
in the evaluator of \link{Section 4.1.7} in that we generate an execution
procedure for each operand.  At simulation time, we call the operand procedures
and apply the Scheme procedure that simulates the operation to the resulting
values.  The simulation procedure is found by looking up the operation name in
the operation table for the machine:

\begin{scheme}
(define (lookup-prim symbol operations)
  (let ((val (assoc symbol operations)))
    (if val
        (cadr val)
        (error "Unknown operation: ASSEMBLE"
               symbol))))
\end{scheme}

\begin{quote}
\heading{\phantomsection\label{Exercise 5.9}Exercise 5.9:} The treatment of machine operations
above permits them to operate on labels as well as on constants and the
contents of registers.  Modify the expression-processing procedures to enforce
the condition that operations can be used only with registers and constants.
\end{quote}

\begin{quote}
\heading{\phantomsection\label{Exercise 5.10}Exercise 5.10:} Design a new syntax for
register-machine instructions and modify the simulator to use your new syntax.
Can you implement your new syntax without changing any part of the simulator
except the syntax procedures in this section?
\end{quote}

\begin{quote}
\heading{\phantomsection\label{Exercise 5.11}Exercise 5.11:} When we introduced \code{save}
and \code{restore} in \link{Section 5.1.4}, we didn't specify what would happen
if you tried to restore a register that was not the last one saved, as in the
sequence

\begin{scheme}
(save y)  (save x)  (restore y)
\end{scheme}

There are several reasonable possibilities for the meaning of \code{restore}:

\begin{enumerate}[a.]

\item
\code{(restore y)} puts into \code{y} the last value saved on the stack,
regardless of what register that value came from.  This is the way our
simulator behaves.  Show how to take advantage of this behavior to eliminate
one instruction from the Fibonacci machine of \link{Section 5.1.4} (\link{Figure 5.12}).

\item
\code{(restore y)} puts into \code{y} the last value saved on the stack, but
only if that value was saved from \code{y}; otherwise, it signals an error.
Modify the simulator to behave this way.  You will have to change \code{save}
to put the register name on the stack along with the value.

\item
\code{(restore y)} puts into \code{y} the last value saved from \code{y}
regardless of what other registers were saved after \code{y} and not restored.
Modify the simulator to behave this way.  You will have to associate a separate
stack with each register.  You should make the \code{initialize\-/stack}
operation initialize all the register stacks.

\end{enumerate}
\end{quote}

\begin{quote}
\heading{\phantomsection\label{Exercise 5.12}Exercise 5.12:} The simulator can be used to help
determine the data paths required for implementing a machine with a given
controller.  Extend the assembler to store the following information in the
machine model:

\begin{itemize}

\item
a list of all instructions, with duplicates removed, sorted by instruction type
(\code{assign}, \code{goto}, and so on);

\item
a list (without duplicates) of the registers used to hold entry points (these
are the registers referenced by \code{goto} instructions);

\item
a list (without duplicates) of the registers that are \code{save}d
or \code{restore}d;

\item
for each register, a list (without duplicates) of the sources from which it is
assigned (for example, the sources for register \code{val} in the factorial
machine of \link{Figure 5.11} are \code{(const 1)} and \code{((op *) (reg n)
(reg val))}).

\end{itemize}

Extend the message-passing interface to the machine to provide access to this
new information.  To test your analyzer, define the Fibonacci machine from
\link{Figure 5.12} and examine the lists you constructed.
\end{quote}

\begin{quote}
\heading{\phantomsection\label{Exercise 5.13}Exercise 5.13:} Modify the simulator so that it
uses the controller sequence to determine what registers the machine has rather
than requiring a list of registers as an argument to \code{make\-/machine}.
Instead of pre-allocating the registers in \code{make\-/machine}, you can
allocate them one at a time when they are first seen during assembly of the
instructions.
\end{quote}

\subsection{Monitoring Machine Performance}
\label{Section 5.2.4}

Simulation is useful not only for verifying the correctness of a proposed
machine design but also for measuring the machine's performance.  For example,
we can install in our simulation program a ``meter'' that measures the number
of stack operations used in a computation.  To do this, we modify our simulated
stack to keep track of the number of times registers are saved on the stack and
the maximum depth reached by the stack, and add a message to the stack's
interface that prints the statistics, as shown below.  We also add an operation
to the basic machine model to print the stack statistics, by initializing
\code{the\-/ops} in \code{make\-/new\-/machine} to

\begin{scheme}
(list (list 'initialize-stack
            (lambda () (stack 'initialize)))
      (list 'print-stack-statistics
            (lambda () (stack 'print-statistics))))
\end{scheme}

\noindent
Here is the new version of \code{make\-/stack}:

\begin{scheme}
(define (make-stack)
  (let ((s '())
        (number-pushes 0)
        (max-depth 0)
        (current-depth 0))
    (define (push x)
      (set! s (cons x s))
      (set! number-pushes (+ 1 number-pushes))
      (set! current-depth (+ 1 current-depth))
      (set! max-depth (max current-depth max-depth)))
    (define (pop)
      (if (null? s)
          (error "Empty stack: POP")
          (let ((top (car s)))
            (set! s (cdr s))
            (set! current-depth (- current-depth 1))
            top)))
    (define (initialize)
      (set! s '())
      (set! number-pushes 0)
      (set! max-depth 0)
      (set! current-depth 0)
      'done)
    (define (print-statistics)
      (newline)
      (display (list 'total-pushes  '= number-pushes
                     'maximum-depth '= max-depth)))
    (define (dispatch message)
      (cond ((eq? message 'push) push)
            ((eq? message 'pop) (pop))
            ((eq? message 'initialize) (initialize))
            ((eq? message 'print-statistics)
             (print-statistics))
            (else (error "Unknown request: STACK" message))))
    dispatch))
\end{scheme}

\noindent
\link{Exercise 5.15} through \link{Exercise 5.19} describe other useful
monitoring and debugging features that can be added to the register-machine
simulator.

\begin{quote}
\heading{\phantomsection\label{Exercise 5.14}Exercise 5.14:} Measure the number of pushes and
the maximum stack depth required to compute \( n! \) for various small values of
\( n \) using the factorial machine shown in \link{Figure 5.11}.  From your data
determine formulas in terms of \( n \) for the total number of push operations
and the maximum stack depth used in computing \( n! \) for any \( n > 1 \). Note
that each of these is a linear function of \( n \) and is thus determined by two
constants.  In order to get the statistics printed, you will have to augment
the factorial machine with instructions to initialize the stack and print the
statistics.  You may want to also modify the machine so that it repeatedly
reads a value for \( n \), computes the factorial, and prints the result (as we
did for the \acronym{GCD} machine in \link{Figure 5.4}), so that you will not
have to repeatedly invoke \code{get\-/register\-/contents},
\code{set\-/register\-/contents!}, and \code{start}.
\end{quote}

\begin{quote}
\heading{\phantomsection\label{Exercise 5.15}Exercise 5.15:} Add \newterm{instruction
counting} to the register machine simulation.  That is, have the machine model
keep track of the number of instructions executed.  Extend the machine model's
interface to accept a new message that prints the value of the instruction
count and resets the count to zero.
\end{quote}

\begin{quote}
\heading{\phantomsection\label{Exercise 5.16}Exercise 5.16:} Augment the simulator to provide
for \newterm{instruction tracing}.  That is, before each instruction is
executed, the simulator should print the text of the instruction.  Make the
machine model accept \code{trace\-/on} and \code{trace\-/off} messages to turn
tracing on and off.
\end{quote}

\begin{quote}
\heading{\phantomsection\label{Exercise 5.17}Exercise 5.17:} Extend the instruction tracing of
\link{Exercise 5.16} so that before printing an instruction, the simulator
prints any labels that immediately precede that instruction in the controller
sequence.  Be careful to do this in a way that does not interfere with
instruction counting (\link{Exercise 5.15}).  You will have to make the
simulator retain the necessary label information.
\end{quote}

\begin{quote}
\heading{\phantomsection\label{Exercise 5.18}Exercise 5.18:} Modify the \code{make\-/register}
procedure of \link{Section 5.2.1} so that registers can be traced.  Registers
should accept messages that turn tracing on and off.  When a register is
traced, assigning a value to the register should print the name of the
register, the old contents of the register, and the new contents being
assigned.  Extend the interface to the machine model to permit you to turn
tracing on and off for designated machine registers.
\end{quote}

\begin{quote}
\heading{\phantomsection\label{Exercise 5.19}Exercise 5.19:} Alyssa P. Hacker wants a
\newterm{breakpoint} feature in the simulator to help her debug her machine
designs.  You have been hired to install this feature for her.  She wants to be
able to specify a place in the controller sequence where the simulator will
stop and allow her to examine the state of the machine.  You are to implement a
procedure

\begin{scheme}
(set-breakpoint ~\( \dark \langle\kern0.08em \)~~\var{\dark machine}~~\( \dark \rangle \)~ ~\( \dark \langle \)~~\var{\dark label}~~\( \dark \rangle \)~ ~\( \dark \langle \)~~\var{\dark n}~~\( \dark \rangle \)~)
\end{scheme}

\noindent
that sets a breakpoint just before the \( n^{\mathrm{th}} \) instruction after the given
label.  For example,

\begin{scheme}
(set-breakpoint gcd-machine 'test-b 4)
\end{scheme}

\noindent
installs a breakpoint in \code{gcd\-/machine} just before the assignment to
register \code{a}.  When the simulator reaches the breakpoint it should print
the label and the offset of the breakpoint and stop executing instructions.
Alyssa can then use \code{get\-/register\-/contents} and
\code{set\-/register\-/contents!} to manipulate the state of the simulated machine.
She should then be able to continue execution by saying

\begin{scheme}
(proceed-machine ~\( \dark \langle\kern0.08em \)~~\var{\dark machine}~~\( \dark \rangle \)~)
\end{scheme}

She should also be able to remove a specific breakpoint by means of

\begin{scheme}
(cancel-breakpoint ~\( \dark \langle\kern0.08em \)~~\var{\dark machine}~~\( \dark \rangle \)~ ~\( \dark \langle \)~~\var{\dark label}~~\( \dark \rangle \)~ ~\( \dark \langle \)~~\var{\dark n}~~\( \dark \rangle \)~)
\end{scheme}

\noindent
or to remove all breakpoints by means of

\begin{scheme}
(cancel-all-breakpoints ~\( \dark \langle\kern0.08em \)~~\var{\dark machine}~~\( \dark \rangle \)~)
\end{scheme}
\end{quote}

\section{Storage Allocation and Garbage Collection}
\label{Section 5.3}

In \link{Section 5.4}, we will show how to implement a Scheme evaluator as a
register machine.  In order to simplify the discussion, we will assume that our
register machines can be equipped with a \newterm{list-structured memory}, in
which the basic operations for manipulating list-structured data are primitive.
Postulating the existence of such a memory is a useful abstraction when one is
focusing on the mechanisms of control in a Scheme interpreter, but this does
not reflect a realistic view of the actual primitive data operations of
contemporary computers.  To obtain a more complete picture of how a Lisp system
operates, we must investigate how list structure can be represented in a way
that is compatible with conventional computer memories.

There are two considerations in implementing list structure.  The first is
purely an issue of representation: how to represent the ``box-and-pointer''
structure of Lisp pairs, using only the storage and addressing capabilities of
typical computer memories.  The second issue concerns the management of memory
as a computation proceeds.  The operation of a Lisp system depends crucially on
the ability to continually create new data objects.  These include objects that
are explicitly created by the Lisp procedures being interpreted as well as
structures created by the interpreter itself, such as environments and argument
lists.  Although the constant creation of new data objects would pose no
problem on a computer with an infinite amount of rapidly addressable memory,
computer memories are available only in finite sizes (more's the pity).  Lisp
systems thus provide an \newterm{automatic storage allocation} facility to
support the illusion of an infinite memory.  When a data object is no longer
needed, the memory allocated to it is automatically recycled and used to
construct new data objects.  There are various techniques for providing such
automatic storage allocation.  The method we shall discuss in this section is
called \newterm{garbage collection}.



\subsection{Memory as Vectors}
\label{Section 5.3.1}

A conventional computer memory can be thought of as an array of cubbyholes,
each of which can contain a piece of information.  Each cubbyhole has a unique
name, called its \newterm{address} or \newterm{location}.  Typical memory
systems provide two primitive operations: one that fetches the data stored in a
specified location and one that assigns new data to a specified location.
Memory addresses can be incremented to support sequential access to some set of
the cubbyholes.  More generally, many important data operations require that
memory addresses be treated as data, which can be stored in memory locations
and manipulated in machine registers.  The representation of list structure is
one application of such \newterm{address arithmetic}.

To model computer memory, we use a new kind of data structure called a
\newterm{vector}.  Abstractly, a vector is a compound data object whose
individual elements can be accessed by means of an integer index in an amount
of time that is independent of the index.\footnote{We could represent memory as
lists of items.  However, the access time would then not be independent of the
index, since accessing the \( n^{\mathrm{th}} \) element of a list requires \( n - 1 \)
\code{cdr} operations.} In order to describe memory operations, we use two
primitive Scheme procedures for manipulating vectors:

\begin{itemize}

\item
\code{(vector\-/ref \( \langle \)\var{vector}\( \rangle \) \( \langle \)\var{n}\( \rangle \))} returns the \( n^{\mathrm{th}} \)
element of the vector.

\item
\code{(vector\-/set! \( \langle \)\var{vector}\( \rangle \) \( \langle \)\var{n}\( \rangle \) \( \langle \)\var{value}\( \rangle \))}
sets the \( n^{\mathrm{th}} \) element of the vector to the designated value.

\end{itemize}

\noindent
For example, if \code{v} is a vector, then \code{(vector\-/ref v 5)} gets the
fifth entry in the vector \code{v} and \code{(vector\-/set! v 5 7)} changes the
value of the fifth entry of the vector \code{v} to 7.\footnote{For
completeness, we should specify a \code{make\-/vector} operation that constructs
vectors.  However, in the present application we will use vectors only to model
fixed divisions of the computer memory.}  For computer memory, this access can
be implemented through the use of address arithmetic to combine a \newterm{base
address} that specifies the beginning location of a vector in memory with an
\newterm{index} that specifies the offset of a particular element of the
vector.

\subsubsection*{Representing Lisp data}

We can use vectors to implement the basic pair structures required for a
list-structured memory.  Let us imagine that computer memory is divided into
two vectors: \code{the\-/cars} and \code{the\-/cdrs}.  We will represent list
structure as follows: A pointer to a pair is an index into the two vectors.
The \code{car} of the pair is the entry in \code{the\-/cars} with the designated
index, and the \code{cdr} of the pair is the entry in \code{the\-/cdrs} with the
designated index.  We also need a representation for objects other than pairs
(such as numbers and symbols) and a way to distinguish one kind of data from
another.  There are many methods of accomplishing this, but they all reduce to
using \newterm{typed pointers}, that is, to extending the notion of ``pointer''
to include information on data type.\footnote{This is precisely the same
``tagged data'' idea we introduced in \link{Chapter 2} for dealing with generic
operations.  Here, however, the data types are included at the primitive
machine level rather than constructed through the use of lists.} The data type
enables the system to distinguish a pointer to a pair (which consists of the
``pair'' data type and an index into the memory vectors) from pointers to other
kinds of data (which consist of some other data type and whatever is being used
to represent data of that type).  Two data objects are considered to be the
same (\code{eq?}) if their pointers are identical.\footnote{Type information
may be encoded in a variety of ways, depending on the details of the machine on
which the Lisp system is to be implemented.  The execution efficiency of Lisp
programs will be strongly dependent on how cleverly this choice is made, but it
is difficult to formulate general design rules for good choices.  The most
straightforward way to implement typed pointers is to allocate a fixed set of
bits in each pointer to be a \newterm{type field} that encodes the data type.
Important questions to be addressed in designing such a representation include
the following: How many type bits are required?  How large must the vector
indices be?  How efficiently can the primitive machine instructions be used to
manipulate the type fields of pointers?  Machines that include special hardware
for the efficient handling of type fields are said to have \newterm{tagged
architectures}.} \link{Figure 5.14} illustrates the use of this method to
represent the list \code{((1 2) 3 4)}, whose box-and-pointer diagram is also
shown.  We use letter prefixes to denote the data-type information.  Thus, a
pointer to the pair with index 5 is denoted \code{p5}, the empty list is
denoted by the pointer \code{e0}, and a pointer to the number 4 is denoted
\code{n4}.  In the box-and-pointer diagram, we have indicated at the lower left
of each pair the vector index that specifies where the \code{car} and
\code{cdr} of the pair are stored.  The blank locations in \code{the\-/cars} and
\code{the\-/cdrs} may contain parts of other list structures (not of interest
here).

A pointer to a number, such as \code{n4}, might consist of a type indicating
numeric data together with the actual representation of the number
4.\footnote{This decision on the representation of numbers determines whether
\code{eq?}, which tests equality of pointers, can be used to test for equality
of numbers.  If the pointer contains the number itself, then equal numbers will
have the same pointer.  But if the pointer contains the index of a location
where the number is stored, equal numbers will be guaranteed to have equal
pointers only if we are careful never to store the same number in more than one
location.}  To deal with numbers that are too large to be represented in the
fixed amount of space allocated for a single pointer, we could use a distinct
\newterm{bignum} data type, for which the pointer designates a list in which
the parts of the number are stored.\footnote{This is just like writing a number
as a sequence of digits, except that each ``digit'' is a number between 0 and
the largest number that can be stored in a single pointer.}

\begin{figure}[tb]
\phantomsection\label{Figure 5.14}
\centering
\begin{comment}
\heading{Figure 5.14:} Box-and-pointer and memory-vector representations of the list \code{((1 2) 3 4)}.

\begin{example}
               +---+---+               +---+---+    +---+---+
((1 2) 3 4) -->| * | *-+-------------->| * | *-+--->| * | / |
               +-|-+---+               +-|-+---+    +-|-+---+
              1  |                    2  |         4  |
                 V                       V            V
               +---+---+    +---+---+  +---+        +---+
               | * | *-+--->| * | / |  | 3 |        | 4 |
               +-|-+---+    +-|-+---+  +---+        +---+
              5  |         7  |
                 V            V
               +---+        +---+
               | 1 |        | 2 |
               +---+        +---+

   Index   0    1    2    3    4    5    6    7    8    ...
         +----+----+----+----+----+----+----+----+----+----
the-cars |    | p5 | n3 |    | n4 | n1 |    | n2 |    | ...
         +----+----+----+----+----+----+----+----+----+----
the-cdrs |    | p2 | p4 |    | e0 | p7 |    | e0 |    | ...
         +----+----+----+----+----+----+----+----+----+----
\end{example}
\end{comment}
\includegraphics[width=91mm]{fig/chap5/Fig5.14a.pdf}
\begin{quote}
\heading{Figure 5.14:} Box-and-pointer and memory-vector representations of the list \code{((1 2) 3 4)}.
\end{quote}
\end{figure}

A symbol might be represented as a typed pointer that designates a sequence of
the characters that form the symbol's printed representation.  This sequence is
constructed by the Lisp reader when the character string is initially
encountered in input.  Since we want two instances of a symbol to be recognized
as the ``same'' symbol by \code{eq?} and we want \code{eq?} to be a simple test
for equality of pointers, we must ensure that if the reader sees the same
character string twice, it will use the same pointer (to the same sequence of
characters) to represent both occurrences.  To accomplish this, the reader
maintains a table, traditionally called the \newterm{obarray}, of all the
symbols it has ever encountered.  When the reader encounters a character string
and is about to construct a symbol, it checks the obarray to see if it has ever
before seen the same character string.  If it has not, it uses the characters
to construct a new symbol (a typed pointer to a new character sequence) and
enters this pointer in the obarray.  If the reader has seen the string before,
it returns the symbol pointer stored in the obarray.  This process of replacing
character strings by unique pointers is called \newterm{interning} symbols.

\subsubsection*{Implementing the primitive list operations}

Given the above representation scheme, we can replace each ``primitive'' list
operation of a register machine with one or more primitive vector operations.
We will use two registers, \code{the\-/cars} and \code{the\-/cdrs}, to identify the
memory vectors, and will assume that \code{vector\-/ref} and \code{vector\-/set!}
are available as primitive operations.  We also assume that numeric operations
on pointers (such as incrementing a pointer, using a pair pointer to index a
vector, or adding two numbers) use only the index portion of the typed pointer.

For example, we can make a register machine support the instructions

\begin{scheme}
(assign ~\( \dark \langle \)~~\( \dark reg_1 \)~~\( \dark \rangle \)~ (op car) (reg ~\( \dark \langle \)~~\( \dark reg_2 \)~~\( \dark \rangle \)~))
(assign ~\( \dark \langle \)~~\( \dark reg_1 \)~~\( \dark \rangle \)~ (op cdr) (reg ~\( \dark \langle \)~~\( \dark reg_2 \)~~\( \dark \rangle \)~))
\end{scheme}

\noindent
if we implement these, respectively, as

\begin{scheme}
(assign ~\( \dark \langle \)~~\( \dark reg_1 \)~~\( \dark \rangle \)~ (op vector-ref) (reg the-cars) (reg ~\( \dark \langle \)~~\( \dark reg_2 \)~~\( \dark \rangle \)~))
(assign ~\( \dark \langle \)~~\( \dark reg_1 \)~~\( \dark \rangle \)~ (op vector-ref) (reg the-cdrs) (reg ~\( \dark \langle \)~~\( \dark reg_2 \)~~\( \dark \rangle \)~))
\end{scheme}

\noindent
The instructions

\begin{scheme}
(perform (op set-car!) (reg ~\( \dark \langle \)~~\( \dark reg_1 \)~~\( \dark \rangle \)~) (reg ~\( \dark \langle \)~~\( \dark reg_2 \)~~\( \dark \rangle \)~))
(perform (op set-cdr!) (reg ~\( \dark \langle \)~~\( \dark reg_1 \)~~\( \dark \rangle \)~) (reg ~\( \dark \langle \)~~\( \dark reg_2 \)~~\( \dark \rangle \)~))
\end{scheme}

\noindent
are implemented as

\begin{scheme}
(perform
 (op vector-set!) (reg the-cars) (reg ~\( \dark \langle \)~~\( \dark reg_1 \)~~\( \dark \rangle \)~) (reg ~\( \dark \langle \)~~\( \dark reg_2 \)~~\( \dark \rangle \)~))
(perform
 (op vector-set!) (reg the-cdrs) (reg ~\( \dark \langle \)~~\( \dark reg_1 \)~~\( \dark \rangle \)~) (reg ~\( \dark \langle \)~~\( \dark reg_2 \)~~\( \dark \rangle \)~))
\end{scheme}

\noindent
\code{Cons} is performed by allocating an unused index and storing the
arguments to \code{cons} in \code{the\-/cars} and \code{the\-/cdrs} at that indexed
vector position.  We presume that there is a special register, \code{free},
that always holds a pair pointer containing the next available index, and that
we can increment the index part of that pointer to find the next free
location.\footnote{There are other ways of finding free storage.  For example,
we could link together all the unused pairs into a \newterm{free list}.  Our
free locations are consecutive (and hence can be accessed by incrementing a
pointer) because we are using a compacting garbage collector, as we will see in
\link{Section 5.3.2}.}  For example, the instruction

\begin{scheme}
(assign ~\( \dark \langle \)~~\( \dark reg_1 \)~~\( \dark \rangle \)~ (op cons) (reg ~\( \dark \langle \)~~\( \dark reg_2 \)~~\( \dark \rangle \)~) (reg ~\( \dark \langle \)~~\( \dark reg_3 \)~~\( \dark \rangle \)~))
\end{scheme}

\noindent
is implemented as the following sequence of vector operations:\footnote{This is
essentially the implementation of \code{cons} in terms of \code{set\-/car!} and
\code{set\-/cdr!}, as described in \link{Section 3.3.1}.  The operation
\code{get\-/new\-/pair} used in that implementation is realized here by the
\code{free} pointer.}

\begin{scheme}
(perform
 (op vector-set!) (reg the-cars) (reg free) (reg ~\( \dark \langle \)~~\( \dark reg_2 \)~~\( \dark \rangle \)~))
(perform
 (op vector-set!) (reg the-cdrs) (reg free) (reg ~\( \dark \langle \)~~\( \dark reg_3 \)~~\( \dark \rangle \)~))
(assign ~\( \dark \langle \)~~\( \dark reg_1 \)~~\( \dark \rangle \)~ (reg free))
(assign free (op +) (reg free) (const 1))
\end{scheme}

\noindent
The \code{eq?} operation

\begin{scheme}
(op eq?) (reg ~\( \dark \langle \)~~\( \dark reg_1 \)~~\( \dark \rangle \)~) (reg ~\( \dark \langle \)~~\( \dark reg_2 \)~~\( \dark \rangle \)~)
\end{scheme}

\noindent
simply tests the equality of all fields in the registers, and predicates such
as \code{pair?}, \code{null?}, \code{symbol?}, and \code{number?} need only
check the type field.

\subsubsection*{Implementing stacks}

Although our register machines use stacks, we need do nothing special here,
since stacks can be modeled in terms of lists.  The stack can be a list of the
saved values, pointed to by a special register \code{the\-/stack}.  Thus, \code{
(save \( \langle \)\var{reg}\( \rangle \))} can be implemented as

\begin{scheme}
(assign the-stack (op cons) (reg ~\( \dark \langle \)~~\var{\dark reg}~~\( \dark \rangle \)~) (reg the-stack))
\end{scheme}

\noindent
Similarly, \code{(restore \( \langle \)\var{reg}\( \rangle \))} can be implemented as

\begin{scheme}
(assign ~\( \dark \langle \)~~\var{\dark reg}~~\( \dark \rangle \)~ (op car) (reg the-stack))
(assign the-stack (op cdr) (reg the-stack))
\end{scheme}

\noindent
and \code{(perform (op initialize\-/stack))} can be implemented as

\begin{scheme}
(assign the-stack (const ()))
\end{scheme}

\noindent
These operations can be further expanded in terms of the vector operations
given above.  In conventional computer architectures, however, it is usually
advantageous to allocate the stack as a separate vector.  Then pushing and
popping the stack can be accomplished by incrementing or decrementing an index
into that vector.

\begin{quote}
\heading{\phantomsection\label{Exercise 5.20}Exercise 5.20:} Draw the box-and-pointer
representation and the memory-vector representation (as in \link{Figure 5.14})
of the list structure produced by

\begin{scheme}
(define x (cons 1 2))
(define y (list x x))
\end{scheme}

\noindent
with the \code{free} pointer initially \code{p1}.  What is the final value of
\code{free} ?  What pointers represent the values of \code{x} and \code{y} ?
\end{quote}

\begin{quote}
\heading{\phantomsection\label{Exercise 5.21}Exercise 5.21:} Implement register machines for
the following procedures.  Assume that the list-structure memory operations are
available as machine primitives.

\begin{enumerate}[a.]

\item
Recursive \code{count\-/leaves}:

\begin{scheme}
(define (count-leaves tree)
  (cond ((null? tree) 0)
        ((not (pair? tree)) 1)
        (else (+ (count-leaves (car tree))
                 (count-leaves (cdr tree))))))
\end{scheme}

\item
Recursive \code{count\-/leaves} with explicit counter:

\begin{scheme}
(define (count-leaves tree)
  (define (count-iter tree n)
    (cond ((null? tree) n)
          ((not (pair? tree)) (+ n 1))
          (else
           (count-iter (cdr tree)
                       (count-iter (car tree) 
                                   n)))))
  (count-iter tree 0))
\end{scheme}
\end{enumerate}
\end{quote}

\begin{quote}
\heading{\phantomsection\label{Exercise 5.22}Exercise 5.22:} \link{Exercise 3.12} of 
\link{Section 3.3.1} presented an \code{append} procedure that appends two lists to form
a new list and an \code{append!} procedure that splices two lists together.
Design a register machine to implement each of these procedures.  Assume that
the list-structure memory operations are available as primitive operations.
\end{quote}

\subsection{Maintaining the Illusion of Infinite Memory}
\label{Section 5.3.2}

The representation method outlined in \link{Section 5.3.1} solves the problem of
implementing list structure, provided that we have an infinite amount of
memory.  With a real computer we will eventually run out of free space in which
to construct new pairs.\footnote{This may not be true eventually, because
memories may get large enough so that it would be impossible to run out of free
memory in the lifetime of the computer.  For example, there are about
\( 3\cdot10^{13} \) microseconds in a year, so if we were to \code{cons} once per
microsecond we would need about \( 10^{15} \) cells of memory to build a machine that
could operate for 30 years without running out of memory.  That much memory
seems absurdly large by today's standards, but it is not physically impossible.
On the other hand, processors are getting faster and a future computer may have
large numbers of processors operating in parallel on a single memory, so it may
be possible to use up memory much faster than we have postulated.}  However,
most of the pairs generated in a typical computation are used only to hold
intermediate results.  After these results are accessed, the pairs are no
longer needed---they are \newterm{garbage}.  For instance, the computation

\begin{scheme}
(accumulate + 0 (filter odd? (enumerate-interval 0 n)))
\end{scheme}

\noindent
constructs two lists: the enumeration and the result of filtering the
enumeration.  When the accumulation is complete, these lists are no longer
needed, and the allocated memory can be reclaimed.  If we can arrange to
collect all the garbage periodically, and if this turns out to recycle memory
at about the same rate at which we construct new pairs, we will have preserved
the illusion that there is an infinite amount of memory.

In order to recycle pairs, we must have a way to determine which allocated
pairs are not needed (in the sense that their contents can no longer influence
the future of the computation).  The method we shall examine for accomplishing
this is known as \newterm{garbage collection}.  Garbage collection is based on
the observation that, at any moment in a Lisp interpretation, the only objects
that can affect the future of the computation are those that can be reached by
some succession of \code{car} and \code{cdr} operations starting from the
pointers that are currently in the machine registers.\footnote{We assume here
that the stack is represented as a list as described in \link{Section 5.3.1}, so
that items on the stack are accessible via the pointer in the stack register.}
Any memory cell that is not so accessible may be recycled.

There are many ways to perform garbage collection.  The method we shall examine
here is called \newterm{stop-and-copy}.  The basic idea is to divide memory
into two halves: ``working memory'' and ``free memory.''  When \code{cons}
constructs pairs, it allocates these in working memory.  When working memory is
full, we perform garbage collection by locating all the useful pairs in working
memory and copying these into consecutive locations in free memory.  (The
useful pairs are located by tracing all the \code{car} and \code{cdr} pointers,
starting with the machine registers.)  Since we do not copy the garbage, there
will presumably be additional free memory that we can use to allocate new
pairs.  In addition, nothing in the working memory is needed, since all the
useful pairs in it have been copied.  Thus, if we interchange the roles of
working memory and free memory, we can continue processing; new pairs will be
allocated in the new working memory (which was the old free memory).  When this
is full, we can copy the useful pairs into the new free memory (which was the
old working memory).\footnote{This idea was invented and first implemented by
Minsky, as part of the implementation of Lisp for the PDP-1 at the
\acronym{MIT} Research Laboratory of Electronics.  It was further developed by
\link{Fenichel and Yochelson (1969)} for use in the Lisp implementation for the
Multics time-sharing system.  Later, \link{Baker (1978)} developed a ``real-time''
version of the method, which does not require the computation to stop during
garbage collection.  Baker's idea was extended by Hewitt, Lieberman, and Moon
(see \link{Lieberman and Hewitt 1983}) to take advantage of the fact that some
structure is more volatile and other structure is more permanent.

An alternative commonly used garbage-collection technique is the
\newterm{mark-sweep} method.  This consists of tracing all the structure
accessible from the machine registers and marking each pair we reach.  We then
scan all of memory, and any location that is unmarked is ``swept up'' as
garbage and made available for reuse.  A full discussion of the mark-sweep
method can be found in \link{Allen 1978}.

The Minsky-Fenichel-Yochelson algorithm is the dominant algorithm in use for
large-memory systems because it examines only the useful part of memory.  This
is in contrast to mark-sweep, in which the sweep phase must check all of
memory.  A second advantage of stop-and-copy is that it is a
\newterm{compacting} garbage collector.  That is, at the end of the
garbage-collection phase the useful data will have been moved to consecutive
memory locations, with all garbage pairs compressed out.  This can be an
extremely important performance consideration in machines with virtual memory,
in which accesses to widely separated memory addresses may require extra paging
operations.}

\subsubsection*{Implementation of a stop-and-copy garbage collector}

We now use our register-machine language to describe the stop-and-copy
algorithm in more detail.  We will assume that there is a register called
\code{root} that contains a pointer to a structure that eventually points at
all accessible data.  This can be arranged by storing the contents of all the
machine registers in a pre-allocated list pointed at by \code{root} just before
starting garbage collection.\footnote{This list of registers does not include
the registers used by the storage-allocation system---\code{root},
\code{the\-/cars}, \code{the\-/cdrs}, and the other registers that will be
introduced in this section.} We also assume that, in addition to the current
working memory, there is free memory available into which we can copy the
useful data.  The current working memory consists of vectors whose base
addresses are in registers called \code{the\-/cars} and \code{the\-/cdrs}, and the
free memory is in registers called \code{new\-/cars} and \code{new\-/cdrs}.

Garbage collection is triggered when we exhaust the free cells in the current
working memory, that is, when a \code{cons} operation attempts to increment the
\code{free} pointer beyond the end of the memory vector.  When the
garbage-collection process is complete, the \code{root} pointer will point into
the new memory, all objects accessible from the \code{root} will have been
moved to the new memory, and the \code{free} pointer will indicate the next
place in the new memory where a new pair can be allocated.  In addition, the
roles of working memory and new memory will have been interchanged---new pairs
will be constructed in the new memory, beginning at the place indicated by
\code{free}, and the (previous) working memory will be available as the new
memory for the next garbage collection.  \link{Figure 5.15} shows the
arrangement of memory just before and just after garbage collection.

\begin{figure}[tp]
\phantomsection\label{Figure 5.15}
\centering
\begin{comment}
\begin{quote}
\heading{Figure 5.15:} Reconfiguration of memory by the garbage-collection process.

\begin{example}
             Just before garbage collection

         +------------------------------------+
the-cars |                                    | working
         | mixture of useful data and garbage | memory
the-cdrs |                                    |
         +------------------------------------+
                                            ^
                                            | free

         +------------------------------------+
new-cars |                                    | free   
         |            free memory             | memory
new-cdrs |                                    |
         +------------------------------------+

             Just after garbage collection

         +------------------------------------+
new-cars |                                    | new  
         |          discarded memory          | free  
new-cdrs |                                    | memory
         +------------------------------------+

         +------------------+-----------------+
the-cars |                  |                 | new
         |   useful data    |    free area    | working
the-cdrs |                  |                 | memory
         +------------------+-----------------+
                              ^
                              | free
\end{example}
\end{quote}
\end{comment}
\includegraphics[width=91mm]{fig/chap5/Fig5.15a.pdf}
\begin{quote}
\heading{Figure 5.15:} Reconfiguration of memory by the garbage-collection process.
\end{quote}
\end{figure}

The state of the garbage-collection process is controlled by maintaining two
pointers: \code{free} and \code{scan}.  These are initialized to point to the
beginning of the new memory.  The algorithm begins by relocating the pair
pointed at by \code{root} to the beginning of the new memory.  The pair is
copied, the \code{root} pointer is adjusted to point to the new location, and
the \code{free} pointer is incremented.  In addition, the old location of the
pair is marked to show that its contents have been moved.  This marking is done
as follows: In the \code{car} position, we place a special tag that signals
that this is an already-moved object.  (Such an object is traditionally called
a \newterm{broken heart}.)\footnote{The term \emph{broken heart} was coined by
David Cressey, who wrote a garbage collector for MDL, a dialect of Lisp
developed at \acronym{MIT} during the early 1970s.}  In the \code{cdr} position
we place a \newterm{forwarding address} that points at the location to which
the object has been moved.

After relocating the root, the garbage collector enters its basic cycle.  At
each step in the algorithm, the \code{scan} pointer (initially pointing at the
relocated root) points at a pair that has been moved to the new memory but
whose \code{car} and \code{cdr} pointers still refer to objects in the old
memory.  These objects are each relocated, and the \code{scan} pointer is
incremented.  To relocate an object (for example, the object indicated by the
\code{car} pointer of the pair we are scanning) we check to see if the object
has already been moved (as indicated by the presence of a broken-heart tag in
the \code{car} position of the object).  If the object has not already been
moved, we copy it to the place indicated by \code{free}, update \code{free},
set up a broken heart at the object's old location, and update the pointer to
the object (in this example, the \code{car} pointer of the pair we are
scanning) to point to the new location.  If the object has already been moved,
its forwarding address (found in the \code{cdr} position of the broken heart)
is substituted for the pointer in the pair being scanned.  Eventually, all
accessible objects will have been moved and scanned, at which point the
\code{scan} pointer will overtake the \code{free} pointer and the process will
terminate.

We can specify the stop-and-copy algorithm as a sequence of instructions for a
register machine.  The basic step of relocating an object is accomplished by a
subroutine called \code{relocate\-/old\-/result\-/in\-/new}.  This subroutine gets its
argument, a pointer to the object to be relocated, from a register named
\code{old}.  It relocates the designated object (incrementing \code{free} in
the process), puts a pointer to the relocated object into a register called
\code{new}, and returns by branching to the entry point stored in the register
\code{relocate\-/continue}.  To begin garbage collection, we invoke this
subroutine to relocate the \code{root} pointer, after initializing \code{free}
and \code{scan}.  When the relocation of \code{root} has been accomplished, we
install the new pointer as the new \code{root} and enter the main loop of the
garbage collector.

\begin{scheme}
begin-garbage-collection
  (assign free (const 0))
  (assign scan (const 0))
  (assign old (reg root))
  (assign relocate-continue (label reassign-root))
  (goto (label relocate-old-result-in-new))
reassign-root
  (assign root (reg new))
  (goto (label gc-loop))
\end{scheme}

\noindent
In the main loop of the garbage collector we must determine whether there are
any more objects to be scanned.  We do this by testing whether the \code{scan}
pointer is coincident with the \code{free} pointer.  If the pointers are equal,
then all accessible objects have been relocated, and we branch to
\code{gc\-/flip}, which cleans things up so that we can continue the interrupted
computation.  If there are still pairs to be scanned, we call the relocate
subroutine to relocate the \code{car} of the next pair (by placing the
\code{car} pointer in \code{old}).  The \code{relocate\-/continue} register is
set up so that the subroutine will return to update the \code{car} pointer.

\begin{scheme}
gc-loop
  (test (op =) (reg scan) (reg free))
  (branch (label gc-flip))
  (assign old (op vector-ref) (reg new-cars) (reg scan))
  (assign relocate-continue (label update-car))
  (goto (label relocate-old-result-in-new))
\end{scheme}

\noindent
At \code{update\-/car}, we modify the \code{car} pointer of the pair being
scanned, then proceed to relocate the \code{cdr} of the pair.  We return to
\code{update\-/cdr} when that relocation has been accomplished.  After relocating
and updating the \code{cdr}, we are finished scanning that pair, so we continue
with the main loop.

\begin{scheme}
update-car
  (perform (op vector-set!)
           (reg new-cars)
           (reg scan)
           (reg new))
  (assign old (op vector-ref) (reg new-cdrs) (reg scan))
  (assign relocate-continue (label update-cdr))
  (goto (label relocate-old-result-in-new))
update-cdr
  (perform (op vector-set!)
           (reg new-cdrs)
           (reg scan)
           (reg new))
  (assign scan (op +) (reg scan) (const 1))
  (goto (label gc-loop))
\end{scheme}

\noindent
The subroutine \code{relocate\-/old\-/result\-/in\-/new} relocates objects as follows:
If the object to be relocated (pointed at by \code{old}) is not a pair, then we
return the same pointer to the object unchanged (in \code{new}).  (For example,
we may be scanning a pair whose \code{car} is the number 4.  If we represent
the \code{car} by \code{n4}, as described in \link{Section 5.3.1}, then we want
the ``relocated'' \code{car} pointer to still be \code{n4}.)  Otherwise, we
must perform the relocation.  If the \code{car} position of the pair to be
relocated contains a broken-heart tag, then the pair has in fact already been
moved, so we retrieve the forwarding address (from the \code{cdr} position of
the broken heart) and return this in \code{new}.  If the pointer in \code{old}
points at a yet-unmoved pair, then we move the pair to the first free cell in
new memory (pointed at by \code{free}) and set up the broken heart by storing a
broken-heart tag and forwarding address at the old location.
\code{Relocate\-/old\-/result\-/in\-/new} uses a register \code{oldcr} to hold the
\code{car} or the \code{cdr} of the object pointed at by
\code{old}.\footnote{The garbage collector uses the low-level predicate
\code{pointer\-/to\-/pair?} instead of the list-structure \code{pair?}  operation
because in a real system there might be various things that are treated as
pairs for garbage-collection purposes.  For example, in a Scheme system that
conforms to the \acronym{IEEE} standard a procedure object may be implemented
as a special kind of ``pair'' that doesn't satisfy the \code{pair?} predicate.
For simulation purposes, \code{pointer\-/to\-/pair?} can be implemented as
\code{pair?}.}

\begin{scheme}
relocate-old-result-in-new
  (test (op pointer-to-pair?) (reg old))
  (branch (label pair))
  (assign new (reg old))
  (goto (reg relocate-continue))
pair
  (assign oldcr (op vector-ref) (reg the-cars) (reg old))
  (test (op broken-heart?) (reg oldcr))
  (branch (label already-moved))
  (assign new (reg free)) ~\textrm{; new location for pair}~
  ~\textrm{;; Update \code{free} pointer.}~
  (assign free (op +) (reg free) (const 1))
  ~\textrm{;; Copy the \code{car} and \code{cdr} to new memory.}~
  (perform (op vector-set!)
           (reg new-cars) (reg new) (reg oldcr))
  (assign oldcr (op vector-ref) (reg the-cdrs) (reg old))
  (perform (op vector-set!)
           (reg new-cdrs) (reg new) (reg oldcr))
  ~\textrm{;; Construct the broken heart.}~
  (perform (op vector-set!)
           (reg the-cars) (reg old) (const broken-heart))
  (perform
   (op vector-set!) (reg the-cdrs) (reg old) (reg new))
  (goto (reg relocate-continue))
already-moved
  (assign new (op vector-ref) (reg the-cdrs) (reg old))
  (goto (reg relocate-continue))
\end{scheme}

\noindent
At the very end of the garbage-collection process, we interchange the role of
old and new memories by interchanging pointers: interchanging \code{the\-/cars}
with \code{new\-/cars}, and \code{the\-/cdrs} with \code{new\-/cdrs}.  We will then
be ready to perform another garbage collection the next time memory runs out.

\begin{scheme}
gc-flip
  (assign temp (reg the-cdrs))
  (assign the-cdrs (reg new-cdrs))
  (assign new-cdrs (reg temp))
  (assign temp (reg the-cars))
  (assign the-cars (reg new-cars))
  (assign new-cars (reg temp))
\end{scheme}

\section{The Explicit-Control Evaluator}
\label{Section 5.4}

In \link{Section 5.1} we saw how to transform simple Scheme programs into
descriptions of register machines.  We will now perform this transformation on
a more complex program, the metacircular evaluator of 
\link{Section 4.1.1}--\link{Section 4.1.4}, which shows how the behavior of a Scheme interpreter
can be described in terms of the procedures \code{eval} and \code{apply}.  The
\newterm{explicit-control evaluator} that we develop in this section shows how
the underlying procedure-calling and argument-passing mechanisms used in the
evaluation process can be described in terms of operations on registers and
stacks.  In addition, the explicit-control evaluator can serve as an
implementation of a Scheme interpreter, written in a language that is very
similar to the native machine language of conventional computers.  The
evaluator can be executed by the register-machine simulator of 
\link{Section 5.2}.  Alternatively, it can be used as a starting point for building a
machine-language implementation of a Scheme evaluator, or even a
special-purpose machine for evaluating Scheme expressions.  \link{Figure 5.16}
shows such a hardware implementation: a silicon chip that acts as an evaluator
for Scheme.  The chip designers started with the data-path and controller
specifications for a register machine similar to the evaluator described in
this section and used design automation programs to construct the
integrated-circuit layout.\footnote{See \link{Batali et al. 1982} for more information
on the chip and the method by which it was designed.}

\begin{figure}[tb]
\phantomsection\label{Figure 5.16}
\centering
\begin{comment}
\heading{Figure 5.16:} A silicon-chip implementation of an evaluator for Scheme.

[A photograph of chip layout]

\end{comment}
\includegraphics[width=91mm]{fig/chap5/chip.jpg}
\begin{quote}
\heading{Figure 5.16:} A silicon-chip implementation of an evaluator for Scheme.
\end{quote}
\end{figure}

\subsubsection*{Registers and operations}

In designing the explicit-control evaluator, we must specify the operations to
be used in our register machine.  We described the metacircular evaluator in
terms of abstract syntax, using procedures such as \code{quoted?} and
\code{make\-/procedure}.  In implementing the register machine, we could expand
these procedures into sequences of elementary list-structure memory operations,
and implement these operations on our register machine.  However, this would
make our evaluator very long, obscuring the basic structure with details.  To
clarify the presentation, we will include as primitive operations of the
register machine the syntax procedures given in \link{Section 4.1.2} and the
procedures for representing environments and other run-time data given in
sections \link{Section 4.1.3} and \link{Section 4.1.4}.  In order to completely specify an
evaluator that could be programmed in a low-level machine language or
implemented in hardware, we would replace these operations by more elementary
operations, using the list-structure implementation we described in 
\link{Section 5.3}.

Our Scheme evaluator register machine includes a stack and seven registers:
\code{exp}, \code{env}, \code{val}, \code{continue}, \code{proc}, \code{argl},
and \code{unev}.  \code{Exp} is used to hold the expression to be evaluated,
and \code{env} contains the environment in which the evaluation is to be
performed.  At the end of an evaluation, \code{val} contains the value obtained
by evaluating the expression in the designated environment.  The
\code{continue} register is used to implement recursion, as explained in
\link{Section 5.1.4}.  (The evaluator needs to call itself recursively, since
evaluating an expression requires evaluating its subexpressions.)  The
registers \code{proc}, \code{argl}, and \code{unev} are used in evaluating
combinations.

We will not provide a data-path diagram to show how the registers and
operations of the evaluator are connected, nor will we give the complete list
of machine operations.  These are implicit in the evaluator's controller, which
will be presented in detail.



\subsection{The Core of the Explicit-Control Evaluator}
\label{Section 5.4.1}

The central element in the evaluator is the sequence of instructions beginning
at \code{eval\-/dispatch}.  This corresponds to the \code{eval} procedure of the
metacircular evaluator described in \link{Section 4.1.1}.  When the controller
starts at \code{eval\-/dispatch}, it evaluates the expression specified by
\code{exp} in the environment specified by \code{env}.  When evaluation is
complete, the controller will go to the entry point stored in \code{continue},
and the \code{val} register will hold the value of the expression.  As with the
metacircular \code{eval}, the structure of \code{eval\-/dispatch} is a case
analysis on the syntactic type of the expression to be evaluated.\footnote{In
our controller, the dispatch is written as a sequence of \code{test} and
\code{branch} instructions.  Alternatively, it could have been written in a
data-directed style (and in a real system it probably would have been) to avoid
the need to perform sequential tests and to facilitate the definition of new
expression types.  A machine designed to run Lisp would probably include a
\code{dispatch\-/on\-/type} instruction that would efficiently execute such
data-directed dispatches.}

\begin{scheme}
eval-dispatch
  (test (op self-evaluating?) (reg exp))
  (branch (label ev-self-eval))
  (test (op variable?) (reg exp))
  (branch (label ev-variable))
  (test (op quoted?) (reg exp))
  (branch (label ev-quoted))
  (test (op assignment?) (reg exp))
  (branch (label ev-assignment))
  (test (op definition?) (reg exp))
  (branch (label ev-definition))
  (test (op if?) (reg exp))
  (branch (label ev-if))
  (test (op lambda?) (reg exp))
  (branch (label ev-lambda))
  (test (op begin?) (reg exp))
  (branch (label ev-begin))
  (test (op application?) (reg exp))
  (branch (label ev-application))
  (goto (label unknown-expression-type))
\end{scheme}

\subsubsection*{Evaluating simple expressions}

Numbers and strings (which are self-evaluating), variables, quotations, and
\code{lambda} expressions have no subexpressions to be evaluated.  For these,
the evaluator simply places the correct value in the \code{val} register and
continues execution at the entry point specified by \code{continue}.
Evaluation of simple expressions is performed by the following controller code:

\begin{scheme}
ev-self-eval
  (assign val (reg exp))
  (goto (reg continue))
ev-variable
  (assign val (op lookup-variable-value) (reg exp) (reg env))
  (goto (reg continue))
ev-quoted
  (assign val (op text-of-quotation) (reg exp))
  (goto (reg continue))
ev-lambda
  (assign unev (op lambda-parameters) (reg exp))
  (assign exp (op lambda-body) (reg exp))
  (assign val (op make-procedure) (reg unev) (reg exp) (reg env))
  (goto (reg continue))
\end{scheme}

\noindent
Observe how \code{ev\-/lambda} uses the \code{unev} and \code{exp} registers to
hold the parameters and body of the lambda expression so that they can be
passed to the \code{make\-/procedure} operation, along with the environment in
\code{env}.

\subsubsection*{Evaluating procedure applications}

A procedure application is specified by a combination containing an operator
and operands.  The operator is a subexpression whose value is a procedure, and
the operands are subexpressions whose values are the arguments to which the
procedure should be applied.  The metacircular \code{eval} handles applications
by calling itself recursively to evaluate each element of the combination, and
then passing the results to \code{apply}, which performs the actual procedure
application.  The explicit-control evaluator does the same thing; these
recursive calls are implemented by \code{goto} instructions, together with use
of the stack to save registers that will be restored after the recursive call
returns.  Before each call we will be careful to identify which registers must
be saved (because their values will be needed later).\footnote{This is an
important but subtle point in translating algorithms from a procedural
language, such as Lisp, to a register-machine language.  As an alternative to
saving only what is needed, we could save all the registers (except \code{val})
before each recursive call. This is called a \newterm{framed-stack} discipline.
This would work but might save more registers than necessary; this could be an
important consideration in a system where stack operations are expensive.
Saving registers whose contents will not be needed later may also hold onto
useless data that could otherwise be garbage-collected, freeing space to be
reused.}

We begin the evaluation of an application by evaluating the operator to produce
a procedure, which will later be applied to the evaluated operands.  To
evaluate the operator, we move it to the \code{exp} register and go to
\code{eval\-/dispatch}.  The environment in the \code{env} register is already
the correct one in which to evaluate the operator.  However, we save \code{env}
because we will need it later to evaluate the operands.  We also extract the
operands into \code{unev} and save this on the stack.  We set up
\code{continue} so that \code{eval\-/dispatch} will resume at
\code{ev\-/appl\-/did\-/operator} after the operator has been evaluated.  First,
however, we save the old value of \code{continue}, which tells the controller
where to continue after the application.

\enlargethispage{\baselineskip}

\begin{scheme}
ev-application
  (save continue)
  (save env)
  (assign unev (op operands) (reg exp))
  (save unev)
  (assign exp (op operator) (reg exp))
  (assign continue (label ev-appl-did-operator))
  (goto (label eval-dispatch))
\end{scheme}

\noindent
Upon returning from evaluating the operator subexpression, we proceed to
evaluate the operands of the combination and to accumulate the resulting
arguments in a list, held in \code{argl}.  First we restore the unevaluated
operands and the environment.  We initialize \code{argl} to an empty list.
Then we assign to the \code{proc} register the procedure that was produced by
evaluating the operator.  If there are no operands, we go directly to
\code{apply\-/dispatch}.  Otherwise we save \code{proc} on the stack and start
the argument-evaluation loop:\footnote{We add to the evaluator data-structure
procedures in \link{Section 4.1.3} the following two procedures for manipulating
argument lists:

\begin{smallscheme}
(define (empty-arglist) '())
(define (adjoin-arg arg arglist) (append arglist (list arg)))
\end{smallscheme}

We also use an additional syntax procedure to test for the last operand in a
combination:

\begin{smallscheme}
(define (last-operand? ops) (null? (cdr ops)))
\end{smallscheme} }

\begin{scheme}
ev-appl-did-operator
  (restore unev)                       ~\textrm{; the operands}~
  (restore env)
  (assign argl (op empty-arglist))
  (assign proc (reg val))              ~\textrm{; the operator}~
  (test (op no-operands?) (reg unev))
  (branch (label apply-dispatch))
  (save proc)
\end{scheme}

\noindent
Each cycle of the argument-evaluation loop evaluates an operand from the list
in \code{unev} and accumulates the result into \code{argl}.  To evaluate an
operand, we place it in the \code{exp} register and go to \code{eval\-/dispatch},
after setting \code{continue} so that execution will resume with the
argument-accumulation phase.  But first we save the arguments accumulated so
far (held in \code{argl}), the environment (held in \code{env}), and the
remaining operands to be evaluated (held in \code{unev}).  A special case is
made for the evaluation of the last operand, which is handled at
\code{ev\-/appl\-/last\-/arg}.

\begin{scheme}
ev-appl-operand-loop
  (save argl)
  (assign exp (op first-operand) (reg unev))
  (test (op last-operand?) (reg unev))
  (branch (label ev-appl-last-arg))
  (save env)
  (save unev)
  (assign continue (label ev-appl-accumulate-arg))
  (goto (label eval-dispatch))
\end{scheme}

\noindent
When an operand has been evaluated, the value is accumulated into the list held
in \code{argl}.  The operand is then removed from the list of unevaluated
operands in \code{unev}, and the argument-evaluation continues.

\begin{scheme}
ev-appl-accumulate-arg
  (restore unev)
  (restore env)
  (restore argl)
  (assign argl (op adjoin-arg) (reg val) (reg argl))
  (assign unev (op rest-operands) (reg unev))
  (goto (label ev-appl-operand-loop))
\end{scheme}

\noindent
Evaluation of the last argument is handled differently.  There is no need to
save the environment or the list of unevaluated operands before going to
\code{eval\-/dispatch}, since they will not be required after the last operand is
evaluated.  Thus, we return from the evaluation to a special entry point
\code{ev\-/appl\-/accum\-/last\-/arg}, which restores the argument list, accumulates
the new argument, restores the saved procedure, and goes off to perform the
application.\footnote{The optimization of treating the last operand specially
is known as \newterm{evlis tail recursion} (see \link{Wand 1980}).  We could be
somewhat more efficient in the argument evaluation loop if we made evaluation
of the first operand a special case too.  This would permit us to postpone
initializing \code{argl} until after evaluating the first operand, so as to
avoid saving \code{argl} in this case.  The compiler in \link{Section 5.5}
performs this optimization.  (Compare the \code{construct\-/arglist} procedure of
\link{Section 5.5.3}.)}

\begin{scheme}
ev-appl-last-arg
  (assign continue (label ev-appl-accum-last-arg))
  (goto (label eval-dispatch))
ev-appl-accum-last-arg
  (restore argl)
  (assign argl (op adjoin-arg) (reg val) (reg argl))
  (restore proc)
  (goto (label apply-dispatch))
\end{scheme}

\noindent
The details of the argument-evaluation loop determine the order in which the
interpreter evaluates the operands of a combination (e.g., left to right or
right to left---see \link{Exercise 3.8}).  This order is not determined by the
metacircular evaluator, which inherits its control structure from the
underlying Scheme in which it is implemented.\footnote{The order of operand
evaluation in the metacircular evaluator is determined by the order of
evaluation of the arguments to \code{cons} in the procedure
\code{list\-/of\-/values} of \link{Section 4.1.1} (see \link{Exercise 4.1}).} Because
the \code{first\-/operand} selector (used in \code{ev\-/appl\-/operand\-/loop} to
extract successive operands from \code{unev}) is implemented as \code{car} and
the \code{rest\-/operands} selector is implemented as \code{cdr}, the
explicit-control evaluator will evaluate the operands of a combination in
left-to-right order.

\subsubsection*{Procedure application}

The entry point \code{apply\-/dispatch} corresponds to the \code{apply} procedure
of the metacircular evaluator.  By the time we get to \code{apply\-/dispatch},
the \code{proc} register contains the procedure to apply and \code{argl}
contains the list of evaluated arguments to which it must be applied.  The
saved value of \code{continue} (originally passed to \code{eval\-/dispatch} and
saved at \code{ev\-/application}), which tells where to return with the result of
the procedure application, is on the stack.  When the application is complete,
the controller transfers to the entry point specified by the saved
\code{continue}, with the result of the application in \code{val}.  As with the
metacircular \code{apply}, there are two cases to consider.  Either the
procedure to be applied is a primitive or it is a compound procedure.

\begin{scheme}
apply-dispatch
  (test (op primitive-procedure?) (reg proc))
  (branch (label primitive-apply))
  (test (op compound-procedure?) (reg proc))
  (branch (label compound-apply))
  (goto (label unknown-procedure-type))
\end{scheme}

\noindent
We assume that each primitive is implemented so as to obtain its arguments from
\code{argl} and place its result in \code{val}.  To specify how the machine
handles primitives, we would have to provide a sequence of controller
instructions to implement each primitive and arrange for \code{primitive\-/apply}
to dispatch to the instructions for the primitive identified by the contents of
\code{proc}.  Since we are interested in the structure of the evaluation
process rather than the details of the primitives, we will instead just use an
\code{apply\-/primitive\-/procedure} operation that applies the procedure in
\code{proc} to the arguments in \code{argl}.  For the purpose of simulating the
evaluator with the simulator of \link{Section 5.2} we use the procedure
\code{apply\-/primitive\-/procedure}, which calls on the underlying Scheme system
to perform the application, just as we did for the metacircular evaluator in
\link{Section 4.1.4}.  After computing the value of the primitive application,
we restore \code{continue} and go to the designated entry point.

\begin{scheme}

primitive-apply
  (assign val (op apply-primitive-procedure)
              (reg proc)
              (reg argl))
  (restore continue)
  (goto (reg continue))
\end{scheme}

\noindent
To apply a compound procedure, we proceed just as with the metacircular
evaluator.  We construct a frame that binds the procedure's parameters to the
arguments, use this frame to extend the environment carried by the procedure,
and evaluate in this extended environment the sequence of expressions that
forms the body of the procedure.  \code{Ev\-/sequence}, described below in
\link{Section 5.4.2}, handles the evaluation of the sequence.

\begin{scheme}
compound-apply
  (assign unev (op procedure-parameters) (reg proc))
  (assign env (op procedure-environment) (reg proc))
  (assign env (op extend-environment)
              (reg unev) (reg argl) (reg env))
  (assign unev (op procedure-body) (reg proc))
  (goto (label ev-sequence))
\end{scheme}

\noindent
\code{Compound\-/apply} is the only place in the interpreter where the \code{env}
register is ever assigned a new value.  Just as in the metacircular evaluator,
the new environment is constructed from the environment carried by the
procedure, together with the argument list and the corresponding list of
variables to be bound.

\subsection{Sequence Evaluation and Tail Recursion}
\label{Section 5.4.2}

The portion of the explicit-control evaluator at \code{ev\-/sequence} is
analogous to the metacircular evaluator's \code{eval\-/sequence} procedure.  It
handles sequences of expressions in procedure bodies or in explicit
\code{begin} expressions.

Explicit \code{begin} expressions are evaluated by placing the sequence of
expressions to be evaluated in \code{unev}, saving \code{continue} on the
stack, and jumping to \code{ev\-/sequence}.

\begin{scheme}
ev-begin
  (assign unev (op begin-actions) (reg exp))
  (save continue)
  (goto (label ev-sequence))
\end{scheme}

\noindent
The implicit sequences in procedure bodies are handled by jumping to
\code{ev\-/sequence} from \code{compound\-/apply}, at which point \code{continue}
is already on the stack, having been saved at \code{ev\-/application}.

The entries at \code{ev\-/sequence} and \code{ev\-/sequence\-/continue} form a loop
that successively evaluates each expression in a sequence.  The list of
unevaluated expressions is kept in \code{unev}.  Before evaluating each
expression, we check to see if there are additional expressions to be evaluated
in the sequence.  If so, we save the rest of the unevaluated expressions (held
in \code{unev}) and the environment in which these must be evaluated (held in
\code{env}) and call \code{eval\-/dispatch} to evaluate the expression.  The two
saved registers are restored upon the return from this evaluation, at
\code{ev\-/sequence\-/continue}.

The final expression in the sequence is handled differently, at the entry point
\code{ev\-/sequence\-/last\-/exp}.  Since there are no more expressions to be
evaluated after this one, we need not save \code{unev} or \code{env} before
going to \code{eval\-/dispatch}.  The value of the whole sequence is the value of
the last expression, so after the evaluation of the last expression there is
nothing left to do except continue at the entry point currently held on the
stack (which was saved by \code{ev\-/application} or \code{ev\-/begin}.)  Rather
than setting up \code{continue} to arrange for \code{eval\-/dispatch} to return
here and then restoring \code{continue} from the stack and continuing at that
entry point, we restore \code{continue} from the stack before going to
\code{eval\-/dispatch}, so that \code{eval\-/dispatch} will continue at that entry
point after evaluating the expression.

\begin{scheme}
ev-sequence
  (assign exp (op first-exp) (reg unev))
  (test (op last-exp?) (reg unev))
  (branch (label ev-sequence-last-exp))
  (save unev)
  (save env)
  (assign continue (label ev-sequence-continue))
  (goto (label eval-dispatch))
ev-sequence-continue
  (restore env)
  (restore unev)
  (assign unev (op rest-exps) (reg unev))
  (goto (label ev-sequence))
ev-sequence-last-exp
  (restore continue)
  (goto (label eval-dispatch))
\end{scheme}

\subsubsection*{Tail recursion}

In \link{Chapter 1} we said that the process described by a procedure such as

\begin{scheme}
(define (sqrt-iter guess x)
  (if (good-enough? guess x)
      guess
      (sqrt-iter (improve guess x) x)))
\end{scheme}

\noindent
is an iterative process.  Even though the procedure is syntactically recursive
(defined in terms of itself), it is not logically necessary for an evaluator to
save information in passing from one call to \code{sqrt\-/iter} to the
next.\footnote{We saw in \link{Section 5.1} how to implement such a process with
a register machine that had no stack; the state of the process was stored in a
fixed set of registers.} An evaluator that can execute a procedure such as
\code{sqrt\-/iter} without requiring increasing storage as the procedure
continues to call itself is called a \newterm{tail-recursive} evaluator.  The
metacircular implementation of the evaluator in \link{Chapter 4} does not
specify whether the evaluator is tail-recursive, because that evaluator
inherits its mechanism for saving state from the underlying Scheme.  With the
explicit-control evaluator, however, we can trace through the evaluation
process to see when procedure calls cause a net accumulation of information on
the stack.

Our evaluator is tail-recursive, because in order to evaluate the final
expression of a sequence we transfer directly to \code{eval\-/dispatch} without
saving any information on the stack.  Hence, evaluating the final expression in
a sequence---even if it is a procedure call (as in \code{sqrt\-/iter}, where the
\code{if} expression, which is the last expression in the procedure body,
reduces to a call to \code{sqrt\-/iter})---will not cause any information to be
accumulated on the stack.\footnote{This implementation of tail recursion in
\code{ev\-/sequence} is one variety of a well-known optimization technique used
by many compilers.  In compiling a procedure that ends with a procedure call,
one can replace the call by a jump to the called procedure's entry point.
Building this strategy into the interpreter, as we have done in this section,
provides the optimization uniformly throughout the language.}

\enlargethispage{\baselineskip}

If we did not think to take advantage of the fact that it was unnecessary to
save information in this case, we might have implemented \code{eval\-/sequence}
by treating all the expressions in a sequence in the same way---saving the
registers, evaluating the expression, returning to restore the registers, and
repeating this until all the expressions have been evaluated:\footnote{We can
define \code{no\-/more\-/exps?} as follows:

\begin{smallscheme}
(define (no-more-exps? seq) (null? seq))
\end{smallscheme}
}

\begin{scheme}
ev-sequence
  (test (op no-more-exps?) (reg unev))
  (branch (label ev-sequence-end))
  (assign exp (op first-exp) (reg unev))
  (save unev)
  (save env)
  (assign continue (label ev-sequence-continue))
  (goto (label eval-dispatch))
ev-sequence-continue
  (restore env)
  (restore unev)
  (assign unev (op rest-exps) (reg unev))
  (goto (label ev-sequence))
ev-sequence-end
  (restore continue)
  (goto (reg continue))
\end{scheme}

\enlargethispage{\baselineskip}

\noindent
This may seem like a minor change to our previous code for evaluation of a
sequence: The only difference is that we go through the save-restore cycle for
the last expression in a sequence as well as for the others.  The interpreter
will still give the same value for any expression.  But this change is fatal to
the tail-recursive implementation, because we must now return after evaluating
the final expression in a sequence in order to undo the (useless) register
saves.  These extra saves will accumulate during a nest of procedure calls.
Consequently, processes such as \code{sqrt\-/iter} will require space
proportional to the number of iterations rather than requiring constant space.
This difference can be significant.  For example, with tail recursion, an
infinite loop can be expressed using only the procedure-call mechanism:

\begin{scheme}
(define (count n)
  (newline) (display n) (count (+ n 1)))
\end{scheme}

\noindent
Without tail recursion, such a procedure would eventually run out of stack
space, and expressing a true iteration would require some control mechanism
other than procedure call.

\subsection{Conditionals, Assignments, and Definitions}
\label{Section 5.4.3}

As with the metacircular evaluator, special forms are handled by selectively
evaluating fragments of the expression.  For an \code{if} expression, we must
evaluate the predicate and decide, based on the value of predicate, whether to
evaluate the consequent or the alternative.

Before evaluating the predicate, we save the \code{if} expression itself so
that we can later extract the consequent or alternative.  We also save the
environment, which we will need later in order to evaluate the consequent or
the alternative, and we save \code{continue}, which we will need later in order
to return to the evaluation of the expression that is waiting for the value of
the \code{if}.

\begin{scheme}
ev-if
  (save exp)                    ~\textrm{; save expression for later}~
  (save env)
  (save continue)
  (assign continue (label ev-if-decide))
  (assign exp (op if-predicate) (reg exp))
  (goto (label eval-dispatch))  ~\textrm{; evaluate the predicate}~
\end{scheme}

\noindent
When we return from evaluating the predicate, we test whether it was true or
false and, depending on the result, place either the consequent or the
alternative in \code{exp} before going to \code{eval\-/dispatch}.  Notice that
restoring \code{env} and \code{continue} here sets up \code{eval\-/dispatch} to
have the correct environment and to continue at the right place to receive the
value of the \code{if} expression.

\begin{scheme}

ev-if-decide
  (restore continue)
  (restore env)
  (restore exp)
  (test (op true?) (reg val))
  (branch (label ev-if-consequent))
ev-if-alternative
  (assign exp (op if-alternative) (reg exp))
  (goto (label eval-dispatch))
ev-if-consequent
  (assign exp (op if-consequent) (reg exp))
  (goto (label eval-dispatch))
\end{scheme}

\enlargethispage{\baselineskip}

\subsubsection*{Assignments and definitions}

Assignments are handled by \code{ev\-/assignment}, which is reached from
\code{eval\-/dispatch} with the assignment expression in \code{exp}.  The code at
\code{ev\-/assignment} first evaluates the value part of the expression and then
installs the new value in the environment.  \code{Set\-/variable\-/value!} is
assumed to be available as a machine operation.

\begin{scheme}
ev-assignment
  (assign unev (op assignment-variable) (reg exp))
  (save unev)                   ~\textrm{; save variable for later}~
  (assign exp (op assignment-value) (reg exp))
  (save env)
  (save continue)
  (assign continue (label ev-assignment-1))
  (goto (label eval-dispatch))  ~\textrm{; evaluate the assignment value}~
ev-assignment-1
  (restore continue)
  (restore env)
  (restore unev)
  (perform
   (op set-variable-value!) (reg unev) (reg val) (reg env))
  (assign val (const ok))
  (goto (reg continue))
\end{scheme}

\noindent
Definitions are handled in a similar way:

\begin{scheme}
ev-definition
  (assign unev (op definition-variable) (reg exp))
  (save unev)                   ~\textrm{; save variable for later}~
  (assign exp (op definition-value) (reg exp))
  (save env)
  (save continue)
  (assign continue (label ev-definition-1))
  (goto (label eval-dispatch))  ~\textrm{; evaluate the definition value}~
ev-definition-1
  (restore continue)
  (restore env)
  (restore unev)
  (perform
   (op define-variable!) (reg unev) (reg val) (reg env))
  (assign val (const ok))
  (goto (reg continue))
\end{scheme}

\begin{quote}
\heading{\phantomsection\label{Exercise 5.23}Exercise 5.23:} Extend the evaluator to handle
derived expressions such as \code{cond}, \code{let}, and so on 
(\link{Section 4.1.2}).  You may ``cheat'' and assume that the syntax transformers such
as \code{cond\-/>if} are available as machine operations.\footnote{This isn't
really cheating.  In an actual implementation built from scratch, we would use
our explicit-control evaluator to interpret a Scheme program that performs
source-level transformations like \code{cond\-/>if} in a syntax phase that runs
before execution.}
\end{quote}

\begin{quote}
\heading{\phantomsection\label{Exercise 5.24}Exercise 5.24:} Implement \code{cond} as a new
basic special form without reducing it to \code{if}.  You will have to
construct a loop that tests the predicates of successive \code{cond} clauses
until you find one that is true, and then use \code{ev\-/sequence} to evaluate
the actions of the clause.
\end{quote}

\begin{quote}
\heading{\phantomsection\label{Exercise 5.25}Exercise 5.25:} Modify the evaluator so that it
uses normal-order evaluation, based on the lazy evaluator of \link{Section 4.2}.
\end{quote}

\subsection{Running the Evaluator}
\label{Section 5.4.4}

With the implementation of the explicit-control evaluator we come to the end of
a development, begun in \link{Chapter 1}, in which we have explored successively
more precise models of the evaluation process.  We started with the relatively
informal substitution model, then extended this in \link{Chapter 3} to the
environment model, which enabled us to deal with state and change.  In the
metacircular evaluator of \link{Chapter 4}, we used Scheme itself as a language
for making more explicit the environment structure constructed during
evaluation of an expression.  Now, with register machines, we have taken a
close look at the evaluator's mechanisms for storage management, argument
passing, and control.  At each new level of description, we have had to raise
issues and resolve ambiguities that were not apparent at the previous, less
precise treatment of evaluation.  To understand the behavior of the
explicit-control evaluator, we can simulate it and monitor its performance.

We will install a driver loop in our evaluator machine.  This plays the role of
the \code{driver\-/loop} procedure of \link{Section 4.1.4}.  The evaluator will
repeatedly print a prompt, read an expression, evaluate the expression by going
to \code{eval\-/dispatch}, and print the result.  The following instructions form
the beginning of the explicit-control evaluator's controller
sequence:\footnote{We assume here that \code{read} and the various printing
operations are available as primitive machine operations, which is useful for
our simulation, but completely unrealistic in practice.  These are actually
extremely complex operations.  In practice, they would be implemented using
low-level input-output operations such as transferring single characters to and
from a device.

To support the \code{get\-/global\-/environment} operation we define

\begin{smallscheme}
(define the-global-environment (setup-environment))
(define (get-global-environment) the-global-environment)
\end{smallscheme}
}

\begin{scheme}
read-eval-print-loop
  (perform (op initialize-stack))
  (perform
   (op prompt-for-input) (const ";;EC-Eval input:"))
  (assign exp (op read))
  (assign env (op get-global-environment))
  (assign continue (label print-result))
  (goto (label eval-dispatch))
print-result
  (perform (op announce-output) (const ";;EC-Eval value:"))
  (perform (op user-print) (reg val))
  (goto (label read-eval-print-loop))
\end{scheme}

\noindent
When we encounter an error in a procedure (such as the ``unknown procedure type
error'' indicated at \code{apply\-/dispatch}), we print an error message and
return to the driver loop.\footnote{There are other errors that we would like
the interpreter to handle, but these are not so simple.  See \link{Exercise 5.30}.}

\begin{scheme}
unknown-expression-type
  (assign val (const unknown-expression-type-error))
  (goto (label signal-error))
unknown-procedure-type
  (restore continue)    ~\textrm{; clean up stack (from \code{apply\-/dispatch})}~
  (assign val (const unknown-procedure-type-error))
  (goto (label signal-error))
signal-error
  (perform (op user-print) (reg val))
  (goto (label read-eval-print-loop))
\end{scheme}

\noindent
For the purposes of the simulation, we initialize the stack each time through
the driver loop, since it might not be empty after an error (such as an
undefined variable) interrupts an evaluation.\footnote{We could perform the
stack initialization only after errors, but doing it in the driver loop will be
convenient for monitoring the evaluator's performance, as described below.}

If we combine all the code fragments presented in 
\link{Section 5.4.1}--\link{Section 5.4.4}, we can create an evaluator machine model that we can
run using the register-machine simulator of \link{Section 5.2}.

\begin{scheme}
(define eceval
  (make-machine
   '(exp env val proc argl continue unev)
   eceval-operations
   '(read-eval-print-loop
     ~\( \dark \langle \)~~\emph{entire machine controller as given above}~~\( \dark \rangle \)~ )))
\end{scheme}

\noindent
We must define Scheme procedures to simulate the operations used as primitives
by the evaluator.  These are the same procedures we used for the metacircular
evaluator in \link{Section 4.1}, together with the few additional ones defined
in footnotes throughout \link{Section 5.4}.

\begin{scheme}
(define eceval-operations
  (list (list 'self-evaluating? self-evaluating)
        ~\( \dark \langle \)~~\emph{complete list of operations for eceval machine}~~\( \dark \rangle \)~))
\end{scheme}

\noindent
Finally, we can initialize the global environment and run the evaluator:

\begin{scheme}
(define the-global-environment (setup-environment))
(start eceval)
~\textit{;;; EC-Eval input:}~
(define (append x y)
  (if (null? x) y (cons (car x) (append (cdr x) y))))
~\textit{;;; EC-Eval value:}~
~\textit{ok}~
~\textit{;;; EC-Eval input:}~
(append '(a b c) '(d e f))
~\textit{;;; EC-Eval value:}~
~\textit{(a b c d e f)}~
\end{scheme}

\noindent
Of course, evaluating expressions in this way will take much longer than if we
had directly typed them into Scheme, because of the multiple levels of
simulation involved.  Our expressions are evaluated by the
explicit-control-evaluator machine, which is being simulated by a Scheme
program, which is itself being evaluated by the Scheme interpreter.

\subsubsection*{Monitoring the performance of the evaluator}

Simulation can be a powerful tool to guide the implementation of evaluators.
Simulations make it easy not only to explore variations of the register-machine
design but also to monitor the performance of the simulated evaluator.  For
example, one important factor in performance is how efficiently the evaluator
uses the stack.  We can observe the number of stack operations required to
evaluate various expressions by defining the evaluator register machine with
the version of the simulator that collects statistics on stack use 
(\link{Section 5.2.4}), and adding an instruction at the evaluator's \code{print\-/result}
entry point to print the statistics:

\begin{scheme}
print-result
  (perform (op print-stack-statistics))   ~\textrm{; added instruction}~
  (perform
   (op announce-output) (const ";;; EC-Eval value:"))
  ~\( \dots \)~ ~\textrm{; same as before}~
\end{scheme}

\noindent
Interactions with the evaluator now look like this:

\begin{scheme}
~\textit{;;; EC-Eval input:}~
(define (factorial n)
  (if (= n 1) 1 (* (factorial (- n 1)) n)))
~\textit{(total-pushes = 3 maximum-depth = 3)}~
~\textit{;;; EC-Eval value:}~
~\textit{ok}~
~\textit{;;; EC-Eval input:}~
(factorial 5)
~\textit{(total-pushes = 144 maximum-depth = 28)}~
~\textit{;;; EC-Eval value:}~
~\textit{120}~
\end{scheme}

\noindent
Note that the driver loop of the evaluator reinitializes the stack at the start
of each interaction, so that the statistics printed will refer only to stack
operations used to evaluate the previous expression.

\begin{quote}
\heading{\phantomsection\label{Exercise 5.26}Exercise 5.26:} Use the monitored stack to
explore the tail-recursive property of the evaluator (\link{Section 5.4.2}).
Start the evaluator and define the iterative \code{factorial} procedure from
\link{Section 1.2.1}:

\begin{scheme}
(define (factorial n)
  (define (iter product counter)
    (if (> counter n)
        product
        (iter (* counter product) (+ counter 1))))
  (iter 1 1))
\end{scheme}

Run the procedure with some small values of \( n \).  Record the maximum stack
depth and the number of pushes required to compute \( n! \) for each of these
values.

\begin{enumerate}[a.]

\item
You will find that the maximum depth required to evaluate \( n! \) is independent
of \( n \).  What is that depth?

\item
Determine from your data a formula in terms of \( n \) for the total number of
push operations used in evaluating \( n! \) for any \( n \ge 1 \).  Note that the
number of operations used is a linear function of \( n \) and is thus determined
by two constants.

\end{enumerate}
\end{quote}

\begin{quote}
\heading{\phantomsection\label{Exercise 5.27}Exercise 5.27:} For comparison with \link{Exercise 5.26}, 
explore the behavior of the following procedure for computing factorials
recursively:

\begin{scheme}
(define (factorial n)
  (if (= n 1) 1 (* (factorial (- n 1)) n)))
\end{scheme}

By running this procedure with the monitored stack, determine, as a function of
\( n \), the maximum depth of the stack and the total number of pushes used in
evaluating \( n! \) for \( n \ge 1 \).  (Again, these functions will be linear.)
Summarize your experiments by filling in the following table with the
appropriate expressions in terms of \( n \):
\begin{comment}

\begin{example}
               Maximum depth       Number of pushes

Recursive
factorial

Iterative
factorial
\end{example}

\end{comment}
\begin{displaymath}
\vbox{
\offinterlineskip
\halign{
\strut 	\hfil \quad #\quad \hfil & \vrule 
	\hfil \quad #\quad \hfil & \vrule
	\hfil \quad #\quad \hfil \cr

				& Maximum depth & Number of pushes \cr
	\noalign{\hrule}
	Recursive 		&  		&  \cr
	factorial 		&  		&  \cr
	\noalign{\hrule}
	Iterative 		&  		&  \cr
	factorial 		&  		&  \cr
}
}
\end{displaymath}
The maximum depth is a measure of the amount of space used by the evaluator in
carrying out the computation, and the number of pushes correlates well with the
time required.
\end{quote}

\begin{quote}
\heading{\phantomsection\label{Exercise 5.28}Exercise 5.28:} Modify the definition of the
evaluator by changing \code{eval\-/sequence} as described in \link{Section 5.4.2}
so that the evaluator is no longer tail-recursive.  Rerun your experiments from
\link{Exercise 5.26} and \link{Exercise 5.27} to demonstrate that both versions
of the \code{factorial} procedure now require space that grows linearly with
their input.
\end{quote}

\begin{quote}
\heading{\phantomsection\label{Exercise 5.29}Exercise 5.29:} Monitor the stack operations in
the tree-recursive Fibonacci computation:

\begin{scheme}
(define (fib n)
  (if (< n 2)
      n
      (+ (fib (- n 1)) (fib (- n 2)))))
\end{scheme}

\begin{enumerate}[a.]

\item
Give a formula in terms of \( n \) for the maximum depth of the stack required to
compute \( {\rm Fib}(n) \) for \( n \ge 2 \).  Hint: In \link{Section 1.2.2} we
argued that the space used by this process grows linearly with \( n \).

\item
Give a formula for the total number of pushes used to compute \( {\rm Fib}(n) \)
for \( n \ge 2 \).  You should find that the number of pushes (which correlates
well with the time used) grows exponentially with \( n \).  Hint: Let
\( S(n) \) be the number of pushes used in computing \( {\rm Fib}(n) \).  You
should be able to argue that there is a formula that expresses \( S(n) \) in
terms of \( S(n - 1) \), \( S(n - 2) \), and some fixed ``overhead''
constant \( k \) that is independent of \( n \).  Give the formula, and say what
\( k \) is.  Then show that \( S(n) \) can be expressed as 
\( a\cdot{\rm Fib}(n + 1) + b \) and give the values of \( a \) and \( b \).

\end{enumerate}
\end{quote}

\begin{quote}
\heading{\phantomsection\label{Exercise 5.30}Exercise 5.30:} Our evaluator currently catches
and signals only two kinds of errors---unknown expression types and unknown
procedure types.  Other errors will take us out of the evaluator
read-eval-print loop.  When we run the evaluator using the register-machine
simulator, these errors are caught by the underlying Scheme system.  This is
analogous to the computer crashing when a user program makes an
error.\footnote{Regrettably, this is the normal state of affairs in
conventional compiler-based language systems such as C.  In \acronym{UNIX}(tm)
the system ``dumps core,'' and in \acronym{DOS}/Windows(tm) it becomes
catatonic.  The Macintosh(tm) displays a picture of an exploding bomb and
offers you the opportunity to reboot the computer---if you're lucky.}  It is a
large project to make a real error system work, but it is well worth the effort
to understand what is involved here.

\begin{enumerate}[a.]

\item
Errors that occur in the evaluation process, such as an attempt to access an
unbound variable, could be caught by changing the lookup operation to make it
return a distinguished condition code, which cannot be a possible value of any
user variable.  The evaluator can test for this condition code and then do what
is necessary to go to \code{signal\-/error}.  Find all of the places in the
evaluator where such a change is necessary and fix them.  This is lots of work.

\item
Much worse is the problem of handling errors that are signaled by applying
primitive procedures, such as an attempt to divide by zero or an attempt to
extract the \code{car} of a symbol.  In a professionally written high-quality
system, each primitive application is checked for safety as part of the
primitive.  For example, every call to \code{car} could first check that the
argument is a pair.  If the argument is not a pair, the application would
return a distinguished condition code to the evaluator, which would then report
the failure.  We could arrange for this in our register-machine simulator by
making each primitive procedure check for applicability and returning an
appropriate distinguished condition code on failure. Then the
\code{primitive\-/apply} code in the evaluator can check for the condition code
and go to \code{signal\-/error} if necessary.  Build this structure and make it
work.  This is a major project.

\end{enumerate}
\end{quote}

\section{Compilation}
\label{Section 5.5}

The explicit-control evaluator of \link{Section 5.4} is a register machine whose
controller interprets Scheme programs.  In this section we will see how to run
Scheme programs on a register machine whose controller is not a Scheme
interpreter.

The explicit-control evaluator machine is universal---it can carry out any
computational process that can be described in Scheme.  The evaluator's
controller orchestrates the use of its data paths to perform the desired
computation.  Thus, the evaluator's data paths are universal: They are
sufficient to perform any computation we desire, given an appropriate
controller.\footnote{This is a theoretical statement.  We are not claiming that
the evaluator's data paths are a particularly convenient or efficient set of
data paths for a general-purpose computer.  For example, they are not very good
for implementing high-performance floating-point calculations or calculations
that intensively manipulate bit vectors.}

Commercial general-purpose computers are register machines organized around a
collection of registers and operations that constitute an efficient and
convenient universal set of data paths.  The controller for a general-purpose
machine is an interpreter for a register-machine language like the one we have
been using.  This language is called the \newterm{native language} of the
machine, or simply \newterm{machine language}.  Programs written in machine
language are sequences of instructions that use the machine's data paths.  For
example, the explicit-control evaluator's instruction sequence can be thought
of as a machine-language program for a general-purpose computer rather than as
the controller for a specialized interpreter machine.

There are two common strategies for bridging the gap between higher-level
languages and register-machine languages.  The explicit-control evaluator
illustrates the strategy of interpretation.  An interpreter written in the
native language of a machine configures the machine to execute programs written
in a language (called the \newterm{source language}) that may differ from the
native language of the machine performing the evaluation.  The primitive
procedures of the source language are implemented as a library of subroutines
written in the native language of the given machine.  A program to be
interpreted (called the \newterm{source program}) is represented as a data
structure.  The interpreter traverses this data structure, analyzing the source
program.  As it does so, it simulates the intended behavior of the source
program by calling appropriate primitive subroutines from the library.

In this section, we explore the alternative strategy of \newterm{compilation}.
A compiler for a given source language and machine translates a source program
into an equivalent program (called the \newterm{object program}) written in the
machine's native language.  The compiler that we implement in this section
translates programs written in Scheme into sequences of instructions to be
executed using the explicit-control evaluator machine's data
paths.\footnote{Actually, the machine that runs compiled code can be simpler
than the interpreter machine, because we won't use the \code{exp} and
\code{unev} registers.  The interpreter used these to hold pieces of
unevaluated expressions.  With the compiler, however, these expressions get
built into the compiled code that the register machine will run.  For the same
reason, we don't need the machine operations that deal with expression syntax.
But compiled code will use a few additional machine operations (to represent
compiled procedure objects) that didn't appear in the explicit-control
evaluator machine.}

Compared with interpretation, compilation can provide a great increase in the
efficiency of program execution, as we will explain below in the overview of
the compiler.  On the other hand, an interpreter provides a more powerful
environment for interactive program development and debugging, because the
source program being executed is available at run time to be examined and
modified.  In addition, because the entire library of primitives is present,
new programs can be constructed and added to the system during debugging.

In view of the complementary advantages of compilation and interpretation,
modern program-development environments pursue a mixed strategy.  Lisp
interpreters are generally organized so that interpreted procedures and
compiled procedures can call each other.  This enables a programmer to compile
those parts of a program that are assumed to be debugged, thus gaining the
efficiency advantage of compilation, while retaining the interpretive mode of
execution for those parts of the program that are in the flux of interactive
development and debugging.  In \link{Section 5.5.7}, after we have implemented
the compiler, we will show how to interface it with our interpreter to produce
an integrated interpreter-compiler development system.

\subsubsection*{An overview of the compiler}

Our compiler is much like our interpreter, both in its structure and in the
function it performs.  Accordingly, the mechanisms used by the compiler for
analyzing expressions will be similar to those used by the interpreter.
Moreover, to make it easy to interface compiled and interpreted code, we will
design the compiler to generate code that obeys the same conventions of
register usage as the interpreter: The environment will be kept in the
\code{env} register, argument lists will be accumulated in \code{argl}, a
procedure to be applied will be in \code{proc}, procedures will return their
answers in \code{val}, and the location to which a procedure should return will
be kept in \code{continue}.  In general, the compiler translates a source
program into an object program that performs essentially the same register
operations as would the interpreter in evaluating the same source program.

This description suggests a strategy for implementing a rudimentary compiler:
We traverse the expression in the same way the interpreter does.  When we
encounter a register instruction that the interpreter would perform in
evaluating the expression, we do not execute the instruction but instead
accumulate it into a sequence.  The resulting sequence of instructions will be
the object code.  Observe the efficiency advantage of compilation over
interpretation.  Each time the interpreter evaluates an expression---for
example, \code{(f 84 96)}---it performs the work of classifying the expression
(discovering that this is a procedure application) and testing for the end of
the operand list (discovering that there are two operands).  With a compiler,
the expression is analyzed only once, when the instruction sequence is
generated at compile time.  The object code produced by the compiler contains
only the instructions that evaluate the operator and the two operands, assemble
the argument list, and apply the procedure (in \code{proc}) to the arguments
(in \code{argl}).

This is the same kind of optimization we implemented in the analyzing evaluator
of \link{Section 4.1.7}.  But there are further opportunities to gain efficiency
in compiled code.  As the interpreter runs, it follows a process that must be
applicable to any expression in the language.  In contrast, a given segment of
compiled code is meant to execute some particular expression.  This can make a
big difference, for example in the use of the stack to save registers.  When
the interpreter evaluates an expression, it must be prepared for any
contingency.  Before evaluating a subexpression, the interpreter saves all
registers that will be needed later, because the subexpression might require an
arbitrary evaluation.  A compiler, on the other hand, can exploit the structure
of the particular expression it is processing to generate code that avoids
unnecessary stack operations.

As a case in point, consider the combination \code{(f 84 96)}.  Before the
interpreter evaluates the operator of the combination, it prepares for this
evaluation by saving the registers containing the operands and the environment,
whose values will be needed later.  The interpreter then evaluates the operator
to obtain the result in \code{val}, restores the saved registers, and finally
moves the result from \code{val} to \code{proc}.  However, in the particular
expression we are dealing with, the operator is the symbol \code{f}, whose
evaluation is accomplished by the machine operation
\code{lookup\-/variable\-/value}, which does not alter any registers.  The compiler
that we implement in this section will take advantage of this fact and generate
code that evaluates the operator using the instruction

\begin{scheme}
(assign proc (op lookup-variable-value)
             (const f)
             (reg env))
\end{scheme}

\noindent
This code not only avoids the unnecessary saves and restores but also assigns
the value of the lookup directly to \code{proc}, whereas the interpreter would
obtain the result in \code{val} and then move this to \code{proc}.

A compiler can also optimize access to the environment.  Having analyzed the
code, the compiler can in many cases know in which frame a particular variable
will be located and access that frame directly, rather than performing the
\code{lookup\-/variable\-/value} search.  We will discuss how to implement such
variable access in \link{Section 5.5.6}.  Until then, however, we will focus on
the kind of register and stack optimizations described above.  There are many
other optimizations that can be performed by a compiler, such as coding
primitive operations ``in line'' instead of using a general \code{apply}
mechanism (see \link{Exercise 5.38}); but we will not emphasize these here.  Our
main goal in this section is to illustrate the compilation process in a
simplified (but still interesting) context.



\subsection{Structure of the Compiler}
\label{Section 5.5.1}

In \link{Section 4.1.7} we modified our original metacircular interpreter to
separate analysis from execution.  We analyzed each expression to produce an
execution procedure that took an environment as argument and performed the
required operations.  In our compiler, we will do essentially the same
analysis.  Instead of producing execution procedures, however, we will generate
sequences of instructions to be run by our register machine.

The procedure \code{compile} is the top-level dispatch in the compiler.  It
corresponds to the \code{eval} procedure of \link{Section 4.1.1}, the
\code{analyze} procedure of \link{Section 4.1.7}, and the \code{eval\-/dispatch}
entry point of the explicit-control-evaluator in \link{Section 5.4.1}.  The
compiler, like the interpreters, uses the expression-syntax procedures defined
in \link{Section 4.1.2}.\footnote{Notice, however, that our compiler is a Scheme
program, and the syntax procedures that it uses to manipulate expressions are
the actual Scheme procedures used with the metacircular evaluator.  For the
explicit-control evaluator, in contrast, we assumed that equivalent syntax
operations were available as operations for the register machine.  (Of course,
when we simulated the register machine in Scheme, we used the actual Scheme
procedures in our register machine simulation.)}  \code{Compile} performs a
case analysis on the syntactic type of the expression to be compiled.  For each
type of expression, it dispatches to a specialized \newterm{code generator}:

\begin{scheme}
(define (compile exp target linkage)
  (cond ((self-evaluating? exp)
         (compile-self-evaluating exp target linkage))
        ((quoted? exp) (compile-quoted exp target linkage))
        ((variable? exp)
         (compile-variable exp target linkage))
        ((assignment? exp)
         (compile-assignment exp target linkage))
        ((definition? exp)
         (compile-definition exp target linkage))
        ((if? exp) (compile-if exp target linkage))
        ((lambda? exp) (compile-lambda exp target linkage))
        ((begin? exp)
         (compile-sequence
          (begin-actions exp) target linkage))
        ((cond? exp) 
         (compile (cond->if exp) target linkage))
        ((application? exp)
         (compile-application exp target linkage))
        (else
         (error "Unknown expression type: COMPILE" exp))))
\end{scheme}

\subsubsection*{Targets and linkages}

\code{Compile} and the code generators that it calls take two arguments in
addition to the expression to compile.  There is a \newterm{target}, which
specifies the register in which the compiled code is to return the value of the
expression.  There is also a \newterm{linkage descriptor}, which describes how
the code resulting from the compilation of the expression should proceed when
it has finished its execution.  The linkage descriptor can require that the
code do one of the following three things:

\begin{itemize}

\item
continue at the next instruction in sequence (this is specified by the linkage
descriptor \code{next}),

\item
return from the procedure being compiled (this is specified by the linkage
descriptor \code{return}), or

\item
jump to a named entry point (this is specified by using the designated label as
the linkage descriptor).

\end{itemize}

\noindent
For example, compiling the expression \code{5} (which is self-evaluating) with
a target of the \code{val} register and a linkage of \code{next} should produce
the instruction

\begin{scheme}
(assign val (const 5))
\end{scheme}

\noindent
Compiling the same expression with a linkage of \code{return} should produce
the instructions

\begin{scheme}
(assign val (const 5))
(goto (reg continue))
\end{scheme}

\noindent
In the first case, execution will continue with the next instruction in the
sequence. In the second case, we will return from a procedure call.  In both
cases, the value of the expression will be placed into the target \code{val}
register.

\subsubsection*{Instruction sequences and stack usage}

Each code generator returns an \newterm{instruction sequence} containing the
object code it has generated for the expression.  Code generation for a
compound expression is accomplished by combining the output from simpler code
generators for component expressions, just as evaluation of a compound
expression is accomplished by evaluating the component expressions.

The simplest method for combining instruction sequences is a procedure called
\code{append\-/instruction\-/sequences}.  It takes as arguments any number of
instruction sequences that are to be executed sequentially; it appends them and
returns the combined sequence.  That is, if \( \langle \)\( seq_1 \)\( \rangle \) and \( \langle \)\( seq_2 \)\( \rangle \) are
sequences of instructions, then evaluating

\begin{scheme}
(append-instruction-sequences ~\( \dark \langle \)~~\( \dark seq_1 \)~~\( \dark \rangle \)~ ~\( \dark \langle \)~~\( \dark seq_2 \)~~\( \dark \rangle \)~)
\end{scheme}

\noindent
produces the sequence

\begin{scheme}
~\( \dark \langle \)~~\( \dark seq_1 \)~~\( \dark \rangle \)~
~\( \dark \langle \)~~\( \dark seq_2 \)~~\( \dark \rangle \)~
\end{scheme}

\noindent
Whenever registers might need to be saved, the compiler's code generators use
\code{preserving}, which is a more subtle method for combining instruction
sequences.  \code{Preserving} takes three arguments: a set of registers and two
instruction sequences that are to be executed sequentially.  It appends the
sequences in such a way that the contents of each register in the set is
preserved over the execution of the first sequence, if this is needed for the
execution of the second sequence.  That is, if the first sequence modifies the
register and the second sequence actually needs the register's original
contents, then \code{preserving} wraps a \code{save} and a \code{restore} of
the register around the first sequence before appending the sequences.
Otherwise, \code{preserving} simply returns the appended instruction sequences.
Thus, for example,

\begin{scheme}
(preserving (list ~\( \dark \langle reg_1 \rangle \)~ ~\( \dark \langle reg_2 \rangle \)~) ~\( \dark \langle seg_1 \rangle \)~ ~\( \dark \langle seg_2 \rangle \)~)
\end{scheme}

produces one of the following four sequences of instructions, depending on how
\( \langle \)\( seq_1 \)\( \rangle \) and \( \langle \)\( seq_2 \)\( \rangle \) use \( \langle \)\( reg_1 \)\( \rangle \) and \( \langle \)\( reg_2 \)\( \rangle \):
\begin{comment}

\begin{smallexample}
<seq_1> | (save <reg_1>)    | (save <reg_2>)    | (save <reg_2>)
<seq_2> | <seq_1>           | <seq_1>           | (save <reg_1>)
        | (restore <reg_1>) | (restore <reg_2>) | <seq_1>
        | <seq_2>           | <seq_2>           | (restore <reg_1>)
        |                   |                   | (restore <reg_2>)
        |                   |                   | <seq_2>
\end{smallexample}

\end{comment}
\begin{displaymath}
\vbox{
\offinterlineskip
\halign{
\strut 	\kern0.8em # \kern0.4em \hfil & \vrule 
	\kern0.8em # \kern0.4em \hfil & \vrule
	\kern0.8em # \kern0.4em \hfil & \vrule
	\kern0.8em # \kern0.4em \hfil \cr

$\langle{seq_1}\rangle$ 			& \hbox{\tt (save} $\langle{reg_1}\rangle${\tt)} & 
\hbox{\tt (save} $\langle{reg_2}\rangle${\tt)} 	& \hbox{\tt (save} $\langle{reg_2}\rangle${\tt)} \cr

$\langle{seq_2}\rangle$ 			& $\langle{seq_1}\rangle$	& 
$\langle{seq_1}\rangle$ 			& \hbox{\tt (save} $\langle{reg_1}\rangle${\tt)} \cr

						& \hbox{\tt (restore} $\langle{reg_1}\rangle${\tt)} & 
\hbox{\tt (restore} $\langle{reg_2}\rangle${\tt)} 	& $\langle{seq_1}\rangle$ \cr

						& $\langle{seq_2}\rangle$ & 
$\langle{seq_2}\rangle$				& \hbox{\tt (restore} $\langle{reg_1}\rangle${\tt)} \cr

						&  &
			      			& \hbox{\tt (restore} $\langle{reg_2}\rangle${\tt)} \cr

						&  &
			      			& $\langle{seq_2}\rangle$ \cr
}
}
\end{displaymath}

\noindent
By using \code{preserving} to combine instruction sequences the compiler avoids
unnecessary stack operations.  This also isolates the details of whether or not
to generate \code{save} and \code{restore} instructions within the
\code{preserving} procedure, separating them from the concerns that arise in
writing each of the individual code generators.  In fact no \code{save} or
\code{restore} instructions are explicitly produced by the code generators.

In principle, we could represent an instruction sequence simply as a list of
instructions.  \code{Append\-/instruction\-/sequences} could then combine
instruction sequences by performing an ordinary list \code{append}.  However,
\code{preserving} would then be a complex operation, because it would have to
analyze each instruction sequence to determine how the sequence uses its
registers.  \code{Preserving} would be inefficient as well as complex, because
it would have to analyze each of its instruction sequence arguments, even
though these sequences might themselves have been constructed by calls to
\code{preserving}, in which case their parts would have already been analyzed.
To avoid such repetitious analysis we will associate with each instruction
sequence some information about its register use.  When we construct a basic
instruction sequence we will provide this information explicitly, and the
procedures that combine instruction sequences will derive register-use
information for the combined sequence from the information associated with the
component sequences.

An instruction sequence will contain three pieces of information:

\begin{itemize}

\item
the set of registers that must be initialized before the instructions in the
sequence are executed (these registers are said to be \newterm{needed} by the
sequence),

\item
the set of registers whose values are modified by the instructions in the
sequence, and

\item
the actual instructions (also called \newterm{statements}) in the sequence.

\end{itemize}

\noindent
We will represent an instruction sequence as a list of its three parts.  The
constructor for instruction sequences is thus

\begin{scheme}
(define (make-instruction-sequence
         needs modifies statements)
  (list needs modifies statements))
\end{scheme}

\noindent
For example, the two-instruction sequence that looks up the value of the
variable \code{x} in the current environment, assigns the result to \code{val},
and then returns, requires registers \code{env} and \code{continue} to have
been initialized, and modifies register \code{val}.  This sequence would
therefore be constructed as

\begin{scheme}
(make-instruction-sequence 
 '(env continue)
 '(val)
 '((assign val
           (op lookup-variable-value) (const x) (reg env))
   (goto (reg continue))))
\end{scheme}

\noindent
We sometimes need to construct an instruction sequence with no statements:

\begin{scheme}
(define (empty-instruction-sequence)
  (make-instruction-sequence '() '() '()))
\end{scheme}

\noindent
The procedures for combining instruction sequences are shown in 
\link{Section 5.5.4}.

\begin{quote}
\heading{\phantomsection\label{Exercise 5.31}Exercise 5.31:} In evaluating a procedure
application, the explicit-control evaluator always saves and restores the
\code{env} register around the evaluation of the operator, saves and restores
\code{env} around the evaluation of each operand (except the final one), saves
and restores \code{argl} around the evaluation of each operand, and saves and
restores \code{proc} around the evaluation of the operand sequence.  For each
of the following combinations, say which of these \code{save} and
\code{restore} operations are superfluous and thus could be eliminated by the
compiler's \code{preserving} mechanism:

\begin{scheme}
(f 'x 'y)
((f) 'x 'y)
(f (g 'x) y)
(f (g 'x) 'y)
\end{scheme}
\end{quote}

\begin{quote}
\heading{\phantomsection\label{Exercise 5.32}Exercise 5.32:} Using the \code{preserving}
mechanism, the compiler will avoid saving and restoring \code{env} around the
evaluation of the operator of a combination in the case where the operator is a
symbol.  We could also build such optimizations into the evaluator.  Indeed,
the explicit-control evaluator of \link{Section 5.4} already performs a similar
optimization, by treating combinations with no operands as a special case.

\begin{enumerate}[a.]

\item
Extend the explicit-control evaluator to recognize as a separate class of
expressions combinations whose operator is a symbol, and to take advantage of
this fact in evaluating such expressions.

\item
Alyssa P. Hacker suggests that by extending the evaluator to recognize more and
more special cases we could incorporate all the compiler's optimizations, and
that this would eliminate the advantage of compilation altogether.  What do you
think of this idea?

\end{enumerate}
\end{quote}

\subsection{Compiling Expressions}
\label{Section 5.5.2}

In this section and the next we implement the code generators to which the
\code{compile} procedure dispatches.

\subsubsection*{Compiling linkage code}

In general, the output of each code generator will end with
instructions---generated by the procedure \code{compile\-/linkage}---that
implement the required linkage.  If the linkage is \code{return} then we must
generate the instruction \code{(goto (reg continue))}.  This needs the
\code{continue} register and does not modify any registers.  If the linkage is
\code{next}, then we needn't include any additional instructions.  Otherwise,
the linkage is a label, and we generate a \code{goto} to that label, an
instruction that does not need or modify any registers.\footnote{This procedure
uses a feature of Lisp called \newterm{backquote} (or \newterm{quasiquote})
that is handy for constructing lists.  Preceding a list with a backquote symbol
is much like quoting it, except that anything in the list that is flagged with
a comma is evaluated.

For example, if the value of \code{linkage} is the symbol \code{branch25},
then the expression 

\begin{smallscheme}
`((goto (label ,linkage)))
\end{smallscheme}

\noindent
evaluates to the list

\begin{smallscheme}
((goto (label branch25)))
\end{smallscheme}

\noindent
Similarly, if the value of \code{x} is the list \code{(a b c)}, then

\begin{smallscheme}
`(1 2 ,(car x))
\end{smallscheme}

\noindent
evaluates to the list

\begin{smallscheme}
(1 2 a).
\end{smallscheme}
}

% For example, if the value of \code{linkage} is the symbol\\ \code{branch25},
% then the expression\\ \code{`((goto (label ,linkage)))}\\ evaluates to the
% list\\ \code{((goto (label branch25)))}.\\ Similarly, if the value of \code{x}
% is the list \code{(a b c)}, then\\ \code{`(1 2 ,(car x))}\\ evaluates to the
% list\\ \code{(1 2 a)}.}

\begin{scheme}
(define (compile-linkage linkage)
  (cond ((eq? linkage 'return)
         (make-instruction-sequence '(continue) '()
          '((goto (reg continue)))))
        ((eq? linkage 'next)
         (empty-instruction-sequence))
        (else
         (make-instruction-sequence '() '()
          `((goto (label ,linkage)))))))
\end{scheme}

\noindent
The linkage code is appended to an instruction sequence by \code{preserving}
the \code{continue} register, since a \code{return} linkage will require the
\code{continue} register: If the given instruction sequence modifies
\code{continue} and the linkage code needs it, \code{continue} will be saved
and restored.

\begin{scheme}
(define (end-with-linkage linkage instruction-sequence)
  (preserving '(continue)
   instruction-sequence
   (compile-linkage linkage)))
\end{scheme}

\subsubsection*{Compiling simple expressions}

The code generators for self-evaluating expressions, quotations, and variables
construct instruction sequences that assign the required value to the target
register and then proceed as specified by the linkage descriptor.

\begin{scheme}
(define (compile-self-evaluating exp target linkage)
  (end-with-linkage linkage
   (make-instruction-sequence '() (list target)
    `((assign ,target (const ,exp))))))
(define (compile-quoted exp target linkage)
  (end-with-linkage linkage
   (make-instruction-sequence '() (list target)
    `((assign ,target (const ,(text-of-quotation exp)))))))
(define (compile-variable exp target linkage)
  (end-with-linkage linkage
   (make-instruction-sequence '(env) (list target)
    `((assign ,target
              (op lookup-variable-value)
              (const ,exp)
              (reg env))))))
\end{scheme}

\noindent
All these assignment instructions modify the target register, and the one that
looks up a variable needs the \code{env} register.

Assignments and definitions are handled much as they are in the interpreter.
We recursively generate code that computes the value to be assigned to the
variable, and append to it a two-instruction sequence that actually sets or
defines the variable and assigns the value of the whole expression (the symbol
\code{ok}) to the target register.  The recursive compilation has target
\code{val} and linkage \code{next} so that the code will put its result into
\code{val} and continue with the code that is appended after it.  The appending
is done preserving \code{env}, since the environment is needed for setting or
defining the variable and the code for the variable value could be the
compilation of a complex expression that might modify the registers in
arbitrary ways.

\begin{scheme}
(define (compile-assignment exp target linkage)
  (let ((var (assignment-variable exp))
        (get-value-code
         (compile (assignment-value exp) 'val 'next)))
    (end-with-linkage linkage
     (preserving '(env)
      get-value-code
      (make-instruction-sequence '(env val) (list target)
       `((perform (op set-variable-value!)
                  (const ,var)
                  (reg val)
                  (reg env))
         (assign ,target (const ok))))))))
(define (compile-definition exp target linkage)
  (let ((var (definition-variable exp))
        (get-value-code
         (compile (definition-value exp) 'val 'next)))
    (end-with-linkage linkage
     (preserving '(env)
      get-value-code
      (make-instruction-sequence '(env val) (list target)
       `((perform (op define-variable!)
                  (const ,var)
                  (reg val)
                  (reg env))
         (assign ,target (const ok))))))))
\end{scheme}

\enlargethispage{\baselineskip}

\noindent
The appended two-instruction sequence requires \code{env} and \code{val} and
modifies the target.  Note that although we preserve \code{env} for this
sequence, we do not preserve \code{val}, because the \code{get\-/value\-/code} is
designed to explicitly place its result in \code{val} for use by this sequence.
(In fact, if we did preserve \code{val}, we would have a bug, because this
would cause the previous contents of \code{val} to be restored right after the
\code{get\-/value\-/code} is run.)

\subsubsection*{Compiling conditional expressions}

The code for an \code{if} expression compiled with a given target and linkage
has the form

\begin{scheme}
~\( \dark \langle \)~~\emph{compilation of predicate, target \emph{val}, linkage \emph{next}}~~\( \dark \rangle \)~
 (test (op false?) (reg val))
 (branch (label false-branch))
true-branch
 ~\( \dark \langle \)~~\emph{compilation of consequent with given target}~ 
  ~\emph{and given linkage or \emph{after-if}}~~\( \dark \rangle \)~
false-branch
 ~\( \dark \langle \)~~\emph{compilation of alternative with given target and linkage}~~\( \dark \rangle \)~
after-if
\end{scheme}

\noindent
To generate this code, we compile the predicate, consequent, and alternative,
and combine the resulting code with instructions to test the predicate result
and with newly generated labels to mark the true and false branches and the end
of the conditional.\footnote{We can't just use the labels \code{true\-/branch},
\code{false\-/branch}, and \code{after\-/if} as shown above, because there might be
more than one \code{if} in the program.  The compiler uses the procedure
\code{make\-/label} to generate labels.  \code{Make\-/label} takes a symbol as
argument and returns a new symbol that begins with the given symbol.  For
example, successive calls to \code{(make\-/label 'a)} would return \code{a1},
\code{a2}, and so on.  \code{Make\-/label} can be implemented similarly to the
generation of unique variable names in the query language, as follows:

\begin{smallscheme}
(define label-counter 0)
(define (new-label-number)
  (set! label-counter (+ 1 label-counter))
  label-counter)
(define (make-label name)
  (string->symbol
    (string-append (symbol->string name)
                   (number->string (new-label-number)))))
\end{smallscheme}
} In this arrangement of code, we must branch around the true branch if the
test is false.  The only slight complication is in how the linkage for the true
branch should be handled.  If the linkage for the conditional is \code{return}
or a label, then the true and false branches will both use this same linkage.
If the linkage is \code{next}, the true branch ends with a jump around the code
for the false branch to the label at the end of the conditional.

\begin{scheme}
(define (compile-if exp target linkage)
  (let ((t-branch (make-label 'true-branch))
        (f-branch (make-label 'false-branch))
        (after-if (make-label 'after-if)))
    (let ((consequent-linkage
           (if (eq? linkage 'next) after-if linkage)))
      (let ((p-code (compile (if-predicate exp) 'val 'next))
            (c-code
             (compile
              (if-consequent exp) target 
                                  consequent-linkage))
            (a-code
             (compile (if-alternative exp) target linkage)))
        (preserving '(env continue)
         p-code
         (append-instruction-sequences
          (make-instruction-sequence '(val) '()
           `((test (op false?) (reg val))
             (branch (label ,f-branch))))
          (parallel-instruction-sequences
           (append-instruction-sequences t-branch c-code)
           (append-instruction-sequences f-branch a-code))
          after-if))))))
\end{scheme}

\noindent
\code{Env} is preserved around the predicate code because it could be needed by
the true and false branches, and \code{continue} is preserved because it could
be needed by the linkage code in those branches.  The code for the true and
false branches (which are not executed sequentially) is appended using a
special combiner \code{parallel\-/instruction\-/sequences} described in 
\link{Section 5.5.4}.

Note that \code{cond} is a derived expression, so all that the compiler needs
to do handle it is to apply the \code{cond\-/>if} transformer (from 
\link{Section 4.1.2}) and compile the resulting \code{if} expression.

\subsubsection*{Compiling sequences}

The compilation of sequences (from procedure bodies or explicit \code{begin}
expressions) parallels their evaluation.  Each expression of the sequence is
compiled---the last expression with the linkage specified for the sequence, and
the other expressions with linkage \code{next} (to execute the rest of the
sequence).  The instruction sequences for the individual expressions are
appended to form a single instruction sequence, such that \code{env} (needed
for the rest of the sequence) and \code{continue} (possibly needed for the
linkage at the end of the sequence) are preserved.

\begin{scheme}
(define (compile-sequence seq target linkage)
  (if (last-exp? seq)
      (compile (first-exp seq) target linkage)
      (preserving
       '(env continue)
       (compile (first-exp seq) target 'next)
       (compile-sequence (rest-exps seq) target linkage))))
\end{scheme}

\subsubsection*{Compiling \code{lambda} expressions}

\code{Lambda} expressions construct procedures.  The object code for a
\code{lambda} expression must have the form

\begin{scheme}
~\( \dark \langle \)~~\emph{construct procedure object and assign it to target register}~~\( \dark \rangle \)~
~\( \dark \langle \)~~\var{\dark linkage}~~\( \dark \rangle \)~
\end{scheme}

\noindent
When we compile the \code{lambda} expression, we also generate the code for the
procedure body.  Although the body won't be executed at the time of procedure
construction, it is convenient to insert it into the object code right after
the code for the \code{lambda}.  If the linkage for the \code{lambda}
expression is a label or \code{return}, this is fine.  But if the linkage is
\code{next}, we will need to skip around the code for the procedure body by
using a linkage that jumps to a label that is inserted after the body.  The
object code thus has the form

\begin{scheme}
~\( \dark \langle \)~~\emph{construct procedure object and assign it to target register}~~\( \dark \rangle \)~
 ~\( \dark \langle \)~~\emph{code for given linkage}~~\( \dark \rangle \)~ ~\emph{or}~ ~\code{(goto (label after\-/lambda))}~
 ~\( \dark \langle \)~~\emph{compilation of procedure body}~~\( \dark \rangle \)~
after-lambda
\end{scheme}

\noindent
\code{Compile\-/lambda} generates the code for constructing the procedure object
followed by the code for the procedure body.  The procedure object will be
constructed at run time by combining the current environment (the environment
at the point of definition) with the entry point to the compiled procedure body
(a newly generated label).\footnote{\label{Footnote 38} We need
machine operations to implement a data structure for representing compiled
procedures, analogous to the structure for compound procedures described in
\link{Section 4.1.3}:

\begin{smallscheme}
(define (make-compiled-procedure entry env)
  (list 'compiled-procedure entry env))
(define (compiled-procedure? proc)
  (tagged-list? proc 'compiled-procedure))
(define (compiled-procedure-entry c-proc) (cadr c-proc))
(define (compiled-procedure-env c-proc) (caddr c-proc))
\end{smallscheme}
}

\begin{scheme}
(define (compile-lambda exp target linkage)
  (let ((proc-entry (make-label 'entry))
        (after-lambda (make-label 'after-lambda)))
    (let ((lambda-linkage
           (if (eq? linkage 'next) after-lambda linkage)))
      (append-instruction-sequences
       (tack-on-instruction-sequence
        (end-with-linkage lambda-linkage
         (make-instruction-sequence '(env) (list target)
          `((assign ,target
                    (op make-compiled-procedure)
                    (label ,proc-entry)
                    (reg env)))))
        (compile-lambda-body exp proc-entry))
       after-lambda))))
\end{scheme}

\enlargethispage{\baselineskip}

\noindent
\code{Compile\-/lambda} uses the special combiner
\code{tack\-/on\-/instruction\-/sequence} rather than
\code{append\-/instruction\-/sequences} (\link{Section 5.5.4}) to append the procedure body to the
\code{lambda} expression code, because the body is not part of the sequence of
instructions that will be executed when the combined sequence is entered;
rather, it is in the sequence only because that was a convenient place to put
it.

\code{Compile\-/lambda\-/body} constructs the code for the body of the procedure.
This code begins with a label for the entry point.  Next come instructions that
will cause the run-time evaluation environment to switch to the correct
environment for evaluating the procedure body---namely, the definition
environment of the procedure, extended to include the bindings of the formal
parameters to the arguments with which the procedure is called.  After this
comes the code for the sequence of expressions that makes up the procedure
body.  The sequence is compiled with linkage \code{return} and target
\code{val} so that it will end by returning from the procedure with the
procedure result in \code{val}.

\begin{scheme}
(define (compile-lambda-body exp proc-entry)
  (let ((formals (lambda-parameters exp)))
    (append-instruction-sequences
     (make-instruction-sequence '(env proc argl) '(env)
      `(,proc-entry
        (assign env
                (op compiled-procedure-env)
                (reg proc))
        (assign env
                (op extend-environment)
                (const ,formals)
                (reg argl)
                (reg env))))
     (compile-sequence (lambda-body exp) 'val 'return))))
\end{scheme}

\subsection{Compiling Combinations}
\label{Section 5.5.3}

The essence of the compilation process is the compilation of procedure
applications.  The code for a combination compiled with a given target and
linkage has the form

\begin{scheme}
~\( \dark \langle \)~~\emph{compilation of operator, target \emph{proc}, linkage \emph{next}}~~\( \dark \rangle \)~
~\( \dark \langle \)~~\emph{evaluate operands and construct argument list in \emph{argl}}~~\( \dark \rangle \)~
~\( \dark \langle \)~~\emph{compilation of procedure call with given target and linkage}~~\( \dark \rangle \)~
\end{scheme}

\noindent
The registers \code{env}, \code{proc}, and \code{argl} may have to be saved and
restored during evaluation of the operator and operands.  Note that this is the
only place in the compiler where a target other than \code{val} is specified.

The required code is generated by \code{compile\-/application}.  This recursively
compiles the operator, to produce code that puts the procedure to be applied
into \code{proc}, and compiles the operands, to produce code that evaluates the
individual operands of the application.  The instruction sequences for the
operands are combined (by \code{construct\-/arglist}) with code that constructs
the list of arguments in \code{argl}, and the resulting argument-list code is
combined with the procedure code and the code that performs the procedure call
(produced by \code{compile\-/procedure\-/call}).  In appending the code sequences,
the \code{env} register must be preserved around the evaluation of the operator
(since evaluating the operator might modify \code{env}, which will be needed to
evaluate the operands), and the \code{proc} register must be preserved around
the construction of the argument list (since evaluating the operands might
modify \code{proc}, which will be needed for the actual procedure application).
\code{Continue} must also be preserved throughout, since it is needed for the
linkage in the procedure call.

\begin{scheme}
(define (compile-application exp target linkage)
  (let ((proc-code (compile (operator exp) 'proc 'next))
        (operand-codes
         (map (lambda 
                (operand) (compile operand 'val 'next))
              (operands exp))))
    (preserving '(env continue)
     proc-code
     (preserving '(proc continue)
      (construct-arglist operand-codes)
      (compile-procedure-call target linkage)))))
\end{scheme}

\noindent
The code to construct the argument list will evaluate each operand into
\code{val} and then \code{cons} that value onto the argument list being
accumulated in \code{argl}.  Since we \code{cons} the arguments onto
\code{argl} in sequence, we must start with the last argument and end with the
first, so that the arguments will appear in order from first to last in the
resulting list.  Rather than waste an instruction by initializing \code{argl}
to the empty list to set up for this sequence of evaluations, we make the first
code sequence construct the initial \code{argl}.  The general form of the
argument-list construction is thus as follows:

\begin{scheme}
~\( \dark \langle \)~~\emph{compilation of last operand, targeted to \emph{val}}~~\( \dark \rangle \)~
(assign argl (op list) (reg val))
~\( \dark \langle \)~~\emph{compilation of next operand, targeted to \emph{val}}~~\( \dark \rangle \)~
(assign argl (op cons) (reg val) (reg argl))
~\( \dots \)~
~\( \dark \langle \)~~\emph{compilation of first operand, targeted to \emph{val}}~~\( \dark \rangle \)~
(assign argl (op cons) (reg val) (reg argl))
\end{scheme}

\noindent
\code{Argl} must be preserved around each operand evaluation except the first
(so that arguments accumulated so far won't be lost), and \code{env} must be
preserved around each operand evaluation except the last (for use by subsequent
operand evaluations).

Compiling this argument code is a bit tricky, because of the special treatment
of the first operand to be evaluated and the need to preserve \code{argl} and
\code{env} in different places.  The \code{construct\-/arglist} procedure takes
as arguments the code that evaluates the individual operands.  If there are no
operands at all, it simply emits the instruction

\begin{scheme}
(assign argl (const ()))
\end{scheme}

\noindent
Otherwise, \code{construct\-/arglist} creates code that initializes \code{argl}
with the last argument, and appends code that evaluates the rest of the
arguments and adjoins them to \code{argl} in succession.  In order to process
the arguments from last to first, we must reverse the list of operand code
sequences from the order supplied by \code{compile\-/application}.

\begin{scheme}
(define (construct-arglist operand-codes)
  (let ((operand-codes (reverse operand-codes)))
    (if (null? operand-codes)
        (make-instruction-sequence '() '(argl)
         '((assign argl (const ()))))
        (let ((code-to-get-last-arg
               (append-instruction-sequences
                (car operand-codes)
                (make-instruction-sequence '(val) '(argl)
                 '((assign argl (op list) (reg val)))))))
          (if (null? (cdr operand-codes))
              code-to-get-last-arg
              (preserving '(env)
               code-to-get-last-arg
               (code-to-get-rest-args
                (cdr operand-codes))))))))

(define (code-to-get-rest-args operand-codes)
  (let ((code-for-next-arg
         (preserving '(argl)
          (car operand-codes)
          (make-instruction-sequence '(val argl) '(argl)
           '((assign argl
              (op cons) (reg val) (reg argl)))))))
    (if (null? (cdr operand-codes))
        code-for-next-arg
        (preserving '(env)
         code-for-next-arg
         (code-to-get-rest-args (cdr operand-codes))))))
\end{scheme}

\subsubsection*{Applying procedures}

After evaluating the elements of a combination, the compiled code must apply
the procedure in \code{proc} to the arguments in \code{argl}.  The code
performs essentially the same dispatch as the \code{apply} procedure in the
metacircular evaluator of \link{Section 4.1.1} or the \code{apply\-/dispatch}
entry point in the explicit-control evaluator of \link{Section 5.4.1}.  It
checks whether the procedure to be applied is a primitive procedure or a
compiled procedure.  For a primitive procedure, it uses
\code{apply\-/primitive\-/procedure}; we will see shortly how it handles compiled
procedures.  The procedure-application code has the following form:

\begin{scheme}
(test (op primitive-procedure?) (reg proc))
 (branch (label primitive-branch))
compiled-branch
 ~\( \dark \langle \)~~\emph{code to apply compiled procedure with given target}~ 
  ~\emph{and appropriate linkage}~~\( \dark \rangle \)~
primitive-branch
 (assign ~\( \dark \langle \)~~\var{\dark target}~~\( \dark \rangle \)~
         (op apply-primitive-procedure)
         (reg proc)
         (reg argl))
 ~\( \dark \langle \)~~\var{\dark linkage}~~\( \dark \rangle \)~
after-call
\end{scheme}

\noindent
Observe that the compiled branch must skip around the primitive branch.
Therefore, if the linkage for the original procedure call was \code{next}, the
compound branch must use a linkage that jumps to a label that is inserted after
the primitive branch.  (This is similar to the linkage used for the true branch
in \code{compile\-/if}.)

\begin{scheme}
(define (compile-procedure-call target linkage)
  (let ((primitive-branch (make-label 'primitive-branch))
        (compiled-branch (make-label 'compiled-branch))
        (after-call (make-label 'after-call)))

    (let ((compiled-linkage
           (if (eq? linkage 'next) after-call linkage)))
      (append-instruction-sequences
       (make-instruction-sequence '(proc) '()
        `((test (op primitive-procedure?) (reg proc))
          (branch (label ,primitive-branch))))
       (parallel-instruction-sequences
        (append-instruction-sequences
         compiled-branch
         (compile-proc-appl target compiled-linkage))
        (append-instruction-sequences
         primitive-branch
         (end-with-linkage linkage
          (make-instruction-sequence '(proc argl)
                                     (list target)
           `((assign ,target
                     (op apply-primitive-procedure)
                     (reg proc)
                     (reg argl)))))))
       after-call))))
\end{scheme}

\noindent
The primitive and compound branches, like the true and false branches in
\code{compile\-/if}, are appended using \code{parallel\-/instruction\-/sequences}
rather than the ordinary \code{append\-/instruction\-/sequences}, because they will
not be executed sequentially.

\subsubsection*{Applying compiled procedures}

The code that handles procedure application is the most subtle part of the
compiler, even though the instruction sequences it generates are very short.  A
compiled procedure (as constructed by \code{compile\-/lambda}) has an entry
point, which is a label that designates where the code for the procedure
starts.  The code at this entry point computes a result in \code{val} and
returns by executing the instruction \code{(goto (reg continue))}.  Thus, we
might expect the code for a compiled-procedure application (to be generated by
\code{compile\-/proc\-/appl}) with a given target and linkage to look like this if
the linkage is a label

\begin{scheme}
(assign continue (label proc-return))
 (assign val (op compiled-procedure-entry) (reg proc))
 (goto (reg val))
proc-return
 (assign ~\( \dark \langle \)~~\var{\dark target}~~\( \dark \rangle \)~ (reg val))   ~\textrm{; included if target is not \code{val}}~
 (goto (label ~\( \dark \langle \)~~\var{\dark linkage}~~\( \dark \rangle \)~))      ~\textrm{; linkage code}~
\end{scheme}

\noindent
or like this if the linkage is \code{return}.

\begin{scheme}
(save continue)
 (assign continue (label proc-return))
 (assign val (op compiled-procedure-entry) (reg proc))
 (goto (reg val))
proc-return
 (assign ~\( \dark \langle \)~~\var{\dark target}~~\( \dark \rangle \)~ (reg val))   ~\textrm{; included if target is not \code{val}}~
 (restore continue)
 (goto (reg continue))         ~\textrm{; linkage code}~
\end{scheme}

\noindent
This code sets up \code{continue} so that the procedure will return to a label
\code{proc\-/return} and jumps to the procedure's entry point.  The code at
\code{proc\-/return} transfers the procedure's result from \code{val} to the
target register (if necessary) and then jumps to the location specified by the
linkage.  (The linkage is always \code{return} or a label, because
\code{compile\-/procedure\-/call} replaces a \code{next} linkage for the
compound-procedure branch by an \code{after\-/call} label.)

In fact, if the target is not \code{val}, that is exactly the code our compiler
will generate.\footnote{Actually, we signal an error when the target is not
\code{val} and the linkage is \code{return}, since the only place we request
\code{return} linkages is in compiling procedures, and our convention is that
procedures return their values in \code{val}.}  Usually, however, the target is
\code{val} (the only time the compiler specifies a different register is when
targeting the evaluation of an operator to \code{proc}), so the procedure
result is put directly into the target register and there is no need to return
to a special location that copies it.  Instead, we simplify the code by setting
up \code{continue} so that the procedure will ``return'' directly to the place
specified by the caller's linkage:

\begin{scheme}
~\( \dark \langle \)~~\emph{set up \emph{continue} for linkage}~~\( \dark \rangle \)~
(assign val (op compiled-procedure-entry) (reg proc))
(goto (reg val))
\end{scheme}

\noindent
If the linkage is a label, we set up \code{continue} so that the procedure will
return to that label.  (That is, the \code{(goto (reg continue))} the procedure
ends with becomes equivalent to the \code{(goto (label \( \langle \)\var{linkage}\( \rangle \)))} at
\code{proc\-/return} above.)

\begin{scheme}
(assign continue (label ~\( \dark \langle \)~~\var{\dark linkage}~~\( \dark \rangle \)~))
(assign val (op compiled-procedure-entry) (reg proc))
(goto (reg val))
\end{scheme}

\noindent
If the linkage is \code{return}, we don't need to set up \code{continue} at
all: It already holds the desired location.  (That is, the \code{(goto (reg
continue))} the procedure ends with goes directly to the place where the
\code{(goto (reg continue))} at \code{proc\-/return} would have gone.)

\begin{scheme}
(assign val (op compiled-procedure-entry) (reg proc))
(goto (reg val))
\end{scheme}

\noindent
With this implementation of the \code{return} linkage, the compiler generates
tail-recursive code.  Calling a procedure as the final step in a procedure body
does a direct transfer, without saving any information on the stack.

Suppose instead that we had handled the case of a procedure call with a linkage
of \code{return} and a target of \code{val} as shown above for a non-\code{val}
target.  This would destroy tail recursion.  Our system would still give the
same value for any expression.  But each time we called a procedure, we would
save \code{continue} and return after the call to undo the (useless) save.
These extra saves would accumulate during a nest of procedure
calls.\footnote{Making a compiler generate tail-recursive code might seem like
a straightforward idea.  But most compilers for common languages, including C
and Pascal, do not do this, and therefore these languages cannot represent
iterative processes in terms of procedure call alone.  The difficulty with tail
recursion in these languages is that their implementations use the stack to
store procedure arguments and local variables as well as return addresses.  The
Scheme implementations described in this book store arguments and variables in
memory to be garbage-collected.  The reason for using the stack for variables
and arguments is that it avoids the need for garbage collection in languages
that would not otherwise require it, and is generally believed to be more
efficient.  Sophisticated Lisp compilers can, in fact, use the stack for
arguments without destroying tail recursion.  (See \link{Hanson 1990} for a
description.)  There is also some debate about whether stack allocation is
actually more efficient than garbage collection in the first place, but the
details seem to hinge on fine points of computer architecture.  (See \link{Appel 1987}
and \link{Miller and Rozas 1994} for opposing views on this issue.)}

\code{Compile\-/proc\-/appl} generates the above procedure-application code by
considering four cases, depending on whether the target for the call is
\code{val} and whether the linkage is \code{return}.  Observe that the
instruction sequences are declared to modify all the registers, since executing
the procedure body can change the registers in arbitrary ways.\footnote{The
variable \code{all\-/regs} is bound to the list of names of all the registers:

\begin{smallscheme}
(define all-regs '(env proc val argl continue))
\end{smallscheme}
} Also note that the code sequence for the case with target \code{val} and
linkage \code{return} is declared to need \code{continue}: Even though
\code{continue} is not explicitly used in the two-instruction sequence, we must
be sure that \code{continue} will have the correct value when we enter the
compiled procedure.

\begin{scheme}
(define (compile-proc-appl target linkage)
  (cond ((and (eq? target 'val) (not (eq? linkage 'return)))
         (make-instruction-sequence '(proc) all-regs
           `((assign continue (label ,linkage))
             (assign val (op compiled-procedure-entry)
                         (reg proc))
             (goto (reg val)))))
        ((and (not (eq? target 'val))
              (not (eq? linkage 'return)))
         (let ((proc-return (make-label 'proc-return)))
           (make-instruction-sequence '(proc) all-regs
            `((assign continue (label ,proc-return))
              (assign val (op compiled-procedure-entry)
                          (reg proc))
              (goto (reg val))
              ,proc-return
              (assign ,target (reg val))
              (goto (label ,linkage))))))
        ((and (eq? target 'val) (eq? linkage 'return))
         (make-instruction-sequence 
          '(proc continue) 
          all-regs
          '((assign val (op compiled-procedure-entry)
                        (reg proc))
            (goto (reg val)))))
        ((and (not (eq? target 'val))
              (eq? linkage 'return))
         (error "return linkage, target not val: COMPILE"
                target))))
\end{scheme}

\subsection{Combining Instruction Sequences}
\label{Section 5.5.4}

This section describes the details on how instruction sequences are represented
and combined.  Recall from \link{Section 5.5.1} that an instruction sequence is
represented as a list of the registers needed, the registers modified, and the
actual instructions.  We will also consider a label (symbol) to be a degenerate
case of an instruction sequence, which doesn't need or modify any registers.
So to determine the registers needed and modified by instruction sequences we
use the selectors

\begin{scheme}
(define (registers-needed s)
  (if (symbol? s) '() (car s)))
(define (registers-modified s)
  (if (symbol? s) '() (cadr s)))
(define (statements s)
  (if (symbol? s) (list s) (caddr s)))
\end{scheme}

\noindent
and to determine whether a given
sequence needs or modifies a given register we use the predicates

\begin{scheme}
(define (needs-register? seq reg)
  (memq reg (registers-needed seq)))
(define (modifies-register? seq reg)
  (memq reg (registers-modified seq)))
\end{scheme}

\noindent
In terms of these predicates and selectors, we can implement the various
instruction sequence combiners used throughout the compiler.

The basic combiner is \code{append\-/instruction\-/sequences}.  This takes as
arguments an arbitrary number of instruction sequences that are to be executed
sequentially and returns an instruction sequence whose statements are the
statements of all the sequences appended together.  The subtle point is to
determine the registers that are needed and modified by the resulting sequence.
It modifies those registers that are modified by any of the sequences; it needs
those registers that must be initialized before the first sequence can be run
(the registers needed by the first sequence), together with those registers
needed by any of the other sequences that are not initialized (modified) by
sequences preceding it.

The sequences are appended two at a time by \code{append\-/2\-/sequences}.  This
takes two instruction sequences \code{seq1} and \code{seq2} and returns the
instruction sequence whose statements are the statements of \code{seq1}
followed by the statements of \code{seq2}, whose modified registers are those
registers that are modified by either \code{seq1} or \code{seq2}, and whose
needed registers are the registers needed by \code{seq1} together with those
registers needed by \code{seq2} that are not modified by \code{seq1}.  (In
terms of set operations, the new set of needed registers is the union of the
set of registers needed by \code{seq1} with the set difference of the registers
needed by \code{seq2} and the registers modified by \code{seq1}.)  Thus,
\code{append\-/instruction\-/sequences} is implemented as follows:

\begin{scheme}
(define (append-instruction-sequences . seqs)
  (define (append-2-sequences seq1 seq2)
    (make-instruction-sequence
     (list-union
      (registers-needed seq1)
      (list-difference (registers-needed seq2)
                       (registers-modified seq1)))
     (list-union (registers-modified seq1)
                 (registers-modified seq2))
     (append (statements seq1) (statements seq2))))
  (define (append-seq-list seqs)
    (if (null? seqs)
        (empty-instruction-sequence)
        (append-2-sequences
         (car seqs)
         (append-seq-list (cdr seqs)))))
  (append-seq-list seqs))
\end{scheme}

\noindent
This procedure uses some simple operations for manipulating sets represented as
lists, similar to the (unordered) set representation described in 
\link{Section 2.3.3}:

\begin{scheme}
(define (list-union s1 s2)
  (cond ((null? s1) s2)
        ((memq (car s1) s2) (list-union (cdr s1) s2))
        (else (cons (car s1) (list-union (cdr s1) s2)))))
(define (list-difference s1 s2)
  (cond ((null? s1) '())
        ((memq (car s1) s2) (list-difference (cdr s1) s2))
        (else (cons (car s1)
                    (list-difference (cdr s1) s2)))))
\end{scheme}

\noindent
\code{Preserving}, the second major instruction sequence combiner, takes a list
of registers \code{regs} and two instruction sequences \code{seq1} and
\code{seq2} that are to be executed sequentially.  It returns an instruction
sequence whose statements are the statements of \code{seq1} followed by the
statements of \code{seq2}, with appropriate \code{save} and \code{restore}
instructions around \code{seq1} to protect the registers in \code{regs} that
are modified by \code{seq1} but needed by \code{seq2}.  To accomplish this,
\code{preserving} first creates a sequence that has the required \code{save}s
followed by the statements of \code{seq1} followed by the required
\code{restore}s.  This sequence needs the registers being saved and restored in
addition to the registers needed by \code{seq1}, and modifies the registers
modified by \code{seq1} except for the ones being saved and restored.  This
augmented sequence and \code{seq2} are then appended in the usual way.  The
following procedure implements this strategy recursively, walking down the list
of registers to be preserved:\footnote{Note that \code{preserving} calls
\code{append} with three arguments.  Though the definition of \code{append}
shown in this book accepts only two arguments, Scheme standardly provides an
\code{append} procedure that takes an arbitrary number of arguments.}

\begin{scheme}
(define (preserving regs seq1 seq2)
  (if (null? regs)
      (append-instruction-sequences seq1 seq2)
      (let ((first-reg (car regs)))
        (if (and (needs-register? seq2 first-reg)
                 (modifies-register? seq1 first-reg))
            (preserving (cdr regs)
             (make-instruction-sequence
              (list-union (list first-reg)
                          (registers-needed seq1))
              (list-difference (registers-modified seq1)
                               (list first-reg))
              (append `((save ,first-reg))
                      (statements seq1)
                      `((restore ,first-reg))))
             seq2)
            (preserving (cdr regs) seq1 seq2)))))
\end{scheme}

\noindent
Another sequence combiner, \code{tack\-/on\-/instruction\-/sequence}, is used by
\code{compile\-/lambda} to append a procedure body to another sequence.  Because
the procedure body is not ``in line'' to be executed as part of the combined
sequence, its register use has no impact on the register use of the sequence in
which it is embedded.  We thus ignore the procedure body's sets of needed and
modified registers when we tack it onto the other sequence.

\begin{scheme}
(define (tack-on-instruction-sequence seq body-seq)
  (make-instruction-sequence
   (registers-needed seq)
   (registers-modified seq)
   (append (statements seq)
           (statements body-seq))))
\end{scheme}

\noindent
\code{Compile\-/if} and \code{compile\-/procedure\-/call} use a special combiner
called \code{parallel\-/instruction\-/sequences} to append the two alternative
branches that follow a test.  The two branches will never be executed
sequentially; for any particular evaluation of the test, one branch or the
other will be entered.  Because of this, the registers needed by the second
branch are still needed by the combined sequence, even if these are modified by
the first branch.

\begin{scheme}
(define (parallel-instruction-sequences seq1 seq2)
  (make-instruction-sequence
   (list-union (registers-needed seq1)
               (registers-needed seq2))
   (list-union (registers-modified seq1)
               (registers-modified seq2))
   (append (statements seq1)
           (statements seq2))))
\end{scheme}

\subsection{An Example of Compiled Code}
\label{Section 5.5.5}

Now that we have seen all the elements of the compiler, let us examine an
example of compiled code to see how things fit together.  We will compile the
definition of a recursive \code{factorial} procedure by calling \code{compile}:

\begin{scheme}
(compile
 '(define (factorial n)
    (if (= n 1)
        1
        (* (factorial (- n 1)) n)))
 'val
 'next)
\end{scheme}

\noindent
We have specified that the value of the \code{define} expression should be
placed in the \code{val} register.  We don't care what the compiled code does
after executing the \code{define}, so our choice of \code{next} as the linkage
descriptor is arbitrary.

\code{Compile} determines that the expression is a definition, so it calls
\code{compile\-/definition} to compile code to compute the value to be assigned
(targeted to \code{val}), followed by code to install the definition, followed
by code to put the value of the \code{define} (which is the symbol \code{ok})
into the target register, followed finally by the linkage code.  \code{Env} is
preserved around the computation of the value, because it is needed in order to
install the definition.  Because the linkage is \code{next}, there is no
linkage code in this case.  The skeleton of the compiled code is thus

\begin{scheme}
~\( \dark \langle \)~~\emph{save \emph{env} if modified by code to compute value}~~\( \dark \rangle \)~
  ~\( \dark \langle \)~~\emph{compilation of definition value, target \emph{val}, linkage \emph{next}}~~\( \dark \rangle \)~
  ~\( \dark \langle \)~~\emph{restore \emph{env} if saved above}~~\( \dark \rangle \)~
  (perform (op define-variable!)
           (const factorial)
           (reg val)
           (reg env))
  (assign val (const ok))
\end{scheme}

\noindent
The expression that is to be compiled to produce the value for the variable
\code{factorial} is a \code{lambda} expression whose value is the procedure
that computes factorials.  \code{Compile} handles this by calling
\code{compile\-/lambda}, which compiles the procedure body, labels it as a new
entry point, and generates the instruction that will combine the procedure body
at the new entry point with the run-time environment and assign the result to
\code{val}.  The sequence then skips around the compiled procedure code, which
is inserted at this point.  The procedure code itself begins by extending the
procedure's definition environment by a frame that binds the formal parameter
\code{n} to the procedure argument.  Then comes the actual procedure body.
Since this code for the value of the variable doesn't modify the \code{env}
register, the optional \code{save} and \code{restore} shown above aren't
generated.  (The procedure code at \code{entry2} isn't executed at this point,
so its use of \code{env} is irrelevant.)  Therefore, the skeleton for the
compiled code becomes

\begin{scheme}
  (assign val
          (op make-compiled-procedure)
          (label entry2)
          (reg env))
  (goto (label after-lambda1))
entry2
  (assign env (op compiled-procedure-env) (reg proc))
  (assign env
          (op extend-environment)
          (const (n))
          (reg argl)
          (reg env))
  ~\( \dark \langle \)~~\emph{compilation of procedure body}~~\( \dark \rangle \)~
after-lambda1
  (perform (op define-variable!)
           (const factorial)
           (reg val)
           (reg env))
  (assign val (const ok))
\end{scheme}

\noindent
A procedure body is always compiled (by \code{compile\-/lambda\-/body}) as a
sequence with target \code{val} and linkage \code{return}.  The sequence in
this case consists of a single \code{if} expression:

\begin{scheme}
(if (= n 1)
    1
    (* (factorial (- n 1)) n))
\end{scheme}

\noindent
\code{Compile\-/if} generates code that first computes the predicate (targeted to
\code{val}), then checks the result and branches around the true branch if the
predicate is false.  \code{Env} and \code{continue} are preserved around the
predicate code, since they may be needed for the rest of the \code{if}
expression.  Since the \code{if} expression is the final expression (and only
expression) in the sequence making up the procedure body, its target is
\code{val} and its linkage is \code{return}, so the true and false branches are
both compiled with target \code{val} and linkage \code{return}.  (That is, the
value of the conditional, which is the value computed by either of its
branches, is the value of the procedure.)

\begin{scheme}
~\( \dark \langle \)~~\emph{save \emph{continue}, \emph{env} if modified by predicate and needed by branches}~~\( \dark \rangle \)~
  ~\( \dark \langle \)~~\emph{compilation of predicate, target \emph{val}, linkage \emph{next}}~~\( \dark \rangle \)~
  ~\( \dark \langle \)~~\emph{restore \emph{continue}, \emph{env} if saved above}~~\( \dark \rangle \)~
  (test (op false?) (reg val))
  (branch (label false-branch4))
true-branch5
  ~\( \dark \langle \)~~\emph{compilation of true branch, target \emph{val}, linkage \emph{return}}~~\( \dark \rangle \)~
false-branch4
  ~\( \dark \langle \)~~\emph{compilation of false branch, target \emph{val}, linkage \emph{return}}~~\( \dark \rangle \)~
after-if3
\end{scheme}

\noindent
The predicate \code{(= n 1)} is a procedure call.  This looks up the operator
(the symbol \code{=}) and places this value in \code{proc}.  It then assembles
the arguments \code{1} and the value of \code{n} into \code{argl}.  Then it
tests whether \code{proc} contains a primitive or a compound procedure, and
dispatches to a primitive branch or a compound branch accordingly.  Both
branches resume at the \code{after\-/call} label.  The requirements to preserve
registers around the evaluation of the operator and operands don't result in
any saving of registers, because in this case those evaluations don't modify
the registers in question.

\begin{scheme}
  (assign proc
          (op lookup-variable-value) (const =) (reg env))
  (assign val (const 1))
  (assign argl (op list) (reg val))
  (assign val
          (op lookup-variable-value) (const n) (reg env))
  (assign argl (op cons) (reg val) (reg argl))
  (test (op primitive-procedure?) (reg proc))
  (branch (label primitive-branch17))
compiled-branch16
  (assign continue (label after-call15))
  (assign val (op compiled-procedure-entry) (reg proc))
  (goto (reg val))
primitive-branch17
  (assign val
          (op apply-primitive-procedure)
          (reg proc)
          (reg argl))
after-call15
\end{scheme}

\noindent
The true branch, which is the constant 1, compiles (with target \code{val} and
linkage \code{return}) to

\begin{scheme}
(assign val (const 1))
(goto (reg continue))
\end{scheme}

\noindent
The code for the false branch is another procedure call, where the procedure
is the value of the symbol \code{*}, and the arguments are \code{n} and the
result of another procedure call (a call to \code{factorial}).  Each of these
calls sets up \code{proc} and \code{argl} and its own primitive and compound
branches.  \link{Figure 5.17} shows the complete compilation of the definition
of the \code{factorial} procedure.  Notice that the possible \code{save} and
\code{restore} of \code{continue} and \code{env} around the predicate, shown
above, are in fact generated, because these registers are modified by the
procedure call in the predicate and needed for the procedure call and the
\code{return} linkage in the branches.

\begin{quote}
\heading{\phantomsection\label{Exercise 5.33}Exercise 5.33:} Consider the following definition
of a factorial procedure, which is slightly different from the one given above:

\begin{scheme}
(define (factorial-alt n)
  (if (= n 1)
      1
      (* n (factorial-alt (- n 1)))))
\end{scheme}

Compile this procedure and compare the resulting code with that produced for
\code{factorial}.  Explain any differences you find.  Does either program
execute more efficiently than the other?
\end{quote}

\begin{quote}
\heading{\phantomsection\label{Exercise 5.34}Exercise 5.34:} Compile the iterative factorial
procedure

\begin{scheme}
(define (factorial n)
  (define (iter product counter)
    (if (> counter n)
        product
        (iter (* counter product)
              (+ counter 1))))
  (iter 1 1))
\end{scheme}

Annotate the resulting code, showing the essential difference between the code
for iterative and recursive versions of \code{factorial} that makes one process
build up stack space and the other run in constant stack space.
\end{quote}

\begin{quote}
\heading{\phantomsection\label{Figure 5.17}Figure 5.17:} \( \downarrow \) Compilation of the definition of the
\code{factorial} procedure.

\begin{smallscheme}
~\textrm{;; construct the procedure and skip over code for the procedure body}~
  (assign val
          (op make-compiled-procedure) 
          (label entry2) 
          (reg env))
  (goto (label after-lambda1))
entry2     ~\textrm{; calls to \code{factorial} will enter here}~
  (assign env (op compiled-procedure-env) (reg proc))
  (assign env
          (op extend-environment) 
          (const (n)) 
          (reg argl) 
          (reg env))
~\textrm{;; begin actual procedure body}~
  (save continue)
  (save env)
~\textrm{;; compute \code{(= n 1)}}~
  (assign proc 
          (op lookup-variable-value) 
          (const =) 
          (reg env))
  (assign val (const 1))
  (assign argl (op list) (reg val))
  (assign val 
          (op lookup-variable-value) 
          (const n) 
          (reg env))
  (assign argl (op cons) (reg val) (reg argl))
  (test (op primitive-procedure?) (reg proc))
  (branch (label primitive-branch17))
compiled-branch16
  (assign continue (label after-call15))
  (assign val (op compiled-procedure-entry) (reg proc))
  (goto (reg val))
primitive-branch17
  (assign val 
          (op apply-primitive-procedure) 
          (reg proc) 
          (reg argl))
after-call15   ~\textrm{; \code{val} now contains result of \code{(= n 1)}}~
  (restore env)
  (restore continue)
  (test (op false?) (reg val))
  (branch (label false-branch4))
true-branch5  ~\textrm{; return 1}~
  (assign val (const 1))
  (goto (reg continue))
false-branch4
~\textrm{;; compute and return \code{(* (factorial (- n 1)) n)}}~
  (assign proc 
          (op lookup-variable-value) 
          (const *) 
          (reg env))
  (save continue)
  (save proc)   ~\textrm{; save \code{*}}~ procedure
  (assign val 
          (op lookup-variable-value) 
          (const n) 
          (reg env))
  (assign argl (op list) (reg val))
  (save argl)   ~\textrm{; save partial argument list for \code{*}}~
~\textrm{;; compute \code{(factorial (- n 1))}, which is the other argument for \code{*}}~
  (assign proc
          (op lookup-variable-value) 
          (const factorial) 
          (reg env))
  (save proc)  ~\textrm{; save \code{factorial} procedure}~
~\textrm{;; compute \code{(- n 1)}, which is the argument for \code{factorial}}~
  (assign proc 
          (op lookup-variable-value)
          (const -) 
          (reg env))
  (assign val (const 1))
  (assign argl (op list) (reg val))
  (assign val 
          (op lookup-variable-value) 
          (const n) 
          (reg env))
  (assign argl (op cons) (reg val) (reg argl))
  (test (op primitive-procedure?) (reg proc))
  (branch (label primitive-branch8))
compiled-branch7
  (assign continue (label after-call6))
  (assign val (op compiled-procedure-entry) (reg proc))
  (goto (reg val))
primitive-branch8
  (assign val 
          (op apply-primitive-procedure) 
          (reg proc) 
          (reg argl))
after-call6   ~\textrm{; \code{val} now contains result of \code{(- n 1)}}~
  (assign argl (op list) (reg val))
  (restore proc) ~\textrm{; restore \code{factorial}}~
~\textrm{;; apply \code{factorial}}~
  (test (op primitive-procedure?) (reg proc))
  (branch (label primitive-branch11))
compiled-branch10
  (assign continue (label after-call9))
  (assign val (op compiled-procedure-entry) (reg proc))
  (goto (reg val))
primitive-branch11
  (assign val 
          (op apply-primitive-procedure) 
          (reg proc) 
          (reg argl))
after-call9      ~\textrm{; \code{val} now contains result of \code{(factorial (- n 1))}}~
  (restore argl) ~\textrm{; restore partial argument list for \code{*}}~
  (assign argl (op cons) (reg val) (reg argl))
  (restore proc) ~\textrm{; restore \code{*}}~
  (restore continue)
~\textrm{;; apply \code{*} and return its value}~
  (test (op primitive-procedure?) (reg proc))
  (branch (label primitive-branch14))
compiled-branch13
~\textrm{;; note that a compound procedure here is called tail-recursively}~
  (assign val (op compiled-procedure-entry) (reg proc))
  (goto (reg val))
primitive-branch14
  (assign val 
          (op apply-primitive-procedure) 
          (reg proc) 
          (reg argl))
  (goto (reg continue))
after-call12
after-if3
after-lambda1
~\textrm{;; assign the procedure to the variable \code{factorial}}~
  (perform (op define-variable!) 
           (const factorial) 
           (reg val) 
           (reg env))
  (assign val (const ok))
\end{smallscheme}

\end{quote}

\begin{quote}
\heading{\phantomsection\label{Exercise 5.35}Exercise 5.35:} What expression was compiled to
produce the code shown in \link{Figure 5.18}?
\end{quote}

\begin{quote}
\heading{\phantomsection\label{Figure 5.18}Figure 5.18:} \( \downarrow \) An example of compiler output.  See
\link{Exercise 5.35}.

\begin{smallscheme}
(assign val 
        (op make-compiled-procedure) 
        (label entry16) 
        (reg env))
  (goto (label after-lambda15))
entry16
  (assign env (op compiled-procedure-env) (reg proc))
  (assign env
          (op extend-environment) 
          (const (x)) 
          (reg argl) 
          (reg env))
  (assign proc 
          (op lookup-variable-value) 
          (const +) 
          (reg env))
  (save continue)
  (save proc)
  (save env)
  (assign proc 
          (op lookup-variable-value) 
          (const g) 
          (reg env))
  (save proc)
  (assign proc 
          (op lookup-variable-value) 
          (const +) 
          (reg env))
  (assign val (const 2))
  (assign argl (op list) (reg val))
  (assign val 
          (op lookup-variable-value) 
          (const x) 
          (reg env))
  (assign argl (op cons) (reg val) (reg argl))
  (test (op primitive-procedure?) (reg proc))
  (branch (label primitive-branch19))
compiled-branch18
  (assign continue (label after-call17))
  (assign val (op compiled-procedure-entry) (reg proc))
  (goto (reg val))
primitive-branch19
  (assign val 
          (op apply-primitive-procedure) 
          (reg proc) 
          (reg argl))
after-call17
  (assign argl (op list) (reg val))
  (restore proc)
  (test (op primitive-procedure?) (reg proc))
  (branch (label primitive-branch22))
compiled-branch21
  (assign continue (label after-call20))
  (assign val (op compiled-procedure-entry) (reg proc))
  (goto (reg val))
primitive-branch22
  (assign val 
          (op apply-primitive-procedure) 
          (reg proc) 
          (reg argl))
after-call20
  (assign argl (op list) (reg val))
  (restore env)
  (assign val 
          (op lookup-variable-value) 
          (const x) 
          (reg env))
  (assign argl (op cons) (reg val) (reg argl))
  (restore proc)
  (restore continue)
  (test (op primitive-procedure?) (reg proc))
  (branch (label primitive-branch25))
compiled-branch24
  (assign val 
          (op compiled-procedure-entry) 
          (reg proc))
  (goto (reg val))
primitive-branch25
  (assign val 
          (op apply-primitive-procedure) 
          (reg proc) 
          (reg argl))
  (goto (reg continue))
after-call23
after-lambda15
  (perform (op define-variable!) 
           (const f) 
           (reg val) 
           (reg env))
  (assign val (const ok))
\end{smallscheme}

\end{quote}

\begin{quote}
\heading{\phantomsection\label{Exercise 5.36}Exercise 5.36:} What order of evaluation does our
compiler produce for operands of a combination?  Is it left-to-right,
right-to-left, or some other order?  Where in the compiler is this order
determined?  Modify the compiler so that it produces some other order of
evaluation.  (See the discussion of order of evaluation for the
explicit-control evaluator in \link{Section 5.4.1}.)  How does changing the
order of operand evaluation affect the efficiency of the code that constructs
the argument list?
\end{quote}

\begin{quote}
\heading{\phantomsection\label{Exercise 5.37}Exercise 5.37:} One way to understand the
compiler's \code{preserving} mechanism for optimizing stack usage is to see
what extra operations would be generated if we did not use this idea.  Modify
\code{preserving} so that it always generates the \code{save} and
\code{restore} operations.  Compile some simple expressions and identify the
unnecessary stack operations that are generated.  Compare the code to that
generated with the \code{preserving} mechanism intact.
\end{quote}

\begin{quote}
\heading{\phantomsection\label{Exercise 5.38}Exercise 5.38:} Our compiler is clever about
avoiding unnecessary stack operations, but it is not clever at all when it
comes to compiling calls to the primitive procedures of the language in terms
of the primitive operations supplied by the machine.  For example, consider how
much code is compiled to compute \code{(+ a 1)}: The code sets up an argument
list in \code{argl}, puts the primitive addition procedure (which it finds by
looking up the symbol \code{+} in the environment) into \code{proc}, and tests
whether the procedure is primitive or compound.  The compiler always generates
code to perform the test, as well as code for primitive and compound branches
(only one of which will be executed).  We have not shown the part of the
controller that implements primitives, but we presume that these instructions
make use of primitive arithmetic operations in the machine's data paths.
Consider how much less code would be generated if the compiler could
\newterm{open-code} primitives---that is, if it could generate code to directly
use these primitive machine operations.  The expression \code{(+ a 1)} might be
compiled into something as simple as\footnote{We have used the same symbol
\code{+} here to denote both the source-language procedure and the machine
operation.  In general there will not be a one-to-one correspondence between
primitives of the source language and primitives of the machine.}

\begin{scheme}
(assign
 val (op lookup-variable-value) (const a) (reg env))
(assign val (op +) (reg val) (const 1))
\end{scheme}

In this exercise we will extend our compiler to support open coding of selected
primitives.  Special-purpose code will be generated for calls to these
primitive procedures instead of the general procedure-application code.  In
order to support this, we will augment our machine with special argument
registers \code{arg1} and \code{arg2}.  The primitive arithmetic operations of
the machine will take their inputs from \code{arg1} and \code{arg2}. The
results may be put into \code{val}, \code{arg1}, or \code{arg2}.

The compiler must be able to recognize the application of an open-coded
primitive in the source program.  We will augment the dispatch in the
\code{compile} procedure to recognize the names of these primitives in addition
to the reserved words (the special forms) it currently
recognizes.\footnote{Making the primitives into reserved words is in general a
bad idea, since a user cannot then rebind these names to different procedures.
Moreover, if we add reserved words to a compiler that is in use, existing
programs that define procedures with these names will stop working.  See
\link{Exercise 5.44} for ideas on how to avoid this problem.} For each special
form our compiler has a code generator.  In this exercise we will construct a
family of code generators for the open-coded primitives.

\begin{enumerate}[a.]

\item
The open-coded primitives, unlike the special forms, all need their operands
evaluated.  Write a code generator \code{spread\-/arguments} for use by all the
open-coding code generators.  \code{Spread\-/arguments} should take an operand
list and compile the given operands targeted to successive argument registers.
Note that an operand may contain a call to an open-coded primitive, so argument
registers will have to be preserved during operand evaluation.

\item
For each of the primitive procedures \code{=}, \code{*}, \code{-}, and
\code{+}, write a code generator that takes a combination with that operator,
together with a target and a linkage descriptor, and produces code to spread
the arguments into the registers and then perform the operation targeted to the
given target with the given linkage.  You need only handle expressions with two
operands.  Make \code{compile} dispatch to these code generators.

\item
Try your new compiler on the \code{factorial} example.  Compare the resulting
code with the result produced without open coding.

\item
Extend your code generators for \code{+} and \code{*} so that they can handle
expressions with arbitrary numbers of operands.  An expression with more than
two operands will have to be compiled into a sequence of operations, each with
only two inputs.

\end{enumerate}
\end{quote}

\subsection{Lexical Addressing}
\label{Section 5.5.6}

One of the most common optimizations performed by compilers is the optimization
of variable lookup.  Our compiler, as we have implemented it so far, generates
code that uses the \code{lookup\-/variable\-/value} operation of the evaluator
machine.  This searches for a variable by comparing it with each variable that
is currently bound, working frame by frame outward through the run-time
environment.  This search can be expensive if the frames are deeply nested or
if there are many variables.  For example, consider the problem of looking up
the value of \code{x} while evaluating the expression \code{(* x y z)} in an
application of the procedure that is returned by

\enlargethispage{\baselineskip}

\begin{scheme}
(let ((x 3) (y 4))
  (lambda (a b c d e)
    (let ((y (* a b x)) (z (+ c d x)))
      (* x y z))))
\end{scheme}

\noindent
Since a \code{let} expression is just syntactic sugar for a \code{lambda}
combination, this expression is equivalent to

\begin{scheme}
((lambda (x y)
   (lambda (a b c d e)
     ((lambda (y z) (* x y z))
      (* a b x)
      (+ c d x))))
 3
 4)
\end{scheme}

\noindent
Each time \code{lookup\-/variable\-/value} searches for \code{x}, it must determine
that the symbol \code{x} is not \code{eq?} to \code{y} or \code{z} (in the
first frame), nor to \code{a}, \code{b}, \code{c}, \code{d}, or \code{e} (in
the second frame).  We will assume, for the moment, that our programs do not
use \code{define}---that variables are bound only with \code{lambda}.  Because
our language is lexically scoped, the run-time environment for any expression
will have a structure that parallels the lexical structure of the program in
which the expression appears.\footnote{This is not true if we allow internal
definitions, unless we scan them out.  See \link{Exercise 5.43}.  } Thus, the
compiler can know, when it analyzes the above expression, that each time the
procedure is applied the variable \code{x} in \code{(* x y z)} will be found
two frames out from the current frame and will be the first variable in that
frame.

We can exploit this fact by inventing a new kind of variable-lookup operation,
\code{lexical\-/address\-/lookup}, that takes as arguments an environment and a
\newterm{lexical address} that consists of two numbers: a \newterm{frame
number}, which specifies how many frames to pass over, and a
\newterm{displacement number}, which specifies how many variables to pass over
in that frame.  \code{Lexical\-/address\-/lookup} will produce the value of the
variable stored at that lexical address relative to the current environment.
If we add the \code{lexical\-/address\-/lookup} operation to our machine, we can
make the compiler generate code that references variables using this operation,
rather than \code{lookup\-/variable\-/value}.  Similarly, our compiled code can use
a new \code{lexical\-/address\-/set!}  operation instead of
\code{set\-/variable\-/value!}.

In order to generate such code, the compiler must be able to determine the
lexical address of a variable it is about to compile a reference to.  The
lexical address of a variable in a program depends on where one is in the code.
For example, in the following program, the address of \code{x} in expression
\( \langle \)\var{e1}\( \kern0.08em\rangle \) is (2, 0)---two frames back and the first variable in the frame.  At
that point \code{y} is at address (0, 0) and \code{c} is at address (1, 2).  In
expression \( \langle \)\var{e2}\( \kern0.09em\rangle \), \code{x} is at (1, 0), \code{y} is at (1, 1), and \code{c}
is at (0, 2).

\begin{scheme}
((lambda (x y)
   (lambda (a b c d e)
     ((lambda (y z) ~\( \dark \langle \)~~\var{\dark e1}~~\( \dark \rangle \)~)
      ~\( \dark \langle \)~~\var{\dark e2}~~\( \dark \rangle \)~
      (+ c d x))))
 3
 4)
\end{scheme}

\noindent
One way for the compiler to produce code that uses lexical addressing is to
maintain a data structure called a \newterm{compile-time environment}.  This
keeps track of which variables will be at which positions in which frames in
the run-time environment when a particular variable-access operation is
executed.  The compile-time environment is a list of frames, each containing a
list of variables.  (There will of course be no values bound to the variables,
since values are not computed at compile time.)  The compile-time environment
becomes an additional argument to \code{compile} and is passed along to each
code generator.  The top-level call to \code{compile} uses an empty
compile-time environment.  When a \code{lambda} body is compiled,
\code{compile\-/lambda\-/body} extends the compile-time environment by a frame
containing the procedure's parameters, so that the sequence making up the body
is compiled with that extended environment.  At each point in the compilation,
\code{compile\-/variable} and \code{compile\-/assignment} use the compile-time
environment in order to generate the appropriate lexical addresses.

\link{Exercise 5.39} through \link{Exercise 5.43} describe how to complete this
sketch of the lexical-addressing strategy in order to incorporate lexical
lookup into the compiler.  \link{Exercise 5.44} describes another use for the
compile-time environment.

\begin{quote}
\heading{\phantomsection\label{Exercise 5.39}Exercise 5.39:} Write a procedure
\code{lexical\-/address\-/lookup} that implements the new lookup operation.  It
should take two arguments---a lexical address and a run-time environment---and
return the value of the variable stored at the specified lexical address.
\code{Lexical\-/address\-/lookup} should signal an error if the value of the
variable is the symbol \code{*unassigned*}.\footnote{This is the modification
to variable lookup required if we implement the scanning method to eliminate
internal definitions (\link{Exercise 5.43}).  We will need to eliminate these
definitions in order for lexical addressing to work.} Also write a procedure
\code{lexical\-/address\-/set!} that implements the operation that changes the
value of the variable at a specified lexical address.
\end{quote}

\begin{quote}
\heading{\phantomsection\label{Exercise 5.40}Exercise 5.40:} Modify the compiler to maintain
the compile-time environment as described above.  That is, add a
compile-time-environment argument to \code{compile} and the various code
generators, and extend it in \code{compile\-/lambda\-/body}.
\end{quote}

\begin{quote}
\heading{\phantomsection\label{Exercise 5.41}Exercise 5.41:} Write a procedure
\code{find\-/variable} that takes as arguments a variable and a compile-time
environment and returns the lexical address of the variable with respect to
that environment.  For example, in the program fragment that is shown above,
the compile-time environment during the compilation of expression \( \langle \)\var{e1}\( \kern0.08em\rangle \) is
\code{((y z) (a b c d e) (x y))}.  \code{Find\-/variable} should produce

\begin{scheme}
(find-variable 'c '((y z) (a b c d e) (x y)))
~\textit{(1 2)}~
(find-variable 'x '((y z) (a b c d e) (x y)))
~\textit{(2 0)}~
(find-variable 'w '((y z) (a b c d e) (x y)))
~\textit{not-found}~
\end{scheme}
\end{quote}

\begin{quote}
\heading{\phantomsection\label{Exercise 5.42}Exercise 5.42:} Using \code{find\-/variable} from
\link{Exercise 5.41}, rewrite \code{compile\-/variable} and
\code{compile\-/assignment} to output lexical-address instructions.  In cases
where \code{find\-/variable} returns \code{not\-/found} (that is, where the
variable is not in the compile-time environment), you should have the code
generators use the evaluator operations, as before, to search for the binding.
(The only place a variable that is not found at compile time can be is in the
global environment, which is part of the run-time environment but is not part
of the compile-time environment.\footnote{Lexical addresses cannot be used to
access variables in the global environment, because these names can be defined
and redefined interactively at any time.  With internal definitions scanned
out, as in \link{Exercise 5.43}, the only definitions the compiler sees are
those at top level, which act on the global environment.  Compilation of a
definition does not cause the defined name to be entered in the compile-time
environment.}  Thus, if you wish, you may have the evaluator operations look
directly in the global environment, which can be obtained with the operation
\code{(op get\-/global\-/environment)}, instead of having them search the whole
run-time environment found in \code{env}.)  Test the modified compiler on a few
simple cases, such as the nested \code{lambda} combination at the beginning of
this section.
\end{quote}

\begin{quote}
\heading{\phantomsection\label{Exercise 5.43}Exercise 5.43:} We argued in \link{Section 4.1.6}
that internal definitions for block structure should not be considered ``real''
\code{define}s.  Rather, a procedure body should be interpreted as if the
internal variables being defined were installed as ordinary \code{lambda}
variables initialized to their correct values using \code{set!}.  
\link{Section 4.1.6} and \link{Exercise 4.16} showed how to modify the metacircular
interpreter to accomplish this by scanning out internal definitions.  Modify
the compiler to perform the same transformation before it compiles a procedure
body.
\end{quote}

\begin{quote}
\heading{\phantomsection\label{Exercise 5.44}Exercise 5.44:} In this section we have focused
on the use of the compile-time environment to produce lexical addresses.  But
there are other uses for compile-time environments.  For instance, in
\link{Exercise 5.38} we increased the efficiency of compiled code by open-coding
primitive procedures.  Our implementation treated the names of open-coded
procedures as reserved words.  If a program were to rebind such a name, the
mechanism described in \link{Exercise 5.38} would still open-code it as a
primitive, ignoring the new binding.  For example, consider the procedure

\begin{scheme}
(lambda (+ * a b x y)
  (+ (* a x) (* b y)))
\end{scheme}

\noindent
which computes a linear combination of \code{x} and \code{y}.  We might call it
with arguments \code{+matrix}, \code{*matrix}, and four matrices, but the
open-coding compiler would still open-code the \code{+} and the \code{*} in
\code{(+ (* a x) (* b y))} as primitive \code{+} and \code{*}.  Modify the
open-coding compiler to consult the compile-time environment in order to
compile the correct code for expressions involving the names of primitive
procedures.  (The code will work correctly as long as the program does not
\code{define} or \code{set!} these names.)
\end{quote}

\subsection{Interfacing Compiled Code to the Evaluator}
\label{Section 5.5.7}

We have not yet explained how to load compiled code into the evaluator machine
or how to run it.  We will assume that the explicit-control-evaluator machine
has been defined as in \link{Section 5.4.4}, with the additional operations
specified in \link{Footnote 38}.  We will implement a procedure
\code{compile\-/and\-/go} that compiles a Scheme expression, loads the resulting
object code into the evaluator machine, and causes the machine to run the code
in the evaluator global environment, print the result, and enter the
evaluator's driver loop.  We will also modify the evaluator so that interpreted
expressions can call compiled procedures as well as interpreted ones.  We can
then put a compiled procedure into the machine and use the evaluator to call
it:

\begin{scheme}
(compile-and-go
 '(define (factorial n)
    (if (= n 1)
        1
        (* (factorial (- n 1)) n))))
~\textit{;;; EC-Eval value:}~
~\textit{ok}~
~\textit{;;; EC-Eval input:}~
(factorial 5)
~\textit{;;; EC-Eval value:}~
~\textit{120}~
\end{scheme}

\noindent
To allow the evaluator to handle compiled procedures (for example, to evaluate
the call to \code{factorial} above), we need to change the code at
\code{apply\-/dispatch} (\link{Section 5.4.1}) so that it recognizes compiled
procedures (as distinct from compound or primitive procedures) and transfers
control directly to the entry point of the compiled code:\footnote{Of course,
compiled procedures as well as interpreted procedures are compound
(nonprimitive).  For compatibility with the terminology used in the
explicit-control evaluator, in this section we will use ``compound'' to mean
interpreted (as opposed to compiled).}

\begin{scheme}
apply-dispatch
  (test (op primitive-procedure?) (reg proc))
  (branch (label primitive-apply))
  (test (op compound-procedure?) (reg proc))
  (branch (label compound-apply))
  (test (op compiled-procedure?) (reg proc))
  (branch (label compiled-apply))
  (goto (label unknown-procedure-type))
compiled-apply
  (restore continue)
  (assign val (op compiled-procedure-entry) (reg proc))
  (goto (reg val))
\end{scheme}

\noindent
Note the restore of \code{continue} at \code{compiled\-/apply}.  Recall that the
evaluator was arranged so that at \code{apply\-/dispatch}, the continuation would
be at the top of the stack.  The compiled code entry point, on the other hand,
expects the continuation to be in \code{continue}, so \code{continue} must be
restored before the compiled code is executed.

To enable us to run some compiled code when we start the evaluator machine, we
add a \code{branch} instruction at the beginning of the evaluator machine,
which causes the machine to go to a new entry point if the \code{flag} register
is set.\footnote{Now that the evaluator machine starts with a \code{branch}, we
must always initialize the \code{flag} register before starting the evaluator
machine.  To start the machine at its ordinary read-eval-print loop, we could
use

\begin{smallscheme}
(define (start-eceval)
  (set! the-global-environment (setup-environment))
  (set-register-contents! eceval 'flag false)
  (start eceval))
\end{smallscheme}
}

\begin{scheme}
(branch (label external-entry))      ~\textrm{; branches if \code{flag} is set}~
read-eval-print-loop
  (perform (op initialize-stack))
  ~\( \dots \)~
\end{scheme}

\enlargethispage{\baselineskip}

\noindent
\code{External\-/entry} assumes that the machine is started with \code{val}
containing the location of an instruction sequence that puts a result into
\code{val} and ends with \code{(goto (reg continue))}.  Starting at this entry
point jumps to the location designated by \code{val}, but first assigns
\code{continue} so that execution will return to \code{print\-/result}, which
prints the value in \code{val} and then goes to the beginning of the
evaluator's read-eval-print loop.\footnote{Since a compiled procedure is an
object that the system may try to print, we also modify the system print
operation \code{user\-/print} (from \link{Section 4.1.4}) so that it will not
attempt to print the components of a compiled procedure:

\begin{smallscheme}
(define (user-print object)
  (cond ((compound-procedure? object)
         (display (list 'compound-procedure
                        (procedure-parameters object)
                        (procedure-body object)
                        '<procedure-env>)))
        ((compiled-procedure? object)
         (display '<compiled-procedure>))
        (else (display object))))
\end{smallscheme}
}

\begin{scheme}
external-entry
  (perform (op initialize-stack))
  (assign env (op get-global-environment))
  (assign continue (label print-result))
  (goto (reg val))
\end{scheme}

\noindent
Now we can use the following procedure to compile a procedure definition,
execute the compiled code, and run the read-eval-print loop so we can try the
procedure.  Because we want the compiled code to return to the location in
\code{continue} with its result in \code{val}, we compile the expression with a
target of \code{val} and a linkage of \code{return}.  In order to transform the
object code produced by the compiler into executable instructions for the
evaluator register machine, we use the procedure \code{assemble} from the
register-machine simulator (\link{Section 5.2.2}).  We then initialize the
\code{val} register to point to the list of instructions, set the \code{flag}
so that the evaluator will go to \code{external\-/entry}, and start the
evaluator.

\begin{scheme}
(define (compile-and-go expression)
  (let ((instructions
         (assemble
          (statements
           (compile expression 'val 'return))
          eceval)))
    (set! the-global-environment (setup-environment))
    (set-register-contents! eceval 'val instructions)
    (set-register-contents! eceval 'flag true)
    (start eceval)))
\end{scheme}

\noindent
If we have set up stack monitoring, as at the end of \link{Section 5.4.4}, we
can examine the stack usage of compiled code:

\begin{scheme}
(compile-and-go
 '(define (factorial n)
    (if (= n 1)
        1
        (* (factorial (- n 1)) n))))
~\textit{(total-pushes = 0 maximum-depth = 0)}~
~\textit{;;; EC-Eval value:}~
~\textit{ok}~
~\textit{;;; EC-Eval input:}~
(factorial 5)
~\textit{(total-pushes = 31 maximum-depth = 14)}~
~\textit{;;; EC-Eval value:}~
~\textit{120}~
\end{scheme}

\noindent
Compare this example with the evaluation of \code{(factorial 5)} using the
interpreted version of the same procedure, shown at the end of 
\link{Section 5.4.4}.  The interpreted version required 144 pushes and a maximum stack
depth of 28.  This illustrates the optimization that results from our
compilation strategy.

\subsubsection*{Interpretation and compilation}

With the programs in this section, we can now experiment with the alternative
execution strategies of interpretation and compilation.\footnote{We can do even
better by extending the compiler to allow compiled code to call interpreted
procedures.  See \link{Exercise 5.47}.}  An interpreter raises the machine to
the level of the user program; a compiler lowers the user program to the level
of the machine language.  We can regard the Scheme language (or any programming
language) as a coherent family of abstractions erected on the machine language.
Interpreters are good for interactive program development and debugging because
the steps of program execution are organized in terms of these abstractions,
and are therefore more intelligible to the programmer.  Compiled code can
execute faster, because the steps of program execution are organized in terms
of the machine language, and the compiler is free to make optimizations that
cut across the higher-level abstractions.\footnote{Independent of the strategy
of execution, we incur significant overhead if we insist that errors
encountered in execution of a user program be detected and signaled, rather
than being allowed to kill the system or produce wrong answers.  For example,
an out-of-bounds array reference can be detected by checking the validity of
the reference before performing it.  The overhead of checking, however, can be
many times the cost of the array reference itself, and a programmer should
weigh speed against safety in determining whether such a check is desirable.  A
good compiler should be able to produce code with such checks, should avoid
redundant checks, and should allow programmers to control the extent and type
of error checking in the compiled code.

Compilers for popular languages, such as C and C++, put hardly any
error-checking operations into running code, so as to make things run as fast
as possible.  As a result, it falls to programmers to explicitly provide error
checking.  Unfortunately, people often neglect to do this, even in critical
applications where speed is not a constraint.  Their programs lead fast and
dangerous lives.  For example, the notorious ``Worm'' that paralyzed the
Internet in 1988 exploited the \acronym{UNIX}(tm) operating system's failure to
check whether the input buffer has overflowed in the finger daemon. (See
\link{Spafford 1989}.)}

The alternatives of interpretation and compilation also lead to different
strategies for porting languages to new computers. Suppose that we wish to
implement Lisp for a new machine.  One strategy is to begin with the
explicit-control evaluator of \link{Section 5.4} and translate its instructions
to instructions for the new machine.  A different strategy is to begin with the
compiler and change the code generators so that they generate code for the new
machine.  The second strategy allows us to run any Lisp program on the new
machine by first compiling it with the compiler running on our original Lisp
system, and linking it with a compiled version of the run-time
library.\footnote{Of course, with either the interpretation or the compilation
strategy we must also implement for the new machine storage allocation, input
and output, and all the various operations that we took as ``primitive'' in our
discussion of the evaluator and compiler.  One strategy for minimizing work
here is to write as many of these operations as possible in Lisp and then
compile them for the new machine.  Ultimately, everything reduces to a small
kernel (such as garbage collection and the mechanism for applying actual
machine primitives) that is hand-coded for the new machine.} Better yet, we can
compile the compiler itself, and run this on the new machine to compile other
Lisp programs.\footnote{This strategy leads to amusing tests of correctness of
the compiler, such as checking whether the compilation of a program on the new
machine, using the compiled compiler, is identical with the compilation of the
program on the original Lisp system.  Tracking down the source of differences
is fun but often frustrating, because the results are extremely sensitive to
minuscule details.}  Or we can compile one of the interpreters of 
\link{Section 4.1} to produce an interpreter that runs on the new machine.

\begin{quote}
\heading{\phantomsection\label{Exercise 5.45}Exercise 5.45:} By comparing the stack operations
used by compiled code to the stack operations used by the evaluator for the
same computation, we can determine the extent to which the compiler optimizes
use of the stack, both in speed (reducing the total number of stack operations)
and in space (reducing the maximum stack depth).  Comparing this optimized
stack use to the performance of a special-purpose machine for the same
computation gives some indication of the quality of the compiler.

\begin{enumerate}[a.]

\item
\link{Exercise 5.27} asked you to determine, as a function of \( n \), the number
of pushes and the maximum stack depth needed by the evaluator to compute \( n! \)
using the recursive factorial procedure given above.  \link{Exercise 5.14} asked
you to do the same measurements for the special-purpose factorial machine shown
in \link{Figure 5.11}. Now perform the same analysis using the compiled
\code{factorial} procedure.

Take the ratio of the number of pushes in the compiled version to the number of
pushes in the interpreted version, and do the same for the maximum stack depth.
Since the number of operations and the stack depth used to compute \( n! \)  are
linear in \( n \), these ratios should approach constants as \( n \) becomes large.
What are these constants?  Similarly, find the ratios of the stack usage in the
special-purpose machine to the usage in the interpreted version.

Compare the ratios for special-purpose versus interpreted code to the ratios
for compiled versus interpreted code.  You should find that the special-purpose
machine does much better than the compiled code, since the hand-tailored
controller code should be much better than what is produced by our rudimentary
general-purpose compiler.

\item
Can you suggest improvements to the compiler that would help it generate code
that would come closer in performance to the hand-tailored version?

\end{enumerate}
\end{quote}

\begin{quote}
\heading{\phantomsection\label{Exercise 5.46}Exercise 5.46:} Carry out an analysis like the
one in \link{Exercise 5.45} to determine the effectiveness of compiling the
tree-recursive Fibonacci procedure

\begin{scheme}
(define (fib n)
  (if (< n 2)
      n
      (+ (fib (- n 1))
         (fib (- n 2)))))
\end{scheme}

\noindent
compared to the effectiveness of using the special-purpose Fibonacci machine of
\link{Figure 5.12}.  (For measurement of the interpreted performance, see
\link{Exercise 5.29}.)  For Fibonacci, the time resource used is not linear in
\( n; \) hence the ratios of stack operations will not approach a limiting value
that is independent of \( n \).
\end{quote}

\begin{quote}
\heading{\phantomsection\label{Exercise 5.47}Exercise 5.47:} This section described how to
modify the explicit-control evaluator so that interpreted code can call
compiled procedures.  Show how to modify the compiler so that compiled
procedures can call not only primitive procedures and compiled procedures, but
interpreted procedures as well.  This requires modifying
\code{compile\-/procedure\-/call} to handle the case of compound (interpreted)
procedures.  Be sure to handle all the same \code{target} and \code{linkage}
combinations as in \code{compile\-/proc\-/appl}.  To do the actual procedure
application, the code needs to jump to the evaluator's \code{compound\-/apply}
entry point.  This label cannot be directly referenced in object code (since
the assembler requires that all labels referenced by the code it is assembling
be defined there), so we will add a register called \code{compapp} to the
evaluator machine to hold this entry point, and add an instruction to
initialize it:

\begin{scheme}
 (assign compapp (label compound-apply))
 (branch (label external-entry)) ~\textrm{;branches if \code{flag} is set}~
read-eval-print-loop ~\( \dots \)~
\end{scheme}

To test your code, start by defining a procedure \code{f} that calls a
procedure \code{g}.  Use \code{compile\-/and\-/go} to compile the definition of
\code{f} and start the evaluator.  Now, typing at the evaluator, define
\code{g} and try to call \code{f}.
\end{quote}

\begin{quote}
\heading{\phantomsection\label{Exercise 5.48}Exercise 5.48:} The \code{compile\-/and\-/go}
interface implemented in this section is awkward, since the compiler can be
called only once (when the evaluator machine is started).  Augment the
compiler-interpreter interface by providing a \code{compile\-/and\-/run} primitive
that can be called from within the explicit-control evaluator as follows:

\begin{scheme}
~\textit{;;; EC-Eval input:}~
(compile-and-run
 '(define (factorial n)
    (if (= n 1) 1 (* (factorial (- n 1)) n))))
~\textit{;;; EC-Eval value:}~
~\textit{ok}~
~\textit{;;; EC-Eval input:}~
(factorial 5)
~\textit{;;; EC-Eval value:}~
~\textit{120}~
\end{scheme}
\end{quote}

\begin{quote}
\heading{\phantomsection\label{Exercise 5.49}Exercise 5.49:} As an alternative to using the
explicit-control evaluator's read-eval-print loop, design a register machine
that performs a read-compile-execute-print loop.  That is, the machine should
run a loop that reads an expression, compiles it, assembles and executes the
resulting code, and prints the result.  This is easy to run in our simulated
setup, since we can arrange to call the procedures \code{compile} and
\code{assemble} as ``register-machine operations.''
\end{quote}

\begin{quote}
\heading{\phantomsection\label{Exercise 5.50}Exercise 5.50:} Use the compiler to compile the
metacircular evaluator of \link{Section 4.1} and run this program using the
register-machine simulator.  (To compile more than one definition at a time,
you can package the definitions in a \code{begin}.)  The resulting interpreter
will run very slowly because of the multiple levels of interpretation, but
getting all the details to work is an instructive exercise.
\end{quote}

\begin{quote}
\heading{\phantomsection\label{Exercise 5.51}Exercise 5.51:} Develop a rudimentary
implementation of Scheme in C (or some other low-level language of your choice)
by translating the explicit-control evaluator of \link{Section 5.4} into C.  In
order to run this code you will need to also provide appropriate
storage-allocation routines and other run-time support.
\end{quote}

\begin{quote}
\heading{\phantomsection\label{Exercise 5.52}Exercise 5.52:} As a counterpoint to 
\link{Exercise 5.51}, modify the compiler so that it compiles Scheme procedures
into sequences of C instructions.  Compile the metacircular evaluator of
\link{Section 4.1} to produce a Scheme interpreter written in C.
\end{quote}

\backmatter

\chapter*{References}
\addcontentsline{toc}{chapter}{References}
\label{References}

\phantomsection \label{Abelson et al. 1992}
Abelson, Harold, Andrew Berlin, Jacob Katzenelson, William McAllister,
Guillermo Rozas, Gerald Jay Sussman, and Jack Wisdom. 1992.  The Supercomputer
Toolkit: A general framework for special-purpose computing.
\textit{International Journal of High-Speed Electronics} 3(3): 337-361.
\href{http://www.hpl.hp.com/techreports/94/HPL-94-30.html}{\code{(Onl)}}

\phantomsection \label{Allen 1978}
Allen, John.  1978.  \textit{Anatomy of Lisp}. New York: McGraw-Hill.

\phantomsection \label{ANSI 1994}
\acronym{ANSI} X3.226-1994. \textit{American National Standard for Information
Sys\-tems---Programming Language---Common Lisp}.

\phantomsection \label{Appel 1987}
Appel, Andrew W.  1987.  Garbage collection can be faster than stack
allocation.  \textit{Information Processing Letters} 25(4): 275-279.
\href{http://citeseer.ist.psu.edu/viewdoc/summary?doi=10.1.1.39.8219}{\code{(Online)}}

\phantomsection \label{Backus 1978}
Backus, John.  1978.  Can programming be liberated from the von Neumann style?
\textit{Communications of the \acronym{ACM}} 21(8): 613-641.
\href{http://www.stanford.edu/class/cs242/readings/backus.pdf}{\code{(Online)}}

\phantomsection \label{Baker (1978)}
Baker, Henry G., Jr.  1978.  List processing in real time on a serial computer.
\textit{Communications of the \acronym{ACM}} 21(4): 280-293.
\href{http://dspace.mit.edu/handle/1721.1/41976}{\code{(Online)}}

\phantomsection \label{Batali et al. 1982}
Batali, John, Neil Mayle, Howard Shrobe, Gerald Jay Sussman, and Daniel Weise.
1982.  The Scheme-81 architecture---System and chip.  In \textit{Proceedings of
the \acronym{MIT} Conference on Advanced Research in \acronym{VLSI}}, edited by
Paul Penfield, Jr. Dedham, MA: Artech House.

\phantomsection \label{Borning (1977)}
Borning, Alan.  1977.  ThingLab---An object-oriented system for building
simulations using constraints. In \textit{Proceedings of the 5th International
Joint Conference on Artificial Intelligence}.
\href{http://ijcai.org/Past\%20Proceedings/IJCAI-77-VOL1/PDF/085.pdf}{\code{(Online)}}

\phantomsection \label{Borodin and Munro (1975)}
Borodin, Alan, and Ian Munro.  1975.  \textit{The Computational Complexity of
Algebraic and Numeric Problems}. New York: American Elsevier.

\phantomsection \label{Chaitin 1975}
Chaitin, Gregory J.  1975.  Randomness and mathematical proof.
\textit{Scientific American} 232(5): 47-52.

\phantomsection \label{Church (1941)}
Church, Alonzo.  1941.  \textit{The Calculi of Lambda-Conversion}.  Princeton,
N.J.: Princeton University Press.

\phantomsection \label{Clark (1978)}
Clark, Keith L.  1978.  Negation as failure.  In \textit{Logic and Data Bases}.
New York: Plenum Press, pp. 293-322.
\href{http://www.doc.ic.ac.uk/~klc/neg.html}{\code{(Online)}}

\phantomsection \label{Clinger (1982)}
Clinger, William.  1982.  Nondeterministic call by need is neither lazy nor by
name. In \textit{Proceedings of the \acronym{ACM} Symposium on Lisp and
Functional Programming}, pp. 226-234.

\phantomsection \label{Clinger and Rees 1991}
Clinger, William, and Jonathan Rees.  1991.  Macros that work.  In
\textit{Proceedings of the 1991 \acronym{ACM} Conference on Principles of
Programming Languages}, pp. 155-162.
\href{http://mumble.net/~jar/pubs/macros_that_work.ps}{\code{(Online)}}

\phantomsection \label{Colmerauer et al. 1973}
Colmerauer A., H. Kanoui, R. Pasero, and P. Roussel.  1973.  Un syst\`eme de
communication homme-machine en fran\c{c}ais.  Technical report, Groupe
Intelligence Artificielle, Universit\'e d'Aix Marseille, Luminy.

\phantomsection \label{Cormen et al. 1990}
Cormen, Thomas, Charles Leiserson, and Ronald Rivest.  1990. \textit{Introduction
to Algorithms}. Cambridge, MA: \acronym{MIT} Press.

\phantomsection \label{Darlington et al. 1982}
Darlington, John, Peter Henderson, and David Turner.  1982.  \textit{Functional
Programming and Its Applications}. New York: Cambridge University Press.

\phantomsection \label{Dijkstra 1968a}
Dijkstra, Edsger W. 1968a.  The structure of the ``\acronym{THE}''
multiprogramming system.  \textit{Communications of the \acronym{ACM}}
11(5): 341-346.
\href{http://www.cs.utexas.edu/users/EWD/ewd01xx/EWD196.PDF}{\code{(Online)}}

\phantomsection \label{Dijkstra 1968b}
Dijkstra, Edsger W. 1968b.  Cooperating sequential processes.  In
\textit{Programming Languages}, edited by F. Genuys. New York: Academic Press,
pp.  43-112.
\href{http://www.cs.utexas.edu/users/EWD/ewd01xx/EWD123.PDF}{\code{(Online)}}

\phantomsection \label{Dinesman 1968}
Dinesman, Howard P.  1968.  \textit{Superior Mathematical Puzzles}.  New York:
Simon and Schuster.

\phantomsection \label{deKleer et al. 1977}
deKleer, Johan, Jon Doyle, Guy Steele, and Gerald J. Sussman.  1977.
\acronym{AMORD}: Explicit control of reasoning.  In \textit{Proceedings of the
\acronym{ACM} Symposium on Artificial Intelligence and Programming Languages},
pp.  116-125.
\href{http://dspace.mit.edu/handle/1721.1/5750}{\code{(Online)}}

\phantomsection \label{Doyle (1979)}
Doyle, Jon. 1979. A truth maintenance system. \textit{Artificial Intelligence}
12: 231-272.
\href{http://dspace.mit.edu/handle/1721.1/5733}{\code{(Online)}}

\phantomsection \label{Feigenbaum and Shrobe 1993}
Feigenbaum, Edward, and Howard Shrobe. 1993. The Japanese National Fifth
Generation Project: Introduction, survey, and evaluation.  In \textit{Future
Generation Computer Systems}, vol. 9, pp. 105-117.

\phantomsection \label{Feeley (1986)}
Feeley, Marc.  1986.  Deux approches \`a l'implantation du language
Scheme.  Masters thesis, Universit\'e de Montr\'eal.

\phantomsection \label{Feeley and Lapalme 1987}
Feeley, Marc and Guy Lapalme.  1987.  Using closures for code generation.
\textit{Journal of Computer Languages} 12(1): 47-66.
\href{http://citeseerx.ist.psu.edu/viewdoc/summary?doi=10.1.1.90.6978}{\code{(Online)}}

Feller, William.  1957.  \textit{An Introduction to Probability Theory and Its
Applications}, volume 1. New York: John Wiley \& Sons.

\phantomsection \label{Fenichel and Yochelson (1969)}
Fenichel, R., and J. Yochelson.  1969.  A Lisp garbage collector for virtual
memory computer systems.  \textit{Communications of the \acronym{ACM}}
12(11): 611-612.

\phantomsection \label{Floyd (1967)}
Floyd, Robert. 1967. Nondeterministic algorithms. \textit{\acronym{JACM}},
14(4): 636-644.

\phantomsection \label{Forbus and deKleer 1993}
Forbus, Kenneth D., and Johan deKleer.  1993. \textit{Building Problem
Solvers}. Cambridge, MA: \acronym{MIT} Press.

\phantomsection \label{Friedman and Wise (1976)}
Friedman, Daniel P., and David S. Wise.  1976.  \acronym{CONS} should not
evaluate its arguments. In \textit{Automata, Languages, and Programming: Third
International Colloquium}, edited by S. Michaelson and R.  Milner, pp. 257-284.
\href{https://www.cs.indiana.edu/cgi-bin/techreports/TRNNN.cgi?trnum=TR44}{\code{(Online)}}

\phantomsection \label{Friedman et al. 1992}
Friedman, Daniel P., Mitchell Wand, and Christopher T. Haynes. 1992.
\textit{Essentials of Programming Languages}.  Cambridge, MA: \acronym{MIT}
Press/ McGraw-Hill.

\phantomsection \label{Gabriel 1988}
Gabriel, Richard P. 1988.  The Why of \emph{Y}.  \textit{Lisp Pointers}
2(2): 15-25.
\href{http://www.dreamsongs.com/Files/WhyOfY.pdf}{\code{(Online)}}

Goldberg, Adele, and David Robson.  1983.  \textit{Smalltalk-80: The Language and
Its Implementation}. Reading, MA: Addison-Wesley.

\phantomsection \label{Gordon et al. 1979}
Gordon, Michael, Robin Milner, and Christopher Wadsworth.  1979.
\textit{Edinburgh LCF}. Lecture Notes in Computer Science, volume 78. New York:
Springer-Verlag.

\phantomsection \label{Gray and Reuter 1993}
Gray, Jim, and Andreas Reuter. 1993. \textit{Transaction Processing: Concepts and
Models}. San Mateo, CA: Morgan-Kaufman.

\phantomsection \label{Green 1969}
Green, Cordell.  1969.  Application of theorem proving to problem solving.  In
\textit{Proceedings of the International Joint Conference on Artificial
Intelligence}, pp. 219-240.
\href{http://citeseer.ist.psu.edu/viewdoc/summary?doi=10.1.1.81.9820}{\code{(Online)}}

\phantomsection \label{Green and Raphael (1968)}
Green, Cordell, and Bertram Raphael.  1968.  The use of theorem-proving
techniques in question-answering systems.  In \textit{Proceedings of the
\acronym{ACM} National Conference}, pp. 169-181.

\phantomsection \label{Griss 1981}
Griss, Martin L.  1981.  Portable Standard Lisp, a brief overview.  Utah
Symbolic Computation Group Operating Note 58, University of Utah.

\phantomsection \label{Guttag 1977}
Guttag, John V.  1977.  Abstract data types and the development of data
structures.  \textit{Communications of the \acronym{ACM}} 20(6): 396-404.
\href{http://www.unc.edu/~stotts/comp723/guttagADT77.pdf}{\code{(Online)}}

\phantomsection \label{Hamming 1980}
Hamming, Richard W.  1980.  \textit{Coding and Information Theory}.  Englewood
Cliffs, N.J.: Prentice-Hall.

\phantomsection \label{Hanson 1990}
Hanson, Christopher P.  1990.  Efficient stack allocation for tail-recur\-sive
languages.  In \textit{Proceedings of \acronym{ACM} Conference on Lisp and
Functional Programming}, pp. 106-118.

\phantomsection \label{Hanson 1991}
Hanson, Christopher P.  1991.  A syntactic closures macro facility.  \textit{Lisp
Pointers}, 4(3).
\href{http://groups.csail.mit.edu/mac/ftpdir/scheme-reports/synclo.ps}{\code{(Online)}}

\phantomsection \label{Hardy 1921}
Hardy, Godfrey H.  1921.  Srinivasa Ramanujan.  \textit{Proceedings of the London
Mathematical Society} XIX(2).

\phantomsection \label{Hardy and Wright 1960}
Hardy, Godfrey H., and E. M. Wright.  1960.  \textit{An Introduction to the
Theory of Numbers}.  4th edition.  New York: Oxford University Press.

\phantomsection \label{Havender (1968)}
Havender, J. 1968. Avoiding deadlocks in multi-tasking systems. \textit{IBM
Systems Journal} 7(2): 74-84.

\phantomsection \label{Hearn 1969}
Hearn, Anthony C.  1969.  Standard Lisp.  Technical report \acronym{AIM}-90,
Artificial Intelligence Project, Stanford University.
\href{http://www.softwarepreservation.org/projects/LISP/stanford/Hearn-StandardLisp-AIM-90.pdf}{\code{(Online)}}

\phantomsection \label{Henderson 1980}
Henderson, Peter. 1980.  \textit{Functional Programming: Application and
Implementation}. Englewood Cliffs, N.J.: Prentice-Hall.

\phantomsection \label{Henderson 1982}
Henderson. Peter. 1982. Functional Geometry. In \textit{Conference Record of the
1982 \acronym{ACM} Symposium on Lisp and Functional Programming}, pp. 179-187.
\href{http://pmh-systems.co.uk/phAcademic/papers/funcgeo.pdf}{\code{(Online)}}
\href{http://eprints.soton.ac.uk/257577/1/funcgeo2.pdf}{\code{(2002 version)}}

\phantomsection \label{Hewitt (1969)}
Hewitt, Carl E.  1969.  \acronym{PLANNER}: A language for proving
theorems in robots.  In \textit{Proceedings of the International Joint
Conference on Artificial Intelligence}, pp. 295-301.
\href{http://dspace.mit.edu/handle/1721.1/6171}{\code{(Online)}}

\phantomsection \label{Hewitt (1977)}
Hewitt, Carl E.  1977.  Viewing control structures as patterns of passing
messages.  \textit{Journal of Artificial Intelligence} 8(3): 323-364.
\href{http://dspace.mit.edu/handle/1721.1/6272}{\code{(Online)}}

\phantomsection \label{Hoare (1972)}
Hoare, C. A. R. 1972.  Proof of correctness of data representations.
\textit{Acta Informatica} 1(1).

\phantomsection \label{Hodges 1983}
Hodges, Andrew. 1983.  \textit{Alan Turing: The Enigma}. New York: Simon and
Schuster.

\phantomsection \label{Hofstadter 1979}
Hofstadter, Douglas R.  1979.  \textit{G\"odel, Escher, Bach: An Eternal Golden
Braid}. New York: Basic Books.

\phantomsection \label{Hughes 1990}
Hughes, R. J. M.  1990.  Why functional programming matters.  In \textit{Research
Topics in Functional Programming}, edited by David Turner.  Reading, MA:
Addison-Wesley, pp. 17-42.
\href{http://www.cs.kent.ac.uk/people/staff/dat/miranda/whyfp90.pdf}{\code{(Online)}}

\phantomsection \label{IEEE 1990}
\acronym{IEEE} Std 1178-1990.  1990.  \textit{\acronym{IEEE} Standard for the
Scheme Programming Language}.

\phantomsection \label{Ingerman et al. 1960}
Ingerman, Peter, Edgar Irons, Kirk Sattley, and Wallace Feurzeig; assisted by
M. Lind, Herbert Kanner, and Robert Floyd.  1960.  \acronym{THUNKS}: A way of
compiling procedure statements, with some comments on procedure declarations.
Unpublished manuscript.  (Also, private communication from Wallace Feurzeig.)

\phantomsection \label{Kaldewaij 1990}
Kaldewaij, Anne. 1990.  \textit{Programming: The Derivation of Algorithms}. New
York: Prentice-Hall.

\phantomsection \label{Knuth (1973)}
Knuth, Donald E.  1973.  \textit{Fundamental Algorithms}. Volume 1 of \textit{The
Art of Computer Programming}.  2nd edition. Reading, MA: Addison-Wesley.

\phantomsection \label{Knuth 1981}
Knuth, Donald E.  1981.  \textit{Seminumerical Algorithms}. Volume 2 of \textit{The
Art of Computer Programming}.  2nd edition. Reading, MA: Addison-Wesley.

\phantomsection \label{Kohlbecker 1986}
Kohlbecker, Eugene Edmund, Jr. 1986.  Syntactic extensions in the programming
language Lisp.  Ph.D. thesis, Indiana University.
\href{http://www.ccs.neu.edu/scheme/pubs/dissertation-kohlbecker.pdf}{\code{(Online)}}

\phantomsection \label{Konopasek and Jayaraman 1984}
Konopasek, Milos, and Sundaresan Jayaraman.  1984.  \textit{The TK!Solver Book: A
Guide to Problem-Solving in Science, Engineering, Business, and
Education}. Berkeley, CA: Osborne/McGraw-Hill.

\phantomsection \label{Kowalski (1973; 1979)}
Kowalski, Robert.  1973.  Predicate logic as a programming language.  Technical
report 70, Department of Computational Logic, School of Artificial
Intelligence, University of Edinburgh.
\href{http://www.doc.ic.ac.uk/~rak/papers/IFIP\%2074.pdf}{\code{(Online)}}

Kowalski, Robert.  1979.  \textit{Logic for Problem Solving}. New York:
North-Holland.

\phantomsection \label{Lamport (1978)}
Lamport, Leslie. 1978.  Time, clocks, and the ordering of events in a
distributed system.  \textit{Communications of the \acronym{ACM}} 21(7): 558-565.
\href{http://www.stanford.edu/class/cs240/readings/lamport.pdf}{\code{(Online)}}

\phantomsection \label{Lampson et al. 1981}
Lampson, Butler, J. J. Horning, R.  London, J. G. Mitchell, and G. K.  Popek.
1981.  Report on the programming language Euclid.  Technical report, Computer
Systems Research Group, University of Toronto.
\href{http://www.bitsavers.org/pdf/xerox/parc/techReports/CSL-81-12_Report_On_The_Programming_Language_Euclid.pdf}{\code{(Online)}}

\phantomsection \label{Landin (1965)}
Landin, Peter.  1965.  A correspondence between Algol 60 and Church's lambda
notation: Part I.  \textit{Communications of the \acronym{ACM}} 8(2): 89-101.

\phantomsection \label{Lieberman and Hewitt 1983}
Lieberman, Henry, and Carl E. Hewitt. 1983. A real-time garbage collector based
on the lifetimes of objects. \textit{Communications of the \acronym{ACM}}
26(6): 419-429.
\href{http://dspace.mit.edu/handle/1721.1/6335}{\code{(Online)}}

\phantomsection \label{Liskov and Zilles (1975)}
Liskov, Barbara H., and Stephen N. Zilles.  1975.  Specification techniques for
data abstractions.  \textit{\acronym{IEEE} Transactions on Software Engineering}
1(1): 7-19.
\href{http://csg.csail.mit.edu/CSGArchives/memos/Memo-117.pdf}{\code{(Online)}}

\phantomsection \label{McAllester (1978; 1980)}
McAllester, David Allen.  1978.  A three-valued truth-maintenance system.  Memo
473, \acronym{MIT} Artificial Intelligence Laboratory.
\href{http://dspace.mit.edu/handle/1721.1/6296}{\code{(Online)}}

McAllester, David Allen.  1980.  An outlook on truth maintenance.  Memo 551,
\acronym{MIT} Artificial Intelligence Laboratory.
\href{http://dspace.mit.edu/handle/1721.1/6327}{\code{(Online)}}

\phantomsection \label{McCarthy 1960}
McCarthy, John.  1960.  Recursive functions of symbolic expressions and their
computation by machine.  \textit{Communications of the \acronym{ACM}}
3(4): 184-195.
\href{http://innovation.it.uts.edu.au/projectjmc/articles/recursive.html}{\code{(Online)}}

\phantomsection \label{McCarthy 1963}
McCarthy, John.  1963.  A basis for a mathematical theory of computation.  In
\textit{Computer Programming and Formal Systems}, edited by P. Braffort and
D. Hirschberg.  North-Holland.
\href{http://innovation.it.uts.edu.au/projectjmc/articles/basis.html}{\code{(Online)}}

\phantomsection \label{McCarthy 1978}
McCarthy, John.  1978.  The history of Lisp.  In \textit{Proceedings of the
\acronym{ACM} \acronym{SIGPLAN} Conference on the History of Programming
Languages}.
\href{http://innovation.it.uts.edu.au/projectjmc/articles/lisp.html}{\code{(Online)}}

\phantomsection \label{McCarthy et al. 1965}
McCarthy, John, P. W. Abrahams, D. J. Edwards, T. P. Hart, and M. I.  Levin.
1965.  \textit{Lisp 1.5 Programmer's Manual}.  2nd edition.  Cambridge, MA:
\acronym{MIT} Press.
\href{http://www.softwarepreservation.org/projects/LISP/book/LISP\%201.5\%20Programmers\%20Manual.pdf/view}{\code{(Online)}}

\phantomsection \label{McDermott and Sussman (1972)}
McDermott, Drew, and Gerald Jay Sussman.  1972. Conniver reference manual.
Memo 259, \acronym{MIT} Artificial Intelligence Laboratory.
\href{http://dspace.mit.edu/handle/1721.1/6203}{\code{(Online)}}

\phantomsection \label{Miller 1976}
Miller, Gary L.  1976.  Riemann's Hypothesis and tests for primality.
\textit{Journal of Computer and System Sciences} 13(3): 300-317.
\href{http://www.cs.cmu.edu/~glmiller/Publications/b2hd-Mi76.html}{\code{(Online)}}

\phantomsection \label{Miller and Rozas 1994}
Miller, James S., and Guillermo J. Rozas. 1994.  Garbage collection is fast,
but a stack is faster.  Memo 1462, \acronym{MIT} Artificial Intelligence
Laboratory.
\href{http://dspace.mit.edu/handle/1721.1/6622}{\code{(Online)}}

\phantomsection \label{Moon 1978}
Moon, David.  1978.  MacLisp reference manual, Version 0.  Technical report,
\acronym{MIT} Laboratory for Computer Science.
\href{http://www.softwarepreservation.org/projects/LISP/MIT/Moon-MACLISP_Reference_Manual-Apr_08_1974.pdf/view}{\code{(Online)}}

\phantomsection \label{Moon and Weinreb 1981}
Moon, David, and Daniel Weinreb.  1981.  Lisp machine manual.  Technical
report, \acronym{MIT} Artificial Intelligence Laboratory.
\href{http://www.unlambda.com/lmman/index.html}{\code{(Online)}}

\phantomsection \label{Morris et al. 1980}
Morris, J. H., Eric Schmidt, and Philip Wadler.  1980.  Experience with an
applicative string processing language.  In \textit{Proceedings of the 7th Annual
\acronym{ACM} \acronym{SIGACT}/\acronym{SIGPLAN} Symposium on the Principles of
Programming Languages}.

\phantomsection \label{Phillips 1934}
Phillips, Hubert.  1934. \textit{The Sphinx Problem Book}.  London: Faber and
Faber.

\phantomsection \label{Pitman 1983}
Pitman, Kent. 1983. The revised MacLisp Manual (Saturday evening edition).
Technical report 295, \acronym{MIT} Laboratory for Computer Science.
\href{http://maclisp.info/pitmanual}{\code{(Online)}}

\phantomsection \label{Rabin 1980}
Rabin, Michael O. 1980. Probabilistic algorithm for testing primality.
\textit{Journal of Number Theory} 12: 128-138.

\phantomsection \label{Raymond 1993}
Raymond, Eric.  1993. \textit{The New Hacker's Dictionary}. 2nd edition.
Cambridge, MA: \acronym{MIT} Press.
\href{http://www.outpost9.com/reference/jargon/jargon_toc.html}{\code{(Online)}}

Raynal, Michel. 1986. \textit{Algorithms for Mutual Exclusion}.  Cambridge, MA:
\acronym{MIT} Press.

\phantomsection \label{Rees and Adams 1982}
Rees, Jonathan A., and Norman I. Adams IV. 1982.  T: A dialect of Lisp or,
lambda: The ultimate software tool.  In \textit{Conference Record of the 1982
\acronym{ACM} Symposium on Lisp and Functional Programming}, pp.  114-122.
\href{http://people.csail.mit.edu/riastradh/t/adams82t.pdf}{\code{(Online)}}

Rees, Jonathan, and William Clinger (eds). 1991.  The \( \rm revised^4 \) report on the
algorithmic language Scheme.  \textit{Lisp Pointers}, 4(3).
\href{http://people.csail.mit.edu/jaffer/r4rs_toc.html}{\code{(Online)}}

\phantomsection \label{Rivest et al. (1977)}
Rivest, Ronald, Adi Shamir, and Leonard Adleman.  1977.  A method for obtaining
digital signatures and public-key cryptosystems. Technical memo LCS/TM82,
\acronym{MIT} Laboratory for Computer Science.
\href{http://people.csail.mit.edu/rivest/Rsapaper.pdf}{\code{(Online)}}

\phantomsection \label{Robinson 1965}
Robinson, J. A. 1965.  A machine-oriented logic based on the resolution
principle.  \textit{Journal of the \acronym{ACM}} 12(1): 23.

\phantomsection \label{Robinson 1983}
Robinson, J. A. 1983.  Logic programming---Past, present, and future.
\textit{New Generation Computing} 1: 107-124.

\phantomsection \label{Spafford 1989}
Spafford, Eugene H.  1989.  The Internet Worm: Crisis and aftermath.
\textit{Communications of the \acronym{ACM}} 32(6): 678-688.
\href{http://citeseerx.ist.psu.edu/viewdoc/download?doi=10.1.1.123.8503&rep=rep1&type=pdf}{\code{(Online)}}

\phantomsection \label{Steele 1977}
Steele, Guy Lewis, Jr.  1977.  Debunking the ``expensive procedure call'' myth.
In \textit{Proceedings of the National Conference of the \acronym{ACM}},
pp. 153-62.
\href{http://dspace.mit.edu/handle/1721.1/5753}{\code{(Online)}}

\phantomsection \label{Steele 1982}
Steele, Guy Lewis, Jr.  1982.  An overview of Common Lisp.  In
\textit{Proceedings of the \acronym{ACM} Symposium on Lisp and Functional
Programming}, pp. 98-107.

\phantomsection \label{Steele 1990}
Steele, Guy Lewis, Jr.  1990.  \textit{Common Lisp: The Language}. 2nd edition.
Digital Press.
\href{http://www.cs.cmu.edu/Groups/AI/html/cltl/cltl2.html}{\code{(Online)}}

\phantomsection \label{Steele and Sussman 1975}
Steele, Guy Lewis, Jr., and Gerald Jay Sussman.  1975.  Scheme: An interpreter
for the extended lambda calculus.  Memo 349, \acronym{MIT} Artificial
Intelligence Laboratory.
\href{http://dspace.mit.edu/handle/1721.1/5794}{\code{(Online)}}

\phantomsection \label{Steele et al. 1983}
Steele, Guy Lewis, Jr., Donald R. Woods, Raphael A. Finkel, Mark R.  Crispin,
Richard M. Stallman, and Geoffrey S. Goodfellow.  1983.  \textit{The Hacker's
Dictionary}. New York: Harper \& Row.
\href{http://www.dourish.com/goodies/jargon.html}{\code{(Online)}}

\phantomsection \label{Stoy 1977}
Stoy, Joseph E.  1977.  \textit{Denotational Semantics}. Cambridge, MA:
\acronym{MIT} Press.

\phantomsection \label{Sussman and Stallman 1975}
Sussman, Gerald Jay, and Richard M. Stallman.  1975.  Heuristic techniques in
computer-aided circuit analysis.  \textit{\acronym{IEEE} Transactions on Circuits
and Systems} CAS-22(11): 857-865.
\href{http://dspace.mit.edu/handle/1721.1/5803}{\code{(Online)}}

\phantomsection \label{Sussman and Steele 1980}
Sussman, Gerald Jay, and Guy Lewis Steele Jr.  1980.  Constraints---A language
for expressing almost-hierachical descriptions.  \textit{AI Journal} 14: 1-39.
\href{http://dspace.mit.edu/handle/1721.1/6312}{\code{(Online)}}

\phantomsection \label{Sussman and Wisdom 1992}
Sussman, Gerald Jay, and Jack Wisdom.  1992. Chaotic evolution of the solar
system.  \textit{Science} 257: 256-262.
\href{http://groups.csail.mit.edu/mac/users/wisdom/ss-chaos.pdf}{\code{(Online)}}

\phantomsection \label{Sussman et al. (1971)}
Sussman, Gerald Jay, Terry Winograd, and Eugene Charniak.  1971.  Microplanner
reference manual.  Memo 203A, \acronym{MIT} Artificial Intelligence Laboratory.
\href{http://dspace.mit.edu/handle/1721.1/6184}{\code{(Online)}}

\phantomsection \label{Sutherland (1963)}
Sutherland, Ivan E.  1963.  \acronym{SKETCHPAD}: A man-machine graphical
communication system.  Technical report 296, \acronym{MIT} Lincoln Laboratory.
\href{http://citeseer.ist.psu.edu/viewdoc/summary?doi=10.1.1.10.4290}{\code{(Onl.)}}

\phantomsection \label{Teitelman 1974}
Teitelman, Warren.  1974.  Interlisp reference manual.  Technical report, Xerox
Palo Alto Research Center.

\phantomsection \label{Thatcher et al. 1978}
Thatcher, James W., Eric G. Wagner, and Jesse B. Wright. 1978.  Data type
specification: Parameterization and the power of specification techniques. In
\textit{Conference Record of the Tenth Annual \acronym{ACM} Symposium on Theory
of Computing}, pp. 119-132.

\phantomsection \label{Turner 1981}
Turner, David.  1981.  The future of applicative languages.  In
\textit{Proceedings of the 3rd European Conference on Informatics}, Lecture Notes
in Computer Science, volume 123. New York: Springer-Verlag, pp.  334-348.

\phantomsection \label{Wand 1980}
Wand, Mitchell.  1980.  Continuation-based program transformation strategies.
\textit{Journal of the \acronym{ACM}} 27(1): 164-180.
\href{http://citeseerx.ist.psu.edu/viewdoc/summary?doi=10.1.1.83.8567}{\code{(Online)}}

\phantomsection \label{Waters (1979)}
Waters, Richard C.  1979.  A method for analyzing loop programs.
\textit{\acronym{IEEE} Transactions on Software Engineering} 5(3): 237-247.

Winograd, Terry.  1971.  Procedures as a representation for data in a computer
program for understanding natural language.  Technical report AI TR-17,
\acronym{MIT} Artificial Intelligence Laboratory.
\href{http://dspace.mit.edu/handle/1721.1/7095}{\code{(Online)}}

\phantomsection \label{Winston 1992}
Winston, Patrick. 1992. \textit{Artificial Intelligence}.  3rd edition.  Reading,
MA: Addison-Wesley.

\phantomsection \label{Zabih et al. 1987}
Zabih, Ramin, David McAllester, and David Chapman.  1987.  Non-deterministic
Lisp with dependency-directed backtracking.  \textit{\acronym{AAAI}-87},
pp. 59-64.
\href{http://www.aaai.org/Papers/AAAI/1987/AAAI87-011.pdf}{\code{(Online)}}

\phantomsection \label{Zippel (1979)}
Zippel, Richard.  1979.  Probabilistic algorithms for sparse polynomials.
Ph.D. dissertation, Department of Electrical Engineering and Computer Science,
\acronym{MIT}.

\phantomsection \label{Zippel 1993}
Zippel, Richard.  1993.  \textit{Effective Polynomial Computation}.  Boston, MA:
Kluwer Academic Publishers.

\chapter*{List of Exercises}
\addcontentsline{toc}{chapter}{List of Exercises}
\label{List of Exercises}

% 
\subsubsection*{第1章} 

\begin{tabular}{llllllllll}
\hyperref[練習問題 1.1]{1.1} &
\hyperref[練習問題 1.2]{1.2} &
\hyperref[練習問題 1.3]{1.3} &
\hyperref[練習問題 1.4]{1.4} &
\hyperref[練習問題 1.5]{1.5} &
\hyperref[練習問題 1.6]{1.6} &
\hyperref[練習問題 1.7]{1.7} &
\hyperref[練習問題 1.8]{1.8} &
\hyperref[練習問題 1.9]{1.9} &
\hyperref[練習問題 1.10]{1.10}
\\ 
\hyperref[練習問題 1.11]{1.11} &
\hyperref[練習問題 1.12]{1.12} &
\hyperref[練習問題 1.13]{1.13} &
\hyperref[練習問題 1.14]{1.14} &
\hyperref[練習問題 1.15]{1.15} &
\hyperref[練習問題 1.16]{1.16} &
\hyperref[練習問題 1.17]{1.17} &
\hyperref[練習問題 1.18]{1.18} &
\hyperref[練習問題 1.19]{1.19} &
\hyperref[練習問題 1.20]{1.20}
\\ 
\hyperref[練習問題 1.21]{1.21} &
\hyperref[練習問題 1.22]{1.22} &
\hyperref[練習問題 1.23]{1.23} &
\hyperref[練習問題 1.24]{1.24} &
\hyperref[練習問題 1.25]{1.25} &
\hyperref[練習問題 1.26]{1.26} &
\hyperref[練習問題 1.27]{1.27} &
\hyperref[練習問題 1.28]{1.28} &
\hyperref[練習問題 1.29]{1.29} &
\hyperref[練習問題 1.30]{1.30}
\\ 
\hyperref[練習問題 1.31]{1.31} &
\hyperref[練習問題 1.32]{1.32} &
\hyperref[練習問題 1.33]{1.33} &
\hyperref[練習問題 1.34]{1.34} &
\hyperref[練習問題 1.35]{1.35} &
\hyperref[練習問題 1.36]{1.36} &
\hyperref[練習問題 1.37]{1.37} &
\hyperref[練習問題 1.38]{1.38} &
\hyperref[練習問題 1.39]{1.39} &
\hyperref[練習問題 1.40]{1.40}
\\ 
\hyperref[練習問題 1.41]{1.41} &
\hyperref[練習問題 1.42]{1.42} &
\hyperref[練習問題 1.43]{1.43} &
\hyperref[練習問題 1.44]{1.44} &
\hyperref[練習問題 1.45]{1.45} &
\hyperref[練習問題 1.46]{1.46} &
\end{tabular} 

\subsubsection*{第2章} 

\begin{tabular}{llllllllll}
\hyperref[練習問題 2.1]{2.1} &
\hyperref[練習問題 2.2]{2.2} &
\hyperref[練習問題 2.3]{2.3} &
\hyperref[練習問題 2.4]{2.4} &
\hyperref[練習問題 2.5]{2.5} &
\hyperref[練習問題 2.6]{2.6} &
\hyperref[練習問題 2.7]{2.7} &
\hyperref[練習問題 2.8]{2.8} &
\hyperref[練習問題 2.9]{2.9} &
\hyperref[練習問題 2.10]{2.10}
\\ 
\hyperref[練習問題 2.11]{2.11} &
\hyperref[練習問題 2.12]{2.12} &
\hyperref[練習問題 2.13]{2.13} &
\hyperref[練習問題 2.14]{2.14} &
\hyperref[練習問題 2.15]{2.15} &
\hyperref[練習問題 2.16]{2.16} &
\hyperref[練習問題 2.17]{2.17} &
\hyperref[練習問題 2.18]{2.18} &
\hyperref[練習問題 2.19]{2.19} &
\hyperref[練習問題 2.20]{2.20}
\\ 
\hyperref[練習問題 2.21]{2.21} &
\hyperref[練習問題 2.22]{2.22} &
\hyperref[練習問題 2.23]{2.23} &
\hyperref[練習問題 2.24]{2.24} &
\hyperref[練習問題 2.25]{2.25} &
\hyperref[練習問題 2.26]{2.26} &
\hyperref[練習問題 2.27]{2.27} &
\hyperref[練習問題 2.28]{2.28} &
\hyperref[練習問題 2.29]{2.29} &
\hyperref[練習問題 2.30]{2.30}
\\ 
\hyperref[練習問題 2.31]{2.31} &
\hyperref[練習問題 2.32]{2.32} &
\hyperref[練習問題 2.33]{2.33} &
\hyperref[練習問題 2.34]{2.34} &
\hyperref[練習問題 2.35]{2.35} &
\hyperref[練習問題 2.36]{2.36} &
\hyperref[練習問題 2.37]{2.37} &
\hyperref[練習問題 2.38]{2.38} &
\hyperref[練習問題 2.39]{2.39} &
\hyperref[練習問題 2.40]{2.40}
\\ 
\hyperref[練習問題 2.41]{2.41} &
\hyperref[練習問題 2.42]{2.42} &
\hyperref[練習問題 2.43]{2.43} &
\hyperref[練習問題 2.44]{2.44} &
\hyperref[練習問題 2.45]{2.45} &
\hyperref[練習問題 2.46]{2.46} &
\hyperref[練習問題 2.47]{2.47} &
\hyperref[練習問題 2.48]{2.48} &
\hyperref[練習問題 2.49]{2.49} &
\hyperref[練習問題 2.50]{2.50}
\\ 
\hyperref[練習問題 2.51]{2.51} &
\hyperref[練習問題 2.52]{2.52} &
\hyperref[練習問題 2.53]{2.53} &
\hyperref[練習問題 2.54]{2.54} &
\hyperref[練習問題 2.55]{2.55} &
\hyperref[練習問題 2.56]{2.56} &
\hyperref[練習問題 2.57]{2.57} &
\hyperref[練習問題 2.58]{2.58} &
\hyperref[練習問題 2.59]{2.59} &
\hyperref[練習問題 2.60]{2.60}
\\ 
\hyperref[練習問題 2.61]{2.61} &
\hyperref[練習問題 2.62]{2.62} &
\hyperref[練習問題 2.63]{2.63} &
\hyperref[練習問題 2.64]{2.64} &
\hyperref[練習問題 2.65]{2.65} &
\hyperref[練習問題 2.66]{2.66} &
\hyperref[練習問題 2.67]{2.67} &
\hyperref[練習問題 2.68]{2.68} &
\hyperref[練習問題 2.69]{2.69} &
\hyperref[練習問題 2.70]{2.70}
\\ 
\hyperref[練習問題 2.71]{2.71} &
\hyperref[練習問題 2.72]{2.72} &
\hyperref[練習問題 2.73]{2.73} &
\hyperref[練習問題 2.74]{2.74} &
\hyperref[練習問題 2.75]{2.75} &
\hyperref[練習問題 2.76]{2.76} &
\hyperref[練習問題 2.77]{2.77} &
\hyperref[練習問題 2.78]{2.78} &
\hyperref[練習問題 2.79]{2.79} &
\hyperref[練習問題 2.80]{2.80}
\\ 
\hyperref[練習問題 2.81]{2.81} &
\hyperref[練習問題 2.82]{2.82} &
\hyperref[練習問題 2.83]{2.83} &
\hyperref[練習問題 2.84]{2.84} &
\hyperref[練習問題 2.85]{2.85} &
\hyperref[練習問題 2.86]{2.86} &
\hyperref[練習問題 2.87]{2.87} &
\hyperref[練習問題 2.88]{2.88} &
\hyperref[練習問題 2.89]{2.89} &
\hyperref[練習問題 2.90]{2.90}
\\ 
\hyperref[練習問題 2.91]{2.91} &
\hyperref[練習問題 2.92]{2.92} &
\hyperref[練習問題 2.93]{2.93} &
\hyperref[練習問題 2.94]{2.94} &
\hyperref[練習問題 2.95]{2.95} &
\hyperref[練習問題 2.96]{2.96} &
\hyperref[練習問題 2.97]{2.97} &
\end{tabular} 

\subsubsection*{第3章} 

\begin{tabular}{llllllllll}
\hyperref[練習問題 3.1]{3.1} &
\hyperref[練習問題 3.2]{3.2} &
\hyperref[練習問題 3.3]{3.3} &
\hyperref[練習問題 3.4]{3.4} &
\hyperref[練習問題 3.5]{3.5} &
\hyperref[練習問題 3.6]{3.6} &
\hyperref[練習問題 3.7]{3.7} &
\hyperref[練習問題 3.8]{3.8} &
\hyperref[練習問題 3.9]{3.9} &
\hyperref[練習問題 3.10]{3.10}
\\ 
\hyperref[練習問題 3.11]{3.11} &
\hyperref[練習問題 3.12]{3.12} &
\hyperref[練習問題 3.13]{3.13} &
\hyperref[練習問題 3.14]{3.14} &
\hyperref[練習問題 3.15]{3.15} &
\hyperref[練習問題 3.16]{3.16} &
\hyperref[練習問題 3.17]{3.17} &
\hyperref[練習問題 3.18]{3.18} &
\hyperref[練習問題 3.19]{3.19} &
\hyperref[練習問題 3.20]{3.20}
\\ 
\hyperref[練習問題 3.21]{3.21} &
\hyperref[練習問題 3.22]{3.22} &
\hyperref[練習問題 3.23]{3.23} &
\hyperref[練習問題 3.24]{3.24} &
\hyperref[練習問題 3.25]{3.25} &
\hyperref[練習問題 3.26]{3.26} &
\hyperref[練習問題 3.27]{3.27} &
\hyperref[練習問題 3.28]{3.28} &
\hyperref[練習問題 3.29]{3.29} &
\hyperref[練習問題 3.30]{3.30}
\\ 
\hyperref[練習問題 3.31]{3.31} &
\hyperref[練習問題 3.32]{3.32} &
\hyperref[練習問題 3.33]{3.33} &
\hyperref[練習問題 3.34]{3.34} &
\hyperref[練習問題 3.35]{3.35} &
\hyperref[練習問題 3.36]{3.36} &
\hyperref[練習問題 3.37]{3.37} &
\hyperref[練習問題 3.38]{3.38} &
\hyperref[練習問題 3.39]{3.39} &
\hyperref[練習問題 3.40]{3.40}
\\ 
\hyperref[練習問題 3.41]{3.41} &
\hyperref[練習問題 3.42]{3.42} &
\hyperref[練習問題 3.43]{3.43} &
\hyperref[練習問題 3.44]{3.44} &
\hyperref[練習問題 3.45]{3.45} &
\hyperref[練習問題 3.46]{3.46} &
\hyperref[練習問題 3.47]{3.47} &
\hyperref[練習問題 3.48]{3.48} &
\hyperref[練習問題 3.49]{3.49} &
\hyperref[練習問題 3.50]{3.50}
\\ 
\hyperref[練習問題 3.51]{3.51} &
\hyperref[練習問題 3.52]{3.52} &
\hyperref[練習問題 3.53]{3.53} &
\hyperref[練習問題 3.54]{3.54} &
\hyperref[練習問題 3.55]{3.55} &
\hyperref[練習問題 3.56]{3.56} &
\hyperref[練習問題 3.57]{3.57} &
\hyperref[練習問題 3.58]{3.58} &
\hyperref[練習問題 3.59]{3.59} &
\hyperref[練習問題 3.60]{3.60}
\\ 
\hyperref[練習問題 3.61]{3.61} &
\hyperref[練習問題 3.62]{3.62} &
\hyperref[練習問題 3.63]{3.63} &
\hyperref[練習問題 3.64]{3.64} &
\hyperref[練習問題 3.65]{3.65} &
\hyperref[練習問題 3.66]{3.66} &
\hyperref[練習問題 3.67]{3.67} &
\hyperref[練習問題 3.68]{3.68} &
\hyperref[練習問題 3.69]{3.69} &
\hyperref[練習問題 3.70]{3.70}
\\ 
\hyperref[練習問題 3.71]{3.71} &
\hyperref[練習問題 3.72]{3.72} &
\hyperref[練習問題 3.73]{3.73} &
\hyperref[練習問題 3.74]{3.74} &
\hyperref[練習問題 3.75]{3.75} &
\hyperref[練習問題 3.76]{3.76} &
\hyperref[練習問題 3.77]{3.77} &
\hyperref[練習問題 3.78]{3.78} &
\hyperref[練習問題 3.79]{3.79} &
\hyperref[練習問題 3.80]{3.80}
\\ 
\hyperref[練習問題 3.81]{3.81} &
\hyperref[練習問題 3.82]{3.82} &
\end{tabular} 

\subsubsection*{第4章} 

\begin{tabular}{llllllllll}
\hyperref[練習問題 4.1]{4.1} &
\hyperref[練習問題 4.2]{4.2} &
\hyperref[練習問題 4.3]{4.3} &
\hyperref[練習問題 4.4]{4.4} &
\hyperref[練習問題 4.5]{4.5} &
\hyperref[練習問題 4.6]{4.6} &
\hyperref[練習問題 4.7]{4.7} &
\hyperref[練習問題 4.8]{4.8} &
\hyperref[練習問題 4.9]{4.9} &
\hyperref[練習問題 4.10]{4.10}
\\ 
\hyperref[練習問題 4.11]{4.11} &
\hyperref[練習問題 4.12]{4.12} &
\hyperref[練習問題 4.13]{4.13} &
\hyperref[練習問題 4.14]{4.14} &
\hyperref[練習問題 4.15]{4.15} &
\hyperref[練習問題 4.16]{4.16} &
\hyperref[練習問題 4.17]{4.17} &
\hyperref[練習問題 4.18]{4.18} &
\hyperref[練習問題 4.19]{4.19} &
\hyperref[練習問題 4.20]{4.20}
\\ 
\hyperref[練習問題 4.21]{4.21} &
\hyperref[練習問題 4.22]{4.22} &
\hyperref[練習問題 4.23]{4.23} &
\hyperref[練習問題 4.24]{4.24} &
\hyperref[練習問題 4.25]{4.25} &
\hyperref[練習問題 4.26]{4.26} &
\hyperref[練習問題 4.27]{4.27} &
\hyperref[練習問題 4.28]{4.28} &
\hyperref[練習問題 4.29]{4.29} &
\hyperref[練習問題 4.30]{4.30}
\\ 
\hyperref[練習問題 4.31]{4.31} &
\hyperref[練習問題 4.32]{4.32} &
\hyperref[練習問題 4.33]{4.33} &
\hyperref[練習問題 4.34]{4.34} &
\hyperref[練習問題 4.35]{4.35} &
\hyperref[練習問題 4.36]{4.36} &
\hyperref[練習問題 4.37]{4.37} &
\hyperref[練習問題 4.38]{4.38} &
\hyperref[練習問題 4.39]{4.39} &
\hyperref[練習問題 4.40]{4.40}
\\ 
\hyperref[練習問題 4.41]{4.41} &
\hyperref[練習問題 4.42]{4.42} &
\hyperref[練習問題 4.43]{4.43} &
\hyperref[練習問題 4.44]{4.44} &
\hyperref[練習問題 4.45]{4.45} &
\hyperref[練習問題 4.46]{4.46} &
\hyperref[練習問題 4.47]{4.47} &
\hyperref[練習問題 4.48]{4.48} &
\hyperref[練習問題 4.49]{4.49} &
\hyperref[練習問題 4.50]{4.50}
\\ 
\hyperref[練習問題 4.51]{4.51} &
\hyperref[練習問題 4.52]{4.52} &
\hyperref[練習問題 4.53]{4.53} &
\hyperref[練習問題 4.54]{4.54} &
\hyperref[練習問題 4.55]{4.55} &
\hyperref[練習問題 4.56]{4.56} &
\hyperref[練習問題 4.57]{4.57} &
\hyperref[練習問題 4.58]{4.58} &
\hyperref[練習問題 4.59]{4.59} &
\hyperref[練習問題 4.60]{4.60}
\\ 
\hyperref[練習問題 4.61]{4.61} &
\hyperref[練習問題 4.62]{4.62} &
\hyperref[練習問題 4.63]{4.63} &
\hyperref[練習問題 4.64]{4.64} &
\hyperref[練習問題 4.65]{4.65} &
\hyperref[練習問題 4.66]{4.66} &
\hyperref[練習問題 4.67]{4.67} &
\hyperref[練習問題 4.68]{4.68} &
\hyperref[練習問題 4.69]{4.69} &
\hyperref[練習問題 4.70]{4.70}
\\ 
\hyperref[練習問題 4.71]{4.71} &
\hyperref[練習問題 4.72]{4.72} &
\hyperref[練習問題 4.73]{4.73} &
\hyperref[練習問題 4.74]{4.74} &
\hyperref[練習問題 4.75]{4.75} &
\hyperref[練習問題 4.76]{4.76} &
\hyperref[練習問題 4.77]{4.77} &
\hyperref[練習問題 4.78]{4.78} &
\hyperref[練習問題 4.79]{4.79} &
\end{tabular} 

\subsubsection*{第5章} 

\begin{tabular}{llllllllll}
\hyperref[練習問題 5.1]{5.1} &
\hyperref[練習問題 5.2]{5.2} &
\hyperref[練習問題 5.3]{5.3} &
\hyperref[練習問題 5.4]{5.4} &
\hyperref[練習問題 5.5]{5.5} &
\hyperref[練習問題 5.6]{5.6} &
\hyperref[練習問題 5.7]{5.7} &
\hyperref[練習問題 5.8]{5.8} &
\hyperref[練習問題 5.9]{5.9} &
\hyperref[練習問題 5.10]{5.10}
\\ 
\hyperref[練習問題 5.11]{5.11} &
\hyperref[練習問題 5.12]{5.12} &
\hyperref[練習問題 5.13]{5.13} &
\hyperref[練習問題 5.14]{5.14} &
\hyperref[練習問題 5.15]{5.15} &
\hyperref[練習問題 5.16]{5.16} &
\hyperref[練習問題 5.17]{5.17} &
\hyperref[練習問題 5.18]{5.18} &
\hyperref[練習問題 5.19]{5.19} &
\hyperref[練習問題 5.20]{5.20}
\\ 
\hyperref[練習問題 5.21]{5.21} &
\hyperref[練習問題 5.22]{5.22} &
\hyperref[練習問題 5.23]{5.23} &
\hyperref[練習問題 5.24]{5.24} &
\hyperref[練習問題 5.25]{5.25} &
\hyperref[練習問題 5.26]{5.26} &
\hyperref[練習問題 5.27]{5.27} &
\hyperref[練習問題 5.28]{5.28} &
\hyperref[練習問題 5.29]{5.29} &
\hyperref[練習問題 5.30]{5.30}
\\ 
\hyperref[練習問題 5.31]{5.31} &
\hyperref[練習問題 5.32]{5.32} &
\hyperref[練習問題 5.33]{5.33} &
\hyperref[練習問題 5.34]{5.34} &
\hyperref[練習問題 5.35]{5.35} &
\hyperref[練習問題 5.36]{5.36} &
\hyperref[練習問題 5.37]{5.37} &
\hyperref[練習問題 5.38]{5.38} &
\hyperref[練習問題 5.39]{5.39} &
\hyperref[練習問題 5.40]{5.40}
\\ 
\hyperref[練習問題 5.41]{5.41} &
\hyperref[練習問題 5.42]{5.42} &
\hyperref[練習問題 5.43]{5.43} &
\hyperref[練習問題 5.44]{5.44} &
\hyperref[練習問題 5.45]{5.45} &
\hyperref[練習問題 5.46]{5.46} &
\hyperref[練習問題 5.47]{5.47} &
\hyperref[練習問題 5.48]{5.48} &
\hyperref[練習問題 5.49]{5.49} &
\hyperref[練習問題 5.50]{5.50}
\\ 
\hyperref[練習問題 5.51]{5.51} &
\hyperref[練習問題 5.52]{5.52} &
\end{tabular} 



\chapter*{List of Figures}
\addcontentsline{toc}{chapter}{List of Figures}
\label{List of Figures}

% 
\subsubsection*{第1章} 

\begin{tabular}{llllllllll}
\hyperref[図1.1]{1.1} &
\hyperref[図1.2]{1.2} &
\hyperref[図1.3]{1.3} &
\hyperref[図1.4]{1.4} &
\hyperref[図1.5]{1.5} &
\end{tabular} 

\subsubsection*{第2章} 

\begin{tabular}{llllllllll}
\hyperref[図2.1]{2.1} &
\hyperref[図2.2]{2.2} &
\hyperref[図2.3]{2.3} &
\hyperref[図2.4]{2.4} &
\hyperref[図2.5]{2.5} &
\hyperref[図2.6]{2.6} &
\hyperref[図2.7]{2.7} &
\hyperref[図2.8]{2.8} &
\hyperref[図2.9]{2.9} &
\hyperref[図2.10]{2.10}
\\ 
\hyperref[図2.11]{2.11} &
\hyperref[図2.12]{2.12} &
\hyperref[図2.13]{2.13} &
\hyperref[図2.14]{2.14} &
\hyperref[図2.15]{2.15} &
\hyperref[図2.16]{2.16} &
\hyperref[図2.17]{2.17} &
\hyperref[図2.18]{2.18} &
\hyperref[図2.19]{2.19} &
\hyperref[図2.20]{2.20}
\\ 
\hyperref[図2.21]{2.21} &
\hyperref[図2.22]{2.22} &
\hyperref[図2.23]{2.23} &
\hyperref[図2.24]{2.24} &
\hyperref[図2.25]{2.25} &
\hyperref[図2.26]{2.26} &
\end{tabular} 

\subsubsection*{第3章} 

\begin{tabular}{llllllllll}
\hyperref[図3.1]{3.1} &
\hyperref[図3.2]{3.2} &
\hyperref[図3.3]{3.3} &
\hyperref[図3.4]{3.4} &
\hyperref[図3.5]{3.5} &
\hyperref[図3.6]{3.6} &
\hyperref[図3.7]{3.7} &
\hyperref[図3.8]{3.8} &
\hyperref[図3.9]{3.9} &
\hyperref[図3.10]{3.10}
\\ 
\hyperref[図3.11]{3.11} &
\hyperref[図3.12]{3.12} &
\hyperref[図3.13]{3.13} &
\hyperref[図3.14]{3.14} &
\hyperref[図3.15]{3.15} &
\hyperref[図3.16]{3.16} &
\hyperref[図3.17]{3.17} &
\hyperref[図3.18]{3.18} &
\hyperref[図3.19]{3.19} &
\hyperref[図3.20]{3.20}
\\ 
\hyperref[図3.21]{3.21} &
\hyperref[図3.22]{3.22} &
\hyperref[図3.23]{3.23} &
\hyperref[図3.24]{3.24} &
\hyperref[図3.25]{3.25} &
\hyperref[図3.26]{3.26} &
\hyperref[図3.27]{3.27} &
\hyperref[図3.28]{3.28} &
\hyperref[図3.29]{3.29} &
\hyperref[図3.30]{3.30}
\\ 
\hyperref[図3.31]{3.31} &
\hyperref[図3.32]{3.32} &
\hyperref[図3.33]{3.33} &
\hyperref[図3.34]{3.34} &
\hyperref[図3.35]{3.35} &
\hyperref[図3.36]{3.36} &
\hyperref[図3.37]{3.37} &
\hyperref[図3.38]{3.38} &
\end{tabular} 

\subsubsection*{第4章} 

\begin{tabular}{llllllllll}
\hyperref[図4.1]{4.1} &
\hyperref[図4.2]{4.2} &
\hyperref[図4.3]{4.3} &
\hyperref[図4.4]{4.4} &
\hyperref[図4.5]{4.5} &
\hyperref[図4.6]{4.6} &
\end{tabular} 

\subsubsection*{第5章} 

\begin{tabular}{llllllllll}
\hyperref[図5.1]{5.1} &
\hyperref[図5.2]{5.2} &
\hyperref[図5.3]{5.3} &
\hyperref[図5.4]{5.4} &
\hyperref[図5.5]{5.5} &
\hyperref[図5.6]{5.6} &
\hyperref[図5.7]{5.7} &
\hyperref[図5.8]{5.8} &
\hyperref[図5.9]{5.9} &
\hyperref[図5.10]{5.10}
\\ 
\hyperref[図5.11]{5.11} &
\hyperref[図5.12]{5.12} &
\hyperref[図5.13]{5.13} &
\hyperref[図5.14]{5.14} &
\hyperref[図5.15]{5.15} &
\hyperref[図5.16]{5.16} &
\hyperref[図5.17]{5.17} &
\hyperref[図5.18]{5.18} &
\end{tabular} 



\begin{comment}
\chapter*{Index}
\addcontentsline{toc}{chapter}{Index}
\label{Index}

\vspace{0.8em}

\begin{quote}
Any inaccuracies in this index may be explained by the fact\\ 
that it has been prepared with the help of a computer. 

---Donald E. Knuth, \textit{Fundamental Algorithms}\\ 
(Volume 1 of \textit{The Art of Computer Programming}) 
\end{quote}

\vspace{1.7em}
\end{comment}

\setindexprenote{\normalsize \begin{quote} Any inaccuracies in this index may be explained by the fact 
that it has been prepared with the help of a computer. 

---Donald E. Knuth, \textit{Fundamental Algorithms}\\ 
(Volume 1 of \textit{The Art of Computer Programming}) \end{quote}}

\printindex

\chapter*{Colophon}
\addcontentsline{toc}{chapter}{Colophon}
\label{Colophon}

\lettrine[lraise=-0.03,loversize=0.08]{O}{n the cover page} is Agostino Ramelli's bookwheel mechanism from 1588. It could be seen as an early hypertext navigation aid. This image of the engraving is hosted by J. E. Johnson of \href{http://newgottland.com/2012/02/09/before-the-ereader-there-was-the-wheelreader/ramelli_bookwheel_1032px/}{New Gottland}. 

The typefaces are Linux Libertine for body text and Linux Biolinum for headings, both by Philipp H. Poll. Typewriter face is Inconsolata created by Raph Levien and supplemented by Dimosthenis Kaponis and Takashi Tanigawa in the form of Inconsolata LGC. 

Graphic design and typography are done by Andres Raba. Texinfo source is converted to LaTeX by a Perl script and compiled to \acronym{PDF} by XeLaTeX. Diagrams are drawn with Inkscape.


\end{document}
